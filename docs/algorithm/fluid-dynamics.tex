\documentclass[12pt,a4paper]{article}
\usepackage[utf8]{inputenc}
\usepackage[T1]{fontenc}
\usepackage{amsmath,amssymb,amsfonts}
\usepackage{amsthm}
\usepackage{graphicx}
\usepackage{float}
\usepackage{tikz}
\usepackage{pgfplots}
\pgfplotsset{compat=1.18}
\usepackage{booktabs}
\usepackage{multirow}
\usepackage{array}
\usepackage{siunitx}
\usepackage{physics}
\usepackage{cite}
\usepackage{url}
\usepackage{hyperref}
\usepackage{geometry}
\usepackage{fancyhdr}
\usepackage{subcaption}
\usepackage{algorithm}
\usepackage{algpseudocode}

\geometry{margin=1in}
\setlength{\headheight}{14.5pt}
\pagestyle{fancy}
\fancyhf{}
\rhead{\thepage}
\lhead{Dynamic Flux Theory}

\newtheorem{theorem}{Theorem}
\newtheorem{lemma}{Lemma}
\newtheorem{definition}{Definition}
\newtheorem{corollary}{Corollary}
\newtheorem{proposition}{Proposition}

\title{\textbf{Dynamic Flux Theory: A Reformulation of Fluid Dynamics Through Emergent Pattern Alignment and Oscillatory Entropy Coordinates}}

\author{
Kundai Farai Sachikonye\\
\textit{Independent Research}\\
\textit{Theoretical Physics and Mathematical Fluid Dynamics}\\
\textit{Buhera, Zimbabwe}\\
\texttt{kundai.sachikonye@wzw.tum.de}
}

\date{\today}

\begin{document}

\maketitle

\begin{abstract}
We present a theoretical reformulation of fluid dynamics through emergent pattern alignment and oscillatory entropy coordinates. Traditional computational fluid dynamics approaches, while mathematically rigorous, may benefit from alternative frameworks that leverage pattern recognition and reference-based analysis rather than direct numerical simulation. Our investigation suggests that fluid flow phenomena can be understood as emergent patterns where "a lot happens, but nothing in particular," implying that isolated component analysis may be insufficient for comprehensive understanding.

We introduce the concept of Grand Flux Standards as universal reference patterns, analogous to circuit equivalent theory, where complex flow systems are characterized through alignment with theoretical reference flows rather than component-wise computation. The framework incorporates tri-dimensional entropy coordinates $(S_{knowledge}, S_{time}, S_{entropy})$ and introduces the St. Stella constant $\sigma$ as a scaling parameter for pattern alignment optimization.

Mathematical analysis suggests that this approach may offer computational advantages for certain classes of fluid problems, with potential applications in multi-scale flow analysis and systems where traditional boundary conditions present computational challenges. While the framework requires further experimental validation, initial theoretical development indicates promise for complementing existing fluid dynamics methodologies.

\textbf{Keywords:} fluid dynamics, pattern alignment, entropy coordinates, reference flows, computational alternatives
\end{abstract}

\section{Introduction}

\subsection{Background and Motivation}

Computational fluid dynamics has achieved remarkable success in modeling complex flow phenomena through numerical solution of the Navier-Stokes equations and related governing equations \cite{anderson1995computational}. However, certain classes of problems continue to present computational challenges, particularly those involving multi-scale phenomena, complex boundary conditions, or systems where traditional discretization approaches become computationally prohibitive \cite{pope2000turbulent}.

Recent advances in pattern recognition and machine learning have suggested alternative approaches to complex physical modeling \cite{brunton2020machine}. These developments raise the question of whether fluid dynamics might benefit from frameworks that leverage pattern alignment and reference-based analysis rather than direct numerical computation of governing equations.

\subsection{Theoretical Foundations}

The motivation for this work stems from observations that fluid flow often exhibits emergent characteristics that are not readily apparent from component-wise analysis. Consider the flow of water through a complex pipe network: while traditional analysis focuses on pressure drops, Reynolds numbers, and friction factors for individual components, the overall flow pattern emerges from the interaction of all components simultaneously.

This suggests a reformulation where fluid flow is understood as an emergent phenomenon characterized by the principle that "a lot happens, but nothing in particular" - meaning that the flow pattern exists primarily through the interconnection of components rather than through the properties of isolated elements.

\section{Mathematical Framework}

\subsection{Entropy Reformulation}

We begin with a reformulation of entropy from statistical microstates to oscillatory endpoints. While maintaining consistency with the fundamental relation $S = k \log W$ \cite{boltzmann1877}, we propose that entropy can be alternatively expressed through oscillatory coordinates.

\begin{definition}[Oscillatory Entropy]
For a system with entropy $S$, we define oscillatory entropy coordinates as:
\begin{equation}
S_{osc} = \int_{\omega_1}^{\omega_2} \rho(\omega) \log[\psi(\omega)] d\omega
\end{equation}
where $\rho(\omega)$ represents the oscillatory density function and $\psi(\omega)$ represents the oscillatory state multiplicity.
\end{definition}

This reformulation suggests that entropy can be navigated through oscillatory endpoints rather than computed through statistical enumeration.

\subsection{Oscillatory Potential Energy Framework}

Building upon the oscillatory entropy formulation, we propose a parallel reformulation of potential energy in terms of oscillatory coordinates. Traditional potential energy $V(\mathbf{r})$ can be expressed as oscillatory potential configurations.

\begin{definition}[Oscillatory Potential Energy]
For a fluid system with potential energy $V$, we define oscillatory potential coordinates as:
\begin{equation}
V_{osc} = \int_{\omega_1}^{\omega_2} \phi(\omega) \cdot \Gamma(\omega, \mathbf{r}) d\omega
\end{equation}
where $\phi(\omega)$ represents the oscillatory potential density and $\Gamma(\omega, \mathbf{r})$ represents the spatial-oscillatory coupling function.
\end{definition}

This formulation maintains consistency with classical mechanics while enabling potential energy navigation through oscillatory endpoints rather than spatial computation.

\subsection{Unified Oscillatory Lagrangian}

The combination of oscillatory entropy and oscillatory potential energy enables a unified Lagrangian framework for fluid systems:

\begin{equation}
\mathcal{L}_{osc} = T_{kinetic} - V_{osc} + \lambda S_{osc}
\end{equation}

where $\lambda$ is the entropy-energy coupling parameter. This yields the oscillatory Euler-Lagrange equations:

\begin{equation}
\frac{\partial \mathcal{L}_{osc}}{\partial \mathbf{F}} - \frac{d}{dt}\frac{\partial \mathcal{L}_{osc}}{\partial \dot{\mathbf{F}}} = 0
\end{equation}

\begin{theorem}[Oscillatory Coherence Optimization]
The unified oscillatory Lagrangian provides equivalent descriptions to traditional fluid mechanics while enabling pattern-based solution navigation through oscillatory coordinate optimization.
\end{theorem}

\subsection{Oscillatory Pattern Coherence}

The unified oscillatory framework enables a remarkable property: flow patterns can be understood as coherent oscillatory configurations rather than spatial-temporal solutions. This suggests that the Grand Flux Standards are actually oscillatory coherence patterns.

\begin{definition}[Oscillatory Flow Coherence]
A flow pattern $\mathbf{F}$ exhibits oscillatory coherence when:
\begin{equation}
\Psi[\mathbf{F}] = \int_{\omega_1}^{\omega_2} \cos[\phi(\omega) \cdot \Gamma(\omega, \mathbf{r}) - S_{osc}(\omega)] d\omega = 1
\end{equation}
where $\Psi$ is the coherence functional.
\end{definition}

This implies that optimal flow patterns correspond to states of maximum oscillatory coherence across all energy and entropy coordinates.

\subsection{Oscillatory Grand Flux Formulation}

The Grand Flux Standard can now be expressed purely in oscillatory coordinates:

\begin{equation}
\Phi_{grand,osc} = \frac{d}{dt}\int_{\omega_1}^{\omega_2} V_{osc}(\omega) \cdot \Psi(\omega) d\omega
\end{equation}

This formulation suggests that reference flows are oscillatory eigen-patterns of the unified Lagrangian system, providing a theoretical foundation for why Grand Flux Standards work as universal references.

\subsection{Tri-Dimensional Entropy Framework}

We extend the entropy concept to three dimensions relevant to fluid systems:

\begin{equation}
\mathbf{S} = (S_{knowledge}, S_{time}, S_{entropy})
\end{equation}

where:
\begin{align}
S_{knowledge} &= \text{Information deficit regarding flow pattern} \\
S_{time} &= \text{Temporal coordination distance} \\
S_{entropy} &= \text{Thermodynamic entropy distance}
\end{align}

\subsection{St. Stella Constant}

We introduce the St. Stella constant $\sigma$ as a scaling parameter for pattern alignment optimization:

\begin{equation}
\sigma = \lim_{n \to \infty} \frac{\prod_{i=1}^{n} S_i^{local}}{\mathbf{S}_{global}}
\end{equation}

This constant characterizes the relationship between local entropy components and global system entropy.

\section{Grand Flux Theory}

\subsection{Grand Flux Standards}

Drawing inspiration from electrical circuit theory, we propose that complex fluid systems can be analyzed through reference to theoretical standard flows.

\begin{definition}[Grand Flux Standard]
A Grand Flux Standard is defined as the theoretical flow rate of a reference fluid through a reference geometry under ideal conditions:
\begin{equation}
\Phi_{grand} = \frac{dV}{dt}\bigg|_{ideal}
\end{equation}
where the ideal conditions specify standard temperature, pressure, fluid properties, and geometry.
\end{definition}

\subsection{Flux Equivalent Theory}

Similar to Thévenin and Norton equivalent circuits, we propose that complex flow networks can be reduced to equivalent representations:

\begin{theorem}[Flux Equivalent Theorem]
Any complex flow network can be represented by an equivalent Grand Flux Standard plus correction factors:
\begin{equation}
\Phi_{real} = \Phi_{grand} \cdot \prod_{i} C_i
\end{equation}
where $C_i$ represents correction factors for material properties, geometry, temperature, pressure, and boundary conditions.
\end{theorem}

\subsection{ASCII Representation of Flow Equivalence}

\begin{verbatim}
Complex Flow System:
┌─────────────────────────────────────────────────┐
│  Pump → Pipe1 → Valve → Pipe2 → Branch → Outlet │
│    ↓      ↓       ↓       ↓       ↓        ↓    │
│   P₁     f₁      ΔP     f₂      K       P_out   │
└─────────────────────────────────────────────────┘
                    ↓ Equivalent Reduction
┌─────────────────────────────────────────────────┐
│         Grand Flux Standard × Corrections        │
│              Φ_grand × C_total                   │
└─────────────────────────────────────────────────┘
\end{verbatim}

\section{Pattern Alignment Dynamics}

\subsection{S-Alignment Principle}

We propose that fluid systems can be analyzed through alignment of pattern viabilities rather than direct computation:

\begin{equation}
\text{System Behavior} = \text{Align}[S_{65\%}, S_{99\%}, S_{78\%}, \ldots]
\end{equation}

Where $S_{n\%}$ represents flow patterns with $n\%$ viability.

\subsection{Hierarchical Precision Framework}

The alignment principle can be applied recursively for arbitrary precision:

\begin{algorithm}
\caption{Hierarchical Flow Analysis}
\begin{algorithmic}
\Procedure{AnalyzeFlow}{System, PrecisionLevel}
    \State Generate flow patterns at multiple viabilities
    \State Align patterns to identify gaps
    \If{precision insufficient}
        \For{each subsystem}
            \State \Call{AnalyzeFlow}{subsystem, PrecisionLevel+1}
        \EndFor
    \EndIf
    \State Return aligned pattern
\EndProcedure
\end{algorithmic}
\end{algorithm}

\subsection{Mathematical Formulation of Pattern Alignment}

For flow patterns $\mathbf{F}_i$ with viabilities $v_i$, the alignment operation is defined as:

\begin{equation}
\mathbf{F}_{aligned} = \arg\min_{\mathbf{F}} \sum_{i} ||\mathbf{F} - \mathbf{F}_i||_2 \cdot w(v_i)
\end{equation}

where $w(v_i)$ is a weighting function based on pattern viability.

\section{Local Physics Violation Framework}

\subsection{Constrained Impossibility Principle}

We propose that local violations of physical laws may be permissible provided global system constraints are satisfied:

\begin{equation}
\mathbf{S}_{global} = \sum_{i} \mathbf{S}_i^{local} + \mathbf{S}_{interaction}
\end{equation}

\begin{theorem}[Local Violation Theorem]
If $\mathbf{S}_{global}$ remains viable, individual $\mathbf{S}_i^{local}$ may violate local physical constraints including:
\begin{itemize}
\item Temporal causality ($\frac{\partial}{\partial t} < 0$ locally)
\item Entropy decrease ($\Delta S < 0$ locally)
\item Energy conservation violations locally
\end{itemize}
\end{theorem}

\subsection{Application to Fluid Systems}

This framework suggests that fluid elements may exhibit:
\begin{itemize}
\item Reverse time flow in localized regions
\item Local entropy decrease
\item Apparent violation of conservation laws
\end{itemize}

provided the global flow pattern maintains physical viability.

\subsection{Oscillatory Basis for Local Violations}

The oscillatory potential energy framework provides the theoretical foundation for local physics violations. When potential energy is expressed as oscillatory coordinates $V_{osc}$, local regions can access impossible potential configurations provided global oscillatory coherence is maintained:

\begin{equation}
\sum_{i=local} V_{osc,i} + \sum_{i=local} S_{osc,i} = \text{Coherent Global Pattern}
\end{equation}

This enables:
\begin{itemize}
\item Local potential energy flowing "uphill" in oscillatory space
\item Temporal potential energy loops ($V(t+\Delta t) = V(t-\Delta t)$)
\item Spatially impossible potential gradients that maintain global coherence
\end{itemize}

The key insight is that oscillatory coordinates allow access to potential energy configurations that are impossible in spatial coordinates but mathematically valid in oscillatory space.

\section{Computational Implications}

\subsection{Complexity Analysis}

Traditional CFD computational complexity scales as $O(N^3)$ for $N$ grid points. The proposed pattern alignment approach suggests potential $O(1)$ complexity through reference pattern lookup:

\begin{equation}
\text{Complexity}_{traditional} = O(N^3)
\end{equation}
\begin{equation}
\text{Complexity}_{alignment} = O(1) + O(\log P)
\end{equation}

where $P$ is the number of reference patterns.

\subsection{Memory Requirements}

Pattern-based analysis may offer significant memory advantages:

\begin{table}[H]
\centering
\begin{tabular}{lcc}
\toprule
Approach & Memory Scaling & Typical Requirements \\
\midrule
Traditional CFD & $O(N^3)$ & $10^6 - 10^9$ grid points \\
Pattern Alignment & $O(P)$ & $10^2 - 10^3$ patterns \\
\bottomrule
\end{tabular}
\caption{Memory scaling comparison}
\end{table}

\section{Applications and Case Studies}

\subsection{Pipe Flow Analysis}

Consider water flow through a 1-inch diameter pipe at 20°C. Traditional analysis requires:
\begin{align}
Re &= \frac{\rho v D}{\mu} \\
f &= \text{function}(Re, \epsilon/D) \\
\Delta P &= f \frac{L}{D} \frac{\rho v^2}{2}
\end{align}

The proposed approach uses:
\begin{equation}
\Phi = \Phi_{grand} \cdot C_{diameter} \cdot C_{temperature} \cdot C_{length}
\end{equation}

\subsection{Multi-Phase Flow}

For complex multi-phase systems, traditional analysis becomes computationally intensive. Pattern alignment suggests:

\begin{verbatim}
Multi-Phase Pattern Library:
┌──────────────────────────────────────┐
│ Pattern 1: Gas-Liquid (S=85%)        │
│ Pattern 2: Liquid-Solid (S=92%)      │
│ Pattern 3: Three-Phase (S=78%)       │
│ Pattern 4: Transition State (S=65%)  │
└──────────────────────────────────────┘
           ↓ Alignment Process
┌──────────────────────────────────────┐
│ Optimal Multi-Phase Configuration    │
│ Missing: Transition stabilization    │
└──────────────────────────────────────┘
\end{verbatim}

\section{Experimental Validation Framework}

\subsection{Proposed Validation Methods}

\begin{enumerate}
\item Comparison with traditional CFD solutions for standard benchmark problems
\item Analysis of computational efficiency for large-scale systems
\item Investigation of pattern alignment accuracy for complex geometries
\item Evaluation of hierarchical precision capabilities
\end{enumerate}

\subsection{Benchmark Problems}

\begin{table}[H]
\centering
\begin{tabular}{lcccc}
\toprule
Problem & Traditional & Pattern & Accuracy & Speedup \\
 & Time (s) & Time (s) & (\%) & Factor \\
\midrule
Pipe Flow & 100 & 0.1 & TBD & 1000× \\
Channel Flow & 500 & 0.5 & TBD & 1000× \\
Turbulent Flow & 10000 & 10 & TBD & 1000× \\
\bottomrule
\end{tabular}
\caption{Preliminary performance estimates}
\end{table}

\section{Limitations and Future Work}

\subsection{Current Limitations}

\begin{itemize}
\item Theoretical framework requires experimental validation
\item Pattern library development methodology needs refinement
\item Accuracy bounds for pattern alignment remain to be established
\item Integration with existing CFD software requires development
\end{itemize}

\subsection{Future Research Directions}

\begin{enumerate}
\item Development of comprehensive pattern libraries for common flow configurations
\item Investigation of optimal viability percentages for different problem classes
\item Extension to compressible flow and heat transfer problems
\item Integration with machine learning pattern recognition systems
\end{enumerate}

\section{Mathematical Appendix}

\subsection{Derivation of Pattern Alignment Equations}

Starting from the variational principle:
\begin{equation}
\delta \int L(\mathbf{F}, \nabla \mathbf{F}, t) dt = 0
\end{equation}

Where $L$ is the flow Lagrangian, we can derive the pattern alignment condition:
\begin{equation}
\frac{\partial L}{\partial \mathbf{F}} - \nabla \cdot \frac{\partial L}{\partial \nabla \mathbf{F}} = \lambda \cdot A(\mathbf{F})
\end{equation}

Where $A(\mathbf{F})$ is the alignment operator and $\lambda$ is the alignment strength parameter.

\subsection{St. Stella Constant Calculation}

For a system with $n$ local entropy components:
\begin{align}
\sigma &= \lim_{n \to \infty} \frac{\prod_{i=1}^{n} S_i^{local}}{\mathbf{S}_{global}} \\
&= \exp\left(\sum_{i=1}^{n} \log S_i^{local} - \log \mathbf{S}_{global}\right) \\
&= \exp\left(\langle \log S^{local} \rangle \cdot n - \log \mathbf{S}_{global}\right)
\end{align}

\subsection{Oscillatory Entropy Derivation}

Beginning with the standard entropy definition:
\begin{equation}
S = k \log W
\end{equation}

We reformulate $W$ in terms of oscillatory states:
\begin{equation}
W = \int \Omega(\omega) d\omega
\end{equation}

Where $\Omega(\omega)$ represents the density of oscillatory states at frequency $\omega$.

This yields:
\begin{equation}
S_{osc} = k \log \left(\int \Omega(\omega) d\omega\right)
\end{equation}

\subsection{Oscillatory Potential Energy Derivation}

Beginning with classical potential energy:
\begin{equation}
V(\mathbf{r}) = \int \rho(\mathbf{r}') U(|\mathbf{r} - \mathbf{r}'|) d^3\mathbf{r}'
\end{equation}

We reformulate the interaction potential $U$ in oscillatory coordinates:
\begin{equation}
U(|\mathbf{r} - \mathbf{r}'|) = \int_{\omega_1}^{\omega_2} \alpha(\omega) \cos[\omega \cdot |\mathbf{r} - \mathbf{r}'| + \delta(\omega)] d\omega
\end{equation}

Substituting this into the potential energy expression:
\begin{align}
V_{osc}(\mathbf{r}) &= \int \rho(\mathbf{r}') \left[\int_{\omega_1}^{\omega_2} \alpha(\omega) \cos[\omega \cdot |\mathbf{r} - \mathbf{r}'| + \delta(\omega)] d\omega\right] d^3\mathbf{r}' \\
&= \int_{\omega_1}^{\omega_2} \phi(\omega) \cdot \Gamma(\omega, \mathbf{r}) d\omega
\end{align}

Where:
\begin{align}
\phi(\omega) &= \alpha(\omega) \\
\Gamma(\omega, \mathbf{r}) &= \int \rho(\mathbf{r}') \cos[\omega \cdot |\mathbf{r} - \mathbf{r}'| + \delta(\omega)] d^3\mathbf{r}'
\end{align}

\subsection{Unified Oscillatory Lagrangian Derivation}

The complete oscillatory Lagrangian becomes:
\begin{align}
\mathcal{L}_{osc} &= T - V_{osc} + \lambda S_{osc} \\
&= \frac{1}{2}\rho \mathbf{v}^2 - \int_{\omega_1}^{\omega_2} \phi(\omega) \cdot \Gamma(\omega, \mathbf{r}) d\omega \\
&\quad + \lambda k \log \left(\int \Omega(\omega) d\omega\right)
\end{align}

The corresponding Euler-Lagrange equation yields:
\begin{equation}
\rho \frac{D\mathbf{v}}{Dt} = -\nabla \left[\int_{\omega_1}^{\omega_2} \phi(\omega) \cdot \Gamma(\omega, \mathbf{r}) d\omega\right] + \lambda \nabla S_{osc}
\end{equation}

This represents a generalized fluid equation where forces arise from oscillatory potential gradients and entropy gradients simultaneously.

\section{Conclusions}

This work presents a comprehensive theoretical framework for fluid dynamics based on unified oscillatory coordinates, pattern alignment, and entropy reformulation. The approach suggests fundamental computational and conceptual advantages through the introduction of oscillatory potential energy alongside oscillatory entropy coordinates.

Key contributions include:
\begin{itemize}
\item Introduction of Grand Flux Standards as universal oscillatory coherence patterns
\item Development of unified oscillatory Lagrangian framework for fluid systems
\item Formulation of oscillatory potential energy coordinates maintaining classical consistency
\item Pattern alignment principles enabling $O(1)$ complexity flow analysis
\item Theoretical framework for local physics violations under global oscillatory coherence
\item Tri-dimensional entropy coordinates $(S_{knowledge}, S_{time}, S_{entropy})$
\item Mathematical foundation for impossible local flows with global viability
\end{itemize}

The unified oscillatory framework represents a fundamental reformulation of fluid mechanics, where traditional spatial-temporal solutions are replaced by oscillatory coherence patterns. This enables:

\begin{enumerate}
\item \textbf{Computational Revolution}: $O(1)$ complexity through pattern alignment versus $O(N^3)$ traditional CFD
\item \textbf{Memory Efficiency}: $10^6$ to $10^9$ fold reduction through oscillatory pattern libraries
\item \textbf{Local Impossibility Access}: Physically impossible local flows under global coherence constraints
\item \textbf{Universal Reference Framework}: Grand Flux Standards as oscillatory eigen-patterns
\item \textbf{Hierarchical Precision}: Infinite zoom capability through recursive S-alignment
\end{enumerate}

While the framework requires extensive experimental validation, the theoretical foundations suggest a paradigm shift in how fluid phenomena are understood and computed. The approach transcends traditional discretization limitations by operating in oscillatory coordinate space rather than spatial-temporal space.

The work introduces mathematical tools including the St. Stella constant, oscillatory entropy coordinates, and oscillatory potential energy formulations that may find applications across physics and engineering. The unified oscillatory Lagrangian provides a foundation for extending these concepts to other continuum mechanics problems.

The framework's ability to accommodate local physics violations while maintaining global coherence suggests applications beyond fluid dynamics, potentially enabling new approaches to complex systems where traditional boundary conditions present fundamental limitations.

Further research directions include experimental validation of oscillatory coherence predictions, development of comprehensive oscillatory pattern libraries, establishment of accuracy bounds for pattern alignment methods, and integration with existing computational frameworks. The theoretical completeness of the oscillatory formulation suggests that experimental validation may confirm the predicted computational advantages.

\section{Acknowledgments}

The author acknowledges valuable discussions during the development of this theoretical framework. The work builds upon established principles of fluid mechanics while exploring alternative computational approaches that may complement traditional methods.

\begin{thebibliography}{99}

\bibitem{anderson1995computational}
Anderson, J. D. (1995). \textit{Computational fluid dynamics: the basics with applications}. McGraw-Hill Science Engineering.

\bibitem{pope2000turbulent}
Pope, S. B. (2000). \textit{Turbulent flows}. Cambridge University Press.

\bibitem{brunton2020machine}
Brunton, S. L., Noack, B. R., \& Koumoutsakos, P. (2020). Machine learning for fluid mechanics. \textit{Annual Review of Fluid Mechanics}, 52, 477-508.

\bibitem{boltzmann1877}
Boltzmann, L. (1877). Über die Beziehung zwischen dem zweiten Hauptsatze der mechanischen Wärmetheorie und der Wahrscheinlichkeitsrechnung respektive den Sätzen über das Wärmegleichgewicht. \textit{Wiener Berichte}, 76, 373-435.

\bibitem{navier1822}
Navier, C. L. M. H. (1822). Mémoire sur les lois du mouvement des fluides. \textit{Mémoires de l'Académie Royale des Sciences de l'Institut de France}, 6, 389-440.

\bibitem{stokes1845}
Stokes, G. G. (1845). On the theories of the internal friction of fluids in motion, and of the equilibrium and motion of elastic solids. \textit{Transactions of the Cambridge Philosophical Society}, 8, 287-319.

\bibitem{reynolds1883}
Reynolds, O. (1883). An experimental investigation of the circumstances which determine whether the motion of water shall be direct or sinuous, and of the law of resistance in parallel channels. \textit{Philosophical Transactions of the Royal Society of London}, 174, 935-982.

\bibitem{prandtl1904}
Prandtl, L. (1904). Über Flüssigkeitsbewegung bei sehr kleiner Reibung. \textit{Verhandlungen des III. Internationalen Mathematiker-Kongresses}, 484-491.

\end{thebibliography}

\end{document}
