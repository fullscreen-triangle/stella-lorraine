\section{Introduction}

Interferometric imaging achieves angular resolution through the synthesis of apertures distributed across baselines that exceed the physical dimensions of individual telescopes. The fundamental relationship
\begin{equation}
\theta_{\text{min}} \approx \frac{\lambda}{D}
\end{equation}
where \(\lambda\) denotes wavelength and \(D\) represents baseline separation, it establishes that angular resolution improves linearly with increasing baseline extent. Very Long Baseline Interferometry (VLBI) has extended terrestrial baselines to continental scales (\(D \sim 10^4\) km), achieving micro-arcsecond resolution in radio wavelengths \cite{thompson2017interferometry}.

However, current implementations face fundamental limitations:

\begin{enumerate}
\item \textbf{Clock Synchronization}: Atomic frequency standards at separated stations exhibit drift rates of \(\sim 10^{-15}\) per day, requiring continuous correction \cite{riehle2018optical}.

\item \textbf{Phase Coherence Loss}: Atmospheric turbulence and baseline-dependent delays introduce phase errors that accumulate over integration times, particularly at optical wavelengths where \(\lambda \sim 500\) nm.

\item \textbf{Post-Processing Latency}: Correlation of signals recorded at distributed stations requires physical transport of recorded data or complex network infrastructure, introducing delays of hours to days \cite{doeleman2008event}.

\item \textbf{Atmospheric Limitations}: Propagation through the atmosphere introduces path-length variations of order \(\sim \lambda\) at optical wavelengths, degrading coherence \cite{roddier1981atmospheric}.
\end{enumerate}

These constraints become increasingly severe as baselines extend and observing wavelengths decrease. At optical wavelengths (\(\lambda \sim 400\)–700 nm), atmospheric seeing limits ground-based resolution to \(\sim 0.1\) arcseconds without adaptive optics, regardless of baseline length \cite{beckers1993adaptive}.

\subsection{Categorical Framework Overview}

Recent developments in categorical state theory \cite{author2024categorical} demonstrate that molecular systems evolve through discrete categorical states defined by irreversible occupation dynamics. Each categorical state \(\mathcal{C}(t)\) encodes complete phase-space information within a coordinate system defined by three entropic dimensions:
\begin{equation}
\mathbf{S} = (S_k, S_t, S_e)
\end{equation}
where \(S_k\) represents knowledge entropy, \(S_t\) represents temporal entropy, and \(S_e\) represents configurational entropy.

The phase-lock network formalism \cite{author2024phaselocks} establishes that molecules exist in synchronized oscillatory relationships, with state transitions governed by:
\begin{equation}
\frac{d\mathcal{C}}{dt} = \mathcal{F}[\mathcal{C}, \mathbf{S}(t)]
\end{equation}

Crucially, categorical state prediction \cite{author2024prediction} demonstrates distance-independent information access: given the categorical state \(\mathcal{C}_A\) at location \(\mathbf{r}_A\), the corresponding state \(\mathcal{C}_B\) at a spatially separated location \(\mathbf{r}_B\) can be determined through categorical coordinate transformation without requiring propagation through the intervening spatial volume.

\begin{figure}[htbp]
    \centering
    \includegraphics[width=0.95\textwidth]{figures/figure_19_gibbs_paradox_resolution.png}
    \caption{\textbf{Resolution of Gibbs' paradox through categorical state irreversibility.}
    (a) Traditional Gibbs paradox: Mixing entropy (red line) shows discontinuity at gas
    similarity parameter $\approx 0.5$. Identical gases ($\Delta S = 0$, green box annotation)
    vs different gases ($\Delta S = k_B \ln(2)$, pink box annotation). Red X marks discontinuity.
    Yellow box: "DISCONTINUITY (Paradox)". (b) Categorical irreversibility: Once mixed
    ($C_{\text{mixed}}$ completed, purple region with blue circles), cannot return to
    $C_{\text{separated}}$ (red region A and blue region B). Green box: "Once mixed
    ($C_{\text{mixed}}$ completed), cannot return to $C_{\text{separated}}$". Red arrow shows
    "MIXING" allowed. Red box: "IMPOSSIBLE (Categorical irreversibility)" with red X. Pink
    region shows $C_{\text{separated}}$ (cannot return). (c) Oscillatory entropy formulation:
    Two oscillating curves (orange = Gas A, blue = Gas B) with formula $S = k_B \ln(\alpha)$
    (yellow box). Gray shaded region shows terminated (mixed state) after red dashed line at
    $t \approx 6$. Annotation: "$\alpha = $ termination probability". (d) Resolution: Smooth
    entropy via categorical completion: Categorical resolution (green curve) shows smooth
    transition from 0 to 1.0 mixing entropy. Traditional paradox (pink dashed line) shows
    discontinuous jump at similarity $\approx 0.5$. Green shaded region labeled "NO DISCONTINUITY
    Paradox resolved". Red dashed line shows transition point. (e) Mixing-separation cycle:
    Categorical irreversibility ensures $\Delta S > 0$: Blue oval ($C_{\text{sep}}$, "Separated")
    connects to purple oval ($C_{\text{mix}}$, "Mixed") via green arrow labeled "MIXING (allowed)"
    with "$\Delta S > 0$ (entropy increases)". Purple oval connects to gray oval ($C_{\text{sep}}?$,
    "Separated?") via red dashed arrow labeled "SEPARATION (forbidden)" with "$\Delta S < 0?$
    (impossible)". Blue box at bottom: "Categorical irreversibility: Once $C_{\text{mix}}$ is
    completed, cannot return to $C_{\text{sep}}$. This resolves Gibbs paradox: Full mixing-separation
    cycle ALWAYS increases entropy. $\oint dS > 0$ (cycle entropy always positive)". Large
    blue region at bottom with KEY INSIGHT: "Gibbs' paradox (150-year-old problem) is resolved
    by categorical irreversibility. Physical configurations are distinguished by their position
    in an irreversible completion sequence. Once mixed ($C_{\text{mixed}}$ completed), cannot
    return to separated ($C_{\text{separated}}$) state. Oscillatory entropy $S = k_B \ln(\alpha)$
    provides smooth transition, eliminating discontinuity." \textbf{Revolutionary resolution}:
    The paradox arises from treating mixing as reversible. Categorical irreversibility shows
    that once gases are mixed (categorical state completed), they cannot be unmixed without
    external work. The discontinuity in traditional formulation is an artifact of assuming
    reversibility. Oscillatory entropy provides smooth transition by recognizing that mixing
    is a gradual completion process, not an instantaneous jump. Parameters: Two-gas system,
    similarity parameter from 0 (identical) to 1 (completely different).}
    \label{fig:gibbs_resolution}
    \end{figure}

\subsection{Application to Interferometry}

This work demonstrates that categorical phase correlation enables interferometric measurements with characteristics fundamentally distinct from conventional approaches:

\begin{itemize}
\item Phase information encoded in categorical states propagates through categorical space rather than physical space, bypassing atmospheric distortion.

\item Trans-Planckian timing precision (\(\Delta t \sim 10^{-15}\) s) from hardware-molecular synchronisation \cite{author2024hardware} maintains phase coherence over arbitrary baselines.

\item Real-time categorical state prediction enables correlation without post-processing delays.

\item Multi-band parallel operation (UV + Visible + IR) provides simultaneous spectroscopic and spatial information.
\end{itemize}

We establish the theoretical framework for categorical baseline interferometry, derive achievable angular resolutions, and present experimental architectures for implementation at scales from local (\(D \sim 1\) km) to planetary (\(D \sim 10^4\) km) baselines.
