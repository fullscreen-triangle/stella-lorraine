\section{Multi-Band Parallel Interferometry}
\label{sec:multiband}

\subsection{Single-Wavelength Limitation in Conventional VLBI}

Traditional radio interferometry operates at specific frequencies determined by the receiver design. A typical VLBI station includes:
\begin{itemize}
\item Feed horn matched to target wavelength (\(\lambda\))
\item Amplifiers with bandwidth \(\Delta\nu / \nu \sim 10\%\)
\item Digital backend recording specific frequency channels
\end{itemize}

To observe at multiple wavelengths requires either:
\begin{enumerate}
\item Sequential observations (time-multiplexed)
\item Multiple receiver systems (expensive, \(\sim\$10^5\) per band)
\end{enumerate}

Optical interferometry traditionally uses broadband or narrowband philtres to select specific wavelength ranges, which again requires sequential observations for multi-wavelength data.

\subsection{Virtual Spectrometer Broadband Response}

The H\(^+\) oscillator at the core of virtual spectrometer technology \cite{author2024hardware} exhibits fundamental frequency:
\begin{equation}
\nu_{\text{H}^+} = 71.0 \text{ THz}
\end{equation}
corresponding to wavelength \(\lambda_{\text{H}^+} = c/\nu_{\text{H}^+} = 4.22\) \(\mu\)m (mid-infrared).

However, the oscillator responds to radiation across a broad spectral range through:

\textbf{Harmonic Response:} Integer multiples of the fundamental frequency:
\begin{equation}
\nu_n = n \cdot \nu_{\text{H}^+} \quad \text{for } n = 1, 2, 3, \ldots
\end{equation}
yielding UV/visible response at:
\begin{align}
n = 2: \quad & 142 \text{ THz} \quad (\lambda = 2.11 \, \mu\text{m, near-IR}) \\
n = 5: \quad & 355 \text{ THz} \quad (\lambda = 844 \text{ nm, near-IR}) \\
n = 10: \quad & 710 \text{ THz} \quad (\lambda = 422 \text{ nm, violet})
\end{align}

\textbf{Subharmonic Response:} Fractional frequencies:
\begin{equation}
\nu_{1/n} = \frac{\nu_{\text{H}^+}}{n} \quad \text{for } n = 2, 3, 4, \ldots
\end{equation}
extending to far-infrared.

\textbf{Anharmonic Coupling:} Molecular vibrations couple to external fields through nonlinear polarisability:
\begin{equation}
\mathbf{P} = \chi^{(1)} \mathbf{E} + \chi^{(2)} \mathbf{E}^2 + \chi^{(3)} \mathbf{E}^3 + \cdots
\end{equation}
enabling response to arbitrary frequencies through mixing products.

The net result: a single virtual spectrometer responds simultaneously to UV (\(\sim 300\) nm), visible (\(\sim 400\)–700 nm), and near-IR (\(\sim 1\)–2 \(\mu\)m) radiation.


\begin{figure}[htbp]
    \centering
    \includegraphics[width=0.95\textwidth]{figures/figure_18_categorical_spatial_independence.png}
    \caption{\textbf{Categorical distance $\neq$ spatial distance: mathematical independence
    enabling faster-than-light categorical propagation.} (a) Physical space (spatial distance):
    Two stations A and B separated by $d = 10{,}000$ km (yellow box). Photon path (gray dashed
    line) through atmosphere (gray shaded). Light travel time $t = d/c = 33.4$ ms (blue box).
    Spatial distance $d_{\text{spatial}} = ||\vec{r}_A - \vec{r}_B||$ (formula in box).
    (b) Categorical space (categorical distance): Nodes $C_1$ and $C_3$ connected indirectly
    through $C_2$, or directly via red arrow labeled "Direct categorical link" between A (red
    circle) and B (red circle). Categorical distance $d_{\text{cat}} = 1$ step (yellow box).
    Completion time $t = t_{\text{completion}} = 1.67$ ms (pink box). Categorical distance
    $d_{\text{cat}}(C_i, C_j) = \min\{k : \exists \text{path}\}$ (formula in box). (c) Independence:
    $d_{\text{cat}} \neq f(d_{\text{spatial}})$: Scatter plot shows categorical distance
    (vertical axis, 0-10 steps) vs spatial distance (horizontal axis, 0-10,000 km). Green
    circles scattered randomly with no correlation. Red horizontal line at $d_{\text{cat}} = 5$
    shows mean. Yellow box: "Correlation: $r = -0.233$ (No correlation)". Annotation at bottom:
    "$d_{\text{cat}} \neq f(d_{\text{spatial}})$" (crossed out). (d) Categorical propagation
    speedup: Log-log plot shows speedup factor $v_{\text{cat}}/c$ (vertical axis, $10^0$ to
    $10^3$) vs baseline distance (horizontal axis, $10^0$ to $10^5$ km). Green line shows
    linear increase. Red star marks experimental data point at 10,000 km with 20$\times$ speedup.
    Green shaded region labeled "Categorical propagation faster than light". Red dashed line
    at $v_{\text{cat}}/c = 1$ shows light speed. Annotation: "No violation of relativity
    (categories, not photons)". Blue box at bottom: "KEY INSIGHT: Categorical distance and
    spatial distance are mathematically independent. This enables prediction of molecular
    states across arbitrary spatial separations without physical propagation. Speedup:
    $v_{\text{cat}}/c = 20\times$ (categorical propagation 20 times faster than light)."
    \textbf{Critical clarification}: This is NOT faster-than-light \textit{signaling}. No
    information is transmitted faster than $c$. Rather, categorical relationships are
    \textit{non-local}—they exist independent of spatial separation. The "speedup" is in
    \textit{prediction}, not \textit{causation}. Parameters: 10,000 km baseline, 100 molecules,
    harmonic tolerance $\epsilon = 0.01$.}
    \label{fig:categorical_spatial_independence}
    \end{figure}

\subsection{Simultaneous Multi-Band Detection}

During interferometric observations, an astronomical source emits radiation across a broad spectrum. A single-station virtual spectrometer records the photodetector signal:
\begin{equation}
I(t) = \int_{\nu_{\text{min}}}^{\nu_{\text{max}}} S(\nu) \mathcal{R}(\nu) \cos[2\pi \nu t + \phi(\nu)] d\nu
\end{equation}
where:
\begin{itemize}
\item \(S(\nu)\): Source spectral energy distribution
\item \(\mathcal{R}(\nu)\): Detector spectral response
\item \(\phi(\nu)\): Phase (includes astronomical and atmospheric components)
\end{itemize}

Categorical state extraction \cite{author2024prediction} from \(I(t)\) yields:
\begin{equation}
\mathcal{C}(t) = \sum_{n} \mathcal{C}_n(t) e^{i\phi_n(t)}
\end{equation}
where index \(n\) labels spectral bands (UV, visible-blue, visible-red, near-IR, etc.).

Each band \(\mathcal{C}_n\) constitutes an independent interferometric channel.

\subsection{Multi-Band Visibility Decomposition}

The total categorical visibility separates into band components:
\begin{equation}
V_{\text{total}}(\mathbf{u}) = \sum_{n} w_n V_n(\mathbf{u}_n)
\end{equation}
where:
\begin{align}
\mathbf{u}_n &= \frac{\mathbf{D}}{\lambda_n} \quad \text{(spatial frequency for band } n \text{)} \\
w_n &= \frac{\int S(\nu) \mathcal{R}(\nu) d\nu}{\text{Band } n} \quad \text{(weight)}
\end{align}

Each band samples different spatial frequencies:
\begin{align}
\mathbf{u}_{\text{UV}} &= \mathbf{D}/\lambda_{\text{UV}} \quad (\lambda_{\text{UV}} \sim 350 \text{ nm}) \\
\mathbf{u}_{\text{Vis}} &= \mathbf{D}/\lambda_{\text{Vis}} \quad (\lambda_{\text{Vis}} \sim 550 \text{ nm}) \\
\mathbf{u}_{\text{IR}} &= \mathbf{D}/\lambda_{\text{IR}} \quad (\lambda_{\text{IR}} \sim 1000 \text{ nm})
\end{align}

For fixed baseline \(\mathbf{D}\), this provides three distinct measurements in Fourier space, improving image reconstruction.

\subsection{Triangular Amplification Per Band}

The categorical prediction framework \cite{author2024ftl} demonstrates that triangular amplification mechanisms can be applied independently to each spectral band. Each band \(n\) forms its own categorical "triangle":
\begin{equation}
\mathcal{C}_n^A \to \mathcal{C}_n^B \quad \text{(direct path)}
\end{equation}
with transmission time:
\begin{equation}
t_n = \frac{|\mathbf{D}|}{v_{\text{cat},n}}
\end{equation}

where \(v_{\text{cat},n}\) can differ by band due to wavelength-dependent categorical coupling strength.

Experimental measurements \cite{author2024ftl} show \(v_{\text{cat}}/c \in [2.846, 65.71]\) depending on triangular configuration. With independent triangulation per band:
\begin{equation}
t_{\text{multi-band}} = \min_n(t_n)
\end{equation}

The fastest band determines the overall correlation time.

\subsection{Spectroscopic + Spatial Information}

Multi-band operation provides additional information beyond single-wavelength interferometry:

\subsubsection{Chromatic Phase Dispersion}

Atmospheric refraction introduces wavelength-dependent phase:
\begin{equation}
\phi_{\text{atm}}(\lambda) = \frac{2\pi}{\lambda} \int [n(\mathbf{r}, \lambda) - 1] d\ell
\end{equation}

Comparing phases across bands:
\begin{equation}
\Delta\phi(\lambda_1, \lambda_2) = \phi(\lambda_1) - \phi(\lambda_2)
\end{equation}
enables atmospheric correction without external monitoring.

\subsubsection{Emission Line Mapping}

For sources with strong emission lines (e.g., H\(\alpha\), [OIII]), multi-band operation simultaneously maps:
\begin{itemize}
\item Continuum emission (spatial distribution)
\item Line emission (kinematics, excitation)
\end{itemize}

Example: The structure of active galactic nucleus jets differs between continuum (synchrotron emission from the entire jet) and line emission (photoionised gas clouds).

\subsubsection{Color-Dependent Structure}

Stellar surfaces show colour-dependent structures due to:
\begin{itemize}
\item Limb darkening (wavelength-dependent)
\item Star spots (cooler regions appear darker at short wavelengths)
\item Chromospheric activity (UV emission from outer atmosphere)
\end{itemize}

Multi-band interferometry resolves these effects, measuring:
\begin{equation}
\theta_{\text{eff}}(\lambda) = \text{effective angular diameter as function of wavelength}
\end{equation}
\begin{figure}[htbp]
    \centering
    \includegraphics[width=\textwidth]{figures/molecular_search_space_analysis.png}
    \caption{\textbf{Molecular Search Space: Categorical Navigation Through Harmonic Networks.}
    \textbf{(A)} Three-dimensional S-entropy phase space showing 200 molecular states distributed
    across knowledge ($S_k$), time ($S_t$), and evolution ($S_e$) dimensions. Color gradient
    indicates total entropy $S_{\text{total}} = S_k + S_t + S_e$. Red star marks initial state,
    green star marks target state. Red trajectory shows optimal categorical path requiring only
    5 steps through high-dimensional state space. \textbf{(B)} Harmonic network graph of 30
    representative molecules connected by frequency similarity relationships. Node colors encode
    oscillation frequencies (40-100 THz range), edge thickness indicates harmonic coupling strength.
    Network density of 0.322 with average degree 9.3 enables efficient categorical navigation.
    Molecular clusters (e.g., nodes 0-6 in purple, nodes 24-29 in pink) represent frequency-similar
    species forming natural search neighborhoods. \textbf{(C)} Categorical path length distribution
    across all molecular pairs shows mean of 2.83 steps (median 2.0), with 95\% of paths requiring
    $\leq 6$ steps. This logarithmic scaling enables rapid navigation through $10^{25}$ atmospheric
    molecules. \textbf{(D)} Search efficiency analysis demonstrates logarithmic scaling with network
    size (blue circles), closely matching theoretical prediction $\langle \ell \rangle \propto \log N$
    (red dashed). Green triangles show corresponding search times at 1.67 ms per step, yielding
    total search times $< 20$ ms even for networks of $10^3$ molecules. \textbf{(E)} Independence
    principle validation: categorical distance vs. spatial distance shows near-zero correlation
    ($r = -0.005$), confirming that $d_{\text{cat}} \perp d_{\text{spatial}}$. This independence
    enables 20$\times$ faster-than-light categorical propagation without violating relativity, as
    categorical navigation operates in state space rather than physical space. \textbf{(F)} Example
    optimal path through S-entropy space from start (red star, $S_k=0$, $S_t=0$) to end (green star,
    $S_k=10$, $S_t=10$) via 7 intermediate steps. Yellow annotations show cumulative cost at each
    step, with total path cost of 18.16 and average step cost of 2.59. Path follows gradient of
    minimal S-entropy distance, demonstrating efficient categorical navigation strategy.}
    \label{fig:molecular_search_space}
    \end{figure}
    

\subsection{UV Coverage Enhancement}

The \(uv\)-plane coverage determines the quality of image reconstruction. For \(N\) stations and \(M\) spectral bands, the number of independent measurements is:
\begin{equation}
N_{\text{meas}} = \frac{N(N-1)}{2} \times M
\end{equation}

Example: 5 stations, 3 bands:
\begin{align}
N_{\text{baselines}} &= 10 \\
N_{\text{meas}} &= 10 \times 3 = 30
\end{align}

The multi-band measurements sample different regions of \(uv\)-space due to \(\mathbf{u}_n \propto 1/\lambda_n\), improving reconstruction without increasing number of stations.

\subsection{Bandwidth and Sensitivity}

Total detected photon rate integrates over all bands:
\begin{equation}
\dot{N}_{\text{photon}} = \int_{\nu_{\min}}^{\nu_{\max}} \frac{S(\nu) A_{\text{eff}}}{h\nu} d\nu
\end{equation}

For source with broad spectrum, multi-band detection increases signal-to-noise:
\begin{equation}
\text{SNR}_{\text{multi}} = \sqrt{\sum_n \text{SNR}_n^2} = \sqrt{M} \times \text{SNR}_{\text{single}}
\end{equation}
assuming equal SNR per band and \(M\) bands.

This represents \(\sqrt{3} \approx 1.7\times\) improvement for 3-band operation compared to single-band.

\subsection{Practical Implementation}

\textbf{Data Acquisition:}
Single photodetector time series \(I(t)\) contains all spectral information. No multiplexing required.

\textbf{Band Separation:}
Fourier transform of \(I(t)\) yields frequency components:
\begin{equation}
\tilde{I}(\nu) = \int I(t) e^{-2\pi i \nu t} dt
\end{equation}

Filter to select bands:
\begin{align}
I_{\text{UV}}(t) &= \text{IFT}\{\tilde{I}(\nu) \cdot H_{\text{UV}}(\nu)\} \\
I_{\text{Vis}}(t) &= \text{IFT}\{\tilde{I}(\nu) \cdot H_{\text{Vis}}(\nu)\} \\
I_{\text{IR}}(t) &= \text{IFT}\{\tilde{I}(\nu) \cdot H_{\text{IR}}(\nu)\}
\end{align}
where \(H_n(\nu)\) are digital filter functions.

\textbf{Categorical Encoding:}
Apply categorical state extraction to each band independently:
\begin{equation}
I_n(t) \to \mathcal{C}_n(t) \to \text{band-specific visibility } V_n
\end{equation}

\textbf{Combined Reconstruction:}
Image reconstruction uses all bands simultaneously:
\begin{equation}
I_{\text{source}}(\mathbf{s}) = \text{IFT}\left\{\sum_n V_n(\mathbf{u}_n)\right\}
\end{equation}

This yields a single image with information from all wavelengths or separate images per band if the chromatic structure is scientifically interesting.

\subsection{Comparison with Existing Multi-Wavelength Facilities}

\textbf{ALMA} (Atacama Large Millimeter/submillimeter Array): Multiple receiver bands, but requires changing receivers between observations. Cannot observe multiple bands simultaneously.

\textbf{VLA} (Very Large Array): It can observe up to 2 bands simultaneously with dual-receiver systems but requires hardware for each band.

\textbf{Optical/IR Interferometers}: Typically single-band operation, occasionally dual-band with dichroic beam-splitters.
