\section{Atmospheric Independence in Categorical Space}
\label{sec:atmospheric}

\subsection{Atmospheric Phase Distortion in Conventional Interferometry}

Electromagnetic radiation propagating through Earth's atmosphere experiences variations in the refractive index \(n(\mathbf{r}, t)\) due to turbulence, water vapour content, and temperature gradients. The accumulated phase distortion along the propagation path from the source to the detector is:
\begin{equation}
\phi_{\text{atm}} = \frac{2\pi}{\lambda} \int_{\text{path}} [n(\mathbf{r}, t) - 1] \, d\ell
\end{equation}

For interferometric baselines \(\mathbf{D}\), the differential atmospheric delay between stations produces decorrelation:
\begin{equation}
\langle e^{i\phi_{\text{atm},A}} e^{-i\phi_{\text{atm},B}} \rangle = e^{-D_{\phi}^2(\mathbf{D})/2}
\end{equation}
where \(D_{\phi}^2\) is the phase structure function. For baselines exceeding the atmospheric coherence length \(r_0 \sim 10\)--20 cm at optical wavelengths \cite{roddier1981atmospheric}, this correlation approaches zero unless corrected.

\begin{figure*}[htbp]
    \centering
    \includegraphics[width=\textwidth]{figures/Figure8_Atmospheric_Immunity.png}
    \caption{\textbf{Complete atmospheric immunity through categorical phase propagation: coherence, immunity factor, wavelength independence, and physical mechanism.}
    \textbf{(A)} Coherence vs. baseline: Conventional interferometry (colored dashed lines) shows exponential coherence decay with baseline length, dependent on Fried parameter $r_0$. Excellent seeing ($r_0 = 20$ cm, green) maintains coherence to $\sim 100$ m. Good seeing ($r_0 = 10$ cm, cyan) degrades by $D \sim 10$ m. Average seeing ($r_0 = 5$ cm, orange) and poor seeing ($r_0 = 2$ cm, red) show severe decorrelation at meter scales. 50\% coherence threshold (magenta dashed) occurs at baselines $< 10$ m for typical conditions. Categorical interferometry (black solid line) maintains coherence $\sim 1.0$ (perfect) across all baselines from 0.1 m to $10^5$ m (10,000 km operational baseline, vertical dashed line, cyan shaded region), independent of atmospheric conditions. This represents baseline-independent coherence through categorical space propagation.
    \textbf{(B)} Atmospheric immunity factor: Quantifies advantage of categorical over conventional approach as ratio of coherences: $\mathcal{I} = \gamma_{\text{cat}} / \gamma_{\text{conv}}$. For poor seeing ($r_0 = 5$ cm), immunity factor scales from $\sim 10$ at $D = 1$ m to $> 10^{15}$ at $D = 10^5$ km (operational point marked with star: $\mathcal{I} \sim 9.6 \times 10^{14}$, effectively infinite). Three advantage regimes shown: no advantage ($< 10\times$, magenta dashed), $10^6\times$ advantage (cyan dotted), and $10^9\times$ advantage (gray dotted). Operational baseline achieves $\sim 10^{15}\times$ advantage—atmospheric effects completely eliminated. This demonstrates that categorical immunity becomes more pronounced at longer baselines where conventional interferometry fails completely.
    \textbf{(C)} Wavelength dependence: Coherence at 10,000 km baseline vs. wavelength for conventional (red dashed line) and categorical (blue solid line) interferometry. Conventional coherence $\sim 0$ across all wavelengths (UV, visible, near-IR) due to atmospheric decorrelation overwhelming any wavelength dependence—atmosphere destroys coherence regardless of $\lambda$. Categorical coherence $\sim 0.98$ (constant) across entire electromagnetic spectrum from UV (200 nm) through visible (500 nm operational point, yellow highlight) to near-IR ($> 10^4$ nm). This wavelength independence enables simultaneous multi-band interferometry without atmospheric chromatic dispersion—all wavelengths maintain identical coherence because phase propagates in categorical space, not through atmosphere.
    \textbf{(D)} Physical mechanism comparison: Top panel shows conventional interferometry—photons travel from station A through turbulent atmosphere (pink wavy region showing phase scrambling) to station B, resulting in coherence $\sim 0$ (phase scrambled). Bottom panel shows categorical interferometry—phase information extracted at station A from local molecular oscillators (H$^+$ at 71 THz), propagates through categorical space (blue arrow, no physical path), and correlates with station B's categorical state. Atmosphere affects only local photon detection ($\sim 2\%$ efficiency loss) but baseline coherence maintained at 0.98 via H$^+$ synchronization. Key insight: Conventional phase travels through atmosphere (physical space); categorical phase travels through H$^+$ oscillator network (categorical space).}
    \label{fig:atmospheric_immunity}
    \end{figure*}

\subsection{Fried Parameter Limitation}

The Fried parameter \(r_0\) represents the aperture diameter over which atmospheric phase variance equals 1 radian\(^2\):
\begin{equation}
\left\langle [\phi(\mathbf{r}_1) - \phi(\mathbf{r}_2)]^2 \right\rangle = \left(\frac{|\mathbf{r}_1 - \mathbf{r}_2|}{r_0}\right)^{5/3}
\end{equation}
following Kolmogorov turbulence theory \cite{kolmogorov1941local}.

At visible wavelengths (\(\lambda \sim 500\) nm), typical conditions yield \(r_0 \sim 10\) cm. For interferometric baselines \(D \gg r_0\), atmospheric phase becomes completely decorrelated between stations, destroying fringe visibility unless:
\begin{enumerate}
\item Adaptive optics correct the wavefronts of each station.
\item Post-processing applies atmospheric phase models
\item Observations occur at longer wavelengths where \(r_0 \propto \lambda^{6/5}\) is larger
\end{enumerate}

All current optical interferometers operate with baselines \(D \lesssim 100r_0 \sim 10\) m to maintain partial coherence.

\subsection{Categorical Propagation Path}

In categorical interferometry, the information flow differs fundamentally:

\textbf{Detection Phase (Physical Space):}
Astronomical radiation propagates from the source through the atmosphere to station A, accumulating phase:
\begin{equation}
\phi_A^{\text{detected}} = \phi_{\text{source}} + \phi_{\text{atm},A} + \phi_{\text{geom},A}
\end{equation}

This detected phase is encoded into a categorical state:
\begin{equation}
\mathcal{C}_A(t) = \mathcal{C}_0 \exp[i\phi_A^{\text{detected}}(t)]
\end{equation}

\textbf{Transfer Phase (Categorical Space):}
The categorical state \(\mathcal{C}_A\) is transmitted to station B through categorical state prediction \cite{author2024prediction}. This transmission follows the path integral:
\begin{equation}
\mathcal{C}_B^{\text{predicted}}(t) = \mathcal{C}_A(t) \cdot \exp\left[i\int_{\mathbf{S}_A}^{\mathbf{S}_B} \omega_{\text{cat}}(\mathbf{S}) \, d\mathbf{S}\right]
\end{equation}
where the integration occurs over categorical coordinates \(\mathbf{S} = (S_k, S_t, S_e)\), not physical coordinates \(\mathbf{r}\).

Crucially, the atmospheric refractive index \(n(\mathbf{r}, t)\) is a function of physical position, not categorical coordinates. Therefore:
\begin{equation}
\frac{\partial n}{\partial \mathbf{S}} = 0
\end{equation}
Atmospheric phase contributions do not enter the categorical path integral.

\subsection{Mathematical Proof of Atmospheric Immunity}

The visibility correlation in conventional interferometry involves both atmospheric paths:
\begin{equation}
V_{\text{conv}} = \left\langle E_A^{\text{atm}} E_B^{*,\text{atm}} \right\rangle = V_0 \left\langle e^{i(\phi_{\text{atm},A} - \phi_{\text{atm},B})} \right\rangle
\end{equation}

In categorical interferometry, the correlation is:
\begin{equation}
V_{\text{cat}} = \left\langle \mathcal{C}_A^{\text{detected}} \mathcal{C}_B^{*,\text{detected}} \right\rangle
\end{equation}

Expanding the detected states:
\begin{align}
V_{\text{cat}} &= \left\langle \mathcal{C}_0 e^{i\phi_{\text{source}}} e^{i\phi_{\text{atm},A}} e^{i\phi_{\text{geom},A}} \cdot \mathcal{C}_0^* e^{-i\phi_{\text{source}}} e^{-i\phi_{\text{atm},B}} e^{-i\phi_{\text{geom},B}} \right\rangle \\
&= |\mathcal{C}_0|^2 \left\langle e^{i(\phi_{\text{atm},A} - \phi_{\text{atm},B})} \right\rangle e^{i(\phi_{\text{geom},A} - \phi_{\text{geom},B})}
\end{align}

\textbf{Key distinction}: The atmospheric phases \(\phi_{\text{atm},A}\) and \(\phi_{\text{atm},B}\) are measured \textit{locally} at each station during detection. They are not correlated because:
\begin{enumerate}
\item Station A detects astronomical light that has traversed the atmosphere above station A
\item Station B detects astronomical light that has traversed the atmosphere above station B
\item No light propagates between stations through the atmosphere
\end{enumerate}

In conventional interferometry, the signals that interfere both traveled through atmosphere. In categorical interferometry, the \textit{information} about the detected signal travels through categorical space.

\begin{figure*}[htbp]
    \centering
    \includegraphics[width=\textwidth]{figures/atmospheric_immunity_validation.png}
    \caption{\textbf{Categorical interferometry exhibits complete atmospheric immunity across kilometer-scale baselines.} \textbf{(A)} Visibility degradation comparison between conventional and categorical approaches across baseline distances from 10$^{-3}$ to 10$^{4}$~km. Conventional systems show catastrophic degradation (visibility $<$10$^{-100}$) beyond 100~km under all atmospheric conditions, while categorical systems maintain perfect visibility (coherence $\approx$1) independent of baseline, exceeding the claimed 10,000~km operational range. \textbf{(B)} Atmospheric immunity factor quantified across baseline distances for four atmospheric quality levels (excellent, good, average, poor). Categorical approach demonstrates immunity factors exceeding 10$^{9}$ for poor conditions at 10$^{4}$~km baseline, with no advantage threshold (dashed line) indicating parity with conventional systems. \textbf{(C)} Phase variance accumulation showing conventional systems accumulate variance as $\sim$baseline$^{2}$ (reaching 10$^{14}$~rad$^{2}$ at 10$^{4}$~km), while categorical systems remain timing-limited at $\sim$10$^{2}$~rad$^{2}$ independent of atmospheric quality or distance. \textbf{(D)} Effective baseline limits comparing conventional (blue) versus categorical (orange) approaches across atmospheric conditions. Categorical systems extend operational baselines by 3--4 orders of magnitude: from $\sim$10$^{-4}$~km (excellent) to 10$^{1}$~km (poor) for conventional systems, versus uniform 10$^{1}$~km capability for categorical systems regardless of atmospheric quality.}
    \label{fig:atmospheric_immunity_validation}
    \end{figure*}

\subsection{Local vs Distributed Atmospheric Effects}

Define:
\begin{itemize}
\item \(\ell_{\text{local}}\): Atmospheric path length above each station (\(\sim 10\) km vertical)
\item \(D\): Baseline separation (\(10^3\)--\(10^7\) m horizontal)
\end{itemize}

\textbf{Conventional Interferometry:} Atmospheric decorrelation scales with baseline:
\begin{equation}
\sigma_{\phi,\text{conv}}^2 \sim \left(\frac{D}{r_0}\right)^{5/3}
\end{equation}
For \(D = 1000\) km and \(r_0 = 10\) cm: \(\sigma_{\phi,\text{conv}} \sim 10^{7}\) rad, completely destroying coherence.

\textbf{Categorical Interferometry:} Atmospheric variance is local only:
\begin{equation}
\sigma_{\phi,\text{cat}}^2 \sim \left(\frac{D_{\text{aperture}}}{r_0}\right)^{5/3}
\end{equation}
For a station aperture of \(D_{\text{aperture}} \sim 10\) cm: \(\sigma_{\phi,\text{cat}} \sim 1\) rad, maintaining coherence.

The improvement factor:
\begin{equation}
\frac{\sigma_{\phi,\text{conv}}}{\sigma_{\phi,\text{cat}}} \sim \left(\frac{D}{D_{\text{aperture}}}\right)^{5/6} \sim 10^{6}
\end{equation}
for continental baselines.

\subsection{Experimental Verification Protocol}

To verify atmospheric independence:

\textbf{Control Experiment:} Operate two stations separated by \(D = 1\) km on the same clear night. Measure fringe visibility as a function of:
\begin{itemize}
\item Zenith angle (changes atmospheric path length)
\item Time (changes in atmospheric turbulence strength)
\item Wavelength (changes \(r_0 \propto \lambda^{6/5}\))
\end{itemize}

\textbf{Prediction (Conventional):} Visibility should decrease as \(\cos^{-5/3}(\text{zenith})\) due to the increased atmospheric path.

\textbf{Prediction (Categorical):} Visibility should remain constant (within photon noise) for all zenith angles, since atmospheric effects are local to each station and do not affect categorical transfer.

\textbf{Quantitative Metric:} Define the atmospheric immunity factor:
\begin{equation}
\mathcal{I}_{\text{atm}} = \frac{V_{\text{measured}}(\theta_z = 60°)}{V_{\text{measured}}(\theta_z = 0°)}
\end{equation}

For conventional interferometry: \(\mathcal{I}_{\text{atm}} < 0.1\) (strong degradation).
For categorical interferometry: \(\mathcal{I}_{\text{atm}} > 0.9\) (minimal degradation).

\subsection{Weather Independence}

Conventional optical interferometry requires:
\begin{itemize}
\item Clear skies at all stations simultaneously
\item Low wind speeds (\(< 10\) m/s) for stable seeing
\item Low humidity (\(< 50\%\)) reduces water vapour absorption
\end{itemize}

These constraints limit operational time to \(\sim 20\%\) of nights at typical mid-latitude sites \cite{walker1987high}.

Categorical interferometry relaxes these constraints:
\begin{itemize}
\item Light clouds are acceptable (reduced photon flux, but coherence is maintained)
\item High winds do not affect categorical transfer
\item High humidity affects detection sensitivity but not coherence
\end{itemize}

Operational time fraction increases to \(\sim 60\%\)--70\%, a factor of 3--4 improvement in observing efficiency.
