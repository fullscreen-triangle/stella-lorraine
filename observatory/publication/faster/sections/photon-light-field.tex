\section{Light Field Equivalence and Geometric Reconstruction Theory}

\subsection{Motivation: Spatial Position as Electromagnetic Context}

In conventional geometric analysis, spatial position is treated as an absolute coordinate in three-dimensional space. However, from an information-theoretic perspective, spatial position can be equivalently characterised by the complete electromagnetic field experienced at that location—the \textit{light field}. This recharacterization suggests a profound equivalence: two spatial locations experiencing identical light fields are, in a fundamental sense, \textit{the same location} from the perspective of electromagnetic information content.

\begin{principle}[Spatial-Electromagnetic Duality]
Spatial position $\mathbf{r} \in \mathbb{R}^3$ can be equivalently characterized by:
\begin{enumerate}[(i)]
\item \textbf{Geometric characterization}: Cartesian coordinates $(x, y, z)$
\item \textbf{Electromagnetic characterization}: Complete spherical light field $\mathcal{L}(\mathbf{r})$
\end{enumerate}

When two locations share identical light fields, $\mathcal{L}(\mathbf{r}_A) = \mathcal{L}(\mathbf{r}_B)$, they are electromagnetically indistinguishable.
\end{principle}

\subsection{Mathematical Representation of Light Fields}

\begin{definition}[Complete Spherical Light Field]
\label{def:spherical_light_field}
A \textbf{complete spherical light field} at spatial position $\mathbf{r} \in \mathbb{R}^3$ and time $t \in \mathbb{R}$ is defined as:

\begin{equation}
\mathcal{L}(\mathbf{r}, t) = \oint_{S^2} \mathcal{I}(\theta, \phi, \lambda, t; \mathbf{r}) \, d\Omega
\end{equation}

where:
\begin{itemize}
\item $\mathcal{I}(\theta, \phi, \lambda, t; \mathbf{r})$: Electromagnetic intensity at spherical angles $(\theta, \phi) \in [0, \pi] \times [0, 2\pi]$, wavelength $\lambda \in \mathbb{R}^+$, time $t$, observed at position $\mathbf{r}$
\item $S^2$: Unit sphere representing all incoming directions
\item $d\Omega = \sin\theta \, d\theta \, d\phi$: Differential solid angle element
\end{itemize}
\end{definition}

\begin{remark}[Informational Content]
The complete light field $\mathcal{L}(\mathbf{r}, t)$ encodes:
\begin{enumerate}
\item \textbf{Angular information}: Electromagnetic intensity from all directions $(\theta, \phi) \in S^2$
\item \textbf{Spectral information}: Wavelength-dependent intensity $\mathcal{I}(\lambda)$ across the electromagnetic spectrum
\item \textbf{Temporal information}: Time evolution $\mathcal{I}(t)$ capturing dynamic field variations
\item \textbf{Polarization information}: Vector field components (implicit in $\mathcal{I}$)
\end{enumerate}

This represents the complete electromagnetic context at position $\mathbf{r}$.
\end{remark}

\subsection{Spherical Harmonic Decomposition}

Light fields admit a natural decomposition in spherical harmonic basis functions.

\begin{definition}[Spherical Harmonic Expansion of Light Fields]
\label{def:spherical_harmonic_expansion}
For fixed wavelength $\lambda$ and time $t$, the angular distribution decomposes as:

\begin{equation}
\mathcal{I}(\theta, \phi; \lambda, t, \mathbf{r}) = \sum_{l=0}^{\infty} \sum_{m=-l}^{l} A_{lm}(\lambda, t, \mathbf{r}) Y_l^m(\theta, \phi)
\end{equation}

where:
\begin{itemize}
\item $Y_l^m(\theta, \phi)$: Spherical harmonic basis functions of degree $l$ and order $m$
\item $A_{lm}(\lambda, t, \mathbf{r})$: Complex expansion coefficients encoding field information
\item $l \in \mathbb{N}_0$: Degree index (representing angular frequency)
\item $m \in \{-l, -l+1, \ldots, l-1, l\}$: Order index
\end{itemize}
\end{definition}

\begin{theorem}[Completeness of Spherical Harmonic Representation]
\label{thm:spherical_harmonic_completeness}
The spherical harmonic basis $\{Y_l^m : l \in \mathbb{N}_0, |m| \leq l\}$ is complete for $L^2(S^2)$. Therefore, any square-integrable light field angular distribution admits unique decomposition in this basis with coefficients:

\begin{equation}
A_{lm}(\lambda, t, \mathbf{r}) = \int_{S^2} \mathcal{I}(\theta, \phi; \lambda, t, \mathbf{r}) \overline{Y_l^m(\theta, \phi)} \, d\Omega
\end{equation}

where $\overline{Y_l^m}$ denotes the complex conjugate.
\end{theorem}

\begin{proof}
This follows from the fundamental completeness theorem for spherical harmonics on the sphere $S^2$. The functions $\{Y_l^m\}$ form an orthonormal basis:
\begin{equation}
\int_{S^2} Y_l^m(\theta, \phi) \overline{Y_{l'}^{m'}(\theta, \phi)} \, d\Omega = \delta_{ll'} \delta_{mm'}
\end{equation}

By the Hilbert space projection theorem, any $\mathcal{I} \in L^2(S^2)$ can be uniquely expressed as a convergent series in this basis. $\square$
\end{proof}

\subsection{Light Field Equivalence Principle}

\begin{definition}[Electromagnetic Equivalence]
\label{def:em_equivalence}
Two spatial positions $\mathbf{r}_A, \mathbf{r}_B \in \mathbb{R}^3$ are \textbf{electromagnetically equivalent} at time $t$ if their light fields coincide:

\begin{equation}
\mathcal{L}(\mathbf{r}_A, t) = \mathcal{L}(\mathbf{r}_B, t)
\end{equation}

Equivalently, in spherical harmonic representation:

\begin{equation}
A_{lm}(\lambda, t, \mathbf{r}_A) = A_{lm}(\lambda, t, \mathbf{r}_B) \quad \forall l, m, \lambda
\end{equation}
\end{definition}

\begin{theorem}[Light Field Equivalence Principle]
\label{thm:light_field_equivalence}
Electromagnetically equivalent positions $\mathbf{r}_A \sim_{\text{EM}} \mathbf{r}_B$ are indistinguishable by any electromagnetic measurement performed locally at those positions.
\end{theorem}


    \begin{figure*}[htbp]
        \centering
        \includegraphics[width=\textwidth]{figures/complementarity_analysis_lower_half.png}
        \caption{
        \textbf{Complementarity analysis of numerical and CV methods: Feature space projections, cross-method correlations, and method performance across spectra.}
        \textbf{(Panel G)} Feature space (PCA) showing PC2 ($-3.5$--$+1.5$, 23.1\% variance) vs. PC1 ($-2$--$+7$, 75.5\% variance). Six spectra labeled S100--S105 shown as purple circles. Cluster of four spectra (S100--S103) at left (PC1 $\sim -1$ to $0$, PC2 $\sim -3$ to $+1$). S105 isolated at right (PC1 $\sim +6$, PC2 $\sim -0.2$). Legend shows orange (Numerical Better), purple (CV Better), gray (Equal). All spectra purple-coded indicating CV method superiority. Annotation: ``G. Feature Space (PCA), Numerical Better, CV Better, Equal, PC2 (23.1\%), PC1 (75.5\%).''
        \textbf{(Panel H)} Feature space (t-SNE) showing t-SNE Dimension 2 ($-30$--$+60$) vs. Dimension 1 ($-70$--$+30$). Six spectra distributed: S105 (bottom-left, $\sim -27, -27$), S100 (top-center, $\sim -40, +55$), S101 (center, $\sim -20, 0$), S102 (upper-right, $\sim 0, +13$), S104 (right, $\sim +20, -2$), S103 (bottom-right, $\sim +10, -27$). Greater separation than PCA indicates nonlinear structure. Annotation: ``H. Feature Space (t-SNE), t-SNE Dimension 2, t-SNE Dimension 1.''
        \textbf{(Panel I)} Feature importance (PCA) showing horizontal bars for 11 features. Top features: Shannon Entropy (orange, $\sim 0.095$, longest), S\_knowledge ($\mu$) (pink, $\sim 0.092$), Velocity ($\mu$) (pink, $\sim 0.090$), Peak Count (orange, $\sim 0.088$). Bottom features: Gini Coeff (orange, $\sim 0.025$, shortest). Orange bars indicate numerical features, pink bars indicate CV features. Legend at right. CV features dominate top importance. Annotation: ``I. Feature Importance (PCA), Shannon Entropy, S\_knowledge ($\mu$), Velocity ($\mu$), Peak Count, S\_time ($\mu$), S\_knowledge ($\sigma$), Radius ($\mu$), S\_entropy ($\mu$), S\_entropy ($\sigma$), S\_time ($\sigma$), Gini Coeff, Numerical Features, CV Features, Feature Importance.''
        \textbf{(Panel J)} Cross-method feature correlation showing two bars. Left bar (Peak Count vs. Droplet Count): teal, $r = 1.000$, perfect correlation. Right bar (Shannon Entropy vs. Mean S\_entropy): teal, $r = 0.951$, strong correlation. Both exceed moderate correlation threshold (gray dashed line at $\sim 0.6$). Text annotation: ``Strong correlation, Moderate correlation.'' Demonstrates high inter-method agreement. Annotation: ``J. Cross-Method Feature Correlation, $r = 1.000$, $r = 0.951$, Pearson Correlation ($r$).''
        \textbf{(Panel K)} Method complementarity by spectrum showing horizontal bars for six spectra. X-axis: Complementarity Score ($-0.35$--$0.00$). All bars salmon-colored, extending leftward (negative scores). S104 shows highest complementarity (shortest bar, $\sim -0.05$). S101 shows lowest (longest bar, $\sim -0.33$). Green box annotation at top: ``High score = methods complement well.'' Negative scores indicate weak complementarity overall. Annotation: ``K. Method Complementarity by Spectrum, High score = methods complement well, S104, S105, S103, S102, S100, S101, Complementarity Score.''
        \textbf{(Panel L)} Summary and recommendations text box with salmon background: ``COMPLEMENTARITY ANALYSIS SUMMARY. METHOD PERFORMANCE: Numerical better: 0/6 spectra (0.0\%), CV better: 6/6 spectra (100.0\%), Equal performance: 0/6 spectra (0.0\%). MEAN CONFIDENCE SCORES: Numerical method: 0.269, CV method: 0.805, Combined method: 0.537, Improvement: -33.3\%. COMPLEMENTARITY: Mean complementarity score: -0.330, Methods show weak complementarity. RECOMMENDATIONS: $\checkmark$ Use NUMERICAL method for: Simple spectra, high-throughput. $\checkmark$ Use CV method for: Complex spectra, isobaric compounds. $\checkmark$ Use COMBINED approach for: Maximum confidence, novel compounds.'' Annotation: ``L. Summary and Recommendations.''
        }
        \label{fig:complementarity_analysis}
        \end{figure*}


\begin{proof}
Electromagnetic measurements at position $\mathbf{r}$ are functionals $\Phi: \mathcal{L}(\mathbf{r}, t) \to \mathbb{R}$ mapping light field to observable values. Examples:
\begin{itemize}
\item Intensity measurement: $\Phi_{\text{int}}[\mathcal{L}] = \int_{S^2} \int_{\lambda} \mathcal{I}(\theta, \phi, \lambda) \, d\lambda \, d\Omega$
\item Directional measurement: $\Phi_{\text{dir}}[\mathcal{L}] = \mathcal{I}(\theta_0, \phi_0, \lambda_0)$ for specified $(\theta_0, \phi_0, \lambda_0)$
\item Spectral measurement: $\Phi_{\text{spec}}[\mathcal{L}] = \int_{S^2} \mathcal{I}(\theta, \phi, \lambda_0) \, d\Omega$ for specified $\lambda_0$
\end{itemize}

If $\mathcal{L}(\mathbf{r}_A, t) = \mathcal{L}(\mathbf{r}_B, t)$, then for any electromagnetic functional $\Phi$:
\begin{equation}
\Phi[\mathcal{L}(\mathbf{r}_A, t)] = \Phi[\mathcal{L}(\mathbf{r}_B, t)]
\end{equation}

Therefore, all electromagnetic measurements yield identical results, establishing indistinguishability. $\square$
\end{proof}

\begin{remark}[Photon Reference Frame Connection]
In relativistic mechanics, photon worldlines satisfy $ds^2 = 0$ (null geodesics), implying zero proper time: $d\tau = 0$. From the photon's perspective, emission and absorption events are \textit{simultaneous}. When two spatial positions experience identical light fields—i.e., they interact with photons identically—they share the same photon reference frame relationships. This provides physical motivation for electromagnetic equivalence: positions with identical light fields have identical photon-mediated information access.
\end{remark}

\subsection{Geometric Reconstruction from Light Fields}

The equivalence principle suggests that spatial geometry can be reconstructed from light field data.

\begin{definition}[Light Field Sampling]
A \textbf{multi-angle, multi-band light field sampling} at position $\mathbf{r}$ consists of measurements:

\begin{equation}
\mathcal{S}(\mathbf{r}) = \{\mathcal{I}(\theta_i, \phi_j, \lambda_k, t; \mathbf{r}) : i \in [1, N_{\theta}], j \in [1, N_{\phi}], k \in [1, N_{\lambda}]\}
\end{equation}

where:
\begin{itemize}
\item $N_{\theta}, N_{\phi}$: Number of angular samples (spatial resolution)
\item $N_{\lambda}$: Number of wavelength bands (spectral resolution)
\item Total samples: $N_{\text{total}} = N_{\theta} \times N_{\phi} \times N_{\lambda}$
\end{itemize}
\end{definition}

\begin{theorem}[Sampling Sufficiency for Reconstruction]
\label{thm:sampling_sufficiency}
For light field band-limited to maximum spherical harmonic degree $L_{\max}$ and wavelength range $[\lambda_{\min}, \lambda_{\max}]$, the sampling $\mathcal{S}(\mathbf{r})$ with:

\begin{align}
N_{\theta} &\geq L_{\max} + 1 \\
N_{\phi} &\geq 2L_{\max} + 1 \\
N_{\lambda} &\geq \frac{\lambda_{\max} - \lambda_{\min}}{\Delta\lambda_{\min}}
\end{align}

is sufficient for perfect reconstruction of $\mathcal{L}(\mathbf{r}, t)$ within the specified bandwidth.
\end{theorem}

\begin{proof}
\textbf{Angular reconstruction}: Spherical harmonic degree $l$ has $2l+1$ independent orders $m \in [-l, l]$. Total coefficients up to degree $L_{\max}$:
\begin{equation}
N_{\text{coeff}} = \sum_{l=0}^{L_{\max}} (2l+1) = (L_{\max} + 1)^2
\end{equation}

By Nyquist-Shannon theorem on the sphere, uniform sampling with $N_{\theta} \geq L_{\max} + 1$ and $N_{\phi} \geq 2L_{\max} + 1$ provides:
\begin{equation}
N_{\text{samples}} = N_{\theta} \times N_{\phi} \geq (L_{\max} + 1)(2L_{\max} + 1) > (L_{\max} + 1)^2 = N_{\text{coeff}}
\end{equation}

guaranteeing unique coefficient determination.

\textbf{Spectral reconstruction}: Wavelength sampling at Nyquist rate $\Delta\lambda_{\min}$ (determined by spectral features) ensures reconstruction across $[\lambda_{\min}, \lambda_{\max}]$.

Combined angular-spectral sampling provides complete light field reconstruction. $\square$
\end{proof}

\subsection{Categorical Encoding of Light Fields}

Light fields can be encoded as categorical states via S-entropy coordinates, enabling the application of categorical completion and triangular amplification mechanisms.

\begin{definition}[Categorical Light Field State]
\label{def:categorical_light_field}
For light field $\mathcal{L}(\mathbf{r}, t)$ with spherical harmonic coefficients $\{A_{lm}(\lambda_k)\}$ across wavelength bands $\{\lambda_k : k \in [1, N_{\lambda}]\}$, the \textbf{categorical state} is:

\begin{equation}
C_{\mathcal{L}}(\mathbf{r}) = \left\{ (s_{k}^{(k)}, s_t^{(k)}, s_e^{(k)}) : k \in [1, N_{\lambda}] \right\}
\end{equation}

where each wavelength band $\lambda_k$ maps to S-entropy coordinates:

\begin{align}
s_k^{(k)} &= H(\{A_{lm}(\lambda_k)\}) + I_{\text{angular}}(\{A_{lm}\}) \\
s_t^{(k)} &= \langle t_{\text{coherence}} \rangle(\lambda_k) + \Delta t_{\text{variation}} \\
s_e^{(k)} &= S_{\text{spectral}}(\lambda_k) + S_{\text{polarization}}
\end{align}

where $H$ denotes Shannon entropy, $I_{\text{angular}}$ quantifies angular information content, $\langle t_{\text{coherence}} \rangle$ measures temporal coherence, and $S_{\text{spectral}}, S_{\text{polarization}}$ encodes spectral and polarisation entropy.
\end{definition}

\begin{theorem}[Categorical Representation Completeness]
\label{thm:categorical_light_field_completeness}
The categorical encoding $C_{\mathcal{L}}(\mathbf{r})$ preserves sufficient information for light field reconstruction up to equivalence class precision determined by S-entropy quantization.
\end{theorem}

\begin{proof}
By Theorem 3.X (Section 3: S-coordinates are sufficient statistics), the tri-dimensional S-entropy coordinates compress infinite oscillatory information into three finite values while preserving optimality for categorical navigation.

For light field encoding:
\begin{itemize}
\item $s_k$: Captures angular and spectral information content (knowledge dimension)
\item $s_t$: Captures temporal dynamics and coherence (time dimension)
\item $s_e$: Captures disorder and constraints (entropy dimension)
\end{itemize}

Each wavelength band's S-coordinates encode the essential geometric and spectral information. The set $\{(s_k^{(k)}, s_t^{(k)}, s_e^{(k)}) : k \in [1, N_{\lambda}]\}$ therefore provides sufficient statistics for light field characterisation within categorical equivalence classes.

Reconstruction: Given $C_{\mathcal{L}}(\mathbf{r})$, the inverse mapping:
\begin{equation}
C_{\mathcal{L}}^{-1}: \{(s_k^{(k)}, s_t^{(k)}, s_e^{(k)})\} \to \{\hat{A}_{lm}(\lambda_k)\} \to \hat{\mathcal{L}}(\mathbf{r}, t)
\end{equation}

recovers light field $\hat{\mathcal{L}}$ satisfying $\mathcal{L} \sim_{\text{cat}} \hat{\mathcal{L}}$ (categorical equivalence). $\square$
\end{proof}

\begin{figure*}[htbp]
    \centering
    \includegraphics[width=0.95\textwidth]{figures/bmd_equivalence_20251105_124315.png}
    \caption{Multi-pathway convergence analysis validating BMD (Biological Maxwell Demon) equivalence across four independent computational methods. \textbf{Top left:} Variance convergence trajectories over $50$ iterations show all four pathways (visual processing, spectral analysis, semantic embedding, hardware sampling) converging to mean final variance $\sim 3.2 \times 10^7$ (black dashed line). \textbf{Top center:} Final variance by pathway: spectral analysis shows highest variance $\sim 1.3 \times 10^8$, while visual processing, semantic embedding, and hardware sampling cluster near mean $3.2 \times 10^7$ (black dashed line). \textbf{Top right:} Relative deviations from mean show visual processing ($-30\%$, coral) and hardware sampling ($+40\%$, coral) exceed $10\%$ threshold (gray dashed line), while semantic embedding ($-20\%$, teal) and spectral analysis ($+300\%$, teal) show larger deviations. \textbf{Middle left:} Pairwise equivalence matrix reveals diagonal self-equivalence (green, score $1.000$) with off-diagonal cross-pathway equivalence $0.800$--$0.975$ (red-yellow gradient), indicating high but incomplete convergence. \textbf{Middle center:} Statistical validation: F-statistic $4.09 \times 10^{17}$ with P-value $0.000000$ confirms significant variance differences; mean variance $3.20 \times 10^7$, variance spread $5.54 \times 10^7$, relative spread $1.73$; equivalence status NOT CONFIRMED, theorem validation $\text{Var}(\Pi_1) = \text{Var}(\Pi_2) = \text{Var}(\Pi_3) = \text{Var}(\Pi_4)$ INCOMPLETE. \textbf{Bottom right:} Convergence rates by pathway show exponential decay: hardware sampling and visual processing (coral) converge fastest ($\sim 10^{-17}$ rate), semantic embedding and spectral analysis (teal) converge slower ($\sim 10^{-18}$ to $10^{-17}$ rate).}
    \label{fig:bmd_equivalence}
    \end{figure*}


\subsection{Multi-Band Parallel Reconstruction}

Each wavelength band provides independent geometric information, enabling parallel reconstruction processes.

\begin{definition}[Per-Band Geometric Validation]
For wavelength band $\lambda_k$, the \textbf{band-specific reconstruction} operates on:

\begin{equation}
\mathcal{L}_k(\mathbf{r}, t) = \sum_{l=0}^{L_{\max}} \sum_{m=-l}^{l} A_{lm}(\lambda_k, t, \mathbf{r}) Y_l^m(\theta, \phi)
\end{equation}

Each band $\mathcal{L}_k$ constitutes an independent measurement of the geometric configuration at $\mathbf{r}$.
\end{definition}

\begin{theorem}[Independent Band Validation]
\label{thm:independent_band_validation}
For $N_{\lambda}$ wavelength bands, successful reconstruction across all bands provides $N_{\lambda}$ independent validations of geometric equivalence. The combined confidence level is:

\begin{equation}
P_{\text{combined}} = 1 - (1 - P_{\text{single}})^{N_{\lambda}}
\end{equation}

where $P_{\text{single}}$ is the single-band validation confidence.
\end{theorem}
