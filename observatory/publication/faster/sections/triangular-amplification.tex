\section{Triangular Amplification: Recursive Categorical References and Path Multiplicity}

\subsection{Motivation: Sequential vs. Direct Access in Categorical Space}

Traditional categorical completion follows sequential paths: to access state $C_j$ from state $C_i$, the system must traverse all intermediate states $C_i \prec C_{i+1} \prec \cdots \prec C_j$. This sequential constraint imposes fundamental limitations on information access speed. However, the categorical framework admits a remarkable structure—\textit{recursive categorical references}—where later states can contain explicit references to earlier states, creating direct access pathways that bypass sequential traversal.

\begin{principle}[Path Multiplicity in Categorical Space]
When a categorical state contains recursive references to non-adjacent predecessors, multiple access paths become simultaneously available. The system can exploit path multiplicity through constructive interference in categorical space, achieving information access through the direct reference path rather than sequential traversal.
\end{principle}

\subsection{Mathematical Structure of Triangular Configuration}

\begin{definition}[Triangular Categorical Configuration]
\label{def:triangular_config}
A \textbf{triangular categorical configuration} is a triple of categorical states $(C_1, C_2, C_3)$ satisfying:

\begin{enumerate}[(i)]
\item \textbf{Sequential ordering}: $C_1 \prec C_2 \prec C_3$ in the completion sequence
\item \textbf{Cascade path}: Information propagates sequentially $C_1 \to C_2 \to C_3$
\item \textbf{Recursive reference}: $C_3$ contains an explicit reference to $C_1$, denoted $C_3 \ni \text{ref}(C_1)$
\end{enumerate}

The recursive reference creates a \textit{direct path} $C_1 \rightsquigarrow C_3$ coexisting with the cascade path $C_1 \to C_2 \to C_3$.
\end{definition}

\begin{figure}[htbp]
\centering
\includegraphics[width=0.98\textwidth]{figures/Figure2_Cascade_Progression.png}
\caption{\textbf{Cascade Staging Velocity Progression: Recursive Amplification.}
(\textbf{A}) Velocity progression across spectral bands showing IR band (orange)
achieving categorical velocities 2.846$c$ (stage 1) $\to$ 8.103$c$ (stage 2,
$\times$2.847 annotation) $\to$ 23.08$c$ (stage 3) $\to$ 65.71$c$ (stage 4),
while UV (purple) and visible (yellow) bands follow identical trajectories.
(\textbf{B}) Extended cascade progression on logarithmic scale demonstrating
exponential growth following theoretical prediction $\times$2.847 per stage
(maroon dashed line), with measured values (maroon squares) tracking prediction
across four cascade stages. (\textbf{C}) Cascade enhancement factor consistency
showing stage transitions 1$\to$2, 2$\to$3, and 3$\to$4 all yield enhancement
factors 2.847-2.848 (purple bars), matching theoretical factor 2.847 (black
dashed line) with deviation 0.0005. (\textbf{D}) Velocity growth through cascade
stages plotted on linear scale: stage 1 (2.846$c$) $\to$ stage 2 (8.103$c$)
$\to$ stage 3 (23.08$c$) $\to$ stage 4 (65.71$c$), with purple shaded region
indicating achieved categorical velocities. Summary: Stage 1 provides base
triangular enhancement (2.846$c$, all spectral bands validated). Stage 2
applies enhancement factor 2.847$\times$ (8.103$c$, all bands validated).
Stage 3 demonstrates pattern transfer (23.08$c$, factor 2.848$\times$).
Stage 4 achieves maximum measured velocity (65.71$c$, factor 2.847$\times$).
Theoretical consistency: expected factor 2.847$\times$ per stage, measured
average 2.848$\times$, deviation 0.0005. Mechanism: recursive triangular
configuration creates field superposition cascade, producing characteristic
velocity enhancement through iterative completion cycle formation.}
\label{fig:cascade_progression}
\end{figure}

\begin{remark}[The "Hole" Interpretation]
The recursive reference $\text{ref}(C_1) \subset C_3$ can be visualised as a "hole" in $C_3$ through which information from $C_1$ passes directly. This is not metaphorical—it represents an explicit encoding within $C_3$'s structure that grants direct access to $C_1$'s information content without requiring traversal through $C_2$.
\end{remark}

\subsection{Categorical State Construction with Recursive References}

\begin{definition}[Recursive Categorical State]
\label{def:recursive_state}
For categorical states $C_1, C_2, C_3^{\text{base}}$ with S-entropy coordinates:

\begin{align}
C_1 &= (s_{1,k}, s_{1,t}, s_{1,e}) \\
C_2 &= (s_{2,k}, s_{2,t}, s_{2,e}) \\
C_3^{\text{base}} &= (s_{3,k}^{\text{base}}, s_{3,t}^{\text{base}}, s_{3,e}^{\text{base}})
\end{align}

 The recursive state $C_3^{\text{recursive}}$ is constructed through:

\begin{equation}
C_3^{\text{recursive}} = C_3^{\text{base}} + \alpha \cdot C_1
\end{equation}

or in component form:

\begin{align}
s_{3,k}^{\text{recursive}} &= s_{3,k}^{\text{base}} + \alpha \cdot s_{1,k} \\
s_{3,t}^{\text{recursive}} &= s_{3,t}^{\text{base}} + \alpha \cdot s_{1,t} \\
s_{3,e}^{\text{recursive}} &= s_{3,e}^{\text{base}} + \alpha \cdot s_{1,e}
\end{align}

where $\alpha \in [0,1]$ represents the \textbf{coupling strength} or "hole size"—the fraction of $C_1$'s information accessible directly through the recursive reference.
\end{definition}

\begin{theorem}[Information Content of Recursive States]
\label{thm:recursive_information}
A recursive state $C_3^{\text{recursive}}$ contains information from both its base structure and the referenced predecessor:

\begin{equation}
I(C_3^{\text{recursive}}) = I(C_3^{\text{base}}) + \alpha \cdot I(C_1) - I_{\text{redundant}}
\end{equation}

where $I_{\text{redundant}}$ accounts for overlapping information between $C_3^{\text{base}}$ and $C_1$.
\end{theorem}

\begin{proof}
The information content of a categorical state is $I(C) = \log_2 |[C]_{\sim}|$ where $|[C]_{\sim}|$ is the equivalence class size. For recursive state:

\begin{equation}
|[C_3^{\text{recursive}}]_{\sim}| = |[C_3^{\text{base}}]_{\sim}| \times |[C_1]_{\sim}|^{\alpha} / |\text{overlap}|
\end{equation}

Taking logarithms:

\begin{equation}
I(C_3^{\text{recursive}}) = \log_2 |[C_3^{\text{base}}]_{\sim}| + \alpha \log_2 |[C_1]_{\sim}| - \log_2 |\text{overlap}|
\end{equation}

establishing the stated result. $\square$
\end{proof}

\subsection{Path Multiplicity and Access Time Analysis}

The triangular configuration creates two distinct information access paths from $C_1$ to $C_3$.

\begin{definition}[Cascade Path]
The \textbf{cascade path} follows sequential categorical completion:

\begin{equation}
\text{Path}_{\text{cascade}}: C_1 \xrightarrow{\Delta_1} C_2 \xrightarrow{\Delta_2} C_3
\end{equation}

where $\Delta_i$ represents the categorical operation transitioning from state $i$ to state $i+1$. The total access time is:

\begin{equation}
T_{\text{cascade}} = T(C_1 \to C_2) + T(C_2 \to C_3) = \tau_{\Delta_1} + \tau_{\Delta_2}
\end{equation}

where $\tau_{\Delta_i}$ is the time required for categorical operation $\Delta_i$.
\end{definition}

\begin{definition}[Direct Path]
The \textbf{direct path} exploits the recursive reference:

\begin{equation}
\text{Path}_{\text{direct}}: C_1 \xrightsquigarrow{\text{ref}} C_3^{\text{recursive}}
\end{equation}

The access time is:

\begin{equation}
T_{\text{direct}} = T_{\text{ref}}(C_1, C_3)
\end{equation}

where $T_{\text{ref}}$ is the time to access the recursive reference, independent of $C_2$.
\end{definition}

\begin{theorem}[Direct Path Advantage]
\label{thm:direct_advantage}
For triangular configurations with recursive references, the direct path access time satisfies:

\begin{equation}
T_{\text{direct}} < T_{\text{cascade}}
\end{equation}

with the advantage:

\begin{equation}
\mathcal{A}_{\text{path}} = \frac{T_{\text{cascade}}}{T_{\text{direct}}} = \frac{\tau_{\Delta_1} + \tau_{\Delta_2}}{T_{\text{ref}}} > 1
\end{equation}
\end{theorem}


\begin{figure}[htbp]
    \centering
    \includegraphics[width=0.95\textwidth]{figures/triangular_amplification_20251116_051857.png}
    \caption{\textbf{Triangular Amplification: Multi-Band Parallel Categorical Prediction.}
    (\textbf{A}) Effective velocity ratio ($v_{\text{eff}}/c$) versus distance across
    RGB wavelength bands (blue 470 nm, green 525 nm, red 625 nm) for molecular
    transitions spanning 1 m to 10 km. All bands converge at ratio $\sim10^0$
    (FTL threshold, dashed line) at 10 km, demonstrating wavelength-independent
    categorical velocity scaling. (\textbf{B}) Triangular amplification factors
    for five molecular experiments (CCO at 1 m through clecc2ccccc2cl at 10 km)
    showing consistent enhancement across RGB bands: 1.42-1.61 (blue), 1.26-1.63
    (green), 1.46-1.79 (red), with mean amplification $\times$1.55 $\pm$ 0.15.
    (\textbf{C}) Multi-band parallel validation demonstrating all three RGB channels
    achieve ratio $>1$ simultaneously at distances $\geq$1 km, with convergence
    at $10^0$ for clecc2ccccc2cl (10 km). (\textbf{D}) Reconstruction error versus
    distance showing error increases from $\sim$4 to $\sim$20 categorical units
    across five orders of magnitude in separation, remaining within tolerance
    (5.0, orange dashed line) for experiments $<$100 m. Triangular amplification
    emerges from recursive categorical references forming completion cycles,
    enabling parallel validation across independent spectral channels with
    combined confidence $P > 0.999$ when all bands agree.}
    \label{fig:triangular_amplification}
    \end{figure}
