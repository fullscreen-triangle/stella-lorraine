\section{Dynamic Categorical Systems: Expressing Physical Evolution in Completion Coordinates}

\subsection{Motivation: Configuration Space vs. Categorical Space}

Traditional dynamics describes physical systems through configuration space coordinates $(q, p)$ representing positions and momenta. The system's evolution follows trajectories $\gamma(t): \mathbb{R} \to \mathbb{R}^{2N}$ determined by Hamilton's equations:

\begin{equation}
\frac{dq_i}{dt} = \frac{\partial H}{\partial p_i}, \quad \frac{dp_i}{dt} = -\frac{\partial H}{\partial q_i}
\end{equation}

However, this description omits a crucial aspect of physical reality: \textit{processes complete}. Once a physical configuration is realized, that particular manifestation cannot be identically re-realized—it has been "used up" in the sequence of physical actualization. This observation motivates an alternative coordinate system based on \textit{categorical completion} rather than spatial configuration.

\begin{principle}[Categorical Coordinate Description]
Physical systems admit dual descriptions:
\begin{enumerate}[(i)]
\item \textbf{Configuration description}: State specified by $(q, p) \in \mathbb{R}^{2N}$ (spatial coordinates)
\item \textbf{Categorical description}: State specified by $(q, p, C) \in \mathbb{R}^{2N} \times \mathcal{C}$ where $C \in \mathcal{C}$ is the categorical position in the completion sequence
\end{enumerate}

The categorical coordinate $C$ tracks \textit{which} realization of configuration $(q, p)$ the system currently occupies, distinguishing multiple temporally separated visits to the same spatial state.
\end{principle}

\subsection{Categorical State Space Structure}

\begin{definition}[Categorical State Space]
\label{def:categorical_state_space}
The \textbf{categorical state space} is a fibered manifold:

\begin{equation}
\mathcal{M}_{\text{cat}} = \mathbb{R}^{2N} \times \mathcal{C}
\end{equation}

where:
\begin{itemize}
\item $\mathbb{R}^{2N}$: Base manifold (traditional phase space)
\item $\mathcal{C}$: Fiber (categorical completion sequence)
\item Projection $\pi: \mathcal{M}_{\text{cat}} \to \mathbb{R}^{2N}$ given by $\pi(q, p, C) = (q, p)$
\end{itemize}

The categorical space $\mathcal{C}$ carries a partial order $\prec$ representing temporal succession of completions.
\end{definition}

\begin{definition}[Categorical Precedence]
For categorical states $C_i, C_j \in \mathcal{C}$, we write $C_i \prec C_j$ (read "$C_i$ precedes $C_j$") if state $C_i$ was completed before state $C_j$ in the temporal sequence of physical processes.

The precedence relation $\prec$ satisfies:
\begin{enumerate}
\item \textbf{Irreflexivity}: $\neg(C \prec C)$
\item \textbf{Antisymmetry}: If $C_i \prec C_j$, then $\neg(C_j \prec C_i)$
\item \textbf{Transitivity}: If $C_i \prec C_j$ and $C_j \prec C_k$, then $C_i \prec C_k$
\end{enumerate}

defining a strict partial order on $\mathcal{C}$.
\end{definition}

\subsection{Dynamics in Categorical Coordinates}

\begin{definition}[Categorical Velocity]
The fundamental dynamical quantity in categorical description is the \textbf{categorical completion rate}:

\begin{equation}
\dot{C}(t) = \frac{dC}{dt}
\end{equation}

measuring the rate at which new categorical states are completed (units: categorical states per second).
\end{definition}

\begin{axiom}[Categorical Irreversibility]
\label{ax:categorical_irreversibility}
Once a categorical state $C$ is completed, it cannot be re-occupied. Therefore:

\begin{equation}
\dot{C}(t) \geq 0 \quad \forall t
\end{equation}

with equality only when no physical processes occur (system at equilibrium).
\end{axiom}

\begin{theorem}[Categorical Dynamics Equations]
\label{thm:categorical_dynamics}
System evolution in categorical coordinates is governed by:

\begin{align}
\frac{dq_i}{dt} &= \frac{\partial H}{\partial p_i} \\
\frac{dp_i}{dt} &= -\frac{\partial H}{\partial q_i} \\
\frac{dC}{dt} &= \Gamma(q, p, C)
\end{align}

where $\Gamma: \mathcal{M}_{\text{cat}} \to \mathbb{R}^+$ is the \textbf{categorical completion function}, determining how rapidly new states complete given the current configuration.
\end{theorem}

\begin{remark}[Coupling Between Spaces]
The categorical completion rate $\Gamma(q, p, C)$ generally depends on both spatial configuration $(q, p)$ and categorical position $C$. This coupling means:
\begin{itemize}
\item Spatial dynamics influence completion rate: energetic configurations complete states faster
\item Categorical history influences spatial dynamics: completed states constrain future configurations
\end{itemize}

This bidirectional coupling is the origin of history-dependent dynamics and irreversibility.
\end{remark}

\begin{figure*}[htbp]
    \centering
    \includegraphics[width=0.95\textwidth]{figures/unpertubed_comparison_20251109_065121.png}
    \caption{Mixed-reseparated versus unperturbed comparison demonstrating categorical memory persistence despite spatial similarity. \textbf{(A)} Physical: Mixed-Reseparated - scatter plot (blue circles, $\sim 20$ molecules) shows position distribution ($x \in [0, 0.5]$, $y \in [0, 1]$) for left container after mixing and re-separation. Black vertical line at $x = 0.25$ marks container midpoint. Blue annotation box: ``Mixed then Re-separated''. Molecules distributed across full vertical extent with slight clustering at $y \sim 0.2$ and $y \sim 0.8$. \textbf{(B)} Physical: Unperturbed - scatter plot (green circles, $\sim 20$ molecules) shows position distribution for container that was never mixed. Green annotation box: ``Never Mixed (Unperturbed)''. Spatial distribution visually similar to panel A with comparable vertical spread and clustering pattern. \textbf{(C)} Spatial similarity: empty plot with red dashed horizontal line at Spatial Similarity $\sim 0.8$ and yellow annotation box: ``Spatially Similar! ($\sim 85$--$95\%$)''. X-axis labeled ``X distribution'' confirms high spatial overlap between mixed-reseparated and unperturbed configurations, validating macroscopic reversibility. \textbf{(D)} Categorical: Mixed-Reseparated - cumulative categorical states $C(t)$ (blue line with shaded area) versus time ($0$--$10$~s) shows monotonic increase from $C = 0$ to $C \approx 20000$ states. White text box: ``Entropy: $S = k_B C = 2.82 \times 10^{-19}$~J/K''. Linear growth rate $\sim 2000$~states/s indicates continuous categorical state completion during mixing-separation cycle. \textbf{(E)} Categorical: Unperturbed - cumulative categorical states $C(t)$ (green line with shaded area) versus time shows similar monotonic increase from $C = 0$ to $C \approx 20000$ states. White text box: ``Entropy: $S = k_B C = 2.75 \times 10^{-19}$~J/K''. Slightly lower final state count compared to mixed-reseparated case. \textbf{(F)} Categorical divergence: bar chart compares final categorical state counts: Mixed-Reseparated $C = 20461$ states (blue bar), Unperturbed $C = 19948$ states (green bar). Red annotation with bracket: ``$\Delta C = 513$ states, DIFFERENT!'' confirms categorical distinction despite spatial similarity. Difference $\Delta C = 513$ represents additional states completed during mixing-separation cycle. \textbf{(G)} Phase-lock network density: time series ($0$--$10$~s) shows phase-lock edge count $|E|$ for Mixed-Reseparated (blue oscillating curve, $|E| \sim 35$--$55$ edges) and Unperturbed (green oscillating curve, $|E| \sim 35$--$45$ edges). Orange shaded region highlights residual difference from mixing phase.\textbf{(H)} Entropy: $S = k_B C$ - dual time series ($0$--$10$~s) shows entropy evolution for Mixed-Reseparated (blue line with shaded area) and Unperturbed (green line with shaded area). Both increase linearly from $S = 0$ to $S \sim 25000 \times 10^{-23}$~J/K with parallel slopes. Yellow annotation box: ``Final Entropies: Mixed: $2.82 \times 10^{-19}$~J/K, Unpert: $2.75 \times 10^{-19}$~J/K, $\Delta S = 7.08 \times 10^{-21}$~J/K''. Orange annotation box: ``Mixed container has HIGHER entropy!'' Entropy difference $\Delta S = 7.08 \times 10^{-21}$~J/K ($\sim 2.5\%$ of total) persists throughout evolution, confirming irreversible entropy production from mixing. \textbf{(I)} The fundamental distinction: white text box provides comprehensive analysis. \textit{Spatial Configuration:} Both containers: LEFT half, Both: $\sim 20$ molecules, Both: Similar distributions, Spatial similarity: $\sim 90\%$, MACROSCOPICALLY IDENTICAL. \textit{Categorical Configuration:} Mixed-Resep: $C = 20461$ states, Unperturbed: $C = 19948$ states, Difference: $\Delta C = 513$, Entropy diff: $\Delta S = 7.08 \times 10^{-21}$~J/K, CATEGORICALLY DISTINCT.}
    \label{fig:categorical_memory}
    \end{figure*}

\subsection{Physical Manifestation: Phase-Lock Networks}

The abstract categorical structure has concrete physical realization through oscillatory phase-lock networks.

\begin{definition}[Phase-Lock Network]
\label{def:phase_lock_network}
For system of $N$ oscillatory components (e.g., molecules), the \textbf{phase-lock network} is a graph $\mathcal{G} = (V, E)$ where:
\begin{itemize}
\item Vertices $V = \{v_1, \ldots, v_N\}$: Individual oscillators
\item Edges $E = \{(v_i, v_j) : |\phi_i - \phi_j| < \phi_{\text{threshold}}\}$: Phase-locked pairs
\end{itemize}

Two oscillators $v_i, v_j$ are phase-locked when their phase difference satisfies:
\begin{equation}
|\Delta\phi_{ij}(t)| = |\phi_i(t) - \phi_j(t) - \phi_{ij}^{\text{eq}}| < \phi_{\text{threshold}} \approx \frac{\pi}{4}
\end{equation}

where $\phi_{ij}^{\text{eq}}$ is the equilibrium phase offset.
\end{definition}

\begin{theorem}[Categorical States as Phase-Lock Configurations]
\label{thm:categorical_phaselock_correspondence}
There exists a bijection between categorical states and equivalence classes of phase-lock network configurations:

\begin{equation}
C \leftrightarrow [\mathcal{G}]_{\sim}
\end{equation}

where $[\mathcal{G}]_{\sim}$ denotes equivalence class of phase-lock graphs producing the same spatial configuration $(q, p)$.
\end{theorem}

\begin{proof}
\textbf{Forward direction} ($C \to [\mathcal{G}]_{\sim}$): Each categorical state $C$ corresponds to a specific physical realization. For oscillatory systems, this realization is characterized by the phase relationships between oscillators. Multiple phase-lock configurations $\{\mathcal{G}_i\}$ can produce the same spatial configuration $(q, p)$ but differ in internal phase structure. These form equivalence class $[\mathcal{G}]_{\sim}$. The categorical state $C$ identifies which particular phase-lock configuration (or equivalence class) the system occupies.

\textbf{Reverse direction} ($ [\mathcal{G}]_{\sim} \to C$): Given a phase-lock configuration $\mathcal{G}$, the system occupies a unique position in the categorical completion sequence. Since phase relationships cannot be identically recreated (oscillations evolve continuously), each phase-lock configuration corresponds to a unique categorical state $C$ in the temporal ordering.

The bijection is established by recognizing that categorical distinguishability precisely captures the multiplicity of phase-lock realizations of a given spatial state. $\square$
\end{proof}

\begin{corollary}[Network Topology Determines Categorical Position]
\label{cor:topology_determines_category}
The categorical position $C$ is determined by phase-lock network topology:

\begin{equation}
C = f(|E|, \text{connectivity}, \text{clustering}, \ldots)
\end{equation}

where $|E|$ is edge count, connectivity measures graph connectedness, and clustering quantifies local network structure.
\end{corollary}

\subsection{Categorical Completion Dynamics}

\begin{definition}[Network Evolution Function]
The phase-lock network evolves according to:

\begin{equation}
\frac{d\mathcal{G}}{dt} = \mathcal{F}[\mathcal{G}, (q, p)]
\end{equation}

where $\mathcal{F}$ is the network evolution functional determining edge formation and removal rates based on current network state and spatial configuration.
\end{definition}

\begin{theorem}[Edge Count Monotonicity]
\label{thm:edge_monotonicity}
For systems approaching equilibrium, the phase-lock network edge count increases monotonically:

\begin{equation}
\frac{d|E|}{dt} \geq 0
\end{equation}

with equality only at equilibrium.
\end{theorem}

\begin{proof}
Phase-lock edges form when oscillators synchronize through interactions. Consider two oscillators $v_i, v_j$ with phase difference $\Delta\phi_{ij}$:

\textbf{Edge formation rate}:
\begin{equation}
r_{\text{form}} \propto P(\Delta\phi_{ij} < \phi_{\text{threshold}}) \times \nu_{\text{interact}}
\end{equation}

where $\nu_{\text{interact}}$ is interaction frequency (collision rate for molecules).

\textbf{Edge removal rate}:
\begin{equation}
r_{\text{remove}} \propto P(\Delta\phi_{ij} > \phi_{\text{threshold}}) \times \gamma_{\text{decohere}}
\end{equation}

where $\gamma_{\text{decohere}}$ is decoherence rate.

For systems not at equilibrium, oscillators explore phase space seeking stable phase relationships. Each interaction is an opportunity to establish new phase-locks. As system approaches equilibrium, more stable phase relationships form, increasing edge count.

At equilibrium, formation and removal rates balance: $r_{\text{form}} = r_{\text{remove}}$, giving $d|E|/dt = 0$.

Therefore: $\frac{d|E|}{dt} \geq 0$ with equality only at equilibrium. $\square$
\end{proof}

\begin{theorem}[Categorical Completion Rate from Network Dynamics]
\label{thm:completion_from_network}
The categorical completion rate equals the rate of phase-lock network evolution:

\begin{equation}
\dot{C}(t) = \kappa \cdot \frac{d|E|}{dt} + \lambda \cdot \text{Tr}\left(\frac{d\mathcal{G}}{dt}\right)
\end{equation}

where $\kappa, \lambda$ are coupling constants and $\text{Tr}(\cdot)$ represents a trace operation over network configuration space.
\end{theorem}

\begin{proof}
By Theorem \ref{thm:categorical_phaselock_correspondence}, categorical states correspond to phase-lock configurations. Categorical completion occurs when phase-lock configuration changes. The rate of configuration change has two components:

\textbf{(1) Topological changes}: Formation/removal of edges, quantified by $d|E|/dt$

\textbf{(2) Structural changes}: Modification of edge weights, phase offsets, connectivity patterns, quantified by trace of network evolution

The categorical completion rate is weighted sum of these contributions:
\begin{equation}
\dot{C} = \kappa \cdot \text{(topological change)} + \lambda \cdot \text{(structural change)}
\end{equation}

establishing the stated result. $\square$
\end{proof}

\subsection{Entropy Production in Categorical Coordinates}

\begin{definition}[Categorical Entropy]
The entropy associated with categorical state $C$ is:

\begin{equation}
S(q, p, C) = k_B \log \Omega_{\text{cat}}(q, p, C)
\end{equation}

where $\Omega_{\text{cat}}(q, p, C)$ counts the number of phase-lock configurations compatible with spatial state $(q, p)$ in categorical state $C$.
\end{definition}

\begin{theorem}[Entropy Production from Categorical Completion]
\label{thm:entropy_production}
The entropy production rate is:

\begin{equation}
\frac{dS}{dt} = k_B \frac{\partial \log \Omega_{\text{cat}}}{\partial C} \cdot \frac{dC}{dt} = k_B \frac{\partial \log \Omega_{\text{cat}}}{\partial C} \cdot \dot{C}
\end{equation}

Since $\Omega_{\text{cat}}$ increases with $C$ (later categorical states have more accessible phase-lock configurations) and $\dot{C} \geq 0$ (Axiom \ref{ax:categorical_irreversibility}), we have:

\begin{equation}
\frac{dS}{dt} \geq 0
\end{equation}

providing categorical derivation of the second law.
\end{theorem}

\begin{proof}
\textbf{Step 1}: By definition, $S = k_B \log \Omega_{\text{cat}}(q, p, C)$.

\textbf{Step 2}: Taking time derivative via chain rule:
\begin{equation}
\frac{dS}{dt} = k_B \frac{\partial \log \Omega_{\text{cat}}}{\partial q} \frac{dq}{dt} + k_B \frac{\partial \log \Omega_{\text{cat}}}{\partial p} \frac{dp}{dt} + k_B \frac{\partial \log \Omega_{\text{cat}}}{\partial C} \frac{dC}{dt}
\end{equation}

\textbf{Step 3}: For processes at constant energy (microcanonical ensemble), the $(q, p)$ terms average to zero over phase space. The categorical term dominates:
\begin{equation}
\frac{dS}{dt} \approx k_B \frac{\partial \log \Omega_{\text{cat}}}{\partial C} \dot{C}
\end{equation}

\textbf{Step 4}: Crucially, $\Omega_{\text{cat}}$ increases with $C$ because:
\begin{itemize}
\item Early categorical states (small $C$): Few phase-lock configurations explored
\item Later categorical states (large $C$): Many phase-lock configurations discovered through system evolution
\end{itemize}

Therefore: $\frac{\partial \Omega_{\text{cat}}}{\partial C} > 0$, which implies $\frac{\partial \log \Omega_{\text{cat}}}{\partial C} > 0$.

\textbf{Step 5}: By Axiom \ref{ax:categorical_irreversibility}, $\dot{C} \geq 0$.

\textbf{Conclusion}: Product of two non-negative quantities is non-negative:
\begin{equation}
\frac{dS}{dt} = k_B \underbrace{\frac{\partial \log \Omega_{\text{cat}}}{\partial C}}_{> 0} \cdot \underbrace{\dot{C}}_{\geq 0} \geq 0
\end{equation}

$\square$
\end{proof}

\begin{corollary}[Thermodynamic Irreversibility from Categorical Dynamics]
\label{cor:irreversibility_categorical}
The second law of thermodynamics ($dS/dt \geq 0$) is a direct consequence of categorical irreversibility ($\dot{C} \geq 0$), not a statistical statement about probability.
\end{corollary}




\subsection{Categorical Distance and Trajectory Optimization}

\begin{definition}[S-Distance Between Categorical States]
For categorical states $C_i, C_j$, the S-distance is:

\begin{equation}
S(C_i, C_j) = \int_{C_i}^{C_j} \|\nabla_C \Omega_{\text{cat}}\| \, dC
\end{equation}

measuring the cumulative change in accessible phase-lock configurations along the categorical path from $C_i$ to $C_j$.
\end{definition}

\begin{theorem}[Categorical Geodesics]
\label{thm:categorical_geodesics}
Physical processes follow geodesics in categorical space—paths minimizing S-distance. The geodesic equation is:

\begin{equation}
\frac{d^2 C}{dt^2} + \Gamma_C \left(\frac{dC}{dt}\right)^2 = 0
\end{equation}

where $\Gamma_C$ is the categorical connection coefficient.
\end{theorem}

\begin{proof}
Physical processes optimize efficiency: they complete categorical states via paths requiring minimal "work" in categorical space. This optimization principle yields geodesic equations analogous to classical mechanics.

The "work" to traverse categorical space is quantified by S-distance. Minimizing $\int S(C_i, C_j) dt$ subject to constraints yields Euler-Lagrange equations equivalent to the stated geodesic equation.

Physical interpretation: Systems naturally evolve along paths of least categorical resistance—sequences of phase-lock configurations that flow naturally from one to the next. $\square$
\end{proof}

\subsection{Multi-Scale Categorical Hierarchies}

\begin{definition}[Hierarchical Categorical Structure]
Real physical systems exhibit nested categorical hierarchies:

\begin{equation}
\mathcal{C}_{\text{total}} = \mathcal{C}_{\text{quantum}} \times \mathcal{C}_{\text{molecular}} \times \mathcal{C}_{\text{mesoscopic}} \times \mathcal{C}_{\text{macroscopic}}
\end{equation}

where each level has its own completion dynamics:

\begin{align}
\dot{C}_{\text{quantum}} &\sim 10^{15} \text{ states/s} \quad \text{(electronic transitions)} \\
\dot{C}_{\text{molecular}} &\sim 10^{12} \text{ states/s} \quad \text{(vibrational modes)} \\
\dot{C}_{\text{mesoscopic}} &\sim 10^{6} \text{ states/s} \quad \text{(collective modes)} \\
\dot{C}_{\text{macroscopic}} &\sim 10^{0} \text{ states/s} \quad \text{(thermodynamic processes)}
\end{align}
\end{definition}

\begin{theorem}[Scale-Separated Categorical Dynamics]
\label{thm:scale_separation}
When categorical completion rates differ by orders of magnitude ($\dot{C}_i \gg \dot{C}_j$), the faster scale reaches quasi-equilibrium while the slower scale evolves:

\begin{equation}
\frac{\dot{C}_i}{\dot{C}_j} \gg 1 \implies C_i(t) \approx C_i^{\text{eq}}[C_j(t)]
\end{equation}

The fast scale $C_i$ adiabatically follows the slow scale $C_j$.
\end{theorem}

\begin{proof}
Consider two-scale system with $\dot{C}_{\text{fast}} = 10^{15}$ states/s and $\dot{C}_{\text{slow}} = 10^{0}$ states/s.

In time $\Delta t = 10^{-12}$ s (one picosecond):
\begin{itemize}
\item Fast scale completes: $\Delta C_{\text{fast}} = \dot{C}_{\text{fast}} \Delta t = 10^{15} \times 10^{-12} = 10^3$ states
\item Slow scale completes: $\Delta C_{\text{slow}} = \dot{C}_{\text{slow}} \Delta t = 10^{0} \times 10^{-12} = 10^{-12}$ states $\approx 0$
\end{itemize}

The fast scale completes thousands of categorical states, while the slow scale is essentially frozen. Therefore, the fast scale equilibrates to the configuration determined by the current slow-scale categorical state.

This establishes adiabatic following: $C_{\text{fast}}(t) \approx C_{\text{fast}}^{\text{eq}}[C_{\text{slow}}(t)]$. $\square$
\end{proof}

\subsection{Oscillatory-Categorical Correspondence}

\begin{principle}[Frequency-Category Duality]
\label{pr:frequency_category_duality}
Categorical states correspond bijectively to oscillatory modes. For system with oscillatory spectrum $\{\omega_n\}$:

\begin{equation}
C_n \leftrightarrow \omega_n
\end{equation}

Each categorical state $C_n$ in the completion sequence corresponds to a distinct oscillatory frequency $\omega_n$.
\end{principle}

\begin{remark}[Physical Basis]
This correspondence arises because:
\begin{enumerate}
\item Physical systems are fundamentally oscillatory (Section 1)
\item Categorical states represent distinct realisations (Theorem \ref{thm:categorical_phaselock_correspondence})
\item Each realisation has a characteristic oscillatory signature
\item Different categorical states have different oscillatory frequencies
\end{enumerate}

Therefore, the categorical completion sequence $\{C_1, C_2, C_3, \ldots\}$ maps to the oscillatory frequency spectrum $\{\omega_1, \omega_2, \omega_3, \ldots\}$.
\end{remark}

\begin{theorem}[Complete Categorical Access via Oscillatory Spectrum]
\label{thm:categorical_access}
A system capable of accessing all oscillatory modes $\{\omega_n\}$ in its spectrum can access all categorical states $\{C_n\}$ in the completion sequence:

\begin{equation}
\text{Access}(\{\omega_n\}_{n=1}^{\infty}) \iff \text{Access}(\{C_n\}_{n=1}^{\infty})
\end{equation}
\end{theorem}

\begin{proof}
\textbf{Forward direction} ($\implies$): Suppose the system can access all oscillatory modes $\{\omega_n\}$. By Principle \ref{pr:frequency_category_duality}, each $\omega_n$ corresponds to a categorical state $C_n$. Therefore, accessing $\{\omega_n\}$ provides access to $\{C_n\}$.

\textbf{Reverse direction} ($\impliedby$): Suppose the system can access all categorical states $\{C_n\}$. Each categorical state corresponds to a phase-lock configuration (Theorem \ref{thm:categorical_phaselock_correspondence}) with a characteristic oscillatory signature $\omega_n$. Therefore, accessing $\{C_n\}$ provides access to $\{\omega_n\}$.

The bijection establishes equivalence. $\square$
\end{proof}

\begin{corollary}[Single-System Categorical Spanning]
\label{cor:single_system_spanning}
A single physical system that can oscillate at all frequencies within its accessible spectrum effectively spans the complete categorical space:

\begin{equation}
\mathcal{O}_{\text{system}} = \{\omega_n\}_{n=1}^{N} \implies \mathcal{C}_{\text{accessible}} = \{C_n\}_{n=1}^{N}
\end{equation}

where $\mathcal{O}_{\text{system}}$ is the oscillatory spectrum and $\mathcal{C}_{\text{accessible}}$ is the accessible categorical subspace.
\end{corollary}

\begin{remark}[Practical Implication]
This corollary has profound implications: rather than requiring separate physical instantiations for each categorical state, a \textit{single system with a rich oscillatory spectrum can access multiple categorical states by changing its oscillatory mode}.

Example: A molecular system with vibrational frequencies $\{\omega_{\text{vib},n}\}$, rotational frequencies $\{\omega_{\text{rot},m}\}$, and electronic frequencies $\{\omega_{\text{elec},k}\}$ can access categorical states:
\begin{equation}
\{C_{nmk}\} = \{\text{states corresponding to } (\omega_{\text{vib},n}, \omega_{\text{rot},m}, \omega_{\text{elec},k})\}
\end{equation}

by modulating its internal oscillatory modes. The system need not physically move through space—it traverses categorical space by modulating its oscillatory state.
\end{remark}

\subsection{Categorical State Prediction}

\begin{definition}[Categorical Prediction Problem]
Given the current categorical state $C_{\text{current}}$ and target S-distance $\Delta S_{\text{target}}$, predict the final categorical state:

\begin{equation}
C_{\text{final}} = C_{\text{current}} + \Delta C(\Delta S_{\text{target}})
\end{equation}

where $\Delta C$ is the categorical displacement corresponding to S-distance $\Delta S_{\text{target}}$.
\end{definition}

\begin{theorem}[Categorical Prediction via Oscillatory Mapping]
\label{thm:categorical_prediction}
Categorical state prediction reduces to oscillatory mode mapping. Given:
\begin{itemize}
\item Current oscillatory state: $\omega_{\text{current}}$
\item Target categorical displacement: $\Delta C$
\item Oscillatory-categorical map: $\omega_n \leftrightarrow C_n$
\end{itemize}

The predicted final state is:
\begin{equation}
\omega_{\text{final}} = \omega_{\text{current}} + \Delta\omega(\Delta C)
\end{equation}

where $\Delta\omega$ is determined by inverting the oscillatory-categorical correspondence.
\end{theorem}

\begin{proof}
\textbf{Step 1}: The current categorical state corresponds to the current oscillatory mode:
\begin{equation}
C_{\text{current}} \leftrightarrow \omega_{\text{current}}
\end{equation}

\textbf{Step 2}: The target categorical state is:
\begin{equation}
C_{\text{target}} = C_{\text{current}} + \Delta C
\end{equation}

\textbf{Step 3}: By oscillatory-categorical correspondence:
\begin{equation}
C_{\text{target}} \leftrightarrow \omega_{\text{target}}
\end{equation}

\textbf{Step 4}: The oscillatory displacement is:
\begin{equation}
\Delta\omega = \omega_{\text{target}} - \omega_{\text{current}}
\end{equation}

determined by the form of the correspondence relation (typically logarithmic: $C \propto \log \omega$ for harmonic oscillators).

\textbf{Conclusion}: Categorical prediction is equivalent to oscillatory mode prediction. If the oscillatory spectrum is known, categorical states can be predicted by identifying the corresponding oscillatory frequencies. $\square$
\end{proof}

\begin{algorithm}[H]
\caption{Categorical State Prediction via Oscillatory Mapping}
\begin{algorithmic}[1]
\Procedure{PredictCategoricalState}{$C_{\text{current}}, \Delta S_{\text{target}}$}
    \State $\omega_{\text{current}} \gets$ MapCategoricalToOscillatory($C_{\text{current}}$)
    \State $\Delta C \gets$ ComputeCategoricalDisplacement($\Delta S_{\text{target}}$)
    \State $\Delta\omega \gets$ ComputeOscillatoryDisplacement($\Delta C$)
    \State $\omega_{\text{target}} \gets \omega_{\text{current}} + \Delta\omega$
    \State $C_{\text{target}} \gets$ MapOscillatorytoCategorical($\omega_{\text{target}}$)
    \State \Return $C_{\text{target}}$
\EndProcedure
\end{algorithmic}
\end{algorithm}

\subsection{Complexity Reduction via Categorical Representation}

\begin{theorem}[Categorical Complexity Reduction]
\label{thm:categorical_complexity}
Expressing dynamics in categorical coordinates reduces computational complexity from exponential to logarithmic:

\begin{equation}
O(2^N) \xrightarrow{\text{categorical}} O(\log N)
\end{equation}

where $N$ is the number of system components (e.g., molecules).
\end{theorem}

\begin{proof}
\textbf{Configuration space complexity}: The traditional description requires tracking all $2^N$ possible spatial configurations of $N$ binary components. For molecular systems with continuous degrees of freedom, complexity is even worse: $O(\infty^N)$.

\textbf{Categorical space complexity}: The categorical description tracks the position in the completion sequence $C \in \{1, 2, 3, \ldots\}$. The categorical state encodes an equivalence class of configurations rather than individual configurations.

For $M$ total accessible categorical states (typically $M \sim \log N$ due to hierarchical organisation), the complexity is $O(\log M) = O(\log \log N) \approx O(\log N)$ for practical systems.

The reduction factor:
\begin{equation}
\text{Reduction} = \frac{O(2^N)}{O(\log N)} = \frac{2^N}{\log N}
\end{equation}

For $N = 100$: $\text{Reduction} \approx \frac{10^{30}}{2.3} \approx 10^{30}$—thirty orders of magnitude!

$\square$
\end{proof}

\begin{corollary}[Tractability of Categorical Dynamics]
Systems intractable in configuration space become tractable in categorical space. Problems requiring $O(2^N)$ operations (exponential, infeasible for $N > 50$) reduce to $O(\log N)$ operations (logarithmic, feasible for arbitrarily large $N$).
\end{corollary}

\begin{figure*}[htbp]
    \centering
    \includegraphics[width=0.95\textwidth]{figures/reseperation_20251109_065105.png}
    \caption{Gibbs paradox resolution through categorical state dynamics demonstrating spatial reversibility with categorical irreversibility across full mixing-separation cycle. \textbf{(A)} Physical configuration - spatially identical to initial: scatter plot shows Container A (blue circles) and Container B (red circles) molecules in position space ($x$, $y$ $\in [0, 1]$) after re-separation. Partition restored at $x = 0.5$ (black dashed line) with Container A occupying left half ($x < 0.5$, $\sim 20$ molecules) and Container B right half ($x > 0.5$, $\sim 20$ molecules). Configuration macroscopically identical to initial state but categorically distinct. \textbf{(B)} Categorical state - DIFFERENT from initial: trajectory plot shows categorical state evolution from Initial (separated, gray region, $C_{\text{init}}$) through Mixed state (yellow region, $C_{\text{mix}}$) to Re-separated state (orange region, $C_{\text{resep}}$). Black arrow indicates irreversible progression across $\sim 40$ categorical state IDs. \textbf{(C)} Residual A-B phase correlations: circular network diagram displays phase-lock coherence matrix for $40$ molecules (blue circles = Container A, red circles = Container B, arranged on circle perimeter).\textbf{(D)} Edge count through full cycle: bar chart compares phase-lock edge counts across three stages: Initial (separated) shows A-A edges $\sim 32$ (blue bar), B-B edges $\sim 20$ (red bar), A-B edges $0$ (orange bar absent). Mixed state: A-A $\sim 32$ (blue), B-B $\sim 20$ (red), A-B $\sim 60$ (orange). Re-separated: A-A $\sim 32$ (blue), B-B $\sim 20$ (red), A-B $\sim 20$ (orange, annotated ``20 residual edges persist!''). Persistent A-B edges after re-separation confirm categorical memory. \textbf{(E)} Entropy through mixing-separation cycle: entropy $S$ (J/K, $\times 10^{-23}$) versus process stage (Initial, Mixed, Re-separated) shows monotonic increase (red line with shaded area) from $S_{\text{init}} \sim 1.0 \times 10^{-23}$~J/K (black circle) through $S_{\text{mix}} \sim 2.0 \times 10^{-23}$~J/K (peak, black circle) to $S_{\text{resep}} \sim 1.3 \times 10^{-23}$~J/K (black circle). Red dashed horizontal line at $S_{\text{init}}$ shows $S_{\text{resep}} > S_{\text{init}}$ ($\Delta S > 0$). \textbf{(F)} Phase coherence matrix: heatmap (colorbar $0.0$--$1.0$, yellow = high coherence, dark red = low coherence) shows $40 \times 40$ molecule-molecule phase coherence after re-separation. Strong diagonal blocks (yellow, coherence $\sim 0.8$--$1.0$) indicate intra-container correlations (molecules $0$--$20$ = Container A, $20$--$40$ = Container B). Off-diagonal blocks (orange/red, coherence $\sim 0.2$--$0.6$) reveal residual inter-container correlations. Orange annotation: ``Residual A-B'' highlights persistent cross-container phase memory. \textbf{(G)} Spatial $\approx$ Initial, Categorical $\neq$ Initial: green text box on white background provides spatial versus categorical distinguishability analysis. \textit{Spatial Configuration:} Molecules in left half (Container A), molecules in right half (Container B), partition at $x = 0.5$, position distribution $\approx$ Initial, velocity distribution $\approx$ Initial, macroscopically IDENTICAL to initial.}
    \label{fig:gibbs_paradox}
    \end{figure*}

\subsection{Summary: Categorical Dynamics Framework}

The dynamic categorical systems framework establishes:

\begin{enumerate}
\item \textbf{Categorical coordinates}: Physical systems admit description through categorical position $C$ in the completion sequence, complementing traditional $(q, p)$ coordinates

\item \textbf{Categorical velocity}: The fundamental dynamical quantity is the completion rate $\dot{C}(t) \geq 0$, which is strictly non-negative due to irreversibility

\item \textbf{Phase-lock realization}: Categorical states correspond to phase-lock network configurations, providing a concrete physical manifestation

\item \textbf{Network evolution}: Phase-lock networks evolve via edge formation/removal, with the edge count increasing monotonically toward equilibrium

\item \textbf{Entropy production}: The second law emerges from categorical irreversibility: $dS/dt = k_B (\partial \log \Omega_{\text{cat}} / \partial C) \dot{C} \geq 0$

\item \textbf{Oscillatory correspondence}: Categorical states map bijectively to oscillatory modes $C_n \leftrightarrow \omega_n$, establishing frequency-category duality

\item \textbf{Complete access principle}: A system accessing all oscillatory modes accesses all categorical states—single system can span categorical space by modulating oscillations

\item \textbf{Prediction via oscillations}: Categorical state prediction reduces to oscillatory mode mapping, enabling efficient computation

\item \textbf{Complexity reduction}: Categorical representation reduces complexity from $O(2^N)$ exponential to $O(\log N)$ logarithmic

\item \textbf{Multi-scale hierarchies}: Nested categorical structures at quantum/molecular/mesoscopic/macroscopic scales with adiabatic separation between levels
\end{enumerate}

This framework reveals that physical dynamics, when expressed in categorical coordinates, exhibit a fundamentally different mathematical structure than traditional phase space dynamics. The irreversibility, entropy production, and complexity reduction emerge naturally from categorical completion principles rather than requiring statistical arguments.

Critically, the oscillatory-categorical correspondence (Principle \ref{pr:frequency_category_duality}) and the complete access theorem (Theorem \ref{thm:categorical_access}) establish that \textit{a single system with a rich oscillatory spectrum can access the complete categorical space}. This suggests that physical measurements and state determination might be achievable through oscillatory mode detection rather than exhaustive configuration space sampling—a principle whose implications will be explored in subsequent sections.
