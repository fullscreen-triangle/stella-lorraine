\section{Discussion}
\label{sec:discussion}

\subsection{Principal Achievements}

This work establishes a non-invasive thermometry protocol achieving a temperature resolution of \(\Delta T \sim 17\) pK through categorical state measurement. The key enabling principles are:

\begin{enumerate}
\item \textbf{Information-Based Measurement}: Temperature inferred from categorical state \(\mathcal{C}(t)\) encoding the momentum distribution, rather than direct kinetic energy measurement. This bypasses quantum backaction constraints.

\item \textbf{Trans-Planckian Precision}: H\(^+\) oscillator timing at 71 THz provides energy resolution \(\Delta E \sim 10^{-34}\) J, \(\sim 10^4\times\) better than the photon recoil limit.

\item \textbf{Zero Energy Input}: Far-detuned optical coupling to a virtual spectrometer introduces heating \(\Delta T < 1\) fK per second—negligible for measurement times \(\sim\) milliseconds.

\item \textbf{Real-Time Monitoring}: Non-destructive measurement enables continuous temperature tracking during cooling, allowing for adaptive protocol optimization.
\end{enumerate}

These capabilities transform temperature measurement in the ultra-cold regime from a crude diagnostic (\(\Delta T / T \sim 1\%\)) into a precision tool (\(\Delta T / T \sim 10^{-7}\)).

\subsection{Comparison with Existing Methods}

\subsubsection{Time-of-Flight Imaging}

\textbf{Conventional Approach:} Release atoms from the trap, image after ballistic expansion, and fit the cloud size to extract temperature.

\textbf{Advantages:}
\begin{itemize}
\item Well-established technique with decades of development
\item Works for wide temperature range (nK to mK)
\item Simple theoretical interpretation
\end{itemize}

\textbf{Limitations:}
\begin{itemize}
\item Destructive: Sample lost after measurement
\item Accuracy: \(\Delta T / T \sim 1\)--5\%
\item Slow: Requires \(t_{\text{TOF}} \sim\) 10–100 ms plus imaging
\item Systematic errors from residual fields during expansion
\end{itemize}

\textbf{Categorical Improvement:}
\begin{itemize}
\item Non-destructive: Enables repeated measurements
\item Accuracy: \(\Delta T / T \sim 10^{-7}\) (\(10^5\times\) better)
\item Fast: Limited by integration time (\(\sim 1\) ms)
\item No free expansion: Immune to systematic field effects
\end{itemize}

\subsubsection{In-Situ Absorption Imaging}

\textbf{Conventional Approach:} A resonant probe beam measures optical density, and fitting to the thermal distribution yields temperature.

\textbf{Advantages:}
\begin{itemize}
\item Non-destructive (in principle)
\item Provides spatial information
\item Faster than time-of-flight
\end{itemize}

\textbf{Limitations:}
\begin{itemize}
\item Photon scattering heats the sample: \(\Delta T \sim 100\) nK per image
\item Requires careful calibration of the imaging system
\item Difficult at very low \(T\) levels where the optical density is small
\item Probe light perturbs quantum states
\end{itemize}

\textbf{Categorical Improvement:}
\begin{itemize}
\item Truly non-invasive: \(\Delta T < 1\) femtoK per measurement
\item No calibration of the absorption cross-section is needed
\item Works down to picokelvin regime
\item Far-detuned light preserves quantum coherence
\end{itemize}

\subsubsection{Thermometry via Spectroscopy}

\textbf{Conventional Approach:} Measure spectral linewidth (Doppler or motional sidebands), extract temperature from width.

\textbf{Advantages:}
\begin{itemize}
\item Non-destructive for weak spectroscopy
\item High precision possible for narrow lines
\item Direct access to the velocity distribution
\end{itemize}

\textbf{Limitations:}
\begin{itemize}
\item Requires resolved sidebands: \(\omega_{\text{trap}} > \Gamma_{\text{line}}\)
\item AC Stark shifts from probe light introduce systematic errors
\item Limited to specific atomic species/transitions
\item Heating from spectroscopy light
\end{itemize}

\textbf{Categorical Improvement:}
\begin{itemize}
\item No sideband resolution requirement
\item No light-shift systematics (far-detuned)
\item Universal: Works for any atomic species
\item Negligible heating
\end{itemize}

\subsection{Experimental Validation Strategy}

\subsubsection{Phase 1: Calibration Against Known Methods}

\textbf{Setup:} Rb-87 BEC at \(T \sim 100\) nK, well-characterised by time-of-flight.

\textbf{Protocol:}
\begin{enumerate}
\item Measure \(T_{\text{TOF}}\) using standard imaging
\item Immediately prepare identical sample
\item Measure \(T_{\text{cat}}\) using categorical thermometry
\item Compare: \(|T_{\text{cat}} - T_{\text{TOF}}| / T_{\text{TOF}} < 0.01\)
\end{enumerate}

\textbf{Goal:} Establish the accuracy of the categorical method in a regime accessible to conventional techniques.

\subsubsection{Phase 2: Sub-Recoil Regime Demonstration}

\textbf{Setup:} Apply Raman sideband cooling to reach \(T < T_{\text{recoil}} = 273\) nK (regime where optical thermometry fails).

\textbf{Protocol:}
\begin{enumerate}
\item Cool to \(T \approx 50\) nK (verified by sideband occupations)
\item Monitor \(S_e(t)\) continuously during cooling
\item Extract \(T_{\text{cat}}(t)\) from categorical coordinates
\item Verify \(T_{\text{cat}}\) consistent with sideband cooling theory
\end{enumerate}

\textbf{Goal:} Demonstrate categorical thermometry in regime inaccessible to standard methods.

\subsubsection{Phase 3: Real-Time Cooling Optimization}

\textbf{Setup:} Evaporative cooling with adaptive protocol based on categorical feedback.

\textbf{Protocol:}
\begin{enumerate}
\item Define target temperature \(T_{\text{target}} = 10\) nK
\item Implement feedback loop: Adjust evaporation parameters to minimize \(\tau_{\text{cool}}\) while reaching \(T_{\text{target}}\)
\item Compare with fixed-parameter evaporation
\end{enumerate}

\textbf{Goal:} Demonstrate the practical advantages of real-time temperature monitoring.

\subsubsection{Phase 4: Picokelvin Resolution}

\textbf{Setup:} Optical lattice clock atoms (Sr, Yb) cooled to sub-nanokelvin in 3D lattice.

\textbf{Protocol:}
\begin{enumerate}
\item Apply final stage cooling to reach \(T \sim 100\) pK (theoretically)
\item Measure \(T_{\text{cat}}\) with maximum integration time (\(\sim 1\) s)
\item Achieve \(\Delta T < 20\) pK resolution
\item Verify the stability of the picokelvin state over time
\end{enumerate}

\textbf{Goal:} Push categorical thermometry to fundamental limits, accessing previously unmeasurable regime.
