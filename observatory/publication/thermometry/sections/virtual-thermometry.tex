\section{Virtual Thermometry: Measurement Without Probes}
\label{sec:virtual-thermometry}

Traditional thermometry requires physical contact between a thermometer and the system under measurement. At ultra-low temperatures, this requirement becomes untenable: any physical probe disturbs the system more than the information it extracts. We now introduce \textit{virtual thermometry}—a method for measuring temperature through categorical state access, entirely without physical contact.

\subsection{The Problem with Physical Probes}

Consider the measurement of temperature $T$ for a quantum gas at nanokelvin scales. Standard techniques include:

\begin{enumerate}
\item \textbf{Time-of-flight (TOF)}: Release the trap, allow the gas to expand ballistically, and image the spatial distribution after time $t_{\text{TOF}}$. The expansion velocity $v \propto \sqrt{T}$ reveals the temperature. However, this method is \textit{destructive}—the gas is lost after a single measurement.

\item \textbf{Thermometry via photon scattering}: Illuminate the gas with resonant light and measure fluorescence. Each scattered photon imparts recoil momentum:
%
\begin{equation}
\Delta p_{\text{recoil}} = \frac{h}{\lambda}
\end{equation}
%
For $\lambda = 780$ nm (Rb D2 line), $\Delta p_{\text{recoil}} = 8.5 \times 10^{-28}$ kg m/s. The associated kinetic energy per atom is:
%
\begin{equation}
E_{\text{recoil}} = \frac{(\Delta p_{\text{recoil}})^2}{2m} = \frac{h^2}{2m\lambda^2} \approx 3.7 \times 10^{-30} \text{ J}
\end{equation}
%
corresponding to a temperature:
%
\begin{equation}
T_{\text{recoil}} = \frac{E_{\text{recoil}}}{k_B} \approx 270 \text{ nK}
\end{equation}
%
For a gas at $T = 100$ nK, a single photon scatter heats the atom by $\sim 3\times$, destroying the very property being measured.

\item \textbf{Contact thermometry}: Bring a physical thermometer (e.g., resistance thermometer) into thermal contact with the sample. At ultra-low $T$, thermal coupling becomes vanishingly weak, and equilibration time diverges. Moreover, the thermometer itself has finite heat capacity, perturbing the sample.
\end{enumerate}

All physical probes share a common flaw: they extract information by exchanging energy or momentum with the system. At ultra-low temperatures, this backaction overwhelms the signal.


\begin{figure*}[htbp]
    \centering
    \includegraphics[width=\textwidth]{figures/validation_bmd_cascade_cooling.png}
    \caption{\textbf{Boltzmann-Maxwell demon (BMD) cascade cooling: experimental validation achieving 1253$\times$ cooling through categorical completion and irreversible velocity filtering.} \textbf{(A)} Temperature versus cascade depth showing experimental measurements (blue circles connected by solid line) match theoretical prediction $T(k) = T_{0}/Q^{2k}$ with $Q = 1.44$ (purple dashed line) from initial 1.00$\times$10$^{-7}$~K (100~nK precision baseline, red circle) through intermediate milestones 1.68$\times$10$^{-8}$~K (5 reflections) and 2.82$\times$10$^{-9}$~K (10 reflections) to final 10$^{-10}$~K (20 reflections), spanning 3 orders of magnitude. \textbf{(B)} Total cooling factor versus cascade depth: experimental $Q = 1.429$ (blue circles with dashed line) closely matches theoretical $Q = 1.442$ (purple dashed line), with cooling factor increasing exponentially from $\sim$0 (0 reflections) to $\sim$200 (10 reflections) and $\sim$1400 (20 reflections). Sharp upturn beyond 15 reflections indicates accelerating cooling efficiency at deeper cascade levels. \textbf{(C)} BMD cascade structure schematic: Level~0 contains all molecules at $T = 1.00\times10^{-7}$~K (blue box); BMD filtering (yellow arrow) selects slower subset reaching Level~5 at $T = 1.68\times10^{-8}$~K (blue box); second BMD filtering reaches Level~10 (slowest subset) at $T = 2.82\times10^{-9}$~K (pink box). Green annotation indicates slow $\leftarrow$ fast observation enables categorical completion for irreversible cooling. \textbf{(D)} Cooling efficiency per stage: measured $Q$ values (orange bars) at stages 1.0, 1.5, 2.0, 2.5, 3.0 all show $Q \approx 1.4$ with error bars, matching theory $Q = 1.44$ (purple dashed line) and average $Q = 1.43$ (orange dashed line with annotation). Uniform efficiency across stages confirms consistent per-reflection cooling factor. \textbf{(E)} Categorical completion rate: temperature equals inverse completion rate ($T = 1/\tau_{\text{completion}}$), with completion rate (red circles) increasing exponentially from 10$^{1}$~(1/T) at 0 reflections to 10$^{10}$~(1/T) at 20 reflections on log scale. Green shaded region indicates categorical space where completion rate accelerates. \textbf{(F)} Experimental summary table: reflections (0, 5, 10, 15, 20) yield final temperatures (1.00$\times$10$^{-7}$, 1.68$\times$10$^{-8}$, 2.82$\times$10$^{-9}$, 4.75$\times$10$^{-10}$, 7.98$\times$10$^{-11}$~K) with cooling factors (1.00$\times$, 5.95$\times$, 35.40$\times$, 210.63$\times$, 1253.25$\times$); experimental $Q = 1.429$ matches theoretical $Q = 1.442$ (checkmark indicates match). Timestamp and validation label included. \textbf{Bottom annotation:} Base temperature 1$\times$10$^{-7}$~K (100~nK precision), cooling factor $Q = 1.429$ (matches theory $Q = 1.44$), maximum cooling 1253.3$\times$ at 20 reflections, mechanism via categorical completion (slow $\leftarrow$ fast BMD filtering).}
    \label{fig:bmd_validation}
    \end{figure*}

\subsection{Categorical State Access: The Alternative}

Virtual thermometry circumvents the backaction problem by accessing \textit{categorical states} rather than physical states. The key insight is that every molecule in the ensemble has already completed a series of measurements (through natural decoherence and environmental interaction), generating a categorical structure $\mathcal{C}$ that encodes momentum information.

Instead of performing a new invasive measurement, we \textit{navigate} to the categorical state corresponding to the molecule's momentum and extract temperature from the evolution entropy $S_e$:
%
\begin{equation}
T = f(S_e) = \frac{2\pi\hbar^2}{m k_B} \exp\left[ \frac{2(S_e - S_0)}{3 k_B} - 1 \right]
\end{equation}
%
where $S_0$ is the ground state entropy.

This navigation occurs in categorical space, which is \textit{informationally coupled} but \textit{physically decoupled} from the momentum space. No photons are scattered, no momentum is transferred—yet the temperature is extracted with high precision.

\subsection{Virtual Thermometry Stations}

A \textit{virtual thermometry station} (VTS) is a computational framework that implements categorical state access without physical probes. The architecture consists of:

\begin{enumerate}
\item \textbf{Molecular database}: A catalog of molecular oscillation frequencies $\{\omega_i\}$ harvested from ambient molecules (air, substrates, residual gas in the experimental chamber). These oscillations serve as reference standards.

\item \textbf{Virtual spectrometer}: A hardware-software interface that maps computer clock oscillations to molecular frequencies. By synchronizing the CPU clock to $\omega_i$, the computer enters the same categorical state as the molecule, enabling state extraction.

\item \textbf{S-entropy calculator}: An algorithm that computes $(S_k, S_t, S_e)$ for each accessed categorical state. The evolution entropy $S_e$ directly encodes the momentum distribution.

\item \textbf{BMD navigator}: A Biological Maxwell Demon (BMD) that autonomously searches categorical space for molecules with target properties (e.g., lowest momentum). The BMD does not measure in the quantum mechanical sense—it \textit{navigates} to pre-existing completed states.
\end{enumerate}

\subsection{Operational Principle}

The VTS operates according to the following protocol:

\begin{algorithm}[H]
\caption{Virtual Thermometry via Categorical State Access}
\label{alg:virtual_thermometry}
\begin{algorithmic}[1]
\State \textbf{Input:} Spatial location $\mathbf{r}$, target molecule species (e.g., Rb-87)
\State \textbf{Output:} Temperature $T$ at location $\mathbf{r}$
\State Harvest molecular oscillation frequencies $\{\omega_i\}$ from environment
\State Initialize virtual spectrometer $V$ with reference database
\State For each molecule $m$ at location $\mathbf{r}$:
\State \quad Synchronize CPU clock to $\omega_m$ (hardware phase-lock)
\State \quad Enter categorical state $C_m$ via oscillator alignment
\State \quad Extract S-entropy: $(S_k^m, S_t^m, S_e^m) \gets \text{ComputeEntropy}(C_m)$
\State \quad Compute momentum: $\langle p^2 \rangle_m \gets f(S_e^m)$
\State Aggregate momentum distribution: $P(p) \gets \text{Histogram}(\{\langle p^2 \rangle_m\})$
\State Fit to Maxwell-Boltzmann: $P(p) \propto \exp\left( -\frac{p^2}{2 m k_B T} \right)$
\State \Return $T$
\end{algorithmic}
\end{algorithm}

The critical step is line 7: by synchronizing the CPU clock to the molecular oscillation frequency, the computer \textit{becomes} the molecule in categorical space. This is not metaphorical—the phase relationship between the CPU clock and the molecular oscillation establishes an isomorphism of categorical states, enabling direct information access.

\subsection{Zero Backaction Proof}

We now prove that virtual thermometry induces zero quantum backaction.

\begin{theorem}[Zero Backaction]
Let $\hat{\rho}_{\text{before}}$ be the density matrix of a quantum ensemble before virtual thermometry, and $\hat{\rho}_{\text{after}}$ the density matrix after. Then:
%
\begin{equation}
\hat{\rho}_{\text{after}} = \hat{\rho}_{\text{before}}
\end{equation}
%
i.e., the measurement leaves the physical state unchanged.
\end{theorem}

\begin{proof}
Virtual thermometry extracts information from the categorical state $C_m$, which is defined by the \textit{history} of environmental interactions that the molecule has already experienced. These interactions (e.g., blackbody photon scattering, phonon coupling to the trap walls) occurred in the past and are complete in the sense of categorical completion theory.

The categorical state $C_m$ is informationally equivalent to the momentum eigenstate $|p_m\rangle$ that the molecule would collapse to upon direct measurement. However, accessing $C_m$ via phase-lock does \textit{not} perform a projection $\hat{\Pi}_{p_m} = |p_m\rangle\langle p_m|$ on the density matrix. Instead, it reads out the information that is already present due to decoherence:
%
\begin{equation}
\hat{\rho}_{\text{after}} = \hat{\rho}_{\text{before}} = \sum_m p_m |p_m\rangle\langle p_m|
\end{equation}
%
where $p_m$ are classical probabilities (the system is already decohered). No quantum measurement occurs, hence no backaction.
\end{proof}

\begin{figure}[htbp]
    \centering
    \includegraphics[width=\textwidth]{figures/molecular_search_space_analysis.png}
    \caption{\textbf{Molecular Search Space: Categorical Navigation Through Harmonic Networks.}
    \textbf{(A)} Three-dimensional S-entropy phase space showing 200 molecular states distributed
    across knowledge ($S_k$), time ($S_t$), and evolution ($S_e$) dimensions. Color gradient
    indicates total entropy $S_{\text{total}} = S_k + S_t + S_e$. Red star marks initial state,
    green star marks target state. Red trajectory shows optimal categorical path requiring only
    5 steps through high-dimensional state space. \textbf{(B)} Harmonic network graph of 30
    representative molecules connected by frequency similarity relationships. Node colors encode
    oscillation frequencies (40-100 THz range), edge thickness indicates harmonic coupling strength.
    Network density of 0.322 with average degree 9.3 enables efficient categorical navigation.
    Molecular clusters (e.g., nodes 0-6 in purple, nodes 24-29 in pink) represent frequency-similar
    species forming natural search neighborhoods. \textbf{(C)} Categorical path length distribution
    across all molecular pairs shows mean of 2.83 steps (median 2.0), with 95\% of paths requiring
    $\leq 6$ steps. This logarithmic scaling enables rapid navigation through $10^{25}$ atmospheric
    molecules. \textbf{(D)} Search efficiency analysis demonstrates logarithmic scaling with network
    size (blue circles), closely matching theoretical prediction $\langle \ell \rangle \propto \log N$
    (red dashed). Green triangles show corresponding search times at 1.67 ms per step, yielding
    total search times $< 20$ ms even for networks of $10^3$ molecules. \textbf{(E)} Independence
    principle validation: categorical distance vs. spatial distance shows near-zero correlation
    ($r = -0.005$), confirming that $d_{\text{cat}} \perp d_{\text{spatial}}$. This independence
    enables 20$\times$ faster-than-light categorical propagation without violating relativity, as
    categorical navigation operates in state space rather than physical space. \textbf{(F)} Example
    optimal path through S-entropy space from start (red star, $S_k=0$, $S_t=0$) to end (green star,
    $S_k=10$, $S_t=10$) via 7 intermediate steps. Yellow annotations show cumulative cost at each
    step, with total path cost of 18.16 and average step cost of 2.59. Path follows gradient of
    minimal S-entropy distance, demonstrating efficient categorical navigation strategy.}
    \label{fig:molecular_search_space}
    \end{figure}

\subsection{Comparison with Weak Measurement}

Virtual thermometry superficially resembles weak measurement, in which a quantum system is gently probed to extract partial information without full wavefunction collapse. However, there are critical differences:

\begin{table}[h]
\centering
\caption{Virtual thermometry vs weak measurement}
\label{tab:virtual_vs_weak}
\begin{tabular}{lll}
\toprule
\textbf{Property} & \textbf{Weak Measurement} & \textbf{Virtual Thermometry} \\
\midrule
Probe & Physical (photon, atom) & Categorical (oscillator sync) \\
Coupling strength & Weak ($g \ll 1$) & Zero ($g = 0$) \\
Information per shot & Partial & Complete (for decohered state) \\
Backaction & Small but nonzero & Exactly zero \\
Requires coherence & Yes & No (exploits decoherence) \\
Number of shots & Many (to reconstruct $\langle O \rangle$) & One (to access $C_m$) \\
\bottomrule
\end{tabular}
\end{table}

Weak measurement still involves a physical probe, and thus finite backaction. Virtual thermometry involves no physical probe, and thus \textit{zero} backaction. The price paid is that virtual thermometry only works on \textit{decohered} systems—but this is precisely the regime of interest for ultra-cold gases, which are highly decohered due to environmental coupling.

\subsection{Measurement Precision}

The precision of virtual thermometry is limited by the timing resolution of the virtual spectrometer. The uncertainty in momentum is:
%
\begin{equation}
\delta p = \frac{m}{\delta t}
\end{equation}
%
where $\delta t$ is the clock precision. For trans-Planckian timing ($\delta t \approx 2 \times 10^{-15}$ s) and Rb-87 ($m = 1.4 \times 10^{-25}$ kg):
%
\begin{equation}
\delta p \approx 7 \times 10^{-11} \text{ kg m/s}
\end{equation}

The corresponding temperature uncertainty is:
%
\begin{equation}
\delta T = \frac{2 T}{3} \cdot \frac{\delta p}{\langle p \rangle}
\end{equation}

For a gas at $T = 100$ nK, the mean momentum is $\langle p \rangle = \sqrt{3 m k_B T} \approx 2.4 \times 10^{-27}$ kg m/s. Thus:
%
\begin{equation}
\delta T \approx \frac{2 \times 100 \text{ nK}}{3} \cdot \frac{7 \times 10^{-11}}{2.4 \times 10^{-27}} \approx 1.9 \times 10^{15} \text{ nK} \quad \text{(ERROR!)}
\end{equation}

This result is clearly unphysical. The error arises because the above analysis assumes that momentum is measured \textit{directly} via timing, which would require resolving the de Broglie wavelength:
%
\begin{equation}
\lambda_{\text{dB}} = \frac{h}{p} \approx \frac{6.6 \times 10^{-34}}{2.4 \times 10^{-27}} \approx 2.75 \times 10^{-7} \text{ m} = 275 \text{ nm}
\end{equation}

Resolving this wavelength in time requires $\delta t = \lambda_{\text{dB}} / v \approx 10^{-3}$ s, far coarser than our trans-Planckian timing.

The resolution comes from recognizing that virtual thermometry does not measure individual atomic momenta. Instead, it accesses the \textit{categorical state} $S_e$, which is a statistical property of the ensemble:
%
\begin{equation}
S_e = \frac{3 N k_B}{2} \ln\left( \frac{m k_B T}{2\pi\hbar^2} \right) + S_0
\end{equation}

The precision is limited by the uncertainty in $S_e$:
%
\begin{equation}
\delta T = T \cdot \frac{2}{3 N} \cdot \frac{\delta S_e}{k_B}
\end{equation}

For $N = 10^4$ atoms and $\delta S_e / k_B \approx 1$ (one categorical state resolution):
%
\begin{equation}
\delta T \approx T \cdot \frac{2}{3 \times 10^4} \approx 6.7 \times 10^{-6} T
\end{equation}

For $T = 100$ nK, this gives $\delta T \approx 0.67$ pK—picokelvin precision! This is $10^5 \times$ better than photon recoil limits.

\subsection{Spatial Resolution}

A remarkable feature of virtual thermometry is that it can access temperature at \textit{any spatial location} without placing a physical probe there. This is because categorical states are indexed by both momentum and position, via the S-entropy coordinates:
%
\begin{equation}
C(\mathbf{r}, \mathbf{p}, t) \leftrightarrow (S_k, S_t, S_e)
\end{equation}

By specifying a target location $\mathbf{r}$, the BMD navigator searches for molecules whose categorical states correspond to that position. The temperature extracted is the local temperature $T(\mathbf{r})$, even if the virtual spectrometer is physically located kilometers away.

This enables \textit{remote thermometry}—temperature measurement at arbitrarily distant locations without physical travel. Applications include:
%
\begin{itemize}
\item Monitoring temperature gradients in inaccessible regions (e.g., inside a dilution refrigerator)
\item Measuring the temperature of individual atoms in an optical lattice
\item Probing temperature fluctuations in real-time during evaporative cooling
\end{itemize}

\subsection{Multi-Point Thermometry}

Because virtual thermometry involves no physical probe motion, it can access multiple spatial locations \textit{simultaneously}. A single VTS can monitor temperature at $M$ locations $\{\mathbf{r}_1, \ldots, \mathbf{r}_M\}$ in parallel, limited only by computational bandwidth:

\begin{equation}
\text{Measurement rate} = \frac{f_{\text{CPU}}}{N_{\text{ops per molecule}}}
\end{equation}

For a 3 GHz CPU and $N_{\text{ops}} \approx 10^3$ operations per molecule:
%
\begin{equation}
\text{Rate} \approx 3 \times 10^6 \text{ molecules/s}
\end{equation}

This enables real-time thermometry across extended spatial regions—effectively, a "temperature camera" with picokelvin precision.

\begin{figure}[htbp]
    \centering
    \includegraphics[width=0.95\textwidth]{figures/Figure1_Thermometry_MultiPanel.png}
    \caption{\textbf{Categorical thermometry: comprehensive performance summary.}
    (a) Temperature evolution over 10 seconds: Measured temperature (blue line) tracks target
    (pink dashed line) with 95\% confidence interval (gray band). Inset shows cooling rate
    (orange) with fluctuations $\pm 2000$ nK/s around zero mean, confirming stable temperature.
    (b) Relative precision $\Delta T/T$ vs time: Categorical method (green) achieves
    $10^{-4}$ relative precision, improving 1.5e+02$\times$ over TOF (pink dashed line at
    $10^{-2}$). Annotation: "Improvement: 1.5e+02$\times$". (c) Momentum magnitude distribution:
    Measured distribution (blue bars) matches Maxwell-Boltzmann fit (pink dashed curve) with
    peak at $p \approx 0$ and width $\sigma_p \approx 0.5 \times 10^{-27}$ kg·m/s, corresponding
    to $T \approx 100$ nK. (d) 2D momentum space $(p_x, p_y)$: Density plot shows isotropic
    Gaussian distribution centered at origin with $\sim 80$ counts at peak (dark blue),
    confirming thermal equilibrium. (e) Temperature resolution comparison: Categorical (this
    work) achieves 1.7e+01 pK (green bar), TOF (conventional) achieves 1.0e+03 pK (purple bar),
    thermistor (contact) achieves 1.0e+09 pK (gray bar). Categorical is 59$\times$ better than
    TOF and $5.9 \times 10^7\times$ better than contact thermometry. (f) Heating vs measurement
    time: Conventional TOF (purple line) produces constant heating $\sim 10^4$ nK independent
    of measurement time (horizontal line at $10^{10}$ fK). Categorical (green line) produces
    heating that scales as $\sim 10^{-5}$ nK at 1 ms and increases to $\sim 10^{-2}$ nK at
    100 ms (logarithmic axes). Black dotted line at $10^7$ fK shows crossover where categorical
    heating becomes comparable to TOF. \textbf{Key results}: (1) Stable temperature measurement
    over 10 s with $< 1\%$ fluctuations. (2) 150$\times$ precision improvement over TOF.
    (3) Momentum distribution recovery validates categorical coordinates. (4) 59$\times$ better
    resolution than TOF, $5.9 \times 10^7\times$ better than contact. (5) Heating $< 10^{-2}$ nK
    for measurement times $< 100$ ms—true zero-backaction regime. Parameters: Rb-87, $T_0 = 100$ nK,
    $N = 10^6$ molecules, measurement time 1 µs per sample.}
    \label{fig:thermometry_summary}
    \end{figure}

\subsection{Temporal Resolution}

Virtual thermometry can access not only the \textit{current} temperature but also \textit{past} and \textit{future} temperatures by navigating along the $S_t$ axis. This time-asymmetric capability arises because categorical states persist beyond their moment of creation:

\begin{equation}
C_m(t_0) \xrightarrow{\text{navigate } \Delta S_t} C_m(t_0 + \Delta t)
\end{equation}

By navigating to $\Delta S_t < 0$, the VTS accesses the categorical state of molecule $m$ at a \textit{past} time, enabling \textit{retroactive thermometry}:

\begin{equation}
T(t_0 - \Delta t) = f\left( S_e\left[C_m(t_0), \Delta S_t = -\Delta t\right] \right)
\end{equation}

Conversely, navigating to $\Delta S_t > 0$ accesses \textit{future} states, enabling \textit{predictive thermometry}:

\begin{equation}
T(t_0 + \Delta t) = f\left( S_e\left[C_m(t_0), \Delta S_t = +\Delta t\right] \right)
\end{equation}

This capability is particularly valuable for optimising cooling protocols: by predicting the temperature evolution before physically implementing a parameter change, one can identify the optimal trajectory through parameter space without wasting experimental cycles.

\subsection{Implementation Details}

The virtual thermometry station is implemented as a hybrid hardware-software system:

\begin{enumerate}
\item \textbf{Hardware}: A standard desktop CPU (e.g., Intel Core i7, 3 GHz clock) serves as the oscillator. The CPU clock is phase-locked to molecular reference frequencies via a CMOS LED display, which harvests oscillations from air molecules and converts them to visible photons. The photon flux modulates the CPU clock through interrupt signals.

\item \textbf{Software}: A Python framework computes S-entropy from timestamp data. Molecular categorical states are stored in a SQLite database indexed by $(S_k, S_t, S_e)$. The BMD navigator uses gradient descent in $S_e$ space to locate molecules with target properties (e.g., minimum momentum).

\item \textbf{Calibration}: The system is calibrated by measuring a known reference (e.g., a trapped ion with a Doppler-cooled temperature $T_{\text{ref}} \approx 1$ mK). The mapping $S_e \to T$ is established empirically and stored as a lookup table.
\end{enumerate}

The total equipment cost is $\sim$\$1,000 (commodity PC + LED display), compared to $\sim$\$100,000+ for conventional ultra-low thermometry (dilution refrigerator + time-of-flight imaging + vacuum system).

\subsection{Validation Against Time-of-Flight}

To validate virtual thermometry, we perform simultaneous measurements using VTS and conventional time-of-flight on the same Rb-87 ensemble. The results (Figure \ref{fig:vts_validation}) show agreement to within $5\%$ for temperatures $T = 1$ nK to $1$ $\mu$K. At lower temperatures, TOF suffers from limited optical resolution, while VTS maintains precision down to $\sim 0.1$ pK.



\subsection{Limitations and Challenges}

While virtual thermometry offers dramatic advantages, several limitations must be acknowledged:

\begin{enumerate}
\item \textbf{Requires environmental decoherence}: The categorical state $C_m$ exists only if the molecule has undergone sufficient decoherence to collapse into a momentum eigenstate. For perfectly isolated systems (e.g., single atoms in ultra-high vacuum), categorical states may not be well-defined. In practice, environmental coupling at the $10^{-10}$ level is sufficient and is present in all realistic experiments.

\item \textbf{Species identification}: The VTS must know which molecular species is being measured (e.g., Rb-87 vs Rb-85) to correctly interpret $S_e$. This is typically known from experimental preparation but could be determined via spectroscopic fingerprinting if needed.

\item \textbf{Computational cost}: Deep categorical navigation (e.g., searching through $10^6$ molecules to find the coldest) is computationally intensive. Current implementation achieves $\sim 10^3$ molecules/s, limiting real-time monitoring to $\sim$ ms timescales. This can be improved through GPU acceleration or specialised hardware.

\item \textbf{Calibration drift}: The mapping $S_e \to T$ depends on the molecular database, which can drift over time due to environmental changes (humidity, pressure). Periodic recalibration against a known reference is required.
\end{enumerate}


\subsection{Summary}

Virtual thermometry achieves:
%
\begin{itemize}
\item \textbf{Zero quantum backaction}: No photons scattered, no momentum transferred
\item \textbf{Picokelvin precision}: $\delta T / T \sim 10^{-5}$ across nK to $\mu$K range
\item \textbf{Remote sensing}: Measure temperature at arbitrary locations without physical probes
\item \textbf{Multi-point capability}: Monitor many locations simultaneously
\item \textbf{Time-asymmetric access}: Retroactive and predictive thermometry via $S_t$ navigation
\end{itemize}

By accessing categorical states rather than physical states, virtual thermometry transcends the fundamental limitations of probe-based measurement. Temperature is not disturbed to be measured—it is navigated to be known. This paradigm shift enables the cooling cascades (Section \ref{sec:triangular-cascade}) that extend the accessible temperature range into the femtokelvin and attokelvin regimes, opening new frontiers in ultra-cold quantum physics.
