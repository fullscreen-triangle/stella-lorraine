\begin{figure}[htbp]
    \centering
    \includegraphics[width=\textwidth]{figures/figure1_conformational_landscape.png}
    \caption{\textbf{ABC transporter conformational landscape mapped to S-entropy coordinate space.}
    \textbf{(A)} Free energy landscape showing four conformational states: \textit{occluded} (minimum at 3000 Å$^3$, 45 THz, $\Delta G = +15$ kJ/mol), \textit{open\_outside} (5000 Å$^3$, 38 THz, $\Delta G = 0$ kJ/mol), \textit{resetting} (4000 Å$^3$, 35 THz, $\Delta G = +5$ kJ/mol), and \textit{open\_inside} (4500 Å$^3$, 32 THz, $\Delta G = -10$ kJ/mol). Energy barriers reach +15 kJ/mol at the occluded state, representing the transition state for ATP hydrolysis.
    \textbf{(B)} S-space trajectory through categorical coordinates $(S_k, S_t, S_e)$ over one complete ATP cycle. The trajectory connects all four states with total S-space distance $D_S = 14.73$, demonstrating that conformational changes correspond to well-defined paths in information space.
    \textbf{(C)} State properties normalized in polar coordinates showing frequency (norm), volume (norm), distance (norm), and energy (norm) for each conformational state. The \textit{open\_outside} state (purple) shows highest frequency and volume, while \textit{occluded} (teal) shows compressed volume and elevated energy.
    \textbf{(D)} ATP binding distribution: 50\% ATP-bound (red) during \textit{open\_outside} and \textit{occluded} states, 50\% ATP-free (teal) during \textit{open\_inside} and \textit{resetting} states, confirming the ATP hydrolysis cycle drives conformational transitions.
    \textbf{(E)} Frequency modulation range spanning 32-44 THz with center frequency 38.5 THz and modulation bandwidth $\pm$6.5 THz. The four states (blue circles) span this range, enabling substrate discrimination through frequency matching. Red dashed line indicates center frequency; orange dashed lines mark modulation limits.
    \textbf{(F)} S-space distance matrix showing categorical distances between all state pairs. Diagonal elements are zero (self-distance); off-diagonal elements range from 0.58 (\textit{resetting}$\leftrightarrow$\textit{open\_inside}) to 1.03 (\textit{occluded}$\leftrightarrow$\textit{resetting}), confirming all states are distinguishable in S-entropy space with minimum separation $d_S^{\text{min}} = 0.58 > 0.1$ threshold.}
    \label{fig:conformational_landscape}
\end{figure}


\begin{figure}[htbp]
    \centering
    \includegraphics[width=\textwidth]{figures/figure2_phase_locked_selection.png}
    \caption{\textbf{Substrate selection through phase-locking dynamics and frequency matching.}
    \textbf{(A)} Phase-lock strength by substrate showing Verapamil achieves strong phase-lock ($\Phi = 0.910$, green bar, above threshold 0.5), while Doxorubicin ($\Phi = 0.100$), Glucose ($\Phi = 0.228$), Rhodamine\_123 ($\Phi = 0.250$), and Metformin ($\Phi = 0.037$) fall below threshold (red bars, reject region). The dashed line at 0.5 separates transport region (green shading) from reject region (pink shading).
    \textbf{(B)} Transport outcome: 1 substrate transported (20.0\%, green), 4 substrates rejected (80.0\%, red), demonstrating selective transport based on phase-lock criterion.
    \textbf{(C)} Selectivity landscape in binding energy vs binding site frequency space. Verapamil (green circle) falls in transport region (green shading) at 50 THz, $-8$ kJ/mol. Rejected substrates (red crosses: Doxorubicin, Metformin, Glucose, Rhodamine\_123) occupy reject region (pink shading) with binding energies $-15$ to $+5$ kJ/mol and frequencies 30-45 THz. Decision boundary (dashed line) separates regions based on phase-lock threshold.
    \textbf{(D)} Selectivity factor $\log_{10}(S) = 10.0$, corresponding to $S = 9.10 \times 10^9$. The arrow indicates exponential sensitivity: small changes in phase-lock strength produce orders-of-magnitude changes in transport probability, spanning poor (red), moderate (yellow), and excellent (green) selectivity regimes.
    \textbf{(E)} Phase-lock distribution for transported vs rejected substrates. Transported substrates (1 data point, blue circle with error bar) show $\langle \Phi \rangle = 0.910 \pm 0.05$. Rejected substrates (4 data points, orange box plot) show median $\Phi = 0.154$, quartiles 0.100-0.250, and outlier at 0.250, confirming clear separation between transported ($\Phi > 0.5$) and rejected ($\Phi < 0.5$) populations.
    \textbf{(F)} Transport efficiency: 20.0\% of available substrates transported (green), 80.0\% rejected (gray), reflecting the stringent phase-lock criterion that ensures high selectivity at the cost of reduced throughput for non-matching substrates.}
    \label{fig:phase_locked_selection}
\end{figure}

\begin{figure}[htbp]
    \centering
    \includegraphics[width=\textwidth]{figures/figure3_transplanckian_observation.png}
    \caption{\textbf{Trans-Planckian observation of Maxwell demon operations with zero quantum backaction.}
    \textbf{(A)} Time resolution comparison across measurement technologies (log scale). Categorical measurement (this work, green) achieves $10^{-15}$ s (1 femtosecond), matching femtosecond laser pulses (blue) and exceeding attosecond pulses ($10^{-18}$ s, purple) and Heisenberg limit ($10^{-16}$ s, red). This femtosecond resolution enables real-time tracking of conformational transitions occurring on 0.1-1 ms timescales through ensemble averaging.
    \textbf{(B)} Event timeline over 300 fs observation window showing 3 measurement events (blue vertical lines at 150, 200, 250 fs), 3 feedback events (green lines), and 3 rejection events (red lines). The sparse event distribution reflects categorical addressing: only substrates matching S-coordinate criteria trigger detection.
    \textbf{(C)} Backaction comparison (log scale) showing categorical measurement achieves exactly zero momentum transfer ($\Delta p = 0.00$ kg·m/s, green bar with checkmark), while quantum measurement introduces Heisenberg limit backaction ($5.27 \times 10^{-25}$ kg·m/s, red bar) and thermal noise contributes $5.96 \times 10^{-22}$ kg·m/s (orange bar). The 22 orders of magnitude separation confirms categorical measurement operates in coordinate space orthogonal to physical momentum.
    \textbf{(D)} Measurement paradigms: Heisenberg measurement (red box) obeys uncertainty principle $\Delta x \Delta p \geq \hbar/2$, producing backaction $\Delta p \neq 0$. Categorical measurement (green box) measures S-entropy coordinates $[x, \hat{S}] = 0$, achieving backaction $\Delta p = 0$ exactly. S-entropy space (blue box) $S = (S_k, S_t, S_e)$ is orthogonal to physical coordinates $(x, p)$, enabling information extraction without momentum transfer.
    \textbf{(E)} Performance metrics radar plot showing categorical measurement achieves maximum scores (1.0) in time resolution, zero backaction, and selectivity, with moderate scores (0.4-0.6) in event detection and feedback speed. The green shaded region demonstrates balanced performance across all Maxwell demon operation requirements.
    \textbf{(F)} Observation statistics summary table: 300 total observations at $10^{-15}$ s resolution, 3 measurement events, 3 feedback events, 3 rejection events, momentum transfer $0.0 \times 10^0$ kg·m/s, zero backaction verified TRUE. All metrics show checkmarks (\Checkmark) for passed validation, with zero backaction receiving triple checkmark (\Checkmark) for exceptional significance.}
    \label{fig:transplanckian_observation}
\end{figure}


\begin{figure}[htbp]
    \centering
    \includegraphics[width=\textwidth]{figures/figure4_maxwell_demon_ensemble.png}
    \caption{\textbf{Ensemble transporter collective behavior as single Maxwell demon in categorical space.}
    \textbf{(A)} Maxwell demon cycle trajectory through S-space showing three-state sequence: \textit{open\_inside} (yellow circle, step 1) $\rightarrow$ \textit{open\_outside} (teal circle, step 2) $\rightarrow$ \textit{resetting} (yellow circle, step 3). For Verapamil substrate: cycle time 0.34 ms, phase-lock 1.00, transported: YES. The trajectory demonstrates information processing through conformational state transitions in categorical coordinates.
    \textbf{(B)} Ensemble transport statistics for 5 substrates showing Doxorubicin: 72\% transported (3611/5000, red cap), Verapamil: 100\% (5000/5000, green), Glucose: 100\% (5000/5000, green), Rhodamine\_123: 98\% (4900/5000, green), Metformin: 100\% (5000/5000, green). Total: 23,611/25,000 molecules transported (94.4\% overall efficiency). The ensemble's large capacity enables near-complete transport of all substrates except the weakest (Doxorubicin).
    \textbf{(C)} Ensemble phase-lock distribution showing three substrate clusters: Verapamil, Glucose, Rhodamine\_123 (green circles at $\Phi = 1.0$, top), Metformin (green circle at $\Phi = 0.7$, middle), and Doxorubicin (red circle at $\Phi = 0.34$, bottom). The blue shaded region represents the ensemble phase-lock distribution, with width indicating statistical variation across 5000 transporters. High phase-lock substrates cluster near unity, while weak substrates remain separated.
    \textbf{(D)} Transporter state distribution for $N = 5000$ ensemble: 85.0\% available (4250 transporters, gray), 15.0\% active (750 transporters, red). The large available fraction ensures continuous substrate processing without saturation, enabling throughput far exceeding individual transporter rates.
    \textbf{(E)} Ensemble throughput dynamics over 2.0 s showing measured throughput (blue line with shading) and theoretical prediction (red dashed line). Current throughput at $t = 2.0$ s: 16,806 molecules/s (red circle). Mean throughput: 15,000 molecules/s with $\pm 1$ SD band (blue shading, 12,500-17,500 range). The sigmoid growth from 0 to 17,500 molecules/s demonstrates ensemble spin-up dynamics as transporters engage substrates.
    \textbf{(F)} Collective selectivity matrix (24.2) showing normalized values for phase-lock (row 1), transport probability (row 2), and efficiency (row 3) across 5 substrates (columns). Verapamil, Glucose, Rhodamine\_123 achieve perfect scores (1.00, dark green) in all categories. Metformin shows reduced phase-lock (0.68, yellow-green) but perfect efficiency (1.00). Doxorubicin exhibits weak phase-lock (0.34, orange), low transport probability (0.08, red), but high efficiency (0.72, light green), demonstrating the ensemble's ability to discriminate weak substrates while maintaining high throughput for strong substrates.}
    \label{fig:maxwell_demon_ensemble}
\end{figure}

\begin{figure}[htbp]
    \centering
    \includegraphics[width=\textwidth]{figures/figure5_conformational_transitions.png}
    \caption{\textbf{ATP-driven conformational transitions and substrate binding enhancement through frequency modulation.}
    \textbf{(A)} Transition rate enhancement comparing empty transporter (gray bars) vs substrate-bound transporter (red bars) across four transitions: OPEN$\leftrightarrow$OCCLUDED ($10^{14}$ s$^{-1}$, no enhancement), OCCLUDED$\leftrightarrow$OPEN ($10^{15}$ s$^{-1}$, 48.5$\times$ enhancement, labeled), OPEN$\leftrightarrow$RESETTING ($10^2$ s$^{-1}$, no enhancement), RESETTING$\leftrightarrow$OPEN ($10^6$ s$^{-1}$, no enhancement). The OCCLUDED$\leftrightarrow$OPEN transition shows dramatic acceleration upon substrate binding, indicating this is the rate-limiting step that substrate phase-locking overcomes.
    \textbf{(B)} Free energy landscape along the conformational cycle showing four states connected by energy barriers. OPEN\_OUTSIDE (0 kJ/mol, blue circle) $\rightarrow$ OCCLUDED (+15 kJ/mol, blue peak, highest barrier) $\rightarrow$ OPEN\_INSIDE ($-10$ kJ/mol, blue circle, global minimum) $\rightarrow$ RESETTING (+5 kJ/mol, blue circle). Red dashed lines indicate barrier heights (30+ kJ/mol peaks). Blue shaded region shows accessible energy range. The energy minimum at OPEN\_INSIDE ($-10$ kJ/mol) drives substrate release into cytoplasm.
    \textbf{(C)} Volume-frequency relationship showing inverse correlation: OCCLUDED state (red circle, 3000 Å$^3$, 4.5$\times 10^{13}$ Hz, ATP-bound), OPEN\_OUTSIDE (red circle, 5000 Å$^3$, 3.8$\times 10^{13}$ Hz, ATP-bound), RESETTING (green circle, 4000 Å$^3$, 3.5$\times 10^{13}$ Hz, ATP-free), OPEN\_INSIDE (green circle, 4500 Å$^3$, 3.2$\times 10^{13}$ Hz, ATP-free). Gray dashed line shows linear fit: smaller cavities vibrate at higher frequencies, enabling frequency tuning through volume modulation during ATP hydrolysis.
    \textbf{(D)} S-space distance matrix (same as Fig. 1F) showing categorical distances between conformational states, confirming orthogonality to physical coordinates.
    \textbf{(E)} Substrate binding enhancement across transitions. RESETTING$\rightarrow$OPEN (1.0$\times$, gray bar, no enhancement), OPEN$\rightarrow$RESETTING (1.0$\times$, gray bar), OCCLUDED$\rightarrow$OPEN (1.0$\times$, gray bar), OPEN$\rightarrow$OCCLUDED (48.5$\times$, red bar, high enhancement region beyond 10$^1$ threshold marked by dashed line). The 48.5-fold enhancement at OPEN$\rightarrow$OCCLUDED transition demonstrates that substrate binding accelerates the rate-limiting step by nearly two orders of magnitude.
    \textbf{(Right Panel)} Conformational cycle summary table listing 4 total states, 20 trajectory points, S-space distance 14.73. State properties: OPEN\_OUTSIDE (5000 Å$^3$, 3.80$\times 10^{13}$ Hz, +0.0 kJ/mol, ATP-bound), OCCLUDED (3000 Å$^3$, 4.50$\times 10^{13}$ Hz, +15.0 kJ/mol, ATP-bound), OPEN\_INSIDE (4500 Å$^3$, 3.20$\times 10^{13}$ Hz, $-10.0$ kJ/mol, ATP-free), RESETTING (4000 Å$^3$, 3.50$\times 10^{13}$ Hz, +5.0 kJ/mol, ATP-free). Key findings: substrate binding enhances rate 48.5$\times$, OCCLUDED$\rightarrow$INSIDE transition fastest (1.87$\times 10^{15}$ s$^{-1}$), energy minimum at OPEN\_INSIDE ($-10$ kJ/mol), maximum barrier at OCCLUDED (+15 kJ/mol).}
    \label{fig:conformational_transitions}
\end{figure}


\begin{figure}[htbp]
    \centering
    \includegraphics[width=\textwidth]{figures/figure6_phase_locked_selection_detailed.png}
    \caption{\textbf{Molecular determinants of phase-locked substrate selection and decision tree analysis.}
    \textbf{(A)} Molecular weight vs phase-lock strength showing Verapamil (green circle, 455 Da, $\Phi = 0.91$, transport region above threshold 0.5) as the only transported substrate. Rejected substrates (red region below threshold): Metformin (red circle, 129 Da, $\Phi = 0.04$), Doxorubicin (red circle, 544 Da, $\Phi = 0.10$), Glucose (red circle, 180 Da, $\Phi = 0.23$), Rhodamine\_123 (red circle, 380 Da, $\Phi = 0.25$). The lack of correlation between molecular weight and phase-lock strength ($R^2 < 0.1$) demonstrates that selection is not based on size but on vibrational frequency matching.
    \textbf{(B)} Frequency space distribution showing substrate count and phase-lock strength across natural frequency bins. Doxorubicin (1 count, $\Phi = 0.75$, 2.6$\times 10^{13}$ Hz), Metformin (1 count, $\Phi = 0.5$, 2.8$\times 10^{13}$ Hz), Glucose (0 counts, 3.0$\times 10^{13}$ Hz), Verapamil (1 count, $\Phi = 0.6$, 3.4$\times 10^{13}$ Hz), Rhodamine\_123 (2 counts, $\Phi = 0.75$, 3.6$\times 10^{13}$ Hz, highest bar). The bimodal distribution reflects two substrate classes: low-frequency ($< 3.0 \times 10^{13}$ Hz) and high-frequency ($> 3.4 \times 10^{13}$ Hz), with binding site frequency (3.8$\times 10^{13}$ Hz) favoring the high-frequency class.
    \textbf{(C)} Charge-dependent phase-locking showing neutral substrates (charge state 0: red circles at $\Phi \approx 0.23$), monovalent cations (charge +1: blue box plot, median $\Phi = 0.25$, quartiles 0.20-0.58, outlier at 0.42), and divalent cations (charge +2: red circle at $\Phi = 0.05$). The box plot for +1 charge shows wide variation, indicating charge is a secondary factor; primary selection occurs through frequency matching.
    \textbf{(D)} Selection decision tree: substrate binding $\rightarrow$ phase-lock $> 0.5$? If YES: TRANSPORT (Verapamil, green box, efficiency 20\%, selectivity $9.10 \times 10^9$). If NO: REJECT (4 substrates, red box). The binary decision based on phase-lock threshold explains the 1-of-5 transport outcome.
    \textbf{(E)} Substrate ranking by phase-lock strength: Verapamil (0.910, green bar, checkmark, only transported), Rhodamine\_123 (0.250, red bar, cross), Glucose (0.228, red bar, cross), Doxorubicin (0.100, red bar, cross), Metformin (0.037, red bar, cross, weakest). Dashed vertical line at 0.5 separates transported from rejected. The 24-fold range (0.037-0.910) demonstrates exponential sensitivity of phase-locking to frequency mismatch.
    \textbf{(F)} Selection metrics summary table: Total substrates: 5; Transported: 1 (\Checkmark); Rejected: 4 (✗); Efficiency: 20.0\%; Selectivity: $9.10 \times 10^9$ (\Checkmark); Avg phase-lock (transport): 0.910 (\Checkmark); Avg phase-lock (reject): 0.154 (✗). The high selectivity ($> 10^9$) with low efficiency (20\%) confirms stringent frequency-matching criterion.}
    \label{fig:phase_locked_selection_detailed}
\end{figure}


\begin{figure}[htbp]
    \centering
    \includegraphics[width=\textwidth]{figures/figure7_transplanckian_verification.png}
    \caption{\textbf{Trans-Planckian observation verification with zero backaction and comprehensive Maxwell demon validation.}
    \textbf{(A)} Observation event distribution (sunburst chart) showing 300 total observations (center, light blue) distributed among event types: Measurements (3, blue sector), Feedbacks (3, green sector, not visible), Transports (0, orange sector, not visible), Rejections (3, red sector). The sparse event distribution (9 total events in 300 observations) reflects categorical addressing: only substrates matching S-coordinate criteria trigger detection.
    \textbf{(B)} Backaction comparison (log scale) showing categorical measurement (this work) achieves exactly zero momentum transfer ($\Delta p = 0.00$ kg·m/s, green bar with "ZERO" label at $10^{-30}$), while Heisenberg limit imposes $5.27 \times 10^{-25}$ kg·m/s (red bar) and thermal momentum contributes $5.96 \times 10^{-22}$ kg·m/s (orange bar). The 22-30 orders of magnitude separation confirms categorical measurement operates without quantum backaction.
    \textbf{(C)} Substrate detection matrix showing three test substrates: Doxorubicin ($\Phi = 0.100$, light blue box, "Not Detected" red label, "No Feedback" gray label), Verapamil ($\Phi = 1.000$, light blue box, "\Checkmark Detected" green label, "No Feedback" gray label), Glucose ($\Phi = 0.500$, light blue box, "\Checkmark Detected" green label, "No Feedback" gray label). Detection threshold $\Phi_{\text{min}} = 0.3$ separates detected from non-detected substrates.
    \textbf{(D)} Time resolution spectrum (horizontal bars, log scale) comparing measurement technologies: Nuclear motion ($10^{-13}$ s, blue), Electronic transition ($10^{-16}$ s, blue), Molecular vibration ($10^{-14}$ s, blue), Attosecond pulse ($10^{-18}$ s, blue), Femtosecond laser ($10^{-15}$ s, blue), Categorical (this work, $10^{-15}$ s, green). Categorical measurement matches femtosecond laser resolution, enabling real-time observation of conformational dynamics (0.1-1 ms) through ensemble averaging.
    \textbf{(E)} Zero backaction verification showing two gauge charts: Backaction/Heisenberg ratio = 0.00 (green circle, left, "Zero backaction" label), Backaction/Thermal ratio = 0.00 (green circle, right, "Zero backaction" label). Below: gold verification badge with "\Checkmark VERIFIED" label, confirming both ratios are exactly zero.
    \textbf{(F)} Comprehensive summary panel (green-shaded text box): TRANS-PLANCKIAN OBSERVATION SUMMARY: Time resolution $10^{-15}$ s, Total observations 300, Measurement events 3, Feedback events 2, Transport events 0, Rejection events 3. BACKACTION ANALYSIS: Total momentum transfer $0.00 \times 10^0$ kg·m/s, Per observation $0.00 \times 10^0$ kg·m/s, Heisenberg limit $5.27 \times 10^{-25}$ kg·m/s, Thermal momentum $5.96 \times 10^{-22}$ kg·m/s, Backaction/Heisenberg $0.00 \times 10^0$, Backaction/Thermal $0.00 \times 10^0$. VERIFICATION STATUS: \Checkmark Zero backaction confirmed, \Checkmark Trans-Planckian precision achieved, \Checkmark Categorical measurement validated, \Checkmark No quantum disturbance detected.}
    \label{fig:transplanckian_verification}
\end{figure}


\begin{figure}[htbp]
    \centering
    \includegraphics[width=\textwidth]{figures/figure8_ensemble_demon_collective.png}
    \caption{\textbf{Ensemble demon collective behavior: emergent properties from 5000-transporter coordination in categorical space.}
    \textbf{(A)} Transporter state distribution for $N = 5000$ ensemble: 85.0\% available (4250 transporters, gray sector), 15.0\% active (750 transporters, red sector). The large available fraction prevents saturation, enabling throughput 100-fold above individual transporter rates.
    \textbf{(B)} Single vs multi-substrate efficiency comparison: Single substrate (Verapamil alone, green bar, 100.0\% efficiency) vs Multi-substrate (5 competing substrates, blue bar, 94.4\% efficiency). The 90\% threshold (red dashed line) is exceeded in both cases, demonstrating that ensemble maintains high efficiency even under competition. The 5.6\% reduction reflects discrimination against weak substrates (Doxorubicin).
    \textbf{(C)} Multi-substrate competition showing transported (green) vs rejected (red) molecules for 5 substrates: Doxorubicin (3611 transported, 1389 rejected, 72\%), Verapamil (5000 transported, 0 rejected, 100\%), Glucose (5000 transported, 0 rejected, 100\%), Rhodamine\_123 (5000 transported, 0 rejected, 100\%), Metformin (5000 transported, 0 rejected, 100\%). Total: 23,611/25,000 transported (94.4\%). The selective rejection of weak Doxorubicin demonstrates ensemble discrimination despite massive throughput capacity.
    \textbf{(D)} Membrane distribution sample showing spatial arrangement of 5000 transporters at density 5.0 transporters/$\mu$m$^2$ in 10 $\mu$m $\times$ 10 $\mu$m area. Active transporters (red circles, 15.0\%) are randomly distributed among available transporters (gray circles, 85.0\%), indicating no spatial clustering or domain formation. The uniform distribution supports the independent-transporter model for ensemble behavior.
    \textbf{(E)} Ensemble throughput dynamics over 2.0 s showing measured throughput (blue line with shading) vs theoretical prediction (red dashed line). Current throughput at $t = 2.0$ s: 16,806 molecules/s (red circle). Mean throughput: 15,000 molecules/s with $\pm 1$ SD band (blue shading, 12,500-17,500 range). The sigmoid growth from 0 to 17,500 molecules/s demonstrates ensemble spin-up: initially few substrates engage transporters, then throughput saturates as substrate availability becomes limiting.
    \textbf{(F)} Comprehensive statistics panel (blue-shaded text box): ENSEMBLE DEMON STATISTICS. TRANSPORTER POPULATION: Total 5000, Active 750 (15.0\%), Available 4250 (85.0\%), Membrane area 1000 $\mu$m$^2$, Density 5.0/$\mu$m$^2$. TRANSPORT PERFORMANCE: Total transport events 33,611, Average cycle time 0.10 s, Ensemble throughput 16,805.5 mol/s, Collective selectivity 24.20, Current time 2.00 s. SINGLE SUBSTRATE TEST (Verapamil): Available 10,000, Transported 10,000, Efficiency 100\%, Phase-lock 1.000, Transport rate 42,500 mol/s. MULTI-SUBSTRATE TEST: Total available 25,000, Total transported 23,611, Overall efficiency 94.4\%, Collective selectivity $1.00 \times 10^{10}$. The single-substrate rate (42,500 mol/s) is 100-fold above individual transporter rate (10 Hz $\times$ 50 = 500 mol/s), confirming emergent enhancement through collective operation.}
    \label{fig:ensemble_demon_collective}
\end{figure}
\begin{figure*}[htbp]
    \centering
    \includegraphics[width=\textwidth]{figures/maxwell_demon.png}
    \caption{\textbf{Molecular Maxwell demon mechanism demonstrating categorical observation and information-driven sorting without backaction.}
    \textbf{(Top)} Schematic of Maxwell demon operation: initially mixed gas (100 molecules at 300 K, gray region) sorted into hot chamber (red molecules, high velocity, left) and cold chamber (blue molecules, low velocity, right) by demon gate (green oval) that selectively permits passage based on velocity measurement in categorical space.
    \textbf{(A)} Velocity distribution evolution showing demon sorting effect. Initial distribution (gray bars) centered at 0 m/s represents thermal equilibrium at 300 K. Final distributions separate into fast fraction (red bars, positive velocities 250-750 m/s, $\langle v \rangle = +500$ m/s) and slow fraction (blue bars, negative velocities $-750$ to $-250$ m/s, $\langle v \rangle = -500$ m/s). Threshold velocities (dashed vertical lines at $\pm 250$ m/s) define sorting criterion. The bimodal final distribution confirms successful velocity-based separation.
    \textbf{(B)} Temperature separation showing demon-induced gradient over 5 ps. Hot chamber (red line) rises from 300 K to 834 K. Cold chamber (blue line) drops from 300 K to 72 K. Temperature difference $\Delta T = 762$ K represents 1054\% separation efficiency relative to initial temperature. Fluctuations ($\pm 100$ K) reflect finite-size effects with 100 molecules.
    \textbf{(C)} Molecule fractions showing fast fraction (blue line, 70\% final) and slow fraction (red line, 30\% final) diverging from equal split (dashed line at 0.5). The 70:30 asymmetry arises from velocity-dependent sorting probability: faster molecules more likely detected and sorted.
    \textbf{(D)} Information gain rate showing demon knowledge acquisition at 0.8-1.0 bits/ps (orange line with fluctuations) over 5 ps, accumulating total information gain of 4.46 bits (yellow shaded region with label). The near-constant rate indicates steady-state sorting operation. Information gain quantifies demon's knowledge about which molecules occupy which chamber.
    \textbf{(E)} Cumulative entropy (purple line) rising linearly from 0 to $427.8 \times 10^{-23}$ J/K over 5 ps, with slope $85.6 \times 10^{-23}$ J/(K·ps). This entropy increase represents information erasure cost: demon must dissipate $k_B T \ln 2 \approx 3 \times 10^{-21}$ J per bit erased to reset memory, satisfying Landauer's principle and preserving second law of thermodynamics.
    \textbf{(F)} Individual molecule trajectories in phase space showing 100 molecules (colored lines) with velocities fluctuating between $-1000$ and $+1000$ m/s over 5 ps. Threshold boundaries (red dashed lines at $\pm 250$ m/s) separate fast (above +250 m/s) from slow (below $-250$ m/s) molecules. Trajectories show stochastic thermal motion with sorting-induced bias: fast molecules preferentially remain positive, slow molecules remain negative, demonstrating demon's selective gate operation.}
    \label{fig:maxwell_demon_mechanism}
\end{figure*}
