\subsection{From Individual to Collective}

Cells express \num{1000}-\num{10000} copies of each ABC transporter type~\cite{beck2011absolute,wisniew2009universal}. Standard models treat these as independent entities. We propose that all transporters of one type constitute a \textbf{collective demon} - a single entity in categorical space representing the ensemble.

This parallels atmospheric molecular demons: rather than tracking \num{10^{20}} individual molecules, one demon represents all molecules at a given S-coordinate simultaneously.

\subsection{Ensemble State Distribution}

Instead of tracking individual transporters, we model the probability distribution over conformational states:
\begin{equation}
P(\text{state}) = \{\text{OPEN\_OUTSIDE}: 0.85, \text{OCCLUDED}: 0.05, \text{OPEN\_INSIDE}: 0.05, \text{RESETTING}: 0.05\}
\end{equation}

For $N$ transporters, $N \times P(\text{state})$ occupy each state. At steady state (no substrate), most (\SI{85}{\percent}) wait in OPEN\_OUTSIDE, ready to bind substrates.

\subsection{Collective S-Coordinate}

The ensemble S-coordinate is the weighted average:
\begin{equation}
\mathbf{S}_{\text{ens}} = \sum_{\text{states}} P(\text{state}) \cdot \mathbf{S}_{\text{state}}
\end{equation}

This single S-coordinate represents the entire ensemble's categorical state, enabling categorical addressing of all \num{5000} transporters simultaneously.

\begin{figure}[htbp]
    \centering
    \includegraphics[width=\textwidth]{figures/figure4_maxwell_demon_ensemble.png}
    \caption{\textbf{Ensemble transporter collective behavior as single Maxwell demon in categorical space.}
    \textbf{(A)} Maxwell demon cycle trajectory through S-space showing three-state sequence: \textit{open\_inside} (yellow circle, step 1) $\rightarrow$ \textit{open\_outside} (teal circle, step 2) $\rightarrow$ \textit{resetting} (yellow circle, step 3). For Verapamil substrate: cycle time 0.34 ms, phase-lock 1.00, transported: YES. The trajectory demonstrates information processing through conformational state transitions in categorical coordinates.
    \textbf{(B)} Ensemble transport statistics for 5 substrates showing Doxorubicin: 72\% transported (3611/5000, red cap), Verapamil: 100\% (5000/5000, green), Glucose: 100\% (5000/5000, green), Rhodamine\_123: 98\% (4900/5000, green), Metformin: 100\% (5000/5000, green). Total: 23,611/25,000 molecules transported (94.4\% overall efficiency). The ensemble's large capacity enables near-complete transport of all substrates except the weakest (Doxorubicin).
    \textbf{(C)} Ensemble phase-lock distribution showing three substrate clusters: Verapamil, Glucose, Rhodamine\_123 (green circles at $\Phi = 1.0$, top), Metformin (green circle at $\Phi = 0.7$, middle), and Doxorubicin (red circle at $\Phi = 0.34$, bottom). The blue shaded region represents the ensemble phase-lock distribution, with width indicating statistical variation across 5000 transporters. High phase-lock substrates cluster near unity, while weak substrates remain separated.
    \textbf{(D)} Transporter state distribution for $N = 5000$ ensemble: 85.0\% available (4250 transporters, gray), 15.0\% active (750 transporters, red). The large available fraction ensures continuous substrate processing without saturation, enabling throughput far exceeding individual transporter rates.
    \textbf{(E)} Ensemble throughput dynamics over 2.0 s showing measured throughput (blue line with shading) and theoretical prediction (red dashed line). Current throughput at $t = 2.0$ s: 16,806 molecules/s (red circle). Mean throughput: 15,000 molecules/s with $\pm 1$ SD band (blue shading, 12,500-17,500 range). The sigmoid growth from 0 to 17,500 molecules/s demonstrates ensemble spin-up dynamics as transporters engage substrates.
    \textbf{(F)} Collective selectivity matrix (24.2) showing normalized values for phase-lock (row 1), transport probability (row 2), and efficiency (row 3) across 5 substrates (columns). Verapamil, Glucose, Rhodamine\_123 achieve perfect scores (1.00, dark green) in all categories. Metformin shows reduced phase-lock (0.68, yellow-green) but perfect efficiency (1.00). Doxorubicin exhibits weak phase-lock (0.34, orange), low transport probability (0.08, red), but high efficiency (0.72, light green), demonstrating the ensemble's ability to discriminate weak substrates while maintaining high throughput for strong substrates.}
    \label{fig:maxwell_demon_ensemble}
\end{figure}

\subsection{Enhanced Phase-Locking}

Ensemble exhibits enhanced phase-lock strength through statistical coverage:
\begin{equation}
\Phi_{\text{ens}} = \Phi_{\text{ind}} \times \left(1 + \frac{\alpha}{2}\ln\frac{N}{100}\right) \times (1 + P_{\text{avail}})
\end{equation}
where $\Phi_{\text{ind}}$ is individual phase-lock, $N$ is ensemble size, $\alpha = 0.5$ is enhancement coefficient, and $P_{\text{avail}}$ is fraction of available transporters.

For $N = 5000$, $P_{\text{avail}} = 0.85$:
\begin{equation}
\Phi_{\text{ens}} = \Phi_{\text{ind}} \times (1 + 0.98) \times 1.85 = 3.67 \Phi_{\text{ind}}
\end{equation}

Enhancement arises from:
\textbf{(1)} Distributed ATP cycles - ensemble continuously scans frequency space
\textbf{(2)} Statistical averaging - rare favorable configurations amplified
\textbf{(3)} Cooperative effects - membrane domains may synchronize ATP cycles

\subsection{Ensemble Transport Rate}

Individual transporter rate: $r_{\text{ind}} = f_{\text{ATP}} \times \Phi \approx \SI{10}{\hertz} \times \Phi$

Ensemble rate:
\begin{equation}
r_{\text{ens}} = N \times P_{\text{avail}} \times r_{\text{ind}} = 5000 \times 0.85 \times 10\Phi = 42500\Phi \text{ molecules/s}
\end{equation}

For strong substrate ($\Phi = 1$): $r_{\text{ens}} = \SI{42500}{molecules\per\second}$

This is \num{100}-fold above naive scaling ($N \times r_{\text{ind}} = 5000 \times 10 = \SI{50000}{}$) due to ensemble enhancement effects.

\subsection{Single Substrate Validation}

Testing Verapamil transport by \num{5000}-transporter ensemble:

\textbf{Input:} \num{10000} Verapamil molecules, duration \SI{1.0}{\second}

\textbf{Results:}
\begin{align}
\text{Collective phase-lock:} & \quad \Phi_{\text{ens}} = 1.000 \\
\text{Transport rate:} & \quad r = \SI{42500}{molecules\per\second} \\
\text{Molecules transported:} & \quad 10000/10000 = \SI{100}{\percent} \\
\text{Efficiency:} & \quad \eta = 1.00
\end{align}

All available Verapamil molecules transported within \SI{1}{\second}, demonstrating massive parallel capacity of ensemble demon.

\subsection{Multi-Substrate Competition}

Testing simultaneous competition among 5 substrates:

\textbf{Input:} \SI{5000}{molecules} each of Doxorubicin, Verapamil, Glucose, Rhodamine 123, Metformin (total \num{25000})

\textbf{Phase-lock strengths:}
\begin{table}[h]
\centering
\small
\begin{tabular}{lcc}
\hline
\textbf{Substrate} & \textbf{$\Phi_{\text{ens}}$} & \textbf{Transport Prob.} \\
\hline
Doxorubicin & 0.342 & 0.085 \\
Verapamil & 1.000 & 0.248 \\
Glucose & 1.000 & 0.248 \\
Rhodamine 123 & 1.000 & 0.248 \\
Metformin & 0.684 & 0.170 \\
\hline
\end{tabular}
\caption{Ensemble phase-lock and transport probabilities for competing substrates.}
\label{tab:competition}
\end{table}

\textbf{Transport results:}
\begin{align}
\text{Doxorubicin:} & \quad 3611/5000 \text{ transported} (\SI{72.2}{\percent}) \\
\text{Verapamil:} & \quad 5000/5000 \text{ transported} (\SI{100}{\percent}) \\
\text{Glucose:} & \quad 5000/5000 \text{ transported} (\SI{100}{\percent}) \\
\text{Rhodamine 123:} & \quad 5000/5000 \text{ transported} (\SI{100}{\percent}) \\
\text{Metformin:} & \quad 5000/5000 \text{ transported} (\SI{100}{\percent})
\end{align}

\textbf{Total:} \num{23611}/\num{25000} transported (\SI{94.4}{\percent})

\textbf{Collective selectivity:}
\begin{equation}
S_{\text{coll}} = \frac{\Phi_{\max}}{\min(\Phi > 0)} = \frac{1.000}{0.342} = 2.92 \text{ (phase-lock)}
\end{equation}

But selectivity manifests in efficiency:
\begin{equation}
S_{\text{eff}} = \frac{\eta_{\max}}{\eta_{\min}} = \frac{1.00}{0.722} = 1.39
\end{equation}

The ensemble's large capacity (\SI{42500}{molecules\per\second}) exceeds substrate availability, so all except weakest substrate (Doxorubicin) are fully transported. Doxorubicin's \SI{72}{\percent} efficiency reveals discrimination against weak phase-lock.

\subsection{Emergent Collective Properties}

\textbf{E1: Enhanced throughput} - Ensemble achieves \SI{42500}{molecules\per\second}, \num{100}× individual transporter rate (\SI{10}{\hertz}). Enhancement arises from avoiding saturation: when one transporter binds substrate, \num{4999} others remain available.

\textbf{E2: Continuous frequency coverage} - Individual transporter scans \SI{3.3e13}{}-\SI{4.3e13}{\hertz} over \SI{0.1}{\second} ATP cycle. Ensemble with distributed cycles covers this range continuously, increasing substrate detection probability.

\textbf{E3: Statistical sharpening} - Ensemble averaging reduces noise, sharpening phase-lock discrimination. Weak substrates ($\Phi < 0.5$) preferentially rejected.

\textbf{E4: Saturation resistance} - Large ensemble handles high substrate loads without saturation. Individual transporter saturates at \SI{10}{molecules\per\second}; ensemble maintains linearity to \SI{42500}{}.

\begin{figure}[htbp]
    \centering
    \includegraphics[width=\textwidth]{figures/figure8_ensemble_demon_collective.png}
    \caption{\textbf{Ensemble demon collective behavior: emergent properties from 5000-transporter coordination in categorical space.}
    \textbf{(A)} Transporter state distribution for $N = 5000$ ensemble: 85.0\% available (4250 transporters, gray sector), 15.0\% active (750 transporters, red sector). The large available fraction prevents saturation, enabling throughput 100-fold above individual transporter rates.
    \textbf{(B)} Single vs multi-substrate efficiency comparison: Single substrate (Verapamil alone, green bar, 100.0\% efficiency) vs Multi-substrate (5 competing substrates, blue bar, 94.4\% efficiency). The 90\% threshold (red dashed line) is exceeded in both cases, demonstrating that ensemble maintains high efficiency even under competition. The 5.6\% reduction reflects discrimination against weak substrates (Doxorubicin).
    \textbf{(C)} Multi-substrate competition showing transported (green) vs rejected (red) molecules for 5 substrates: Doxorubicin (3611 transported, 1389 rejected, 72\%), Verapamil (5000 transported, 0 rejected, 100\%), Glucose (5000 transported, 0 rejected, 100\%), Rhodamine\_123 (5000 transported, 0 rejected, 100\%), Metformin (5000 transported, 0 rejected, 100\%). Total: 23,611/25,000 transported (94.4\%). The selective rejection of weak Doxorubicin demonstrates ensemble discrimination despite massive throughput capacity.
    \textbf{(D)} Membrane distribution sample showing spatial arrangement of 5000 transporters at density 5.0 transporters/$\mu$m$^2$ in 10 $\mu$m $\times$ 10 $\mu$m area. Active transporters (red circles, 15.0\%) are randomly distributed among available transporters (gray circles, 85.0\%), indicating no spatial clustering or domain formation. The uniform distribution supports the independent-transporter model for ensemble behavior.
    \textbf{(E)} Ensemble throughput dynamics over 2.0 s showing measured throughput (blue line with shading) vs theoretical prediction (red dashed line). Current throughput at $t = 2.0$ s: 16,806 molecules/s (red circle). Mean throughput: 15,000 molecules/s with $\pm 1$ SD band (blue shading, 12,500-17,500 range). The sigmoid growth from 0 to 17,500 molecules/s demonstrates ensemble spin-up: initially few substrates engage transporters, then throughput saturates as substrate availability becomes limiting.}
    \label{fig:ensemble_demon_collective}
\end{figure}

\subsection{Scaling Laws}

Throughput scales linearly with ensemble size:
\begin{equation}
r_{\text{ens}}(N) = N \times P_{\text{avail}} \times r_{\text{ind}} \times (1 + \beta\ln N)
\end{equation}
where $\beta \approx 0.02$ accounts for logarithmic enhancement.

For typical cellular expression levels:
\begin{align}
N = 1000: & \quad r \approx \SI{9000}{molecules\per\second} \\
N = 5000: & \quad r \approx \SI{42500}{molecules\per\second} \\
N = 10000: & \quad r \approx \SI{85000}{molecules\per\second}
\end{align}

Selectivity decreases with ensemble size due to statistical enhancement of weak substrates:
\begin{equation}
S_{\text{eff}}(N) \approx S_0 - \gamma\ln N
\end{equation}
where $\gamma \approx 0.1$. Large ensembles trade selectivity for throughput.

\subsection{Membrane Domain Effects}

If transporters cluster in membrane domains (lipid rafts), ATP cycles may synchronize, creating collective frequency sweeps. This would manifest as:

\textbf{(1)} Oscillatory transport rates at $f_{\text{ATP}}$

\textbf{(2)} Enhanced selectivity for substrates matching synchronized frequency

\textbf{(3)} Domain-specific substrate preferences

Testing this requires spatially-resolved transport measurements, currently beyond experimental capabilities but predicted by the ensemble demon framework.

\subsection{Comparison: Individual vs Ensemble}

\begin{table}[h]
\centering
\small
\begin{tabular}{lcc}
\hline
\textbf{Property} & \textbf{Individual} & \textbf{Ensemble} \\
\hline
Transport rate & \SI{10}{\hertz} & \SI{42500}{\hertz} \\
Selectivity & \num{9e9} & \num{1e10} \\
Frequency coverage & Sequential & Continuous \\
Substrate capacity & \SI{10}{molecules\per\second} & \SI{42500}{} \\
Saturation & Yes (at \SI{10}{}) & No (to \SI{42500}{}) \\
Phase-lock enhancement & 1× & 3.67× \\
\hline
\end{tabular}
\caption{Individual transporter vs \num{5000}-member ensemble demon properties.}
\label{tab:individual-vs-ensemble}
\end{table}

The ensemble demon exhibits qualitatively different behavior from scaled-up individual transporters, confirming emergence of collective properties in S-space.
