\subsection{Physical vs Categorical Observables}

Physical observables (position $\mathbf{x}$, momentum $\mathbf{p}$) describe location and motion in 3D space. Categorical observables describe information content, independent of physical coordinates. We introduce S-entropy coordinates:

\begin{definition}[S-Entropy Coordinates]
For a molecular system with internal degrees of freedom, the S-entropy coordinates are:
\begin{align}
S_k &= -\sum_i p_i^{(k)} \ln p_i^{(k)} && \text{(knowledge)} \\
S_t &= -\sum_i p_i^{(t)} \ln p_i^{(t)} && \text{(temporal)} \\
S_e &= -\sum_i p_i^{(e)} \ln p_i^{(e)} && \text{(evolution)}
\end{align}
where $p_i^{(\alpha)}$ are probability distributions over discrete internal states.
\end{definition}

\textbf{$S_k$ (knowledge):} Entropy quantifying "which state is the system in?" For transporter: which conformation? which substrate bound?

\textbf{$S_t$ (temporal):} Entropy of "when do transitions occur?" Encoded in phase of oscillatory dynamics.

\textbf{$S_e$ (evolution):} Entropy of "how will system evolve?" Determined by amplitudes and couplings.

\begin{figure}[htbp]
    \centering
    \includegraphics[width=\textwidth]{figures/figure1_conformational_landscape.png}
    \caption{\textbf{ABC transporter conformational landscape mapped to S-entropy coordinate space.}
    \textbf{(A)} Free energy landscape showing four conformational states: \textit{occluded} (minimum at 3000 Å$^3$, 45 THz, $\Delta G = +15$ kJ/mol), \textit{open\_outside} (5000 Å$^3$, 38 THz, $\Delta G = 0$ kJ/mol), \textit{resetting} (4000 Å$^3$, 35 THz, $\Delta G = +5$ kJ/mol), and \textit{open\_inside} (4500 Å$^3$, 32 THz, $\Delta G = -10$ kJ/mol). Energy barriers reach +15 kJ/mol at the occluded state, representing the transition state for ATP hydrolysis.
    \textbf{(B)} S-space trajectory through categorical coordinates $(S_k, S_t, S_e)$ over one complete ATP cycle. The trajectory connects all four states with total S-space distance $D_S = 14.73$, demonstrating that conformational changes correspond to well-defined paths in information space.
    \textbf{(C)} State properties normalized in polar coordinates showing frequency (norm), volume (norm), distance (norm), and energy (norm) for each conformational state. The \textit{open\_outside} state (purple) shows highest frequency and volume, while \textit{occluded} (teal) shows compressed volume and elevated energy.
    \textbf{(D)} ATP binding distribution: 50\% ATP-bound (red) during \textit{open\_outside} and \textit{occluded} states, 50\% ATP-free (teal) during \textit{open\_inside} and \textit{resetting} states, confirming the ATP hydrolysis cycle drives conformational transitions.
    \textbf{(E)} Frequency modulation range spanning 32-44 THz with center frequency 38.5 THz and modulation bandwidth $\pm$6.5 THz. The four states (blue circles) span this range, enabling substrate discrimination through frequency matching. Red dashed line indicates center frequency; orange dashed lines mark modulation limits.
    \textbf{(F)} S-space distance matrix showing categorical distances between all state pairs. Diagonal elements are zero (self-distance); off-diagonal elements range from 0.58 (\textit{resetting}$\leftrightarrow$\textit{open\_inside}) to 1.03 (\textit{occluded}$\leftrightarrow$\textit{resetting}), confirming all states are distinguishable in S-entropy space with minimum separation $d_S^{\text{min}} = 0.58 > 0.1$ threshold.}
    \label{fig:conformational_landscape}
\end{figure}

\subsection{Dual Coordinate Systems}

\begin{theorem}[Physical-Categorical Orthogonality]
Physical coordinates $(\mathbf{x}, \mathbf{p})$ and S-entropy coordinates $(S_k, S_t, S_e)$ are orthogonal:
\begin{equation}
[\hat{O}_{\text{phys}}, \hat{O}_{\text{cat}}] = 0
\end{equation}
for any physical observable $\hat{O}_{\text{phys}}$ and categorical observable $\hat{O}_{\text{cat}}$.
\end{theorem}

\begin{proof}
Physical observables are operators on wavefunctions: $\hat{O}_{\text{phys}}|\psi\rangle$. Categorical observables are functionals of probability distributions: $\hat{O}_{\text{cat}}[|\psi|^2]$. Since $\hat{O}_{\text{cat}}$ depends only on $|\psi|^2$ (a scalar), not on the phase of $\psi$:
\begin{equation}
\hat{O}_{\text{phys}}\hat{O}_{\text{cat}}[|\psi|^2] = \hat{O}_{\text{cat}}[\hat{O}_{\text{phys}}|\psi|^2] = \hat{O}_{\text{cat}}\hat{O}_{\text{phys}}[|\psi|^2]
\end{equation}
Therefore $[\hat{O}_{\text{phys}}, \hat{O}_{\text{cat}}] = 0$.
\end{proof}

\textbf{Consequence:} Measuring S-coordinates does not disturb physical coordinates. This circumvents the Heisenberg uncertainty principle $\Delta x \Delta p \geq \hbar/2$, which constrains only physical observables.

\subsection{Transporter Conformational States in S-Space}

We map the four ABC transporter conformational states to S-entropy coordinates:

\textbf{OPEN\_OUTSIDE:} Ready to bind substrate. High uncertainty about which substrate will bind ($S_k = 0.10$, low knowledge). Beginning of cycle ($S_t = 0.00$). High evolution potential ($S_e = 1.00$).

\textbf{OCCLUDED:} Substrate trapped. High knowledge of substrate identity ($S_k = 0.90$). Quarter through cycle ($S_t = 0.25$). Mid-evolution as ATP hydrolyzes ($S_e = 0.50$).

\textbf{OPEN\_INSIDE:} Substrate released. Low knowledge, substrate gone ($S_k = 0.20$). Halfway through cycle ($S_t = 0.50$). Low evolution, stable state ($S_e = 0.30$).

\textbf{RESETTING:} Returning to initial. Very low knowledge ($S_k = 0.05$). Three-quarters through cycle ($S_t = 0.75$). High evolution during active transition ($S_e = 0.80$).

\subsection{S-Space Distance and Trajectory}

The categorical distance between states $i$ and $j$ is:
\begin{equation}
d_S(i,j) = \sqrt{(S_k^{(i)} - S_k^{(j)})^2 + (S_t^{(i)} - S_t^{(j)})^2 + (S_e^{(i)} - S_e^{(j)})^2}
\end{equation}

\textbf{Validation:} Minimum inter-state distance $d_S^{\min} = 0.58$, confirming states are distinguishable in S-space. Over 5 ATP cycles, transporter traverses S-space distance:
\begin{equation}
D_{S,\text{total}} = \sum_{i=0}^{19} d_S(i, i+1) = 14.73
\end{equation}

\subsection{Vibrational Frequencies in S-Space}

Each conformational state has characteristic vibrational frequency of binding site cavity:

\begin{table}[h]
\centering
\small
\begin{tabular}{lcc}
\hline
\textbf{State} & \textbf{Frequency (Hz)} & \textbf{Volume (Å$^3$)} \\
\hline
OPEN\_OUTSIDE & \num{3.8e13} & 5000 \\
OCCLUDED & \num{4.5e13} & 3000 \\
OPEN\_INSIDE & \num{3.2e13} & 4500 \\
RESETTING & \num{3.5e13} & 4000 \\
\hline
\end{tabular}
\caption{Binding site frequencies and cavity volumes for each conformational state.}
\label{tab:frequencies}
\end{table}

Frequency modulation range: $\Delta\omega = \SI{1.3e13}{\hertz}$ (from \SI{3.2e13}{} to \SI{4.5e13}{\hertz}). This exceeds typical molecular vibration linewidths ($\sim$\SI{e11}{\hertz}), enabling discrimination of substrates with different vibrational frequencies.

\subsection{Categorical Addressing}

\begin{definition}[Categorical Addressing Operator]
The operator $\Lambda_{\mathbf{S}_*}$ selects all molecules within categorical distance $\epsilon$ of target S-coordinate $\mathbf{S}_*$:
\begin{equation}
\Lambda_{\mathbf{S}_*}[\mathcal{M}] = \{\mathscr{I} \in \mathcal{M} : d_S(\mathscr{I}, \mathbf{S}_*) < \epsilon\}
\end{equation}
\end{definition}

Categorical addressing enables interaction with molecules based on information content (S-coordinates) rather than physical location. This is the mechanism for substrate "measurement" without physical disturbance - detection occurs in S-space, orthogonal to physical space.
