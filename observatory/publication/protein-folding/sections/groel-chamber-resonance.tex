
\subsection{GroEL Structure and Dynamics}

The GroEL chaperonin consists of two stacked heptameric rings, each forming a cylindrical cavity. We focus on the cis cavity (containing the substrate protein) which undergoes dramatic conformational changes during the ATP cycle.

\begin{definition}[GroEL Cavity Geometry]
The cis cavity is characterized by:
\begin{itemize}
\item Radius: $R_{\text{cavity}} = 4.5 \pm 0.5$ nm (ATP-dependent)
\item Height: $H_{\text{cavity}} = 8.5 \pm 1.0$ nm (ATP-dependent)
\item Volume: $V_{\text{cavity}} = \pi R_{\text{cavity}}^2 H_{\text{cavity}} \approx 540 \pm 100$ nm$^3$
\item Wall thickness: $\sim$2 nm (14 subunits, each 57 kDa)
\end{itemize}
\end{definition}

The cavity wall is not rigid but exhibits collective vibrational modes arising from the coupled motion of the seven subunits.

\subsection{Cavity Vibrational Modes}

The cavity can be modeled as an elastic shell with vibrational modes. For a cylindrical cavity of radius $R$ and height $H$, the normal modes are:

\begin{equation}
\psi_{n,m,\ell}(r,\theta,z) = J_n(k_{nm}r)e^{in\theta}\sin\left(\frac{\ell\pi z}{H}\right)
\end{equation}

where:
\begin{itemize}
\item $n$ is the azimuthal mode number (0, 1, 2, ...)
\item $m$ is the radial mode number (1, 2, 3, ...)
\item $\ell$ is the axial mode number (1, 2, 3, ...)
\item $J_n$ is the Bessel function of order $n$
\item $k_{nm}$ is the $m$-th zero of $J_n$
\end{itemize}

The corresponding frequencies are:

\begin{equation}
\omega_{n,m,\ell} = c_{\text{eff}}\sqrt{k_{nm}^2 + \left(\frac{\ell\pi}{H}\right)^2}
\end{equation}

where $c_{\text{eff}}$ is the effective sound velocity in the protein-water composite forming the cavity wall.

For protein material: $c_{\text{eff}} \approx 2000$ m/s.

The fundamental mode ($n=0, m=1, \ell=1$) has $k_{01} = 2.405/R$ and:

\begin{equation}
\omega_{0,1,1} = c_{\text{eff}}\sqrt{\frac{(2.405)^2}{R^2} + \frac{\pi^2}{H^2}}
\end{equation}

For $R = 4.5$ nm and $H = 8.5$ nm:

\begin{equation}
\omega_{0,1,1} = 2000\sqrt{\frac{5.78}{(4.5 \times 10^{-9})^2} + \frac{9.87}{(8.5 \times 10^{-9})^2}} \approx 6.7 \times 10^{13} \text{ rad/s}
\end{equation}

or $f_{0,1,1} \approx 1.1 \times 10^{13}$ Hz.

Critically, this is approximately equal to $\omega_{\text{O}_2} = 10^{13}$ Hz, confirming that the GroEL cavity is naturally resonant with the O$_2$ master clock.

\begin{figure*}[htbp]
    \centering
    \includegraphics[width=\textwidth]{figures/FIGURE_2_CYCLE_DYNAMICS.png}
    \caption{\textbf{Cycle-by-cycle folding dynamics showing ATP-driven resonance tuning in GroEL cavity.}
    \textbf{(A)} Stability evolution across 11 ATP cycles. Final stability (purple line with circles) oscillates between 0.45-0.85, with gold stars marking best cycles (cycles 1, 2, 7, 11). Mean stability (green dashed line with squares) remains relatively constant at $\sim$0.60, indicating consistent phase-locking quality. Cycle 11 achieves final stability 0.841, exceeding success threshold (not shown). Best cycles correspond to optimal GroEL cavity frequency matching with hydrogen bond natural frequencies.
    \textbf{(B)} Phase coherence evolution showing inverse relationship with stability. Final variance (red line with circles) decreases from 0.122 (cycle 1) to 0.035 (cycle 11), representing 71.2\% reduction. Minimum variance (orange dashed line with squares) remains low at 0.01-0.04 across cycles. Lower variance indicates better phase coherence: when hydrogen bonds oscillate in phase, protein structure stabilizes. Variance peaks at cycles 1, 8 correspond to stability troughs, confirming anticorrelation.
    \textbf{(C)} GroEL cavity frequency modulation showing ATP-driven resonance tuning. Cavity frequency (teal line with circles) increases from 5 THz (cycle 1) to 50 THz (cycle 9), then drops to 5 THz (cycle 10). Red dashed line marks O$_2$ master clock at 10 THz. Frequency crosses O$_2$ harmonics at 0.5$\times$, 1.0$\times$, 2.0$\times$, 3.0$\times$, 5.0$\times$ (gray dashed lines with labels). This systematic frequency scanning enables GroEL to sequentially phase-lock different hydrogen bond subsets with distinct natural frequencies.
    \textbf{(D)} Phase space trajectory showing convergence from initial state (red circle, variance 0.122, stability 0.501) to folded state (red circle, variance 0.035, stability 0.841). Intermediate cycles (colored circles, gradient from purple to yellow) trace path through phase space. Trajectory shows non-monotonic convergence: stability can decrease temporarily (cycles 3-4, 6-7) while system explores configuration space. Final convergence is rapid (cycle 10→11), indicating cooperative phase-locking of remaining bonds. Gray lines connect consecutive cycles.
    \textbf{Bottom Panel - Cycle-by-Cycle Statistics:}
    \textit{Overall:} 11 total cycles, folded at cycle 11, final stability 0.841, final variance 0.035.
    \textit{Best cycles:} Cycle 1 (stability 0.501, variance 0.122), Cycle 2 (0.631, 0.103), Cycle 7 (0.641, 0.071), Cycle 11 (0.841, 0.035).
    \textit{Frequency modulation:} Min 5.0 THz, max 50.0 THz, range 45.0 THz.
    \textit{Convergence:} Initial stability 0.501 → final 0.841 (67.8\% improvement). Initial variance 0.122 → final 0.035 (71.2\% reduction).
    \textit{Phase-lock quality:} 11 cycles to convergence, 100.0\% convergence rate, success YES.
    This quantitative summary confirms successful folding through systematic ATP-driven frequency scanning and progressive phase-locking.}
    \label{fig:cycle_dynamics}
\end{figure*}

\subsection{ATP-Driven Cavity Modulation}

ATP binding, hydrolysis, and product release drive conformational changes in the GroEL subunits, modulating the cavity geometry and hence its vibrational frequencies.

\begin{definition}[ATP Cycle Phases]
The ATP cycle consists of four phases parameterized by phase angle $\phi \in [0, 2\pi]$:
\begin{enumerate}
\item ATP Binding: $0 < \phi < \pi/2$
\item Transition State (ATP $\to$ ADP + Pi): $\pi/2 < \phi < \pi$
\item ADP + Pi State: $\pi < \phi < 3\pi/2$
\item ADP Release: $3\pi/2 < \phi < 2\pi$
\end{enumerate}
\end{definition}

Structural studies show the cavity radius varies as:

\begin{equation}
R(\phi) = R_0\left[1 + A_R\cos(\phi - \phi_R)\right]
\end{equation}

with $A_R \approx 0.15$ (15\% modulation) and $\phi_R \approx \pi$ (maximum contraction near transition state).

Similarly, the cavity height varies:

\begin{equation}
H(\phi) = H_0\left[1 + A_H\cos(\phi - \phi_H)\right]
\end{equation}

with $A_H \approx 0.10$ (10\% modulation) and $\phi_H \approx 0$ (maximum expansion at ATP binding).

The cavity volume is:

\begin{equation}
V(\phi) = \pi R(\phi)^2 H(\phi) \approx V_0\left[1 + 2A_R\cos(\phi - \phi_R) + A_H\cos(\phi - \phi_H)\right]
\end{equation}

where $V_0 = \pi R_0^2 H_0 \approx 540$ nm$^3$.

\subsection{Frequency Modulation}

The cavity vibrational frequencies depend on geometry:

\begin{equation}
\omega_{n,m,\ell}(\phi) = c_{\text{eff}}\sqrt{\frac{k_{nm}^2}{R(\phi)^2} + \frac{\ell^2\pi^2}{H(\phi)^2}}
\end{equation}

For the fundamental mode:

\begin{equation}
\omega_{0,1,1}(\phi) = \omega_{0,1,1}^{(0)}\sqrt{\frac{1}{[1 + A_R\cos(\phi - \phi_R)]^2} + \frac{1}{[1 + A_H\cos(\phi - \phi_H)]^2}}
\end{equation}

This gives a frequency modulation of approximately:

\begin{equation}
\frac{\Delta\omega}{\omega_0} \approx 2A_R + A_H \approx 0.4 \quad (40\%)
\end{equation}

Therefore, the cavity fundamental frequency varies over range:

\begin{equation}
\omega_{\text{cavity}} \in [0.8\omega_0, 1.4\omega_0] \approx [8 \times 10^{12}, 1.5 \times 10^{13}] \text{ Hz}
\end{equation}

\subsection{Harmonic Frequency Scanning}

The crucial insight is that GroEL performs \textit{harmonic frequency scanning}. The cavity does not just oscillate at one frequency but at multiple harmonics simultaneously.

Define the cavity frequency spectrum:

\begin{equation}
\Omega_{\text{cavity}}(\phi) = \left\{h \cdot \omega_{\text{base}}(\phi) : h \in \mathcal{H}\right\}
\end{equation}

where $\omega_{\text{base}}(\phi) = \omega_{0,1,1}(\phi)$ is the fundamental frequency and $\mathcal{H} = \{1, 2, 3, 5, 7, 11, 13, ...\}$ is the set of harmonic numbers.

During an ATP cycle, the fundamental frequency sweeps from $0.8\omega_0$ to $1.4\omega_0$. Each harmonic $h$ sweeps over range:

\begin{equation}
h\omega_{\text{cavity}} \in [0.8h\omega_0, 1.4h\omega_0]
\end{equation}

For proton oscillators with frequencies $\omega_{\text{H}^+} \approx 4\times 10^{13}$ Hz, the relevant harmonic is $h \approx 4$:

\begin{equation}
4\omega_{\text{cavity}} \in [3.2\omega_0, 5.6\omega_0] \approx [3.2 \times 10^{13}, 6.0 \times 10^{13}] \text{ Hz}
\end{equation}

This range encompasses typical hydrogen bond frequencies, enabling phase-locking.

\begin{figure*}[htbp]
    \centering
    \includegraphics[width=\textwidth]{figures/FIGURE_5_3D_PHASE_SPACE.png}
    \caption{\textbf{3D phase space analysis reveals stability-variance-coherence trajectory during folding.}
    \textbf{(A)} 3D phase space trajectory showing folding progression in stability-variance-coherence coordinates. Red circle marks start (cycle 1): low stability ($\sim$0.5), high variance ($\sim$0.11), low coherence ($\sim$0.2). Black star marks folded state (cycle 11): high stability ($\sim$0.85), low variance ($\sim$0.04), high coherence ($\sim$0.7). Yellow diamond marks critical transition (cycle 5). Colored spheres show intermediate cycles (purple → yellow gradient). Gray lines connect consecutive cycles. The trajectory demonstrates convergence: system moves from disordered initial state (high variance, low coherence) to ordered final state (low variance, high coherence) along increasing stability axis. This 3D visualization reveals that folding is a directed process in phase space, not random exploration.
    \textbf{(B)} S-V projection (stability vs variance) showing anticorrelation. Trajectory moves from bottom-right (low stability 0.5, high variance 0.11) to top-left (high stability 0.85, low variance 0.04). Gray lines connect cycles. This 2D projection shows that increased stability always accompanies decreased variance, confirming that phase-locking (low variance) produces structural stability.
    \textbf{(C)} S-P projection (stability vs phase coherence) showing positive correlation. Trajectory moves from bottom-left (low stability 0.5, low coherence 0.0) to top-right (high stability 0.85, high coherence 0.7). This demonstrates that phase coherence (synchronization of hydrogen bond oscillators) directly produces stability.
    \textbf{(D)} Phase space velocity showing rate of change in S-V-P space. Red line with shaded region shows velocity oscillating between 0.0-0.6 across cycles. Star marks highest velocity (0.6 at cycle 11, final convergence). Velocity peaks during critical transitions (cycles 2, 5, 11) when multiple bonds form simultaneously. Low velocity during intermediate cycles (4, 7, 9) indicates plateau phases where system consolidates previous gains.
    \textbf{(E)} Phase space distance from origin showing cumulative progress. Purple line decreases from 1.0 (cycle 1) to 0.8 (cycle 5), then increases sharply to 1.4 (cycle 11). The initial decrease represents movement toward intermediate attractor; final increase represents escape to folded state. Distance oscillations (cycles 6-10) show system exploring local minima before final convergence.}
    \label{fig:3d_phase_space}
\end{figure*}

\subsection{Multi-Cycle Frequency Coverage}

A single ATP cycle scans a limited frequency range. Multiple cycles with different harmonic emphasis provide comprehensive coverage.

\begin{definition}[Cycle Harmonic Sequence]
Define the dominant harmonic for cycle $c$ as:
\begin{equation}
h_c = h_1 + (c-1) \mod M
\end{equation}
where $h_1$ is the initial harmonic and $M$ is the harmonic spacing.
\end{definition}

For example, with $h_1 = 1$ and $M = 3$, the sequence is:
\begin{align}
\text{Cycle 1:} &\quad h = 1, 4, 7, 10, ... \\
\text{Cycle 2:} &\quad h = 2, 5, 8, 11, ... \\
\text{Cycle 3:} &\quad h = 3, 6, 9, 12, ... \\
\text{Cycle 4:} &\quad h = 1, 4, 7, 10, ... \quad \text{(repeats)}
\end{align}

Each cycle emphasizes different harmonics, scanning different frequency regions.

The total frequency coverage after $N_{\text{cycles}}$ cycles is:

\begin{equation}
\bigcup_{c=1}^{N_{\text{cycles}}} \Omega_{\text{cavity}}^{(c)}
\end{equation}

where $\Omega_{\text{cavity}}^{(c)}$ is the frequency spectrum in cycle $c$.

\subsection{Phase-Locking Windows}

For a hydrogen bond with frequency $\omega_j$, phase-locking to the cavity occurs when:

\begin{equation}
|h\omega_{\text{cavity}}(\phi) - \omega_j| < K_{\text{GroEL},j}
\end{equation}

for some harmonic $h$ and some phase $\phi$ during the cycle.

Define the phase-locking window:

\begin{equation}
\mathcal{W}_j^{(c)} = \left\{\phi : |h\omega_{\text{cavity}}(\phi) - \omega_j| < K_{\text{GroEL},j}, \, h = h_c\right\}
\end{equation}

The fraction of the cycle where bond $j$ is phase-locked is:

\begin{equation}
f_j^{(c)} = \frac{|\mathcal{W}_j^{(c)}|}{2\pi}
\end{equation}

Bonds with $f_j^{(c)} > f_{\text{crit}} \approx 0.3$ (locked for $>30\%$ of cycle) are considered phase-locked in that cycle.

\subsection{Cycle-by-Cycle Bond Formation}

As the protein evolves through ATP cycles, hydrogen bonds progressively phase-lock to the cavity and stabilize.

\begin{definition}[Formation Cycle]
The formation cycle $C_j$ for bond $j$ is the first cycle where:
\begin{enumerate}
\item Phase-lock strength $\Lambda_j^{(c)} > \Lambda_{\text{crit}} \approx 0.7$
\item Phase coherence with neighbors $\langle r_{\text{local}} \rangle > 0.7$
\item Stability persists through subsequent cycles
\end{enumerate}
\end{definition}

Empirically, bonds form in a hierarchical sequence:

\begin{equation}
C_{\text{core}} < C_{\text{secondary}} < C_{\text{tertiary}}
\end{equation}

where:
\begin{itemize}
\item Core bonds (beta-sheets, alpha-helices) form in cycles 1-3
\item Secondary contacts (loop stabilization) form in cycles 4-7
\item Tertiary contacts (domain interfaces) form in cycles 8-12
\end{itemize}

\subsection{ATP Cycle Timing and O$_2$ Synchronization}

The ATP cycle period is $T_{\text{ATP}} \approx 1$ second, giving $\omega_{\text{ATP}} = 2\pi/T_{\text{ATP}} \approx 6.28$ rad/s or $f_{\text{ATP}} \approx 1$ Hz.

This appears vastly slower than the O$_2$ master clock at $10^{13}$ Hz. However, they are harmonically related:

\begin{equation}
\omega_{\text{ATP}} = n_{\text{ATP}} \omega_{\text{O}_2}
\end{equation}

where $n_{\text{ATP}} = \omega_{\text{ATP}}/\omega_{\text{O}_2} \approx 6 \times 10^{-13}$.

This is not a direct harmonic ($n_{\text{ATP}}$ is not an integer) but rather:

\begin{equation}
\omega_{\text{ATP}} = \frac{n_1}{n_2}\omega_{\text{O}_2}
\end{equation}

where $n_1$ and $n_2$ are coprime integers with $n_1/n_2 \approx 6 \times 10^{-13}$.

In practice, $n_1 = 1$ and $n_2 \approx 1.6 \times 10^{12}$, meaning the ATP cycle is synchronized to approximately the $10^{12}$-th subharmonic of O$_2$.

This deep subharmonic relationship ensures that ATP cycle phase is locked to the O$_2$ master clock, making GroEL's operation coherent with cellular oscillatory dynamics.

\subsection{Resonance Quality Factor}

The quality factor of the cavity resonance is:

\begin{equation}
Q = \frac{\omega_0}{\Delta\omega}
\end{equation}

where $\Delta\omega$ is the resonance linewidth.

For the GroEL cavity, damping arises from:
\begin{itemize}
\item Water viscosity: $\gamma_{\text{water}} \approx 10^{10}$ s$^{-1}$
\item Protein internal friction: $\gamma_{\text{protein}} \approx 10^9$ s$^{-1}$
\end{itemize}

The total damping is $\gamma_{\text{tot}} \approx 10^{10}$ s$^{-1}$, giving:

\begin{equation}
Q = \frac{\omega_0}{\gamma_{\text{tot}}} \approx \frac{10^{13}}{10^{10}} = 10^3
\end{equation}

This high Q factor indicates sharp resonances, allowing precise frequency discrimination.

The resonance linewidth is:

\begin{equation}
\Delta\omega = \frac{\omega_0}{Q} \approx \frac{10^{13}}{10^3} = 10^{10} \text{ rad/s}
\end{equation}

or $\Delta f \approx 1.6 \times 10^9$ Hz (1.6 GHz).

Hydrogen bonds with frequencies differing by less than $\Delta f$ cannot be distinguished by the cavity resonance and will phase-lock together.


\begin{figure*}[htbp]
    \centering
    \includegraphics[width=\textwidth]{figures/folding_dynamics_panel.png}
    \caption{\textbf{Comprehensive protein folding dynamics showing phase-locked GroEL-mediated folding mechanism.}
    \textbf{(A)} Network stability and variance evolution across 30 ATP cycles. Final stability (dark green line with circles) oscillates between 0.45-0.65, with gold stars marking best cycles (cycles 2, 11). Mean stability (green dashed line with squares) remains constant at $\sim$0.60. Success threshold at 0.7 (gray dotted line) is approached but not exceeded, indicating partial folding. Variance (red line, right y-axis) oscillates between 0.04-0.12 with anticorrelated relationship to stability: high variance (0.10-0.12) corresponds to low stability (0.45-0.50), confirming that phase coherence drives structural stability.
    \textbf{(B)} GroEL cavity frequency scanning showing systematic modulation across 30 cycles. Cavity frequency (teal circles, size proportional to cycle number) increases from 5 THz (cycle 1, small) to 50 THz (cycle 9, large), then decreases. Red dashed line marks O$_2$ master clock at 10 THz. Gray dashed lines indicate O$_2$ harmonics: 0.5$\times$ (5 THz), 1.0$\times$ (10 THz), 2.0$\times$ (20 THz), 3.0$\times$ (30 THz), 5.0$\times$ (50 THz). Cavity frequency crosses each harmonic sequentially, enabling resonant coupling to hydrogen bonds with natural frequencies matching these harmonics. This systematic scanning ensures all bond subsets encounter their resonance condition.
    \textbf{(C)} H-bond formation timeline showing 8 bonds (y-axis, labeled with IDs 77233, 133117, 199331, 55235, 8111199, 177333, 99221, 155355) forming across 30 cycles (x-axis). Horizontal bars show formation duration, colored by phase coherence (colorbar 0.0-1.0): dark red = low coherence (0.0-0.2), orange = medium (0.4-0.6), green = high (0.8-1.0). Phase coherence values labeled on bars: bond 155355 (0.83, green), bond 177333 (0.85, green), bond 55235 (0.85, green), bond 199331 (0.64, yellow), bond 133117 (0.34, orange), bond 77233 (0.64, yellow), bond 99221 (0.29, red). Purple dashed line marks folding nucleus at cycle 2. Green dashed lines mark subsequent critical events. Bonds 155355, 177333, 55235 achieve high coherence (0.83-0.85), indicating successful phase-locking. Bond 99221 has low coherence (0.29), suggesting incomplete folding.
    \textbf{(D)} Cumulative bond phase-locking showing stepwise bond formation. Blue line (cumulative bonds) increases from 0 to 8 in discrete steps: 0→3 bonds (cycle 0→2), 3→6 bonds (cycle 2→5), 6→8 bonds (cycle 5→30). Red circles mark formation events. Green dashed line marks total bonds (8). Yellow dashed line marks folding nucleus formation at cycle 2 (3 bonds). Blue shaded region emphasizes cumulative progress. The stepwise progression demonstrates cooperative phase-locking: nucleus forms rapidly (3 bonds in 2 cycles), then remaining bonds add progressively (5 bonds over 28 cycles). This two-phase behavior (fast nucleation + slow completion) is characteristic of phase-locked folding.}
    \label{fig:folding_dynamics_comprehensive}
\end{figure*}

\subsection{Coupling Strength Distribution}

The coupling between GroEL cavity and protein hydrogen bonds varies spatially. For a spherical protein of radius $R_p$ centered in a spherical cavity of radius $R_c$:

\begin{equation}
K_{\text{GroEL}}(r) = K_0\exp\left(-\frac{R_c - r}{d_0}\right)
\end{equation}

where $r$ is the distance from the protein center and $d_0 \approx 1$ nm is the coupling length.

Surface bonds ($r \approx R_p$) experience coupling:

\begin{equation}
K_{\text{surface}} = K_0\exp\left(-\frac{R_c - R_p}{d_0}\right)
\end{equation}

For $R_p = 3$ nm, $R_c = 4.5$ nm, $d_0 = 1$ nm:

\begin{equation}
K_{\text{surface}} = K_0 e^{-1.5} \approx 0.22 K_0
\end{equation}

Core bonds ($r \approx 0$) experience much weaker coupling:

\begin{equation}
K_{\text{core}} = K_0 e^{-4.5} \approx 0.011 K_0
\end{equation}

This spatial gradient means surface bonds phase-lock first, followed by progressively deeper bonds as the protein compacts.

\subsection{Energy Landscape Modification}

The GroEL cavity modifies the protein's energy landscape through the phase-locking potential:

\begin{equation}
V_{\text{GroEL}}[\{\phi_j\}] = -\sum_j K_{\text{GroEL},j}\cos(\phi_j - h\phi_{\text{cavity}})
\end{equation}

This adds to the intrinsic protein potential:

\begin{equation}
V_{\text{protein}}[\{\phi_j\}] = -\sum_{j,k} K_{jk}\cos(\phi_j - \phi_k)
\end{equation}

The total free energy is:

\begin{equation}
F[\{\phi_j\}] = V_{\text{protein}}[\{\phi_j\}] + V_{\text{GroEL}}[\{\phi_j\}] - TS[\{\phi_j\}]
\end{equation}

The GroEL potential creates new local minima corresponding to partially folded states where some bonds are phase-locked to the cavity while others remain disordered.

These metastable intermediates act as stepping stones in the folding pathway, allowing the protein to traverse rugged energy landscape regions that would otherwise be kinetically inaccessible.

\subsection{Theoretical Folding Time Prediction}

The time required for a protein to fold in GroEL depends on:
\begin{enumerate}
\item Number of hydrogen bonds: $N_{\text{bonds}}$
\item Frequency distribution width: $\Delta\omega_{\text{bonds}}$
\item Cavity coupling strength: $\langle K_{\text{GroEL}}\rangle$
\item Harmonic scanning rate: $\dot{h} \approx 1$ per cycle
\end{enumerate}

The number of cycles required is approximately:

\begin{equation}
N_{\text{cycles}} \approx \frac{\Delta\omega_{\text{bonds}}}{\Delta\omega_{\text{cavity}}} \cdot \frac{N_{\text{bonds}}}{N_{\text{parallel}}}
\end{equation}

where $\Delta\omega_{\text{cavity}} \approx 0.4\omega_0$ is the frequency range scanned per cycle and $N_{\text{parallel}}$ is the number of bonds that can phase-lock simultaneously.

For typical proteins:
\begin{itemize}
\item $N_{\text{bonds}} \approx 50-200$
\item $\Delta\omega_{\text{bonds}}/\omega_0 \approx 0.2-0.5$ (20-50\% frequency spread)
\item $N_{\text{parallel}} \approx 10-20$ (limited by spatial clustering)
\end{itemize}

This gives:

\begin{equation}
N_{\text{cycles}} \approx \frac{0.2-0.5}{0.4} \cdot \frac{50-200}{10-20} \approx 1.25-6.25 \text{ cycles}
\end{equation}

With safety factor for difficult cases and backtracking:

\begin{equation}
N_{\text{cycles}} \approx 2-15 \text{ cycles}
\end{equation}

At 1 second per cycle, this predicts folding times of 2-15 seconds, in agreement with experimental observations.

\subsection{Summary}

The GroEL cavity functions as an ATP-driven resonance chamber that:

\begin{enumerate}
\item \textbf{Resonates with O$_2$ master clock}: Fundamental frequency $\sim 10^{13}$ Hz matches cytoplasmic O$_2$ vibrations.

\item \textbf{Scans frequency space}: ATP-driven geometry changes sweep the fundamental frequency over 40\% range, with harmonics covering proton oscillation frequencies.

\item \textbf{Provides phase-locking potential}: Couples to protein hydrogen bonds with spatially varying strength, creating metastable intermediates.

\item \textbf{Operates in synchronized cycles}: ATP cycle timing is locked to O$_2$ master clock through deep subharmonic relationship.

\item \textbf{Enables multi-cycle folding}: Sequential harmonic scanning over multiple cycles allows complete frequency coverage for complex proteins.
\end{enumerate}

This establishes GroEL as an active folding catalyst that systematically guides proteins through phase space to achieve the native phase-locked state.
