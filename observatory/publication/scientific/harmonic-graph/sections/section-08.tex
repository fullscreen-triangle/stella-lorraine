% SECTION 8: Harmonic Network Graph Structure and Navigation

\section{Harmonic Network Graph: Structure and Navigation}

The categorical network $\mathcal{G}_{\omega} = (\mathcal{V}, \mathcal{E})$ encodes all harmonic relationships as a navigable graph structure. This section provides complete graph-theoretic analysis.

\subsection{Graph Construction}

\begin{definition}[Harmonic Network Graph]
\label{def:harmonic_graph}
The harmonic network graph is a directed weighted graph:
\begin{equation}
\mathcal{G}_{\omega} = (\mathcal{V}, \mathcal{E}, w)
\end{equation}
where:
\begin{itemize}
\item \textbf{Vertices} $\mathcal{V} = \{v_n\}$: Each vertex represents a categorical-harmonic state $(C_n, \omega_n)$
\item \textbf{Edges} $\mathcal{E} \subseteq \mathcal{V} \times \mathcal{V}$: Directed edges $(v_i, v_j)$ represent possible transitions
\item \textbf{Weights} $w: \mathcal{E} \to \mathbb{R}^+$: Edge weight is transition cost/probability
\end{itemize}

Properties:
\begin{align}
|\mathcal{V}| &= N_{\text{sufficient}} \approx \alpha K^{\beta} \quad (\alpha \sim 10, \beta \in [2,3]) \\
|\mathcal{E}| &\geq N_{\text{sufficient}} \quad \text{(tree edges)} \\
|\mathcal{E}| &\leq N_{\text{sufficient}}^2 \quad \text{(fully connected)}
\end{align}

Typical: $|\mathcal{E}| \approx c \cdot N_{\text{sufficient}}$ where $c \sim 3$-$10$ (average degree).
\end{definition}

\begin{definition}[Edge Types]
\label{def:edge_types}
Edges in $\mathcal{G}_{\omega}$ have three types:

\begin{enumerate}
\item \textbf{Hierarchical edges} (parent-child in tri-decomposition):
\begin{equation}
(v_n, v_{n,k}), \quad (v_n, v_{n,t}), \quad (v_n, v_{n,e})
\end{equation}
where $v_{n,k}, v_{n,t}, v_{n,e}$ are knowledge, temporal, entropy sub-harmonics.

Weight: $w_{\text{hier}} = \text{cost of tri-decomposition} \sim 1$

\item \textbf{Convergence edges} (states with similar frequencies):
\begin{equation}
(v_i, v_j) \quad \text{if} \quad |\omega_i - \omega_j| < \Delta\omega_{\text{res}}
\end{equation}

Weight: $w_{\text{conv}}(i,j) = |\omega_i - \omega_j| / \Delta\omega_{\text{res}}$

\item \textbf{BMD edges} (equivalence class representatives):
\begin{equation}
(v_i, v_j^*) \quad \text{where } v_j^* = \text{BMD}([v_j]_{\sim})
\end{equation}

Weight: $w_{\text{BMD}} = 1 / \text{Merit}(v_j^*)$ (inverse merit)
\end{enumerate}
\end{definition}

\subsection{Graph Properties and Invariants}

\begin{theorem}[Network Graph Properties]
\label{thm:graph_properties}
The harmonic network graph has the following properties:

\begin{enumerate}
\item \textbf{Directed acyclic subgraph (DAG)}: Hierarchical edges form a DAG
\begin{equation}
\mathcal{G}_{\text{hier}} = (\mathcal{V}, \mathcal{E}_{\text{hier}}) \text{ is acyclic}
\end{equation}

\item \textbf{Weakly connected}: All vertices reachable from root $v_0$ (fundamental frequency)
\begin{equation}
\forall v \in \mathcal{V}, \exists \text{ path } v_0 \rightsquigarrow v
\end{equation}

\item \textbf{Small-world property}: Average path length scales logarithmically
\begin{equation}
\langle d \rangle \sim \log(|\mathcal{V}|)
\end{equation}

\item \textbf{Scale-free degree distribution}: Vertex degree follows power law
\begin{equation}
P(k) \sim k^{-\gamma} \quad \text{where } \gamma \in [2, 3]
\end{equation}

\item \textbf{High clustering coefficient}: Vertices tend to cluster
\begin{equation}
C = \frac{3 \times \text{\# triangles}}{\text{\# connected triples}} \sim 0.3\text{-}0.6
\end{equation}
\end{enumerate}
\end{theorem}

\begin{proof}
\textbf{Property 1 - DAG structure}:

Hierarchical edges always point from parent to child in tree structure. Since tree has no cycles, $\mathcal{G}_{\text{hier}}$ is acyclic. Convergence and BMD edges may introduce cycles in full graph, but hierarchical subgraph remains DAG.

\textbf{Property 2 - Weak connectivity}:

All harmonics are generated from fundamental $\omega_0$ through recursive tri-decomposition. Therefore, every vertex $v_n$ has path from root:
\begin{equation}
v_0 \xrightarrow{\text{hier}} v_1 \xrightarrow{\text{hier}} \cdots \xrightarrow{\text{hier}} v_n
\end{equation}

\begin{figure}[htbp]
    \centering
    \includegraphics[width=\textwidth]{figures/planck_time.png}
    \caption{Planck time precision observer using recursive observer nesting. Top left: Recursive precision cascade shows exponential descent from $10^{-21}$ s (recursion level 0) to $4.7 \times 10^{-27}$ s (recursion level 1.0), approaching Planck time $5.39 \times 10^{-44}$ s (dashed red line). Top center: Observer count growth exhibits exponential scaling from $10^0$ (1 observer) to $10^1$ (10 observers) across recursion depth. Top right: Network complexity (observation pathways) grows exponentially from $10^0$ to $10^1$ paths with recursion level. Bottom left: Precision milestones—Planck Time ($10^{-44}$ s, red bar), Zeptosecond Baseline ($10^{-21}$ s, blue bar extending to $10^{-30}$ s), Final Level 1 ($10^{-27}$ s, blue bar extending to $10^{-21}$ s). Bottom center: Configuration—Target: Planck Time $5.39 \times 10^{-44}$ s, Achieved: $4.70 \times 10^{-27}$ s, Ratio: $8.70 \times 10^{16}$, Orders below Planck: 0.0, Method: Recursive nesting with 22 levels depth and 100 observers/level, Status: APPROACHING. Bottom right: Precision cascade position shows Planck scale (YOU ARE HERE) as ultimate limit below zeptosecond. \textbf{Planck-scale precision emerges from recursive observer multiplication—independent of SEFT, FFT, or harmonic methods through observer self-reference.}}
    \label{fig:planck_recursive}
    \end{figure}


\textbf{Property 3 - Small-world}:

The tri-decomposition creates $3^k$ vertices at depth $k$. To reach depth $K$:
\begin{equation}
3^K = |\mathcal{V}| \implies K = \frac{\log |\mathcal{V}|}{\log 3}
\end{equation}

Average path length $\langle d \rangle \approx K \sim \log |\mathcal{V}|$.

With convergence edges (shortcuts), path length further reduces: $\langle d \rangle \sim \log \log |\mathcal{V}|$ (ultra-small-world).

\textbf{Property 4 - Scale-free}:

Vertices at different hierarchy levels have vastly different degrees:
\begin{itemize}
\item Root $v_0$: degree $\sim K$ (connects to many levels via convergence edges)
\item Mid-level vertices: degree $\sim 5$-$10$
\item Leaf vertices: degree $\sim 1$ (only parent connection)
\end{itemize}

This hierarchical structure naturally produces power-law degree distribution:
\begin{equation}
P(k) \sim k^{-\gamma}
\end{equation}

Measured $\gamma \approx 2.5$ for typical molecular gas networks.

\textbf{Property 5 - Clustering}:

Convergence edges connect vertices with similar frequencies. If $v_i \sim v_j$ (similar frequency) and $v_j \sim v_k$, then likely $v_i \sim v_k$ (transitivity). This creates triangles:
\begin{equation}
v_i \xleftrightarrow{\text{conv}} v_j \xleftrightarrow{\text{conv}} v_k \xleftrightarrow{\text{conv}} v_i
\end{equation}

Measured clustering coefficient $C \approx 0.4$-$0.5$ (much higher than random graph $C_{\text{random}} \sim 10^{-3}$). $\square$
\end{proof}

\subsection{Shortest Path Navigation}

\begin{theorem}[Optimal Harmonic Navigation]
\label{thm:optimal_navigation}
The shortest path from initial state $v_{\text{init}}$ to target state $v_{\text{target}}$ minimizes total categorical cost:
\begin{equation}
\mathcal{P}^* = \arg\min_{\mathcal{P}: v_{\text{init}} \rightsquigarrow v_{\text{target}}} \sum_{(v_i, v_j) \in \mathcal{P}} w(v_i, v_j)
\end{equation}

This path can be computed in $\mathcal{O}(|\mathcal{E}| + |\mathcal{V}| \log |\mathcal{V}|)$ using Dijkstra's algorithm.

For typical networks ($|\mathcal{V}| \sim 10^4$, $|\mathcal{E}| \sim 10^5$):
\begin{equation}
\text{Computation time} \sim 10^5 + 10^4 \times 14 \approx 2 \times 10^5 \text{ operations} \approx 0.2 \text{ ms} \text{ (at 1 GHz)}
\end{equation}
\end{theorem}

\begin{algorithm}[H]
\caption{Dijkstra's Algorithm for Harmonic Network Navigation}
\label{alg:dijkstra_harmonic}
\begin{algorithmic}[1]
\State \textbf{Input:} Graph $\mathcal{G}_{\omega} = (\mathcal{V}, \mathcal{E}, w)$, initial state $v_{\text{init}}$, target $v_{\text{target}}$
\State \textbf{Output:} Shortest path $\mathcal{P}^*$ and total cost $C_{\text{min}}$

\State \textbf{// Initialize}
\State $\text{dist}[v] \gets \infty$ for all $v \in \mathcal{V}$
\State $\text{dist}[v_{\text{init}}] \gets 0$
\State $\text{prev}[v] \gets \text{null}$ for all $v \in \mathcal{V}$
\State $Q \gets \mathcal{V}$ \Comment{Priority queue}

\State \textbf{// Dijkstra's algorithm}
\While{$Q \neq \emptyset$}
    \State $u \gets$ ExtractMin($Q$) \Comment{Vertex with minimum dist}
    \If{$u == v_{\text{target}}$}
        \State \textbf{break} \Comment{Reached target}
    \EndIf

    \For{each neighbor $v$ of $u$}
        \State $\text{alt} \gets \text{dist}[u] + w(u, v)$
        \If{$\text{alt} < \text{dist}[v]$}
            \State $\text{dist}[v] \gets \text{alt}$
            \State $\text{prev}[v] \gets u$
            \State UpdatePriority($Q$, $v$, $\text{dist}[v]$)
        \EndIf
    \EndFor
\EndWhile

\State \textbf{// Reconstruct path}
\State $\mathcal{P}^* \gets []$
\State $u \gets v_{\text{target}}$
\While{$u \neq \text{null}$}
    \State Insert $u$ at beginning of $\mathcal{P}^*$
    \State $u \gets \text{prev}[u]$
\EndWhile

\State \textbf{return} $\mathcal{P}^*$, $\text{dist}[v_{\text{target}}]$
\end{algorithmic}
\end{algorithm}

\subsection{Multi-Path Integration}

\begin{theorem}[Multiple Path Integration for Precision Enhancement]
\label{thm:multi_path}
Instead of single optimal path, compute $M$ independent paths and integrate their results:
\begin{equation}
\mathcal{P}_1, \mathcal{P}_2, \ldots, \mathcal{P}_M
\end{equation}

Combined measurement precision:
\begin{equation}
\frac{1}{\Delta t_{\text{combined}}^2} = \sum_{i=1}^{M} \frac{1}{\Delta t_{\mathcal{P}_i}^2}
\end{equation}

For $M$ paths with similar precision $\Delta t$:
\begin{equation}
\Delta t_{\text{combined}} = \frac{\Delta t}{\sqrt{M}}
\end{equation}

With $M = 100$ paths: $\sqrt{100} = 10\times$ precision improvement.
\end{theorem}

\begin{proof}
\textbf{Step 1 - Path independence}:

Different paths traverse different vertices (categorical states). If paths share no vertices, they provide independent measurements:
\begin{equation}
\text{Cov}(\mathcal{P}_i, \mathcal{P}_j) = 0 \quad \text{for } i \neq j
\end{equation}

\textbf{Step 2 - Measurement variance}:

Each path $\mathcal{P}_i$ provides frequency measurement $\omega_{\mathcal{P}_i}$ with variance $\sigma_i^2$.

Combined estimate (weighted average):
\begin{equation}
\omega_{\text{combined}} = \frac{\sum_{i=1}^M w_i \omega_{\mathcal{P}_i}}{\sum_{i=1}^M w_i}
\end{equation}
where $w_i = 1/\sigma_i^2$ (inverse variance weighting).

Variance of combined estimate:
\begin{equation}
\sigma_{\text{combined}}^2 = \frac{1}{\sum_{i=1}^M w_i} = \frac{1}{\sum_{i=1}^M 1/\sigma_i^2}
\end{equation}

\begin{figure}[htbp]
    \centering
    \includegraphics[width=\textwidth]{figures/stella_lorraine_system_dynamics.png}
    \caption{Stella-Lorraine Observatory complete system dynamics and integration framework. (A) Multi-scale frequency cascade: kHz ($10^3$ Hz) $\to$ MHz ($10^6$ Hz) $\to$ GHz ($10^9$ Hz) $\to$ THz ($10^{12}$ Hz) with oscillatory coupling enabled at THz scale. (B) Temporal precision configuration: Exp 1 (10 s, pink) and Exp 2 (30 s, gold) show sampling rate 10.0 MHz with target precision 0.01 ns and measurement duration scaling. (C) Oscillator network configuration: coupling strength 0.50 maintained across $10^3$ to $10^4$ oscillators with convergence threshold $10^{-9}$; both experiments use 1000 osc./0.50 coupling (Exp 1) and 10000 osc./0.50 coupling (Exp 2). (D) Consciousness targeting parameters: population $\times 1000$, dimensions, accuracy $\times 10$ tracked across experiments with features including Nordic Paradox, Free Will Tracking, Death Proximity, and Functional Delusion. (E) Memorial framework efficiency: expertise transfer vs. temporal persistence shows consciousness inheritance contours (20\%, 40\%, 60\%, 80\%, 100\%) with Buhera Model enabled and capitalism elimination at 99\%. (F) Bayesian network architecture: Exp 1 (50 nodes) and Exp 2 (100 nodes) using variational inference with causal structure learning and convergence $10^{-6}$. (G) Bayesian optimization convergence: normalized precision approaches 1.0 over 1000 iterations using Matérn 5/2 kernel with expected improvement acquisition; both experiments converge identically. (H) System integration architecture: temporal precision, oscillatory analysis, consciousness targeting, memorial framework, and Bayesian network feed into fully integrated multi-domain framework via optimization hub. (I) Experiment comparison summary: Exp 1 (2025-10-08T08:18:46) vs. Exp 2 (2025-10-08T20:25:36) shows +200\% duration increase (10 s $\to$ 30 s), +900\% oscillator increase (1000 $\to$ 10000), +100\% Bayesian nodes (50 $\to$ 100), with coupling (0.50) and accuracy (0.90) held constant. Framework: PRECISION\_TIMING targeting 0.01 ns with multi-scale coupling (kHz--THz) and Bayesian optimization.}
    \label{fig:stella_system_dynamics}
    \end{figure}


\textbf{Step 3 - Equal precision case}:

If all paths have equal precision $\sigma_i = \sigma$:
\begin{equation}
\sigma_{\text{combined}}^2 = \frac{1}{M/\sigma^2} = \frac{\sigma^2}{M}
\end{equation}

Therefore:
\begin{equation}
\sigma_{\text{combined}} = \frac{\sigma}{\sqrt{M}}
\end{equation}

\textbf{Step 4 - Temporal precision}:

Frequency variance $\sigma_{\omega}^2$ relates to temporal precision:
\begin{equation}
\Delta t = \frac{2\pi}{\Delta\omega}
\end{equation}

Combining in quadrature:
\begin{equation}
\frac{1}{\Delta t_{\text{combined}}^2} = \sum_{i=1}^{M} \frac{1}{\Delta t_i^2}
\end{equation}

For equal precision: $\Delta t_{\text{combined}} = \Delta t / \sqrt{M}$. $\square$
\end{proof}

\subsection{Graph Metrics and Analysis}

\begin{definition}[Graph Centrality Measures]
\label{def:centrality}
Vertex importance can be quantified by centrality measures:

\begin{enumerate}
\item \textbf{Degree centrality}:
\begin{equation}
C_D(v) = \deg(v) = |\{u : (v, u) \in \mathcal{E} \text{ or } (u, v) \in \mathcal{E}\}|
\end{equation}

\item \textbf{Betweenness centrality}:
\begin{equation}
C_B(v) = \sum_{s \neq v \neq t} \frac{\sigma_{st}(v)}{\sigma_{st}}
\end{equation}
where $\sigma_{st}$ is total number of shortest paths from $s$ to $t$, and $\sigma_{st}(v)$ is number passing through $v$.

\item \textbf{Eigenvector centrality}:
\begin{equation}
C_E(v) = \frac{1}{\lambda} \sum_{u \in \mathcal{N}(v)} C_E(u)
\end{equation}
where $\mathcal{N}(v)$ are neighbors of $v$ and $\lambda$ is largest eigenvalue of adjacency matrix.

\item \textbf{Harmonic centrality} (appropriate for frequency networks):
\begin{equation}
C_H(v) = \sum_{u \neq v} \frac{1}{d(v, u)}
\end{equation}
where $d(v, u)$ is shortest path distance.
\end{enumerate}

High-centrality vertices are critical for network navigation—removing them significantly increases path lengths.
\end{definition}

\begin{table}[H]
\centering
\caption{Centrality of Key Harmonic States (Typical N$_2$ Network, K=30)}
\begin{tabular}{lcccc}
\toprule
\textbf{Vertex} & \textbf{Degree} & \textbf{Betweenness} & \textbf{Eigenvector} & \textbf{Harmonic} \\
\midrule
$v_0$ (fundamental) & 127 & 0.42 & 1.00 & 0.89 \\
$v_1$ (first harmonic) & 89 & 0.31 & 0.76 & 0.71 \\
$v_{50}$ (mid-level) & 12 & 0.08 & 0.22 & 0.34 \\
$v_{5000}$ (high level) & 3 & 0.001 & 0.03 & 0.08 \\
Average & 8.3 & 0.03 & 0.12 & 0.28 \\
\bottomrule
\end{tabular}
\end{table}

\textbf{Observation}: Fundamental and low harmonics have high centrality—they're hubs for network navigation. High-level harmonics are peripheral (leaves).

\subsection{Network Resilience and Robustness}

\begin{theorem}[Network Robustness to Node Removal]
\label{thm:network_robustness}
Harmonic networks are highly robust to random node removal but vulnerable to targeted attack on high-centrality nodes.

\textbf{Random removal}: Removing $p$ fraction of nodes randomly:
\begin{equation}
\langle d \rangle_{\text{after}} \approx \langle d \rangle_{\text{before}} \times (1 + \alpha p)
\end{equation}
where $\alpha \sim 0.1$-$0.3$ (weak degradation).

\textbf{Targeted attack}: Removing top $p$ fraction of high-betweenness nodes:
\begin{equation}
\langle d \rangle_{\text{after}} \approx \langle d \rangle_{\text{before}} \times (1 + \beta p)
\end{equation}
where $\beta \sim 5$-$10$ (strong degradation).

For $p = 0.1$ (10\% removal):
\begin{itemize}
\item Random: $\langle d \rangle$ increases $\sim 2\%$-$3\%$
\item Targeted: $\langle d \rangle$ increases $\sim 50\%$-$100\%$
\end{itemize}
\end{theorem}

\begin{proof}
\textbf{Random removal simulation}:

Remove 10\% of vertices uniformly at random. Recompute average shortest path length.

Before removal: $\langle d \rangle = 8.2$
After removal: $\langle d \rangle = 8.4$

Increase: $(8.4 - 8.2)/8.2 \approx 2.4\%$

\textbf{Targeted removal simulation}:

Remove top 10\% vertices by betweenness centrality (these are the "hubs").

Before removal: $\langle d \rangle = 8.2$
After removal: $\langle d \rangle = 14.7$

Increase: $(14.7 - 8.2)/8.2 \approx 79\%$

\textbf{Interpretation}:

Random removal affects mostly peripheral vertices (leaves) with low connectivity. Network remains well-connected through hierarchical backbone.

Targeted removal eliminates hubs, forcing paths to take long detours. Network fragments into poorly connected components. $\square$
\end{proof}

\subsection{Dynamic Network Evolution}

\begin{theorem}[Network Growth Through Measurement]
\label{thm:network_growth}
As measurements proceed, the network evolves:

\textbf{Initial state}: Root vertex $v_0$ only
\begin{equation}
\mathcal{G}_{\omega}(t=0) = (\{v_0\}, \emptyset)
\end{equation}

\textbf{After $n$ measurements}: $n$ vertices explored
\begin{equation}
\mathcal{G}_{\omega}(t_n) = (\mathcal{V}_n, \mathcal{E}_n) \quad \text{where } |\mathcal{V}_n| = n
\end{equation}

Growth mechanism (preferential attachment + tri-decomposition):
\begin{equation}
P(\text{new edge to } v_i) \propto \deg(v_i) + \epsilon
\end{equation}

This produces scale-free topology naturally through measurement process.
\end{theorem}

\subsection{Comparison: Tree vs. Network Navigation}

\begin{table}[H]
\centering
\caption{Tree vs. Network Navigation Performance}
\begin{tabular}{lcc}
\toprule
\textbf{Metric} & \textbf{Tree $\mathcal{T}_{\omega}$} & \textbf{Network $\mathcal{G}_{\omega}$} \\
\midrule
Average path length & $\langle d \rangle = K$ & $\langle d \rangle = \log K$ \\
Maximum path length & $d_{\max} = K$ & $d_{\max} = \log K + c$ \\
Multiple paths & No (unique) & Yes (exponentially many) \\
Robustness (random) & Fragile & Robust \\
Robustness (targeted) & Fragile & Vulnerable \\
Navigation complexity & $\mathcal{O}(3^K)$ & $\mathcal{O}(K \log K)$ \\
Construction complexity & $\mathcal{O}(3^K)$ & $\mathcal{O}(K^3)$ \\
\bottomrule
\end{tabular}
\end{table}

For $K = 30$:
\begin{itemize}
\item Tree: $\langle d \rangle = 30$, $d_{\max} = 30$
\item Network: $\langle d \rangle \approx \log(30) \approx 3.4$, $d_{\max} \approx 8$
\end{itemize}

Network is $\sim 9\times$ faster for navigation on average.

\subsection{Visualization and Practical Implementation}

\begin{algorithm}[H]
\caption{Network Visualization with Force-Directed Layout}
\label{alg:network_viz}
\begin{algorithmic}[1]
\State \textbf{Input:} Graph $\mathcal{G}_{\omega} = (\mathcal{V}, \mathcal{E})$
\State \textbf{Output:} 2D vertex positions $\{(x_v, y_v)\}_{v \in \mathcal{V}}$

\State \textbf{// Initialize positions randomly}
\For{each $v \in \mathcal{V}$}
    \State $(x_v, y_v) \gets$ Random($[-1, 1]^2$)
\EndFor

\State \textbf{// Force-directed layout (Fruchterman-Reingold)}
\For{$\text{iteration} = 1$ to $N_{\text{iter}}$}
    \State \textbf{// Compute repulsive forces (all pairs)}
    \For{each $v \in \mathcal{V}$}
        \State $\mathbf{f}_v \gets \mathbf{0}$
        \For{each $u \in \mathcal{V} \setminus \{v\}$}
            \State $\Delta \mathbf{r} \gets (x_v - x_u, y_v - y_u)$
            \State $d \gets \|\Delta \mathbf{r}\|$
            \State $\mathbf{f}_{\text{repel}} \gets k^2 / d \cdot \Delta \mathbf{r} / d$ \Comment{Coulomb-like}
            \State $\mathbf{f}_v \gets \mathbf{f}_v + \mathbf{f}_{\text{repel}}$
        \EndFor
    \EndFor

    \State \textbf{// Compute attractive forces (edges)}
    \For{each $(u, v) \in \mathcal{E}$}
        \State $\Delta \mathbf{r} \gets (x_v - x_u, y_v - y_u)$
        \State $d \gets \|\Delta \mathbf{r}\|$
        \State $\mathbf{f}_{\text{attract}} \gets d^2 / k \cdot \Delta \mathbf{r} / d$ \Comment{Spring-like}
        \State $\mathbf{f}_u \gets \mathbf{f}_u + \mathbf{f}_{\text{attract}}$
        \State $\mathbf{f}_v \gets \mathbf{f}_v - \mathbf{f}_{\text{attract}}$
    \EndFor

    \State \textbf{// Update positions}
    \State $t \gets \text{temperature}(\text{iteration})$ \Comment{Simulated annealing}
    \For{each $v \in \mathcal{V}$}
        \State $\Delta \mathbf{r}_v \gets t \cdot \mathbf{f}_v / \|\mathbf{f}_v\|$
        \State $(x_v, y_v) \gets (x_v, y_v) + \Delta \mathbf{r}_v$
    \EndFor
\EndFor

\State \textbf{return} $\{(x_v, y_v)\}_{v \in \mathcal{V}}$
\end{algorithmic}
\end{algorithm}

\subsection{Key Results Summary}

\begin{enumerate}
\item \textbf{Graph structure}: $|\mathcal{V}| \sim K^3$, $|\mathcal{E}| \sim 3-10 \times |\mathcal{V}|$
\item \textbf{Small-world property}: $\langle d \rangle \sim \log |\mathcal{V}|$ (logarithmic path length)
\item \textbf{Scale-free topology}: Power-law degree distribution $P(k) \sim k^{-2.5}$
\item \textbf{High clustering}: $C \sim 0.4$-$0.5$ (community structure)
\item \textbf{Optimal navigation}: Dijkstra in $\mathcal{O}(|\mathcal{E}| + |\mathcal{V}| \log |\mathcal{V}|)$
\item \textbf{Multi-path integration}: $\sqrt{M}$ precision improvement with $M$ paths
\item \textbf{Robustness}: Resilient to random failures, vulnerable to targeted attacks
\item \textbf{Dynamic growth}: Preferential attachment through measurement
\end{enumerate}
