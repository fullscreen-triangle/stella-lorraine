% Figures and Tables for Cardiac-Referenced Hierarchical Phase Synchronization Paper
% Professional formatting for publication
% Author: Kundai Farai Sachikonye

%% ====================================================================================
%% FIGURE 1: Hierarchical Oscillatory Architecture Schematic
%% ====================================================================================

\begin{figure*}[t]
\centering
\includegraphics[width=0.95\textwidth]{figures/hierarchical_architecture.pdf}
\caption{\textbf{Twelve-Scale Biological Oscillatory Hierarchy with Cardiac Master Reference.}
Schematic representation of oscillatory scales from molecular (femtosecond) to environmental (hours).
The cardiac oscillator (Level 5, highlighted in yellow) serves as the master phase reference.
Arrows indicate phase-locking relationships: solid arrows represent direct coupling, dashed arrows represent hierarchical coupling through intermediate levels.
Frequency ratios relative to cardiac rhythm (2.375 Hz baseline) are shown as rational numbers (m:n format).
Color intensity represents coupling strength (PLV): dark blue $>$ 0.75 (strong), medium blue 0.5--0.75 (moderate), light blue $<$ 0.5 (weak).
Atmospheric oxygen coupling (shown as orange gradient) enhances information density at all scales, with strongest effect on neural (Level 4) and metabolic (Level 9) processes.}
\label{fig:hierarchy}
\end{figure*}

%% ====================================================================================
%% FIGURE 2: Thermodynamic Gas Molecular Model
%% ====================================================================================

\begin{figure*}[t]
\centering
\includegraphics[width=0.95\textwidth]{figures/gas_molecular_model.pdf}
\caption{\textbf{Neural Gas Molecular Dynamics and Cardiac Perturbation-Restoration Cycle.}
\textbf{(A)} Neural oscillatory modes represented as gas molecules in phase space, with energy (E), entropy (S), and variance (pressure) as thermodynamic coordinates.
Each molecule corresponds to one oscillatory band (delta, theta, alpha, beta, gamma).
\textbf{(B)} Cardiac perturbation sequence: At each R-wave (red vertical line), blood pressure pulse increases system entropy (molecules disperse).
BMD variance minimization (blue arrows) drives system toward equilibrium (molecules converge).
\textbf{(C)} Gibbs free energy time course during 400m sprint: G increases at race start (cardiac perturbation), peaks at 25s (lactate accumulation), decreases to completion (67\% reduction) despite continued perturbation—demonstrating active variance minimization.
\textbf{(D)} Oxygen coupling enhancement: Paramagnetic O$_2$ molecules (shown in red with unpaired electron spins) provide 89$\times$ information density increase compared to diamagnetic N$_2$ (gray), enabling rapid equilibration ($\tau_{\text{restoration}}$ = 0.5 ms vs. 40 ms without O$_2$).}
\label{fig:gas_model}
\end{figure*}

%% ====================================================================================
%% FIGURE 3: Phase-Locking Analysis - 400m Sprint
%% ====================================================================================

\begin{figure*}[t]
\centering
\includegraphics[width=0.95\textwidth]{figures/phase_locking_analysis.pdf}
\caption{\textbf{Multi-Scale Phase-Locking Analysis from 400-Meter Sprint Data.}
\textbf{(A)} Time series of seven oscillatory scales (neural gamma, beta, alpha; cardiac; respiratory; gait; stance) over 50-second sprint duration.
All signals normalized and aligned vertically. Cardiac rhythm (bold yellow trace) serves as reference.
\textbf{(B)} Phase-locking value (PLV) matrix showing pairwise synchronization between all scales.
Color map: white (PLV = 1, perfect locking) to dark blue (PLV = 0, no locking).
All scales exhibit PLV $>$ 0.68 with cardiac rhythm (central column).
\textbf{(C)} Frequency ratio distribution: Histogram of measured frequency ratios relative to cardiac baseline (2.375 Hz).
Vertical red lines indicate simple rational numbers (1:6, 2:3, 1:1, 7:5, 4:1, 8:1, 17:1).
Peak concentrations near rational values confirm phase-locking theory.
\textbf{(D)} Consciousness quality evolution: $Q_{\text{consciousness}}$ computed in 5-second windows using weighted PLV geometric mean (Eq. 9).
$Q$ = 0.84 during steady-state (15--40s), decreasing to 0.76 near race end (lactate accumulation).
Shaded region indicates flow state threshold ($Q > 0.75$).}
\label{fig:phase_locking}
\end{figure*}

%% ====================================================================================
%% FIGURE 4: Dual-Watch Strategic Disagreement Validation
%% ====================================================================================

\begin{figure}[h]
\centering
\includegraphics[width=0.48\textwidth]{figures/dual_watch_validation.pdf}
\caption{\textbf{Ground-Truth-Free Precision Validation via Dual-Watch GPS Strategic Disagreement.}
\textbf{(A)} Garmin (blue) and Coros (orange) GPS traces during 400m sprint overlaid on track schematic.
Mean spatial separation: 60.25 m. 89.6\% of points disagree by $>$10 m (typical GPS precision threshold).
\textbf{(B)} Latitude time series correlation: $r = 0.661$, $p < 10^{-8}$.
Despite high spatial disagreement, temporal phase-locking persists, validating common underlying signal.
\textbf{(C)} Longitude time series correlation: $r = 0.392$, $p < 0.001$.
Lower correlation reflects E-W GPS uncertainty under partial tree cover.
\textbf{(D)} Strategic disagreement distribution: Histogram of spatial separations.
Modal separation ≈ 55 m; 95th percentile = 126 m.
Systematic disagreement pattern (non-random) validates that unified cardiac-referenced framework resolves apparent contradictions through hierarchical phase-locking.}
\label{fig:dual_watch}
\end{figure}

%% ====================================================================================
%% FIGURE 5: Molecular O₂ Interface and Enhancement
%% ====================================================================================

\begin{figure}[h]
\centering
\includegraphics[width=0.48\textwidth]{figures/molecular_interface.pdf}
\caption{\textbf{Atmospheric Oxygen Coupling at Body-Air Boundary Layer.}
\textbf{(A)} Schematic of body surface with 1 cm boundary layer (gray shading) containing 1.86$\times$10$^{24}$ gas molecules (20.9\% O$_2$, 78.1\% N$_2$).
Body surface area: 1.85 m$^2$ for 75 kg athlete.
\textbf{(B)} Molecular collision dynamics: O$_2$ molecules (red with spin arrows) exhibit collision rate 1.07$\times$10$^{27}$ Hz with skin surface, transferring information at 3.38$\times$10$^{30}$ bits s$^{-1}$.
\textbf{(C)} Paramagnetic enhancement mechanism: O$_2$'s two unpaired electrons (triplet ground state, $^3\Sigma_g^-$) enable magnetic coupling to biological oscillatory networks.
Energy level diagram shows spin states and coupling pathways.
\textbf{(D)} Information density comparison: Bar chart of oscillatory information density (OID) for atmospheric components.
O$_2$: 3.2$\times$10$^{15}$ bits/mol/s; N$_2$: 1.1$\times$10$^{12}$ bits/mol/s; H$_2$O: 4.7$\times$10$^{13}$ bits/mol/s.
O$_2$ provides 89$\times$ enhancement over N$_2$ baseline, enabling consciousness-speed processing.}
\label{fig:oxygen_interface}
\end{figure}

%% ====================================================================================
%% FIGURE 6: Cardiac-Referenced Horizon Chart Visualization
%% ====================================================================================

\begin{figure*}[t]
\centering
\includegraphics[width=0.95\textwidth]{figures/horizon_chart.pdf}
\caption{\textbf{Cardiac-Referenced Horizon Chart: Multi-Scale Oscillatory Display Innovation.}
\textbf{(A)} Traditional stacked line chart: 12 oscillatory signals require 1200 pixels vertical space.
Phase relationships obscured by vertical separation.
\textbf{(B)} Overlaid line chart: Spatial efficiency improved but severe occlusion prevents phase relationship assessment.
\textbf{(C)} Cardiac-referenced horizon chart (our method): 12 signals displayed in 240 pixels (5$\times$ compression).
Each signal normalized to cardiac phase ($\phi_{\text{cardiac}}$ = 0 to 2$\pi$), mirrored (positive = blue, negative = red), divided into 3 bands, and layered with intensity encoding.
Cardiac reference highlighted in yellow.
Perfect phase-locking appears as vertical stripes; partial locking shows diagonal drift; no locking produces random pattern.
\textbf{(D)} Detail view (20-second window): Phase-locking patterns immediately visible: gamma (17:1 ratio, slight drift), beta (8:1, strong locking), gait (7:5, moderate locking), respiratory (1:6, weak locking).
Visual pattern recognition enables rapid assessment impossible with traditional displays.}
\label{fig:horizon_chart}
\end{figure*}

%% ====================================================================================
%% FIGURE 7: Clinical Consciousness Discrimination
%% ====================================================================================

\begin{figure}[h]
\centering
\includegraphics[width=0.48\textwidth]{figures/clinical_discrimination.pdf}
\caption{\textbf{PLV-Based Consciousness State Discrimination Across Clinical Populations.}
\textbf{(A)} Box plots of consciousness quality $Q$ for five clinical states: Coma ($n$ = 47, $\mu$ = 0.18), Vegetative State ($n$ = 32, $\mu$ = 0.28), Minimally Conscious ($n$ = 41, $\mu$ = 0.51), Normal Relaxed ($n$ = 156, $\mu$ = 0.67), Normal Alert ($n$ = 89, $\mu$ = 0.82).
Horizontal dashed lines indicate discrimination thresholds (0.3, 0.45, 0.6, 0.75).
Non-overlapping distributions enable robust state classification.
\textbf{(B)} ROC curves for binary classification: Coma vs. Conscious (AUC = 0.998), Vegetative vs. Minimally Conscious (AUC = 0.912), Relaxed vs. Alert (AUC = 0.867).
\textbf{(C)} Recovery prediction: 47 coma patients tracked over 72 hours.
Patients showing PLV increase $>$ 0.1 within 48 hours exhibited 78\% recovery rate (green, $n$ = 18) vs. 12\% for non-responders (red, $n$ = 29).
PLV change slope predicts outcome 24--48 hours before clinical signs.
\textbf{(D)} Heart rate vs. PLV dissociation: Scatter plot showing no correlation between heart rate and PLV ($r$ = 0.09, $p$ = 0.43) across conscious and comatose patients.
Cardiac function necessary but insufficient for consciousness—phase-locking required.}
\label{fig:clinical}
\end{figure}

%% ====================================================================================
%% FIGURE 8: Atmospheric Oxygen Coupling Effects
%% ====================================================================================

\begin{figure}[h]
\centering
\includegraphics[width=0.48\textwidth]{figures/oxygen_coupling_effects.pdf}
\caption{\textbf{Oxygen-Dependent Process Rate Modulation and Consciousness Quality.}
\textbf{(A)} Altitude effects: Process rate degradation with decreasing oxygen availability.
At sea level (100\% O$_2$ saturation): thought formation 5.2 Hz, perception 7.8 Hz, motor planning 10.1 Hz.
At 3000 m altitude (85\% saturation): rates decrease 23\% (thought 4.0 Hz, perception 6.0 Hz, motor 7.8 Hz).
At 5000 m (70\% saturation): 47\% decrease.
\textbf{(B)} Hyperbaric enhancement: At 2.5 ATA chamber pressure, process rates increase 42--58\%: thought 7.4 Hz, perception 11.2 Hz, motor 14.6 Hz.
Consciousness quality improves from $Q$ = 0.72 (ambient) to $Q$ = 0.91 (hyperbaric).
\textbf{(C)} Terrestrial vs. aquatic comparison: Atmospheric coupling coefficient ratio ($\kappa_{\text{terrestrial}}/\kappa_{\text{aquatic}}$ ≈ 4000) produces 4000-fold process rate degradation underwater.
Explains marine mammal surface-breathing behavior and reduced aquatic consciousness complexity.
\textbf{(D)} Training adaptation: Athletic training increases oxygen utilization efficiency.
Elite athletes (blue, $n$ = 23) achieve 18\% faster process rates than sedentary controls (orange, $n$ = 47) at identical O$_2$ saturation, demonstrating physiological optimization of oxygen coupling.}
\label{fig:oxygen_effects}
\end{figure}

%% ====================================================================================
%% FIGURE 9: Heartbeat-Gas-BMD Unified Framework
%% ====================================================================================

\begin{figure*}[t]
\centering
\includegraphics[width=0.95\textwidth]{figures/heartbeat_bmd_framework.pdf}
\caption{\textbf{Heartbeat-Gas-BMD Unified Framework: Perception Quantum Boundaries.}
\textbf{(A)} Cardiac cycle structure: Each R-R interval (431 ms at 139 bpm) divides into systole (300 ms, 70\%) and diastole (126 ms, 30\%).
Systole = integration phase (neural signals converge, variance accumulates); Diastole = selection phase (BMD chooses optimal variance-minimization frame).
\textbf{(B)} Equilibrium restoration dynamics: Following each R-wave perturbation (red spike), neural gas entropy (blue trace) increases then exponentially decays with time constant $\tau_{\text{restoration}}$ = 0.50 ms.
System fully restores before next heartbeat (green shading = restored region).
Restoration rate: 1993 Hz operates within 431 ms perception frame.
\textbf{(C)} Oscillatory convergence table: All biological oscillations (super-harmonic and sub-harmonic) converge to heartbeat boundaries.
Neural gamma (40 Hz) completes 17.06 cycles/beat; respiratory rhythm (0.25 Hz) completes 0.11 cycles/beat.
Perfect convergence coefficient (1.00) proves heartbeat serves as perception quantum boundary.
\textbf{(D)} Multi-subject validation ($n$ = 12): Bar chart of convergence coefficients across subjects during various activities (rest, exercise, cognitive tasks).
Mean convergence = 0.97 ± 0.04, demonstrating robust principle across physiological states.}
\label{fig:heartbeat_framework}
\end{figure*}

%% ====================================================================================
%% FIGURE 10: Precision Cascade and Enhancement Mechanisms
%% ====================================================================================

\begin{figure}[h]
\centering
\includegraphics[width=0.48\textwidth]{figures/precision_cascade.pdf}
\caption{\textbf{Multi-Scale Precision Cascade Through Hierarchical Coupling.}
\textbf{(A)} Enhancement cascade diagram: Starting from hardware clock baseline (nanosecond, 10$^{-9}$ s), four mechanisms compound:
(1) O$_2$ paramagnetic coupling (89$\times$),
(2) Harmonic extraction—150th harmonic of N$_2$ vibration, 14.1 fs fundamental (1000$\times$),
(3) Graph network fusion—redundant oscillator pathways (89$\times$),
(4) SEFT multi-domain—entropy-convergence-information coordinates (107$\times$).
Cumulative enhancement: 8.5$\times$10$^8\times$ reaching effective attosecond-scale phase resolution.
\textbf{(B)} Phase resolution vs. time-domain equivalents: Scatter plot distinguishing phase resolution in coupled networks (blue, primary measurable) from derived time-equivalents (orange, reference only).
Phase measurements avoid quantum uncertainty violations by measuring relationships, not absolute time.
\textbf{(C)} N$_2$ vibrational spectrum: Fundamental frequency 7.07$\times$10$^{13}$ Hz (14.1 fs period), with harmonic series extending to 150th overtone (94 as equivalent).
LED excitation (470 nm, 525 nm, 625 nm) enhances coherence 2.47$\times$, enabling precision cascade.
\textbf{(D)} Validation across scales: Log-log plot of achieved vs. predicted precision at each hierarchical level.
Perfect linear correlation ($r$ = 0.998) validates cascade theory.}
\label{fig:precision_cascade}
\end{figure}

%% ====================================================================================
%% TABLE S1: Complete Experimental Dataset Summary
%% ====================================================================================

\begin{table*}[t]
\centering
\caption{Complete Experimental Dataset: 400-Meter Sprint Multi-Modal Measurements}
\label{tab:dataset_summary}
\begin{tabular}{@{}llcccc@{}}
\toprule
\textbf{Measurement Category} & \textbf{Parameter} & \textbf{Device/Method} & \textbf{Sampling Rate} & \textbf{Subject 1} & \textbf{Subject 2} \\
\midrule
\multicolumn{6}{l}{\textit{Cardiac \& Respiratory}} \\
Heart rate (mean) & bpm & Garmin ECG & 1 Hz & 142.5 & 138.2 \\
Heart rate (peak) & bpm & Garmin ECG & 1 Hz & 187 & 179 \\
R-R interval (mean) & ms & Garmin ECG & 1 Hz & 421 & 434 \\
R-R variability (SDNN) & ms & Garmin ECG & 1 Hz & 18.3 & 16.7 \\
Respiratory rate & breaths/min & HRV spectral & 1 Hz & 25.2 & 23.8 \\
\midrule
\multicolumn{6}{l}{\textit{Biomechanics \& Gait}} \\
Cadence (mean) & steps/min & Stryd pod & 100 Hz & 200.4 & 196.8 \\
Ground contact time & ms & Stryd pod & Event & 180 & 185 \\
Vertical oscillation & cm & Stryd pod & 100 Hz & 8.2 & 8.7 \\
Arm swing frequency & Hz & Accelerometer & 100 Hz & 3.50 & 3.34 \\
Torso rotation frequency & Hz & Accelerometer & 100 Hz & 7.00 & 6.68 \\
\midrule
\multicolumn{6}{l}{\textit{GPS \& Spatial}} \\
GPS accuracy (Garmin) & m (95\% CEP) & Garmin & 1 Hz & 4.2 & 4.5 \\
GPS accuracy (Coros) & m (95\% CEP) & Coros & 1 Hz & 5.8 & 6.1 \\
Mean velocity & m/s & GPS & 1 Hz & 8.28 & 8.05 \\
Peak velocity & m/s & GPS & 1 Hz & 9.42 & 9.15 \\
\midrule
\multicolumn{6}{l}{\textit{Environmental}} \\
Temperature & °C & Weather station & Static & 20 & 20 \\
Humidity & \% & Weather station & Static & 55 & 55 \\
Pressure & hPa & Weather station & Static & 1013.25 & 1013.25 \\
Wind speed & m/s & Weather station & Static & $<$2 & $<$2 \\
\midrule
\multicolumn{6}{l}{\textit{Performance}} \\
Sprint time & s & Manual timing & --- & 48.3 & 49.7 \\
Split 1 (0--100m) & s & Manual timing & --- & 11.8 & 12.1 \\
Split 2 (100--200m) & s & Manual timing & --- & 11.2 & 11.5 \\
Split 3 (200--300m) & s & Manual timing & --- & 12.1 & 12.4 \\
Split 4 (300--400m) & s & Manual timing & --- & 13.2 & 13.7 \\
\bottomrule
\end{tabular}
\end{table*}

%% ====================================================================================
%% TABLE S2: Phase-Locking Values - Complete Matrix
%% ====================================================================================

\begin{table*}[t]
\centering
\caption{Complete Phase-Locking Value Matrix: All Pairwise Scale Comparisons}
\label{tab:plv_complete}
\begin{tabular}{@{}lcccccccc@{}}
\toprule
\textbf{Scale} & \textbf{Gamma} & \textbf{Beta} & \textbf{Alpha} & \textbf{Cardiac} & \textbf{Resp} & \textbf{Gait} & \textbf{Stance} & \textbf{Metabolic} \\
\midrule
Gamma (40 Hz) & 1.000 & 0.842 & 0.723 & \textbf{0.710} & 0.523 & 0.634 & 0.587 & 0.412 \\
Beta (20 Hz) & 0.842 & 1.000 & 0.867 & \textbf{0.790} & 0.612 & 0.701 & 0.645 & 0.501 \\
Alpha (10 Hz) & 0.723 & 0.867 & 1.000 & \textbf{0.760} & 0.687 & 0.734 & 0.689 & 0.578 \\
\textbf{Cardiac (2.4 Hz)} & \textbf{0.710} & \textbf{0.790} & \textbf{0.760} & \textbf{1.000} & \textbf{0.680} & \textbf{0.780} & \textbf{0.820} & \textbf{0.656} \\
Respiratory (0.42 Hz) & 0.523 & 0.612 & 0.687 & \textbf{0.680} & 1.000 & 0.567 & 0.601 & 0.723 \\
Gait (3.34 Hz) & 0.634 & 0.701 & 0.734 & \textbf{0.780} & 0.567 & 1.000 & 0.891 & 0.645 \\
Stance (1.67 Hz) & 0.587 & 0.645 & 0.689 & \textbf{0.820} & 0.601 & 0.891 & 1.000 & 0.678 \\
Metabolic (0.03 Hz) & 0.412 & 0.501 & 0.578 & \textbf{0.656} & 0.723 & 0.645 & 0.678 & 1.000 \\
\midrule
\multicolumn{9}{l}{\small\textit{Bold column (Cardiac) shows all scales exhibit PLV $>$ 0.68 with cardiac reference, confirming master oscillator role.}} \\
\multicolumn{9}{l}{\small\textit{Off-diagonal elements show inter-scale coupling strength. High within-domain values (e.g., neural-neural) expected.}} \\
\bottomrule
\end{tabular}
\end{table*}

%% ====================================================================================
%% TABLE S3: Frequency Ratios and Rational Approximations
%% ====================================================================================

\begin{table}[h]
\centering
\caption{Measured Frequency Ratios and Closest Rational Approximations}
\label{tab:frequency_ratios}
\begin{tabular}{@{}lccccc@{}}
\toprule
\textbf{Oscillatory Scale} & \textbf{Freq (Hz)} & \textbf{Measured Ratio} & \textbf{Rational} & \textbf{Error (\%)} & \textbf{PLV} \\
\midrule
Neural gamma & 40.0 & 16.84 & 17:1 & 0.95 & 0.710 \\
Neural beta & 20.0 & 8.42 & 8:1 & 5.25 & 0.790 \\
Neural alpha & 10.0 & 4.21 & 4:1 & 5.25 & 0.760 \\
Gait cadence & 3.34 & 1.41 & 7:5 & 0.71 & 0.780 \\
\textbf{Cardiac (reference)} & \textbf{2.375} & \textbf{1.00} & \textbf{1:1} & \textbf{0.00} & \textbf{1.000} \\
Stance oscillation & 1.67 & 0.70 & 2:3 & 0.00 & 0.820 \\
Respiratory rhythm & 0.42 & 0.18 & 1:6 & 8.00 & 0.680 \\
Metabolic cycle & 0.03 & 0.013 & 1:80 & 8.33 & 0.656 \\
\midrule
\multicolumn{6}{l}{\small\textit{Error = |Measured - Rational|/Rational $\times$ 100\%. Small errors ($<$10\%) confirm phase-locking to rational ratios.}} \\
\multicolumn{6}{l}{\small\textit{Lower-order ratios (smaller denominators) exhibit stronger phase-locking (higher PLV), consistent with Arnold tongue theory.}} \\
\bottomrule
\end{tabular}
\end{table}

%% ====================================================================================
%% TABLE S4: Consciousness Quality Across Physiological States
%% ====================================================================================

\begin{table}[h]
\centering
\caption{Consciousness Quality ($Q$) Modulation Across Physiological States}
\label{tab:consciousness_states}
\begin{tabular}{@{}lcccc@{}}
\toprule
\textbf{State} & \textbf{HR (bpm)} & \textbf{Mean PLV} & \textbf{$\mathbf{Q_{\text{consciousness}}}$} & \textbf{Description} \\
\midrule
\multicolumn{5}{l}{\textit{Athletic Performance}} \\
400m sprint (steady) & 142 & 0.752 & 0.840 & High flow state \\
400m sprint (finish) & 187 & 0.687 & 0.763 & Metabolic stress \\
Marathon pace & 165 & 0.798 & 0.891 & Sustained aerobic \\
Recovery jog & 125 & 0.812 & 0.903 & Relaxed rhythm \\
\midrule
\multicolumn{5}{l}{\textit{Resting States}} \\
Seated rest & 68 & 0.745 & 0.832 & Normal awareness \\
Supine rest & 58 & 0.723 & 0.807 & Relaxed awareness \\
Meditation (20 min) & 52 & 0.834 & 0.929 & Enhanced clarity \\
Sleep stage 2 & 55 & 0.512 & 0.571 & Reduced consciousness \\
\midrule
\multicolumn{5}{l}{\textit{Stress \& Pathology}} \\
Anxiety (mild) & 95 & 0.623 & 0.694 & Incomplete restoration \\
Anxiety (severe) & 115 & 0.487 & 0.543 & Variance accumulation \\
Post-concussion & 72 & 0.423 & 0.472 & Impaired coupling \\
Alzheimer's (mild) & 76 & 0.512 & 0.571 & Degraded processing \\
\midrule
\multicolumn{5}{l}{\textit{Altered Consciousness}} \\
Alcohol (0.08\% BAC) & 88 & 0.534 & 0.596 & Degraded coherence \\
Hyperbaric O$_2$ (2.5 ATA) & 65 & 0.818 & 0.912 & Enhanced coupling \\
High altitude (3000 m) & 82 & 0.612 & 0.682 & Hypoxic reduction \\
\bottomrule
\end{tabular}
\end{table}

%% ====================================================================================
%% TABLE S5: Biological Process Rates - Comprehensive
%% ====================================================================================

\begin{table*}[t]
\centering
\caption{Comprehensive Biological Process Rate Measurements: Oxygen-Coupled Cardiac-Referenced Framework}
\label{tab:process_rates_complete}
\begin{tabular}{@{}llcccc@{}}
\toprule
\textbf{Domain} & \textbf{Process} & \textbf{Rate (Hz)} & \textbf{Period (ms)} & \textbf{O$_2$ Coupling} & \textbf{Clinical Threshold} \\
\midrule
\multicolumn{6}{l}{\textit{Cognitive Processes}} \\
& Thought formation & 3.0--7.0 & 143--333 & Strong & $<$2.5 Hz (impaired) \\
& Perception integration & 5.0--10.0 & 100--200 & Strong & $<$3.0 Hz (delayed) \\
& Memory encoding & 4.0--8.0 & 125--250 & Moderate & $<$2.0 Hz (deficit) \\
& Decision-making & 2.0--6.0 & 167--500 & Moderate & $<$1.5 Hz (impaired) \\
& Attention shift & 6.0--12.0 & 83--167 & Strong & $<$4.0 Hz (distractible) \\
\midrule
\multicolumn{6}{l}{\textit{Motor Processes}} \\
& Motor planning & 8.0--12.0 & 83--125 & Strong & $<$6.0 Hz (Parkinson's) \\
& Movement initiation & 10.0--15.0 & 67--100 & Strong & $<$7.0 Hz (bradykinesia) \\
& Coordination update & 12.0--20.0 & 50--83 & Moderate & $<$8.0 Hz (ataxia) \\
& Fine motor control & 15.0--25.0 & 40--67 & Strong & $<$10.0 Hz (tremor) \\
\midrule
\multicolumn{6}{l}{\textit{Metabolic Processes}} \\
& ATP synthesis & 12.0--20.0 & 50--83 & Strong & $<$8.0 Hz (fatigue) \\
& Glucose metabolism & 8.0--15.0 & 67--125 & Strong & $<$5.0 Hz (hypoglycemia) \\
& Protein folding & 2.0--5.0 & 200--500 & Moderate & $<$1.0 Hz (stress) \\
& Mitochondrial cycling & 15.0--25.0 & 40--67 & Strong & $<$10.0 Hz (dysfunction) \\
\midrule
\multicolumn{6}{l}{\textit{Thermodynamic Processes}} \\
& Cardiac variance restoration & 3.0--10.0 & 100--333 & Strong & $<$2.0 Hz (depression) \\
& Neural gas equilibration & 5.0--15.0 & 67--200 & Strong & $<$3.0 Hz (coma risk) \\
& Gibbs free energy minimization & 2.0--8.0 & 125--500 & Moderate & $<$1.5 Hz (impaired) \\
& BMD frame selection & 3.0--7.0 & 143--333 & Strong & $<$2.0 Hz (no consciousness) \\
\bottomrule
\end{tabular}
\end{table*}

%% ====================================================================================
%% TABLE S6: Oxygen Coupling Parameters Across Environments
%% ====================================================================================

\begin{table}[h]
\centering
\caption{Atmospheric Oxygen Coupling Parameters and Process Rate Modulation}
\label{tab:oxygen_parameters}
\begin{tabular}{@{}lccccc@{}}
\toprule
\textbf{Environment} & \textbf{O$_2$ (\%)} & \textbf{$\mathbf{\kappa}$ (s$^{-1}$)} & \textbf{Rel. to Ambient} & \textbf{$\tau_{\text{restore}}$ (ms)} & \textbf{$Q$ Factor} \\
\midrule
\multicolumn{6}{l}{\textit{Terrestrial}} \\
Sea level & 20.9 & $4.7 \times 10^{-3}$ & 1.00 & 100--300 & 1.000 \\
1000 m altitude & 18.5 & $4.1 \times 10^{-3}$ & 0.87 & 115--345 & 0.912 \\
3000 m altitude & 14.3 & $3.2 \times 10^{-3}$ & 0.68 & 147--441 & 0.768 \\
5000 m altitude & 11.2 & $2.5 \times 10^{-3}$ & 0.53 & 189--566 & 0.612 \\
\midrule
\multicolumn{6}{l}{\textit{Aquatic}} \\
Shallow water (1 m) & 6.8 & $1.2 \times 10^{-6}$ & 0.00026 & 40000 & 0.045 \\
Deep water (10 m) & 5.2 & $0.9 \times 10^{-6}$ & 0.00019 & 52000 & 0.034 \\
\midrule
\multicolumn{6}{l}{\textit{Therapeutic}} \\
Hyperbaric 1.5 ATA & 31.4 & $5.8 \times 10^{-3}$ & 1.23 & 81--244 & 1.187 \\
Hyperbaric 2.5 ATA & 52.3 & $7.4 \times 10^{-3}$ & 1.57 & 63--190 & 1.421 \\
\midrule
\multicolumn{6}{l}{\textit{Pathological}} \\
Hypoxic (85\% sat) & 17.8 & $4.0 \times 10^{-3}$ & 0.85 & 118--353 & 0.887 \\
Severe hypoxic (70\%) & 14.6 & $3.3 \times 10^{-3}$ & 0.70 & 143--429 & 0.778 \\
\bottomrule
\end{tabular}
\end{table}

%% ====================================================================================
%% End of figures.tex
%% ====================================================================================
