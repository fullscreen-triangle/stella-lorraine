\section{Electrical Circuit Complementarity}

\subsection{Motivation}

The dual-membrane complementarity can be understood through a familiar analogy from electrical engineering: \emph{the complementarity of ammeter and voltmeter measurements}. This grounds the abstract concept in concrete measurement physics.

\subsection{The Ammeter/Voltmeter Constraint}

\begin{theorem}[Measurement Apparatus Complementarity]
An ammeter and voltmeter cannot be connected in series to simultaneously measure current and voltage at the same circuit point.
\end{theorem}

\begin{proof}
\textbf{Ammeter requirements:}
\begin{itemize}
\item Must be in \emph{series} with the circuit
\item Ideally, zero impedance: $Z_A \to 0$
\item Measures current: $I = \text{reading}$
\end{itemize}

\textbf{Voltmeter requirements:}
\begin{itemize}
\item It must be in \emph{parallel} across components.
\item Ideally infinite impedance: $Z_V \to \infty$
\item Measures voltage: $V = \text{reading}$
\end{itemize}

If both are placed in series:
\begin{equation}
Z_{\text{total}} = Z_A + Z_V \to \infty
\end{equation}

The circuit is effectively open, the current drops to zero, and the measurement fails. The configurations are \emph{mutually exclusive}.
\end{proof}

\subsection{Measurement vs. Derivation}

Although you cannot measure both simultaneously, you can measure one and \emph{derive} the other:

\subsubsection{Ammeter Mode (Direct Current Measurement)}

\begin{enumerate}
\item Connect ammeter in series: \verb|---[A]---[R]---|
\item \textbf{Directly measure}: $I$
\item \textbf{Calculate}: $V = I \times R$ (using Ohm's law)
\end{enumerate}

The voltage is \emph{derived}, not measured.

\subsubsection{Voltmeter Mode (Direct Voltage Measurement)}

\begin{enumerate}
\item Connect voltmeter in parallel: \verb|---|[V]|---[R]---|[V]|---|
\item \textbf{Directly measure}: $V$
\item \textbf{Calculate}: $I = V / R$ (using Ohm's law)
\end{enumerate}

The current is \emph{derived}, not measured.

\subsection{Mapping to Dual-Membrane}

\begin{center}
\begin{tabular}{|l|l|}
\hline
\textbf{Electrical Circuit} & \textbf{Dual-Membrane} \\
\hline
Ammeter (measures $I$) & Observe front face \\
Voltmeter (measures $V$) & Observe back face \\
Ohm's law: $V = IR$ & Conjugate transform: $\mathbf{S}_{\text{back}} = T(\mathbf{S}_{\text{front}})$ \\
Direct measurement & Observable face \\
Derived calculation & Hidden face (calculated) \\
Switch ammeter $\leftrightarrow$ voltmeter & Switch front $\leftrightarrow$ back \\
Cannot measure both & Complementarity constraint \\
\hline
\end{tabular}
\end{center}

\subsection{Dual-Membrane as Electrical Circuit}

We model the dual-membrane pixel demon as an electrical circuit:

\begin{definition}[Dual-Membrane Circuit]
A dual-membrane circuit consists of:
\begin{itemize}
\item \textbf{Observable components} (front face): resistors $\{R_1, R_2, \ldots, R_n\}$
\item \textbf{Hidden components} (back face): conjugate resistors $\{R_1^*, R_2^*, \ldots, R_n^*\}$
\item \textbf{Measurement mode}: ammeter (front) or voltmeter (back)
\end{itemize}
\end{definition}

For phase conjugate transformation:
\begin{equation}
R_i^* = -R_i
\end{equation}

(Negative resistance represents active components or phase-shifted impedance.)

\subsection{Circuit Balance}

\begin{theorem}[Kirchhoff's Laws for Dual Circuits]
The complete dual-membrane circuit (both faces) satisfies:
\begin{align}
\sum_{\text{nodes}} I_{\text{front}} + \sum_{\text{nodes}} I_{\text{back}} &= 0 \quad \text{(KCL)} \\
\sum_{\text{loop}} V_{\text{front}} + \sum_{\text{loop}} V_{\text{back}} &= 0 \quad \text{(KVL)}
\end{align}
\end{theorem}

\begin{proof}
For phase conjugate, $I_{\text{back}} = -I_{\text{front}}$ and $V_{\text{back}} = -V_{\text{front}}$:
\begin{align}
\sum I_{\text{front}} + \sum (-I_{\text{front}}) &= 0 \\
\sum V_{\text{front}} + \sum (-V_{\text{front}}) &= 0
\end{align}
Both laws are satisfied identically.
\end{proof}

The circuit is electrically balanced even though only one face is observable.

\begin{figure}[htbp]
    \centering
    \includegraphics[width=\textwidth]{figures/figure_5_circuit_complementarity.png}
    \caption{\textbf{Electrical circuit analogy for dual-membrane complementarity.}
    (\textbf{A}) Ammeter configuration: Ammeter (blue) inserted in series measures
    current $I$ directly. Low impedance ($Z \approx 0$) allows current flow.
    Voltage $V$ must be calculated via Ohm's law: $V = IR$.
    (\textbf{B}) Voltmeter configuration: Voltmeter (orange) connected in parallel
    measures voltage $V$ directly. High impedance ($Z \to \infty$) prevents current
    draw. Current $I$ must be calculated: $I = V/R$.
    (\textbf{C}) Measurement incompatibility: Ammeter and voltmeter cannot be placed
    in series simultaneously. Ammeter requires low impedance (all current flows),
    voltmeter requires high impedance (no current flows)—these requirements are
    mutually exclusive. Physical conflict prevents simultaneous direct measurement.
    (\textbf{D}) Mapping to dual-membrane: Ammeter mode $\leftrightarrow$ front face
    (observable); voltmeter mode $\leftrightarrow$ back face (hidden);
    Ohm's law $\leftrightarrow$ conjugate transformation; switching measurement
    apparatus $\leftrightarrow$ switching observable face. Same fundamental constraint:
    measurement apparatus determines which quantity is directly accessible.
    (\textbf{E}) Dual-membrane as electrical circuit: Front face (direct measurement
    of $S_k$, derived $T(S_k)$) and back face (direct measurement of $T(S_k)$,
    derived $S_k$) are related by conjugate transform, analogous to $V = IR$
    relationship. Cannot observe both simultaneously, just as ammeter and voltmeter
    cannot both be in series. This demonstrates that dual-membrane complementarity
    is not quantum-mechanical but reflects classical measurement incompatibility
    present in everyday electrical circuits.}
    \label{fig:circuit_complementarity}
    \end{figure}

\subsection{Measurement Incompatibility}

\begin{theorem}[Simultaneous Measurement Impossibility]
Attempting to directly measure both $\mathbf{S}_{\text{front}}$ and $\mathbf{S}_{\text{back}}$ simultaneously yields an error.
\end{theorem}

\begin{proof}
Direct measurement requires setting the measuring apparatus to a specific mode (ammeter or voltmeter, front or back). The apparatus state is a discrete variable:
\begin{equation}
\text{Mode} \in \{\text{FRONT}, \text{BACK}\}
\end{equation}

It cannot be in both states simultaneously. Any attempt to access both faces directly constitutes an invalid operation, analogous to connecting an ammeter and a voltmeter in series.
\end{proof}

\subsection{Circuit Representation Implementation}

\begin{algorithm}
\caption{Dual-Membrane Circuit Measurement}
\begin{algorithmic}[1]
\State \textbf{Input:} Circuit $C$, Mode $M \in \{\text{FRONT}, \text{BACK}\}$
\State \textbf{Output:} Measured components, Derived components
\State
\If{$M = \text{FRONT}$}
    \State // Direct measurement (ammeter mode)
    \State $\text{Components}_{\text{front}} \gets \text{MeasureObservable}(C)$
    \State $\text{Type}_{\text{front}} \gets \text{DIRECT}$
    \State
    \State // Derived calculation (Ohm's law)
    \State $\text{Components}_{\text{back}} \gets T(\text{Components}_{\text{front}})$
    \State $\text{Type}_{\text{back}} \gets \text{DERIVED}$
\Else
    \State // Direct measurement (voltmeter mode)
    \State $\text{Components}_{\text{back}} \gets \text{MeasureObservable}(C)$
    \State $\text{Type}_{\text{back}} \gets \text{DIRECT}$
    \State
    \State // Derived calculation (inverse transform)
    \State $\text{Components}_{\text{front}} \gets T^{-1}(\text{Components}_{\text{back}})$
    \State $\text{Type}_{\text{front}} \gets \text{DERIVED}$
\EndIf
\State
\State \Return $(\text{Components}_{\text{front}}, \text{Type}_{\text{front}}), (\text{Components}_{\text{back}}, \text{Type}_{\text{back}})$
\end{algorithmic}
\end{algorithm}

\subsection{Observable vs. Hidden Components}

At any time, some circuit components are directly observable, while others are hidden:

\begin{equation}
\text{Observable}(t) = \begin{cases}
\{R_1, R_2, \ldots, R_n\} & \text{if Mode} = \text{FRONT} \\
\{R_1^*, R_2^*, \ldots, R_n^*\} & \text{if Mode} = \text{BACK}
\end{cases}
\end{equation}

\begin{equation}
\text{Hidden}(t) = \begin{cases}
\{R_1^*, R_2^*, \ldots, R_n^*\} & \text{if Mode} = \text{FRONT} \\
\{R_1, R_2, \ldots, R_n\} & \text{if Mode} = \text{BACK}
\end{cases}
\end{equation}

The hidden components exist (the circuit requires them for balance) but cannot be directly measured in the current mode.

\subsection{Physical Significance}

This circuit analogy demonstrates that complementarity is not unique to quantum mechanics:

\begin{enumerate}
\item \textbf{Measurement Apparatus Determines Observable}: What you can measure depends on your apparatus configuration (ammeter vs. voltmeter, front vs. back).

\item \textbf{Both Quantities Exist}: Even though you cannot measure both simultaneously, current and voltage both exist in the circuit. Similarly, both categorical faces exist even though only one is observable.

\item \textbf{Complete Description Requires Both}: To fully characterize a circuit, you need both $I$ and $V$. To fully characterize a categorical state, you need both front and back faces.

\item \textbf{Derivation ≠ Measurement}: Calculating $V$ from $I \times R$ is not the same as measuring $V$ with a voltmeter. Calculating the back face from the front face transform is not the same as observing the back face directly.
\end{enumerate}

\subsection{Experimental Validation}

The circuit complementarity can be validated:

\begin{enumerate}
\item \textbf{Test 1}: Measure front face, calculate back face using $T$
\item \textbf{Test 2}: Switch to back face, measure directly, verify matches calculated value
\item \textbf{Test 3}: Attempt simultaneous measurement of both faces, verify error/impossibility
\item \textbf{Test 4}: Verify circuit balance (Kirchhoff's laws) using front + back components
\end{enumerate}

These tests confirm that the dual-membrane behaves exactly like an electrical circuit with ammeter/voltmeter complementarity.

\subsection{Generalization}

The ammeter/voltmeter constraint is one instance of a general principle:

\begin{theorem}[Measurement Apparatus Complementarity]
Any two observables that require mutually exclusive measurement apparatus configurations cannot be measured simultaneously.
\end{theorem}

Examples:
\begin{itemize}
\item Position/Momentum: Require different apparatus (quantum mechanics)
\item Current/Voltage: Require different apparatus (electrical engineering)
\item Front/Back Face: Require different apparatus mode (categorical dynamics)
\end{itemize}

This places categorical complementarity in the same category as other well-established complementarity principles in physics and engineering.
