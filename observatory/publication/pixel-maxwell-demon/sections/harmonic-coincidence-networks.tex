\section{Harmonic Coincidence Networks}

\subsection{Motivation}

Querying categorical states requires summing over molecular ensembles, which naively scales as $\mathcal{O}(N)$ where $N$ is the number of molecules. For atmospheric conditions ($N \sim 10^{25}$), this is computationally prohibitive. We introduce \emph{harmonic coincidence networks} that enable $\mathcal{O}(1)$ queries.

\subsection{Integer Frequency Ratios}

\begin{definition}[Harmonic Coincidence]
Two oscillators with frequencies $f_1$ and $f_2$ are in \emph{harmonic coincidence} if:
\begin{equation}
\frac{f_1}{f_2} = \frac{m}{n}, \quad m, n \in \mathbb{Z}^+, \quad \gcd(m,n) = 1
\end{equation}
\end{definition}

For atmospheric molecules at $T = 288$ K:

\begin{center}
\begin{tabular}{|l|c|c|}
\hline
\textbf{Pair} & $f_1/f_2$ & \textbf{Ratio} \\
\hline
O$_2$ / N$_2$ & $4.7 \times 10^{13} / 7.0 \times 10^{13}$ & $\approx 2/3$ \\
N$_2$ / H$_2$O & $7.0 \times 10^{13} / 1.1 \times 10^{14}$ & $\approx 7/11$ \\
O$_2$ / H$_2$O & $4.7 \times 10^{13} / 1.1 \times 10^{14}$ & $\approx 3/7$ \\
\hline
\end{tabular}
\end{center}

\subsection{Coincidence Network Construction}

\begin{definition}[Harmonic Coincidence Network]
A harmonic coincidence network $G = (V, E)$ is a graph where:
\begin{itemize}
\item Vertices $V$: oscillators (molecular species)
\item Edges $E$: harmonic coincidences within tolerance $\epsilon$
\end{itemize}

An edge exists between oscillators $i$ and $j$ if:
\begin{equation}
\left|\frac{f_i}{f_j} - \frac{m}{n}\right| < \epsilon \quad \text{for some } m, n \in \mathbb{Z}^+, m, n \leq N_{\max}
\end{equation}
\end{definition}

\subsection{Network-Based Query}

\begin{theorem}[Coincidence Network Query Complexity]
Given a harmonic coincidence network with $k$ components, a categorical state query has complexity:
\begin{equation}
\mathcal{O}(k) \text{ where } k \ll N
\end{equation}
\end{theorem}

\begin{proof}
The categorical state is determined by the network structure, not individual oscillators:
\begin{equation}
\mathbf{S} = F_{\text{network}}(\{n_i, f_i, \phi_i\}_{i=1}^{k})
\end{equation}

where $k$ is the number of molecular species (typically $k \approx 10$ for the atmosphere). Each species aggregates information from all its constituent molecules through:
\begin{equation}
n_i = \sum_{j \in \text{type}_i} 1, \quad \phi_i = \arg\left(\sum_{j \in \text{type}_i} e^{i\phi_j}\right)
\end{equation}

This aggregation is performed once during network initialization, so queries are $\mathcal{O}(k) \approx \mathcal{O}(1)$.
\end{proof}

\subsection{Information Density at Frequency}

\begin{definition}[Oscillator Information Density]
The information density at frequency $f$ is:
\begin{equation}
\rho(f) = \sum_{i: |f_i - f| < \Delta f} n_i \log_2(f_i / f_{\text{ref}})
\end{equation}
\end{definition}

This can be queried efficiently using the network structure:

\begin{algorithm}
\caption{Query Information Density by Frequency}
\begin{algorithmic}[1]
\State \textbf{Input:} Network $G$, Target frequency $f$, Bandwidth $\Delta f$
\State \textbf{Output:} Information density $\rho(f)$
\State
\State $\rho \gets 0$
\For{each vertex $v \in V(G)$}
    \If{$|f_v - f| < \Delta f$}
        \State $\rho \gets \rho + n_v \log_2(f_v / f_{\text{ref}})$
    \EndIf
\EndFor
\State
\State \Return $\rho$
\end{algorithmic}
\end{algorithm}

Complexity: $\mathcal{O}(|V|) = \mathcal{O}(k)$ where $k$ is the number of species.

\subsection{Phase Coherence Clusters}

\begin{definition}[Phase Cluster]
A phase cluster $C$ is a subset of oscillators with phase variance below threshold:
\begin{equation}
\text{Var}(\{\phi_i : i \in C\}) < \epsilon_{\text{phase}}
\end{equation}
\end{definition}

Phase clusters emerge naturally in harmonic coincidence networks:

\begin{theorem}[Harmonic Phase Locking]
Oscillators in harmonic coincidence tend to phase-lock over time:
\begin{equation}
\frac{d\phi_i}{dt} = \omega_i + K \sum_{j \sim i} \sin(\phi_j - \phi_i)
\end{equation}
where $j \sim i$ denotes harmonic coincidence.
\end{theorem}

This is the Kuramoto model applied to molecular oscillators.

\subsection{Cascade Network Enhancement}

The harmonic network structure enables cascaded observations:

\begin{definition}[Network Cascade]
In a cascade of depth $n$, each observation queries the network state, which depends on all previous observations:
\begin{equation}
\mathbf{S}_n = F_{\text{network}}(\mathbf{S}_0, \mathbf{S}_1, \ldots, \mathbf{S}_{n-1})
\end{equation}
\end{definition}

\begin{theorem}[Cascade Information Scaling]
The information gained from $N$ cascaded network queries is:
\begin{equation}
I_N = \sum_{n=1}^{N} (n+1)^2 I_0 = I_0 \frac{N(N+1)(2N+1)}{6}
\end{equation}
\end{theorem}

\begin{proof}
At cascade level $n$, the query accesses:
\begin{itemize}
\item Direct network state: $I_0$ bits
\item Reflections from $n$ previous states: $n I_0$ bits
\item Cross-correlations: $\binom{n}{2} I_0$ bits
\end{itemize}

Total at level $n$:
\begin{equation}
I_n = I_0 \left(1 + n + \binom{n}{2}\right) = I_0 \frac{(n+1)(n+2)}{2}
\end{equation}

Summing over $N$ levels:
\begin{equation}
I_N = \sum_{n=0}^{N-1} I_n = I_0 \sum_{n=0}^{N-1} \frac{(n+1)(n+2)}{2}
\end{equation}

This evaluates to the stated form.
\end{proof}

For $N = 50$ cascades with $I_0 = 1$ bit:
\begin{equation}
I_{50} = \frac{50 \times 51 \times 101}{6} = 42,925 \text{ bits}
\end{equation}

compared to linear scaling: $I_{50,\text{linear}} = 50$ bits, an enhancement of $858\times$.

\subsection{Network Sparsity}

\begin{theorem}[Atmospheric Network Sparsity]
For atmospheric molecules with tolerance $\epsilon = 10^{-3}$, the harmonic coincidence network has:
\begin{equation}
|E| \approx \mathcal{O}(k^2) \text{ where } k \ll N
\end{equation}
\end{theorem}

\begin{proof}
Each of $k$ species can connect to at most $\mathcal{O}(k)$ other species (those within harmonic coincidence tolerance). Therefore:
\begin{equation}
|E| \leq k \times k = k^2
\end{equation}

For atmospheres with $k = 10$ species:
\begin{equation}
|E| \leq 100 \text{ edges}
\end{equation}

This is vastly smaller than the $N \sim 10^{25}$ molecules represented.
\end{proof}

\begin{figure}[htbp]
\centering
\includegraphics[width=0.9\textwidth]{figures/moriarty_publication_figure.png}
\caption{\textbf{Dual-membrane image representation: ``Moriarty'' case study.}
(\textbf{a}) Original photograph: Italian Greyhound ``Moriarty'', professional
canine model, photographed in outdoor setting (Croatia). Subject exhibits
direct eye contact, alert posture, and professional modeling behavior.
(\textbf{b}) Grayscale conversion: Input to dual-membrane analysis. Intensity
values $I \in [0, 255]$ normalized to $[0, 1]$ before $S_k$ transformation.
(\textbf{c}) Front face $S_k$ coordinates: Observable membrane state derived
from pixel intensities via Eq.~\ref{eq:sk_from_intensity}. Blue regions
(negative $S_k$) correspond to darker image areas, red regions (positive $S_k$)
to lighter areas. Spatial structure encodes image content in categorical
coordinate space.
(\textbf{d}) Back face $S_k$ coordinates: Conjugate membrane state computed
as $S_k^{\text{back}} = -S_k^{\text{front}}$. Color pattern is inverted
relative to front face, reflecting phase conjugate relationship.
(\textbf{e}) Conjugate relationship verification: Scatter plot of all
$128 \times 128 = 16,\!384$ pixel pairs. Perfect linear anti-correlation
($r = -1.0000$, red line) confirms conjugate constraint
$S_k^{\text{back}} = -S_k^{\text{front}}$ holds exactly for every pixel.
(\textbf{f}) Conjugate sum spatial distribution:
$S_k^{\text{front}} + S_k^{\text{back}}$ across image. Uniform white
(zero sum) with deviations $< 10^{-7}$ validates information conservation
and demonstrates that conjugate relationship is maintained spatially.
This figure demonstrates the complete dual-membrane representation of a
real-world image, confirming theoretical predictions: (1) each pixel has
two conjugate states, (2) states obey exact anti-correlation, (3) information
is conserved between faces, (4) spatial structure is preserved in categorical
coordinates.}
\label{fig:moriarty_publication}
\end{figure}

\subsection{Real-Time Query Performance}

\begin{theorem}[Real-Time Network Query]
A categorical state query on a harmonic coincidence network can be performed in real-time:
\begin{equation}
t_{\text{query}} = \mathcal{O}(k) \times t_{\text{op}} < 1 \mu\text{s}
\end{equation}
for typical computational systems.
\end{theorem}

\begin{proof}
With $k = 10$ species and modern processors performing $t_{\text{op}} \sim 1$ ns per operation:
\begin{equation}
t_{\text{query}} = 10 \times 1 \text{ ns} = 10 \text{ ns} \ll 1 \mu\text{s}
\end{equation}
\end{proof}

This enables real-time categorical observation at kilohertz or higher rates.

\subsection{Network-Accelerated Image Processing}

For a pixel demon grid of size $N_x \times N_y$:

\begin{theorem}[Grid Query Complexity]
Computing the categorical state for all pixels has complexity:
\begin{equation}
\mathcal{O}(N_x \times N_y \times k)
\end{equation}
independent of molecular count.
\end{theorem}

For a $1024 \times 1024$ image with $k = 10$ species:
\begin{equation}
\text{Operations} = 1024^2 \times 10 \approx 10^7
\end{equation}

At 1 ns/operation: $t_{\text{frame}} \approx 10$ ms, enabling $\sim 100$ fps real-time processing.

\subsection{Network Topology and Information Flow}

\begin{definition}[Information Path]
An information path in network $G$ is a sequence of edges:
\begin{equation}
P = (v_1, v_2), (v_2, v_3), \ldots, (v_{n-1}, v_n)
\end{equation}
where each edge represents harmonic coincidence.
\end{definition}

\begin{theorem}[Path Information Transfer]
Information flows along paths with efficiency:
\begin{equation}
\eta(P) = \prod_{(v_i, v_{i+1}) \in P} \cos(\Delta\phi_{i,i+1})
\end{equation}
where $\Delta\phi_{i,i+1}$ is the phase difference.
\end{theorem}

Phase-locked paths ($\Delta\phi \approx 0$) have $\eta \approx 1$, enabling efficient information transfer.


\begin{figure}[htbp]
    \centering
    \includegraphics[width=0.9\textwidth]{figures/figure_2_grid_patterns_real.png}
    \caption{\textbf{Spatial conjugate patterns in dual-membrane grid using real image data.}
    Small-scale demonstration ($8 \times 8$ pixel grid) extracted from ``Moriarty'' photograph
    to illustrate pixel-level conjugate structure.
    (\textbf{A}) Front face $S_k$ coordinates showing spatial gradient from negative values
    (blue, upper-left) to positive values (red, lower-right). Mean $\mu = 0.506$,
    standard deviation $\sigma = 0.177$.
    (\textbf{B}) Back face $S_k$ coordinates displaying inverted gradient (red upper-left,
    blue lower-right), confirming conjugate relationship. Mean $\mu = -0.506$,
    $\sigma = 0.177$ (identical magnitude, opposite sign).
    (\textbf{C}) Conjugate sum $S_k^{\text{front}} + S_k^{\text{back}}$ across grid.
    Uniform pale yellow indicates near-perfect cancellation: $\mu_{\text{sum}} = 0.000$,
    $\sigma_{\text{sum}} = 0.000$ (to displayed precision). Maximum deviation $< 0.04$,
    demonstrating pixel-by-pixel conjugate constraint.
    (\textbf{D}) Difference map $S_k^{\text{front}} - S_k^{\text{back}}$ showing
    amplified signal (mean $\mu = 1.012$, $\sigma = 0.354$). Spatial structure preserved
    with enhanced contrast, confirming that conjugate faces contain identical information
    with opposite sign. This small-scale analysis validates that conjugate relationship
    holds locally at individual pixel level, not merely as global statistical property.}
    \label{fig:grid_real_data}
    \end{figure}

\subsection{Application to Dual-Membrane}

Each face of the dual membrane has its own harmonic network:

\begin{align}
G_{\text{front}} &= (V_f, E_f) \\
G_{\text{back}} &= (V_b, E_b)
\end{align}

The conjugate transformation maps networks:
\begin{equation}
G_{\text{back}} = T_G(G_{\text{front}})
\end{equation}

For phase conjugation, frequencies are preserved, but phases are inverted:
\begin{equation}
T_G: (V, E, \{\phi_i\}) \mapsto (V, E, \{-\phi_i\})
\end{equation}

This preserves network topology while inverting phase relationships.
