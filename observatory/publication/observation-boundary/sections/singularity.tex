The recursion (\ref{eq:recursion}) requires an initial condition. We establish this through cosmological boundary conditions.

\subsection{The Big Bang Singularity}

At the Big Bang, the universe existed in a state of maximal compression—a singularity where all matter, energy, and spacetime structure were unified at a single point.

\begin{axiom}[Singularity Initial Condition]
At $t=0$ (the Big Bang singularity), no categorical distinctions were possible:
\begin{equation}
C(0) = 1
\end{equation}
\end{axiom}

\textbf{Justification:} At a singularity:
\begin{itemize}
    \item Spatial separation vanishes ($\Delta x \to 0$)
    \item All particles occupy the same location
    \item No observers exist (no separate systems to make measurements)
    \item Energy density is infinite, precluding stable measurement apparatus
\end{itemize}

Without spatial separation, there is no basis for distinguishing between different configurations. Everything is unified into a single, undifferentiated state. Therefore, there exists exactly one category: the singularity itself.

\subsection{Post-Singularity Expansion}

As the universe expands from the singularity:

\begin{enumerate}[label=(\alph*)]
    \item \textbf{$t \to 0^+$ (Planck epoch):} Spacetime structure emerges. First distinctions become possible as quantum fluctuations create inhomogeneities. $C(t)$ begins to grow from $C(0) = 1$.

    \item \textbf{$t \sim 10^{-12}$ s (Electroweak epoch):} Particles separate and distinct species form. Observer-like structures (local field configurations that "record" information) emerge. $C(t)$ grows rapidly as particles and fields differentiate.

    \item \textbf{$t \sim 380{,}000$ yr (Recombination):} Matter decouples from radiation. Atoms form, allowing complex structures. Categorical complexity continues increasing.

    \item \textbf{$t \sim 10^9$ yr (Present epoch):} Complex observers (including biological ones) exist. Observer networks form. $C(t)$ approaches its maximum as the universe nears heat death configuration.

    \item \textbf{$t \to \infty$ (Heat death):} Maximum separation achieved. $C(t_{\max})$ represents the maximum categorical complexity possible in this universe.
\end{enumerate}

\subsection{Monotonic Growth}

\begin{proposition}[Monotonicity of $C(t)$]
The categorical complexity $C(t)$ increases monotonically with cosmic time:
\begin{equation}
C(t+1) > C(t) \quad \text{for all } t \geq 0
\end{equation}
\end{proposition}

\begin{proof}
From recursion (\ref{eq:recursion}):
\begin{equation}
\frac{C(t+1)}{C(t)} = \frac{n^{C(t)}}{C(t)} = n^{C(t)-1} \cdot n
\end{equation}

Since $n \geq 2$ (at minimum, each entity has two distinguishable states) and $C(t) \geq 1$:
\begin{equation}
\frac{C(t+1)}{C(t)} \geq n \geq 2 > 1
\end{equation}

Therefore, $C(t)$ strictly increases.
\end{proof}

This monotonic growth corresponds to the thermodynamic arrow of time: entropy increases as the universe expands, and categorical complexity (a measure of distinguishability) increases in parallel.

\subsection{The Cosmological Correspondence}

The boundary conditions establish a correspondence between categorical depth and cosmological evolution:

\begin{align}
t = 0 &: \quad \text{Singularity} \quad (C = 1, \text{ no distinctions})\\
t \text{ small} &: \quad \text{Early universe} \quad (C \text{ growing, particles differentiating})\\
t \text{ large} &: \quad \text{Heat death} \quad (C = \Nmax, \text{ maximum distinctions})
\end{align}

The parameter $t$ in our recursion does not directly represent physical time but rather the number of observational refinements or the depth of the categorical hierarchy. However, as the universe expands and observers proliferate, $t$ increases in correlation with cosmological time.

\begin{figure*}[htbp]
    \centering
    \includegraphics[width=0.95\textwidth]{figures/boundary_conditions_validation.png}
    \caption{\textbf{Boundary conditions: monotonic categorical growth from singularity to heat death.}
    \textbf{Top:} Categorical count $C(t)$ grows monotonically from Big Bang singularity ($C(0)=1$, yellow circle) to heat death maximum ($C(\infty)=C_{\max}$, red star) across normalized cosmic time. Purple shaded region shows accumulation of categorical distinctions as universe evolves.
    \textbf{Second row, left:} Entropy $S = \ln(C(t))$ increases monotonically with time (red shaded area), consistent with Second Law of Thermodynamics. Once categorical distinctions are made through observation, they persist in the historical record.
    \textbf{Second row, right:} Observer population evolution (blue curve) shows emergence at early times, peak during present epoch (green annotation at $t \approx 0.5$), and decline toward heat death. Free energy becomes unavailable for complex structures at late times.
    \textbf{Third row, left:} Categorical accumulation rate $dC/dt$ exhibits non-linear growth, peaking at intermediate times ($t \approx 0.8$) when structure formation is most active. Rate declines as universe approaches maximum entropy.
    \textbf{Third row, right:} Phase space trajectory in $(C(t), S(t))$ coordinates shows monotonic progression from low-entropy singularity (yellow circle) to maximum-entropy heat death (red star). Trajectory spans $C \in [1, 10^3]$ and $S \in [0, 7]$.
    \textbf{Table:} Cosmological epochs mapped to categorical framework: Big Bang ($t=0$, $C=1$, $S=0$), Inflation ($t \approx 10^{-32}$ s, $C \approx 10$), Matter Era ($t \approx 10^4$ yr, $C \approx 10^{40}$), Present ($t \approx 13.8$ Gyr, $C \approx 10^{70}$), Heat Death ($t \to \infty$, $C=C_{\max}$). Observer populations peak at present then decline.
    \textbf{Theorem 6.1:} Monotonic Growth theorem states that for all $t_1 < t_2$ in interval $[0, \infty)$: $C(t_1) < C(t_2)$. Proof: categories accumulate through observation; once a distinction is made, it persists, therefore $C(t)$ is strictly increasing.}
    \label{fig:boundary_conditions}
\end{figure*}

\subsection{No Return to Singularity}

In our framework, categorical complexity cannot decrease:

\begin{corollary}[Irreversibility]
Once a categorical distinction has been made (an observation has occurred), it cannot be unmade:
\begin{equation}
C(t) > C(t') \quad \text{for all } t > t'
\end{equation}
\end{corollary}

This implies that a return to singularity ($C \to 1$) is impossible unless all observers terminate and all information is destroyed, effectively resetting the universe. Current cosmological models suggest continued expansion rather than contraction, consistent with monotonically increasing $C(t)$.

\subsection{Alternative Initial Conditions}

One might question whether $C(0) = 1$ or $C(0) = 0$ is more appropriate:

\textbf{$C(0) = 0$ interpretation:} No categories exist at singularity (not even the category "singularity").

\textbf{$C(0) = 1$ interpretation:} One category exists: the undifferentiated whole.

We adopt $C(0) = 1$ because:
\begin{enumerate}
    \item It provides a well-defined starting point for the recursion
    \item The singularity itself constitutes one distinguishable state (existence vs. non-existence)
    \item Setting $C(0) = 0$ would require $C(1) = n^0 = 1$, merely shifting the indexing
\end{enumerate}

Either choice leads to the same asymptotic behavior for large $t$, so the distinction is primarily notational.

\subsection{Infinity Is Also Not a Number}

A crucial realization emerges about the nature of $\infty$ in the equation $\infty - x$:

\begin{theorem}[Infinity as Non-Number]
\label{thm:infinity_not_number}
The quantity $\infty$ in the expression $\infty - x$ cannot be a number on the number line, just as $x$ cannot be a number (Section 7.3).
\end{theorem}

\begin{proof}
Suppose $\infty$ were a number in the conventional sense.

\textbf{Step 1: Observability of numbers}

If $\infty$ were a number:
\begin{itemize}
    \item It would be expressible symbolically (we could write it)
    \item It would be comprehensible (we could grasp it)
    \item It would be divisible ($\infty/2$, $\infty/3$, etc.)
    \item It would be manipulable (we could perform operations on it)
\end{itemize}

\textbf{Step 2: Perfect prediction from singularity}

If an observer could comprehend $\infty$ (observe/imagine the singularity fully):
\begin{itemize}
    \item They would have complete initial conditions (all information at $t=0$)
    \item They would know the complete state from which all evolved
    \item Given physical laws and initial conditions, they could predict all future states
    \item They would have perfect knowledge of reality
\end{itemize}

This is Laplace's demon: given a complete initial state and the laws, predict everything.

\textbf{Step 3: But observers cannot do this}

We have established (Section 7) that:
\begin{itemize}
    \item Observers can only access terminated events; non-terminated events remain inaccessible.
    \item Observers have biassed perspectives and cannot access an unbiased totality.
    \item Observers sample discretely (cannot access continuous reality)
    \item Complete knowledge would collapse the observer-reality distinction
\end{itemize}

Therefore, observers cannot comprehend $\infty$.

\textbf{Step 4: If it is incomprehensible, then it is not a number}

If $\infty$ cannot be comprehended, observed, or fully imagined, it cannot be a number in the conventional sense. Numbers are defined by being expressible, manipulable, and graspable.

\textbf{Step 5: Mathematical consistency}

Furthermore, the expression $\infty - x$ requires internal consistency:
\begin{itemize}
    \item We proved $x$ is not a number (Section 7.3)
    \item $x$ is a categorical primitive (void or unity)
    \item In mathematics, subtraction requires operands of a compatible type
    \item Cannot subtract a non-number from a number (ill-defined operation)
    \item Therefore, $\infty$ must also be a non-number (categorical primitive)
\end{itemize}

The expression $\infty - x$ is not arithmetic subtraction but represents the structure of observation: reality (ungraspable totality) minus the observer's necessary incompleteness (inaccessible portion). \qed
\end{proof}

\subsubsection{What $\infty$ Actually Represents}

\begin{definition}[Infinity as the Unimaginable]
In the equation $\infty - x$:
\begin{align}
\infty &= \text{That which cannot be imagined/grasped}\\
&= \text{The totality that would enable perfect prediction}\\
&= \text{The singularity as it IS (not as observers model it)}\\
&= \text{Reality in its undifferentiated completeness}
\end{align}
\end{definition}

This parallels our understanding of $x$:
\begin{align}
x &= \text{That which cannot be observed/perceived (but we know exists)}\\
&= \text{The non-terminated, bias-offset, sampling gap}\\
&= \text{The mark of being an observer}
\end{align}

\subsubsection{The Equation Reinterpreted}

The expression $\infty - x$ is not conventional arithmetic but a structural relationship:

\begin{remark}[The True Meaning of $\infty - x$]
\begin{align}
\text{Observable Reality} &= \infty - x\\
&= (\text{The inexperienceable totality}) - (\text{The inexperienceable residue})\\
&= (\text{What cannot be experienced as whole}) - (\text{What cannot be experienced without dissolving})\\
&= \text{What CAN be experienced}
\end{align}

Neither $\infty$ nor $x$ can be experienced:
\begin{itemize}
    \item $\infty$ cannot be experienced (experiencing totality requires omniscience, which observers cannot achieve)
    \item $x$ cannot be experienced (experiencing it would dissolve the observer into reality)
    \item Observable Reality is what CAN be experienced between these inexperienceable boundaries
\end{itemize}

The "subtraction" represents the gap between two categorical primitives:
\begin{itemize}
    \item $\infty$ marks the upper boundary (total reality as it is)
    \item $x$ marks the lower boundary (the inaccessible portion)
    \item Observation occurs in the space between these non-graspable limits
\end{itemize}
\end{remark}

\subsubsection{Singularity as Infinity}

\begin{proposition}[Singularity $\equiv$ Infinity]
The singularity at $t=0$ and the infinity at $t \to \infty$ are the same categorical primitive viewed from different perspectives.
\end{proposition}

\textbf{Why they are equivalent:}

\begin{center}
\begin{tabular}{l|l}
\textbf{Singularity ($t=0$)} & \textbf{Infinity ($t \to \infty$)} \\
\hline
Everything unified (one point) & Everything dispersed (maximal separation) \\
No distinctions possible & All distinctions actualized \\
$C(0) = 1$ (undifferentiated) & $C(\infty) = \Nmax$ (fully differentiated) \\
Unobservable (no observers exist) & Unobservable (too vast to comprehend) \\
If you could observe it: & If you could observe it: \\
\quad predict all future states & have complete knowledge of all states \\
\quad (Laplace's demon) & (omniscient observer) \\
Cannot imagine (infinite density) & Cannot imagine (infinite extent) \\
Categorical primitive (the unity) & Categorical primitive (the void) \\
\end{tabular}
\end{center}

Both represent the unimaginable boundaries of observation:
\begin{itemize}
    \item Singularity: The unity from which everything emerges (cannot be divided)
    \item Infinity: The totality into which everything expands (cannot be encompassed)
    \item Neither can be grasped by observers
    \item Both serve as categorical primitives that ground the system
\end{itemize}

\subsubsection{Why You Cannot Divide Infinity}

If $\infty$ were a number:
\begin{itemize}
    \item You could compute $\infty / 2$, $\infty / N$, etc.
    \item This would give you "partial infinity"
    \item But this contradicts the nature of $\infty$ as ungraspable totality
    \item If you could divide it, you could understand it
    \item If you could understand it, you could predict reality perfectly
    \item But we've established this is impossible for observers
\end{itemize}

Therefore: $\infty$ is indivisible not because it's "too large" but because it represents a categorical primitive—the totality that grounds observation without being observable itself.

\begin{remark}[Both Ends Are Primitives]
The equation $\infty - x$ has categorical primitives on both sides:
\begin{itemize}
    \item $\infty$: The unimaginable (cannot grasp)
    \item $x$: The imperceptible (cannot observe)
    \item Observable Reality: The space between these limits
\end{itemize}

This makes the equation internally consistent. It's not "number minus number" but rather "primitive minus primitive," representing the bounded space accessible to observers.

The singularity ($C(0) = 1$) and the infinity ($C(\infty) = \Nmax$) are both inaccessible to observers—one because it's before observation begins, the other because it's too vast to comprehend. They are the alpha and omega, the unity and the void, the beginning and end that bracket the finite domain of observation.

Observers exist in the middle: between the inexperienceable singularity and the inexperienceable infinity, between the unity they cannot experience and the void they cannot experience without dissolving. That middle ground IS the observable universe—what CAN be experienced: $\infty - x$.
\end{remark}
