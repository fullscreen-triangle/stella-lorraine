To count categorical distinctions, we must first establish what constitutes a "distinction." In this work, we adopt an operational definition: a categorical distinction exists if and only if an observer can differentiate between two configurations through measurement or observation.

\subsection{The Role of Observers}

\begin{definition}[Observer]
An \emph{observer} is a physical system capable of making measurements that distinguish between different configurations of other systems. Formally, an observer $O$ possesses:
\begin{enumerate}[label=(\roman*)]
    \item A measurement apparatus with finite resolution
    \item A memory system to record measurement outcomes
    \item An internal state that changes in response to observations
\end{enumerate}
\end{definition}

Observers are necessary for categorical distinction because physical configurations do not inherently partition themselves. A molecule in one vibrational state versus another represents two "different" configurations only if some system can distinguish between them. Without observers, there are no categories—only undifferentiated physical reality.

\subsection{Observation Requires Termination}

A crucial constraint governs observation: to observe a system's state, the observation process must complete (terminate). This is not merely a practical limitation but follows from information theory.

\begin{proposition}[Observation Termination]
Let $O$ be an observer attempting to measure system $S$. The measurement produces a definite outcome only when the interaction between $O$ and $S$ is complete, establishing a correlated state $O \otimes S$ in which $O$'s memory encodes information about $S$.
\end{proposition}

Incomplete or ongoing processes cannot be observed because they do not yet have definite outcomes. This has important consequences: if a process never terminates, it remains unobservable, contributing no categorical distinctions to our count.

\begin{figure*}[htbp]
    \centering
    \includegraphics[width=0.95\textwidth]{figures/observer_dependent_structure.png}
    \caption{\textbf{Observer-dependent categorical structure and $\infty - x$ framework.}
    \textbf{Top:} Observable $\infty - x$ (blue curve with circles) increases from $\approx 0$ to $\approx 1.2$ while inaccessible $x$ (red curve with squares) decreases from $\approx 1.0$ to $\approx 0$ as observer count grows from 0 to 50. Curves cross at $\approx 20$ observers where observable and inaccessible fractions are equal.
    \textbf{Middle-left:} Ratio $x/(\infty - x)$ (purple curve with triangles) versus number of observers shows rapid decay from $\approx 10$ at 5 observers to $\approx 0.5$ at 30 observers. Green dashed line marks observed dark matter ratio $\approx 5.4$; intersection occurs at $\approx 10$ observers.
    \textbf{Middle-right:} Convergence analysis shows $|\text{Ratio} - 5.4|$ on log scale versus number of observers. Error drops from $\approx 10^3$ at 5 observers to minimum $\approx 10^{-1}$ at $\approx 10$ observers (red circles), then stabilizes.
    \textbf{Bottom-left:} Observer network diagram shows 5 observers O1-O5 (blue circles) connected by dashed lines representing mutual observation. Recursive constraint: observers must observe observers; no single observer accesses complete information.
    \textbf{Bottom-right:} Key insights box (cyan) summarizes $\infty - x$ structure: Observable $= \infty - x$, Inaccessible $= x$, Ratio $x/(\infty - x) = 5.4$. Physical correspondence to dark matter:ordinary matter ratio $= 5.4:1$ emerges from counting; correspondence presented without claiming causation.
    \textbf{Bottom banner:} Orange disclaimer states analysis is purely combinatorial. Categorical distinctions counted and emergent ratios reported; physical interpretation left to domain specialists.}
    \label{fig:observer_dependent}
\end{figure*}

\subsection{Partial Information and Observer Networks}

No single observer can access complete information about a macroscopic system. Each observer has:
\begin{itemize}
    \item \textbf{Finite spatial range:} Cannot observe arbitrarily distant regions
    \item \textbf{Finite temporal range:} Finite lifetime bounds observation duration
    \item \textbf{Finite resolution:} Cannot distinguish arbitrarily fine differences
\end{itemize}

To reconstruct complete system information, it is necessary to integrate observations from multiple observers. However, other observers are themselves physical systems that must be observed. This creates recursive structure: to know the complete state requires observing all observers, including the observer doing the observing.

\begin{definition}[Observer Network]
An \emph{observer network} $\mathcal{N} = \{O_1, O_2, \ldots, O_N\}$ is a collection of $N$ observers, each of which observes:
\begin{enumerate}[label=(\roman*)]
    \item Some portion of the physical system
    \item Some subset of other observers in $\mathcal{N}$
\end{enumerate}
\end{definition}

Complete information reconstruction requires accounting for all information distributed across the observer network, including information about which observers observed which parts of the system.

\subsection{Categories as Observer-Dependent Distinctions}

We formalize categorical distinctions as follows:

\begin{definition}[Category]
Given an observer network $\mathcal{N}$ at time $t$, a \emph{category} $C$ is an equivalence class of physical configurations that cannot be distinguished by any observer in $\mathcal{N}$ given their collective observations up to time $t$.
\end{definition}

Two configurations belong to the same category if no observer in the network can tell them apart. They belong to different categories if at least one observer can distinguish them.

The number of categories $C(t)$ at time $t$ equals the number of distinguishable equivalence classes. This number depends on:
\begin{itemize}
    \item The physical configuration of the system
    \item The number and capabilities of observers
    \item The history of observations made
\end{itemize}

Crucially, categories proliferate as observers proliferate and as the history of observation lengthens, because new distinctions become possible. This proliferation follows mathematical rules we derive in subsequent sections.
