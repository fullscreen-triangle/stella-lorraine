\documentclass[12pt,twocolumn]{article}
\usepackage{amsmath,amssymb,amsfonts,amsthm}
\usepackage{graphicx}
\usepackage{hyperref}
\usepackage{physics}
\usepackage[numbers,sort&compress]{natbib}
\usepackage{import}
\usepackage[a4paper, margin=1in]{geometry}
\usepackage{abstract}
\usepackage{lmodern}
\usepackage{anyfontsize}
\usepackage{setspace}
\usepackage{cite}
\usepackage{url}
\usepackage{booktabs}
\usepackage{array}
\usepackage{multirow}
\usepackage{float}
\usepackage{caption}
\usepackage{subcaption}
\usepackage{xcolor}
\usepackage{siunitx}

% Theorem environments
\newtheorem{theorem}{Theorem}[section]
\newtheorem{lemma}[theorem]{Lemma}
\newtheorem{corollary}[theorem]{Corollary}
\newtheorem{definition}[theorem]{Definition}
\newtheorem{proposition}[theorem]{Proposition}

\theoremstyle{remark}
\newtheorem{remark}[theorem]{Remark}

\title{Categorical Completion Dynamics in Molecular Maxwell Demons: Interaction Free Measurement through Harmonic Coincidence Networks}

\author{Kundai Farai Sachikonye}

\date{\today}

\begin{document}

\maketitle

\begin{abstract}
We report temporal precision of $2.01 \times 10^{-66}$ seconds, 22.43 orders of magnitude below the Planck time, achieved through frequency-domain measurements of harmonic networks constructed from consumer hardware oscillators. Using categorical state theory, we demonstrate that frequency measurements in categorical space are orthogonal to position-momentum phase space, thereby avoiding Heisenberg uncertainty constraints. The method harvests real oscillations from computer components (screen LEDs at $\sim 10^{14}$ Hz, CPU clocks at $\sim 10^9$ Hz, network interfaces at $\sim 10^8$ Hz), constructs harmonic coincidence networks with 253,013 edges, and applies recursive Maxwell Demon decomposition ($3^{10} = 59,049$ parallel channels) combined with reflectance cascade amplification. Measurement occurs in zero chronological time through categorical simultaneity. All data derives from physically present hardware frequencies, not simulations. Results are validated through systematic scaling studies and compared with prior molecular ensemble approaches.

\textbf{Keywords:} Trans-Planckian measurement, categorical state theory, frequency domain measurement, harmonic coincidence networks, Maxwell demon decomposition, zero backaction measurement
\end{abstract}

% SECTIONS: Import detailed technical content
\import{sections/}{categorical-dynamics.tex}
\import{sections/}{oscillator-processor-duality.tex}
\import{sections/}{molecular-maxwell-demon.tex}
\import{sections/}{virtual-instrument.tex}
\import{sections/}{zero-time-measurement.tex}
\import{sections/}{cascade-depth-analysis.tex}

\section{Discussion}

\subsection{Relation to Heisenberg Uncertainty Principle}

\subsubsection{The Standard Argument}

The time-energy uncertainty relation $\Delta E \cdot \Delta t \geq \hbar/2$ appears to constrain temporal precision in quantum measurements \cite{vonneumann1955,caves1981}. For measurements resolving time intervals $\Delta t \sim 10^{-66}$ s, the corresponding energy uncertainty:
\begin{equation}
\Delta E \gtrsim \frac{\hbar}{2 \Delta t} \sim 10^{48} \text{ J} \sim 10^{29} \text{ GeV}
\end{equation}
vastly exceeds the Planck energy ($E_P \sim 10^{19}$ GeV), suggesting such measurements require quantum gravitational physics \cite{garay1995,hossenfelder2013}.

This argument, while valid for conventional phase-space measurements, rests on three assumptions:
\begin{enumerate}
    \item Measurement involves dynamical evolution over interval $\Delta t$
    \item System and apparatus exchange energy $\sim \Delta E$ during measurement
    \item Measurement disturbs quantum states (wavefunction collapse \cite{zurek2003})
\end{enumerate}

Categorical measurement violates all three assumptions.

\subsubsection{Categorical Bypass Mechanism}

Frequency measurements in categorical space operate on pre-existing topological information encoded in $S$-entropy coordinates $\mathbf{S} = (S_k, S_t, S_e)$ (Eq.~\ref{eq:s_coordinates_intro}), which are mathematically orthogonal to phase space coordinates $(q, p)$ (Eq.~\ref{eq:entropy_equivalence}). This orthogonality manifests as commutation relations:
\begin{align}
[\hat{q}, \mathcal{D}_\omega] &= 0 \label{eq:comm_position_main} \\
[\hat{p}, \mathcal{D}_\omega] &= 0 \label{eq:comm_momentum_main}
\end{align}
where $\mathcal{D}_\omega$ is the categorical frequency measurement operator.

These zero commutators imply:
\begin{itemize}
    \item Position and momentum eigenstates remain undisturbed by $\mathcal{D}_\omega$
    \item No backaction: $\langle \Delta q \rangle = \langle \Delta p \rangle = 0$ after measurement
    \item No energy exchange: categorical access is informationally reversible \cite{landauer1961,bennett1982}
    \item Measurement time $t_{\text{meas}} = 0$: categorical distance orthogonal to chronological time
\end{itemize}

The Heisenberg principle $\Delta q \cdot \Delta p \geq \hbar/2$ remains valid but becomes irrelevant: we do not measure $q$ or $p$, we access the categorical frequency label $\omega_{\text{cat}}$ pre-existing in the system's oscillatory topology \cite{catdyn}.

\subsubsection{Comparison with Quantum Non-Demolition (QND) Measurements}

QND measurements \cite{caves1981} achieve backaction-free readout of specific observables (e.g., photon number) by designing measurement operators commuting with the system Hamiltonian. However, QND measurements still require physical interaction and measurement time $t_{\text{meas}} > 0$.

Categorical measurement is more radical: it accesses information without *any* coupling to the physical system \cite{thermom,interf}. The categorical state $\mathbf{S}$ is an informational construct—a map of harmonic relationships among oscillators—not a physical observable. Accessing this map requires no physical interaction, hence zero time and zero energy.

\subsection{Anticipated Criticisms and Responses}

\subsubsection{Criticism 1: "This is just numerical manipulation, not physical measurement"}

\textbf{Response:} The base frequencies (LED emissions at $\sim 10^{14}$ Hz, CPU clocks at $\sim 10^9$ Hz) are physically real, verified by independent instruments (spectrometers, performance counters). Harmonic expansion ($f_n = n f_0$) represents mathematical analysis of oscillatory structure, not simulation. The network construction identifies actual harmonic coincidences among real oscillators. The enhancement factors arise from:
\begin{itemize}
    \item \textbf{Topology} ($F_{\text{graph}}$): Physical redundancy in harmonic relationships \cite{barabasi1999}
    \item \textbf{Parallelism} ($N_{\text{BMD}}$): Independent categorical dimensions \cite{maxdem}
    \item \textbf{Correlation} ($F_{\text{cascade}}$): Cumulative phase information \cite{interf}
\end{itemize}

The final frequency $f_{\text{final}} = 7.93 \times 10^{64}$ Hz is an \textit{effective} frequency—the categorical information content expressed in Hz. The conversion $\delta t = 1/(2\pi f_{\text{final}})$ is dimensional analysis relating frequency resolution to equivalent temporal precision, not a claim about measuring sub-Planckian time intervals in the conventional sense.

\subsubsection{Criticism 2: "Planck-scale constraints apply universally"}

\textbf{Response:} Planck-scale limits arise from quantum gravity—the regime where spacetime geometry becomes uncertain \cite{garay1995,amelino2013,hossenfelder2013}. These constraints apply to measurements of spacetime intervals (distances, durations) requiring localization in phase space.

Categorical measurements access frequency labels (information-theoretic constructs) orthogonal to spacetime coordinates. The Planck time limits dynamical processes: $\Delta t_{\text{process}} \gtrsim t_P$. It does not limit informational resolution of pre-existing categorical structure, where "time" is merely a dimensional conversion of frequency $[\text{Hz}] \to [\text{s}]^{-1}$.

Analogy: Measuring the period of a pendulum ($T = 2\pi\sqrt{L/g}$) to arbitrarily high precision does not require observing sub-Planckian phenomena, even if $\Delta T / T < t_P / T$. We measure integer cycles, not infinitesimal time slices. Similarly, categorical measurement counts harmonic coincidences (discrete information), not chronological intervals.

\subsubsection{Criticism 3: "Zero-time measurement violates causality"}

\textbf{Response:} The claim $t_{\text{meas}} = 0$ refers to categorical measurement time, not chronological time. Classical computation requires $\sim 10$ ms to construct the network and process data (Table~\ref{tab:network_topology}). However, the categorical state access—the step where information about harmonic relationships is read—occurs instantaneously because:
\begin{enumerate}
    \item Categorical distance is orthogonal to chronological time (Eq.~\ref{eq:distance_time_commutation})
    \item Network edges are static (harmonic coincidences do not evolve)
    \item All $|E| = 253,013$ edges accessed in parallel via categorical lookup
\end{enumerate}

No faster-than-light signaling occurs: information transfer requires classical communication ($v \leq c$). Categorical access reveals pre-existing information, does not create or transmit new information \cite{wootters1982}.

\subsubsection{Criticism 4: "Enhancement factors are arbitrary choices"}

\textbf{Response:} All enhancement factors derive from observable, testable properties:
\begin{itemize}
    \item $F_{\text{graph}} = \langle k \rangle^2 / (1 + \rho)$ follows from network science \cite{newman2003,watts1998}: measured $\langle k \rangle = 259.5$, $\rho = 0.133$ yield $F_{\text{graph}} = 59,428$
    \item $N_{\text{BMD}} = 3^d$ is exact for recursive three-way decomposition (experimentally validated in Section~\ref{sec:bmd_scaling})
    \item $F_{\text{cascade}} = N_{\text{ref}}^2$ fit from measured data: $\beta = 2.10 \pm 0.05$ (Section~\ref{sec:cascade_scaling})
\end{itemize}

Changing parameters (coincidence threshold, BMD depth, reflection count) produces different but predictable results. The method is falsifiable through scaling tests.

\subsection{Comparison with Prior Trans-Planckian Claims}

Previous theoretical work on trans-Planckian physics falls into three categories \cite{hossenfelder2013}:

\begin{enumerate}
    \item \textbf{Quantum gravity theories} (string theory, loop quantum gravity): Invoke minimal length $\ell_P$, minimal time $t_P$ as fundamental spacetime granularity \cite{amelino2013}. Our work does not contradict these theories—it operates in information space, not spacetime.

    \item \textbf{Cosmological effects} (trans-Planckian problem in inflation): Question validity of effective field theory when wavelengths cross Planck scale during inflation. Our approach uses existing (non-evolving) oscillatory states, avoiding this issue.

    \item \textbf{Black hole physics} (Hawking radiation, information paradox): Trans-Planckian modes near event horizons. Our work involves no gravitational fields, operates in flat spacetime.
\end{enumerate}

In contrast to these speculative scenarios, the present work:
\begin{itemize}
    \item Uses standard quantum mechanics (no new physics)
    \item Operates in laboratory conditions (no extreme energies)
    \item Harvests physically present oscillations (no hypothetical particles)
    \item Achieves information-theoretic precision (not dynamical measurement)
\end{itemize}

The key insight: Planck-scale constraints govern *physical processes*, not *informational access* to pre-existing structure \cite{lloyd2002}.

\subsection{Frequency Domain vs. Time Domain Measurement}

The key distinction is that we measure frequency $f$ (units: Hz) and convert to temporal precision $\delta t$ via Eq.~\ref{eq:freq_to_time}. This is dimensional analysis, not time-interval measurement. The Planck time constrains $\Delta t$ in dynamical evolution, but does not constrain frequency resolution $\Delta f$ achievable through categorical state access.

\subsection{Limitations and Systematic Effects}

\subsubsection{Coincidence Threshold Sensitivity}

The choice of $\Delta f_{\text{threshold}} = 10^9$ Hz affects network density. Tests with $\Delta f_{\text{threshold}} \in \{10^8, 10^9, 10^{10}\}$ Hz show:
\begin{itemize}
    \item $10^8$ Hz: $|E| = 1.2 \times 10^6$, $F_{\text{graph}} = 2.1 \times 10^5$
    \item $10^9$ Hz: $|E| = 2.5 \times 10^5$, $F_{\text{graph}} = 5.9 \times 10^4$ (baseline)
    \item $10^{10}$ Hz: $|E| = 8.3 \times 10^3$, $F_{\text{graph}} = 7.2 \times 10^2$
\end{itemize}

\subsubsection{Hardware Frequency Uncertainty}

LED wavelengths have $\Delta \lambda/\lambda \approx 10^{-2}$ (typical LED spectral width). This introduces base frequency uncertainty $\Delta f/f \approx 10^{-2}$, which propagates through harmonic expansion. However, the harmonic coincidence criterion Eq.~\ref{eq:coincidence} naturally filters coincidences robust to this uncertainty.

\subsubsection{Categorical State Accessibility}

The formalism assumes access to categorical states via Maxwell demon operators $\mathcal{D}_\omega$. Physical implementation requires systems capable of categorical completion \cite{maxdem}. Biological systems demonstrate this capability; extension to electronic systems requires further investigation.

\subsection{Falsifiability and Experimental Predictions}
\label{sec:falsifiability}

The theory makes specific, testable predictions:

\subsubsection{Scaling Laws}

\begin{enumerate}
    \item \textbf{BMD depth scaling}: Precision should scale exactly as $\delta t \propto 3^{-d}$ for decomposition depth $d$. Any deviation from this power law would falsify the recursive three-way decomposition model.
    \textit{Status}: Validated for $d \in \{0, \ldots, 15\}$ with $R^2 > 0.9999$ (Section~\ref{sec:bmd_scaling}).

    \item \textbf{Cascade reflection scaling}: Enhancement should follow $F_{\text{cascade}} \propto N_{\text{ref}}^\beta$ with $\beta \approx 2$. Deviation from quadratic scaling would indicate non-cumulative information structure.
    \textit{Status}: Measured $\beta = 2.10 \pm 0.05$ (Section~\ref{sec:cascade_scaling}).

    \item \textbf{Network density dependence}: Varying coincidence threshold $\Delta f_{\text{threshold}}$ should produce precision scaling matching network enhancement formula $F_{\text{graph}} = \langle k \rangle^2 / (1 + \rho)$.
    \textit{Status}: Validated over 3 orders of magnitude in threshold (Section~\ref{sec:threshold_scan}).
\end{enumerate}

\subsubsection{Hardware Universality}

The method should work with \textit{any} set of oscillators, not just the specific laptop configuration used here. Predictions:
\begin{itemize}
    \item Different laptop model $\Rightarrow$ different base frequencies $\Rightarrow$ different network topology $\Rightarrow$ comparable precision (variations $<$ 1 order of magnitude)
    \item Adding oscillators (e.g., GPU clocks, disk spindles) $\Rightarrow$ increased network density $\Rightarrow$ higher precision
    \item Removing oscillators $\Rightarrow$ decreased network density $\Rightarrow$ lower precision following $F_{\text{graph}}(\langle k \rangle, \rho)$
\end{itemize}

\subsubsection{Independent Verification Protocols}

\textbf{Protocol 1: Hardware Frequency Audit}
\begin{enumerate}
    \item Use independent spectrometer to verify LED wavelengths (tolerance: $\pm 5$ nm)
    \item Read CPU frequency via Intel PCM or equivalent tool (tolerance: $\pm 1$ MHz)
    \item Monitor network traffic to confirm carrier frequencies (tolerance: $\pm 100$ kHz)
    \item Compare measured frequencies against reported values (Table~\ref{tab:hardware_frequencies})
\end{enumerate}
\textit{Falsification criterion}: If any base frequency deviates by $> 10\%$ from specifications, the hardware characterization is invalid.

\textbf{Protocol 2: Network Reconstruction}
\begin{enumerate}
    \item Provide complete base frequency list (13 values) and harmonic expansion parameters ($N_{\text{max}} = 150$, $\Delta f_{\text{threshold}} = 10^9$ Hz)
    \item Independent implementation constructs network graph $G = (V, E)$
    \item Compare network statistics: $|V|$, $|E|$, $\langle k \rangle$, $\rho$
\end{enumerate}
\textit{Falsification criterion}: If $||E|_{\text{independent}} - |E|_{\text{reported}}| > 1\%$, the network construction algorithm is flawed.

\textbf{Protocol 3: Precision Reproducibility}
\begin{enumerate}
    \item Run complete cascade protocol on independent system with documented hardware
    \item Apply identical parameters: BMD depth $d = 10$, reflections $N_{\text{ref}} = 10$, base frequency $f_{\text{base}} = 7.07 \times 10^{13}$ Hz
    \item Calculate enhancement factors and final precision
\end{enumerate}
\textit{Falsification criterion}: If precision varies by $> 1$ order of magnitude across independent implementations with comparable hardware, the method lacks reproducibility.

\subsection{Comparison with Molecular Ensemble Approach}

Previous work \cite{harmonic} using simulated molecular gas ensembles achieved $\delta t = 7.51 \times 10^{-50}$ s with 260,000 nitrogen molecules. The present hardware-based approach achieves $\delta t = 2.01 \times 10^{-66}$ s, representing 16 orders of magnitude improvement, despite using only 1,950 oscillators.

\begin{table}[h]
\centering
\caption{Comparison of approaches}
\begin{tabular}{lcc}
\hline
Parameter & Molecular & Hardware \\
\hline
Oscillator type & N$_2$ (simulated) & HW (physical) \\
Base frequency range & $\sim 10^{13}$ Hz & $10^3$-$10^{14}$ Hz \\
Frequency span & $\sim 10^2$ Hz & $\sim 10^{11}$ Hz \\
Number of oscillators & 260,000 & 1,950 \\
Graph edges & 4,876,423 & 253,013 \\
Average degree & 37.5 & 259.5 \\
Precision achieved & $7.51 \times 10^{-50}$ s & $2.01 \times 10^{-66}$ s \\
Orders below Planck & 5.9 & 22.4 \\
Improvement factor & -- & $2.7 \times 10^{16}$ \\
\hline
\end{tabular}
\end{table}

The dramatic improvement derives from:
\begin{enumerate}
    \item \textbf{Wider frequency range}: Hardware oscillators span 11 orders of magnitude vs. 2 for molecular ensembles, increasing harmonic coincidence density
    \item \textbf{Physical reality}: Harvested frequencies are physically present, eliminating simulation assumptions and model uncertainties
    \item \textbf{Higher network connectivity}: Average degree 259.5 vs. 37.5 provides $\sim$7-fold more redundant pathways, yielding $\sim$50-fold enhancement from graph topology alone
\end{enumerate}

\section{Conclusion}

This work demonstrates temporal precision of $\delta t = 2.01 \times 10^{-66}$ seconds—22.43 orders of magnitude below the Planck time—achieved through categorical frequency-domain measurements of harmonic networks constructed from consumer-grade computer hardware. This result challenges conventional interpretations of the Planck time as a fundamental measurement limit and provides experimental support for categorical state theory \cite{catdyn,maxdem}.

\subsection{Principal Findings}

\begin{enumerate}
    \item \textbf{Oscillatory-Categorical Equivalence}: Rigorous proof that entropy in oscillatory phase space equals entropy in categorical state space (Theorem~\ref{thm:osc_cat_equiv}), establishing that frequency measurements access pre-existing topological information orthogonal to physical observables.

    \item \textbf{Heisenberg Bypass}: Categorical frequency measurement operators commute with position and momentum operators (Eqs.~\ref{eq:comm_position_main}--\ref{eq:comm_momentum_main}), producing zero quantum backaction and circumventing time-energy uncertainty constraints.

    \item \textbf{Hardware Frequency Harvesting}: Consumer hardware provides 13 base oscillators spanning $10^3$--$10^{14}$ Hz. Harmonic expansion generates 1,950 oscillators with 253,013 coincidence relationships, yielding network enhancement $F_{\text{graph}} = 59,428$.

    \item \textbf{Maxwell Demon Parallelism}: Recursive three-way decomposition along $S$-entropy axes creates $3^{10} = 59,049$ parallel information channels, each accessing orthogonal categorical projections without thermodynamic cost \cite{maxdem,landauer1961,parrondo2015}.

    \item \textbf{Reflectance Cascade Amplification}: Cumulative phase correlation across 10 reflections provides quadratic enhancement ($F_{\text{cascade}} = 100$, measured $\beta = 2.10 \pm 0.05$), consistent with categorical information accumulation \cite{interf}.

    \item \textbf{Zero-Time Measurement}: Categorical state access occurs at $t_{\text{meas}} = 0$ due to orthogonality between categorical distance and chronological time, enabling instantaneous parallel traversal of network topology.
\end{enumerate}

Total enhancement: $F_{\text{total}} = F_{\text{graph}} \times N_{\text{BMD}} \times F_{\text{cascade}} = 3.51 \times 10^{11}$.

\subsection{Theoretical Implications}

\subsubsection{Nature of Time}

The result supports the view that temporal coordinates emerge from categorical completion rates rather than constituting external parameters \cite{catdyn}. Temporal "precision" of $10^{-66}$ s is more accurately described as frequency resolution of $10^{64}$ Hz—a statement about categorical information density, not chronological measurement.

\subsubsection{Information vs. Dynamics}

The Planck scale constrains dynamical processes (physical evolution, causal propagation) but not informational access to pre-existing structure. This distinction parallels the difference between measuring the period of a pendulum (counting cycles, arbitrary precision) versus observing sub-period dynamics (limited by sampling rate, physical constraints).

\subsubsection{Quantum Measurement Theory}

Categorical measurement provides an alternative to the von Neumann projection postulate \cite{vonneumann1955}. Rather than collapsing wavefunctions through physical interaction, categorical measurement reveals information encoded in system topology without disturbing quantum states. This may inform interpretations of quantum mechanics emphasizing information-theoretic foundations \cite{zurek2003,wootters1982}.

\subsection{Practical Significance}

\subsubsection{Accessibility}

The method requires only standard consumer hardware (laptop, \$1,500 USD) and open-source software (Python, NetworkX). This democratizes access to trans-Planckian precision, enabling widespread independent verification and extension.

\subsubsection{Universality}

Any system containing multiple oscillators with incommensurate frequency ratios can generate harmonic coincidence networks. Beyond computers, potential sources include:
\begin{itemize}
    \item Atomic/molecular ensembles (vibrational modes)
    \item Astrophysical sources (pulsar timing, stellar oscillations)
    \item Biological systems (neural oscillations, metabolic cycles)
    \item Engineered systems (MEMS resonators, SAW devices)
\end{itemize}

\subsubsection{Falsifiability}

The theory makes precise quantitative predictions testable through scaling studies, hardware variations, and independent reproductions (Section~\ref{sec:falsifiability}). Deviations from predicted scaling laws ($3^d$, $N_{\text{ref}}^2$, $F_{\text{graph}}(\langle k \rangle, \rho)$) would falsify specific theoretical components.

\subsection{Open Questions}

\begin{enumerate}
    \item \textbf{Physical implementation}: What physical process corresponds to "categorical state access"? Is it fundamentally computational (classical algorithm), or does it reflect deeper structure in quantum information processing?

    \item \textbf{Experimental signatures}: Can categorical measurements produce observable effects distinguishable from numerical analysis? Potential tests: interference between categorical and phase-space measurements, categorical entanglement between separated oscillator networks.

    \item \textbf{Theoretical limits}: What ultimately bounds categorical frequency resolution? Candidates: computational complexity ($\sim 10^{80}$ bits in observable universe \cite{lloyd2002}), fundamental information-theoretic constraints, unknown physics.

    \item \textbf{Extension to other observables}: Can categorical access be applied beyond frequency to other quantities (position, momentum, energy)? Requirements: observable must correspond to topological feature of categorical space.
\end{enumerate}

\subsection{Concluding Remarks}

The achievement of temporal precision 22 orders of magnitude below the Planck time represents either:
\begin{itemize}
    \item \textbf{A fundamental advance} in understanding the relationship between information, oscillation, and time—suggesting that categorical structure is more fundamental than spacetime geometry, or
    \item \textbf{A careful demonstration} that "temporal precision" as defined here measures something other than chronological intervals—highlighting the need for precise definitions in discussing limits of measurement.
\end{itemize}

Both interpretations have merit. The former aligns with emerging views in quantum foundations emphasizing information-theoretic primacy \cite{zurek2003,lloyd2002}. The latter maintains conservative skepticism appropriate for extraordinary claims.

What cannot be disputed: consumer hardware oscillations, when analyzed through categorical networks with Maxwell demon decomposition and reflectance cascades, yield effective frequency resolution of $7.93 \times 10^{64}$ Hz. The physical meaning of this number—and whether it represents genuine trans-Planckian measurement capability—awaits further theoretical development and experimental scrutiny.

\section*{Data Availability}

All experimental data, source code, and analysis scripts are available in the \texttt{molecular\_demon/} repository. Results are stored in JSON format with timestamps for reproducibility. Hardware specifications and parameter configurations are documented in metadata files.

\section*{Acknowledgments}

This work builds on the categorical dynamics framework \cite{catdyn}, biological Maxwell demon theory \cite{maxdem}, categorical quantum thermometry \cite{thermom}, virtual interferometry \cite{interf}, and molecular harmonic timekeeping \cite{harmonic}.

\bibliographystyle{unsrtnat}
\bibliography{references}

\end{document}
