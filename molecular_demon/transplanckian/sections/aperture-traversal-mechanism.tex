%==============================================================================
\section{Aperture Traversal and Resolution of Maxwell's Demon}
\label{sec:aperture}
%==============================================================================

\subsection{Phase-Lock Network Topology}

\begin{definition}[Phase-Lock Network]
\label{def:phase_lock}
A phase-lock network $\mathcal{G}_{\text{PL}} = (V, E)$ comprises:
\begin{itemize}
\item Vertices $V$: Oscillatory modes (molecular vibrations, rotations, translations)
\item Edges $E$: Phase-lock relationships mediated by position-dependent interactions
\end{itemize}
\end{definition}

\begin{theorem}[Kinetic Independence]
\label{thm:kinetic_independence}
Phase-lock network topology is independent of molecular kinetic energy:
\begin{equation}
\frac{\partial \mathcal{G}_{\text{PL}}}{\partial E_{\text{kin}}} = 0
\end{equation}
\end{theorem}

\begin{proof}
Phase-lock edges form through:

\textbf{Van der Waals forces:}
\begin{equation}
U_{\text{vdW}}(\mathbf{r}) = -\frac{C_6}{r^6}
\end{equation}
depends on polarizability $C_6$ and separation $r$, not velocity.

\textbf{Dipole interactions:}
\begin{equation}
U_{\text{dipole}}(\mathbf{r}, \boldsymbol{\mu}_1, \boldsymbol{\mu}_2) \propto \frac{\boldsymbol{\mu}_1 \cdot \boldsymbol{\mu}_2}{r^3}
\end{equation}
depends on dipole moments $\boldsymbol{\mu}$ and geometry, not kinetic energy.

\textbf{Vibrational coupling:} Synchronization through collisions depends on normal mode frequencies $\omega_i$, not translational velocity.

None of these interactions depend on $E_{\text{kin}} = \tfrac{1}{2}mv^2$. Network structure $\mathcal{G}_{\text{PL}}$ determined by spatial configuration $(n,\ell,m)$ and electronic structure, independent of velocity. Therefore:
\begin{equation}
\frac{\partial \mathcal{G}_{\text{PL}}}{\partial E_{\text{kin}}} = 0
\end{equation}
\end{proof}

\subsection{Aperture Traversal Mechanism}

\begin{definition}[Categorical Aperture]
\label{def:aperture}
A categorical aperture is an accessible transition between partition states $(n_i,\ell_i,m_i,s_i) \to (n_f,\ell_f,m_f,s_f)$ satisfying selection rules and energy conservation.
\end{definition}

\begin{theorem}[Aperture Traversal]
\label{thm:aperture_traversal}
Categorical completion proceeds through aperture traversal without measurement, sorting, or information acquisition.
\end{theorem}

\begin{proof}
Categorical state $(n,\ell,m,s)$ specifies partition cell in bounded phase space. Accessible neighboring states form aperture set:
\begin{equation}
\mathcal{A}(n,\ell,m,s) = \{(n',\ell',m',s') : |\Delta\ell| = 1, |\Delta m| \leq 1, \Delta s = 0\}
\end{equation}

System evolves by traversing apertures according to selection rules (Theorem \ref{thm:selection_rules}). No external agent required:
\begin{itemize}
\item No measurement: aperture structure pre-exists in partition geometry
\item No decision: selection rules are deterministic constraints
\item No sorting: velocity-independent (Theorem \ref{thm:kinetic_independence})
\item No information cost: Landauer erasure applies only to acquired information; pre-existing structure incurs no cost
\end{itemize}

What Maxwell interpreted as "demon sorting" is categorical completion through topologically determined apertures.
\end{proof}

\subsection{Heat-Entropy Decoupling}

\begin{theorem}[Heat-Entropy Distinction]
\label{thm:heat_entropy_decoupling}
Heat and entropy are fundamentally decoupled at microscopic scale:
\begin{align}
\text{Heat:} & \quad \text{Statistical emergent property (bidirectional)} \\
\text{Entropy:} & \quad \text{Categorical fundamental property (monotonic)}
\end{align}
\end{theorem}

\begin{proof}
\textbf{Heat} is energy transfer due to temperature difference:
\begin{equation}
dQ = TdS - PdV
\end{equation}

In individual molecular collisions, energy can flow either direction. For molecule crossing boundary with kinetic energy $E_{\text{kin}}$:
\begin{equation}
\Delta Q = E_{\text{kin}}^{\text{after}} - E_{\text{kin}}^{\text{before}}
\end{equation}

can be positive (cold to hot) or negative (hot to cold) with fluctuating probability.

\textbf{Entropy} is categorical completion:
\begin{equation}
S = \kB M \ln n
\end{equation}

When molecule transfers between containers, both undergo categorical completion:
\begin{itemize}
\item Losing container: $N-1$ molecules form new phase-lock network, $S_A' > S_A$
\item Receiving container: $N+1$ molecules add cross-container correlations, $S_B' > S_B$
\end{itemize}

Total entropy change:
\begin{equation}
\Delta S_{\text{total}} = (S_A' - S_A) + (S_B' - S_B) > 0
\end{equation}

regardless of heat flow direction. Individual collision can transfer heat from cold to hot ($\Delta Q > 0$) while increasing entropy ($\Delta S > 0$).

Maxwell conflated heat and entropy because macroscopically $dS = dQ/T$. At single-molecule level, they decouple.
\end{proof}

\subsection{Symmetric Entropy Increase}

\begin{theorem}[Door Operation Entropy]
\label{thm:door_entropy}
Every door operation increases entropy in both containers:
\begin{equation}
\Delta S_A > 0 \quad \text{and} \quad \Delta S_B > 0
\end{equation}
\end{theorem}

\begin{proof}
Consider transfer of one molecule from container A to container B.

\textbf{Container A (losing):} $N$ molecules $\to$ $N-1$ molecules. Phase-lock network must reconfigure:
\begin{equation}
\mathcal{G}_A = (V_A, E_A) \to \mathcal{G}_A' = (V_A', E_A')
\end{equation}

where $|V_A'| = |V_A| - 1$. Remaining molecules form new network, which by categorical completion satisfies:
\begin{equation}
S_A' = \kB M_A' \ln n_A' > S_A
\end{equation}

because phase-lock network must explore new configurations to compensate for missing node.

\textbf{Container B (receiving):} $N$ molecules $\to$ $N+1$ molecules. Network gains node:
\begin{equation}
\mathcal{G}_B = (V_B, E_B) \to \mathcal{G}_B' = (V_B', E_B')
\end{equation}

where $|V_B'| = |V_B| + 1$. New molecule creates additional phase-lock edges with existing molecules (mixing-type process):
\begin{equation}
S_B' = \kB M_B' \ln n_B' > S_B
\end{equation}

Total entropy increases symmetrically:
\begin{equation}
\Delta S_{\text{total}} = \underbrace{(S_A' - S_A)}_{> 0} + \underbrace{(S_B' - S_B)}_{> 0} > 0
\end{equation}

This holds regardless of molecular velocity, container temperatures, or door operation strategy.
\end{proof}

\begin{corollary}[Demon Dissolution]
\label{cor:demon_dissolution}
Maxwell's demon cannot decrease entropy because every door operation increases entropy in both containers through categorical completion.
\end{corollary}

The "demon" is projection of categorical dynamics onto kinetic observables. Observer measuring only velocities sees structured evolution resembling intelligent sorting. Observer measuring partition coordinates sees deterministic aperture traversal through pre-existing network topology.

\begin{figure*}[htbp]
    \centering
    \includegraphics[width=0.95\textwidth]{figures/maxwell_demon_resolution_panel.png}
    \caption{\textbf{Maxwell's Demon Resolution: Entropy Increases for ANY Molecule Transfer Regardless of Velocity.}
    The figure demonstrates that entropy increases in both containers for slow, medium, and fast molecules, with identical entropy changes regardless of velocity ($\Delta S_A > 0$ and $\Delta S_B > 0$ in all cases). Three scenarios are shown vertically: slow molecule ($v \approx 100$ m/s, top row), medium molecule ($v \approx 400$ m/s, middle row), and fast molecule ($v \approx 800$ m/s, bottom row). Each scenario progresses through three stages (left to right): before transfer, during transfer, and after transfer, with quantified entropy changes shown in the rightmost panels.

    \textbf{Slow Molecule (v ≈ 100 m/s):}
    \textbf{Before:} Container A (green box) contains multiple green molecules with the partition door closed. Container B (purple box) contains multiple purple molecules. The networks are separate.
    \textbf{During:} The blue molecule (highlighted) transfers from A to B through the open door.
    \textbf{After:} Container A has $N-1$ molecules in the reconfigured network. Container B has $N+1$ molecules with new phase-lock edges (the purple cluster is denser).
    \textbf{Entropy changes:} $\Delta S_A = +0.07 \times 10^{-21}$ J/K (categorical completion), $\Delta S_B = +0.28 \times 10^{-21}$ J/K (mixing densification). Both positive (✓ BOTH > 0).

    \textbf{Medium Molecule (v ≈ 400 m/s):}
    \textbf{Before:} Same initial configuration as the slow case.
    \textbf{During:} The orange molecule (highlighted, medium velocity) transfers from A to B.
    \textbf{After:} Container A reconfigures with $N-1$ molecules. Container B densifies with $N+1$ molecules.
    \textbf{Entropy changes:} $\Delta S_A = +0.07 \times 10^{-21}$ J/K, $\Delta S_B = +0.28 \times 10^{-21}$ J/K. Identical to slow case (✓ BOTH > 0).

    \textbf{Fast Molecule (v ≈ 800 m/s):}
    \textbf{Before:} Same initial configuration.
    \textbf{During:} The red molecule (highlighted, fast velocity) transfers from A to B.
    \textbf{After:} Container A reconfigures, and Container B densifies.
    \textbf{Entropy changes:} $\Delta S_A = +0.07 \times 10^{-21}$ J/K, $\Delta S_B = +0.28 \times 10^{-21}$ J/K. Again identical (✓ BOTH > 0).
    }
    \label{fig:maxwell_demon_resolution}
    \end{figure*}

\subsection{Velocity-Entropy Independence}

\begin{theorem}[Orthogonality of Velocity and Entropy]
\label{thm:velocity_entropy}
Entropy is independent of velocity distribution:
\begin{equation}
\frac{\partial \Omega}{\partial v_i} = 0
\end{equation}
where $\Omega$ is number of spatial configurations.
\end{theorem}

\begin{proof}
Boltzmann entropy:
\begin{equation}
S = \kB \ln \Omega
\end{equation}

where $\Omega$ counts spatial arrangements. For $N$ particles in volume $V$:
\begin{equation}
\Omega = \frac{V^N}{N! \lambda^{3N}}
\end{equation}

where $\lambda$ is de Broglie wavelength $\lambda = h/p = h/(mv)$.

Temperature dependence enters through $\lambda \propto 1/\sqrt{T}$, but this affects phase space volume factor, not spatial configuration count. For distinguishable particles at fixed positions:
\begin{equation}
\Omega_{\text{config}} = \frac{V^N}{N!}
\end{equation}

independent of velocities $\{v_i\}$. Entropy counts arrangements, not velocities:
\begin{equation}
\frac{\partial \Omega_{\text{config}}}{\partial v_i} = 0
\end{equation}

Sorting by velocity does not sort by entropy. The demon manipulates velocity (kinetic face) while entropy (categorical face) remains protected through orthogonality.
\end{proof}

\subsection{Temperature Emergence}

\begin{theorem}[Temperature as Statistical Property]
\label{thm:temperature_emergence}
Temperature emerges from phase-lock cluster statistics:
\begin{equation}
T = \mathcal{F}[\{\mathcal{G}_\alpha\}]
\end{equation}
where $\{\mathcal{G}_\alpha\}$ is ensemble of phase-lock clusters.
\end{theorem}

\begin{proof}
Temperature is not molecular property but ensemble average. From Theorem \ref{thm:temperature}:
\begin{equation}
T = \frac{U}{\kB M}
\end{equation}

Internal energy $U$ and categorical dimensions $M$ are collective properties of phase-lock network $\mathcal{G}_{\text{PL}}$. Individual molecule has velocity $v_i$ but no temperature. Temperature emerges from averaging:
\begin{equation}
\langle v^2 \rangle = \frac{1}{N}\sum_{i=1}^N v_i^2 = \frac{3\kB T}{m}
\end{equation}

Phase-lock clusters with high connectivity have high categorical density, correlating with high kinetic energy. But correlation is not causation---network structure determines categorical dimensions $M$, which determine temperature $T$, which correlates with average kinetic energy. The causal chain:
\begin{equation}
\mathcal{G}_{\text{PL}} \to M \to T \to \langle E_{\text{kin}} \rangle
\end{equation}

not $\langle E_{\text{kin}} \rangle \to T$. Demon observes kinetic energy but cannot access categorical structure directly.
\end{proof}

\begin{figure}[htbp]
    \centering
    \includegraphics[width=\textwidth]{figures/partition_traversal_panel.png}
    \caption{Partition traversal dynamics during resonant coupling demonstrating systematic occupation evolution, charge redistribution, and information crystallization across quantum state transitions.
    \textbf{(A) Partition occupation evolution:} Heat map showing temporal evolution of partition element occupation over 50 coupling cycles. Color gradient from white (unoccupied) to dark green (fully occupied) reveals systematic traversal pattern with elements 20--10 showing sequential activation and deactivation cycles.
    \textbf{(B) Charge redistribution during coupling:} Oscillatory charge exchange between system (blue) and apparatus (red) over $12\omega t$ time units. Sinusoidal patterns demonstrate periodic energy transfer with complete charge redistribution cycles, maintaining total charge conservation throughout coupling process.
    \textbf{(C) Partition trajectory $(n, l)$:} Two-dimensional trajectory in quantum number space showing path from start (green circle) to end (red square) positions. Blue trajectory points demonstrate systematic traversal through allowed quantum states with complexity $l$ ranging 0--6 and depth $n$ spanning 1--7.
    \textbf{(D) Information crystallization from partition completion:} Information accumulation showing rapid initial growth (green bars per cycle) reaching saturation at 7 bits. Red cumulative curve demonstrates exponential approach to maximum information content, indicating complete partition characterization after $\sim$10 coupling cycles.
    \textbf{(E) Energy as carrier of partition transitions:} Energy exchange histogram showing transition energies from 1--11 $\times 10^{-19}$ J. Orange bars demonstrate increasing energy requirements for higher-order partition transitions ($\Delta\xi$ from 2--10), with maximum energy at $\Delta\xi = 10$.
    \textbf{(F) Allowed states $|m| \leq l$:} Magnetic quantum number distribution heat map showing allowed $m$ values ($-4$ to $+4$) for each complexity level $l$ (0--4). Green regions indicate accessible states, red regions show forbidden combinations, demonstrating angular momentum selection rules during partition traversal.}
    \label{fig:partition_traversal}
\end{figure}


\subsection{Information Complementarity}

\begin{theorem}[Conjugate Information Faces]
\label{thm:information_complementarity}
Kinetic and categorical information are conjugate observables that cannot be simultaneously specified with arbitrary precision.
\end{theorem}

\begin{proof}
Information possesses two faces:
\begin{itemize}
\item \textbf{Kinetic face}: Velocities $\{v_i\}$, temperatures $T$, molecular speeds
\item \textbf{Categorical face}: Phase-lock networks $\mathcal{G}_{\text{PL}}$, partition coordinates $(n,\ell,m,s)$
\end{itemize}

Measuring kinetic face (velocities) disturbs categorical face (network structure) and vice versa. This is analogous to position-momentum uncertainty:
\begin{equation}
\Delta_{\text{kinetic}} \cdot \Delta_{\text{categorical}} \geq \text{const}
\end{equation}

Maxwell observed kinetic face exclusively. The "demon" is categorical face dynamics projected onto kinetic observables. Observer confined to one face perceives complementary face as external intelligent agent.

When observer gains access to both faces simultaneously (as in modern mass spectrometry measuring both velocities and partition coordinates), demon dissolves---structured evolution is revealed as deterministic aperture traversal.
\end{proof}

The resolution of Maxwell's demon requires no information-theoretic arguments, no quantum considerations, no measurement costs. The Second Law is preserved through geometric necessity: categorical completion densifies phase-lock networks monotonically, with heat flow direction and velocity distributions being orthogonal observables that fluctuate independently.
