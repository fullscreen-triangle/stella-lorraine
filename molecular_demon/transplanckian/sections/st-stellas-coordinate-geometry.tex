%==============================================================================
\section{S-Entropy Coordinate Geometry and Ternary Representation}
\label{sec:stellas}
%==============================================================================

\subsection{Three-Dimensional Entropy Space}

\begin{definition}[S-Entropy Coordinates]
\label{def:s_entropy}
The S-entropy coordinate space $\Sspace = [0,1]^3$ comprises three fundamental dimensions:
\begin{align}
S_k &\in [0,1] \quad \text{(knowledge entropy)} \label{eq:Sk} \\
S_t &\in [0,1] \quad \text{(temporal entropy)} \label{eq:St} \\
S_e &\in [0,1] \quad \text{(evolution entropy)} \label{eq:Se}
\end{align}
\end{definition}

The three coordinates emerge from categorical decomposition of bounded systems:

\textbf{Knowledge entropy} $S_k = -\log_2 P_{\text{config}}$ measures information deficit---how many categorical distinctions remain to be specified to fully determine the system state. At $S_k = 0$, configuration is fully specified. At $S_k = 1$, maximal uncertainty.

\textbf{Temporal entropy} $S_t = \log_{10}(\tau/\tau_0)$ measures temporal distance from reference timescale $\tau_0$, capturing hierarchical structure from molecular vibrations ($\tau \sim 10^{-14}$ s) to macroscopic equilibration ($\tau \sim 10^{0}$ s).

\textbf{Evolution entropy} $S_e = -\sum_i p_i \log_2 p_i$ measures phase distribution entropy, quantifying diversity of oscillatory modes in the system.

\begin{proposition}[Metric Structure]
\label{prop:metric}
S-entropy space admits metric:
\begin{equation}
d_{\Sspace}(\Scoord_1, \Scoord_2) = \sqrt{(S_{k,1} - S_{k,2})^2 + (S_{t,1} - S_{t,2})^2 + (S_{e,1} - S_{e,2})^2}
\end{equation}
satisfying triangle inequality:
\begin{equation}
d_{\Sspace}(\Scoord_i, \Scoord_k) \leq d_{\Sspace}(\Scoord_i, \Scoord_j) + d_{\Sspace}(\Scoord_j, \Scoord_k)
\end{equation}
\end{proposition}

\begin{proof}
The Euclidean metric on $\mathbb{R}^3$ restricted to $[0,1]^3$ inherits all metric properties. Triangle inequality follows from Cauchy-Schwarz inequality applied to coordinate differences.
\end{proof}

\begin{figure}[H]
    \centering
    \includegraphics[width=\textwidth]{figures/topology_categories_panel.png}
    \caption{Topological structure of categorical spaces showing partial ordering, dimensional relationships, and completion dynamics in hierarchical categorical measurement systems.
    \textbf{(A) Partial order:} Completion precedence structure showing hierarchical dependencies between categorical states with directed connectivity indicating measurement ordering constraints.
    \textbf{(B) Tri-dimensional S-space:} Three-dimensional coordinate system $(S_k, S_t, S_e)$ with yellow point indicating specific categorical state location within unit cube geometry.
    \textbf{(C) $3^k$ branching structure:} Hierarchical tree showing exponential branching with root C and ternary subdivision creating multi-colored terminal nodes representing categorical state endpoints.
    \textbf{(D) Scale ambiguity:} Identical triangular structures at Level n and Level n+1 demonstrating scale-invariant topology with ambiguity parameter $\Psi_n$ indicating measurement uncertainty.
    \textbf{(E) Completion trajectory:} Fraction completed $\gamma(t)$ approaching unity asymptotically (green curve) with completion target (red dashed line) showing bounded convergence dynamics.
    \textbf{(F) Asymptotic slowing:} Completion rate $\dot{C}(t) \to 0$ (red curve) with completion time $T$ (dotted line) demonstrating deceleration in categorical state enumeration approaching completeness.}
    \label{fig:topology_categories}
    \end{figure}

\subsection{Categorical Distance and Physical Distance}

\begin{theorem}[Distance Inequivalence]
\label{thm:distance_inequivalence}
Categorical distance $d_{\Sspace}$ does not correspond to physical distance $d_{\text{phys}}$:
\begin{equation}
d_{\Sspace}(\Scoord_i, \Scoord_j) \neq f(d_{\text{phys}}(\mathbf{r}_i, \mathbf{r}_j))
\end{equation}
for any function $f$.
\end{theorem}

\begin{proof}
Consider two molecular configurations:
\begin{itemize}
\item Configuration A: Two molecules separated by $d_{\text{phys}} = 1$ nm in same categorical state (identical vibrational modes, same phase)
\item Configuration B: Two molecules separated by $d_{\text{phys}} = 1$ nm in different categorical states (different vibrational modes, opposite phase)
\end{itemize}

Physical distance identical: $d_{\text{phys}}^A = d_{\text{phys}}^B = 1$ nm.

Categorical distance differs:
\begin{align}
d_{\Sspace}^A &= 0 \quad \text{(same categorical state)} \\
d_{\Sspace}^B &> 0 \quad \text{(different categorical states)}
\end{align}

Therefore, no function $f$ exists mapping physical distance to categorical distance. The two metrics are inequivalent.
\end{proof}

This inequivalence is central to resolving Maxwell's demon: spatial proximity does not imply categorical proximity. Molecules can be physically adjacent yet categorically distant, and vice versa.

\subsection{Ternary Representation}

\begin{definition}[Ternary Digit (Trit)]
\label{def:trit}
A ternary digit (trit) $\trit \in \{0, 1, 2\}$ encodes position along one of three S-entropy axes:
\begin{align}
\trit = 0 &\leftrightarrow \text{refinement along } S_k \\
\trit = 1 &\leftrightarrow \text{refinement along } S_t \\
\trit = 2 &\leftrightarrow \text{refinement along } S_e
\end{align}
\end{definition}

\begin{theorem}[Trit-Coordinate Correspondence]
\label{thm:trit_coord}
A $k$-trit ternary string addresses exactly one cell in the $3^k$ hierarchical partition of $\Sspace$.
\end{theorem}

\begin{proof}
At recursion level $k$, $\Sspace$ is partitioned into $3^k$ cells through recursive subdivision. Each subdivision divides one cube into 3 subcubes along one axis.

A $k$-trit string $T = \trit_1 \trit_2 \cdots \trit_k$ specifies navigation path:
\begin{itemize}
\item $\trit_1 \in \{0,1,2\}$: Select one of 3 cells at depth 1 (3 cells total)
\item $\trit_2 \in \{0,1,2\}$: Select one of 3 subcells at depth 2 (9 cells total)
\item $\trit_k \in \{0,1,2\}$: Select one of 3 subcells at depth $k$ ($3^k$ cells total)
\end{itemize}

Each trit value determines which axis to subdivide along:
\begin{itemize}
\item $\trit_i = 0$: Subdivide interval $[0,1] \to [0, 1/3], [1/3, 2/3], [2/3, 1]$ along $S_k$ axis
\item $\trit_i = 1$: Subdivide along $S_t$ axis
\item $\trit_i = 2$: Subdivide along $S_e$ axis
\end{itemize}

The mapping $\phi: \{0,1,2\}^k \to \{\text{cells at depth } k\}$ is bijective:
\begin{itemize}
\item Injective: Different strings $T \neq T'$ specify different navigation paths, hence different final cells
\item Surjective: Every cell at depth $k$ is reachable by some string of length $k$
\end{itemize}

Therefore, $k$-trit strings correspond one-to-one with cells at depth $k$.
\end{proof}

\subsection{Continuous Emergence}

\begin{theorem}[Continuous Emergence]
\label{thm:continuous_emergence}
As $k \to \infty$, the discrete $3^k$ cell structure converges to continuous space $[0,1]^3$:
\begin{equation}
\lim_{k \to \infty} \text{Cell}(\trit_1, \ldots, \trit_k) = \Scoord \in [0,1]^3
\end{equation}
with the ternary expansion:
\begin{equation}
S_\alpha = \sum_{i=1}^\infty \frac{\trit_i^{(\alpha)}}{3^i}, \quad \alpha \in \{k, t, e\}
\end{equation}
converging to unique point in the continuum.
\end{theorem}

\begin{proof}
For coordinate $S_\alpha$, the $k$-trit approximation is:
\begin{equation}
S_\alpha^{(k)} = \sum_{i=1}^k \frac{\trit_i^{(\alpha)}}{3^i}
\end{equation}

This is a geometric series with ratio $1/3$. For any $\epsilon > 0$, choose $k$ such that:
\begin{equation}
\left|S_\alpha - S_\alpha^{(k)}\right| = \sum_{i=k+1}^\infty \frac{\trit_i^{(\alpha)}}{3^i} \leq \sum_{i=k+1}^\infty \frac{2}{3^i} = \frac{2}{3^k} \cdot \frac{1}{1-1/3} = \frac{1}{3^{k-1}} < \epsilon
\end{equation}

For $k > \log_3(1/\epsilon) + 1$, approximation error $< \epsilon$. Therefore:
\begin{equation}
\lim_{k \to \infty} S_\alpha^{(k)} = S_\alpha
\end{equation}

Convergence is uniform over $[0,1]$. The infinite ternary string specifies unique point in continuum, bridging discrete computation and continuous dynamics.
\end{proof}

\begin{figure}[htbp]
    \centering
    \includegraphics[width=\textwidth]{figures/figure1_ternary_encoding.png}
    \caption{\textbf{Ternary encoding enables hierarchical partition refinement with exponential convergence to the continuum limit.}
    \textbf{(A)} 3D entropy coordinate space $(S_k, S_t, S_e)$ shows trajectory (blue line) connecting initial states (blue circles) through intermediate states (orange circles) to final states (red circles). The trajectory explores the unit cube $[0,1]^3$ systematically, with each step corresponding to a ternary branch decision. Initial states cluster near $(0.2, 0.1, 0.1)$, intermediate states near $(0.5, 0.5, 0.3)$, and final states near $(0.7, 0.6, 0.6)$. The smooth progression confirms deterministic evolution through categorical space with no discontinuous jumps.
    \textbf{(B)} Hierarchical partition refinement shows exponential increase in resolution with hierarchy depth $k$. $k=1$ ($3^3=27$ cells): single blue square, coarse partition. $k=2$ ($3^6=729$ cells): $9 \times 9$ grid with colored regions, medium resolution. Red box highlights one cell for further refinement. $k=3$ ($3^9=19{,}683$ cells): $27 \times 27$ grid with fine-grained color structure. Red box shows target cell. $k=4$ ($3^{12}=531{,}441$ cells): $81 \times 81$ grid approaching continuum. Each level provides $3^3 = 27$-fold increase in total cells and $3$-fold increase in linear resolution. This exponential refinement enables arbitrary precision in categorical addressing: at depth $k$, cell volume $V(k) = 3^{-3k}$ and linear resolution $\delta x = 3^{-k}$.
    \textbf{(C)} Ternary address encoding shows complete tree structure from root (top, purple) through intermediate levels to leaf nodes (bottom, colored circles). The tree has depth $d \sim 5$ with $3^d \approx 243$ leaf nodes. Example address highlighted in red: path $0210\_3$ (base-3) corresponds to sequence Branch 0 $\to$ Branch 2 $\to$ Branch 1 $\to$ Branch 0, uniquely identifying one categorical state. Each ternary digit (trit) encodes one branching decision, with $k$ trits providing $3^k$ distinct addresses. The addressing is bijective: every categorical state has a unique ternary address, and every ternary address corresponds to a unique state.
    \textbf{(D)} Convergence to continuum shows cell volume $V(k) = 3^{-3k}$ (blue line with circles) versus number of trits $k$. Volume decreases exponentially from $V(1) = 10^{-2}$ to $V(20) = 10^{-29}$. Machine precision limit (red dashed line at $\sim 10^{-16}$) is crossed at $k \approx 11$ trits. Continuum limit (gray shaded region below $10^{-16}$) is reached at $k > 11$. For $k = 20$ trits, cell volume $V(20) \approx 10^{-29}$ is $13$ orders of magnitude below machine precision, enabling sub-atomic resolution in categorical addressing. This demonstrates that ternary encoding with $k \sim 20$ trits provides sufficient resolution to address individual quantum states in molecular systems, supporting the claim of trans-Planckian temporal resolution through categorical state counting.}
    \label{fig:ternary_encoding}
    \end{figure}

\subsection{Trajectory Encoding}

\begin{proposition}[Position-Trajectory Duality]
\label{prop:trajectory}
A ternary string encodes both position (final cell) and trajectory (navigation path):
\begin{equation}
T = \trit_1 \trit_2 \cdots \trit_k \quad \Rightarrow \quad \begin{cases}
\text{Position: Cell at depth } k \\
\text{Trajectory: Sequence of refinements}
\end{cases}
\end{equation}
\end{proposition}

\begin{proof}
Position interpretation: Apply Theorem \ref{thm:trit_coord}---string $T$ addresses unique cell.

Trajectory interpretation: Each trit $\trit_i$ specifies operation at step $i$:
\begin{itemize}
\item $\trit_i = 0$: Refine along $S_k$ axis (knowledge accumulation)
\item $\trit_i = 1$: Refine along $S_t$ axis (temporal progression)
\item $\trit_i = 2$: Refine along $S_e$ axis (evolutionary development)
\end{itemize}

The sequence $\trit_1 \to \trit_2 \to \cdots \to \trit_k$ describes path through $\Sspace$ from origin $(0,0,0)$ to final position. Reading the string forward gives trajectory; evaluating the string gives position. Address IS trajectory.
\end{proof}

This duality eliminates the von Neumann separation between data (position) and instructions (trajectory) at the representational level.

\subsection{Information Density Enhancement}

\begin{proposition}[Ternary Advantage]
\label{prop:ternary_advantage}
Ternary representation provides information density enhancement over binary:
\begin{equation}
\frac{3^k}{2^k} = \left(\frac{3}{2}\right)^k = 1.5^k
\end{equation}
\end{proposition}

\begin{proof}
Binary string of length $k$ encodes $2^k$ values. Ternary string of length $k$ encodes $3^k$ values. Density ratio:
\begin{equation}
\rho_{\text{ternary}} = \frac{3^k}{2^k} = \left(\frac{3}{2}\right)^k
\end{equation}

For $k = 20$ trits:
\begin{equation}
\rho_{\text{ternary}}^{(20)} = 1.5^{20} = 3325.26 \approx 10^{3.5}
\end{equation}

A 20-trit string encodes $3^{20} = 3.49 \times 10^9$ values compared to 20-bit string's $2^{20} = 1.05 \times 10^6$ values---over 3000 times more information in same string length.
\end{proof}

\subsection{Ternary Operations}

\begin{definition}[Ternary Projection]
\label{def:projection}
Extract coordinate along one axis:
\begin{equation}
\pi_\alpha(T) = \sum_{i: \trit_i^{(\alpha)} \neq \text{null}} \frac{\trit_i^{(\alpha)}}{3^i}, \quad \alpha \in \{k, t, e\}
\end{equation}
\end{definition}

\begin{definition}[Categorical Completion]
\label{def:completion}
Extend partial string to full representation:
\begin{equation}
\mathcal{C}(T_{\text{partial}}) = T_{\text{partial}} \oplus T_{\text{completion}}
\end{equation}
where $\oplus$ denotes concatenation and $T_{\text{completion}}$ is determined by minimizing categorical distance to accessible states.
\end{definition}

\begin{definition}[Trajectory Composition]
\label{def:composition}
Concatenate two trajectory segments:
\begin{equation}
T_3 = T_1 \circ T_2 = \trit_1^{(1)} \cdots \trit_{k_1}^{(1)} \trit_1^{(2)} \cdots \trit_{k_2}^{(2)}
\end{equation}
navigating first through $T_1$ then through $T_2$.
\end{definition}

These operations (project, complete, compose) replace Boolean logic (AND, OR, NOT) as fundamental computational primitives in ternary architecture.

\begin{figure}[htbp]
    \centering
    \includegraphics[width=\textwidth]{figures/panel_ternary_computation_2.png}
    \caption{\textbf{Ternary Computation as Gas Dynamics: Oscillator = Processor.}
    \textbf{Top Left - Ternary computation trajectories:} Three-dimensional plot showing individual molecular trajectories in S-entropy space. Axes: $S_k$ (knowledge), $S_t$ (time), $S_e$ (evolution) (all range $-0.05$ to 0.30). Yellow lines: trajectory paths for multiple molecules. Yellow sphere at origin: starting configuration. Trajectories explore bounded region, demonstrating confined dynamics in categorical phase space.
    \textbf{Top Center - Ensemble equilibration:} Three traces showing mean S-coordinates versus computation step (0-140): blue ($S_k$, categorical), orange ($S_t$, oscillatory), green ($S_e$, partition). Vertical axis: mean S-coordinate ($-0.10$ to 0.30). All three converge from initial values ($\sim$0.25) to equilibrium ($\sim$0.25) after $\sim$40 steps. Convergence demonstrates that computation = thermalization. Gray shaded regions show fluctuations around equilibrium values.
    \textbf{Top Right - Ternary operations in S-space:} Three-dimensional coordinate system showing three primitive operations. Blue arrow: Op 0 (Oscillate, refines $S_k$). Green arrow: Op 1 (Categorize, refines $S_t$). Red arrow: Op 2 (Partition, refines $S_e$). Axes: $S_k$, $S_t$, $S_e$ (all range 0.0-1.0). Operations act directly on three-dimensional S-entropy structure.
    \textbf{Middle Left - Thermodynamics from ternary computation:} Two traces versus computation step (0-140): red line (temperature $T$ in kelvin, left axis, range 180-280 K), blue dashed line (pressure $P$ in bar, right axis, range 0.50-0.75 bar). Both quantities equilibrate after $\sim$40 steps. Temperature and pressure computed directly from ternary trajectory statistics, not from energy or force.
    \textbf{Middle Center - Trit state evolution:} Heat map showing trit values for single molecule (12-trit register) over 100 computation steps. Horizontal axis: computation step (0-100). Vertical axis: trit position (0-10). Color coding: blue (trit 0, oscillatory), yellow (trit 1, categorical), red (trit 2, partition). Balanced color distribution indicates equal exploration of all three perspectives over time.}
    \label{fig:ternary_computation_2}
    \end{figure}

\subsection{Hardware Realization}

\begin{proposition}[Three-Phase Oscillator Encoding]
\label{prop:three_phase}
Three-phase oscillators with phase separation $2\pi/3$ provide natural ternary encoding:
\begin{equation}
\phi_i = \frac{2\pi i}{3}, \quad i \in \{0, 1, 2\}
\end{equation}
maps to trit values $\trit \in \{0, 1, 2\}$.
\end{proposition}

\begin{proof}
Three-phase system has three oscillators with phases:
\begin{align}
\phi_0 &= 0 \\
\phi_1 &= 2\pi/3 \\
\phi_2 &= 4\pi/3
\end{align}

At any instant, exactly one oscillator is in dominant phase (maximum amplitude). Define mapping:
\begin{equation}
\trit(t) = \argmax_{i \in \{0,1,2\}} |\cos(\omega t + \phi_i)|
\end{equation}

This encoding is bijective: each trit value corresponds to unique phase relationship. Physical implementation using three-phase AC power (ubiquitous in industrial applications) provides immediate hardware substrate for ternary logic.
\end{proof}

\subsection{Navigation Complexity}

\begin{theorem}[Logarithmic Navigation]
\label{thm:navigation}
Reaching target cell in $\Sspace$ requires $O(\log_3 n)$ operations for partition depth $n$.
\end{theorem}

\begin{proof}
At depth $k$, there are $3^k$ cells. To specify unique cell requires $k$ trits (Theorem \ref{thm:trit_coord}). For $3^k \approx n$ cells:
\begin{equation}
k = \log_3 n
\end{equation}

Each trit specifies one subdivision operation (constant time $O(1)$). Total operations:
\begin{equation}
\mathcal{O}(k) = \mathcal{O}(\log_3 n)
\end{equation}

Compared to binary search $O(\log_2 n)$, ternary navigation has same asymptotic complexity but with additional advantage: three-dimensional position is intrinsically encoded rather than requiring separate coordinate transformations.
\end{proof}

\begin{figure}[htbp]
    \centering
    \includegraphics[width=\textwidth]{figures/panel_ttr_d3.png}
    \caption{\textbf{Ternary Trajectory Recorder (TTR): $3^k$ Hierarchy Validation.}
    \textbf{(Top Left)} Trajectories in $3^k$ space for single molecule. Purple lines: trajectory path through three-dimensional S-entropy coordinates $(S_k, S_t, S_e)$. Green sphere: starting configuration. Red sphere: ending configuration. Trajectory explores bounded region [0.30, 0.70]$^3$, demonstrating confined dynamics in categorical phase space. Multiple trajectories shown to illustrate ensemble behavior.
    \textbf{(Top Center)} Trit sequence encodes trajectory as colored bar code. Horizontal axis: step number (0-50). Vertical axis: trit value (0, 1, 2). Blue bars: trit 0 (oscillatory perspective, refine $S_k$). Green bars: trit 1 (categorical perspective, refine $S_t$). Red bars: trit 2 (partition perspective, refine $S_e$). Balanced color distribution indicates equal usage of all three perspectives.
    \textbf{(Top Right)} Perspective balance quantifies trit distribution. Three bars show probability of each perspective: blue (oscillatory, 0.33), green (categorical, 0.32), red (partition, 0.33). Black dashed line: uniform distribution (1/3 $\approx$ 0.333). All three perspectives balanced to within 1\%, validating triple equivalence. Vertical axis: probability (0.00-0.35).
    \textbf{(Middle Left)} Mean squared displacement (MSD) distribution. Three-dimensional surface shows MSD versus depth and steps. Color gradient from purple (low MSD, $\sim$0.010) to yellow (high MSD, $\sim$0.030). Two traces overlaid: orange (radius of gyration), yellow (trajectory length/10). Surface demonstrates diffusive exploration of phase space.
    \textbf{(Middle Center)} Trajectory statistics distribution. Histogram shows count versus trit value (0.2-1.4). Peak at value $\sim$1.2 with count $\sim$8. Distribution skewed toward higher values, indicating preferential occupation of certain categorical regions. Vertical axis: count (0-8).
    \textbf{(Bottom Right)} Transition matrix shows perspective-switching probabilities. Heat map displays transition probability from one perspective (rows: Osc, Cat, Part) to another (columns: Osc, Cat, Part). }
    \label{fig:ttr_validation}
    \end{figure}

The ternary representation in $S$-entropy space thus provides:
\begin{itemize}
\item Natural encoding of three-dimensional structure
\item Information density enhancement $(3/2)^k$
\item Position-trajectory duality (address IS path)
\item Continuous emergence through infinite limits
\item Logarithmic navigation complexity
\item Direct hardware mapping to three-phase oscillators
\end{itemize}

This establishes ternary as the natural mathematical representation of bounded oscillatory systems in three-dimensional categorical space.
