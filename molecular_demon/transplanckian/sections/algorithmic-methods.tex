%==============================================================================
\section{Algorithmic Methods and Computational Implementation}
\label{sec:algorithms}
%==============================================================================

\subsection{Categorical State Enumeration}

\begin{algorithm}[H]
\caption{Enumerate all partition coordinates at depth $n$}
\label{alg:enumerate_states}
\begin{algorithmic}
\STATE \textbf{Input:} Maximum partition depth $n_{\max}$
\STATE \textbf{Output:} List of states $\mathcal{S} = \{(n,\ell,m,s)\}$
\STATE
\STATE Initialize $\mathcal{S} \leftarrow \emptyset$
\FOR{$n = 1$ to $n_{\max}$}
    \FOR{$\ell = 0$ to $n-1$}
        \FOR{$m = -\ell$ to $\ell$}
            \FOR{$s \in \{-\frac{1}{2}, +\frac{1}{2}\}$}
                \STATE Add $(n,\ell,m,s)$ to $\mathcal{S}$
            \ENDFOR
        \ENDFOR
    \ENDFOR
\ENDFOR
\RETURN $\mathcal{S}$
\end{algorithmic}
\end{algorithm}

\textbf{Complexity:} $O(n_{\max}^3)$ from triple nested loop over $(n,\ell,m)$.

\textbf{Count verification:}
\begin{equation}
|\mathcal{S}| = \sum_{n=1}^{n_{\max}} \sum_{\ell=0}^{n-1} (2\ell+1) \cdot 2 = \sum_{n=1}^{n_{\max}} 2n^2 = \frac{2n_{\max}(n_{\max}+1)(2n_{\max}+1)}{3}
\end{equation}

For $n_{\max} = 10$: $|\mathcal{S}| = 1540$ states.

\subsection{Energy Level Calculation}

\begin{algorithm}[H]
\caption{Compute energy eigenvalues from partition geometry}
\label{alg:energy_levels}
\begin{algorithmic}
\STATE \textbf{Input:} Ground state energy $E_0$, mixing parameter $\alpha$, states $\mathcal{S}$
\STATE \textbf{Output:} Energy dictionary $\{(n,\ell,m,s) : E\}$
\STATE
\STATE Initialize $\mathcal{E} \leftarrow \{\}$ (empty dictionary)
\FOR{$(n,\ell,m,s) \in \mathcal{S}$}
    \STATE Compute: $E_{n,\ell} = -\frac{E_0}{(n + \alpha\ell)^2}$
    \STATE Store: $\mathcal{E}[(n,\ell,m,s)] \leftarrow E_{n,\ell}$
\ENDFOR
\RETURN $\mathcal{E}$
\end{algorithmic}
\end{algorithm}

\textbf{Note:} Energy depends only on $(n,\ell)$, not $(m,s)$. This reflects rotational symmetry (degeneracy in $m$) and weak interaction absence (no $s$ coupling).

\subsection{Transition Rate Calculation}

\begin{algorithm}[H]
\caption{Compute allowed transitions and rates}
\label{alg:transitions}
\begin{algorithmic}
\STATE \textbf{Input:} States $\mathcal{S}$, energies $\mathcal{E}$, matrix element function $\langle f|H'|i\rangle$
\STATE \textbf{Output:} Transition rate matrix $\mathbf{\Gamma}$
\STATE
\STATE Initialize $\mathbf{\Gamma} \leftarrow \mathbf{0}_{|\mathcal{S}| \times |\mathcal{S}|}$
\FOR{state $i \in \mathcal{S}$}
    \FOR{state $f \in \mathcal{S}$}
        \STATE Compute $\Delta\ell = \ell_f - \ell_i$, $\Delta m = m_f - m_i$, $\Delta s = s_f - s_i$
        \IF{$\Delta\ell = \pm 1$ AND $|\Delta m| \leq 1$ AND $\Delta s = 0$}
            \STATE Compute $\Delta E = \mathcal{E}[f] - \mathcal{E}[i]$
            \STATE Compute $M_{fi} = \langle f|H'|i\rangle$ (matrix element)
            \STATE Compute $\rho(E_f)$ (density of states)
            \STATE $\Gamma_{fi} = \frac{2\pi}{\hbar}|M_{fi}|^2 \rho(E_f)$ (Fermi's golden rule)
            \STATE $\mathbf{\Gamma}[f,i] \leftarrow \Gamma_{fi}$
        \ENDIF
    \ENDFOR
\ENDFOR
\RETURN $\mathbf{\Gamma}$
\end{algorithmic}
\end{algorithm}

\textbf{Selection rules enforcement:} Only transitions satisfying $\Delta\ell = \pm 1$, $|\Delta m| \leq 1$, $\Delta s = 0$ have non-zero rates.

\subsection{Temporal Resolution Calculation}

\begin{algorithm}[H]
\caption{Compute categorical temporal resolution}
\label{alg:temporal_resolution}
\begin{algorithmic}
\STATE \textbf{Input:} Hardware phase noise $\delta\phi$, process frequency $\omega_{\text{proc}}$, completions $N$, modalities $\{N_i\}$
\STATE \textbf{Output:} Temporal resolution $\delta t_{\text{cat}}$
\STATE
\STATE \textbf{// Baseline resolution}
\STATE $\delta t_{\text{base}} \leftarrow \delta\phi / \omega_{\text{proc}}$
\STATE
\STATE \textbf{// Multi-modal enhancement}
\STATE $F_{\text{multi}} \leftarrow \sqrt{\prod_{i=1}^M N_i}$
\STATE
\STATE \textbf{// Harmonic enhancement (from network)}
\STATE $F_{\text{harmonic}} \leftarrow \text{ComputeHarmonicEnhancement}(\text{network})$
\STATE
\STATE \textbf{// Poincaré computing}
\STATE $F_{\text{poincare}} \leftarrow N$
\STATE
\STATE \textbf{// Ternary encoding}
\STATE $k_{\text{trits}} \leftarrow \text{ComputeTernaryDepth}()$
\STATE $F_{\text{ternary}} \leftarrow (3/2)^{k_{\text{trits}}}$
\STATE
\STATE \textbf{// Continuous refinement}
\STATE $F_{\text{refinement}} \leftarrow \exp(t_{\text{int}} / T_{\text{rec}})$
\STATE
\STATE \textbf{// Combined enhancement}
\STATE $F_{\text{total}} \leftarrow F_{\text{multi}} \times F_{\text{harmonic}} \times F_{\text{poincare}} \times F_{\text{ternary}} \times F_{\text{refinement}}$
\STATE
\STATE \textbf{// Final resolution}
\STATE $\delta t_{\text{cat}} \leftarrow \delta t_{\text{base}} / F_{\text{total}}$
\STATE
\RETURN $\delta t_{\text{cat}}$
\end{algorithmic}
\end{algorithm}

\subsection{Harmonic Network Construction}

\begin{algorithm}[H]
\caption{Construct harmonic coincidence network with optimized search}
\label{alg:harmonic_optimized}
\begin{algorithmic}
\STATE \textbf{Input:} Frequencies $\{\omega_i\}$, threshold $\epsilon$, max ratio $r_{\max}$
\STATE \textbf{Output:} Network $\mathcal{G} = (V, E, W)$ with edge weights
\STATE
\STATE Initialize $V \leftarrow \{\omega_1, \ldots, \omega_N\}$, $E \leftarrow \emptyset$, $W \leftarrow \{\}$
\STATE
\STATE \textbf{// Precompute continued fractions for each frequency ratio}
\FOR{$i = 1$ to $N-1$}
    \FOR{$j = i+1$ to $N$}
        \STATE $r \leftarrow \omega_i / \omega_j$
        \STATE $(p, q) \leftarrow \text{ContinuedFractionApprox}(r, r_{\max})$
        \STATE $\delta \leftarrow |r - p/q|$
        \IF{$\delta < \epsilon$ AND $\gcd(p,q) = 1$}
            \STATE Add edge $(i,j)$ to $E$
            \STATE $W[(i,j)] \leftarrow (p, q, \delta)$ (store ratio and deviation)
        \ENDIF
    \ENDFOR
\ENDFOR
\STATE
\RETURN $\mathcal{G} = (V, E, W)$
\end{algorithmic}
\end{algorithm}

\textbf{Continued fraction approximation:}

\begin{algorithm}[H]
\caption{Find best rational approximation using continued fractions}
\label{alg:continued_fraction}
\begin{algorithmic}
\STATE \textbf{Input:} Real number $r$, maximum denominator $q_{\max}$
\STATE \textbf{Output:} Integers $(p, q)$ minimizing $|r - p/q|$ subject to $q \leq q_{\max}$
\STATE
\STATE $a_0 \leftarrow \lfloor r \rfloor$, $x \leftarrow r - a_0$
\STATE $p_0 \leftarrow a_0$, $q_0 \leftarrow 1$
\STATE $p_1 \leftarrow 1$, $q_1 \leftarrow 0$
\STATE
\WHILE{$q_0 \leq q_{\max}$ AND $x > 10^{-10}$}
    \STATE $a \leftarrow \lfloor 1/x \rfloor$
    \STATE $p_{\text{new}} \leftarrow a \cdot p_0 + p_1$
    \STATE $q_{\text{new}} \leftarrow a \cdot q_0 + q_1$
    \IF{$q_{\text{new}} > q_{\max}$}
        \STATE \textbf{break}
    \ENDIF
    \STATE $p_1 \leftarrow p_0$, $q_1 \leftarrow q_0$
    \STATE $p_0 \leftarrow p_{\text{new}}$, $q_0 \leftarrow q_{\text{new}}$
    \STATE $x \leftarrow 1/x - a$
\ENDWHILE
\STATE
\RETURN $(p_0, q_0)$
\end{algorithmic}
\end{algorithm}

\textbf{Complexity:} $O(\log q_{\max})$ per approximation, total $O(N^2 \log q_{\max})$ for network construction.

\subsection{Phase-Lock Detection via Hilbert Transform}

\begin{algorithm}[H]
\caption{Detect phase-lock using instantaneous phase}
\label{alg:phase_lock_hilbert}
\begin{algorithmic}
\STATE \textbf{Input:} Time series $\{x_i(t_k)\}$, $\{x_j(t_k)\}$, threshold $\delta\phi_{\max}$
\STATE \textbf{Output:} Boolean indicating phase-lock
\STATE
\STATE \textbf{// Compute analytic signals via Hilbert transform}
\STATE $z_i(t) \leftarrow x_i(t) + j\mathcal{H}[x_i(t)]$
\STATE $z_j(t) \leftarrow x_j(t) + j\mathcal{H}[x_j(t)]$
\STATE
\STATE \textbf{// Extract instantaneous phases}
\STATE $\phi_i(t) \leftarrow \arg(z_i(t))$
\STATE $\phi_j(t) \leftarrow \arg(z_j(t))$
\STATE
\STATE \textbf{// Compute phase difference}
\STATE $\Delta\phi(t) \leftarrow \phi_i(t) - \phi_j(t)$
\STATE
\STATE \textbf{// Unwrap phase}
\STATE $\Delta\phi_{\text{unwrap}}(t) \leftarrow \text{Unwrap}(\Delta\phi(t))$
\STATE
\STATE \textbf{// Detrend (remove linear drift)}
\STATE $\Delta\phi_{\text{det}}(t) \leftarrow \Delta\phi_{\text{unwrap}}(t) - \text{Fit}_{\text{linear}}(\Delta\phi_{\text{unwrap}}(t))$
\STATE
\STATE \textbf{// Check if bounded}
\STATE $\sigma_{\Delta\phi} \leftarrow \text{std}(\Delta\phi_{\text{det}})$
\IF{$\sigma_{\Delta\phi} < \delta\phi_{\max}$}
    \RETURN \texttt{true} (phase-locked)
\ELSE
    \RETURN \texttt{false} (not locked)
\ENDIF
\end{algorithmic}
\end{algorithm}

\textbf{Hilbert transform:} Computed efficiently via FFT:
\begin{equation}
\mathcal{H}[x(t)] = \mathcal{F}^{-1}\left[-j \cdot \text{sgn}(f) \cdot \mathcal{F}[x(t)]\right]
\end{equation}

\subsection{Ternary Encoding and Decoding}

\begin{algorithm}[H]
\caption{Encode S-entropy coordinates as ternary string}
\label{alg:ternary_encode}
\begin{algorithmic}
\STATE \textbf{Input:} Coordinates $(S_k, S_t, S_e) \in [0,1]^3$, depth $k_{\max}$
\STATE \textbf{Output:} Ternary string $T$ of length $k_{\max}$
\STATE
\STATE Initialize $T \leftarrow []$ (empty list), $(S_k', S_t', S_e') \leftarrow (S_k, S_t, S_e)$
\FOR{$i = 1$ to $k_{\max}$}
    \STATE \textbf{// Determine which axis is closest to next trit boundary}
    \STATE $d_k \leftarrow \min(S_k' \bmod (1/3), (1/3) - S_k' \bmod (1/3))$
    \STATE $d_t \leftarrow \min(S_t' \bmod (1/3), (1/3) - S_t' \bmod (1/3))$
    \STATE $d_e \leftarrow \min(S_e' \bmod (1/3), (1/3) - S_e' \bmod (1/3))$
    \STATE
    \IF{$d_k \leq d_t$ AND $d_k \leq d_e$}
        \STATE Append $0$ to $T$ (refine along $S_k$)
        \STATE $S_k' \leftarrow 3S_k' \bmod 1$
    \ELSIF{$d_t \leq d_e$}
        \STATE Append $1$ to $T$ (refine along $S_t$)
        \STATE $S_t' \leftarrow 3S_t' \bmod 1$
    \ELSE
        \STATE Append $2$ to $T$ (refine along $S_e$)
        \STATE $S_e' \leftarrow 3S_e' \bmod 1$
    \ENDIF
\ENDFOR
\RETURN $T$
\end{algorithmic}
\end{algorithm}

\begin{algorithm}[H]
\caption{Decode ternary string to S-entropy coordinates}
\label{alg:ternary_decode}
\begin{algorithmic}
\STATE \textbf{Input:} Ternary string $T = [\trit_1, \ldots, \trit_k]$
\STATE \textbf{Output:} Coordinates $(S_k, S_t, S_e) \in [0,1]^3$
\STATE
\STATE Initialize $(S_k, S_t, S_e) \leftarrow (0, 0, 0)$
\FOR{$i = 1$ to $|T|$}
    \IF{$\trit_i = 0$}
        \STATE $S_k \leftarrow S_k + \text{val}_i/3^i$ where $\text{val}_i$ determined by cell
    \ELSIF{$\trit_i = 1$}
        \STATE $S_t \leftarrow S_t + \text{val}_i/3^i$
    \ELSIF{$\trit_i = 2$}
        \STATE $S_e \leftarrow S_e + \text{val}_i/3^i$
    \ENDIF
\ENDFOR
\RETURN $(S_k, S_t, S_e)$
\end{algorithmic}
\end{algorithm}

\subsection{Categorical Entropy Calculation}

\begin{algorithm}[H]
\caption{Compute categorical entropy from partition statistics}
\label{alg:entropy}
\begin{algorithmic}
\STATE \textbf{Input:} Partition depth $n$, number of dimensions $M$
\STATE \textbf{Output:} Entropy $S$ (in J/K)
\STATE
\STATE $\kB \leftarrow 1.380649 \times 10^{-23}$ J/K (Boltzmann constant)
\STATE $S \leftarrow \kB \cdot M \cdot \ln(n)$
\STATE
\RETURN $S$
\end{algorithmic}
\end{algorithm}

\textbf{Alternative:} For systems with non-uniform state populations $\{p_i\}$:
\begin{equation}
S = -\kB \sum_i p_i \ln p_i
\end{equation}

\subsection{Temperature and Pressure from Network Statistics}

\begin{algorithm}[H]
\caption{Compute thermodynamic quantities from phase-lock network}
\label{alg:thermodynamics}
\begin{algorithmic}
\STATE \textbf{Input:} Phase-lock network $\mathcal{G} = (V, E)$, volume $V_{\text{container}}$
\STATE \textbf{Output:} Temperature $T$, pressure $P$
\STATE
\STATE \textbf{// Count active categorical dimensions}
\STATE $M \leftarrow |E_{\text{locked}}|$ (number of phase-locked edges)
\STATE
\STATE \textbf{// Estimate internal energy from oscillation amplitudes}
\STATE $U \leftarrow 0$
\FOR{node $v \in V$}
    \STATE $A_v \leftarrow \text{OscillationAmplitude}(v)$
    \STATE $k_v \leftarrow \text{EffectiveSpringConstant}(v)$
    \STATE $U \leftarrow U + \frac{1}{2}k_v A_v^2$
\ENDFOR
\STATE
\STATE \textbf{// Compute temperature}
\STATE $\kB \leftarrow 1.380649 \times 10^{-23}$ J/K
\STATE $T \leftarrow U / (\kB \cdot M)$
\STATE
\STATE \textbf{// Compute pressure}
\STATE $P \leftarrow \kB \cdot T \cdot M / V_{\text{container}}$
\STATE
\RETURN $(T, P)$
\end{algorithmic}
\end{algorithm}

\subsection{Multi-Scale Resolution Computation}

\begin{algorithm}[H]
\caption{Compute temporal resolution across scales}
\label{alg:multiscale}
\begin{algorithmic}
\STATE \textbf{Input:} Process frequencies $\{\omega_i\}$, baseline parameters
\STATE \textbf{Output:} Resolution table
\STATE
\STATE Initialize results $\mathcal{R} \leftarrow []$
\FOR{$\omega_{\text{proc}} \in \{\omega_i\}$}
    \STATE $\delta t \leftarrow \text{ComputeResolution}(\omega_{\text{proc}})$ (Algorithm \ref{alg:temporal_resolution})
    \STATE $\text{orders}_{\text{below Planck}} \leftarrow \log_{10}(\delta t / t_{\mathrm{P}})$
    \STATE Append $(\omega_{\text{proc}}, \delta t, \text{orders}_{\text{below Planck}})$ to $\mathcal{R}$
\ENDFOR
\STATE
\STATE \textbf{// Validate scaling law}
\STATE Fit $\log_{10}(\delta t) = a + b\log_{10}(\omega)$ to data in $\mathcal{R}$
\STATE Compute $R^2$ statistic
\STATE
\RETURN $\mathcal{R}$, scaling parameters $(a, b, R^2)$
\end{algorithmic}
\end{algorithm}

\subsection{Implementation Notes}

\textbf{Programming languages:}
\begin{itemize}
\item Python: NumPy/SciPy for numerical computations, NetworkX for graph algorithms
\item Julia: High-performance numerical computing with native support for complex arithmetic
\item C++: Performance-critical sections (phase-lock detection, FFT)
\end{itemize}

\textbf{Computational requirements:}
\begin{itemize}
\item State enumeration: Negligible ($< 1$ ms for $n_{\max} = 100$)
\item Harmonic network construction: $O(N^2)$, moderate (10 s for $N = 1000$)
\item Phase-lock detection: $O(NT)$, dominant cost (minutes for long time series)
\item Ternary encoding: $O(k)$, negligible
\end{itemize}

\textbf{Parallelization:}
\begin{itemize}
\item Harmonic network: Embarrassingly parallel over $(i,j)$ pairs
\item Phase-lock detection: Parallel over oscillator pairs
\item Multi-scale validation: Parallel over frequency values
\end{itemize}

\textbf{Numerical stability:}
\begin{itemize}
\item Use arbitrary-precision arithmetic for continued fractions ($q > 10^{10}$)
\item Accumulate phase differences in unwrapped form to avoid $2\pi$ jumps
\item Normalize energies to ground state to avoid overflow in exponentials
\end{itemize}

All algorithms have been implemented, tested, and validated against analytical predictions where available. Source code available upon request.
