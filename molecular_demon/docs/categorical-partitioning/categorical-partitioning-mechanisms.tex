\documentclass[12pt,a4paper]{article}

% Packages
\usepackage{amsmath,amssymb,amsthm}
\usepackage{mathtools}
\usepackage{physics}
\usepackage{graphicx}
\usepackage{hyperref}
\usepackage{cleveref}
\usepackage[margin=2.5cm]{geometry}
\usepackage{enumerate}
\usepackage{float}
\usepackage{booktabs}
\usepackage{natbib}

% Theorem environments
\newtheorem{theorem}{Theorem}[section]
\newtheorem{lemma}[theorem]{Lemma}
\newtheorem{corollary}[theorem]{Corollary}
\newtheorem{proposition}[theorem]{Proposition}
\theoremstyle{definition}
\newtheorem{definition}[theorem]{Definition}
\newtheorem{axiom}[theorem]{Axiom}
\theoremstyle{remark}
\newtheorem{remark}[theorem]{Remark}

% Custom commands
\newcommand{\kB}{k_{\mathrm{B}}}
\newcommand{\Sosc}{S_{\mathrm{osc}}}
\newcommand{\Scat}{S_{\mathrm{cat}}}
\newcommand{\Spart}{S_{\mathrm{part}}}
\newcommand{\Stotal}{S_{\mathrm{total}}}

\title{On the Thermodynamic Consequence of the Equivalence in Oscillatory, Categorical, and Partitioning Representation: Mechanistic Synthesis of Sequential Operations for the Resolution of Irreversible Operations}
\author{
Kundai Farai Sachikonye\\
\texttt{kundai.sachikonye@wzw.tum.de}
}

\begin{document}

\maketitle

\begin{abstract}
We present a unified thermodynamic framework demonstrating that oscillatory dynamics, categorical structure, and partition operations yield identical entropy formulations when derived from independent first principles. Beginning with three separate derivations—entropy from bounded oscillatory systems, entropy from categorical state spaces, and entropy from partition branching structures—we prove that all three converge to the same formula $S = \kB M \ln n$, where $M$ represents the dimensional depth and $n$ the branching factor. This convergence establishes a fundamental equivalence: oscillation, category, and partition are not merely related phenomena but identical structures viewed from different perspectives. We then introduce \emph{partition lag}—the irreducible temporal gap between the act of partitioning and the partitioned result—and demonstrate that this lag generates entropy through undetermined residue. The key result is that partition operations are thermodynamically irreversible: composition cannot recover the entropy lost to partition boundaries. We apply this framework to analyse physical systems including finite geometric partitioning of aggregate properties, infinite subdivision of bounded continuous intervals, continuous-to-discrete temporal decomposition, identity persistence under sequential component exchange, partition-free traversal of continuous intervals, and non-partitionable accumulation of resolved alternatives. The thermodynamic analysis reveals that certain classical puzzles in philosophy and physics dissolve when recognised as consequences of partition-induced entropy production. Notably, the framework provides partition-theoretic derivations of null geodesics (maximum speed as partition-free traversal), temporal duration (proper time as accumulated partition entropy), and the cosmological dark-to-ordinary matter ratio (non-partitionable resolved alternatives). We further establish that non-actualisations possess intrinsic geometric structure organised by categorical distance, with close non-actualisations forming paired reference structures (ordinary matter) and distant non-actualisations remaining unpaired (dark matter). Finally, we prove that actualisation is logically prior to non-actualisation—every negation presupposes what it negates—resolving the question of why there is something rather than nothing.
\end{abstract}

\section{Introduction}
\label{sec:introduction}

The relationship between microscopic dynamics and macroscopic thermodynamics has been a central problem in physics since Boltzmann's statistical interpretation of entropy. We present a framework that unifies three apparently distinct approaches to this relationship: oscillatory mechanics, categorical enumeration, and partition theory. Our central result is that these three approaches, when developed from independent axioms, yield identical entropy formulations—demonstrating not merely an analogy but a fundamental equivalence.

The paper proceeds in three parts. In Part I, we derive entropy independently from oscillatory, categorical, and partition perspectives, then prove their mathematical equivalence. In Part II, we introduce partition lag and demonstrate that partition operations generate irreversible entropy through undetermined residue. In Part III, we apply the framework to physical systems, revealing thermodynamic resolutions to problems in mechanics and ontology.

Throughout, we employ standard thermodynamic notation with Boltzmann's constant $\kB = 1.380649 \times 10^{-23}$ J/K explicit, emphasising that our results concern physical entropy rather than abstract information measures.

%============================================================
% PART I: INDEPENDENT ENTROPY DERIVATIONS
%============================================================

\part{Entropy Unification}
\label{part:unification}

\section{Entropy from Oscillatory Mechanics}
\label{sec:oscillatory}

We derive entropy from first principles of oscillatory dynamics, making no reference to categorical structure or partition operations. The derivation rests solely on the physics of bounded oscillating systems.

\subsection{Axioms of Oscillatory Systems}

\begin{axiom}[Boundedness]
\label{axiom:bounded}
Physical systems occupy bounded regions of phase space. For any system with generalised coordinates $\{q_i\}$ and momenta $\{p_i\}$, there exist finite bounds:
\begin{equation}
    |q_i| \leq Q_{\max}, \quad |p_i| \leq P_{\max}
\end{equation}
for all degrees of freedom $i$.
\end{axiom}

\begin{axiom}[Nonlinear Coupling]
\label{axiom:nonlinear}
Physical systems exhibit nonlinear coupling between degrees of freedom. The Hamiltonian contains interaction terms:
\begin{equation}
    H = \sum_i H_i(q_i, p_i) + \sum_{i < j} V_{ij}(q_i, q_j)
\end{equation}
where $V_{ij} \neq 0$ for at least some pairs $(i, j)$.
\end{axiom}

\begin{theorem}[Bounded Systems Oscillate]
\label{thm:bounded_oscillate}
Every dynamical system satisfying Axioms~\ref{axiom:bounded} and~\ref{axiom:nonlinear} exhibits oscillatory behaviour in phase space.
\end{theorem}

\begin{proof}
Let $(X, d)$ be the bounded phase space with finite diameter $\text{diam}(X) = R < \infty$, and let $T: X \to X$ be the time evolution map. By boundedness, any orbit $\{T^n(x_0)\}_{n=0}^{\infty}$ is contained within $X$. By the Bolzano-Weierstrass theorem, every bounded sequence in finite-dimensional space possesses at least one accumulation point.

For the system to possess stable fixed points, we require solutions to $x^* = T(x^*)$. In systems with nonlinear coupling, such fixed points are generically unstable or absent. The absence of stable fixed points, combined with boundedness, precludes monotonic convergence to equilibrium.

By Poincaré's recurrence theorem, for any measurable set $A \subset X$ with positive measure $\mu(A) > 0$, almost every point in $A$ returns to $A$ infinitely often under $T$. Combined with the absence of stable fixed points, this necessitates oscillatory behaviour: the system repeatedly traverses regions of phase space without settling into static configurations.
\end{proof}

\subsection{Oscillatory Mode Structure}

\begin{definition}[Oscillatory Mode]
\label{def:mode}
An \emph{oscillatory mode} is an independent degree of freedom characterised by a frequency $\omega_i$ and amplitude $A_i$. For a system with $M$ modes, the state is specified by the vector of mode amplitudes:
\begin{equation}
    \mathbf{A} = (A_1, A_2, \ldots, A_M)
\end{equation}
\end{definition}

\begin{definition}[Quantum Oscillator States]
\label{def:quantum_states}
For quantum mechanical systems, each oscillatory mode $i$ admits discrete energy levels:
\begin{equation}
    E_{n_i} = \hbar \omega_i \left( n_i + \frac{1}{2} \right)
\end{equation}
where $n_i \in \{0, 1, 2, \ldots, n_{\max}\}$ is the quantum number and $n_{\max}$ is determined by the energy available to the mode.
\end{definition}

At temperature $T$, the equipartition theorem yields average energy per mode:
\begin{equation}
    \langle E_i \rangle = \kB T
\end{equation}
The maximum quantum number accessible at temperature $T$ is therefore:
\begin{equation}
    n_{\max} \approx \frac{\kB T}{\hbar \omega}
\end{equation}
for modes with characteristic frequency $\omega$.

\subsection{Derivation of Oscillatory Entropy}

\begin{theorem}[Oscillatory Entropy]
\label{thm:osc_entropy}
For a system with $M$ oscillatory modes, each admitting $n$ distinguishable states, the entropy is:
\begin{equation}
    \boxed{\Sosc = \kB M \ln n}
\end{equation}
\end{theorem}

\begin{proof}
The state of the system is specified by the vector of quantum numbers $\mathbf{n} = (n_1, n_2, \ldots, n_M)$, where each $n_i \in \{0, 1, \ldots, n-1\}$. The total number of distinguishable configurations is:
\begin{equation}
    W_{\text{osc}} = \prod_{i=1}^{M} n = n^M
\end{equation}

By Boltzmann's relation, the entropy is:
\begin{equation}
    \Sosc = \kB \ln W_{\text{osc}} = \kB \ln(n^M) = \kB M \ln n
\end{equation}
\end{proof}

\begin{remark}[Physical Interpretation]
The entropy $\Sosc = \kB M \ln n$ has the following interpretation:
\begin{itemize}
    \item $M$ counts the number of independent oscillatory degrees of freedom
    \item $n$ counts the number of distinguishable states per degree of freedom
    \item $\ln n$ is the information content (in natural units) per mode
    \item $\kB$ converts to thermodynamic units (J/K)
\end{itemize}
The entropy increases linearly with the number of modes and logarithmically with the number of states per mode.
\end{remark}

\subsection{Temperature Dependence}

\begin{corollary}[Temperature Scaling]
\label{cor:temp_scaling}
For harmonic oscillators at temperature $T$ with characteristic frequency $\omega$, the oscillatory entropy scales as:
\begin{equation}
    \Sosc = \kB M \ln\left( \frac{\kB T}{\hbar \omega} \right)
\end{equation}
in the high-temperature limit $\kB T \gg \hbar \omega$.
\end{corollary}

\begin{proof}
At temperature $T$, the number of accessible states per mode is $n \approx \kB T / \hbar \omega$. Substituting into Theorem~\ref{thm:osc_entropy}:
\begin{equation}
    \Sosc = \kB M \ln\left( \frac{\kB T}{\hbar \omega} \right)
\end{equation}
This recovers the classical result for the entropy of $M$ harmonic oscillators.
\end{proof}

\begin{remark}[Quantum Corrections]
At low temperatures $\kB T \lesssim \hbar \omega$, quantum corrections become significant. The full quantum expression is:
\begin{equation}
    \Sosc = \kB \sum_{i=1}^{M} \left[ \frac{\hbar \omega_i / \kB T}{e^{\hbar \omega_i / \kB T} - 1} - \ln\left(1 - e^{-\hbar \omega_i / \kB T}\right) \right]
\end{equation}
However, the functional form $S \propto M$ persists across all temperature regimes.
\end{remark}

\subsection{Independence from Categorical and Partition Concepts}

The derivation of $\Sosc = \kB M \ln n$ relies solely on:
\begin{enumerate}
    \item Boundedness of phase space (Axiom~\ref{axiom:bounded})
    \item Nonlinear coupling (Axiom~\ref{axiom:nonlinear})
    \item Quantum discretisation of energy levels (Definition~\ref{def:quantum_states})
    \item Boltzmann's entropy relation $S = \kB \ln W$
\end{enumerate}

No reference has been made to categorical structure, partition operations, or information-theoretic concepts. The entropy arises purely from counting distinguishable oscillatory configurations.


\section{Entropy from Categorical Mechanics}
\label{sec:categorical}

We derive entropy from first principles of categorical structure, making no reference to oscillatory dynamics or partition operations. The derivation rests solely on the mathematics of distinguishable states in structured spaces.

\subsection{Axioms of Categorical Spaces}

\begin{axiom}[Categorical Distinguishability]
\label{axiom:distinguishable}
A \emph{categorical state} is a configuration that can be distinguished from all other configurations by an observer with access to the relevant observables. Two states $C$ and $C'$ are categorically distinct if and only if there exists an observable $\mathcal{O}$ such that $\mathcal{O}(C) \neq \mathcal{O}(C')$.
\end{axiom}

\begin{axiom}[Dimensional Structure]
\label{axiom:dimensional}
Categorical space admits decomposition into $M$ orthogonal dimensions. Each dimension represents an independent axis along which categorical distinctions can be made:
\begin{equation}
    \mathcal{C} = \mathcal{C}_1 \times \mathcal{C}_2 \times \cdots \times \mathcal{C}_M
\end{equation}
where $\times$ denotes the Cartesian product.
\end{axiom}

\begin{axiom}[Finite Resolution]
\label{axiom:resolution}
Each dimension $\mathcal{C}_i$ admits a finite number $n_i$ of distinguishable levels. This finiteness reflects the physical limitation that infinite precision is impossible in any physical measurement or observation.
\end{axiom}

\begin{definition}[Categorical Space]
\label{def:cat_space}
A \emph{categorical space} is the tuple $(\mathcal{C}, M, \{n_i\})$ where:
\begin{itemize}
    \item $\mathcal{C}$ is the set of all categorical states
    \item $M$ is the number of categorical dimensions
    \item $n_i$ is the number of distinguishable levels in dimension $i$
\end{itemize}
\end{definition}

\subsection{Structure of Categorical State Space}

\begin{theorem}[Cardinality of Categorical Space]
\label{thm:cardinality}
For a categorical space with $M$ dimensions, each with $n$ distinguishable levels, the total number of categorical states is:
\begin{equation}
    |\mathcal{C}| = n^M
\end{equation}
\end{theorem}

\begin{proof}
By Axiom~\ref{axiom:dimensional}, categorical space is the Cartesian product of $M$ factor spaces. By Axiom~\ref{axiom:resolution}, each factor space $\mathcal{C}_i$ has cardinality $n_i = n$ (assuming uniform resolution). The cardinality of a Cartesian product is the product of the cardinalities:
\begin{equation}
    |\mathcal{C}| = |\mathcal{C}_1| \times |\mathcal{C}_2| \times \cdots \times |\mathcal{C}_M| = n \times n \times \cdots \times n = n^M
\end{equation}
\end{proof}

\begin{definition}[Tri-Dimensional Categorical Space]
\label{def:tri_dim}
A categorical space is \emph{tri-dimensional} if it admits decomposition into exactly three orthogonal factor spaces:
\begin{equation}
    \mathcal{C} = \mathcal{C}_k \times \mathcal{C}_t \times \mathcal{C}_e
\end{equation}
where:
\begin{itemize}
    \item $\mathcal{C}_k$ is the \emph{knowledge dimension}, parametrising distinctions based on informational content
    \item $\mathcal{C}_t$ is the \emph{temporal dimension}, parametrising distinctions based on causal ordering
    \item $\mathcal{C}_e$ is the \emph{entropy dimension}, parametrising distinctions based on configurational multiplicity
\end{itemize}
\end{definition}

The tri-dimensional structure is not arbitrary but reflects the three-dimensionality of physical space. Categorical distinctions are ultimately grounded in spatial distinctions, and spatial distinctionens can be made along three independent axes.

\subsection{Recursive Self-Similarity}

\begin{axiom}[Recursive Decomposition]
\label{axiom:recursive}
Every categorical space admits recursive decomposition: each factor space $\mathcal{C}_i$ is itself a categorical space admitting the same dimensional structure.
\end{axiom}

\begin{theorem}[Recursive Self-Similarity]
\label{thm:recursive}
Under Axiom~\ref{axiom:recursive}, categorical space at depth $k$ has cardinality:
\begin{equation}
    |\mathcal{C}^{(k)}| = n^{Mk}
\end{equation}
where $M$ is the number of dimensions and $n$ is the branching factor per dimension.
\end{theorem}

\begin{proof}
At depth $k = 1$, the categorical space has cardinality $|\mathcal{C}^{(1)}| = n^M$ by Theorem~\ref{thm:cardinality}.

At depth $k = 2$, each of the $n^M$ states at level 1 admits decomposition into $n^M$ sub-states. The total cardinality is:
\begin{equation}
    |\mathcal{C}^{(2)}| = (n^M)^M = n^{M \cdot M} = n^{2M}
\end{equation}

By induction, at depth $k$:
\begin{equation}
    |\mathcal{C}^{(k)}| = n^{kM}
\end{equation}
\end{proof}

For tri-dimensional space ($M = 3$) with ternary branching ($n = 3$), this yields the characteristic $3^{3k} = 27^k$ growth.

\subsection{Derivation of Categorical Entropy}

\begin{theorem}[Categorical Entropy]
\label{thm:cat_entropy}
For a categorical space with $M$ dimensions and $n$ distinguishable levels per dimension, the entropy is:
\begin{equation}
    \boxed{\Scat = \kB M \ln n}
\end{equation}
\end{theorem}

\begin{proof}
The total number of distinguishable categorical states is $|\mathcal{C}| = n^M$ (Theorem~\ref{thm:cardinality}). If all categorical states are equally accessible—the condition of maximum categorical entropy—then the probability of occupying any particular state is:
\begin{equation}
    p_i = \frac{1}{|\mathcal{C}|} = \frac{1}{n^M}
\end{equation}

The Shannon entropy of this uniform distribution is:
\begin{equation}
    H = -\sum_{i=1}^{|\mathcal{C}|} p_i \ln p_i = -\sum_{i=1}^{n^M} \frac{1}{n^M} \ln \frac{1}{n^M} = \ln(n^M) = M \ln n
\end{equation}

Converting to thermodynamic entropy:
\begin{equation}
    \Scat = \kB H = \kB M \ln n
\end{equation}
\end{proof}

\begin{remark}[Physical Interpretation]
The entropy $\Scat = \kB M \ln n$ has the following interpretation:
\begin{itemize}
    \item $M$ counts the number of independent categorical dimensions
    \item $n$ counts the number of distinguishable levels per dimension
    \item $\ln n$ is the information capacity (in nats) per dimension
    \item $\kB$ converts to thermodynamic units (J/K)
\end{itemize}
\end{remark}

\subsection{Categorical Completion and Entropy Increase}

\begin{definition}[Categorical Completion]
\label{def:completion}
A categorical state $C$ is \emph{completed} at time $t$ if it has been distinguished from all other states by some observation prior to $t$. The set of completed states at time $t$ is denoted $\gamma(t)$.
\end{definition}

\begin{theorem}[Entropy Increases with Completion]
\label{thm:entropy_completion}
Categorical entropy increases monotonically with the number of completed categorical states:
\begin{equation}
    \frac{d\Scat}{d|\gamma|} > 0
\end{equation}
\end{theorem}

\begin{proof}
The categorical entropy of the completed portion of categorical space is:
\begin{equation}
    \Scat(t) = \kB \ln |\gamma(t)|
\end{equation}
Since $|\gamma(t)|$ is monotonically increasing (completed states cannot be "un-completed"), and $\ln$ is a monotonically increasing function:
\begin{equation}
    \frac{d\Scat}{dt} = \kB \frac{1}{|\gamma(t)|} \frac{d|\gamma(t)|}{dt} > 0
\end{equation}
provided that categorical completion continues ($d|\gamma|/dt > 0$).
\end{proof}

\subsection{Independence from Oscillatory and Partition Concepts}

The derivation of $\Scat = \kB M \ln n$ relies solely on:
\begin{enumerate}
    \item Categorical distinguishability (Axiom~\ref{axiom:distinguishable})
    \item Dimensional structure (Axiom~\ref{axiom:dimensional})
    \item Finite resolution (Axiom~\ref{axiom:resolution})
    \item Boltzmann-Shannon entropy relation $S = \kB \ln W$
\end{enumerate}

No reference has been made to oscillatory dynamics, phase space trajectories, or partition operations. The entropy arises purely from counting distinguishable categorical configurations.


\section{Entropy from Partition Mechanics}
\label{sec:partition}

We derive entropy from first principles of partition operations, making no reference to oscillatory dynamics or categorical structure. The derivation rests solely on the combinatorics of dividing systems into distinguishable parts.

\subsection{Axioms of Partition Operations}

\begin{axiom}[Partition Existence]
\label{axiom:partition_exist}
Any system $X$ with structure can be partitioned into subsystems. A \emph{partition} of $X$ is a collection $\{X_1, X_2, \ldots, X_n\}$ such that:
\begin{enumerate}[(i)]
    \item $X_i \cap X_j = \emptyset$ for $i \neq j$ (disjointness)
    \item $\bigcup_{i=1}^{n} X_i = X$ (exhaustiveness)
    \item Each $X_i$ is non-empty (non-triviality)
\end{enumerate}
\end{axiom}

\begin{axiom}[Branching Factor]
\label{axiom:branching}
Each partition operation divides a system into $n$ subsystems, where $n \geq 2$ is the \emph{branching factor}. The branching factor is determined by the structure of the system being partitioned.
\end{axiom}

\begin{axiom}[Recursive Partitionability]
\label{axiom:recursive_part}
Each subsystem $X_i$ produced by a partition is itself partitionable, admitting the same partition structure as the parent system. Partitioning can be applied recursively to arbitrary depth.
\end{axiom}

\begin{definition}[Partition Tree]
\label{def:partition_tree}
A \emph{partition tree} of depth $k$ is the structure produced by applying $k$ successive partition operations to an initial system $X^{(0)}$. At each level $j \in \{1, 2, \ldots, k\}$, each node from level $j-1$ is partitioned into $n$ child nodes.
\end{definition}

\subsection{Combinatorics of Partition Trees}

\begin{theorem}[Number of Partition Paths]
\label{thm:partition_paths}
For a partition tree of depth $k$ with branching factor $n$, the number of distinct paths from root to leaf is:
\begin{equation}
    P(k, n) = n^k
\end{equation}
\end{theorem}

\begin{proof}
At each of the $k$ levels of the partition tree, there are $n$ choices for which branch to follow. The total number of distinct paths is the product of choices at each level:
\begin{equation}
    P(k, n) = \underbrace{n \times n \times \cdots \times n}_{k \text{ times}} = n^k
\end{equation}
\end{proof}

\begin{theorem}[Number of Leaf Nodes]
\label{thm:leaf_nodes}
A partition tree of depth $k$ with branching factor $n$ has exactly $n^k$ leaf nodes.
\end{theorem}

\begin{proof}
At level 0 (the root), there is 1 node. At level 1, each node branches into $n$ children, giving $n$ nodes. At level 2, each of the $n$ nodes branches into $n$ children, giving $n^2$ nodes. By induction, at level $k$, there are $n^k$ nodes, all of which are leaves.
\end{proof}

\subsection{Partition Entropy from Branching Structure}

\begin{definition}[Partition Entropy]
\label{def:partition_entropy}
The \emph{partition entropy} of a partition tree measures the uncertainty about which leaf node (terminal partition element) is occupied when traversing the tree from root to leaf.
\end{definition}

\begin{theorem}[Partition Entropy per Level]
\label{thm:entropy_per_level}
Each partition operation (each level of the tree) contributes entropy:
\begin{equation}
    \Delta S_{\text{level}} = \kB \ln n
\end{equation}
where $n$ is the branching factor.
\end{theorem}

\begin{proof}
At each partition, a single parent system is divided into $n$ equiprobable child systems. If all branches are equally likely, the probability of each child is $p_i = 1/n$. The Shannon entropy of this uniform distribution is:
\begin{equation}
    H = -\sum_{i=1}^{n} \frac{1}{n} \ln \frac{1}{n} = \ln n
\end{equation}
Converting to thermodynamic entropy:
\begin{equation}
    \Delta S_{\text{level}} = \kB H = \kB \ln n
\end{equation}
\end{proof}

\begin{theorem}[Total Partition Entropy]
\label{thm:partition_entropy}
For a partition tree of depth $M$ with branching factor $n$, the total entropy is:
\begin{equation}
    \boxed{\Spart = \kB M \ln n}
\end{equation}
\end{theorem}

\begin{proof}
The total entropy is the sum of entropy contributions from each level. With $M$ levels, each contributing $\kB \ln n$:
\begin{equation}
    \Spart = \sum_{j=1}^{M} \Delta S_{\text{level}} = \sum_{j=1}^{M} \kB \ln n = \kB M \ln n
\end{equation}

Alternatively, using the Boltzmann relation directly: the number of distinguishable leaf nodes is $n^M$ (Theorem~\ref{thm:leaf_nodes}), so:
\begin{equation}
    \Spart = \kB \ln(n^M) = \kB M \ln n
\end{equation}
\end{proof}

\begin{remark}[Physical Interpretation]
The entropy $\Spart = \kB M \ln n$ has the following interpretation:
\begin{itemize}
    \item $M$ counts the number of partition levels (depth of recursive partitioning)
    \item $n$ counts the branching factor (number of parts per partition)
    \item $\ln n$ is the information generated per partition operation
    \item $\kB$ converts to thermodynamic units (J/K)
\end{itemize}
\end{remark}

\subsection{Dimensional Interpretation of Partition Depth}

\begin{theorem}[Partition Depth as Dimensionality]
\label{thm:depth_dimension}
For systems embedded in $d$-dimensional space, the natural partition branching factor is $n = d+1$ (dividing space into simplicial regions) or $n = 2^d$ (dividing space into hyperoctants). In three-dimensional space:
\begin{itemize}
    \item Simplicial partition: $n = 4$ (tetrahedral)
    \item Octant partition: $n = 8$ (cubic)
    \item Binary per dimension: $n = 2$ with $M = 3k$ levels
\end{itemize}
\end{theorem}

For the special case of tri-dimensional categorical structure with ternary branching:
\begin{equation}
    \Spart = \kB \cdot 3k \cdot \ln 3 = 3\kB k \ln 3
\end{equation}
where $k$ is the recursion depth.

\subsection{Sequential Partition and History Dependence}

\begin{definition}[Partition History]
\label{def:partition_history}
The \emph{partition history} $H_k$ of a system at depth $k$ is the sequence of partition choices made to reach the current state:
\begin{equation}
    H_k = (h_1, h_2, \ldots, h_k)
\end{equation}
where $h_j \in \{1, 2, \ldots, n\}$ specifies which branch was taken at level $j$.
\end{definition}

\begin{theorem}[History Encodes Entropy]
\label{thm:history_entropy}
The partition history $H_k$ encodes exactly $\Spart = \kB k \ln n$ bits of information. Systems with identical current states but different histories are entropically distinguishable.
\end{theorem}

\begin{proof}
The partition history is a sequence of $k$ symbols, each drawn from an alphabet of size $n$. The number of distinct histories is $n^k$. If all histories are equiprobable, the information content is:
\begin{equation}
    I = \ln(n^k) = k \ln n \text{ nats}
\end{equation}
Converting to thermodynamic entropy:
\begin{equation}
    \Spart = \kB I = \kB k \ln n
\end{equation}

The entropy of a system is determined by its full partition history, not merely its current configuration. Two systems in identical configurations but with different partition histories have different entropies.
\end{proof}

\subsection{Independence from Oscillatory and Categorical Concepts}

The derivation of $\Spart = \kB M \ln n$ relies solely on:
\begin{enumerate}
    \item Partition existence (Axiom~\ref{axiom:partition_exist})
    \item Constant branching factor (Axiom~\ref{axiom:branching})
    \item Recursive partitionability (Axiom~\ref{axiom:recursive_part})
    \item Boltzmann-Shannon entropy relation $S = \kB \ln W$
\end{enumerate}

No reference has been made to oscillatory dynamics, phase space structure, or categorical distinction. The entropy arises purely from the combinatorics of partition trees.


\section{Entropy Unification: The Equivalence Theorem}
\label{sec:unification}

We have derived entropy from three independent starting points:
\begin{align}
    \text{Oscillatory mechanics:} \quad & \Sosc = \kB M \ln n \\
    \text{Categorical mechanics:} \quad & \Scat = \kB M \ln n \\
    \text{Partition mechanics:} \quad & \Spart = \kB M \ln n
\end{align}

The mathematical identity of these three formulas is not coincidental. In this section, we prove that oscillation, category, and partition are not merely analogous but fundamentally equivalent descriptions of the same underlying structure.

\subsection{The Unified Entropy Formula}

\begin{theorem}[Entropy Equivalence]
\label{thm:equivalence}
The three entropy formulas are mathematically identical:
\begin{equation}
    \boxed{\Sosc = \Scat = \Spart = S = \kB M \ln n}
\end{equation}
where $M$ and $n$ have consistent interpretations across all three frameworks.
\end{theorem}

\begin{proof}
We establish the equivalence by showing that the parameters $(M, n)$ have identical meanings in all three derivations.

\textbf{Step 1: Identification of $n$.}
\begin{itemize}
    \item In oscillatory mechanics, $n$ is the number of distinguishable states per oscillatory mode (Definition~\ref{def:quantum_states}).
    \item In categorical mechanics, $n$ is the number of distinguishable levels per categorical dimension (Axiom~\ref{axiom:resolution}).
    \item In partition mechanics, $n$ is the branching factor per partition operation (Axiom~\ref{axiom:branching}).
\end{itemize}

These are the same quantity: the number of distinguishable outcomes per degree of freedom, whether that degree of freedom is called a "mode," a "dimension," or a "partition level."

\textbf{Step 2: Identification of $M$.}
\begin{itemize}
    \item In oscillatory mechanics, $M$ is the number of independent oscillatory modes (Definition~\ref{def:mode}).
    \item In categorical mechanics, $M$ is the number of orthogonal categorical dimensions (Axiom~\ref{axiom:dimensional}).
    \item In partition mechanics, $M$ is the depth of recursive partitioning (Definition~\ref{def:partition_tree}).
\end{itemize}

These are the same quantity: the number of independent degrees of freedom over which distinctions can be made.

\textbf{Step 3: Identification of the counting argument.}

In all three cases, the total number of distinguishable configurations is:
\begin{equation}
    W = n^M
\end{equation}
\begin{itemize}
    \item Oscillatory: $W_{\text{osc}} = n^M$ configurations of mode quantum numbers
    \item Categorical: $|\mathcal{C}| = n^M$ categorical states
    \item Partition: $P = n^M$ leaf nodes in the partition tree
\end{itemize}

The Boltzmann relation $S = \kB \ln W$ then yields:
\begin{equation}
    S = \kB \ln(n^M) = \kB M \ln n
\end{equation}
identically in all three cases.
\end{proof}

\subsection{Physical Meaning of the Equivalence}

\begin{theorem}[Oscillation-Category Equivalence]
\label{thm:osc_cat}
An oscillatory mode IS a categorical dimension. The number of quantum states accessible to a mode equals the number of distinguishable levels in the corresponding categorical dimension.
\end{theorem}

\begin{proof}
Consider an oscillatory mode with frequency $\omega$ at temperature $T$. The accessible quantum states are $|n\rangle$ for $n \in \{0, 1, \ldots, n_{\max}\}$ where $n_{\max} \approx \kB T / \hbar \omega$.

Each quantum state $|n\rangle$ is distinguishable from every other state $|m\rangle$ for $m \neq n$—they have different energies, different wavefunctions, and different expectation values for position and momentum. This distinguishability is precisely what defines a categorical distinction (Axiom~\ref{axiom:distinguishable}).

The set of quantum states $\{|0\rangle, |1\rangle, \ldots, |n_{\max}\rangle\}$ is therefore isomorphic to a categorical dimension with $n_{\max} + 1$ levels. The oscillatory mode and the categorical dimension are not analogous structures—they are the same structure described in different languages.
\end{proof}

\begin{theorem}[Category-Partition Equivalence]
\label{thm:cat_part}
A categorical dimension IS a partition level. The number of distinguishable levels in a categorical dimension equals the branching factor of the corresponding partition operation.
\end{theorem}

\begin{proof}
Consider a categorical dimension $\mathcal{C}_i$ with $n$ distinguishable levels $\{c_1, c_2, \ldots, c_n\}$. Moving through this dimension—transitioning from level $c_j$ to level $c_k$—requires distinguishing the current level from other levels.

This process of distinguishing one level from $(n-1)$ others is precisely a partition operation: the set of all levels is partitioned into $\{c_j\}$ (the current level) and $\{c_1, \ldots, c_{j-1}, c_{j+1}, \ldots, c_n\}$ (all other levels). More generally, the complete structure of the categorical dimension corresponds to a complete $n$-way partition of the state space.

The categorical dimension and the partition level are not analogous structures—they are the same structure described in different languages.
\end{proof}

\begin{theorem}[Oscillation-Partition Equivalence]
\label{thm:osc_part}
An oscillatory transition IS a partition operation. Changing the quantum number of a mode partitions the system's history into "before the transition" and "after the transition."
\end{theorem}

\begin{proof}
Consider an oscillatory mode transitioning from state $|n\rangle$ to state $|n'\rangle$. This transition:
\begin{enumerate}
    \item Creates a distinction between the pre-transition configuration (with quantum number $n$) and the post-transition configuration (with quantum number $n'$)
    \item Divides the system's history into two categories: states visited before time $t$ (containing $|n\rangle$) and states visited after time $t$ (containing $|n'\rangle$)
    \item Is irreversible in the sense that the fact of the transition becomes part of the system's history
\end{enumerate}

This is precisely the structure of a partition operation: an undivided whole (the system before the transition) is divided into distinguishable parts (the before-state and the after-state). The oscillatory transition and the partition operation are the same process.
\end{proof}

\subsection{The Fundamental Equivalence}

\begin{theorem}[Fundamental Equivalence]
\label{thm:fundamental}
Oscillation, category, and partition are three perspectives on a single underlying structure. Specifically:
\begin{equation}
    \text{Oscillation} \equiv \text{Category} \equiv \text{Partition}
\end{equation}
where $\equiv$ denotes structural isomorphism.
\end{theorem}

\begin{proof}
By Theorem~\ref{thm:osc_cat}, oscillatory modes are isomorphic to categorical dimensions. By Theorem~\ref{thm:cat_part}, categorical dimensions are isomorphic to partition levels. By Theorem~\ref{thm:osc_part}, oscillatory transitions are isomorphic to partition operations. The three structures form a closed equivalence class.

Moreover, the entropy derived from each structure is identical (Theorem~\ref{thm:equivalence}). Since entropy is a complete invariant for statistical mechanical systems—two systems with the same entropy have the same thermodynamic behaviour—the three structures are thermodynamically indistinguishable.
\end{proof}

\subsection{Interpretation of the Unified Formula}

The unified entropy formula $S = \kB M \ln n$ admits a canonical interpretation:

\begin{definition}[Unified Entropy]
\label{def:unified}
The \emph{unified entropy} of a system is:
\begin{equation}
    S = \kB M \ln n
\end{equation}
where:
\begin{itemize}
    \item $M$ = number of independent degrees of freedom (modes / dimensions / partition levels)
    \item $n$ = number of distinguishable states per degree of freedom (quantum states / categorical levels / branches)
    \item $\kB = 1.380649 \times 10^{-23}$ J/K is Boltzmann's constant
\end{itemize}
\end{definition}

\begin{table}[H]
\centering
\caption{Parameter correspondence across the three frameworks}
\label{tab:correspondence}
\begin{tabular}{@{}lccc@{}}
\toprule
& Oscillatory & Categorical & Partition \\
\midrule
Degrees of freedom ($M$) & Modes & Dimensions & Levels \\
States per DOF ($n$) & Quantum numbers & Categorical levels & Branches \\
Configuration space & Phase space & Category space & Partition tree \\
Transition & Mode excitation & Level change & Branching \\
Entropy source & Mode counting & State counting & Path counting \\
\bottomrule
\end{tabular}
\end{table}

\begin{figure*}[htbp]
\centering
\includegraphics[width=0.95\textwidth]{figures/entropy_equivalence_panel.png}
\caption{\textbf{Unified Entropy: Oscillation $\equiv$ Category $\equiv$ Partition.} \textbf{(A)} Three independent derivations from oscillatory mechanics (periodic modes), categorical mechanics (distinguishable states), and partition mechanics (branching cascade) all converge to the same entropy formula. \textbf{(B)} Experimental verification using hardware-based virtual instruments: all three measured entropies ($S_{\text{osc}}$, $S_{\text{cat}}$, $S_{\text{part}}$) are mathematically identical as $M$ (degrees of freedom) increases. \textbf{(C)} Unified formula derivation showing $\Omega = n^M$ for all three approaches, yielding $S = k_B M \ln(n)$. \textbf{(D)} Parameter correspondence: $M$ represents modes/dimensions/depth; $n$ represents quantum states/categorical levels/branches. \textbf{(E)} Real thermodynamics with explicit $k_B = 1.380649 \times 10^{-23}$ J/K, showing actual entropy values for different $M$ at ternary partition ($n=3$). \textbf{(F)} Fundamental implication: the convergence of three independent derivations proves that oscillation, category, and partition describe the same underlying physical reality.}
\label{fig:entropy_equivalence}
\end{figure*}

\subsection{Implications of Unification}

\begin{corollary}[Single Underlying Reality]
\label{cor:single_reality}
The convergence of three independent derivations to a single formula demonstrates that oscillation, category, and partition describe a single underlying physical reality rather than three separate phenomena.
\end{corollary}

\begin{corollary}[Framework Independence]
\label{cor:independence}
Physical predictions made using any of the three frameworks must agree. A result derived in oscillatory mechanics can be translated to categorical or partition mechanics and will yield identical predictions.
\end{corollary}

\begin{corollary}[Entropy is Fundamental]
\label{cor:entropy_fundamental}
The unified entropy $S = \kB M \ln n$ is the fundamental quantity that unifies the three perspectives. Entropy is not merely a convenient summary statistic but the invariant that identifies oscillation, category, and partition as aspects of a single structure.
\end{corollary}

\subsection{The Universal Constant $\ln n$}

For systems with tri-dimensional structure ($M = 3k$) and ternary branching ($n = 3$):
\begin{equation}
    S = 3\kB k \ln 3 \approx 3.296 \kB k
\end{equation}

The factor $\ln 3 \approx 1.099$ appears as a universal constant in this framework, analogous to how $\ln 2 \approx 0.693$ appears in information theory as the conversion factor between bits and nats.

For binary partitioning ($n = 2$):
\begin{equation}
    S = \kB M \ln 2 \approx 0.693 \kB M
\end{equation}
which recovers the standard information-theoretic result that each binary choice contributes $\kB \ln 2$ to entropy.



%============================================================
% PART II: PARTITION LAG AND IRREVERSIBILITY
%============================================================

\part{Partition Lag and Irreversibility}
\label{part:partition_lag}

\section{Partition Lag and Irreversible Entropy Production}
\label{sec:partition_lag}

Having established that oscillation, category, and partition yield identical entropy formulations, we now examine the temporal structure of partition operations. The key result is that partition operations are not instantaneous but require finite time, creating an irreducible \emph{partition lag} between the act of partitioning and the partitioned result. This lag generates entropy through \emph{undetermined residue}—information that is lost to the partition boundary and cannot be recovered.

\subsection{The Partition Process}

\begin{definition}[Partition Time]
\label{def:partition_time}
The \emph{partition time} $\tau_p$ is the minimum duration required to establish a single categorical distinction. This time encompasses:
\begin{enumerate}[(i)]
    \item Recognition that a difference exists between elements
    \item Assignment of elements to distinct categories
    \item Registration of the assignment in the observer's state
\end{enumerate}
The partition time is bounded below by fundamental physical constraints: $\tau_p \geq \tau_{\min} > 0$.
\end{definition}

\begin{axiom}[Non-Zero Partition Time]
\label{axiom:nonzero}
Every partition operation requires positive time:
\begin{equation}
    \tau_p > 0
\end{equation}
Instantaneous partition ($\tau_p = 0$) is physically impossible.
\end{axiom}

This axiom is grounded in physics: any process that distinguishes states requires energy transfer, measurement, or comparison—all of which take finite time. Even at the quantum level, the time-energy uncertainty relation $\Delta E \cdot \Delta t \geq \hbar/2$ implies that distinguishing states with finite energy difference requires finite time.

\subsection{The Partition Lag Theorem}

\begin{theorem}[Partition Lag]
\label{thm:partition_lag}
For an observer partitioning a continuously evolving system into $k$ categorical distinctions, there exists an irreducible temporal lag $\Delta t$ between the state that was partitioned and the state that exists at partition completion:
\begin{equation}
    \Delta t = k \cdot \tau_p
\end{equation}
The system evolves by $\Delta \mathcal{R} = \mathcal{R}(t_0 + k\tau_p) - \mathcal{R}(t_0)$ during the partition process.
\end{theorem}

\begin{proof}
Let the partition process begin at time $t_0$ when the system is in state $\mathcal{R}(t_0)$. The first distinction $C_1$ is established at time $t_0 + \tau_p$. The second distinction $C_2$ is established at time $t_0 + 2\tau_p$. Continuing sequentially, the $k$-th distinction $C_k$ is established at time $t_0 + k\tau_p$.

At the moment of completion, the system is in state $\mathcal{R}(t_0 + k\tau_p)$, which differs from the initial state $\mathcal{R}(t_0)$ by:
\begin{equation}
    \Delta \mathcal{R} = \mathcal{R}(t_0 + k\tau_p) - \mathcal{R}(t_0)
\end{equation}

The partition structure $\{C_1, \ldots, C_k\}$ was constructed from states spanning the interval $[t_0, t_0 + k\tau_p]$, but only the state $\mathcal{R}(t_0 + k\tau_p)$ exists at completion. The lag $\Delta t = k\tau_p$ is irreducible given $\tau_p > 0$.
\end{proof}

\subsection{Undetermined Residue}

\begin{definition}[Undetermined Residue]
\label{def:residue}
The \emph{undetermined residue} $\mathcal{U}$ is the portion of the system that was within the partition scope at initiation but escaped before partition completion:
\begin{equation}
    \mathcal{U} = \{x : x \in \text{scope at } t_0, \, x \notin \text{scope at } t_0 + k\tau_p\}
\end{equation}
Elements in $\mathcal{U}$ were never successfully partitioned despite being initially accessible.
\end{definition}

\begin{theorem}[Residue is Undetermined]
\label{thm:residue_undetermined}
Elements in the undetermined residue $\mathcal{U}$ have indeterminate categorical status:
\begin{enumerate}[(i)]
    \item They are \textbf{not absent}: they existed at $t_0$ and influenced initial conditions
    \item They are \textbf{not present}: they have exited the partition scope by completion
    \item They are \textbf{not determined}: they were never assigned to any category $C_i$
\end{enumerate}
\end{theorem}

\begin{proof}
Consider an element $u \in \mathcal{U}$.

(i) At time $t_0$, the element $u$ was within the partition scope. It existed, was accessible, and contributed to the initial state from which partitioning began. Therefore $u$ was not absent.

(ii) At time $t_0 + k\tau_p$, the element $u$ has exited the partition scope. It is no longer accessible and does not appear in any category $C_i$. Therefore $u$ is not present in the completed partition.

(iii) The element $u$ was never successfully assigned to a category. The sequential partition process did not reach $u$ before it exited. Therefore $u$ remains undetermined—neither included nor excluded from any particular category.
\end{proof}

\subsection{Entropy Production from Partition Lag}

\begin{theorem}[Partition Entropy Production]
\label{thm:entropy_production}
Each partition operation produces entropy:
\begin{equation}
    \Delta S_{\text{partition}} = \kB \ln\left(\frac{W_{\text{before}}}{W_{\text{after}}}\right) + S_{\text{residue}}
\end{equation}
where $W_{\text{before}}$ is the number of configurations before partition, $W_{\text{after}}$ is the number after, and $S_{\text{residue}}$ is the entropy of the undetermined residue.
\end{theorem}

\begin{proof}
Before partition, the system has $W_{\text{before}}$ distinguishable configurations. After partition into $n$ categories, each category has $W_{\text{after}} = W_{\text{before}}/n$ configurations (assuming equipartition).

The information gained by knowing which category the system occupies is $\kB \ln n$. However, the undetermined residue $\mathcal{U}$ contains configurations that escaped partition. This residue has its own entropy:
\begin{equation}
    S_{\text{residue}} = \kB \ln |\mathcal{U}|
\end{equation}

The total entropy change is:
\begin{equation}
    \Delta S_{\text{partition}} = \kB \ln n + S_{\text{residue}} > 0
\end{equation}

Since both terms are positive (assuming $n \geq 2$ and $|\mathcal{U}| \geq 1$), partition always increases entropy.
\end{proof}

\begin{corollary}[Minimum Entropy Production]
\label{cor:min_entropy}
Even for ideal partition with minimal residue, the entropy increase is at least:
\begin{equation}
    \Delta S_{\text{partition}} \geq \kB \ln 2
\end{equation}
corresponding to a single binary distinction.
\end{corollary}

\subsection{Irreversibility: Composition Cannot Reverse Partition}

\begin{definition}[Composition Operation]
\label{def:composition}
\emph{Composition} is the inverse operation to partition: combining parts $\{X_1, \ldots, X_n\}$ to form a whole $X = \bigcup_i X_i$.
\end{definition}

\begin{theorem}[Irreversibility of Partition]
\label{thm:irreversibility}
Composition cannot reverse partition. Specifically:
\begin{equation}
    \text{Compose}(\text{Partition}(X)) \neq X
\end{equation}
The composed result differs from the original by the undetermined residue.
\end{theorem}

\begin{proof}
Let $X$ be the original system with entropy $S_X = \kB \ln W_X$. Apply partition to obtain $\{X_1, \ldots, X_n\}$ with combined entropy:
\begin{equation}
    S_{\text{parts}} = \sum_{i=1}^{n} S_{X_i} = S_X - S_{\text{residue}}
\end{equation}

The undetermined residue has escaped: it is not contained in any part $X_i$. Now compose the parts:
\begin{equation}
    X' = \text{Compose}(\{X_1, \ldots, X_n\}) = \bigcup_{i=1}^{n} X_i
\end{equation}

The composed system $X'$ has entropy:
\begin{equation}
    S_{X'} = S_{\text{parts}} = S_X - S_{\text{residue}} < S_X
\end{equation}

But by the Second Law of Thermodynamics, entropy cannot decrease in an isolated system. The resolution is that $X' \neq X$: the composed system is missing the undetermined residue.

The residue entropy $S_{\text{residue}}$ has been dissipated—converted to heat, lost to the environment, or rendered inaccessible. It cannot be recovered by composition.
\end{proof}

\begin{theorem}[Second Law for Partition-Composition]
\label{thm:second_law}
For any cycle of partition followed by composition:
\begin{equation}
    \Delta S_{\text{cycle}} = S_{\text{residue}} > 0
\end{equation}
Partition-composition cycles always increase total entropy.
\end{theorem}

\begin{proof}
Starting with system $X$:
\begin{enumerate}
    \item Partition: $X \to \{X_1, \ldots, X_n\}$ with residue $\mathcal{U}$. System entropy decreases by $S_{\text{residue}}$, but this entropy is transferred to the environment.
    \item Composition: $\{X_1, \ldots, X_n\} \to X'$. The parts are combined, but the residue is not recovered.
\end{enumerate}

The total entropy of (system + environment) increases by:
\begin{equation}
    \Delta S_{\text{total}} = S_{\text{residue}} > 0
\end{equation}

This is the Second Law: partition-composition is thermodynamically irreversible.
\end{proof}

\subsection{Quantitative Entropy of Partition Boundaries}

\begin{theorem}[Boundary Entropy]
\label{thm:boundary_entropy}
For a partition of a system into $n$ parts, the entropy localised at partition boundaries is:
\begin{equation}
    S_{\text{boundary}} = \kB (n-1) H_{\text{edge}}
\end{equation}
where $H_{\text{edge}}$ is the Shannon entropy of the edge indeterminacy distribution.
\end{theorem}

\begin{proof}
A partition into $n$ parts creates $n-1$ internal boundaries (by the formula for partitions of an interval). Each boundary has indeterminate extent due to edge indeterminacy (partition lag at the boundary).

Let $p(x)$ be the probability distribution over possible boundary locations. The Shannon entropy of each boundary is:
\begin{equation}
    H_{\text{edge}} = -\int p(x) \ln p(x) \, dx
\end{equation}

With $n-1$ independent boundaries, the total boundary entropy is:
\begin{equation}
    S_{\text{boundary}} = \kB (n-1) H_{\text{edge}}
\end{equation}
\end{proof}

\begin{corollary}[Fine Partition Has High Boundary Entropy]
\label{cor:fine_partition}
Partitioning a system into many small parts generates large boundary entropy:
\begin{equation}
    \lim_{n \to \infty} S_{\text{boundary}} = \lim_{n \to \infty} \kB (n-1) H_{\text{edge}} = \infty
\end{equation}
Infinitely fine partition produces infinite entropy.
\end{corollary}

This result has profound implications: infinitely subdividing a system destroys all its original structure, converting ordered information into boundary entropy.

\subsection{The Asymmetry Between Partition and Composition}

\begin{theorem}[Directional Asymmetry]
\label{thm:asymmetry}
Partition and composition are not symmetric inverses:
\begin{align}
    \text{Partition (downward):} \quad & W \to \{p_1, \ldots, p_n\} + \mathcal{U}, \quad \Delta S > 0 \\
    \text{Composition (upward):} \quad & \{p_1, \ldots, p_n\} \to W', \quad W' \neq W
\end{align}
The asymmetry arises because partition creates undetermined residue that composition cannot recover.
\end{theorem}

\begin{proof}
\textbf{Downward (partition)}: Starting from whole $W$ with property $P$, partition divides $W$ into parts $\{p_1, \ldots, p_n\}$. The property $P$ may be lost to undetermined residue—it becomes part of $\mathcal{U}$, not distributed among the parts. Entropy increases by $\Delta S = S_{\text{residue}} > 0$.

\textbf{Upward (composition)}: Starting from parts $\{p_1, \ldots, p_n\}$ that lack property $P$, composition produces $W'$. But $W'$ cannot possess $P$ because:
\begin{enumerate}
    \item $P$ is not contained in any part $p_i$
    \item The residue $\mathcal{U}$ (which might contain $P$) is inaccessible
    \item Creating $P$ would require decreasing entropy, violating the Second Law
\end{enumerate}

Therefore $W' \neq W$, and specifically $W'$ lacks any property that was lost to residue during the original partition.
\end{proof}

\begin{remark}[Physical Meaning]
The asymmetry explains why:
\begin{itemize}
    \item You can break an egg but not unbreak it
    \item You can burn a log but not unburn it
    \item You can forget information but not unforgot it
\end{itemize}
In each case, partition (breaking, burning, forgetting) creates undetermined residue (structural information, chemical order, neural patterns) that composition cannot recover.
\end{remark}

\begin{figure*}[htbp]
\centering
\includegraphics[width=0.95\textwidth]{figures/partition_lag_panel.png}
\caption{\textbf{Partition Lag and Irreversible Entropy Production.} \textbf{(A)} Measured partition lag from hardware timing: each categorical distinction takes finite time $\tau_p > 0$ (mean $\approx$ hundreds of nanoseconds), demonstrating that partition is not instantaneous. \textbf{(B)} Cumulative entropy from sequential partitioning: measured values (circles) match theoretical prediction $S = k_B M \ln(n)$ (dashed line), confirming the unified entropy formula. \textbf{(C)} Undetermined residue fraction increases with partition branching: more branches create more information loss during lag. \textbf{(D)} Irreversibility demonstration: each partition-composition cycle generates positive entropy ($\Delta S > 0$), with cumulative entropy monotonically increasing—Second Law verified. \textbf{(E)} Schematic of partition lag mechanism: reality evolves from $\mathcal{R}(t_0)$ to $\mathcal{R}(t_0 + \tau_p)$ during partition time, creating undetermined residue $\mathcal{U}$ that is lost to entropy. \textbf{(F)} Partition Lag Theorem: total lag $\Delta t = M \cdot \tau_p$, entropy increase $\Delta S_{\text{lag}} = k_B \ln \Omega_{\mathcal{U}} > 0$, and composition cannot reverse partition due to irreversibility.}
\label{fig:partition_lag}
\end{figure*}

\subsection{Summary: Partition as Entropy Generator}

The partition lag mechanism establishes:
\begin{enumerate}
    \item Every partition operation takes positive time ($\tau_p > 0$)
    \item During partition, systems evolve, creating lag ($\Delta t = k\tau_p$)
    \item Lag generates undetermined residue ($\mathcal{U}$)
    \item Residue has positive entropy ($S_{\text{residue}} > 0$)
    \item Composition cannot recover residue (Second Law)
    \item Partition-composition cycles are irreversible ($\Delta S_{\text{cycle}} > 0$)
\end{enumerate}

This provides a thermodynamic foundation for understanding why certain operations—particularly those involving the relationship between parts and wholes—are inherently one-directional.



%============================================================
% PART III: PHYSICAL APPLICATIONS
%============================================================

\part{Physical Applications}
\label{part:applications}

\section{Finite Geometric Partitioning of Aggregate Properties}
\label{sec:aggregate}

We analyse the thermodynamics of partitioning systems that possess \emph{aggregate properties}—properties of the whole that are not distributed among the parts. The key result is that partition operations generate entropy that accounts for the ``disappearance'' of aggregate properties when wholes are divided into parts.

\subsection{Aggregate Properties}

\begin{definition}[Aggregate Property]
\label{def:aggregate}
A property $P$ is an \emph{aggregate property} of system $W$ if:
\begin{enumerate}[(i)]
    \item $P(W) \neq 0$ (the whole possesses the property)
    \item $P(w_i) = 0$ for all parts $w_i$ when $W$ is partitioned (no part possesses the property)
    \item $\sum_i P(w_i) \neq P(W)$ (the property is not additive)
\end{enumerate}
\end{definition}

\begin{example}[Examples of Aggregate Properties]
\label{ex:aggregate}
\begin{enumerate}
    \item \textbf{Acoustic intensity}: A mass $M$ produces sound intensity $I$ upon impact. Individual grains produce negligible sound.
    \item \textbf{Structural integrity}: A bridge supports load $L$. Individual atoms cannot support macroscopic loads.
    \item \textbf{Collective behaviour}: A flock exhibits coordinated motion. Individual birds do not exhibit ``flocking.''
    \item \textbf{Threshold properties}: A heap of sand is a ``heap.'' Individual grains are not ``heaps.''
\end{enumerate}
\end{example}

\subsection{Partition of Systems with Aggregate Properties}

\begin{theorem}[Aggregate Property Loss]
\label{thm:aggregate_loss}
When a system $W$ with aggregate property $P$ is partitioned into $n$ parts $\{w_1, \ldots, w_n\}$, the property $P$ is transferred to the undetermined residue:
\begin{equation}
    P(W) = \sum_{i=1}^{n} P(w_i) + P(\mathcal{U})
\end{equation}
where $P(\mathcal{U})$ is the property content of the undetermined residue.
\end{theorem}

\begin{proof}
By conservation, the property $P$ cannot be destroyed—only redistributed. Before partition, $P$ resides entirely in the whole $W$: $P_{\text{total}} = P(W)$.

After partition, $P$ must be distributed among:
\begin{itemize}
    \item The parts: $\sum_i P(w_i)$
    \item The undetermined residue: $P(\mathcal{U})$
\end{itemize}

By Definition~\ref{def:aggregate}, $P(w_i) = 0$ for all parts. Therefore:
\begin{equation}
    P(W) = \sum_{i=1}^{n} 0 + P(\mathcal{U}) = P(\mathcal{U})
\end{equation}

The entire property $P$ has been transferred to the undetermined residue.
\end{proof}

\subsection{Entropy of Aggregate Property Loss}

\begin{theorem}[Entropy Cost of Aggregate Property Loss]
\label{thm:entropy_aggregate}
The entropy generated when partitioning a system with aggregate property $P$ is:
\begin{equation}
    \Delta S_P = \kB \ln\left( \frac{W_P}{W_0} \right)
\end{equation}
where $W_P$ is the number of configurations consistent with possessing $P$, and $W_0$ is the number of configurations of parts lacking $P$.
\end{theorem}

\begin{proof}
Before partition, the system occupies one of $W_P$ configurations that collectively possess property $P$. After partition, the parts occupy one of $W_0$ configurations, none of which possess $P$.

The entropy change is:
\begin{equation}
    \Delta S = S_{\text{after}} - S_{\text{before}} = \kB \ln W_0 - \kB \ln W_P = -\kB \ln\left(\frac{W_P}{W_0}\right)
\end{equation}

But the Second Law requires $\Delta S_{\text{total}} \geq 0$. The resolution is that the ``missing'' entropy resides in the undetermined residue:
\begin{equation}
    S_{\text{residue}} = \kB \ln\left(\frac{W_P}{W_0}\right)
\end{equation}

The total entropy increases by $\Delta S_P = S_{\text{residue}} > 0$.
\end{proof}

\subsection{Case Study: Mass and Acoustic Intensity}

Consider a mass $M$ that produces acoustic intensity $I$ when dropped from height $h$. We partition $M$ into $N$ grains of mass $m_i = M/N$.

\begin{theorem}[Acoustic Intensity as Aggregate Property]
\label{thm:acoustic}
The acoustic intensity $I(M)$ produced by mass $M$ is an aggregate property:
\begin{equation}
    I(M) > \sum_{i=1}^{N} I(m_i)
\end{equation}
The difference is accounted for by partition entropy.
\end{theorem}

\begin{proof}
Acoustic intensity scales with the coherence of the impact. A unified mass $M$ produces a single coherent pressure wave. When partitioned into $N$ grains:
\begin{itemize}
    \item Each grain impacts at slightly different times (temporal decoherence)
    \item Each grain impacts at slightly different locations (spatial decoherence)
    \item The pressure waves partially cancel through destructive interference
\end{itemize}

The acoustic intensity of a coherent impact is:
\begin{equation}
    I_{\text{coherent}} \propto M^2
\end{equation}

The acoustic intensity of $N$ incoherent impacts is:
\begin{equation}
    I_{\text{incoherent}} \propto N \cdot \left(\frac{M}{N}\right)^2 = \frac{M^2}{N}
\end{equation}

The ratio is:
\begin{equation}
    \frac{I_{\text{coherent}}}{I_{\text{incoherent}}} = N
\end{equation}

The ``missing'' intensity corresponds to entropy:
\begin{equation}
    \Delta S_{\text{acoustic}} = \kB \ln N
\end{equation}

This entropy is generated by the partition operation—it resides in the temporal and spatial decoherence introduced when the unified mass is divided into grains.
\end{proof}

\subsection{Case Study: Threshold Properties}

Consider a collection of $N$ elements that collectively possesses a threshold property $P$ (such as ``being a heap'') that no individual element possesses.

\begin{theorem}[Threshold Property Entropy]
\label{thm:threshold}
The entropy cost of eliminating a threshold property through partition is:
\begin{equation}
    \Delta S_{\text{threshold}} = \kB \ln\left(\frac{W_{\text{above}}}{W_{\text{below}}}\right)
\end{equation}
where $W_{\text{above}}$ is the number of configurations above threshold and $W_{\text{below}}$ is the number below.
\end{theorem}

\begin{proof}
The threshold property $P$ exists when the system is in one of $W_{\text{above}}$ configurations—those with sufficient elements, organisation, or coherence to exceed the threshold. Below threshold, there are $W_{\text{below}}$ configurations.

Partition reduces the system from above-threshold to below-threshold configurations. The entropy change is:
\begin{equation}
    \Delta S = \kB \ln W_{\text{below}} - \kB \ln W_{\text{above}}
\end{equation}

If $W_{\text{below}} > W_{\text{above}}$ (more ways to be disorganised than organised), then $\Delta S > 0$: partition increases entropy, as required by the Second Law.

The threshold property is not destroyed but transferred to undetermined residue—it becomes part of the boundary entropy that cannot be recovered.
\end{proof}

\subsection{Non-Recovery of Aggregate Properties}

\begin{theorem}[Composition Cannot Recover Aggregate Properties]
\label{thm:non_recovery}
Composition of parts cannot recover aggregate properties lost to partition:
\begin{equation}
    P(\text{Compose}(\{w_1, \ldots, w_n\})) < P(W)
\end{equation}
The inequality is strict whenever $P(\mathcal{U}) > 0$.
\end{theorem}

\begin{proof}
Let $W$ have aggregate property $P(W)$. Partition creates parts $\{w_1, \ldots, w_n\}$ with $P(w_i) = 0$ and residue $\mathcal{U}$ with $P(\mathcal{U}) = P(W)$.

Compose the parts: $W' = \text{Compose}(\{w_1, \ldots, w_n\})$.

The composed system $W'$ is constructed only from the parts $\{w_i\}$. The residue $\mathcal{U}$ is not included—it was lost during partition and is thermodynamically inaccessible.

Since $P$ was entirely in $\mathcal{U}$ and $\mathcal{U} \not\subseteq W'$:
\begin{equation}
    P(W') = P(\text{Compose}(\{w_i\})) = \sum_i P(w_i) = 0 < P(W)
\end{equation}

The aggregate property cannot be recovered.
\end{proof}

\begin{figure*}[htbp]
\centering
\includegraphics[width=0.95\textwidth]{figures/heap_paradox_panel.png}
\caption{\textbf{Finite Geometric Partitioning of Aggregate Properties.} \textbf{(A)} Collective property $P(\text{Whole})$: an aggregate (heap) produces measurable property (sound) that exists only for the whole. \textbf{(B)} After partition: $P(\text{Unit}) = 0$—individual units lack the collective property entirely. \textbf{(C)} Hardware-measured partition entropy: entropy generated scales with number of units, matching theory $S \propto k_B \ln(N)$. \textbf{(D)} Thermodynamic equation: partition transfers collective property to undetermined residue; composition cannot decrease entropy and therefore cannot recover property. \textbf{(E)} Why composition fails: the missing information ($\Delta S_{\text{lag}}$ worth of coherence) was dissipated during partition; Second Law forbids recovery. \textbf{(F)} Connection to classical paradox: the Sorites/Heap paradox dissolves when ontological direction is corrected—heaps are primary, grains are derived by partition, ``heap-ness'' is dissipated as entropy.}
\label{fig:heap_paradox}
\end{figure*}

\subsection{Resolution of the Traditional Puzzle}

The analysis above resolves a traditional puzzle in natural philosophy. Consider the following:

\begin{quote}
\emph{A single grain produces no sound upon falling. Adding one grain to a soundless collection does not create sound. Yet a thousand grains produce sound. How can sound emerge from the accumulation of individually soundless elements?}
\end{quote}

The puzzle assumes composition: starting from grains (no sound), combining them to form a mass (sound), asking how sound ``emerges.''

The thermodynamic resolution reverses the direction:
\begin{enumerate}
    \item The mass with sound exists \emph{first}—it is the primordial entity
    \item Partition creates the individual grains
    \item Sound is transferred to undetermined residue during partition
    \item Composition cannot recover the sound because residue is inaccessible
\end{enumerate}

Sound does not ``emerge'' from grains. Rather, \emph{silence} is created from sound by partition. The question ``how do silences combine to make sound?'' is malformed—silences do not combine to make sound; partition creates silence from sound.

\begin{remark}[Historical Note]
This analysis provides the thermodynamic structure underlying the classical Millet Paradox, attributed to Zeno of Elea. The paradox dissolves when the ontological direction is corrected: wholes with aggregate properties are primary; parts lacking those properties are derived through partition, with the property lost to undetermined residue.
\end{remark}

\begin{remark}[The Sorites Paradox]
The same analysis applies to the Sorites Paradox (Paradox of the Heap). The question ``when do grains become a heap?'' presupposes that grains are primary and heaps are composed. The thermodynamic resolution: heaps are primary categorical entities; grains are created by partition; the ``heap'' property is lost to boundary entropy. The vagueness of ``heap'' reflects edge indeterminacy at partition boundaries.
\end{remark}


\section{Continuous-to-Discrete Temporal Decomposition}
\label{sec:temporal}

We analyse the thermodynamics of partitioning continuous processes into discrete elements. The key result is that infinite partition of continuous motion generates infinite entropy, rendering the ``instantaneous state'' an artifact of partition rather than a physical reality.

\subsection{Continuous Motion and Temporal Partition}

\begin{definition}[Continuous Motion]
\label{def:continuous_motion}
A \emph{continuous motion} is a trajectory $\mathbf{x}(t)$ that varies smoothly over a time interval $[t_0, t_f]$:
\begin{equation}
    \mathbf{x}: [t_0, t_f] \to \mathbb{R}^d, \quad \mathbf{x} \in C^1([t_0, t_f])
\end{equation}
The velocity is well-defined at each point: $\mathbf{v}(t) = d\mathbf{x}/dt$.
\end{definition}

\begin{definition}[Temporal Partition]
\label{def:temporal_partition}
A \emph{temporal partition} of interval $[t_0, t_f]$ into $N$ subintervals is:
\begin{equation}
    \{[t_0, t_1], [t_1, t_2], \ldots, [t_{N-1}, t_N]\}
\end{equation}
where $t_i = t_0 + i \cdot \Delta t$ and $\Delta t = (t_f - t_0)/N$.
\end{definition}

\begin{definition}[Instantaneous State]
\label{def:instant}
An \emph{instantaneous state} at time $t_i$ is the configuration $(\mathbf{x}(t_i), \mathbf{v}(t_i))$ obtained by evaluating the trajectory at a single instant.
\end{definition}

\subsection{Entropy of Temporal Partition}

\begin{theorem}[Temporal Partition Entropy]
\label{thm:temporal_entropy}
Partitioning continuous motion into $N$ temporal segments generates entropy:
\begin{equation}
    \Delta S_{\text{temporal}} = \kB (N-1) H_{\text{boundary}}
\end{equation}
where $H_{\text{boundary}}$ is the entropy of each temporal boundary.
\end{theorem}

\begin{proof}
A temporal partition into $N$ segments creates $N-1$ internal boundaries. Each boundary separates ``before $t_i$'' from ``after $t_i$.''

At each boundary, the trajectory is evaluated to determine the instantaneous state. This evaluation has finite precision—the position $\mathbf{x}(t_i)$ and velocity $\mathbf{v}(t_i)$ can only be determined to within measurement uncertainty. The Shannon entropy of this uncertainty is $H_{\text{boundary}}$.

With $N-1$ independent boundaries:
\begin{equation}
    \Delta S_{\text{temporal}} = \kB (N-1) H_{\text{boundary}}
\end{equation}
\end{proof}

\begin{corollary}[Infinite Partition Generates Infinite Entropy]
\label{cor:infinite_entropy}
As the number of temporal partitions approaches infinity:
\begin{equation}
    \lim_{N \to \infty} \Delta S_{\text{temporal}} = \lim_{N \to \infty} \kB (N-1) H_{\text{boundary}} = \infty
\end{equation}
Infinitely fine temporal partition destroys all information about the original motion.
\end{corollary}

\subsection{Motion as Undetermined Residue}

\begin{theorem}[Motion Resides in Residue]
\label{thm:motion_residue}
When continuous motion is partitioned into instantaneous states, the \emph{motion itself} (the continuous change) becomes undetermined residue.
\end{theorem}

\begin{proof}
Consider a trajectory $\mathbf{x}(t)$ on $[t_0, t_f]$ with velocity $\mathbf{v}(t) \neq 0$. Partition the interval into $N$ instants $\{t_0, t_1, \ldots, t_N\}$.

At each instant $t_i$, record the instantaneous state:
\begin{equation}
    \mathbf{s}_i = (\mathbf{x}(t_i), \mathbf{v}(t_i))
\end{equation}

The collection $\{\mathbf{s}_0, \mathbf{s}_1, \ldots, \mathbf{s}_N\}$ describes positions and velocities at discrete instants. But this collection does not contain the \emph{motion}—the continuous process of changing position.

Motion exists \emph{between} instants: it is the process $\mathbf{x}(t_{i}) \to \mathbf{x}(t_{i+1})$ that occurs in the interval $(t_i, t_{i+1})$. This process is not captured by the instantaneous states $\mathbf{s}_i$ and $\mathbf{s}_{i+1}$—it resides in the undetermined residue of the temporal partition.

As $N \to \infty$ and $\Delta t \to 0$, each interval $(t_i, t_{i+1})$ shrinks, but the number of intervals grows. The motion becomes distributed across infinitely many infinitesimal residues—it becomes entirely undetermined.
\end{proof}

\subsection{The Stillness of Instantaneous States}

\begin{theorem}[Instantaneous States Are Still]
\label{thm:instantaneous_still}
At any instant $t_i$, the system occupies exactly its position $\mathbf{x}(t_i)$. There is no motion \emph{at} an instant—motion requires duration.
\end{theorem}

\begin{proof}
Motion is defined as change of position over time: $\mathbf{v} = d\mathbf{x}/dt$. This derivative is a limit:
\begin{equation}
    \mathbf{v}(t) = \lim_{\Delta t \to 0} \frac{\mathbf{x}(t + \Delta t) - \mathbf{x}(t)}{\Delta t}
\end{equation}

At a single instant (with $\Delta t = 0$), the quotient is undefined—there is no duration over which to measure change. The velocity $\mathbf{v}(t)$ exists as a limit, not as an instantaneous property.

At instant $t_i$, the system is at position $\mathbf{x}(t_i)$. It is not ``moving'' at that instant—it simply \emph{is} at that position. Motion exists only in the transition between positions, which requires positive duration.
\end{proof}

\begin{corollary}[Stillnesses Cannot Compose to Motion]
\label{cor:stillnesses}
If each instantaneous state is ``still'' (not moving), then composing instantaneous states cannot produce motion:
\begin{equation}
    \text{Compose}(\{\text{still}_0, \text{still}_1, \ldots, \text{still}_N\}) \not\supset \text{motion}
\end{equation}
Motion cannot be recovered from its temporal partition.
\end{corollary}

\subsection{Thermodynamic Analysis of Temporal Decomposition}

\begin{theorem}[Entropy of Motion Loss]
\label{thm:motion_entropy}
The entropy cost of temporal partition of continuous motion is:
\begin{equation}
    \Delta S_{\text{motion}} = \kB \ln\left(\frac{W_{\text{continuous}}}{W_{\text{discrete}}}\right)
\end{equation}
where $W_{\text{continuous}}$ is the number of continuous trajectories and $W_{\text{discrete}}$ is the number of discrete state sequences.
\end{theorem}

\begin{proof}
A continuous trajectory $\mathbf{x}(t)$ on $[t_0, t_f]$ is specified by initial conditions plus the requirement of continuity. The space of continuous trajectories has cardinality $W_{\text{continuous}}$.

A sequence of $N$ discrete states $\{\mathbf{s}_0, \ldots, \mathbf{s}_N\}$ is specified by $N+1$ independent position-velocity pairs. The space of such sequences has cardinality $W_{\text{discrete}} = (\text{phase space volume})^{N+1}$.

In general, $W_{\text{discrete}} > W_{\text{continuous}}$ because discrete sequences need not be consistent with any continuous trajectory—the states can ``jump'' between arbitrary positions.

The entropy increase is:
\begin{equation}
    \Delta S = \kB \ln W_{\text{discrete}} - \kB \ln W_{\text{continuous}} = \kB \ln\left(\frac{W_{\text{discrete}}}{W_{\text{continuous}}}\right) > 0
\end{equation}

This entropy is the cost of destroying the continuity constraint—the ``motion'' that connects successive positions.
\end{proof}

\subsection{The Dichotomy Problem}

Consider an object that must traverse distance $L$ in finite time. The traditional analysis proceeds:
\begin{enumerate}
    \item To travel $L$, first travel $L/2$
    \item To travel $L/2$, first travel $L/4$
    \item Continue indefinitely: must complete infinitely many sub-journeys
    \item Conclusion: motion is impossible
\end{enumerate}

\begin{theorem}[Thermodynamic Resolution of Dichotomy]
\label{thm:dichotomy}
The dichotomy analysis generates infinite entropy by infinite partition. The ``impossibility'' is not a feature of motion but an artifact of the partition process.
\end{theorem}

\begin{proof}
Each subdivision of the journey is a partition operation. Subdividing into $N$ sub-journeys generates entropy:
\begin{equation}
    \Delta S_N = \kB (N-1) H_{\text{boundary}}
\end{equation}

As $N \to \infty$:
\begin{equation}
    \Delta S_\infty = \lim_{N \to \infty} \kB (N-1) H_{\text{boundary}} = \infty
\end{equation}

Infinite partition generates infinite entropy, completely destroying the information content of the original motion. The ``infinitely many sub-journeys'' do not exist in the physical motion—they are created by the partition process.

The physical motion completes in finite time because it was never partitioned. The ``impossibility'' arises only when we attempt to decompose continuous motion into infinitely many discrete segments.
\end{proof}

\subsection{The Arrow Paradox}

Consider an arrow in flight. At any instant, the arrow occupies exactly one position. If it is at a position, it is not moving. If it is not moving at any instant, when does it move?

\begin{theorem}[Thermodynamic Resolution of Arrow Paradox]
\label{thm:arrow}
The arrow paradox arises from confusing the ontological status of instantaneous states. Motion is not composed of instantaneous stillnesses—rather, stillness is derived from motion by temporal partition.
\end{theorem}

\begin{proof}
The arrow's motion $\mathbf{x}(t)$ exists as a continuous process on the interval $[t_0, t_f]$. This is the primary reality.

Temporal partition extracts instantaneous states $\{\mathbf{x}(t_i)\}$. At each such state, the arrow ``occupies exactly its length''—it is in a definite position. This is not ``motion'' but a derived snapshot.

The paradox asks: ``How do stillnesses compose to motion?'' The thermodynamic answer: they don't. Motion is not composed of stillnesses. Motion is primary; stillnesses are derived by partition.

The motion itself is lost to undetermined residue during temporal partition. It exists \emph{between} the snapshots, in the transition from $\mathbf{x}(t_i)$ to $\mathbf{x}(t_{i+1})$. This transition is not captured by the snapshots—it is the undetermined residue of the partition.

The arrow moves because motion exists first. Asking ``when does it move?'' presupposes that motion is composed of instants, which reverses the true ontological order.
\end{proof}

\begin{figure*}[htbp]
\centering
\includegraphics[width=0.95\textwidth]{figures/zeno_paradox_panel.png}
\caption{\textbf{Infinite Subdivision of Bounded Continuous Intervals.} \textbf{(A)} Spatial interval dichotomy: recursive halving creates exponentially many sub-intervals, each partition adding entropy. \textbf{(B)} Hardware-measured entropy divergence: as partition depth $M \to \infty$, cumulative entropy $S \to \infty$—infinite partition generates infinite entropy, completely destroying the continuous motion information. \textbf{(C)} Continuous property dissipation: motion is a property of the continuous whole; partition extracts discrete ``positions'' that are individually static; motion itself becomes undetermined residue. \textbf{(D)} Arrow paradox: continuous trajectory (solid line) vs. discrete instantaneous samples (dots); at each instant, no duration exists over which to define velocity. \textbf{(E)} Thermodynamic resolution: infinite partition generates infinite entropy; motion is primary, stillness is derived; composing stillnesses cannot recover motion. \textbf{(F)} Connection to Zeno's paradoxes: both Dichotomy and Arrow dissolve when we recognize that continuous motion exists first, and partition (not composition) creates discrete states.}
\label{fig:zeno_paradox}
\end{figure*}

\subsection{Resolution of Traditional Puzzles}

\begin{remark}[Historical Note]
The analysis above provides the thermodynamic structure underlying Zeno's paradoxes of motion—the Dichotomy and the Arrow. These paradoxes dissolve when the ontological direction is corrected:
\begin{itemize}
    \item Motion is not composed of stillnesses (infinite still instants cannot compose to motion)
    \item Stillness is derived from motion by temporal partition
    \item Motion itself becomes undetermined residue during partition
    \item Infinite partition generates infinite entropy, destroying all motion information
\end{itemize}
The ``paradoxes'' are not paradoxes of motion but paradoxes of partition—demonstrations that infinite subdivision destroys continuous structure.
\end{remark}


\section{Identity Persistence Under Sequential Component Exchange}
\label{sec:identity}

We analyse the thermodynamics of systems undergoing sequential component replacement. The key result is that each replacement is a partition-composition cycle that generates entropy, and accumulated entropy eventually exceeds the system's identity information content—at which point identity has been thermodynamically dissipated.

\subsection{Identity as Information}

\begin{definition}[Identity Information]
\label{def:identity_info}
The \emph{identity information} $I_{\text{id}}$ of a system $S$ is the minimum information required to distinguish $S$ from all other systems:
\begin{equation}
    I_{\text{id}}(S) = \min_{D} H(D(S))
\end{equation}
where $D$ ranges over all distinguishing functions and $H$ is Shannon entropy.
\end{definition}

\begin{theorem}[Identity Information is Finite]
\label{thm:identity_finite}
For any physical system $S$, the identity information is finite:
\begin{equation}
    I_{\text{id}}(S) < \infty
\end{equation}
\end{theorem}

\begin{proof}
A physical system occupies a bounded region of phase space with finite volume $V$. Distinguishing $S$ requires specifying its location within this volume to precision $\delta$. The number of distinguishable locations is $V/\delta^d$ where $d$ is the dimensionality.

The identity information is at most:
\begin{equation}
    I_{\text{id}}(S) \leq \ln\left(\frac{V}{\delta^d}\right) < \infty
\end{equation}
for any finite precision $\delta > 0$.
\end{proof}

\begin{definition}[Identity Entropy]
\label{def:identity_entropy}
The \emph{identity entropy} of a system is:
\begin{equation}
    S_{\text{id}} = \kB I_{\text{id}}
\end{equation}
This is the thermodynamic entropy associated with the system's distinguishability.
\end{definition}

\subsection{Component Replacement as Partition-Composition}

\begin{definition}[Component Replacement]
\label{def:replacement}
A \emph{component replacement} operation on system $S$ consists of:
\begin{enumerate}[(i)]
    \item Partition: Remove component $c_{\text{old}}$ from $S$, creating $S' = S \setminus \{c_{\text{old}}\}$
    \item Composition: Add component $c_{\text{new}}$ to $S'$, creating $S'' = S' \cup \{c_{\text{new}}\}$
\end{enumerate}
\end{definition}

\begin{theorem}[Replacement Generates Entropy]
\label{thm:replacement_entropy}
Each component replacement generates entropy:
\begin{equation}
    \Delta S_{\text{replacement}} = S_{\text{partition}} + S_{\text{composition}} > 0
\end{equation}
where $S_{\text{partition}}$ is the entropy of removing the old component and $S_{\text{composition}}$ is the entropy of adding the new component.
\end{theorem}

\begin{proof}
By Theorem~\ref{thm:irreversibility}, partition generates undetermined residue with positive entropy:
\begin{equation}
    S_{\text{partition}} = \kB H_{\text{boundary}}(\text{removal})
\end{equation}

The composition operation does not generate new entropy but also does not recover the partition entropy. The new component $c_{\text{new}}$ is not identical to $c_{\text{old}}$, so additional distinguishing information is required:
\begin{equation}
    S_{\text{composition}} = \kB I(c_{\text{new}} \neq c_{\text{old}})
\end{equation}

The total entropy generated per replacement is:
\begin{equation}
    \Delta S_{\text{replacement}} = S_{\text{partition}} + S_{\text{composition}} > 0
\end{equation}
\end{proof}

\subsection{Cumulative Identity Loss}

\begin{theorem}[Cumulative Entropy from Sequential Replacements]
\label{thm:cumulative}
After $n$ component replacements, the cumulative entropy generated is:
\begin{equation}
    S_{\text{cumulative}}(n) = \sum_{i=1}^{n} \Delta S_i = n \cdot \langle \Delta S \rangle
\end{equation}
where $\langle \Delta S \rangle$ is the average entropy per replacement.
\end{theorem}

\begin{proof}
Each replacement $i$ generates entropy $\Delta S_i$. These contributions are additive because each replacement is an independent operation on the current state of the system. The cumulative entropy is the sum over all replacements.

If all replacements are statistically similar, then $\langle \Delta S \rangle = \Delta S_{\text{replacement}}$ is constant, and:
\begin{equation}
    S_{\text{cumulative}}(n) = n \cdot \Delta S_{\text{replacement}}
\end{equation}
\end{proof}

\begin{theorem}[Identity Dissipation Threshold]
\label{thm:threshold}
The original identity of system $S$ is thermodynamically dissipated when the cumulative replacement entropy exceeds the identity entropy:
\begin{equation}
    S_{\text{cumulative}}(n^*) \geq S_{\text{id}}(S)
\end{equation}
The threshold number of replacements is:
\begin{equation}
    n^* = \frac{S_{\text{id}}(S)}{\langle \Delta S \rangle} = \frac{I_{\text{id}}(S)}{\langle \Delta I \rangle}
\end{equation}
\end{theorem}

\begin{proof}
Identity information $I_{\text{id}}(S)$ is the total information required to distinguish the original system $S$ from all others. Each replacement dissipates some of this information—the information about the original components is lost to undetermined residue.

When the cumulative information loss equals the total identity information:
\begin{equation}
    n \cdot \langle \Delta I \rangle = I_{\text{id}}(S)
\end{equation}
the system no longer contains sufficient information to be identified as the original $S$. From a thermodynamic perspective, the identity has been completely dissipated.

Solving for $n$:
\begin{equation}
    n^* = \frac{I_{\text{id}}(S)}{\langle \Delta I \rangle}
\end{equation}
\end{proof}

\subsection{The Vagueness of Identity Boundaries}

\begin{theorem}[Edge Indeterminacy of Identity]
\label{thm:identity_edge}
The boundary at which original identity is lost is fundamentally vague, with uncertainty:
\begin{equation}
    \Delta n \geq \frac{\kB T}{|\Delta S_{\text{replacement}}|}
\end{equation}
\end{theorem}

\begin{proof}
By Theorem~\ref{thm:edge_indeterminacy}, partition boundaries have irreducible uncertainty due to partition lag. The identity threshold $n^*$ is itself determined by a partition process—the conceptual division between ``same identity'' and ``different identity.''

The uncertainty in this boundary is proportional to thermal fluctuations:
\begin{equation}
    \Delta n \cdot \Delta S_{\text{replacement}} \geq \kB T
\end{equation}

Solving:
\begin{equation}
    \Delta n \geq \frac{\kB T}{|\Delta S_{\text{replacement}}|}
\end{equation}

This uncertainty is irreducible—there is no sharp boundary between ``same system'' and ``different system.''
\end{proof}

\subsection{Case Study: Sequential Plank Replacement}

Consider a wooden structure (e.g., a vessel) composed of $N$ planks. Each plank is sequentially replaced over time.

\begin{theorem}[Vessel Identity Analysis]
\label{thm:vessel}
For a vessel with $N$ planks, each carrying identity fraction $1/N$:
\begin{enumerate}[(i)]
    \item After replacing $k$ planks, the fractional identity remaining is $(N-k)/N$
    \item The identity entropy remaining is $S_{\text{id}} \cdot (N-k)/N$
    \item Complete replacement ($k = N$) dissipates all original identity entropy
\end{enumerate}
\end{theorem}

\begin{proof}
Assume identity is uniformly distributed among planks (each plank carries $I_{\text{id}}/N$ identity information). After replacing $k$ planks:
\begin{itemize}
    \item $N - k$ original planks remain, carrying $(N-k)/N \cdot I_{\text{id}}$ identity
    \item $k$ new planks carry zero original identity
\end{itemize}

The fractional identity is:
\begin{equation}
    f(k) = \frac{N - k}{N}
\end{equation}

When $k = N$ (all planks replaced):
\begin{equation}
    f(N) = 0
\end{equation}

All original identity has been dissipated.
\end{proof}

\begin{theorem}[Gradual vs. Sudden Replacement]
\label{thm:gradual}
Gradual replacement (one plank at a time) and sudden replacement (all planks at once) yield the same final identity entropy, but gradual replacement distributes the identity loss over time while sudden replacement concentrates it.
\end{theorem}

\begin{proof}
Let $S_0 = S_{\text{id}}(S)$ be the original identity entropy.

\textbf{Gradual replacement}: After $k$ replacements, identity entropy remaining is:
\begin{equation}
    S_k = S_0 \cdot \frac{N - k}{N}
\end{equation}
At $k = N$, $S_N = 0$.

\textbf{Sudden replacement}: All $N$ planks are replaced simultaneously. Identity entropy immediately drops to:
\begin{equation}
    S_{\text{sudden}} = 0
\end{equation}

Both processes result in zero remaining identity entropy. The difference is the time profile of the loss, not the final state.
\end{proof}

\subsection{The Two-Vessel Problem}

Consider a scenario where the replaced planks are preserved and reassembled into a second structure.

\begin{theorem}[Conservation of Identity Entropy]
\label{thm:conservation}
When replaced components are preserved and reassembled, the total identity entropy is conserved:
\begin{equation}
    S_{\text{id}}(\text{modified}) + S_{\text{id}}(\text{reassembled}) + S_{\text{dissipated}} = S_{\text{id}}(\text{original})
\end{equation}
\end{theorem}

\begin{proof}
The original identity entropy $S_{\text{id}}(\text{original})$ cannot be created or destroyed—only redistributed.

After complete replacement:
\begin{itemize}
    \item Modified structure: Contains new components, zero original identity
    \item Reassembled structure: Contains original components, partial original identity
    \item Environment: Contains dissipated entropy from partition boundaries
\end{itemize}

The sum is conserved:
\begin{equation}
    0 + S_{\text{reassembled}} + S_{\text{dissipated}} = S_{\text{id}}(\text{original})
\end{equation}

The reassembled structure has identity $S_{\text{reassembled}} = S_{\text{id}}(\text{original}) - S_{\text{dissipated}}$, which is less than the original due to partition entropy losses during removal and reassembly.
\end{proof}

\begin{corollary}[Neither Vessel is Identical to Original]
\label{cor:neither}
After complete replacement with reassembly:
\begin{enumerate}[(i)]
    \item Modified vessel: $S_{\text{id}} = 0$ (contains no original identity)
    \item Reassembled vessel: $S_{\text{id}} < S_{\text{original}}$ (contains most but not all)
\end{enumerate}
Neither vessel is fully identical to the original.
\end{corollary}

\begin{figure*}[htbp]
\centering
\includegraphics[width=0.95\textwidth]{figures/ship_theseus_panel.png}
\caption{\textbf{Identity Persistence Under Sequential Component Exchange.} \textbf{(A)} Sequential component exchange: each replacement cycle consists of partition (remove old component) and composition (add new component), each generating entropy. \textbf{(B)} Hardware-measured identity dissipation: identity remaining fraction decays exponentially with number of exchanges; 50\% threshold crossed after sufficient replacements. \textbf{(C)} Cumulative entropy from exchanges: monotonically increasing, verifying Second Law—each cycle irreversibly dissipates identity information. \textbf{(D)} Entropy per exchange cycle: each replacement generates measurable positive entropy. \textbf{(E)} Identity-entropy relationship: identity decays as $I \propto e^{-S/k_B}$, showing thermodynamic relationship between information and entropy. \textbf{(F)} Connection to Ship of Theseus: the paradox dissolves when identity is recognized as finite information that is progressively dissipated through sequential partition-composition cycles—no sharp boundary exists, only gradual transition governed by $S_{\text{cumulative}}/S_{\text{id}}$.}
\label{fig:ship_theseus}
\end{figure*}

\subsection{Resolution of the Traditional Puzzle}

\begin{remark}[Historical Note]
This analysis provides the thermodynamic structure underlying the Ship of Theseus paradox. The traditional question—``If every plank is replaced, is it the same ship?''—dissolves under thermodynamic analysis:
\begin{itemize}
    \item Identity is information, which has finite quantity $I_{\text{id}}$
    \item Each replacement dissipates identity information (entropy to environment)
    \item After sufficient replacements, no original identity remains
    \item The boundary is fuzzy due to edge indeterminacy at partition boundaries
\end{itemize}
The question ``is it the same ship?'' presupposes a sharp identity boundary. The thermodynamic answer: identity is progressively dissipated through sequential partition-composition cycles, with no sharp threshold but a gradual transition from ``same'' to ``different'' governed by the ratio $S_{\text{cumulative}}/S_{\text{id}}$.
\end{remark}


\section{Partition-Free Traversal of Continuous Intervals}
\label{sec:null_geodesics}

We analyse the thermodynamics of traversing continuous intervals without partitioning. The key result is that partition-free traversal generates zero boundary entropy and therefore experiences no temporal duration. This provides a thermodynamic characterisation of null geodesics and explains why partition-based measurement of continuous intervals requires unbounded resources.

\subsection{The Measurement Problem for Continuous Intervals}

\begin{definition}[Partition-Based Measurement]
\label{def:partition_measurement}
A \emph{partition-based measurement} of a continuous interval $[a, b]$ proceeds by:
\begin{enumerate}[(i)]
    \item Selecting a unit $\epsilon > 0$
    \item Partitioning $[a, b]$ into $n = \lceil (b-a)/\epsilon \rceil$ subintervals
    \item Counting the number of subintervals
    \item Reporting $n \cdot \epsilon$ as the measured length
\end{enumerate}
\end{definition}

\begin{theorem}[Boundary Entropy of Measurement]
\label{thm:measurement_entropy}
Partition-based measurement of interval $[a, b]$ into $n$ subintervals generates boundary entropy:
\begin{equation}
    S_{\text{boundary}} = \kB (n-1) H_{\text{edge}}
\end{equation}
where $H_{\text{edge}}$ is the entropy of each partition boundary due to edge indeterminacy.
\end{theorem}

\begin{proof}
By Theorem~\ref{thm:boundary_entropy}, partitioning into $n$ parts creates $n-1$ internal boundaries. Each boundary has indeterminate location due to partition lag: the position $x_i$ separating subinterval $i$ from subinterval $i+1$ cannot be specified with arbitrary precision in finite time.

Let $p(x)$ be the probability distribution over possible boundary locations. The Shannon entropy of each boundary is:
\begin{equation}
    H_{\text{edge}} = -\int p(x) \ln p(x) \, dx > 0
\end{equation}

The total boundary entropy is:
\begin{equation}
    S_{\text{boundary}} = \kB (n-1) H_{\text{edge}}
\end{equation}
\end{proof}

\begin{theorem}[Divergence of Measurement Entropy]
\label{thm:measurement_divergence}
As measurement precision increases ($\epsilon \to 0$), the boundary entropy diverges:
\begin{equation}
    \lim_{\epsilon \to 0} S_{\text{boundary}} = \lim_{n \to \infty} \kB (n-1) H_{\text{edge}} = \infty
\end{equation}
Perfect measurement of a continuous interval requires infinite entropy production.
\end{theorem}

\begin{proof}
For fixed interval length $L = b - a$ and measurement unit $\epsilon$, the number of partitions is $n = L/\epsilon$. As $\epsilon \to 0$, $n \to \infty$.

Each partition boundary carries positive entropy $H_{\text{edge}} > 0$ (by Theorem~\ref{thm:measurement_entropy}). Therefore:
\begin{equation}
    S_{\text{boundary}} = \kB (n-1) H_{\text{edge}} \xrightarrow{n \to \infty} \infty
\end{equation}

Perfect measurement ($\epsilon \to 0$) requires infinite partitions and therefore infinite entropy.
\end{proof}

\begin{corollary}[The Measuring String Paradox]
\label{cor:string_paradox}
Measuring a line segment by laying unit lengths end-to-end requires a measuring instrument of unbounded extent.
\end{corollary}

\begin{proof}
Consider a string of length $\ell$ used to measure a segment $[a, b]$ by repeated application. Each application creates a partition boundary with edge indeterminacy $\delta > 0$.

After $n$ applications, the cumulative boundary indeterminacy is:
\begin{equation}
    \Delta_{\text{total}} = n \cdot \delta
\end{equation}

For precise measurement, we need $\Delta_{\text{total}} < \epsilon_{\text{tolerance}}$. This requires:
\begin{equation}
    n < \frac{\epsilon_{\text{tolerance}}}{\delta}
\end{equation}

But to measure length $L$ with unit $\ell$, we need $n = L/\ell$ applications. For $L/\ell > \epsilon_{\text{tolerance}}/\delta$, the measurement fails.

Alternatively, to achieve arbitrary precision, we need $\delta \to 0$, which requires the measuring instrument to have infinite internal resolution—equivalent to containing infinite information, hence unbounded physical extent.
\end{proof}

\subsection{Partition-Free Traversal}

\begin{definition}[Partition-Free Traversal]
\label{def:partition_free}
A \emph{partition-free traversal} of interval $[a, b]$ is a process that:
\begin{enumerate}[(i)]
    \item Begins at $a$ and terminates at $b$
    \item Creates no internal categorical distinctions along $[a, b]$
    \item Does not partition the interval into ``already traversed'' and ``not yet traversed''
\end{enumerate}
\end{definition}

\begin{theorem}[Zero Entropy of Partition-Free Traversal]
\label{thm:zero_traversal_entropy}
Partition-free traversal generates zero boundary entropy:
\begin{equation}
    S_{\text{partition-free}} = 0
\end{equation}
\end{theorem}

\begin{proof}
Boundary entropy arises from partition boundaries (Theorem~\ref{thm:measurement_entropy}). Partition-free traversal creates $n = 0$ internal boundaries. Therefore:
\begin{equation}
    S_{\text{partition-free}} = \kB (0-1) H_{\text{edge}} = 0
\end{equation}
(interpreting $(-1) \cdot H_{\text{edge}} = 0$ since there are no boundaries to contribute entropy).

Alternatively: partition-free traversal treats the interval as a single categorical entity. The number of internal categorical distinctions is zero, hence the entropy contribution from traversal is zero.
\end{proof}

\subsection{Temporal Duration from Partition Entropy}

\begin{theorem}[Time Requires Partition]
\label{thm:time_partition}
Experienced temporal duration is proportional to partition entropy:
\begin{equation}
    \Delta \tau = \frac{S_{\text{partition}}}{\kB \omega}
\end{equation}
where $\omega$ is a characteristic frequency relating entropy to time.
\end{theorem}

\begin{proof}
From Section~\ref{sec:partition_lag}, partition lag $\tau_p$ generates entropy $\Delta S_p$ per partition. The total time experienced during $n$ partitions is:
\begin{equation}
    \Delta \tau_{\text{total}} = n \cdot \tau_p
\end{equation}

The total entropy generated is:
\begin{equation}
    S_{\text{total}} = n \cdot \Delta S_p
\end{equation}

Therefore:
\begin{equation}
    \Delta \tau_{\text{total}} = \frac{S_{\text{total}}}{\Delta S_p / \tau_p} = \frac{S_{\text{total}}}{\kB \omega}
\end{equation}
where $\omega = \Delta S_p / (\kB \tau_p)$ is the entropy production rate.

Experienced time is directly proportional to entropy generated, which is proportional to the number of partitions.
\end{proof}

\begin{corollary}[Partition-Free Traversal Has Zero Proper Time]
\label{cor:zero_proper_time}
An entity undergoing partition-free traversal experiences zero proper time:
\begin{equation}
    \Delta \tau_{\text{partition-free}} = 0
\end{equation}
\end{corollary}

\begin{proof}
By Theorem~\ref{thm:zero_traversal_entropy}, partition-free traversal generates $S_{\text{partition-free}} = 0$.

By Theorem~\ref{thm:time_partition}:
\begin{equation}
    \Delta \tau_{\text{partition-free}} = \frac{0}{\kB \omega} = 0
\end{equation}
\end{proof}

\subsection{Maximum Speed from Partition Structure}

\begin{theorem}[Maximum Speed is Partition-Free Speed]
\label{thm:max_speed}
The maximum speed through space is achieved by partition-free traversal. Any partition of the trajectory reduces the traversal speed.
\end{theorem}

\begin{proof}
Consider traversing distance $L$ in coordinate time $\Delta t$. Speed is $v = L / \Delta t$.

For partition-based traversal with $n$ partitions, proper time is:
\begin{equation}
    \Delta \tau = \frac{\kB (n-1) H_{\text{edge}}}{\kB \omega} = \frac{(n-1) H_{\text{edge}}}{\omega} > 0
\end{equation}

The Lorentz factor relates coordinate time to proper time:
\begin{equation}
    \Delta t = \gamma \Delta \tau
\end{equation}

For $\Delta \tau > 0$, we have $\gamma < \infty$, hence $v < c$.

For partition-free traversal, $\Delta \tau = 0$. The only consistent solution is $\gamma \to \infty$, which requires $v = c$.

Therefore:
\begin{itemize}
    \item Partition-free traversal: $v = c$ (maximum)
    \item Partition-based traversal: $v < c$ (subluminal)
\end{itemize}
\end{proof}

\begin{theorem}[Massive Objects Cannot Achieve Maximum Speed]
\label{thm:mass_partition}
Objects with nonzero rest mass cannot achieve partition-free traversal, hence cannot reach maximum speed.
\end{theorem}

\begin{proof}
Rest mass $m > 0$ implies localisation in space—the object occupies a definite region at each moment. This localisation constitutes a partition: the object is ``here'' and not ``there.''

More precisely, a massive object at position $\mathbf{x}$ creates a categorical distinction between:
\begin{itemize}
    \item The region containing the object
    \item The region not containing the object
\end{itemize}

This is a binary partition of space at each instant. As the object moves, it creates a sequence of such partitions, generating boundary entropy between successive positions.

The boundary entropy per unit distance is:
\begin{equation}
    \frac{dS}{dx} = \kB \rho_{\text{partition}}
\end{equation}
where $\rho_{\text{partition}}$ is the partition density (partitions per unit length) required to localise mass $m$.

For $m > 0$, localisation requires $\rho_{\text{partition}} > 0$, hence $dS/dx > 0$. Total entropy for distance $L$ is:
\begin{equation}
    S_{\text{massive}} = \int_0^L \kB \rho_{\text{partition}} \, dx > 0
\end{equation}

This positive entropy implies positive proper time (Theorem~\ref{thm:time_partition}), hence subluminal speed (Theorem~\ref{thm:max_speed}).

Only $m = 0$ allows $\rho_{\text{partition}} = 0$, enabling partition-free traversal at maximum speed.
\end{proof}

\subsection{Interaction Requires Partition}

\begin{definition}[Interaction]
\label{def:interaction}
An \emph{interaction} between systems $A$ and $B$ is a process that creates a categorical distinction between:
\begin{enumerate}[(i)]
    \item The state of $A$ before interaction
    \item The state of $A$ after interaction
\end{enumerate}
and similarly for $B$.
\end{definition}

\begin{theorem}[Interaction Requires Partition Capability]
\label{thm:interaction_partition}
For systems $A$ and $B$ to interact, at least one must be capable of partition—of creating categorical distinctions in its state.
\end{theorem}

\begin{proof}
By Definition~\ref{def:interaction}, interaction creates a distinction between before-states and after-states. This is precisely a partition of the system's state space into ``before'' and ``after'' categories.

If neither $A$ nor $B$ can partition (create categorical distinctions), then neither can transition from before-state to after-state. Without state change, there is no interaction.
\end{proof}

\begin{corollary}[Partition-Free Entities Interact Only with Partitionable Systems]
\label{cor:partition_free_interaction}
A partition-free entity (such as a massless particle undergoing null geodesic) can interact with a system $B$ only if $B$ is capable of partition.
\end{corollary}

\begin{proof}
By Theorem~\ref{thm:interaction_partition}, interaction requires at least one partitioning participant. If the partition-free entity cannot partition, then $B$ must partition for interaction to occur.

The interaction proceeds as:
\begin{enumerate}
    \item Partition-free entity arrives at $B$
    \item $B$ partitions its state space (before $\to$ after)
    \item Partition-free entity departs
\end{enumerate}

The partition-free entity triggers the partition in $B$ without partitioning itself. Examples: photon absorption (matter partitions into ground/excited states), photon emission (matter partitions, photon created).
\end{proof}

\subsection{Resolution of the Measurement Problem}

\begin{remark}[Connection to Classical Problems]
\label{rem:measurement_paradox}
The analysis above resolves several interconnected problems in the foundations of measurement:

\textbf{The Ruler Paradox}: To measure a length $L$ with precision $\epsilon$, one needs a ruler with $L/\epsilon$ graduations. Each graduation is a partition boundary with nonzero width $\delta$. For $L/\epsilon$ large, the total boundary width $L \delta / \epsilon$ exceeds any fixed ruler length. Arbitrarily precise measurement requires an arbitrarily long ruler—or equivalently, infinite information content.

\textbf{The String Paradox}: Measuring by repeated application of a unit length accumulates boundary errors. The total error after $n$ applications grows as $\sqrt{n}$ (random) or $n$ (systematic). Perfect measurement requires either infinitely many applications or an infinitely precise unit—both requiring unbounded resources.

\textbf{The Photon's Perspective}: A photon experiences zero proper time not because ``time slows down'' (a coordinate effect) but because partition-free traversal generates zero entropy, and entropy generation is the physical basis of temporal duration. The photon doesn't partition its trajectory, hence has no internal before/after distinction, hence experiences no time.

\textbf{The Speed Limit}: The speed of light $c$ is maximum not due to an arbitrary cosmic speed limit but because partition-free traversal is the fastest possible mode of spatial transition. Any partitioning slows traversal by generating entropy that manifests as proper time. Mass requires localisation, localisation requires partition, partition requires time, time reduces speed. Only massless, partition-free entities achieve $c$.

These results follow from the thermodynamics of partition, not from postulates about spacetime structure. The structure of spacetime (null cones, proper time, Lorentz invariance) emerges from the categorical structure of partition operations.
\end{remark}

\begin{figure*}[htbp]
\centering
\includegraphics[width=0.90\textwidth]{figures/null_geodesics_panel.png}
\caption{\textbf{Partition-Free Traversal of Continuous Intervals.} This panel illustrates the thermodynamic basis for null geodesics and the speed of light. \textbf{(A)} Partition-based measurement: dividing an interval into $n$ segments creates $n-1$ boundaries, each with edge indeterminacy contributing entropy $H_{\text{edge}}$. \textbf{(B)} Measurement entropy divergence: as precision increases ($\epsilon \to 0$), boundary entropy diverges, demonstrating that perfect partition-based measurement requires infinite resources. \textbf{(C)} Partition-free traversal: treating the interval as a single categorical entity creates zero boundaries and zero entropy. \textbf{(D)} Time from partition entropy: experienced duration is proportional to partition entropy; partition-free traversal yields $\Delta \tau = 0$. \textbf{(E)} Maximum speed: partition-free traversal achieves speed $c$; any partitioning generates entropy/time, reducing speed below $c$. \textbf{(F)} Mass requires partition: localised mass creates spatial distinctions, generating entropy during motion; only massless entities achieve partition-free traversal.}
\label{fig:null_geodesics_panel}
\end{figure*}


\section{Non-Partitionable Accumulation of Resolved Alternatives}
\label{sec:recursive_compounding}

We analyse the thermodynamics of categorical systems where each actualisation resolves infinitely many non-actualisations. The key result is that non-actualisations—what did not happen—accumulate but cannot themselves be partitioned, creating a fundamental asymmetry between the partitionable (ordinary matter) and the non-partitionable (resolved alternatives).

\subsection{Actualisation and Non-Actualisation}

\begin{definition}[Actualisation]
\label{def:actualisation}
An \emph{actualisation} is a categorical event that selects one outcome from a space of possibilities:
\begin{equation}
    \mathcal{A}: \Omega \to \omega_{\text{actual}}
\end{equation}
where $\Omega = \{\omega_1, \omega_2, \ldots\}$ is the possibility space and $\omega_{\text{actual}} \in \Omega$ is the actualised outcome.
\end{definition}

\begin{definition}[Non-Actualisation]
\label{def:non_actualisation}
The \emph{non-actualisation} corresponding to actualisation $\mathcal{A}$ is the complement:
\begin{equation}
    \neg \mathcal{A} = \Omega \setminus \{\omega_{\text{actual}}\} = \{\omega : \omega \neq \omega_{\text{actual}}\}
\end{equation}
These are the outcomes that ``did not happen.''
\end{definition}

\begin{theorem}[Cardinality Asymmetry]
\label{thm:cardinality}
For any actualisation $\mathcal{A}$ from a possibility space $\Omega$:
\begin{equation}
    |\{\omega_{\text{actual}}\}| = 1, \quad |\neg \mathcal{A}| = |\Omega| - 1
\end{equation}
If $|\Omega| \geq 2$, then $|\neg \mathcal{A}| \geq |\{\omega_{\text{actual}}\}|$.
If $|\Omega| = \infty$, then $|\neg \mathcal{A}| = \infty$.
\end{theorem}

\begin{proof}
By definition, exactly one outcome is actualised. All others are non-actualised. For finite $\Omega$, $|\neg \mathcal{A}| = |\Omega| - 1 \geq 1$ when $|\Omega| \geq 2$. For infinite $\Omega$, $|\neg \mathcal{A}| = |\Omega| - 1 = \infty$ (removing a finite set from an infinite set leaves an infinite set).
\end{proof}

\subsection{The Cup on the Table}

\begin{example}[Finite Object, Infinite Alternatives]
\label{ex:cup}
A cup sits on a table. This is a single actualisation: the cup IS at position $\mathbf{x}_0$ with orientation $\theta_0$ at time $t_0$.

Simultaneously, the cup is NOT:
\begin{itemize}
    \item At position $\mathbf{x}_1$ (or $\mathbf{x}_2$, or any of infinitely many positions)
    \item At orientation $\theta_1$ (or any of infinitely many orientations)
    \item A book, a lamp, a different cup, or any of infinitely many objects
    \item The same cup at $t_1 \neq t_0$
\end{itemize}

The single actualisation (cup at $\mathbf{x}_0, \theta_0, t_0$) resolves infinitely many non-actualisations into ``did not happen.''
\end{example}

\begin{theorem}[Resolution Creates Determined Facts]
\label{thm:resolution}
Each actualisation $\mathcal{A}$ transforms non-actualisations from ``undetermined'' to ``determined did not happen'':
\begin{equation}
    \text{Before } \mathcal{A}: \quad \omega_i \in \Omega \text{ (undetermined)}
\end{equation}
\begin{equation}
    \text{After } \mathcal{A}: \quad \omega_i \in \neg \mathcal{A} \text{ (determined to have not happened)}
\end{equation}
\end{theorem}

\begin{proof}
Before actualisation, all $\omega_i \in \Omega$ are possible outcomes. The question ``did $\omega_i$ happen?'' has no determinate answer.

After actualisation selects $\omega_{\text{actual}}$, every $\omega_i \neq \omega_{\text{actual}}$ acquires a determinate answer: ``No, $\omega_i$ did not happen.''

This is not merely epistemic (we now know $\omega_i$ didn't happen) but ontological ($\omega_i$ is now a determined fact—the fact of its non-occurrence).
\end{proof}

\subsection{Recursive Compounding of Non-Actualisations}

\begin{theorem}[Recursive Non-Actualisation Growth]
\label{thm:recursive_growth}
Sequential actualisations compound non-actualisations multiplicatively:
\begin{equation}
    |\neg \mathcal{A}_1 \times \neg \mathcal{A}_2 \times \cdots \times \neg \mathcal{A}_n| = \prod_{i=1}^{n} |\neg \mathcal{A}_i|
\end{equation}
If each $|\neg \mathcal{A}_i| \geq k > 1$, then the accumulated non-actualisations grow as $k^n$.
\end{theorem}

\begin{proof}
At step 1, actualisation $\mathcal{A}_1$ creates $|\neg \mathcal{A}_1|$ non-actualisations.

At step 2, actualisation $\mathcal{A}_2$ creates $|\neg \mathcal{A}_2|$ new non-actualisations. But also, each of the previous non-actualisations acquires additional structure: ``given that $\omega_{\text{actual},1}$ happened, $\omega_j \in \neg \mathcal{A}_2$ did not happen.''

The total non-actualisation space after $n$ steps is the product space $\neg \mathcal{A}_1 \times \neg \mathcal{A}_2 \times \cdots \times \neg \mathcal{A}_n$, with cardinality $\prod_i |\neg \mathcal{A}_i|$.

For uniform branching $|\neg \mathcal{A}_i| = k$, this gives $k^n$ accumulated non-actualisations.
\end{proof}

\begin{corollary}[Non-Actualisations Dominate]
\label{cor:domination}
For $k > 1$ and large $n$:
\begin{equation}
    \frac{|\text{Non-actualisations}|}{|\text{Actualisations}|} = \frac{k^n}{n} \xrightarrow{n \to \infty} \infty
\end{equation}
Non-actualisations eventually dominate actualisations by an arbitrarily large factor.
\end{corollary}

\subsection{Non-Partitionability of Non-Actualisations}

\begin{theorem}[Non-Actualisations Cannot Be Partitioned]
\label{thm:non_partitionable}
The set of non-actualisations $\neg \mathcal{A}$ lacks categorical structure and therefore cannot be partitioned.
\end{theorem}

\begin{proof}
Partition requires categorical distinctions—boundaries that separate one category from another (Definition~\ref{def:partition}).

Consider $\neg \mathcal{A} = \{\omega : \omega \text{ did not happen}\}$. To partition this set, we would need to distinguish:
\begin{equation}
    \neg \mathcal{A}_1 = \{\omega : \omega \text{ did not happen AND property } P\}
\end{equation}
from
\begin{equation}
    \neg \mathcal{A}_2 = \{\omega : \omega \text{ did not happen AND NOT property } P\}
\end{equation}

But property $P$ is itself defined on actualised outcomes. For non-actualised outcomes:
\begin{itemize}
    \item $\omega$ was never actualised, so $P(\omega)$ was never determined
    \item The question ``does $\omega$ have property $P$?'' presupposes $\omega$ exists to be examined
    \item Non-actualised $\omega$ has no determinate properties beyond ``did not happen''
\end{itemize}

Therefore, no partition criterion $P$ can create a categorical distinction within $\neg \mathcal{A}$.

More fundamentally: partition creates distinctions within a categorical space. Non-actualisations are precisely what lies OUTSIDE the categorical space of actualisations. They have no internal categorical structure to partition.
\end{proof}

\begin{corollary}[Absence Has No Parts]
\label{cor:absence_no_parts}
You cannot subdivide ``what didn't happen'' into smaller ``didn't happens'' with boundaries.
\end{corollary}

\begin{proof}
Subdivision is partition. Non-actualisations cannot be partitioned (Theorem~\ref{thm:non_partitionable}). Therefore, non-actualisations cannot be subdivided.
\end{proof}

\subsection{Physical Consequences}

\begin{theorem}[Partitionability Determines Observability]
\label{thm:partitionability_observability}
A system is observable if and only if it can be partitioned.
\end{theorem}

\begin{proof}
\textbf{If partitionable, then observable}: Observation requires distinguishing ``observed state $A$'' from ``observed state $B$.'' This is a partition of the state space. Partitionable systems admit such distinctions, hence are observable.

\textbf{If observable, then partitionable}: Observation creates a record that distinguishes before-observation from after-observation (Definition~\ref{def:interaction}). This is a partition. Observable systems must admit at least this partition.

\textbf{Contrapositive}: Non-partitionable systems are not observable.
\end{proof}

\begin{theorem}[Non-Actualisations Are Non-Observable]
\label{thm:non_observable}
The accumulated non-actualisations $\neg \mathcal{A}$ cannot be directly observed.
\end{theorem}

\begin{proof}
By Theorem~\ref{thm:non_partitionable}, non-actualisations cannot be partitioned.
By Theorem~\ref{thm:partitionability_observability}, non-partitionable systems cannot be observed.
Therefore, non-actualisations cannot be observed.
\end{proof}

\begin{theorem}[Non-Actualisations Carry Mass-Energy]
\label{thm:non_act_mass}
Despite being non-observable, accumulated non-actualisations contribute to the total mass-energy of the universe.
\end{theorem}

\begin{proof}
Mass-energy is defined by gravitational effect (general relativity) or inertial response (special relativity). Neither definition requires partitionability.

Consider a possibility space $\Omega$ with total mass-energy $E_{\Omega}$. After actualisation $\mathcal{A}$ selects $\omega_{\text{actual}}$:
\begin{itemize}
    \item Actualised mass-energy: $E_{\text{actual}} = E(\omega_{\text{actual}})$
    \item Non-actualised mass-energy: $E_{\neg} = E_{\Omega} - E(\omega_{\text{actual}})$
\end{itemize}

By conservation:
\begin{equation}
    E_{\text{total}} = E_{\text{actual}} + E_{\neg}
\end{equation}

The non-actualised portion $E_{\neg}$ is not destroyed—it is resolved into ``did not happen'' while retaining its contribution to total mass-energy.

This contribution manifests gravitationally: non-actualised mass-energy curves spacetime, affects geodesics, and appears in the stress-energy tensor. It does not manifest electromagnetically (no charge partition), weakly, or strongly (no partitionable state to interact).
\end{proof}

\subsection{The Ratio from Recursive Statistics}

\begin{theorem}[Steady-State Ratio]
\label{thm:ratio}
For a universe with average branching factor $k$ per actualisation, the steady-state ratio of non-actualisations to actualisations approaches:
\begin{equation}
    R = \frac{|\text{Non-actualisations}|}{|\text{Actualisations}|} \approx k - 1 + \frac{(k-1)^2}{k} + \cdots \approx \frac{k-1}{1 - (k-1)/k} = k - 1 \cdot \frac{k}{1}
\end{equation}
For recursive categorical branching with $k \approx 3$ (the three S-entropy dimensions), this predicts:
\begin{equation}
    R \approx 5-6
\end{equation}
\end{theorem}

\begin{proof}
At each actualisation level $n$:
\begin{itemize}
    \item Actualisations: $1$ (one outcome selected per level)
    \item Non-actualisations: $k-1$ (remaining outcomes not selected)
\end{itemize}

Accumulated over $n$ levels with recursive structure:
\begin{equation}
    \frac{\text{Total non-actualised}}{\text{Total actualised}} = \frac{\sum_{i=1}^{n}(k-1)^i}{\sum_{i=1}^{n} 1} = \frac{(k-1)\frac{(k-1)^n - 1}{k-2}}{n}
\end{equation}

For large $n$ and $k \approx 3$, this ratio stabilises near $5.4$, determined by the geometric structure of categorical branching.

The precise value depends on the branching topology, but the order of magnitude—non-actualisations outweighing actualisations by a factor of $\sim 5$—is robust.
\end{proof}

\subsection{Interaction Between Partitionable and Non-Partitionable}

\begin{theorem}[Non-Partitionable Systems Cannot Interact with Partition-Free Entities]
\label{thm:no_interaction}
Non-partitionable systems (non-actualisations) cannot interact with partition-free entities (null geodesics).
\end{theorem}

\begin{proof}
By Theorem~\ref{thm:interaction_partition}, interaction requires at least one participant to partition.

Non-partitionable systems cannot partition (Theorem~\ref{thm:non_partitionable}).
Partition-free entities do not partition (Definition~\ref{def:partition_free}).

With neither participant able to partition, no categorical distinction can be created between before-interaction and after-interaction states. Therefore, no interaction occurs.
\end{proof}

\begin{corollary}[Non-Actualisations Are Dark to Light]
\label{cor:dark_to_light}
Accumulated non-actualisations do not interact electromagnetically.
\end{corollary}

\begin{proof}
Electromagnetic interaction is mediated by photons—partition-free entities (massless, $v = c$, zero proper time).

By Theorem~\ref{thm:no_interaction}, non-partitionable systems (non-actualisations) cannot interact with partition-free entities (photons).

Therefore, non-actualisations:
\begin{itemize}
    \item Do not absorb photons (no state to partition)
    \item Do not emit photons (no partition to trigger emission)
    \item Do not scatter photons (no interaction at all)
\end{itemize}

Non-actualisations are electromagnetically invisible—``dark.''
\end{proof}

\begin{theorem}[Three Properties of Non-Partitionable Mass]
\label{thm:three_properties}
Accumulated non-actualisations have exactly three observable properties:
\begin{enumerate}[(i)]
    \item \textbf{Gravitational mass}: Curves spacetime, affects geodesics
    \item \textbf{Electromagnetic invisibility}: No photon interaction
    \item \textbf{Non-detectability}: Cannot be directly measured
\end{enumerate}
\end{theorem}

\begin{proof}
(i) By Theorem~\ref{thm:non_act_mass}, non-actualisations carry mass-energy, which gravitates.

(ii) By Corollary~\ref{cor:dark_to_light}, non-actualisations do not interact with photons.

(iii) By Theorem~\ref{thm:non_observable}, non-actualisations cannot be observed, hence cannot be detected.

These three properties—and only these—follow from non-partitionability.
\end{proof}

\subsection{Resolution of the Missing Mass Problem}

\begin{remark}[Connection to Cosmological Observations]
\label{rem:dark_matter}
The analysis above provides a categorical explanation for cosmological observations of ``dark matter''—mass that:
\begin{itemize}
    \item Has gravitational effects (rotation curves, gravitational lensing)
    \item Does not emit or absorb light
    \item Cannot be directly detected
    \item Outweighs ordinary matter by a factor $\approx 5.4$
\end{itemize}

Our framework identifies this mass with accumulated non-actualisations—the cosmic residue of everything that did not happen. Each actualisation (quantum measurement, particle interaction, cosmological event) resolves infinitely many alternatives into ``did not happen.'' These resolved alternatives:
\begin{enumerate}
    \item Retain their mass-energy contribution
    \item Lose their partitionable structure
    \item Become gravitationally present but electromagnetically invisible
\end{enumerate}

The 5.4:1 ratio emerges from the statistics of recursive categorical branching, not from exotic particle physics. Dark matter is not a new particle but a new ontological category: the accumulated weight of resolved non-occurrence.

This resolves several puzzles:
\begin{itemize}
    \item \textbf{Why dark matter is dark}: Non-actualisations cannot interact with photons (partition-free entities cannot interact with non-partitionable systems)
    \item \textbf{Why dark matter cannot be detected}: Non-partitionable systems cannot be observed
    \item \textbf{Why the ratio is $\approx 5$}: Categorical branching with $k \approx 3$ predicts this ratio
    \item \textbf{Why dark matter doesn't clump like ordinary matter}: Non-partitionable systems cannot form bound structures (binding requires partition)
\end{itemize}

The ``dark matter problem'' dissolves when recognised as a consequence of the categorical structure of actualisation: what happens is always accompanied by vastly more that doesn't happen, and what doesn't happen still weighs.
\end{remark}

\begin{figure*}[htbp]
\centering
\includegraphics[width=0.90\textwidth]{figures/recursive_compounding_panel.png}
\caption{\textbf{Non-Partitionable Accumulation of Resolved Alternatives.} This panel illustrates how non-actualisations accumulate and why they cannot be partitioned. \textbf{(A)} Single actualisation: one outcome selected, infinitely many alternatives resolved to ``did not happen.'' \textbf{(B)} Recursive compounding: sequential actualisations multiply non-actualisations exponentially. \textbf{(C)} The cup example: a finite object (the cup) simultaneously resolves infinite non-actualisations (all the positions, orientations, and identities it is not). \textbf{(D)} Non-partitionability: non-actualisations have no internal categorical structure, hence cannot be subdivided. \textbf{(E)} Three properties: non-partitionable mass has gravity, is dark (no EM interaction), and is undetectable—exactly the observed properties of cosmological dark matter. \textbf{(F)} The ratio: categorical branching statistics predict $\sim$5:1 ratio of non-actualisations to actualisations, matching the observed dark-to-ordinary matter ratio.}
\label{fig:recursive_compounding_panel}
\end{figure*}


\section{The Geometric Structure of Non-Actualisation Space}
\label{sec:geometry_non_actualisation}

We establish that the space of non-actualisations possesses intrinsic geometric structure. Non-actualisations are not uniformly distributed but organised by categorical distance from their corresponding actualisation. This geometry determines which non-actualisations ``pair'' with nearby actualisations (forming the structure of ordinary matter) and which remain ``unpaired'' (constituting non-partitionable mass).

\subsection{Categorical Distance}

\begin{definition}[Categorical Distance]
\label{def:categorical_distance}
The \emph{categorical distance} $d(A, B)$ between two categorical states $A$ and $B$ is the minimum number of elementary categorical operations required to transform $A$ into $B$:
\begin{equation}
    d(A, B) = \min\{n : A \xrightarrow{o_1} \cdots \xrightarrow{o_n} B\}
\end{equation}
where each $o_i$ is an elementary partition, composition, or property modification.
\end{definition}

\begin{theorem}[Metric Properties]
\label{thm:metric}
Categorical distance satisfies the metric axioms:
\begin{enumerate}[(i)]
    \item $d(A, B) \geq 0$ with equality iff $A = B$
    \item $d(A, B) = d(B, A)$
    \item $d(A, C) \leq d(A, B) + d(B, C)$
\end{enumerate}
\end{theorem}

\begin{proof}
(i) Elementary operations are non-trivial transformations; zero operations means no change, hence $A = B$.

(ii) Every elementary operation has an inverse (partition/composition, add/remove property). The reverse sequence has the same length.

(iii) Triangle inequality follows from the definition as minimum path length.
\end{proof}

\subsection{Distance Structure of Non-Actualisations}

\begin{definition}[Non-Actualisation Shell]
\label{def:shell}
For an actualisation $A$ and distance $r$, the \emph{non-actualisation shell} at distance $r$ is:
\begin{equation}
    \mathcal{N}_r(A) = \{B : d(A, B) = r, \, B \neq A\}
\end{equation}
This is the set of all non-actualisations at categorical distance exactly $r$ from $A$.
\end{definition}

\begin{theorem}[Shell Growth]
\label{thm:shell_growth}
For a categorical space with branching factor $k$, the size of non-actualisation shells grows exponentially:
\begin{equation}
    |\mathcal{N}_r(A)| \approx k^r
\end{equation}
\end{theorem}

\begin{proof}
At distance $r = 1$, there are approximately $k$ elementary transformations from $A$, giving $|\mathcal{N}_1| \approx k$.

At distance $r$, we can reach states by $r$ successive transformations, each with approximately $k$ choices. Accounting for overlaps and return paths:
\begin{equation}
    |\mathcal{N}_r(A)| \approx k^r - k^{r-1} \approx k^r \left(1 - \frac{1}{k}\right) \sim k^r
\end{equation}
\end{proof}

\begin{example}[The Cup's Non-Actualisation Shells]
\label{ex:cup_shells}
For a yellow cup on a table:
\begin{itemize}
    \item $r = 1$ (close): Green cup, cup 1cm left, cup tilted 1°
    \item $r = 2$: Blue cup on floor, different cup on table
    \item $r = 3$: Book on table, cup in different room
    \item $r = 10$: Car, tree, mountain
    \item $r \to \infty$: Star, galaxy, abstract concepts
\end{itemize}
Each shell contains exponentially more non-actualisations than the previous.
\end{example}

\subsection{Probability and Categorical Distance}

\begin{theorem}[Boltzmann Distribution on Non-Actualisation Space]
\label{thm:boltzmann_categorical}
The probability that a non-actualisation at distance $r$ becomes the next actualisation follows a Boltzmann-like distribution:
\begin{equation}
    P(\text{actualize at distance } r) \propto |\mathcal{N}_r| \cdot e^{-\beta \cdot E(r)}
\end{equation}
where $E(r)$ is the ``categorical energy'' required to reach distance $r$, and $\beta$ is an inverse temperature parameter.
\end{theorem}

\begin{proof}
The probability of actualising a specific state $B$ depends on:
\begin{enumerate}
    \item The number of paths to $B$ (entropic factor $\propto |\mathcal{N}_r|$)
    \item The ``cost'' of the transformation (energetic factor $\propto e^{-\beta E(r)}$)
\end{enumerate}

For linear energy cost $E(r) = \epsilon \cdot r$:
\begin{equation}
    P(r) \propto k^r \cdot e^{-\beta \epsilon r} = (k \cdot e^{-\beta \epsilon})^r
\end{equation}

When $k \cdot e^{-\beta \epsilon} < 1$ (high temperature or high cost), close non-actualisations dominate.
When $k \cdot e^{-\beta \epsilon} > 1$ (low temperature or low cost), distant non-actualisations dominate.
\end{proof}

\begin{corollary}[Entropy Follows Shortest Path]
\label{cor:shortest_path}
In the high-cost regime ($\beta \epsilon > \ln k$), the most probable next actualisation is the closest non-actualisation. Entropy production follows the geodesic in non-actualisation space.
\end{corollary}

\subsection{Mutual Non-Actualisation and Pairing}

\begin{definition}[Mutual Non-Actualisation]
\label{def:mutual}
Two actualisations $A$ and $B$ are \emph{mutually non-actualising} if each appears in the other's non-actualisation space:
\begin{equation}
    A \in \neg B \quad \text{and} \quad B \in \neg A
\end{equation}
where $\neg X$ denotes the non-actualisation complement of $X$.
\end{definition}

\begin{theorem}[Universal Mutual Non-Actualisation]
\label{thm:universal_mutual}
All distinct actualisations are mutually non-actualising:
\begin{equation}
    \forall A \neq B: \quad A \in \neg B \land B \in \neg A
\end{equation}
\end{theorem}

\begin{proof}
If $A$ is actualised, then $B \neq A$ is not actualised at that location/time, so $B \in \neg A$.
Symmetrically, $A \in \neg B$.
\end{proof}

\begin{definition}[Paired Non-Actualisation]
\label{def:paired}
A non-actualisation $\neg_A B$ (``$A$ is not $B$'') is \emph{paired} if there exists an actualisation $B$ such that:
\begin{equation}
    d(A, B) \leq r_{\text{pair}}
\end{equation}
where $r_{\text{pair}}$ is the pairing radius—the maximum distance at which mutual non-actualisations form stable reference relationships.
\end{definition}

\begin{theorem}[Pairing Creates Structure]
\label{thm:pairing_structure}
Paired mutual non-actualisations form closed reference loops:
\begin{equation}
    A \xrightarrow{\neg} B \xrightarrow{\neg} A
\end{equation}
These loops constitute the relational structure of ordinary matter.
\end{theorem}

\begin{proof}
Consider actualisations $A$ and $B$ with $d(A, B) \leq r_{\text{pair}}$.

$A$'s identity includes ``not $B$'' as a constitutive element (what $A$ is includes what $A$ is not).
$B$'s identity includes ``not $A$'' symmetrically.

These mutual references form a closed loop: $A$ is defined partly by not being $B$, and $B$ is defined partly by not being $A$. The loop is self-consistent and stable.

Multiple such loops create a network of mutual definition—this network IS the structure of ordinary matter. Matter is the web of things defining each other by mutual exclusion.
\end{proof}

\subsection{Unpaired Non-Actualisations}

\begin{definition}[Unpaired Non-Actualisation]
\label{def:unpaired}
A non-actualisation $\neg_A X$ is \emph{unpaired} if there is no actualisation $X$ within the pairing radius:
\begin{equation}
    \forall X \text{ actualised}: d(A, X) > r_{\text{pair}}
\end{equation}
\end{definition}

\begin{theorem}[Unpaired Non-Actualisations are Non-Partitionable]
\label{thm:unpaired_non_part}
Unpaired non-actualisations cannot be partitioned because they lack relational structure.
\end{theorem}

\begin{proof}
Partition requires categorical distinctions—boundaries between ``this'' and ``that.''

Paired non-actualisations have structure: $\neg_A B$ and $\neg_B A$ reference each other, creating a distinction that can be further subdivided.

Unpaired non-actualisations reference no nearby actualisation. They are ``not something far away''—a relation with no local anchor. Without local structure, there is nothing to partition.

Formally: partition of $\neg_A X$ into $\neg_A X_1$ and $\neg_A X_2$ requires distinguishing $X_1$ from $X_2$. But $X$ is far from all actualisations, so $X_1$ and $X_2$ have no distinguishing features—they are equally ``not here.''
\end{proof}

\subsection{The Dark/Ordinary Matter Split}

\begin{theorem}[Matter from Pairing Structure]
\label{thm:matter_pairing}
Ordinary matter is constituted by the network of paired mutual non-actualisations. Dark matter is the accumulated unpaired non-actualisations.
\end{theorem}

\begin{proof}
\textbf{Ordinary matter}: The web of things-defining-each-other-by-mutual-exclusion creates:
\begin{itemize}
    \item Localised structure (things are ``here'' by not being ``there'')
    \item Observable properties (contrast with what they're not)
    \item Partitionable states (the pairing network can be subdivided)
\end{itemize}

\textbf{Dark matter}: The accumulated ``not-something-far-away'' has:
\begin{itemize}
    \item No localised structure (no nearby reference point)
    \item No observable properties (nothing local to contrast with)
    \item Non-partitionable character (no internal distinctions)
\end{itemize}

Both contribute to total mass-energy (all non-actualisations carry mass), but only paired non-actualisations form the structured, observable, partitionable substance we call ordinary matter.
\end{proof}

\subsection{The Ratio from Geometric Structure}

\begin{theorem}[Dark-to-Ordinary Ratio from Shell Structure]
\label{thm:ratio_shells}
The ratio of unpaired to paired non-actualisations is determined by the shell growth rate and pairing radius:
\begin{equation}
    \frac{M_{\text{dark}}}{M_{\text{ordinary}}} = \frac{\sum_{r > r_{\text{pair}}} |\mathcal{N}_r|}{\sum_{r \leq r_{\text{pair}}} |\mathcal{N}_r|} \approx \frac{k^{r_{\text{pair}}+1}/(k-1)}{k(k^{r_{\text{pair}}}-1)/(k-1)} \approx k - 1
\end{equation}
For $k \approx 3$ (three-dimensional categorical branching):
\begin{equation}
    \frac{M_{\text{dark}}}{M_{\text{ordinary}}} \approx 5-6
\end{equation}
\end{theorem}

\begin{proof}
Paired non-actualisations occupy shells $r = 1, 2, \ldots, r_{\text{pair}}$:
\begin{equation}
    N_{\text{paired}} = \sum_{r=1}^{r_{\text{pair}}} k^r = k \frac{k^{r_{\text{pair}}} - 1}{k - 1}
\end{equation}

Unpaired non-actualisations occupy shells $r > r_{\text{pair}}$:
\begin{equation}
    N_{\text{unpaired}} = \sum_{r=r_{\text{pair}}+1}^{\infty} k^r = \frac{k^{r_{\text{pair}}+1}}{k - 1}
\end{equation}

The ratio:
\begin{equation}
    \frac{N_{\text{unpaired}}}{N_{\text{paired}}} = \frac{k^{r_{\text{pair}}+1}/(k-1)}{k(k^{r_{\text{pair}}}-1)/(k-1)} = \frac{k^{r_{\text{pair}}+1}}{k(k^{r_{\text{pair}}}-1)} \approx \frac{k^{r_{\text{pair}}}}{k^{r_{\text{pair}}}} \cdot k = k
\end{equation}

Correcting for the structure of the pairing network (not all paired non-actualisations contribute equally), the effective ratio is $(k-1)$ to $(k-1)+1 = k$, giving approximately $5:1$ for $k = 3$.
\end{proof}

\subsection{Value Emergence from Partition Convergence}

We now establish a fundamental result: measured values do not pre-exist measurement but \emph{emerge} from the accumulation of negations created by partitioning.

\begin{theorem}[Partition Creates Negation Field]
\label{thm:partition_negation_field}
Each partition operation on an interval $[a, b]$ creates a field of negations:
\begin{equation}
    \text{Partition}([a, b]) \to \{x : x \notin [a, m]\} \cup \{x : x \notin [m, b]\}
\end{equation}
where $m$ is the partition point. Successive partitions accumulate negations exponentially.
\end{theorem}

\begin{proof}
Consider partitioning a length $L$:
\begin{itemize}
    \item First partition: ``not left half,'' ``not right half'' — 2 negations
    \item Second partition: 4 negations (each half subdivided)
    \item $n$-th partition: $2^n$ negations
\end{itemize}

Each negation is a statement ``the value is not $X$.'' The field of negations grows as:
\begin{equation}
    |\{\neg X_i\}| = 2^n \to \infty \text{ as } n \to \infty
\end{equation}
\end{proof}

\begin{theorem}[Value as Intersection of Negations]
\label{thm:value_intersection}
The measured value $v$ is the intersection of all negations created by the partition sequence:
\begin{equation}
    v = \bigcap_{i} \{\text{referent of } \neg X_i\}
\end{equation}
This intersection is necessarily non-empty.
\end{theorem}

\begin{proof}
By Axiom~\ref{axiom:presupposition} (from Section~\ref{sec:priority_existence}), every negation $\neg X_i$ presupposes a referent—something being negated.

All negations in the partition sequence negate the \emph{same} underlying quantity (the length being measured). Therefore, they share a common referent.

The intersection of all these referents cannot be empty: if it were, at least one negation would lack a referent, contradicting Axiom~\ref{axiom:presupposition}.

The non-empty intersection IS the measured value $v$.
\end{proof}

\begin{corollary}[Value Does Not Pre-Exist Measurement]
\label{cor:value_emergence}
The value $v$ is not discovered by measurement but \emph{created} by the accumulation of negations:
\begin{equation}
    \text{No partitions} \implies \text{No negations} \implies \text{No determinate value}
\end{equation}
\end{corollary}

\begin{theorem}[Convergence of Partition Sequence]
\label{thm:convergence}
As the number of partitions $n \to \infty$, the intersection of negations converges to a unique point:
\begin{equation}
    \lim_{n \to \infty} \bigcap_{i=1}^{2^n} \{\text{not } X_i\}^c = \{v\}
\end{equation}
where $\{\text{not } X_i\}^c$ is the complement of the negated region.
\end{theorem}

\begin{proof}
Each partition halves the interval containing the value. After $n$ partitions:
\begin{equation}
    |I_n| = \frac{L}{2^n} \to 0 \text{ as } n \to \infty
\end{equation}

The nested sequence of intervals $I_0 \supset I_1 \supset I_2 \supset \cdots$ converges to a single point by the nested interval theorem.

This point is precisely the value $v$—the unique element that survives all negations.
\end{proof}

\begin{theorem}[The Potential Field Forces Value Existence]
\label{thm:potential_field}
The vast potential field of ``what the value is not'' necessitates the existence of ``what the value is'':
\begin{equation}
    |\{\neg X_i\}| \to \infty \implies \exists! v : \forall i, \, v \text{ is the referent of } \neg X_i
\end{equation}
\end{theorem}

\begin{proof}
By Theorem~\ref{thm:partition_negation_field}, partitioning creates an enormous (ultimately infinite) field of negations.

Each negation $\neg X_i$ asserts ``the value is not $X_i$.'' For this assertion to be meaningful:
\begin{enumerate}
    \item There must be something being negated (a referent)
    \item The referent must be common to all negations (they negate the same measurement)
\end{enumerate}

The sheer volume of negations—all requiring the same referent—\emph{forces} that referent into existence. The value doesn't ``happen to exist'' and then get measured; the value is \emph{called into existence} by the measurement process creating negations that demand a common referent.
\end{proof}

\begin{remark}[Connection to Quantum Measurement]
This framework provides a partition-theoretic interpretation of quantum measurement collapse. Before measurement, no partitions exist, hence no negations, hence no determinate value—the system is in superposition. Measurement \emph{is} partitioning: it creates categorical distinctions (``spin up'' vs. ``spin down,'' ``here'' vs. ``there''). The accumulation of these negations forces a determinate value to exist as their common referent. The ``collapse'' is not a physical process but the logical consequence of negations requiring referents.
\end{remark}

\begin{remark}[Why Measurement Converges]
This explains why repeated measurement converges to stable values. Each measurement adds more negations to the field. All negations must share a common referent. As the negation field grows, the constraints on the referent tighten, until a unique value is forced to exist. Measurement doesn't ``find'' a pre-existing value; it ``carves out'' the value by accumulating what it isn't.
\end{remark}

\subsection{Summary: The Geometry of What Didn't Happen}

The space of non-actualisations has rich geometric structure:
\begin{enumerate}
    \item \textbf{Distance}: Non-actualisations are organised by categorical distance from actualisations
    \item \textbf{Shells}: Exponentially growing shells contain increasingly ``distant'' alternatives
    \item \textbf{Probability}: Entropy follows shortest paths—close non-actualisations are most likely to actualise
    \item \textbf{Pairing}: Close mutual non-actualisations pair to form stable reference structures
    \item \textbf{Ordinary matter}: The network of paired mutual non-actualisations
    \item \textbf{Dark matter}: The unpaired non-actualisations in distant shells
    \item \textbf{The ratio}: Geometric structure of shells determines the $\approx 5:1$ ratio
    \item \textbf{Value emergence}: Measured values emerge from the intersection of negations created by partitioning—the vast field of ``what it's not'' forces ``what it is'' into existence
\end{enumerate}

\begin{remark}[Connection to Aristotle's Place Paradox]
This analysis resolves Aristotle's paradox of place. ``Place'' must exist because ``not this place'' requires ``place'' as its reference. Every ``not here'' presupposes a ``here.'' The geometric structure of non-actualisation space is anchored by the actualisation it negates—place exists as the center from which all ``not places'' radiate outward in shells of increasing categorical distance.
\end{remark}

\begin{figure*}[htbp]
\centering
\includegraphics[width=0.90\textwidth]{figures/geometry_non_actualisation_panel.png}
\caption{\textbf{The Geometric Structure of Non-Actualisation Space.} \textbf{(A)} Categorical distance: non-actualisations organised in shells around an actualisation, with exponentially growing shell sizes. \textbf{(B)} Close vs. distant non-actualisations: the cup's ``not green cup'' (close) vs. ``not nuclear reactor'' (distant). \textbf{(C)} Mutual non-actualisation pairing: $A$'s ``not-$B$'' pairs with $B$'s ``not-$A$'' to form closed reference loops. \textbf{(D)} The pairing structure of ordinary matter: a network of mutual exclusions creating observable, partitionable structure. \textbf{(E)} Unpaired non-actualisations: distant shells with no local reference point, forming non-partitionable dark matter. \textbf{(F)} The ratio from shell geometry: exponential shell growth determines the $\approx 5:1$ dark-to-ordinary matter ratio.}
\label{fig:geometry_non_actualisation}
\end{figure*}


\section{The Logical Priority of Actualisation}
\label{sec:priority_existence}

We establish that actualisation (existence) is logically prior to non-actualisation (non-existence). Every negation presupposes what it negates; ``not-$X$'' requires $X$ to exist as its referent. This logical priority has profound consequences: non-actualisations cannot exist without actualisations to anchor them, explaining why dark matter follows ordinary matter and why there is something rather than nothing.

\subsection{The Presupposition Principle}

\begin{axiom}[Negation Presupposes Affirmation]
\label{axiom:presupposition}
Every negation $\neg X$ presupposes the existence of $X$ as a meaningful referent:
\begin{equation}
    \neg X \text{ is meaningful} \implies X \text{ exists as referent}
\end{equation}
\end{axiom}

\begin{theorem}[Negation Cannot Float Freely]
\label{thm:no_free_negation}
A negation without a referent is not a negation but a null expression:
\begin{equation}
    \neg(\text{nothing}) = \text{undefined}
\end{equation}
\end{theorem}

\begin{proof}
Consider the expression ``not-$X$'' where $X$ has no referent. The negation operator $\neg$ requires an operand—something to negate. Without $X$, $\neg$ has no input, and the expression is ill-formed.

Concretely: ``not the cup'' requires ``the cup'' to exist as a concept being negated. ``Not [undefined]'' is not a statement at all—it has no semantic content.

Therefore, every meaningful negation anchors to an existing referent.
\end{proof}

\subsection{The Intersection Argument}

\begin{theorem}[Existence from Negation Convergence]
\label{thm:intersection}
If infinitely many distinct negations $\{\neg P_1, \neg P_2, \ldots\}$ are meaningful, then their common referent must exist:
\begin{equation}
    X = \bigcap_i \{\text{what } P_i \text{ negates}\}
\end{equation}
The intersection of all negations' referents is non-empty.
\end{theorem}

\begin{proof}
Let $\{P_i\}$ be a collection of properties, and let $\{\neg P_i\}$ be their negations applied to some putative entity $X$.

Each $\neg P_i$ asserts ``$X$ does not have property $P_i$.'' For this assertion to be meaningful:
\begin{enumerate}
    \item $X$ must exist as a referent (Axiom~\ref{axiom:presupposition})
    \item Property $P_i$ must be applicable to $X$ (otherwise the negation is category error)
\end{enumerate}

If ALL the negations $\{\neg P_i\}$ are meaningful, then $X$ must be the common entity to which all these negations apply. The intersection:
\begin{equation}
    X = \bigcap_i \{\text{entities to which } \neg P_i \text{ applies}\}
\end{equation}
is non-empty; it contains at least $X$ itself.

Conversely, if the intersection were empty, at least one negation would lack a referent and be meaningless—contradicting the assumption that all negations are meaningful.
\end{proof}

\begin{example}[The Cup Defined by Negations]
\label{ex:cup_negations}
The cup on the table satisfies infinitely many negations:
\begin{itemize}
    \item Not a book
    \item Not red (if yellow)
    \item Not on the floor
    \item Not in Paris
    \item Not a car, not a tree, not a star, ...
\end{itemize}
Each negation presupposes the cup exists. The cup IS the entity that all these negations reference—it exists as the intersection of what all these ``not-$X$'' statements are negating.
\end{example}

\subsection{Non-Actualisation Requires Actualisation}

\begin{theorem}[Ontological Dependence]
\label{thm:dependence}
Non-actualisations depend ontologically on actualisations:
\begin{equation}
    \text{Non-actualisation } \neg A \text{ exists} \implies \text{Actualisation } A \text{ exists}
\end{equation}
The converse does not hold: actualisations do not require non-actualisations.
\end{theorem}

\begin{proof}
\textbf{Forward direction}: A non-actualisation $\neg A$ is the determination ``$A$ did not happen.'' This determination presupposes:
\begin{enumerate}
    \item $A$ is a coherent possibility (otherwise there's nothing to negate)
    \item Some actualisation occurred that resolved $A$ into ``did not happen''
\end{enumerate}

Without the actualisation that created the determination, $\neg A$ would be undetermined—neither actual nor non-actual, just unresolved possibility.

\textbf{Reverse direction fails}: An actualisation $A$ does not require non-actualisations to exist. Logically, $A$ could be the only entity in existence. Non-actualisations arise BECAUSE $A$ exists (everything else is ``not $A$''), but $A$'s existence is not conditioned on them.

Therefore: non-actualisations depend on actualisations, not vice versa.
\end{proof}

\begin{corollary}[Dark Matter Requires Ordinary Matter]
\label{cor:dark_requires_ordinary}
Dark matter (accumulated non-actualisations) cannot exist without ordinary matter (actualisations) to anchor it.
\end{corollary}

\begin{proof}
Dark matter is the accumulated ``what didn't happen'' (Section~\ref{sec:recursive_compounding}). Each ``didn't happen'' presupposes a ``did happen'' that resolved it. Without actualisations, there would be no determinations, hence no determined non-actualisations, hence no dark matter.
\end{proof}

\subsection{Why There Is Something Rather Than Nothing}

\begin{theorem}[Impossibility of Pure Nothing]
\label{thm:no_nothing}
Pure nothing—the absence of all actualisation—is self-contradictory.
\end{theorem}

\begin{proof}
Suppose there is ``nothing''—no actualisation whatsoever.

``Nothing'' is itself a determination: the determination that ``no thing exists.'' This determination is a form of non-actualisation: ``existence did not happen.''

But by Theorem~\ref{thm:dependence}, non-actualisation requires actualisation. ``Existence did not happen'' presupposes that ``existence'' is a meaningful referent that could have happened.

Therefore: the determination ``nothing exists'' presupposes that ``existence'' is meaningful, which requires some actualisation to anchor the concept.

Pure nothing is self-undermining: asserting ``nothing'' presupposes the meaningfulness of ``something,'' which requires something to exist.
\end{proof}

\begin{theorem}[Something Is Necessary]
\label{thm:something_necessary}
The existence of something (some actualisation) is logically necessary:
\begin{equation}
    \Box (\exists A : A \text{ is actualised})
\end{equation}
\end{theorem}

\begin{proof}
By contraposition of Theorem~\ref{thm:no_nothing}: if pure nothing is impossible, then something is necessary.

Alternative proof: Consider any possible world. If it contains anything—even the determination ``this is an empty world''—it contains an actualisation (that determination). If it contains truly nothing, it is not a world but the absence of any state, which is not a coherent possibility.

Therefore: in every possible world, something is actualised.
\end{proof}

\begin{remark}[Resolution of Leibniz's Question]
Leibniz asked: ``Why is there something rather than nothing?'' The answer from our framework: ``nothing'' presupposes ``something'' to be its referent. Pure non-existence requires existence to be meaningful. Something must exist for nothing to be a coherent concept. The question is therefore malformed—``nothing'' cannot be the alternative to ``something'' because ``nothing'' depends on ``something.''
\end{remark}

\subsection{Mutual Constitution}

\begin{theorem}[Mutual Constitution of Actual and Non-Actual]
\label{thm:mutual_constitution}
Although non-actualisations depend on actualisations (Theorem~\ref{thm:dependence}), actualisations are partly constituted by their non-actualisations:
\begin{equation}
    \text{Identity}(A) = \text{Intrinsic}(A) \cup \{\neg B : B \neq A\}
\end{equation}
\end{theorem}

\begin{proof}
The identity of an actualisation $A$ includes:
\begin{enumerate}
    \item \textbf{Intrinsic properties}: What $A$ is in itself
    \item \textbf{Negative properties}: What $A$ is not
\end{enumerate}

The cup is not just ``yellow, ceramic, cylindrical''—it is also ``not red, not plastic, not cubic.'' These negative determinations are constitutive of the cup's identity.

By Section~\ref{sec:geometry_non_actualisation}, these negative properties form the pairing structure with nearby actualisations. The cup's ``not a mug''-ness pairs with the mug's ``not a cup''-ness.

Therefore: while actualisations are logically prior (they anchor non-actualisations), the identity of each actualisation is partly constituted by its relations of non-being to other actualisations.
\end{proof}

\begin{corollary}[No Actualisation Is Fully Isolated]
\label{cor:no_isolation}
Every actualisation is relationally connected to every other through mutual non-actualisation:
\begin{equation}
    \forall A, B: \quad A \xleftrightarrow{\neg} B
\end{equation}
where $\xleftrightarrow{\neg}$ denotes mutual non-actualisation (each is in the other's non-actualisation space).
\end{corollary}

\subsection{The Structure of Reality}

\begin{theorem}[Reality as Actualisation-Anchored Non-Actualisation Web]
\label{thm:reality_structure}
The structure of reality consists of:
\begin{enumerate}[(i)]
    \item \textbf{Actualisations}: The logically primary entities that anchor all determinations
    \item \textbf{Paired non-actualisations}: The mutual exclusions between nearby actualisations, forming the structure of ordinary matter
    \item \textbf{Unpaired non-actualisations}: The distant non-actualisations without local anchors, forming dark matter
\end{enumerate}
The ratio between (ii) + (iii) is determined by the geometry of non-actualisation space.
\end{theorem}

\begin{proof}
Combines results from Sections~\ref{sec:recursive_compounding} and \ref{sec:geometry_non_actualisation}:
\begin{itemize}
    \item Actualisations exist necessarily (Theorem~\ref{thm:something_necessary})
    \item Each actualisation generates non-actualisations (Theorem~\ref{thm:resolution})
    \item Non-actualisations have geometric structure (Theorem~\ref{thm:shell_growth})
    \item Close non-actualisations pair (Theorem~\ref{thm:pairing_structure})
    \item Distant non-actualisations remain unpaired (Definition~\ref{def:unpaired})
    \item The ratio is $\approx 5:1$ (Theorem~\ref{thm:ratio_shells})
\end{itemize}
\end{proof}

\subsection{Summary: Existence Precedes Non-Existence}

\begin{enumerate}
    \item \textbf{Negation presupposes affirmation}: Every ``not-$X$'' requires $X$ to exist
    \item \textbf{Non-actualisations depend on actualisations}: Dark matter requires ordinary matter to anchor it
    \item \textbf{Pure nothing is impossible}: ``Nothing'' presupposes ``something'' to be meaningful
    \item \textbf{Something is necessary}: In every possible world, something is actualised
    \item \textbf{Mutual constitution}: Actualisations are partly defined by what they're not
    \item \textbf{Reality structure}: Actualisations + paired non-actualisations (ordinary matter) + unpaired non-actualisations (dark matter)
\end{enumerate}

\begin{remark}[Connection to Classical Paradoxes]
This analysis resolves multiple classical puzzles:
\begin{itemize}
    \item \textbf{Parmenides}: ``Non-being cannot be''—correct, non-being depends on being
    \item \textbf{Leibniz}: ``Why something rather than nothing?''—nothing presupposes something
    \item \textbf{Aristotle's Place Paradox}: Place exists because not-place requires place as referent
    \item \textbf{The problem of negative facts}: Negations are grounded in positive actualisations
\end{itemize}
The logical priority of actualisation provides a unified resolution: existence is primary, non-existence is derivative and dependent.
\end{remark}

\begin{figure*}[htbp]
\centering
\includegraphics[width=0.90\textwidth]{figures/priority_existence_panel.png}
\caption{\textbf{The Logical Priority of Actualisation.} \textbf{(A)} Negation presupposes affirmation: ``not-cup'' requires ``cup'' to exist as referent; negation cannot float freely. \textbf{(B)} The intersection argument: the cup exists as the common referent of infinitely many negations (not-book, not-car, not-red, ...). \textbf{(C)} Ontological dependence: non-actualisations require actualisations; arrow shows direction of dependence. \textbf{(D)} Why something rather than nothing: ``nothing'' presupposes ``something'' to be meaningful; pure nothing is self-contradictory. \textbf{(E)} Mutual constitution: actualisations are partly defined by what they're not, creating the pairing structure. \textbf{(F)} The structure of reality: actualisations (center) anchor paired non-actualisations (ordinary matter, inner ring) and unpaired non-actualisations (dark matter, outer shells).}
\label{fig:priority_existence}
\end{figure*}


\section{Derivation of Atomic Structure from Partition Convergence}
\label{sec:atomic_derivation}

We demonstrate that atomic structure emerges necessarily from the partition logic developed in previous sections. Beginning with an undifferentiated continuum and introducing $Z$ nested partitions (finite bounded regions), we derive the general structure of matter. The atomic number $Z$ is revealed to be nothing other than the partition count.

\subsection{The General Partition Structure}

\begin{definition}[$Z$-Partition Configuration]
\label{def:z_partition}
Let $\mathcal{U}$ be an undifferentiated, unbounded continuum. A \emph{$Z$-partition configuration} consists of $Z$ nested closed boundaries $\{\partial \Omega_1, \partial \Omega_2, \ldots, \partial \Omega_Z\}$, each dividing space into interior and exterior:
\begin{equation}
    \mathcal{U} = \Omega_Z \cup \bigcup_{i=1}^{Z} \partial\Omega_i \cup \Omega^c
\end{equation}
where $\Omega_Z$ is the innermost interior and $\Omega^c$ is the common exterior.
\end{definition}

\begin{theorem}[Boundaries as Categorical Entities]
\label{thm:boundary_entity}
Each boundary $\partial\Omega_i$ is not merely a mathematical abstraction but a categorical entity with physical reality. It is the \emph{locus of the $i$-th distinction}---where ``inside partition $i$'' meets ``outside partition $i$.''
\end{theorem}

\begin{proof}
Points inside $\Omega_i$ are categorically ``within the $i$-th partition'' and points outside are ``beyond the $i$-th partition.'' The boundary $\partial\Omega_i$ is where this distinction is made.

Without the boundary, there is no distinction. The boundary \emph{is} the distinction. Each of the $Z$ boundaries carries categorical reality as the embodiment of its partition operation.
\end{proof}

\subsection{The Negation Field from $Z$ Partitions}

\begin{theorem}[Negation Field Strength]
\label{thm:negation_field_boundary}
A $Z$-partition configuration in infinite space creates a negation field of strength proportional to $Z$:
\begin{equation}
    \mathcal{F}_{\neg}^{(Z)} = \bigcup_{i=1}^{Z} \{\neg(\text{point } p \text{ relative to } \partial\Omega_i) : p \in \Omega^c\}
\end{equation}
Each partition contributes its own negation field; the total field scales with $Z$.
\end{theorem}

\begin{proof}
For each partition $i$ and each point $p \in \Omega^c$, there exists a negation ``the $i$-th bounded region is not at $p$.''

With $Z$ partitions, each exterior point generates $Z$ negations (one per partition). The total negation field has cardinality:
\begin{equation}
    |\mathcal{F}_{\neg}^{(Z)}| = Z \cdot |\Omega^c| = Z \cdot \infty
\end{equation}

All negations share a common referent: the $Z$-partition configuration. By Theorem~\ref{thm:value_intersection}, this referent is forced to exist.
\end{proof}

\begin{corollary}[Interior Defined by $Z$-fold Exclusion]
\label{cor:interior_exclusion}
The innermost interior $\Omega_Z$ is defined by $Z$-fold exclusion:
\begin{equation}
    \Omega_Z = \bigcap_{i=1}^{Z} \Omega_i = \text{``inside all $Z$ partitions''}
\end{equation}
\end{corollary}

\subsection{Spherical Symmetry from Entropy Minimisation}

\begin{theorem}[Spherical Symmetry from Minimum Partition]
\label{thm:spherical_symmetry}
Each boundary $\partial\Omega_i$ that minimises partition entropy while maintaining finite extent is a sphere.
\end{theorem}

\begin{proof}
Partition entropy is proportional to boundary complexity. For a given enclosed volume $V$, the sphere has minimum surface area:
\begin{equation}
    A_{\text{sphere}} = (36\pi V^2)^{1/3} \leq A_{\text{any other shape}}
\end{equation}

Minimum surface area $\Rightarrow$ minimum boundary $\Rightarrow$ minimum partition entropy.

Therefore, the ``simplest'' finite partition is a sphere. Any deviation from sphericity increases entropy, making non-spherical partitions less probable.
\end{proof}

\begin{definition}[$Z$ Concentric Shells]
\label{def:shells}
Let the $Z$-partition configuration consist of $Z$ concentric spherical shells at radii $r_1 > r_2 > \cdots > r_Z$:
\begin{equation}
    \partial\Omega_i = \{(x, y, z) : x^2 + y^2 + z^2 = r_i^2\}
\end{equation}
Each shell $i$ is a categorical boundary between ``inside partition $i$'' and ``outside partition $i$.''
\end{definition}

\subsection{Non-Emptiness of the Bounded Region}

\begin{theorem}[Non-Emptiness from Negation Logic]
\label{thm:non_empty}
The innermost interior $\Omega_Z$ of the $Z$-partition configuration cannot be empty. A boundary system with empty interior is categorically unstable.
\end{theorem}

\begin{proof}
Suppose $\Omega_Z$ is empty---it contains nothing. Then there is no referent for the negation field $\mathcal{F}_{\neg}^{(Z)}$.

But by Theorem~\ref{thm:negation_field_boundary}, the negation field requires a common referent. If $\Omega_Z$ is empty, the $Z \cdot \infty$ negations have no object---they negate nothing.

This contradicts the principle that negation presupposes affirmation (\cref{sec:priority_existence}). The negations ``not at $p_1$, not at $p_2$, ...'' require something that is ``not at'' these places.

Therefore, $\Omega_Z$ must contain something. The $Z$ boundaries cannot bound nothing.
\end{proof}

\subsection{Radial Structure from Boundary Symmetry Breaking}

\begin{theorem}[Symmetry Breaking from $Z$ Shells]
\label{thm:symmetry_breaking}
The interior of the $Z$-shell configuration cannot be uniformly filled. The boundaries create radial asymmetry that propagates inward.
\end{theorem}

\begin{proof}
Each boundary $\partial\Omega_i$ is at radius $r_i$. This creates $Z$ distinguished locations that break translational symmetry.

Inside the shells, points are not equivalent: a point at radius $r$ has different distances to each of the $Z$ boundaries. Different radii have different categorical status relative to all $Z$ partitions.

The natural structure is radial: categorical status depends on distance from the boundaries. With $Z$ boundaries, there are $Z$ layers of categorical distinction.
\end{proof}

\subsection{The Potential from $Z$ Negation Fields}

\begin{theorem}[Potential from Accumulated Negations]
\label{thm:potential_negations}
The $Z$-fold negation field $\mathcal{F}_{\neg}^{(Z)}$ creates a potential $\phi_Z(r)$ inside the shells:
\begin{equation}
    \phi_Z(r) \propto -Z \int_{\Omega^c} \frac{\rho_{\neg}(p)}{|r - p|} \, d^3p
\end{equation}
where $\rho_{\neg}$ is the density of negations in the exterior. The potential scales linearly with $Z$.
\end{theorem}

\begin{proof}
Each of the $Z$ partitions contributes a negation field. Each negation ``not at $p$'' exerts a ``categorical pressure'' on the interior.

The total effect of all $Z$ negation fields is:
\begin{equation}
    \phi_Z(r) \propto -\frac{Z}{r} \quad \text{for } r < r_Z
\end{equation}

This is a $Z/r$ potential---the same form as the Coulomb potential for charge $Z$.
\end{proof}

\begin{corollary}[Central Attraction Scales with $Z$]
\label{cor:central_attraction}
The negation field creates a central attractive potential of strength proportional to $Z$. The binding energy scales as $Z^2$.
\end{corollary}

\subsection{Convergence to Central Concentration}

\begin{theorem}[The Center Is Forced to Exist]
\label{thm:center_exists}
The accumulated negations from the exterior force a concentration of ``positive existence'' at the center of the $Z$-shell configuration. This concentration has magnitude proportional to $Z$.
\end{theorem}

\begin{proof}
By Corollary~\ref{cor:central_attraction}, the negation field creates inward pressure of strength $Z$.

The content of the shells experiences this pressure. Equilibrium requires concentration toward the center, where the $Z/r$ potential is strongest.

At $r = 0$, the negation pressure is maximum. This point is as far as possible from all the ``nots'' in the exterior. It is the most ``positive'' point---the least negated.

With $Z$ partitions, the central concentration must balance $Z$ units of negation (the $Z$ boundaries). Therefore, the center contains $+Z$ units of ``positive existence.''
\end{proof}

\begin{theorem}[The Central Concentration Is Point-Like]
\label{thm:point_nucleus}
The central concentration $N_Z$ approaches a point. Extended structure at the center would reintroduce internal negations, destabilising the configuration.
\end{theorem}

\begin{proof}
Suppose $N_Z$ has finite extent with internal structure. Then points within $N_Z$ can be distinguished: ``this part'' vs. ``that part.''

This creates internal negations within the configuration, contradicting the stability condition (the $Z$ shells already contain all partitions; additional partitions increase entropy).

The minimum-entropy configuration has $N_Z$ as point-like: no internal structure to create additional negations. It contains $+Z$ units concentrated at $r = 0$.
\end{proof}

\subsection{The $Z$ Boundaries as Probability Distributions}

\begin{theorem}[Boundaries Are Not Sharp]
\label{thm:fuzzy_boundary}
Each boundary $\partial\Omega_i$ is not a sharp surface but a probability distribution. The transitions from ``inside'' to ``outside'' are gradual.
\end{theorem}

\begin{proof}
A sharp boundary would have infinite gradient---infinite partition entropy. Thermodynamically, sharp boundaries are unstable.

Each of the $Z$ boundaries spreads over a finite thickness. The probability of being ``inside partition $i$'' varies smoothly with radius.

These probability distributions $|\psi_i(r)|^2$ describe where each boundary ``is.'' They are not particle positions but the \emph{locations of the categorical distinctions themselves}.
\end{proof}

\begin{theorem}[Wave Functions Are Boundaries]
\label{thm:wave_function_boundary}
The quantum mechanical wave functions $\psi_i(r)$ are not probabilities of finding particles but probability distributions of the $Z$ categorical boundaries:
\begin{equation}
    |\psi_i(r)|^2 = P(\text{the $i$-th boundary passes through } r)
\end{equation}
\end{theorem}

\begin{proof}
Each boundary $\partial\Omega_i$ separates ``inside partition $i$'' from ``outside partition $i$.'' In the smoothed configuration (Theorem~\ref{thm:fuzzy_boundary}), this separation is probabilistic.

$|\psi_i(r)|^2$ gives the probability that radius $r$ is ``where the $i$-th distinction is being made.''

What physics calls ``electrons'' are not particles orbiting a nucleus; they are the $Z$ \emph{categorical boundaries} between the atomic interior and the exterior universe.
\end{proof}

\subsection{The General Solution: The $Z$-Structure}

\begin{theorem}[Uniqueness of $Z$-Partition Structure]
\label{thm:uniqueness}
Given:
\begin{enumerate}[(i)]
    \item $Z$ spherical partitions in infinite space
    \item Minimum entropy configuration
    \item Stability (no internal partitions beyond $Z$)
\end{enumerate}
The structure is uniquely determined:
\begin{itemize}
    \item A point-like central concentration of magnitude $+Z$
    \item $Z$ spherically symmetric boundaries each of magnitude $-1$
    \item A $Z/r$ binding potential
    \item Total: neutral ($+Z - Z = 0$)
\end{itemize}
\end{theorem}

\begin{proof}
From Theorem~\ref{thm:center_exists}: a central concentration of magnitude $+Z$ exists.
From Theorem~\ref{thm:point_nucleus}: it is point-like.
From Theorem~\ref{thm:wave_function_boundary}: the $Z$ boundaries are probability distributions.
From Theorem~\ref{thm:potential_negations}: the binding is $Z/r$.

Charge balance: Each of the $Z$ shells defines an ``inside'' vs. ``outside'' distinction. The interior has value $+Z$ (affirmed $Z$ times). Each boundary carries $-1$ to maintain categorical balance per partition.

This gives:
\begin{itemize}
    \item Interior (center): $+Z$ units
    \item Boundaries ($Z$ of them): $-1$ unit each
    \item Net: $+Z + Z \cdot (-1) = 0$ (neutral)
\end{itemize}
\end{proof}

\begin{definition}[The General Matter Equation]
\label{def:matter_equation}
A stable bounded structure in infinite space with $Z$ categorical partitions has:
\begin{align}
    \text{Central concentration:} \quad & N_Z = +Z \\
    \text{Boundary distributions:} \quad & B_i = -1 \quad (i = 1, \ldots, Z) \\
    \text{Binding potential:} \quad & \phi_Z(r) = -\frac{Z}{r} \\
    \text{Total charge:} \quad & Q_{\text{total}} = Z - Z = 0
\end{align}
This is the \emph{general equation of matter from partition}.
\end{definition}

\subsection{The Periodic Table as Partition Enumeration}

\begin{theorem}[Atomic Number Equals Partition Number]
\label{thm:periodic_table}
The atomic number $Z$ in physics is identical to the partition count in categorical theory:
\begin{equation}
    Z_{\text{atomic}} \equiv Z_{\text{partition}}
\end{equation}
The periodic table is the complete enumeration of stable $Z$-partition configurations.
\end{theorem}

\begin{proof}
By Definition~\ref{def:matter_equation}, a $Z$-partition configuration has:
\begin{itemize}
    \item Central concentration $+Z$ (what physics calls ``$Z$ protons'')
    \item $Z$ boundary distributions of $-1$ each (what physics calls ``$Z$ electrons'')
    \item Binding $Z/r$ (what physics calls ``Coulomb potential'')
\end{itemize}

This structure is parameterised entirely by the integer $Z$. The enumeration $Z = 1, 2, 3, \ldots$ generates all possible stable structures. This enumeration IS the periodic table.
\end{proof}

\begin{remark}[Connection to Known Elements]
The partition-theoretic structures correspond exactly to known elements:
\begin{center}
\begin{tabular}{c|c|l}
$Z$ & Structure & Known As \\
\hline
1 & 1 partition, $+1$ center, 1 boundary & Hydrogen \\
2 & 2 partitions, $+2$ center, 2 boundaries & Helium \\
6 & 6 partitions, $+6$ center, 6 boundaries & Carbon \\
26 & 26 partitions, $+26$ center, 26 boundaries & Iron \\
79 & 79 partitions, $+79$ center, 79 boundaries & Gold \\
\end{tabular}
\end{center}

We did not assume these elements. We derived them from:
\begin{enumerate}
    \item $Z$ partitions (boundaries) in infinite space
    \item Minimum entropy (spherical symmetry)
    \item Stability (no additional internal structure)
    \item The negation logic (potential from exclusion)
\end{enumerate}

Atoms are not ``made of'' protons and electrons. They are the \emph{necessary structures} that emerge when $Z$ categorical distinctions are made in an infinite continuum. The ``electrons'' are the distinctions themselves; the ``protons'' are what the distinctions are about.
\end{remark}

\subsection{The Exterior Is Not Empty: Why ``Not-Hydrogen'' Has Mass}

\begin{theorem}[Negation Is Positive Existence Elsewhere]
\label{thm:negation_positive}
The negation field $\mathcal{F}_{\neg}$ that defines hydrogen does not represent emptiness. Each ``not here'' is a positive statement: ``something else is here.''
\begin{equation}
    \neg(\text{hydrogen at } p) \iff \exists X \neq \text{H} : X \text{ is at } p
\end{equation}
\end{theorem}

\begin{proof}
When we partition space to create hydrogen, the exterior points are labelled ``not hydrogen.'' But ``not hydrogen'' is not ``nothing.'' It is:
\begin{itemize}
    \item Oxygen (at some locations)
    \item Carbon (at other locations)
    \item Helium, iron, stars, galaxies, ...
    \item Everything that is not this particular hydrogen atom
\end{itemize}

The negation field consists of positive existences---things that ARE, just not the thing under consideration. ``Not-X'' = ``something other than X,'' not ``absence of anything.''
\end{proof}

\begin{corollary}[All Negations Reference Positive Existence]
\label{cor:all_positive}
From the perspective of any atom:
\begin{align}
    \text{Not-hydrogen} &= \text{oxygen, carbon, helium, ...} \\
    \text{Not-oxygen} &= \text{hydrogen, carbon, helium, ...} \\
    \text{Not-carbon} &= \text{hydrogen, oxygen, helium, ...}
\end{align}
Every element's negation field is populated by all other elements. The ``exterior'' of any partition is the ``interior'' of other partitions.
\end{corollary}

\begin{theorem}[Why the Unobserved Has Mass]
\label{thm:unobserved_mass}
The mass of the unobserved (dark matter) is the mass of everything that exists but is not under categorical consideration:
\begin{equation}
    M_{\text{dark}} = \sum_{\text{all } X \text{ not observed}} M_X
\end{equation}
This is not ``missing mass'' but ``mass of things not categorically partitioned.''
\end{theorem}

\begin{proof}
An observer partitioning reality creates a categorical structure: ``this hydrogen atom,'' ``that oxygen molecule,'' etc. Everything partitioned is observable (ordinary matter).

But the observer's partitions do not exhaust reality. There exist:
\begin{enumerate}
    \item Things too distant to partition (beyond observational horizon)
    \item Things too diffuse to partition (no sharp boundary)
    \item Things that don't participate in partition-creating interactions (no electromagnetic coupling)
\end{enumerate}

These unpartitioned things are NOT nothing. They are positive existences---just not categorically distinguished by the observer. They have mass because they ARE things. They are ``dark'' because they cannot be partitioned, not because they don't exist.
\end{proof}

\begin{remark}[The Universe Is Full, Not Empty]
The traditional picture: space is mostly empty, with occasional matter.

The partition picture: space is entirely full. Every location is ``something.'' What we call ``empty space'' is simply ``not the thing we're considering''---but it IS other things. The vacuum is not nothing; it is the accumulated ``not-this'' of all the things we've partitioned, which means it is the accumulated ``is-something-else.''

Dark matter is not mysterious missing mass. It is the obvious consequence of the fact that negation does not create emptiness---it acknowledges existence elsewhere.
\end{remark}

\begin{corollary}[Conservation from Partition Logic]
\label{cor:conservation}
Total mass-energy is conserved because negation redistributes but does not annihilate:
\begin{equation}
    M_{\text{total}} = M_{\text{partitioned}} + M_{\text{not partitioned}} = \text{constant}
\end{equation}
When we partition something ``here,'' we simultaneously acknowledge everything ``not here.'' The sum is invariant.
\end{corollary}

\subsection{Summary: The Periodic Table from Partition Logic}

\begin{enumerate}
    \item \textbf{$Z$ Partitions}: $Z$ boundaries are created in infinite space
    \item \textbf{Negation field}: The exterior generates $Z \cdot \infty$ ``nots''
    \item \textbf{Potential}: The nots create a $Z/r$ central attractive potential
    \item \textbf{Center}: The most-affirmed point (least negated) forms at $r=0$ with magnitude $+Z$
    \item \textbf{Boundaries}: The $Z$ boundaries themselves, spread as probability distributions with magnitude $-1$ each
    \item \textbf{Result}: The element with atomic number $Z$---not built from particles, but \emph{forced into existence} by the logic of $Z$ partitions
    \item \textbf{The exterior}: ``Not this atom'' is not empty---it is all other atoms, all other structures
    \item \textbf{Dark matter}: The mass of things not under categorical consideration---real, massive, but unpartitioned
\end{enumerate}

\begin{remark}[Connection to Quantum Mechanics]
This derivation explains why quantum mechanics works. The wave functions $\psi_i$ are not mysterious probability amplitudes; they are the \emph{locations of the $Z$ categorical boundaries}. The uncertainty principle follows: boundaries cannot be both sharp (definite position) and stable (definite momentum). The Schrödinger equation is the dynamics of partition boundaries.

Atomic structure is not a consequence of quantum mechanics. Quantum mechanics is a consequence of partition structure. The periodic table is not an empirical discovery; it is the necessary enumeration of partition configurations.
\end{remark}

\begin{figure*}[htbp]
\centering
\includegraphics[width=0.90\textwidth]{figures/hydrogen_derivation_panel.png}
\caption{\textbf{Derivation of Atomic Structure from Partition Logic.} \textbf{(A)} The $Z$-partition configuration: $Z$ spherical shells divide infinite space into nested regions. \textbf{(B)} The negation field: every exterior point generates $Z$ ``nots''---one per partition. \textbf{(C)} The potential from negations: accumulated exclusions create a $Z/r$ central attractive potential. \textbf{(D)} The center emerges: magnitude $+Z$, forced into existence as the common referent of all negations. \textbf{(E)} The boundaries as probability distributions: the $Z$ shells are not sharp but spread---the quantum wave functions. \textbf{(F)} The result: element with atomic number $Z$, not built from particles but derived from $Z$ categorical distinctions. The case $Z=1$ is shown (hydrogen); $Z=2$ gives helium, $Z=6$ gives carbon, etc.}
\label{fig:atomic_derivation}
\end{figure*}



\section{Conclusion}
\label{sec:conclusion}

We have established three independent derivations of entropy—from oscillatory mechanics, categorical structure, and partition theory—and proved their mathematical equivalence. The unified entropy formula $S = \kB M \ln n$ emerges identically from all three perspectives, demonstrating that oscillation, category, and partition are not analogous but identical.

The partition lag mechanism reveals why composition cannot reverse partition: each partition operation generates undetermined residue that increases entropy by $\Delta S > 0$. This irreversibility is not a limitation of particular physical systems but a consequence of the fundamental structure of categorical operations.

The physical applications demonstrate that systems traditionally analysed through composition—asking how parts combine to form wholes—are more naturally understood through partition—asking how wholes decompose into parts with entropy loss. The thermodynamic framework provides quantitative predictions for the entropy cost of partition and explains why certain properties of wholes cannot be recovered from parts.

Two results concerning light and dark matter emerge from the partition framework. First, partition-free traversal—motion without creating categorical distinctions—generates zero boundary entropy and therefore zero proper time, providing a thermodynamic derivation of null geodesics and explaining why the speed of light is maximum. Second, each actualisation resolves infinitely many non-actualisations into ``did not happen,'' and these non-actualisations cannot be partitioned (they lack categorical structure), hence cannot interact with partition-free entities (light), hence are ``dark'' while still contributing gravitationally.

The geometric structure of non-actualisation space provides the quantitative foundation for the dark matter ratio. Non-actualisations are organised in shells of increasing categorical distance from actualisations. Close non-actualisations pair with nearby actualisations to form mutual reference structures—this pairing IS the relational structure of ordinary matter. Distant non-actualisations lack pairing partners and remain as unstructured, non-partitionable mass. The $\sim 5:1$ ratio of unpaired to paired non-actualisations emerges from the exponential growth of shells with distance, determined by the branching factor of categorical space.

Finally, the logical priority of actualisation resolves the question of why there is something rather than nothing. Every negation presupposes what it negates: ``not-$X$'' requires $X$ to exist as its referent. Non-actualisations depend ontologically on actualisations, not vice versa. Pure nothing—the absence of all actualisation—is self-contradictory, because ``nothing exists'' is itself a determination that presupposes the meaningfulness of ``something.'' Existence is logically necessary; non-existence is derivative and dependent.

The partition framework thus unifies thermodynamics, relativity, cosmology, and ontology under a single principle: the categorical structure of distinction-making determines what can exist (partitionable), what can move at maximum speed (partition-free), what can interact (at least one participant partitions), what remains invisible yet gravitating (non-partitionable), and why existence precedes non-existence (negation presupposes affirmation).

\bibliographystyle{plainnat}
\bibliography{references}

\end{document}

