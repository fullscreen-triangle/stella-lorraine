\section{Continuous-to-Discrete Temporal Decomposition}
\label{sec:temporal}

We analyse the thermodynamics of partitioning continuous processes into discrete elements. The key result is that infinite partition of continuous motion generates infinite entropy, rendering the ``instantaneous state'' an artifact of partition rather than a physical reality.

\subsection{Continuous Motion and Temporal Partition}

\begin{definition}[Continuous Motion]
\label{def:continuous_motion}
A \emph{continuous motion} is a trajectory $\mathbf{x}(t)$ that varies smoothly over a time interval $[t_0, t_f]$:
\begin{equation}
    \mathbf{x}: [t_0, t_f] \to \mathbb{R}^d, \quad \mathbf{x} \in C^1([t_0, t_f])
\end{equation}
The velocity is well-defined at each point: $\mathbf{v}(t) = d\mathbf{x}/dt$.
\end{definition}

\begin{definition}[Temporal Partition]
\label{def:temporal_partition}
A \emph{temporal partition} of interval $[t_0, t_f]$ into $N$ subintervals is:
\begin{equation}
    \{[t_0, t_1], [t_1, t_2], \ldots, [t_{N-1}, t_N]\}
\end{equation}
where $t_i = t_0 + i \cdot \Delta t$ and $\Delta t = (t_f - t_0)/N$.
\end{definition}

\begin{definition}[Instantaneous State]
\label{def:instant}
An \emph{instantaneous state} at time $t_i$ is the configuration $(\mathbf{x}(t_i), \mathbf{v}(t_i))$ obtained by evaluating the trajectory at a single instant.
\end{definition}

\subsection{Entropy of Temporal Partition}

\begin{theorem}[Temporal Partition Entropy]
\label{thm:temporal_entropy}
Partitioning continuous motion into $N$ temporal segments generates entropy:
\begin{equation}
    \Delta S_{\text{temporal}} = \kB (N-1) H_{\text{boundary}}
\end{equation}
where $H_{\text{boundary}}$ is the entropy of each temporal boundary.
\end{theorem}

\begin{proof}
A temporal partition into $N$ segments creates $N-1$ internal boundaries. Each boundary separates ``before $t_i$'' from ``after $t_i$.''

At each boundary, the trajectory is evaluated to determine the instantaneous state. This evaluation has finite precision—the position $\mathbf{x}(t_i)$ and velocity $\mathbf{v}(t_i)$ can only be determined to within measurement uncertainty. The Shannon entropy of this uncertainty is $H_{\text{boundary}}$.

With $N-1$ independent boundaries:
\begin{equation}
    \Delta S_{\text{temporal}} = \kB (N-1) H_{\text{boundary}}
\end{equation}
\end{proof}

\begin{corollary}[Infinite Partition Generates Infinite Entropy]
\label{cor:infinite_entropy}
As the number of temporal partitions approaches infinity:
\begin{equation}
    \lim_{N \to \infty} \Delta S_{\text{temporal}} = \lim_{N \to \infty} \kB (N-1) H_{\text{boundary}} = \infty
\end{equation}
Infinitely fine temporal partition destroys all information about the original motion.
\end{corollary}

\subsection{Motion as Undetermined Residue}

\begin{theorem}[Motion Resides in Residue]
\label{thm:motion_residue}
When continuous motion is partitioned into instantaneous states, the \emph{motion itself} (the continuous change) becomes undetermined residue.
\end{theorem}

\begin{proof}
Consider a trajectory $\mathbf{x}(t)$ on $[t_0, t_f]$ with velocity $\mathbf{v}(t) \neq 0$. Partition the interval into $N$ instants $\{t_0, t_1, \ldots, t_N\}$.

At each instant $t_i$, record the instantaneous state:
\begin{equation}
    \mathbf{s}_i = (\mathbf{x}(t_i), \mathbf{v}(t_i))
\end{equation}

The collection $\{\mathbf{s}_0, \mathbf{s}_1, \ldots, \mathbf{s}_N\}$ describes positions and velocities at discrete instants. But this collection does not contain the \emph{motion}—the continuous process of changing position.

Motion exists \emph{between} instants: it is the process $\mathbf{x}(t_{i}) \to \mathbf{x}(t_{i+1})$ that occurs in the interval $(t_i, t_{i+1})$. This process is not captured by the instantaneous states $\mathbf{s}_i$ and $\mathbf{s}_{i+1}$—it resides in the undetermined residue of the temporal partition.

As $N \to \infty$ and $\Delta t \to 0$, each interval $(t_i, t_{i+1})$ shrinks, but the number of intervals grows. The motion becomes distributed across infinitely many infinitesimal residues—it becomes entirely undetermined.
\end{proof}

\subsection{The Stillness of Instantaneous States}

\begin{theorem}[Instantaneous States Are Still]
\label{thm:instantaneous_still}
At any instant $t_i$, the system occupies exactly its position $\mathbf{x}(t_i)$. There is no motion \emph{at} an instant—motion requires duration.
\end{theorem}

\begin{proof}
Motion is defined as change of position over time: $\mathbf{v} = d\mathbf{x}/dt$. This derivative is a limit:
\begin{equation}
    \mathbf{v}(t) = \lim_{\Delta t \to 0} \frac{\mathbf{x}(t + \Delta t) - \mathbf{x}(t)}{\Delta t}
\end{equation}

At a single instant (with $\Delta t = 0$), the quotient is undefined—there is no duration over which to measure change. The velocity $\mathbf{v}(t)$ exists as a limit, not as an instantaneous property.

At instant $t_i$, the system is at position $\mathbf{x}(t_i)$. It is not ``moving'' at that instant—it simply \emph{is} at that position. Motion exists only in the transition between positions, which requires positive duration.
\end{proof}

\begin{corollary}[Stillnesses Cannot Compose to Motion]
\label{cor:stillnesses}
If each instantaneous state is ``still'' (not moving), then composing instantaneous states cannot produce motion:
\begin{equation}
    \text{Compose}(\{\text{still}_0, \text{still}_1, \ldots, \text{still}_N\}) \not\supset \text{motion}
\end{equation}
Motion cannot be recovered from its temporal partition.
\end{corollary}

\subsection{Thermodynamic Analysis of Temporal Decomposition}

\begin{theorem}[Entropy of Motion Loss]
\label{thm:motion_entropy}
The entropy cost of temporal partition of continuous motion is:
\begin{equation}
    \Delta S_{\text{motion}} = \kB \ln\left(\frac{W_{\text{continuous}}}{W_{\text{discrete}}}\right)
\end{equation}
where $W_{\text{continuous}}$ is the number of continuous trajectories and $W_{\text{discrete}}$ is the number of discrete state sequences.
\end{theorem}

\begin{proof}
A continuous trajectory $\mathbf{x}(t)$ on $[t_0, t_f]$ is specified by initial conditions plus the requirement of continuity. The space of continuous trajectories has cardinality $W_{\text{continuous}}$.

A sequence of $N$ discrete states $\{\mathbf{s}_0, \ldots, \mathbf{s}_N\}$ is specified by $N+1$ independent position-velocity pairs. The space of such sequences has cardinality $W_{\text{discrete}} = (\text{phase space volume})^{N+1}$.

In general, $W_{\text{discrete}} > W_{\text{continuous}}$ because discrete sequences need not be consistent with any continuous trajectory—the states can ``jump'' between arbitrary positions.

The entropy increase is:
\begin{equation}
    \Delta S = \kB \ln W_{\text{discrete}} - \kB \ln W_{\text{continuous}} = \kB \ln\left(\frac{W_{\text{discrete}}}{W_{\text{continuous}}}\right) > 0
\end{equation}

This entropy is the cost of destroying the continuity constraint—the ``motion'' that connects successive positions.
\end{proof}

\subsection{The Dichotomy Problem}

Consider an object that must traverse distance $L$ in finite time. The traditional analysis proceeds:
\begin{enumerate}
    \item To travel $L$, first travel $L/2$
    \item To travel $L/2$, first travel $L/4$
    \item Continue indefinitely: must complete infinitely many sub-journeys
    \item Conclusion: motion is impossible
\end{enumerate}

\begin{theorem}[Thermodynamic Resolution of Dichotomy]
\label{thm:dichotomy}
The dichotomy analysis generates infinite entropy by infinite partition. The ``impossibility'' is not a feature of motion but an artifact of the partition process.
\end{theorem}

\begin{proof}
Each subdivision of the journey is a partition operation. Subdividing into $N$ sub-journeys generates entropy:
\begin{equation}
    \Delta S_N = \kB (N-1) H_{\text{boundary}}
\end{equation}

As $N \to \infty$:
\begin{equation}
    \Delta S_\infty = \lim_{N \to \infty} \kB (N-1) H_{\text{boundary}} = \infty
\end{equation}

Infinite partition generates infinite entropy, completely destroying the information content of the original motion. The ``infinitely many sub-journeys'' do not exist in the physical motion—they are created by the partition process.

The physical motion completes in finite time because it was never partitioned. The ``impossibility'' arises only when we attempt to decompose continuous motion into infinitely many discrete segments.
\end{proof}

\subsection{The Arrow Paradox}

Consider an arrow in flight. At any instant, the arrow occupies exactly one position. If it is at a position, it is not moving. If it is not moving at any instant, when does it move?

\begin{theorem}[Thermodynamic Resolution of Arrow Paradox]
\label{thm:arrow}
The arrow paradox arises from confusing the ontological status of instantaneous states. Motion is not composed of instantaneous stillnesses—rather, stillness is derived from motion by temporal partition.
\end{theorem}

\begin{proof}
The arrow's motion $\mathbf{x}(t)$ exists as a continuous process on the interval $[t_0, t_f]$. This is the primary reality.

Temporal partition extracts instantaneous states $\{\mathbf{x}(t_i)\}$. At each such state, the arrow ``occupies exactly its length''—it is in a definite position. This is not ``motion'' but a derived snapshot.

The paradox asks: ``How do stillnesses compose to motion?'' The thermodynamic answer: they don't. Motion is not composed of stillnesses. Motion is primary; stillnesses are derived by partition.

The motion itself is lost to undetermined residue during temporal partition. It exists \emph{between} the snapshots, in the transition from $\mathbf{x}(t_i)$ to $\mathbf{x}(t_{i+1})$. This transition is not captured by the snapshots—it is the undetermined residue of the partition.

The arrow moves because motion exists first. Asking ``when does it move?'' presupposes that motion is composed of instants, which reverses the true ontological order.
\end{proof}

\begin{figure*}[htbp]
\centering
\includegraphics[width=0.95\textwidth]{figures/zeno_paradox_panel.png}
\caption{\textbf{Infinite Subdivision of Bounded Continuous Intervals.} \textbf{(A)} Spatial interval dichotomy: recursive halving creates exponentially many sub-intervals, each partition adding entropy. \textbf{(B)} Hardware-measured entropy divergence: as partition depth $M \to \infty$, cumulative entropy $S \to \infty$—infinite partition generates infinite entropy, completely destroying the continuous motion information. \textbf{(C)} Continuous property dissipation: motion is a property of the continuous whole; partition extracts discrete ``positions'' that are individually static; motion itself becomes undetermined residue. \textbf{(D)} Arrow paradox: continuous trajectory (solid line) vs. discrete instantaneous samples (dots); at each instant, no duration exists over which to define velocity. \textbf{(E)} Thermodynamic resolution: infinite partition generates infinite entropy; motion is primary, stillness is derived; composing stillnesses cannot recover motion. \textbf{(F)} Connection to Zeno's paradoxes: both Dichotomy and Arrow dissolve when we recognize that continuous motion exists first, and partition (not composition) creates discrete states.}
\label{fig:zeno_paradox}
\end{figure*}

\subsection{Resolution of Traditional Puzzles}

\begin{remark}[Historical Note]
The analysis above provides the thermodynamic structure underlying Zeno's paradoxes of motion—the Dichotomy and the Arrow. These paradoxes dissolve when the ontological direction is corrected:
\begin{itemize}
    \item Motion is not composed of stillnesses (infinite still instants cannot compose to motion)
    \item Stillness is derived from motion by temporal partition
    \item Motion itself becomes undetermined residue during partition
    \item Infinite partition generates infinite entropy, destroying all motion information
\end{itemize}
The ``paradoxes'' are not paradoxes of motion but paradoxes of partition—demonstrations that infinite subdivision destroys continuous structure.
\end{remark}

