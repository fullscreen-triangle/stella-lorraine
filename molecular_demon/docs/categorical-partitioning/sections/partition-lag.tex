\section{Partition Lag and Irreversible Entropy Production}
\label{sec:partition_lag}

Having established that oscillation, category, and partition yield identical entropy formulations, we now examine the temporal structure of partition operations. The key result is that partition operations are not instantaneous but require finite time, creating an irreducible \emph{partition lag} between the act of partitioning and the partitioned result. This lag generates entropy through \emph{undetermined residue}—information that is lost to the partition boundary and cannot be recovered.

\subsection{The Partition Process}

\begin{definition}[Partition Time]
\label{def:partition_time}
The \emph{partition time} $\tau_p$ is the minimum duration required to establish a single categorical distinction. This time encompasses:
\begin{enumerate}[(i)]
    \item Recognition that a difference exists between elements
    \item Assignment of elements to distinct categories
    \item Registration of the assignment in the observer's state
\end{enumerate}
The partition time is bounded below by fundamental physical constraints: $\tau_p \geq \tau_{\min} > 0$.
\end{definition}

\begin{axiom}[Non-Zero Partition Time]
\label{axiom:nonzero}
Every partition operation requires positive time:
\begin{equation}
    \tau_p > 0
\end{equation}
Instantaneous partition ($\tau_p = 0$) is physically impossible.
\end{axiom}

This axiom is grounded in physics: any process that distinguishes states requires energy transfer, measurement, or comparison—all of which take finite time. Even at the quantum level, the time-energy uncertainty relation $\Delta E \cdot \Delta t \geq \hbar/2$ implies that distinguishing states with finite energy difference requires finite time.

\subsection{The Partition Lag Theorem}

\begin{theorem}[Partition Lag]
\label{thm:partition_lag}
For an observer partitioning a continuously evolving system into $k$ categorical distinctions, there exists an irreducible temporal lag $\Delta t$ between the state that was partitioned and the state that exists at partition completion:
\begin{equation}
    \Delta t = k \cdot \tau_p
\end{equation}
The system evolves by $\Delta \mathcal{R} = \mathcal{R}(t_0 + k\tau_p) - \mathcal{R}(t_0)$ during the partition process.
\end{theorem}

\begin{proof}
Let the partition process begin at time $t_0$ when the system is in state $\mathcal{R}(t_0)$. The first distinction $C_1$ is established at time $t_0 + \tau_p$. The second distinction $C_2$ is established at time $t_0 + 2\tau_p$. Continuing sequentially, the $k$-th distinction $C_k$ is established at time $t_0 + k\tau_p$.

At the moment of completion, the system is in state $\mathcal{R}(t_0 + k\tau_p)$, which differs from the initial state $\mathcal{R}(t_0)$ by:
\begin{equation}
    \Delta \mathcal{R} = \mathcal{R}(t_0 + k\tau_p) - \mathcal{R}(t_0)
\end{equation}

The partition structure $\{C_1, \ldots, C_k\}$ was constructed from states spanning the interval $[t_0, t_0 + k\tau_p]$, but only the state $\mathcal{R}(t_0 + k\tau_p)$ exists at completion. The lag $\Delta t = k\tau_p$ is irreducible given $\tau_p > 0$.
\end{proof}

\subsection{Undetermined Residue}

\begin{definition}[Undetermined Residue]
\label{def:residue}
The \emph{undetermined residue} $\mathcal{U}$ is the portion of the system that was within the partition scope at initiation but escaped before partition completion:
\begin{equation}
    \mathcal{U} = \{x : x \in \text{scope at } t_0, \, x \notin \text{scope at } t_0 + k\tau_p\}
\end{equation}
Elements in $\mathcal{U}$ were never successfully partitioned despite being initially accessible.
\end{definition}

\begin{theorem}[Residue is Undetermined]
\label{thm:residue_undetermined}
Elements in the undetermined residue $\mathcal{U}$ have indeterminate categorical status:
\begin{enumerate}[(i)]
    \item They are \textbf{not absent}: they existed at $t_0$ and influenced initial conditions
    \item They are \textbf{not present}: they have exited the partition scope by completion
    \item They are \textbf{not determined}: they were never assigned to any category $C_i$
\end{enumerate}
\end{theorem}

\begin{proof}
Consider an element $u \in \mathcal{U}$.

(i) At time $t_0$, the element $u$ was within the partition scope. It existed, was accessible, and contributed to the initial state from which partitioning began. Therefore $u$ was not absent.

(ii) At time $t_0 + k\tau_p$, the element $u$ has exited the partition scope. It is no longer accessible and does not appear in any category $C_i$. Therefore $u$ is not present in the completed partition.

(iii) The element $u$ was never successfully assigned to a category. The sequential partition process did not reach $u$ before it exited. Therefore $u$ remains undetermined—neither included nor excluded from any particular category.
\end{proof}

\subsection{Entropy Production from Partition Lag}

\begin{theorem}[Partition Entropy Production]
\label{thm:entropy_production}
Each partition operation produces entropy:
\begin{equation}
    \Delta S_{\text{partition}} = \kB \ln\left(\frac{W_{\text{before}}}{W_{\text{after}}}\right) + S_{\text{residue}}
\end{equation}
where $W_{\text{before}}$ is the number of configurations before partition, $W_{\text{after}}$ is the number after, and $S_{\text{residue}}$ is the entropy of the undetermined residue.
\end{theorem}

\begin{proof}
Before partition, the system has $W_{\text{before}}$ distinguishable configurations. After partition into $n$ categories, each category has $W_{\text{after}} = W_{\text{before}}/n$ configurations (assuming equipartition).

The information gained by knowing which category the system occupies is $\kB \ln n$. However, the undetermined residue $\mathcal{U}$ contains configurations that escaped partition. This residue has its own entropy:
\begin{equation}
    S_{\text{residue}} = \kB \ln |\mathcal{U}|
\end{equation}

The total entropy change is:
\begin{equation}
    \Delta S_{\text{partition}} = \kB \ln n + S_{\text{residue}} > 0
\end{equation}

Since both terms are positive (assuming $n \geq 2$ and $|\mathcal{U}| \geq 1$), partition always increases entropy.
\end{proof}

\begin{corollary}[Minimum Entropy Production]
\label{cor:min_entropy}
Even for ideal partition with minimal residue, the entropy increase is at least:
\begin{equation}
    \Delta S_{\text{partition}} \geq \kB \ln 2
\end{equation}
corresponding to a single binary distinction.
\end{corollary}

\subsection{Irreversibility: Composition Cannot Reverse Partition}

\begin{definition}[Composition Operation]
\label{def:composition}
\emph{Composition} is the inverse operation to partition: combining parts $\{X_1, \ldots, X_n\}$ to form a whole $X = \bigcup_i X_i$.
\end{definition}

\begin{theorem}[Irreversibility of Partition]
\label{thm:irreversibility}
Composition cannot reverse partition. Specifically:
\begin{equation}
    \text{Compose}(\text{Partition}(X)) \neq X
\end{equation}
The composed result differs from the original by the undetermined residue.
\end{theorem}

\begin{proof}
Let $X$ be the original system with entropy $S_X = \kB \ln W_X$. Apply partition to obtain $\{X_1, \ldots, X_n\}$ with combined entropy:
\begin{equation}
    S_{\text{parts}} = \sum_{i=1}^{n} S_{X_i} = S_X - S_{\text{residue}}
\end{equation}

The undetermined residue has escaped: it is not contained in any part $X_i$. Now compose the parts:
\begin{equation}
    X' = \text{Compose}(\{X_1, \ldots, X_n\}) = \bigcup_{i=1}^{n} X_i
\end{equation}

The composed system $X'$ has entropy:
\begin{equation}
    S_{X'} = S_{\text{parts}} = S_X - S_{\text{residue}} < S_X
\end{equation}

But by the Second Law of Thermodynamics, entropy cannot decrease in an isolated system. The resolution is that $X' \neq X$: the composed system is missing the undetermined residue.

The residue entropy $S_{\text{residue}}$ has been dissipated—converted to heat, lost to the environment, or rendered inaccessible. It cannot be recovered by composition.
\end{proof}

\begin{theorem}[Second Law for Partition-Composition]
\label{thm:second_law}
For any cycle of partition followed by composition:
\begin{equation}
    \Delta S_{\text{cycle}} = S_{\text{residue}} > 0
\end{equation}
Partition-composition cycles always increase total entropy.
\end{theorem}

\begin{proof}
Starting with system $X$:
\begin{enumerate}
    \item Partition: $X \to \{X_1, \ldots, X_n\}$ with residue $\mathcal{U}$. System entropy decreases by $S_{\text{residue}}$, but this entropy is transferred to the environment.
    \item Composition: $\{X_1, \ldots, X_n\} \to X'$. The parts are combined, but the residue is not recovered.
\end{enumerate}

The total entropy of (system + environment) increases by:
\begin{equation}
    \Delta S_{\text{total}} = S_{\text{residue}} > 0
\end{equation}

This is the Second Law: partition-composition is thermodynamically irreversible.
\end{proof}

\subsection{Quantitative Entropy of Partition Boundaries}

\begin{theorem}[Boundary Entropy]
\label{thm:boundary_entropy}
For a partition of a system into $n$ parts, the entropy localised at partition boundaries is:
\begin{equation}
    S_{\text{boundary}} = \kB (n-1) H_{\text{edge}}
\end{equation}
where $H_{\text{edge}}$ is the Shannon entropy of the edge indeterminacy distribution.
\end{theorem}

\begin{proof}
A partition into $n$ parts creates $n-1$ internal boundaries (by the formula for partitions of an interval). Each boundary has indeterminate extent due to edge indeterminacy (partition lag at the boundary).

Let $p(x)$ be the probability distribution over possible boundary locations. The Shannon entropy of each boundary is:
\begin{equation}
    H_{\text{edge}} = -\int p(x) \ln p(x) \, dx
\end{equation}

With $n-1$ independent boundaries, the total boundary entropy is:
\begin{equation}
    S_{\text{boundary}} = \kB (n-1) H_{\text{edge}}
\end{equation}
\end{proof}

\begin{corollary}[Fine Partition Has High Boundary Entropy]
\label{cor:fine_partition}
Partitioning a system into many small parts generates large boundary entropy:
\begin{equation}
    \lim_{n \to \infty} S_{\text{boundary}} = \lim_{n \to \infty} \kB (n-1) H_{\text{edge}} = \infty
\end{equation}
Infinitely fine partition produces infinite entropy.
\end{corollary}

This result has profound implications: infinitely subdividing a system destroys all its original structure, converting ordered information into boundary entropy.

\subsection{The Asymmetry Between Partition and Composition}

\begin{theorem}[Directional Asymmetry]
\label{thm:asymmetry}
Partition and composition are not symmetric inverses:
\begin{align}
    \text{Partition (downward):} \quad & W \to \{p_1, \ldots, p_n\} + \mathcal{U}, \quad \Delta S > 0 \\
    \text{Composition (upward):} \quad & \{p_1, \ldots, p_n\} \to W', \quad W' \neq W
\end{align}
The asymmetry arises because partition creates undetermined residue that composition cannot recover.
\end{theorem}

\begin{proof}
\textbf{Downward (partition)}: Starting from whole $W$ with property $P$, partition divides $W$ into parts $\{p_1, \ldots, p_n\}$. The property $P$ may be lost to undetermined residue—it becomes part of $\mathcal{U}$, not distributed among the parts. Entropy increases by $\Delta S = S_{\text{residue}} > 0$.

\textbf{Upward (composition)}: Starting from parts $\{p_1, \ldots, p_n\}$ that lack property $P$, composition produces $W'$. But $W'$ cannot possess $P$ because:
\begin{enumerate}
    \item $P$ is not contained in any part $p_i$
    \item The residue $\mathcal{U}$ (which might contain $P$) is inaccessible
    \item Creating $P$ would require decreasing entropy, violating the Second Law
\end{enumerate}

Therefore $W' \neq W$, and specifically $W'$ lacks any property that was lost to residue during the original partition.
\end{proof}

\begin{remark}[Physical Meaning]
The asymmetry explains why:
\begin{itemize}
    \item You can break an egg but not unbreak it
    \item You can burn a log but not unburn it
    \item You can forget information but not unforgot it
\end{itemize}
In each case, partition (breaking, burning, forgetting) creates undetermined residue (structural information, chemical order, neural patterns) that composition cannot recover.
\end{remark}

\begin{figure*}[htbp]
\centering
\includegraphics[width=0.95\textwidth]{figures/partition_lag_panel.png}
\caption{\textbf{Partition Lag and Irreversible Entropy Production.} \textbf{(A)} Measured partition lag from hardware timing: each categorical distinction takes finite time $\tau_p > 0$ (mean $\approx$ hundreds of nanoseconds), demonstrating that partition is not instantaneous. \textbf{(B)} Cumulative entropy from sequential partitioning: measured values (circles) match theoretical prediction $S = k_B M \ln(n)$ (dashed line), confirming the unified entropy formula. \textbf{(C)} Undetermined residue fraction increases with partition branching: more branches create more information loss during lag. \textbf{(D)} Irreversibility demonstration: each partition-composition cycle generates positive entropy ($\Delta S > 0$), with cumulative entropy monotonically increasing—Second Law verified. \textbf{(E)} Schematic of partition lag mechanism: reality evolves from $\mathcal{R}(t_0)$ to $\mathcal{R}(t_0 + \tau_p)$ during partition time, creating undetermined residue $\mathcal{U}$ that is lost to entropy. \textbf{(F)} Partition Lag Theorem: total lag $\Delta t = M \cdot \tau_p$, entropy increase $\Delta S_{\text{lag}} = k_B \ln \Omega_{\mathcal{U}} > 0$, and composition cannot reverse partition due to irreversibility.}
\label{fig:partition_lag}
\end{figure*}

\subsection{Summary: Partition as Entropy Generator}

The partition lag mechanism establishes:
\begin{enumerate}
    \item Every partition operation takes positive time ($\tau_p > 0$)
    \item During partition, systems evolve, creating lag ($\Delta t = k\tau_p$)
    \item Lag generates undetermined residue ($\mathcal{U}$)
    \item Residue has positive entropy ($S_{\text{residue}} > 0$)
    \item Composition cannot recover residue (Second Law)
    \item Partition-composition cycles are irreversible ($\Delta S_{\text{cycle}} > 0$)
\end{enumerate}

This provides a thermodynamic foundation for understanding why certain operations—particularly those involving the relationship between parts and wholes—are inherently one-directional.

