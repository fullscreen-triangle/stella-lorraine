\section{Entropy Unification: The Equivalence Theorem}
\label{sec:unification}

We have derived entropy from three independent starting points:
\begin{align}
    \text{Oscillatory mechanics:} \quad & \Sosc = \kB M \ln n \\
    \text{Categorical mechanics:} \quad & \Scat = \kB M \ln n \\
    \text{Partition mechanics:} \quad & \Spart = \kB M \ln n
\end{align}

The mathematical identity of these three formulas is not coincidental. In this section, we prove that oscillation, category, and partition are not merely analogous but fundamentally equivalent descriptions of the same underlying structure.

\subsection{The Unified Entropy Formula}

\begin{theorem}[Entropy Equivalence]
\label{thm:equivalence}
The three entropy formulas are mathematically identical:
\begin{equation}
    \boxed{\Sosc = \Scat = \Spart = S = \kB M \ln n}
\end{equation}
where $M$ and $n$ have consistent interpretations across all three frameworks.
\end{theorem}

\begin{proof}
We establish the equivalence by showing that the parameters $(M, n)$ have identical meanings in all three derivations.

\textbf{Step 1: Identification of $n$.}
\begin{itemize}
    \item In oscillatory mechanics, $n$ is the number of distinguishable states per oscillatory mode (Definition~\ref{def:quantum_states}).
    \item In categorical mechanics, $n$ is the number of distinguishable levels per categorical dimension (Axiom~\ref{axiom:resolution}).
    \item In partition mechanics, $n$ is the branching factor per partition operation (Axiom~\ref{axiom:branching}).
\end{itemize}

These are the same quantity: the number of distinguishable outcomes per degree of freedom, whether that degree of freedom is called a "mode," a "dimension," or a "partition level."

\textbf{Step 2: Identification of $M$.}
\begin{itemize}
    \item In oscillatory mechanics, $M$ is the number of independent oscillatory modes (Definition~\ref{def:mode}).
    \item In categorical mechanics, $M$ is the number of orthogonal categorical dimensions (Axiom~\ref{axiom:dimensional}).
    \item In partition mechanics, $M$ is the depth of recursive partitioning (Definition~\ref{def:partition_tree}).
\end{itemize}

These are the same quantity: the number of independent degrees of freedom over which distinctions can be made.

\textbf{Step 3: Identification of the counting argument.}

In all three cases, the total number of distinguishable configurations is:
\begin{equation}
    W = n^M
\end{equation}
\begin{itemize}
    \item Oscillatory: $W_{\text{osc}} = n^M$ configurations of mode quantum numbers
    \item Categorical: $|\mathcal{C}| = n^M$ categorical states
    \item Partition: $P = n^M$ leaf nodes in the partition tree
\end{itemize}

The Boltzmann relation $S = \kB \ln W$ then yields:
\begin{equation}
    S = \kB \ln(n^M) = \kB M \ln n
\end{equation}
identically in all three cases.
\end{proof}

\subsection{Physical Meaning of the Equivalence}

\begin{theorem}[Oscillation-Category Equivalence]
\label{thm:osc_cat}
An oscillatory mode IS a categorical dimension. The number of quantum states accessible to a mode equals the number of distinguishable levels in the corresponding categorical dimension.
\end{theorem}

\begin{proof}
Consider an oscillatory mode with frequency $\omega$ at temperature $T$. The accessible quantum states are $|n\rangle$ for $n \in \{0, 1, \ldots, n_{\max}\}$ where $n_{\max} \approx \kB T / \hbar \omega$.

Each quantum state $|n\rangle$ is distinguishable from every other state $|m\rangle$ for $m \neq n$—they have different energies, different wavefunctions, and different expectation values for position and momentum. This distinguishability is precisely what defines a categorical distinction (Axiom~\ref{axiom:distinguishable}).

The set of quantum states $\{|0\rangle, |1\rangle, \ldots, |n_{\max}\rangle\}$ is therefore isomorphic to a categorical dimension with $n_{\max} + 1$ levels. The oscillatory mode and the categorical dimension are not analogous structures—they are the same structure described in different languages.
\end{proof}

\begin{theorem}[Category-Partition Equivalence]
\label{thm:cat_part}
A categorical dimension IS a partition level. The number of distinguishable levels in a categorical dimension equals the branching factor of the corresponding partition operation.
\end{theorem}

\begin{proof}
Consider a categorical dimension $\mathcal{C}_i$ with $n$ distinguishable levels $\{c_1, c_2, \ldots, c_n\}$. Moving through this dimension—transitioning from level $c_j$ to level $c_k$—requires distinguishing the current level from other levels.

This process of distinguishing one level from $(n-1)$ others is precisely a partition operation: the set of all levels is partitioned into $\{c_j\}$ (the current level) and $\{c_1, \ldots, c_{j-1}, c_{j+1}, \ldots, c_n\}$ (all other levels). More generally, the complete structure of the categorical dimension corresponds to a complete $n$-way partition of the state space.

The categorical dimension and the partition level are not analogous structures—they are the same structure described in different languages.
\end{proof}

\begin{theorem}[Oscillation-Partition Equivalence]
\label{thm:osc_part}
An oscillatory transition IS a partition operation. Changing the quantum number of a mode partitions the system's history into "before the transition" and "after the transition."
\end{theorem}

\begin{proof}
Consider an oscillatory mode transitioning from state $|n\rangle$ to state $|n'\rangle$. This transition:
\begin{enumerate}
    \item Creates a distinction between the pre-transition configuration (with quantum number $n$) and the post-transition configuration (with quantum number $n'$)
    \item Divides the system's history into two categories: states visited before time $t$ (containing $|n\rangle$) and states visited after time $t$ (containing $|n'\rangle$)
    \item Is irreversible in the sense that the fact of the transition becomes part of the system's history
\end{enumerate}

This is precisely the structure of a partition operation: an undivided whole (the system before the transition) is divided into distinguishable parts (the before-state and the after-state). The oscillatory transition and the partition operation are the same process.
\end{proof}

\subsection{The Fundamental Equivalence}

\begin{theorem}[Fundamental Equivalence]
\label{thm:fundamental}
Oscillation, category, and partition are three perspectives on a single underlying structure. Specifically:
\begin{equation}
    \text{Oscillation} \equiv \text{Category} \equiv \text{Partition}
\end{equation}
where $\equiv$ denotes structural isomorphism.
\end{theorem}

\begin{proof}
By Theorem~\ref{thm:osc_cat}, oscillatory modes are isomorphic to categorical dimensions. By Theorem~\ref{thm:cat_part}, categorical dimensions are isomorphic to partition levels. By Theorem~\ref{thm:osc_part}, oscillatory transitions are isomorphic to partition operations. The three structures form a closed equivalence class.

Moreover, the entropy derived from each structure is identical (Theorem~\ref{thm:equivalence}). Since entropy is a complete invariant for statistical mechanical systems—two systems with the same entropy have the same thermodynamic behaviour—the three structures are thermodynamically indistinguishable.
\end{proof}

\subsection{Interpretation of the Unified Formula}

The unified entropy formula $S = \kB M \ln n$ admits a canonical interpretation:

\begin{definition}[Unified Entropy]
\label{def:unified}
The \emph{unified entropy} of a system is:
\begin{equation}
    S = \kB M \ln n
\end{equation}
where:
\begin{itemize}
    \item $M$ = number of independent degrees of freedom (modes / dimensions / partition levels)
    \item $n$ = number of distinguishable states per degree of freedom (quantum states / categorical levels / branches)
    \item $\kB = 1.380649 \times 10^{-23}$ J/K is Boltzmann's constant
\end{itemize}
\end{definition}

\begin{table}[H]
\centering
\caption{Parameter correspondence across the three frameworks}
\label{tab:correspondence}
\begin{tabular}{@{}lccc@{}}
\toprule
& Oscillatory & Categorical & Partition \\
\midrule
Degrees of freedom ($M$) & Modes & Dimensions & Levels \\
States per DOF ($n$) & Quantum numbers & Categorical levels & Branches \\
Configuration space & Phase space & Category space & Partition tree \\
Transition & Mode excitation & Level change & Branching \\
Entropy source & Mode counting & State counting & Path counting \\
\bottomrule
\end{tabular}
\end{table}

\begin{figure*}[htbp]
\centering
\includegraphics[width=0.95\textwidth]{figures/entropy_equivalence_panel.png}
\caption{\textbf{Unified Entropy: Oscillation $\equiv$ Category $\equiv$ Partition.} \textbf{(A)} Three independent derivations from oscillatory mechanics (periodic modes), categorical mechanics (distinguishable states), and partition mechanics (branching cascade) all converge to the same entropy formula. \textbf{(B)} Experimental verification using hardware-based virtual instruments: all three measured entropies ($S_{\text{osc}}$, $S_{\text{cat}}$, $S_{\text{part}}$) are mathematically identical as $M$ (degrees of freedom) increases. \textbf{(C)} Unified formula derivation showing $\Omega = n^M$ for all three approaches, yielding $S = k_B M \ln(n)$. \textbf{(D)} Parameter correspondence: $M$ represents modes/dimensions/depth; $n$ represents quantum states/categorical levels/branches. \textbf{(E)} Real thermodynamics with explicit $k_B = 1.380649 \times 10^{-23}$ J/K, showing actual entropy values for different $M$ at ternary partition ($n=3$). \textbf{(F)} Fundamental implication: the convergence of three independent derivations proves that oscillation, category, and partition describe the same underlying physical reality.}
\label{fig:entropy_equivalence}
\end{figure*}

\subsection{Implications of Unification}

\begin{corollary}[Single Underlying Reality]
\label{cor:single_reality}
The convergence of three independent derivations to a single formula demonstrates that oscillation, category, and partition describe a single underlying physical reality rather than three separate phenomena.
\end{corollary}

\begin{corollary}[Framework Independence]
\label{cor:independence}
Physical predictions made using any of the three frameworks must agree. A result derived in oscillatory mechanics can be translated to categorical or partition mechanics and will yield identical predictions.
\end{corollary}

\begin{corollary}[Entropy is Fundamental]
\label{cor:entropy_fundamental}
The unified entropy $S = \kB M \ln n$ is the fundamental quantity that unifies the three perspectives. Entropy is not merely a convenient summary statistic but the invariant that identifies oscillation, category, and partition as aspects of a single structure.
\end{corollary}

\subsection{The Universal Constant $\ln n$}

For systems with tri-dimensional structure ($M = 3k$) and ternary branching ($n = 3$):
\begin{equation}
    S = 3\kB k \ln 3 \approx 3.296 \kB k
\end{equation}

The factor $\ln 3 \approx 1.099$ appears as a universal constant in this framework, analogous to how $\ln 2 \approx 0.693$ appears in information theory as the conversion factor between bits and nats.

For binary partitioning ($n = 2$):
\begin{equation}
    S = \kB M \ln 2 \approx 0.693 \kB M
\end{equation}
which recovers the standard information-theoretic result that each binary choice contributes $\kB \ln 2$ to entropy.

