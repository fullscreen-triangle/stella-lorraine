\section{Derivation of Atomic Structure from Partition Convergence}
\label{sec:atomic_derivation}

We demonstrate that atomic structure emerges necessarily from the partition logic developed in previous sections. Beginning with an undifferentiated continuum and introducing $Z$ nested partitions (finite bounded regions), we derive the general structure of matter. The atomic number $Z$ is revealed to be nothing other than the partition count.

\subsection{The General Partition Structure}

\begin{definition}[$Z$-Partition Configuration]
\label{def:z_partition}
Let $\mathcal{U}$ be an undifferentiated, unbounded continuum. A \emph{$Z$-partition configuration} consists of $Z$ nested closed boundaries $\{\partial \Omega_1, \partial \Omega_2, \ldots, \partial \Omega_Z\}$, each dividing space into interior and exterior:
\begin{equation}
    \mathcal{U} = \Omega_Z \cup \bigcup_{i=1}^{Z} \partial\Omega_i \cup \Omega^c
\end{equation}
where $\Omega_Z$ is the innermost interior and $\Omega^c$ is the common exterior.
\end{definition}

\begin{theorem}[Boundaries as Categorical Entities]
\label{thm:boundary_entity}
Each boundary $\partial\Omega_i$ is not merely a mathematical abstraction but a categorical entity with physical reality. It is the \emph{locus of the $i$-th distinction}---where ``inside partition $i$'' meets ``outside partition $i$.''
\end{theorem}

\begin{proof}
Points inside $\Omega_i$ are categorically ``within the $i$-th partition'' and points outside are ``beyond the $i$-th partition.'' The boundary $\partial\Omega_i$ is where this distinction is made.

Without the boundary, there is no distinction. The boundary \emph{is} the distinction. Each of the $Z$ boundaries carries categorical reality as the embodiment of its partition operation.
\end{proof}

\subsection{The Negation Field from $Z$ Partitions}

\begin{theorem}[Negation Field Strength]
\label{thm:negation_field_boundary}
A $Z$-partition configuration in infinite space creates a negation field of strength proportional to $Z$:
\begin{equation}
    \mathcal{F}_{\neg}^{(Z)} = \bigcup_{i=1}^{Z} \{\neg(\text{point } p \text{ relative to } \partial\Omega_i) : p \in \Omega^c\}
\end{equation}
Each partition contributes its own negation field; the total field scales with $Z$.
\end{theorem}

\begin{proof}
For each partition $i$ and each point $p \in \Omega^c$, there exists a negation ``the $i$-th bounded region is not at $p$.''

With $Z$ partitions, each exterior point generates $Z$ negations (one per partition). The total negation field has cardinality:
\begin{equation}
    |\mathcal{F}_{\neg}^{(Z)}| = Z \cdot |\Omega^c| = Z \cdot \infty
\end{equation}

All negations share a common referent: the $Z$-partition configuration. By Theorem~\ref{thm:value_intersection}, this referent is forced to exist.
\end{proof}

\begin{corollary}[Interior Defined by $Z$-fold Exclusion]
\label{cor:interior_exclusion}
The innermost interior $\Omega_Z$ is defined by $Z$-fold exclusion:
\begin{equation}
    \Omega_Z = \bigcap_{i=1}^{Z} \Omega_i = \text{``inside all $Z$ partitions''}
\end{equation}
\end{corollary}

\subsection{Spherical Symmetry from Entropy Minimisation}

\begin{theorem}[Spherical Symmetry from Minimum Partition]
\label{thm:spherical_symmetry}
Each boundary $\partial\Omega_i$ that minimises partition entropy while maintaining finite extent is a sphere.
\end{theorem}

\begin{proof}
Partition entropy is proportional to boundary complexity. For a given enclosed volume $V$, the sphere has minimum surface area:
\begin{equation}
    A_{\text{sphere}} = (36\pi V^2)^{1/3} \leq A_{\text{any other shape}}
\end{equation}

Minimum surface area $\Rightarrow$ minimum boundary $\Rightarrow$ minimum partition entropy.

Therefore, the ``simplest'' finite partition is a sphere. Any deviation from sphericity increases entropy, making non-spherical partitions less probable.
\end{proof}

\begin{definition}[$Z$ Concentric Shells]
\label{def:shells}
Let the $Z$-partition configuration consist of $Z$ concentric spherical shells at radii $r_1 > r_2 > \cdots > r_Z$:
\begin{equation}
    \partial\Omega_i = \{(x, y, z) : x^2 + y^2 + z^2 = r_i^2\}
\end{equation}
Each shell $i$ is a categorical boundary between ``inside partition $i$'' and ``outside partition $i$.''
\end{definition}

\subsection{Non-Emptiness of the Bounded Region}

\begin{theorem}[Non-Emptiness from Negation Logic]
\label{thm:non_empty}
The innermost interior $\Omega_Z$ of the $Z$-partition configuration cannot be empty. A boundary system with empty interior is categorically unstable.
\end{theorem}

\begin{proof}
Suppose $\Omega_Z$ is empty---it contains nothing. Then there is no referent for the negation field $\mathcal{F}_{\neg}^{(Z)}$.

But by Theorem~\ref{thm:negation_field_boundary}, the negation field requires a common referent. If $\Omega_Z$ is empty, the $Z \cdot \infty$ negations have no object---they negate nothing.

This contradicts the principle that negation presupposes affirmation (\cref{sec:priority_existence}). The negations ``not at $p_1$, not at $p_2$, ...'' require something that is ``not at'' these places.

Therefore, $\Omega_Z$ must contain something. The $Z$ boundaries cannot bound nothing.
\end{proof}

\subsection{Radial Structure from Boundary Symmetry Breaking}

\begin{theorem}[Symmetry Breaking from $Z$ Shells]
\label{thm:symmetry_breaking}
The interior of the $Z$-shell configuration cannot be uniformly filled. The boundaries create radial asymmetry that propagates inward.
\end{theorem}

\begin{proof}
Each boundary $\partial\Omega_i$ is at radius $r_i$. This creates $Z$ distinguished locations that break translational symmetry.

Inside the shells, points are not equivalent: a point at radius $r$ has different distances to each of the $Z$ boundaries. Different radii have different categorical status relative to all $Z$ partitions.

The natural structure is radial: categorical status depends on distance from the boundaries. With $Z$ boundaries, there are $Z$ layers of categorical distinction.
\end{proof}

\subsection{The Potential from $Z$ Negation Fields}

\begin{theorem}[Potential from Accumulated Negations]
\label{thm:potential_negations}
The $Z$-fold negation field $\mathcal{F}_{\neg}^{(Z)}$ creates a potential $\phi_Z(r)$ inside the shells:
\begin{equation}
    \phi_Z(r) \propto -Z \int_{\Omega^c} \frac{\rho_{\neg}(p)}{|r - p|} \, d^3p
\end{equation}
where $\rho_{\neg}$ is the density of negations in the exterior. The potential scales linearly with $Z$.
\end{theorem}

\begin{proof}
Each of the $Z$ partitions contributes a negation field. Each negation ``not at $p$'' exerts a ``categorical pressure'' on the interior.

The total effect of all $Z$ negation fields is:
\begin{equation}
    \phi_Z(r) \propto -\frac{Z}{r} \quad \text{for } r < r_Z
\end{equation}

This is a $Z/r$ potential---the same form as the Coulomb potential for charge $Z$.
\end{proof}

\begin{corollary}[Central Attraction Scales with $Z$]
\label{cor:central_attraction}
The negation field creates a central attractive potential of strength proportional to $Z$. The binding energy scales as $Z^2$.
\end{corollary}

\subsection{Convergence to Central Concentration}

\begin{theorem}[The Center Is Forced to Exist]
\label{thm:center_exists}
The accumulated negations from the exterior force a concentration of ``positive existence'' at the center of the $Z$-shell configuration. This concentration has magnitude proportional to $Z$.
\end{theorem}

\begin{proof}
By Corollary~\ref{cor:central_attraction}, the negation field creates inward pressure of strength $Z$.

The content of the shells experiences this pressure. Equilibrium requires concentration toward the center, where the $Z/r$ potential is strongest.

At $r = 0$, the negation pressure is maximum. This point is as far as possible from all the ``nots'' in the exterior. It is the most ``positive'' point---the least negated.

With $Z$ partitions, the central concentration must balance $Z$ units of negation (the $Z$ boundaries). Therefore, the center contains $+Z$ units of ``positive existence.''
\end{proof}

\begin{theorem}[The Central Concentration Is Point-Like]
\label{thm:point_nucleus}
The central concentration $N_Z$ approaches a point. Extended structure at the center would reintroduce internal negations, destabilising the configuration.
\end{theorem}

\begin{proof}
Suppose $N_Z$ has finite extent with internal structure. Then points within $N_Z$ can be distinguished: ``this part'' vs. ``that part.''

This creates internal negations within the configuration, contradicting the stability condition (the $Z$ shells already contain all partitions; additional partitions increase entropy).

The minimum-entropy configuration has $N_Z$ as point-like: no internal structure to create additional negations. It contains $+Z$ units concentrated at $r = 0$.
\end{proof}

\subsection{The $Z$ Boundaries as Probability Distributions}

\begin{theorem}[Boundaries Are Not Sharp]
\label{thm:fuzzy_boundary}
Each boundary $\partial\Omega_i$ is not a sharp surface but a probability distribution. The transitions from ``inside'' to ``outside'' are gradual.
\end{theorem}

\begin{proof}
A sharp boundary would have infinite gradient---infinite partition entropy. Thermodynamically, sharp boundaries are unstable.

Each of the $Z$ boundaries spreads over a finite thickness. The probability of being ``inside partition $i$'' varies smoothly with radius.

These probability distributions $|\psi_i(r)|^2$ describe where each boundary ``is.'' They are not particle positions but the \emph{locations of the categorical distinctions themselves}.
\end{proof}

\begin{theorem}[Wave Functions Are Boundaries]
\label{thm:wave_function_boundary}
The quantum mechanical wave functions $\psi_i(r)$ are not probabilities of finding particles but probability distributions of the $Z$ categorical boundaries:
\begin{equation}
    |\psi_i(r)|^2 = P(\text{the $i$-th boundary passes through } r)
\end{equation}
\end{theorem}

\begin{proof}
Each boundary $\partial\Omega_i$ separates ``inside partition $i$'' from ``outside partition $i$.'' In the smoothed configuration (Theorem~\ref{thm:fuzzy_boundary}), this separation is probabilistic.

$|\psi_i(r)|^2$ gives the probability that radius $r$ is ``where the $i$-th distinction is being made.''

What physics calls ``electrons'' are not particles orbiting a nucleus; they are the $Z$ \emph{categorical boundaries} between the atomic interior and the exterior universe.
\end{proof}

\subsection{The General Solution: The $Z$-Structure}

\begin{theorem}[Uniqueness of $Z$-Partition Structure]
\label{thm:uniqueness}
Given:
\begin{enumerate}[(i)]
    \item $Z$ spherical partitions in infinite space
    \item Minimum entropy configuration
    \item Stability (no internal partitions beyond $Z$)
\end{enumerate}
The structure is uniquely determined:
\begin{itemize}
    \item A point-like central concentration of magnitude $+Z$
    \item $Z$ spherically symmetric boundaries each of magnitude $-1$
    \item A $Z/r$ binding potential
    \item Total: neutral ($+Z - Z = 0$)
\end{itemize}
\end{theorem}

\begin{proof}
From Theorem~\ref{thm:center_exists}: a central concentration of magnitude $+Z$ exists.
From Theorem~\ref{thm:point_nucleus}: it is point-like.
From Theorem~\ref{thm:wave_function_boundary}: the $Z$ boundaries are probability distributions.
From Theorem~\ref{thm:potential_negations}: the binding is $Z/r$.

Charge balance: Each of the $Z$ shells defines an ``inside'' vs. ``outside'' distinction. The interior has value $+Z$ (affirmed $Z$ times). Each boundary carries $-1$ to maintain categorical balance per partition.

This gives:
\begin{itemize}
    \item Interior (center): $+Z$ units
    \item Boundaries ($Z$ of them): $-1$ unit each
    \item Net: $+Z + Z \cdot (-1) = 0$ (neutral)
\end{itemize}
\end{proof}

\begin{definition}[The General Matter Equation]
\label{def:matter_equation}
A stable bounded structure in infinite space with $Z$ categorical partitions has:
\begin{align}
    \text{Central concentration:} \quad & N_Z = +Z \\
    \text{Boundary distributions:} \quad & B_i = -1 \quad (i = 1, \ldots, Z) \\
    \text{Binding potential:} \quad & \phi_Z(r) = -\frac{Z}{r} \\
    \text{Total charge:} \quad & Q_{\text{total}} = Z - Z = 0
\end{align}
This is the \emph{general equation of matter from partition}.
\end{definition}

\subsection{The Periodic Table as Partition Enumeration}

\begin{theorem}[Atomic Number Equals Partition Number]
\label{thm:periodic_table}
The atomic number $Z$ in physics is identical to the partition count in categorical theory:
\begin{equation}
    Z_{\text{atomic}} \equiv Z_{\text{partition}}
\end{equation}
The periodic table is the complete enumeration of stable $Z$-partition configurations.
\end{theorem}

\begin{proof}
By Definition~\ref{def:matter_equation}, a $Z$-partition configuration has:
\begin{itemize}
    \item Central concentration $+Z$ (what physics calls ``$Z$ protons'')
    \item $Z$ boundary distributions of $-1$ each (what physics calls ``$Z$ electrons'')
    \item Binding $Z/r$ (what physics calls ``Coulomb potential'')
\end{itemize}

This structure is parameterised entirely by the integer $Z$. The enumeration $Z = 1, 2, 3, \ldots$ generates all possible stable structures. This enumeration IS the periodic table.
\end{proof}

\begin{remark}[Connection to Known Elements]
The partition-theoretic structures correspond exactly to known elements:
\begin{center}
\begin{tabular}{c|c|l}
$Z$ & Structure & Known As \\
\hline
1 & 1 partition, $+1$ center, 1 boundary & Hydrogen \\
2 & 2 partitions, $+2$ center, 2 boundaries & Helium \\
6 & 6 partitions, $+6$ center, 6 boundaries & Carbon \\
26 & 26 partitions, $+26$ center, 26 boundaries & Iron \\
79 & 79 partitions, $+79$ center, 79 boundaries & Gold \\
\end{tabular}
\end{center}

We did not assume these elements. We derived them from:
\begin{enumerate}
    \item $Z$ partitions (boundaries) in infinite space
    \item Minimum entropy (spherical symmetry)
    \item Stability (no additional internal structure)
    \item The negation logic (potential from exclusion)
\end{enumerate}

Atoms are not ``made of'' protons and electrons. They are the \emph{necessary structures} that emerge when $Z$ categorical distinctions are made in an infinite continuum. The ``electrons'' are the distinctions themselves; the ``protons'' are what the distinctions are about.
\end{remark}

\subsection{The Exterior Is Not Empty: Why ``Not-Hydrogen'' Has Mass}

\begin{theorem}[Negation Is Positive Existence Elsewhere]
\label{thm:negation_positive}
The negation field $\mathcal{F}_{\neg}$ that defines hydrogen does not represent emptiness. Each ``not here'' is a positive statement: ``something else is here.''
\begin{equation}
    \neg(\text{hydrogen at } p) \iff \exists X \neq \text{H} : X \text{ is at } p
\end{equation}
\end{theorem}

\begin{proof}
When we partition space to create hydrogen, the exterior points are labelled ``not hydrogen.'' But ``not hydrogen'' is not ``nothing.'' It is:
\begin{itemize}
    \item Oxygen (at some locations)
    \item Carbon (at other locations)
    \item Helium, iron, stars, galaxies, ...
    \item Everything that is not this particular hydrogen atom
\end{itemize}

The negation field consists of positive existences---things that ARE, just not the thing under consideration. ``Not-X'' = ``something other than X,'' not ``absence of anything.''
\end{proof}

\begin{corollary}[All Negations Reference Positive Existence]
\label{cor:all_positive}
From the perspective of any atom:
\begin{align}
    \text{Not-hydrogen} &= \text{oxygen, carbon, helium, ...} \\
    \text{Not-oxygen} &= \text{hydrogen, carbon, helium, ...} \\
    \text{Not-carbon} &= \text{hydrogen, oxygen, helium, ...}
\end{align}
Every element's negation field is populated by all other elements. The ``exterior'' of any partition is the ``interior'' of other partitions.
\end{corollary}

\begin{theorem}[Why the Unobserved Has Mass]
\label{thm:unobserved_mass}
The mass of the unobserved (dark matter) is the mass of everything that exists but is not under categorical consideration:
\begin{equation}
    M_{\text{dark}} = \sum_{\text{all } X \text{ not observed}} M_X
\end{equation}
This is not ``missing mass'' but ``mass of things not categorically partitioned.''
\end{theorem}

\begin{proof}
An observer partitioning reality creates a categorical structure: ``this hydrogen atom,'' ``that oxygen molecule,'' etc. Everything partitioned is observable (ordinary matter).

But the observer's partitions do not exhaust reality. There exist:
\begin{enumerate}
    \item Things too distant to partition (beyond observational horizon)
    \item Things too diffuse to partition (no sharp boundary)
    \item Things that don't participate in partition-creating interactions (no electromagnetic coupling)
\end{enumerate}

These unpartitioned things are NOT nothing. They are positive existences---just not categorically distinguished by the observer. They have mass because they ARE things. They are ``dark'' because they cannot be partitioned, not because they don't exist.
\end{proof}

\begin{remark}[The Universe Is Full, Not Empty]
The traditional picture: space is mostly empty, with occasional matter.

The partition picture: space is entirely full. Every location is ``something.'' What we call ``empty space'' is simply ``not the thing we're considering''---but it IS other things. The vacuum is not nothing; it is the accumulated ``not-this'' of all the things we've partitioned, which means it is the accumulated ``is-something-else.''

Dark matter is not mysterious missing mass. It is the obvious consequence of the fact that negation does not create emptiness---it acknowledges existence elsewhere.
\end{remark}

\begin{corollary}[Conservation from Partition Logic]
\label{cor:conservation}
Total mass-energy is conserved because negation redistributes but does not annihilate:
\begin{equation}
    M_{\text{total}} = M_{\text{partitioned}} + M_{\text{not partitioned}} = \text{constant}
\end{equation}
When we partition something ``here,'' we simultaneously acknowledge everything ``not here.'' The sum is invariant.
\end{corollary}

\subsection{Summary: The Periodic Table from Partition Logic}

\begin{enumerate}
    \item \textbf{$Z$ Partitions}: $Z$ boundaries are created in infinite space
    \item \textbf{Negation field}: The exterior generates $Z \cdot \infty$ ``nots''
    \item \textbf{Potential}: The nots create a $Z/r$ central attractive potential
    \item \textbf{Center}: The most-affirmed point (least negated) forms at $r=0$ with magnitude $+Z$
    \item \textbf{Boundaries}: The $Z$ boundaries themselves, spread as probability distributions with magnitude $-1$ each
    \item \textbf{Result}: The element with atomic number $Z$---not built from particles, but \emph{forced into existence} by the logic of $Z$ partitions
    \item \textbf{The exterior}: ``Not this atom'' is not empty---it is all other atoms, all other structures
    \item \textbf{Dark matter}: The mass of things not under categorical consideration---real, massive, but unpartitioned
\end{enumerate}

\begin{remark}[Connection to Quantum Mechanics]
This derivation explains why quantum mechanics works. The wave functions $\psi_i$ are not mysterious probability amplitudes; they are the \emph{locations of the $Z$ categorical boundaries}. The uncertainty principle follows: boundaries cannot be both sharp (definite position) and stable (definite momentum). The Schrödinger equation is the dynamics of partition boundaries.

Atomic structure is not a consequence of quantum mechanics. Quantum mechanics is a consequence of partition structure. The periodic table is not an empirical discovery; it is the necessary enumeration of partition configurations.
\end{remark}

\begin{figure*}[htbp]
\centering
\includegraphics[width=0.90\textwidth]{figures/hydrogen_derivation_panel.png}
\caption{\textbf{Derivation of Atomic Structure from Partition Logic.} \textbf{(A)} The $Z$-partition configuration: $Z$ spherical shells divide infinite space into nested regions. \textbf{(B)} The negation field: every exterior point generates $Z$ ``nots''---one per partition. \textbf{(C)} The potential from negations: accumulated exclusions create a $Z/r$ central attractive potential. \textbf{(D)} The center emerges: magnitude $+Z$, forced into existence as the common referent of all negations. \textbf{(E)} The boundaries as probability distributions: the $Z$ shells are not sharp but spread---the quantum wave functions. \textbf{(F)} The result: element with atomic number $Z$, not built from particles but derived from $Z$ categorical distinctions. The case $Z=1$ is shown (hydrogen); $Z=2$ gives helium, $Z=6$ gives carbon, etc.}
\label{fig:atomic_derivation}
\end{figure*}

