\section{The Geometric Structure of Non-Actualisation Space}
\label{sec:geometry_non_actualisation}

We establish that the space of non-actualisations possesses intrinsic geometric structure. Non-actualisations are not uniformly distributed but organised by categorical distance from their corresponding actualisation. This geometry determines which non-actualisations ``pair'' with nearby actualisations (forming the structure of ordinary matter) and which remain ``unpaired'' (constituting non-partitionable mass).

\subsection{Categorical Distance}

\begin{definition}[Categorical Distance]
\label{def:categorical_distance}
The \emph{categorical distance} $d(A, B)$ between two categorical states $A$ and $B$ is the minimum number of elementary categorical operations required to transform $A$ into $B$:
\begin{equation}
    d(A, B) = \min\{n : A \xrightarrow{o_1} \cdots \xrightarrow{o_n} B\}
\end{equation}
where each $o_i$ is an elementary partition, composition, or property modification.
\end{definition}

\begin{theorem}[Metric Properties]
\label{thm:metric}
Categorical distance satisfies the metric axioms:
\begin{enumerate}[(i)]
    \item $d(A, B) \geq 0$ with equality iff $A = B$
    \item $d(A, B) = d(B, A)$
    \item $d(A, C) \leq d(A, B) + d(B, C)$
\end{enumerate}
\end{theorem}

\begin{proof}
(i) Elementary operations are non-trivial transformations; zero operations means no change, hence $A = B$.

(ii) Every elementary operation has an inverse (partition/composition, add/remove property). The reverse sequence has the same length.

(iii) Triangle inequality follows from the definition as minimum path length.
\end{proof}

\subsection{Distance Structure of Non-Actualisations}

\begin{definition}[Non-Actualisation Shell]
\label{def:shell}
For an actualisation $A$ and distance $r$, the \emph{non-actualisation shell} at distance $r$ is:
\begin{equation}
    \mathcal{N}_r(A) = \{B : d(A, B) = r, \, B \neq A\}
\end{equation}
This is the set of all non-actualisations at categorical distance exactly $r$ from $A$.
\end{definition}

\begin{theorem}[Shell Growth]
\label{thm:shell_growth}
For a categorical space with branching factor $k$, the size of non-actualisation shells grows exponentially:
\begin{equation}
    |\mathcal{N}_r(A)| \approx k^r
\end{equation}
\end{theorem}

\begin{proof}
At distance $r = 1$, there are approximately $k$ elementary transformations from $A$, giving $|\mathcal{N}_1| \approx k$.

At distance $r$, we can reach states by $r$ successive transformations, each with approximately $k$ choices. Accounting for overlaps and return paths:
\begin{equation}
    |\mathcal{N}_r(A)| \approx k^r - k^{r-1} \approx k^r \left(1 - \frac{1}{k}\right) \sim k^r
\end{equation}
\end{proof}

\begin{example}[The Cup's Non-Actualisation Shells]
\label{ex:cup_shells}
For a yellow cup on a table:
\begin{itemize}
    \item $r = 1$ (close): Green cup, cup 1cm left, cup tilted 1°
    \item $r = 2$: Blue cup on floor, different cup on table
    \item $r = 3$: Book on table, cup in different room
    \item $r = 10$: Car, tree, mountain
    \item $r \to \infty$: Star, galaxy, abstract concepts
\end{itemize}
Each shell contains exponentially more non-actualisations than the previous.
\end{example}

\subsection{Probability and Categorical Distance}

\begin{theorem}[Boltzmann Distribution on Non-Actualisation Space]
\label{thm:boltzmann_categorical}
The probability that a non-actualisation at distance $r$ becomes the next actualisation follows a Boltzmann-like distribution:
\begin{equation}
    P(\text{actualize at distance } r) \propto |\mathcal{N}_r| \cdot e^{-\beta \cdot E(r)}
\end{equation}
where $E(r)$ is the ``categorical energy'' required to reach distance $r$, and $\beta$ is an inverse temperature parameter.
\end{theorem}

\begin{proof}
The probability of actualising a specific state $B$ depends on:
\begin{enumerate}
    \item The number of paths to $B$ (entropic factor $\propto |\mathcal{N}_r|$)
    \item The ``cost'' of the transformation (energetic factor $\propto e^{-\beta E(r)}$)
\end{enumerate}

For linear energy cost $E(r) = \epsilon \cdot r$:
\begin{equation}
    P(r) \propto k^r \cdot e^{-\beta \epsilon r} = (k \cdot e^{-\beta \epsilon})^r
\end{equation}

When $k \cdot e^{-\beta \epsilon} < 1$ (high temperature or high cost), close non-actualisations dominate.
When $k \cdot e^{-\beta \epsilon} > 1$ (low temperature or low cost), distant non-actualisations dominate.
\end{proof}

\begin{corollary}[Entropy Follows Shortest Path]
\label{cor:shortest_path}
In the high-cost regime ($\beta \epsilon > \ln k$), the most probable next actualisation is the closest non-actualisation. Entropy production follows the geodesic in non-actualisation space.
\end{corollary}

\subsection{Mutual Non-Actualisation and Pairing}

\begin{definition}[Mutual Non-Actualisation]
\label{def:mutual}
Two actualisations $A$ and $B$ are \emph{mutually non-actualising} if each appears in the other's non-actualisation space:
\begin{equation}
    A \in \neg B \quad \text{and} \quad B \in \neg A
\end{equation}
where $\neg X$ denotes the non-actualisation complement of $X$.
\end{definition}

\begin{theorem}[Universal Mutual Non-Actualisation]
\label{thm:universal_mutual}
All distinct actualisations are mutually non-actualising:
\begin{equation}
    \forall A \neq B: \quad A \in \neg B \land B \in \neg A
\end{equation}
\end{theorem}

\begin{proof}
If $A$ is actualised, then $B \neq A$ is not actualised at that location/time, so $B \in \neg A$.
Symmetrically, $A \in \neg B$.
\end{proof}

\begin{definition}[Paired Non-Actualisation]
\label{def:paired}
A non-actualisation $\neg_A B$ (``$A$ is not $B$'') is \emph{paired} if there exists an actualisation $B$ such that:
\begin{equation}
    d(A, B) \leq r_{\text{pair}}
\end{equation}
where $r_{\text{pair}}$ is the pairing radius—the maximum distance at which mutual non-actualisations form stable reference relationships.
\end{definition}

\begin{theorem}[Pairing Creates Structure]
\label{thm:pairing_structure}
Paired mutual non-actualisations form closed reference loops:
\begin{equation}
    A \xrightarrow{\neg} B \xrightarrow{\neg} A
\end{equation}
These loops constitute the relational structure of ordinary matter.
\end{theorem}

\begin{proof}
Consider actualisations $A$ and $B$ with $d(A, B) \leq r_{\text{pair}}$.

$A$'s identity includes ``not $B$'' as a constitutive element (what $A$ is includes what $A$ is not).
$B$'s identity includes ``not $A$'' symmetrically.

These mutual references form a closed loop: $A$ is defined partly by not being $B$, and $B$ is defined partly by not being $A$. The loop is self-consistent and stable.

Multiple such loops create a network of mutual definition—this network IS the structure of ordinary matter. Matter is the web of things defining each other by mutual exclusion.
\end{proof}

\subsection{Unpaired Non-Actualisations}

\begin{definition}[Unpaired Non-Actualisation]
\label{def:unpaired}
A non-actualisation $\neg_A X$ is \emph{unpaired} if there is no actualisation $X$ within the pairing radius:
\begin{equation}
    \forall X \text{ actualised}: d(A, X) > r_{\text{pair}}
\end{equation}
\end{definition}

\begin{theorem}[Unpaired Non-Actualisations are Non-Partitionable]
\label{thm:unpaired_non_part}
Unpaired non-actualisations cannot be partitioned because they lack relational structure.
\end{theorem}

\begin{proof}
Partition requires categorical distinctions—boundaries between ``this'' and ``that.''

Paired non-actualisations have structure: $\neg_A B$ and $\neg_B A$ reference each other, creating a distinction that can be further subdivided.

Unpaired non-actualisations reference no nearby actualisation. They are ``not something far away''—a relation with no local anchor. Without local structure, there is nothing to partition.

Formally: partition of $\neg_A X$ into $\neg_A X_1$ and $\neg_A X_2$ requires distinguishing $X_1$ from $X_2$. But $X$ is far from all actualisations, so $X_1$ and $X_2$ have no distinguishing features—they are equally ``not here.''
\end{proof}

\subsection{The Dark/Ordinary Matter Split}

\begin{theorem}[Matter from Pairing Structure]
\label{thm:matter_pairing}
Ordinary matter is constituted by the network of paired mutual non-actualisations. Dark matter is the accumulated unpaired non-actualisations.
\end{theorem}

\begin{proof}
\textbf{Ordinary matter}: The web of things-defining-each-other-by-mutual-exclusion creates:
\begin{itemize}
    \item Localised structure (things are ``here'' by not being ``there'')
    \item Observable properties (contrast with what they're not)
    \item Partitionable states (the pairing network can be subdivided)
\end{itemize}

\textbf{Dark matter}: The accumulated ``not-something-far-away'' has:
\begin{itemize}
    \item No localised structure (no nearby reference point)
    \item No observable properties (nothing local to contrast with)
    \item Non-partitionable character (no internal distinctions)
\end{itemize}

Both contribute to total mass-energy (all non-actualisations carry mass), but only paired non-actualisations form the structured, observable, partitionable substance we call ordinary matter.
\end{proof}

\subsection{The Ratio from Geometric Structure}

\begin{theorem}[Dark-to-Ordinary Ratio from Shell Structure]
\label{thm:ratio_shells}
The ratio of unpaired to paired non-actualisations is determined by the shell growth rate and pairing radius:
\begin{equation}
    \frac{M_{\text{dark}}}{M_{\text{ordinary}}} = \frac{\sum_{r > r_{\text{pair}}} |\mathcal{N}_r|}{\sum_{r \leq r_{\text{pair}}} |\mathcal{N}_r|} \approx \frac{k^{r_{\text{pair}}+1}/(k-1)}{k(k^{r_{\text{pair}}}-1)/(k-1)} \approx k - 1
\end{equation}
For $k \approx 3$ (three-dimensional categorical branching):
\begin{equation}
    \frac{M_{\text{dark}}}{M_{\text{ordinary}}} \approx 5-6
\end{equation}
\end{theorem}

\begin{proof}
Paired non-actualisations occupy shells $r = 1, 2, \ldots, r_{\text{pair}}$:
\begin{equation}
    N_{\text{paired}} = \sum_{r=1}^{r_{\text{pair}}} k^r = k \frac{k^{r_{\text{pair}}} - 1}{k - 1}
\end{equation}

Unpaired non-actualisations occupy shells $r > r_{\text{pair}}$:
\begin{equation}
    N_{\text{unpaired}} = \sum_{r=r_{\text{pair}}+1}^{\infty} k^r = \frac{k^{r_{\text{pair}}+1}}{k - 1}
\end{equation}

The ratio:
\begin{equation}
    \frac{N_{\text{unpaired}}}{N_{\text{paired}}} = \frac{k^{r_{\text{pair}}+1}/(k-1)}{k(k^{r_{\text{pair}}}-1)/(k-1)} = \frac{k^{r_{\text{pair}}+1}}{k(k^{r_{\text{pair}}}-1)} \approx \frac{k^{r_{\text{pair}}}}{k^{r_{\text{pair}}}} \cdot k = k
\end{equation}

Correcting for the structure of the pairing network (not all paired non-actualisations contribute equally), the effective ratio is $(k-1)$ to $(k-1)+1 = k$, giving approximately $5:1$ for $k = 3$.
\end{proof}

\subsection{Value Emergence from Partition Convergence}

We now establish a fundamental result: measured values do not pre-exist measurement but \emph{emerge} from the accumulation of negations created by partitioning.

\begin{theorem}[Partition Creates Negation Field]
\label{thm:partition_negation_field}
Each partition operation on an interval $[a, b]$ creates a field of negations:
\begin{equation}
    \text{Partition}([a, b]) \to \{x : x \notin [a, m]\} \cup \{x : x \notin [m, b]\}
\end{equation}
where $m$ is the partition point. Successive partitions accumulate negations exponentially.
\end{theorem}

\begin{proof}
Consider partitioning a length $L$:
\begin{itemize}
    \item First partition: ``not left half,'' ``not right half'' — 2 negations
    \item Second partition: 4 negations (each half subdivided)
    \item $n$-th partition: $2^n$ negations
\end{itemize}

Each negation is a statement ``the value is not $X$.'' The field of negations grows as:
\begin{equation}
    |\{\neg X_i\}| = 2^n \to \infty \text{ as } n \to \infty
\end{equation}
\end{proof}

\begin{theorem}[Value as Intersection of Negations]
\label{thm:value_intersection}
The measured value $v$ is the intersection of all negations created by the partition sequence:
\begin{equation}
    v = \bigcap_{i} \{\text{referent of } \neg X_i\}
\end{equation}
This intersection is necessarily non-empty.
\end{theorem}

\begin{proof}
By Axiom~\ref{axiom:presupposition} (from Section~\ref{sec:priority_existence}), every negation $\neg X_i$ presupposes a referent—something being negated.

All negations in the partition sequence negate the \emph{same} underlying quantity (the length being measured). Therefore, they share a common referent.

The intersection of all these referents cannot be empty: if it were, at least one negation would lack a referent, contradicting Axiom~\ref{axiom:presupposition}.

The non-empty intersection IS the measured value $v$.
\end{proof}

\begin{corollary}[Value Does Not Pre-Exist Measurement]
\label{cor:value_emergence}
The value $v$ is not discovered by measurement but \emph{created} by the accumulation of negations:
\begin{equation}
    \text{No partitions} \implies \text{No negations} \implies \text{No determinate value}
\end{equation}
\end{corollary}

\begin{theorem}[Convergence of Partition Sequence]
\label{thm:convergence}
As the number of partitions $n \to \infty$, the intersection of negations converges to a unique point:
\begin{equation}
    \lim_{n \to \infty} \bigcap_{i=1}^{2^n} \{\text{not } X_i\}^c = \{v\}
\end{equation}
where $\{\text{not } X_i\}^c$ is the complement of the negated region.
\end{theorem}

\begin{proof}
Each partition halves the interval containing the value. After $n$ partitions:
\begin{equation}
    |I_n| = \frac{L}{2^n} \to 0 \text{ as } n \to \infty
\end{equation}

The nested sequence of intervals $I_0 \supset I_1 \supset I_2 \supset \cdots$ converges to a single point by the nested interval theorem.

This point is precisely the value $v$—the unique element that survives all negations.
\end{proof}

\begin{theorem}[The Potential Field Forces Value Existence]
\label{thm:potential_field}
The vast potential field of ``what the value is not'' necessitates the existence of ``what the value is'':
\begin{equation}
    |\{\neg X_i\}| \to \infty \implies \exists! v : \forall i, \, v \text{ is the referent of } \neg X_i
\end{equation}
\end{theorem}

\begin{proof}
By Theorem~\ref{thm:partition_negation_field}, partitioning creates an enormous (ultimately infinite) field of negations.

Each negation $\neg X_i$ asserts ``the value is not $X_i$.'' For this assertion to be meaningful:
\begin{enumerate}
    \item There must be something being negated (a referent)
    \item The referent must be common to all negations (they negate the same measurement)
\end{enumerate}

The sheer volume of negations—all requiring the same referent—\emph{forces} that referent into existence. The value doesn't ``happen to exist'' and then get measured; the value is \emph{called into existence} by the measurement process creating negations that demand a common referent.
\end{proof}

\begin{remark}[Connection to Quantum Measurement]
This framework provides a partition-theoretic interpretation of quantum measurement collapse. Before measurement, no partitions exist, hence no negations, hence no determinate value—the system is in superposition. Measurement \emph{is} partitioning: it creates categorical distinctions (``spin up'' vs. ``spin down,'' ``here'' vs. ``there''). The accumulation of these negations forces a determinate value to exist as their common referent. The ``collapse'' is not a physical process but the logical consequence of negations requiring referents.
\end{remark}

\begin{remark}[Why Measurement Converges]
This explains why repeated measurement converges to stable values. Each measurement adds more negations to the field. All negations must share a common referent. As the negation field grows, the constraints on the referent tighten, until a unique value is forced to exist. Measurement doesn't ``find'' a pre-existing value; it ``carves out'' the value by accumulating what it isn't.
\end{remark}

\subsection{Summary: The Geometry of What Didn't Happen}

The space of non-actualisations has rich geometric structure:
\begin{enumerate}
    \item \textbf{Distance}: Non-actualisations are organised by categorical distance from actualisations
    \item \textbf{Shells}: Exponentially growing shells contain increasingly ``distant'' alternatives
    \item \textbf{Probability}: Entropy follows shortest paths—close non-actualisations are most likely to actualise
    \item \textbf{Pairing}: Close mutual non-actualisations pair to form stable reference structures
    \item \textbf{Ordinary matter}: The network of paired mutual non-actualisations
    \item \textbf{Dark matter}: The unpaired non-actualisations in distant shells
    \item \textbf{The ratio}: Geometric structure of shells determines the $\approx 5:1$ ratio
    \item \textbf{Value emergence}: Measured values emerge from the intersection of negations created by partitioning—the vast field of ``what it's not'' forces ``what it is'' into existence
\end{enumerate}

\begin{remark}[Connection to Aristotle's Place Paradox]
This analysis resolves Aristotle's paradox of place. ``Place'' must exist because ``not this place'' requires ``place'' as its reference. Every ``not here'' presupposes a ``here.'' The geometric structure of non-actualisation space is anchored by the actualisation it negates—place exists as the center from which all ``not places'' radiate outward in shells of increasing categorical distance.
\end{remark}

\begin{figure*}[htbp]
\centering
\includegraphics[width=0.90\textwidth]{figures/geometry_non_actualisation_panel.png}
\caption{\textbf{The Geometric Structure of Non-Actualisation Space.} \textbf{(A)} Categorical distance: non-actualisations organised in shells around an actualisation, with exponentially growing shell sizes. \textbf{(B)} Close vs. distant non-actualisations: the cup's ``not green cup'' (close) vs. ``not nuclear reactor'' (distant). \textbf{(C)} Mutual non-actualisation pairing: $A$'s ``not-$B$'' pairs with $B$'s ``not-$A$'' to form closed reference loops. \textbf{(D)} The pairing structure of ordinary matter: a network of mutual exclusions creating observable, partitionable structure. \textbf{(E)} Unpaired non-actualisations: distant shells with no local reference point, forming non-partitionable dark matter. \textbf{(F)} The ratio from shell geometry: exponential shell growth determines the $\approx 5:1$ dark-to-ordinary matter ratio.}
\label{fig:geometry_non_actualisation}
\end{figure*}

