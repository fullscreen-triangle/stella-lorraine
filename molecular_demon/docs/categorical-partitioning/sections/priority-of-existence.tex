\section{The Logical Priority of Actualisation}
\label{sec:priority_existence}

We establish that actualisation (existence) is logically prior to non-actualisation (non-existence). Every negation presupposes what it negates; ``not-$X$'' requires $X$ to exist as its referent. This logical priority has profound consequences: non-actualisations cannot exist without actualisations to anchor them, explaining why dark matter follows ordinary matter and why there is something rather than nothing.

\subsection{The Presupposition Principle}

\begin{axiom}[Negation Presupposes Affirmation]
\label{axiom:presupposition}
Every negation $\neg X$ presupposes the existence of $X$ as a meaningful referent:
\begin{equation}
    \neg X \text{ is meaningful} \implies X \text{ exists as referent}
\end{equation}
\end{axiom}

\begin{theorem}[Negation Cannot Float Freely]
\label{thm:no_free_negation}
A negation without a referent is not a negation but a null expression:
\begin{equation}
    \neg(\text{nothing}) = \text{undefined}
\end{equation}
\end{theorem}

\begin{proof}
Consider the expression ``not-$X$'' where $X$ has no referent. The negation operator $\neg$ requires an operand—something to negate. Without $X$, $\neg$ has no input, and the expression is ill-formed.

Concretely: ``not the cup'' requires ``the cup'' to exist as a concept being negated. ``Not [undefined]'' is not a statement at all—it has no semantic content.

Therefore, every meaningful negation anchors to an existing referent.
\end{proof}

\subsection{The Intersection Argument}

\begin{theorem}[Existence from Negation Convergence]
\label{thm:intersection}
If infinitely many distinct negations $\{\neg P_1, \neg P_2, \ldots\}$ are meaningful, then their common referent must exist:
\begin{equation}
    X = \bigcap_i \{\text{what } P_i \text{ negates}\}
\end{equation}
The intersection of all negations' referents is non-empty.
\end{theorem}

\begin{proof}
Let $\{P_i\}$ be a collection of properties, and let $\{\neg P_i\}$ be their negations applied to some putative entity $X$.

Each $\neg P_i$ asserts ``$X$ does not have property $P_i$.'' For this assertion to be meaningful:
\begin{enumerate}
    \item $X$ must exist as a referent (Axiom~\ref{axiom:presupposition})
    \item Property $P_i$ must be applicable to $X$ (otherwise the negation is category error)
\end{enumerate}

If ALL the negations $\{\neg P_i\}$ are meaningful, then $X$ must be the common entity to which all these negations apply. The intersection:
\begin{equation}
    X = \bigcap_i \{\text{entities to which } \neg P_i \text{ applies}\}
\end{equation}
is non-empty; it contains at least $X$ itself.

Conversely, if the intersection were empty, at least one negation would lack a referent and be meaningless—contradicting the assumption that all negations are meaningful.
\end{proof}

\begin{example}[The Cup Defined by Negations]
\label{ex:cup_negations}
The cup on the table satisfies infinitely many negations:
\begin{itemize}
    \item Not a book
    \item Not red (if yellow)
    \item Not on the floor
    \item Not in Paris
    \item Not a car, not a tree, not a star, ...
\end{itemize}
Each negation presupposes the cup exists. The cup IS the entity that all these negations reference—it exists as the intersection of what all these ``not-$X$'' statements are negating.
\end{example}

\subsection{Non-Actualisation Requires Actualisation}

\begin{theorem}[Ontological Dependence]
\label{thm:dependence}
Non-actualisations depend ontologically on actualisations:
\begin{equation}
    \text{Non-actualisation } \neg A \text{ exists} \implies \text{Actualisation } A \text{ exists}
\end{equation}
The converse does not hold: actualisations do not require non-actualisations.
\end{theorem}

\begin{proof}
\textbf{Forward direction}: A non-actualisation $\neg A$ is the determination ``$A$ did not happen.'' This determination presupposes:
\begin{enumerate}
    \item $A$ is a coherent possibility (otherwise there's nothing to negate)
    \item Some actualisation occurred that resolved $A$ into ``did not happen''
\end{enumerate}

Without the actualisation that created the determination, $\neg A$ would be undetermined—neither actual nor non-actual, just unresolved possibility.

\textbf{Reverse direction fails}: An actualisation $A$ does not require non-actualisations to exist. Logically, $A$ could be the only entity in existence. Non-actualisations arise BECAUSE $A$ exists (everything else is ``not $A$''), but $A$'s existence is not conditioned on them.

Therefore: non-actualisations depend on actualisations, not vice versa.
\end{proof}

\begin{corollary}[Dark Matter Requires Ordinary Matter]
\label{cor:dark_requires_ordinary}
Dark matter (accumulated non-actualisations) cannot exist without ordinary matter (actualisations) to anchor it.
\end{corollary}

\begin{proof}
Dark matter is the accumulated ``what didn't happen'' (Section~\ref{sec:recursive_compounding}). Each ``didn't happen'' presupposes a ``did happen'' that resolved it. Without actualisations, there would be no determinations, hence no determined non-actualisations, hence no dark matter.
\end{proof}

\subsection{Why There Is Something Rather Than Nothing}

\begin{theorem}[Impossibility of Pure Nothing]
\label{thm:no_nothing}
Pure nothing—the absence of all actualisation—is self-contradictory.
\end{theorem}

\begin{proof}
Suppose there is ``nothing''—no actualisation whatsoever.

``Nothing'' is itself a determination: the determination that ``no thing exists.'' This determination is a form of non-actualisation: ``existence did not happen.''

But by Theorem~\ref{thm:dependence}, non-actualisation requires actualisation. ``Existence did not happen'' presupposes that ``existence'' is a meaningful referent that could have happened.

Therefore: the determination ``nothing exists'' presupposes that ``existence'' is meaningful, which requires some actualisation to anchor the concept.

Pure nothing is self-undermining: asserting ``nothing'' presupposes the meaningfulness of ``something,'' which requires something to exist.
\end{proof}

\begin{theorem}[Something Is Necessary]
\label{thm:something_necessary}
The existence of something (some actualisation) is logically necessary:
\begin{equation}
    \Box (\exists A : A \text{ is actualised})
\end{equation}
\end{theorem}

\begin{proof}
By contraposition of Theorem~\ref{thm:no_nothing}: if pure nothing is impossible, then something is necessary.

Alternative proof: Consider any possible world. If it contains anything—even the determination ``this is an empty world''—it contains an actualisation (that determination). If it contains truly nothing, it is not a world but the absence of any state, which is not a coherent possibility.

Therefore: in every possible world, something is actualised.
\end{proof}

\begin{remark}[Resolution of Leibniz's Question]
Leibniz asked: ``Why is there something rather than nothing?'' The answer from our framework: ``nothing'' presupposes ``something'' to be its referent. Pure non-existence requires existence to be meaningful. Something must exist for nothing to be a coherent concept. The question is therefore malformed—``nothing'' cannot be the alternative to ``something'' because ``nothing'' depends on ``something.''
\end{remark}

\subsection{Mutual Constitution}

\begin{theorem}[Mutual Constitution of Actual and Non-Actual]
\label{thm:mutual_constitution}
Although non-actualisations depend on actualisations (Theorem~\ref{thm:dependence}), actualisations are partly constituted by their non-actualisations:
\begin{equation}
    \text{Identity}(A) = \text{Intrinsic}(A) \cup \{\neg B : B \neq A\}
\end{equation}
\end{theorem}

\begin{proof}
The identity of an actualisation $A$ includes:
\begin{enumerate}
    \item \textbf{Intrinsic properties}: What $A$ is in itself
    \item \textbf{Negative properties}: What $A$ is not
\end{enumerate}

The cup is not just ``yellow, ceramic, cylindrical''—it is also ``not red, not plastic, not cubic.'' These negative determinations are constitutive of the cup's identity.

By Section~\ref{sec:geometry_non_actualisation}, these negative properties form the pairing structure with nearby actualisations. The cup's ``not a mug''-ness pairs with the mug's ``not a cup''-ness.

Therefore: while actualisations are logically prior (they anchor non-actualisations), the identity of each actualisation is partly constituted by its relations of non-being to other actualisations.
\end{proof}

\begin{corollary}[No Actualisation Is Fully Isolated]
\label{cor:no_isolation}
Every actualisation is relationally connected to every other through mutual non-actualisation:
\begin{equation}
    \forall A, B: \quad A \xleftrightarrow{\neg} B
\end{equation}
where $\xleftrightarrow{\neg}$ denotes mutual non-actualisation (each is in the other's non-actualisation space).
\end{corollary}

\subsection{The Structure of Reality}

\begin{theorem}[Reality as Actualisation-Anchored Non-Actualisation Web]
\label{thm:reality_structure}
The structure of reality consists of:
\begin{enumerate}[(i)]
    \item \textbf{Actualisations}: The logically primary entities that anchor all determinations
    \item \textbf{Paired non-actualisations}: The mutual exclusions between nearby actualisations, forming the structure of ordinary matter
    \item \textbf{Unpaired non-actualisations}: The distant non-actualisations without local anchors, forming dark matter
\end{enumerate}
The ratio between (ii) + (iii) is determined by the geometry of non-actualisation space.
\end{theorem}

\begin{proof}
Combines results from Sections~\ref{sec:recursive_compounding} and \ref{sec:geometry_non_actualisation}:
\begin{itemize}
    \item Actualisations exist necessarily (Theorem~\ref{thm:something_necessary})
    \item Each actualisation generates non-actualisations (Theorem~\ref{thm:resolution})
    \item Non-actualisations have geometric structure (Theorem~\ref{thm:shell_growth})
    \item Close non-actualisations pair (Theorem~\ref{thm:pairing_structure})
    \item Distant non-actualisations remain unpaired (Definition~\ref{def:unpaired})
    \item The ratio is $\approx 5:1$ (Theorem~\ref{thm:ratio_shells})
\end{itemize}
\end{proof}

\subsection{Summary: Existence Precedes Non-Existence}

\begin{enumerate}
    \item \textbf{Negation presupposes affirmation}: Every ``not-$X$'' requires $X$ to exist
    \item \textbf{Non-actualisations depend on actualisations}: Dark matter requires ordinary matter to anchor it
    \item \textbf{Pure nothing is impossible}: ``Nothing'' presupposes ``something'' to be meaningful
    \item \textbf{Something is necessary}: In every possible world, something is actualised
    \item \textbf{Mutual constitution}: Actualisations are partly defined by what they're not
    \item \textbf{Reality structure}: Actualisations + paired non-actualisations (ordinary matter) + unpaired non-actualisations (dark matter)
\end{enumerate}

\begin{remark}[Connection to Classical Paradoxes]
This analysis resolves multiple classical puzzles:
\begin{itemize}
    \item \textbf{Parmenides}: ``Non-being cannot be''—correct, non-being depends on being
    \item \textbf{Leibniz}: ``Why something rather than nothing?''—nothing presupposes something
    \item \textbf{Aristotle's Place Paradox}: Place exists because not-place requires place as referent
    \item \textbf{The problem of negative facts}: Negations are grounded in positive actualisations
\end{itemize}
The logical priority of actualisation provides a unified resolution: existence is primary, non-existence is derivative and dependent.
\end{remark}

\begin{figure*}[htbp]
\centering
\includegraphics[width=0.90\textwidth]{figures/priority_existence_panel.png}
\caption{\textbf{The Logical Priority of Actualisation.} \textbf{(A)} Negation presupposes affirmation: ``not-cup'' requires ``cup'' to exist as referent; negation cannot float freely. \textbf{(B)} The intersection argument: the cup exists as the common referent of infinitely many negations (not-book, not-car, not-red, ...). \textbf{(C)} Ontological dependence: non-actualisations require actualisations; arrow shows direction of dependence. \textbf{(D)} Why something rather than nothing: ``nothing'' presupposes ``something'' to be meaningful; pure nothing is self-contradictory. \textbf{(E)} Mutual constitution: actualisations are partly defined by what they're not, creating the pairing structure. \textbf{(F)} The structure of reality: actualisations (center) anchor paired non-actualisations (ordinary matter, inner ring) and unpaired non-actualisations (dark matter, outer shells).}
\label{fig:priority_existence}
\end{figure*}

