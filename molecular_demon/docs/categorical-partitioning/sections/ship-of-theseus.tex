\section{Identity Persistence Under Sequential Component Exchange}
\label{sec:identity}

We analyse the thermodynamics of systems undergoing sequential component replacement. The key result is that each replacement is a partition-composition cycle that generates entropy, and accumulated entropy eventually exceeds the system's identity information content—at which point identity has been thermodynamically dissipated.

\subsection{Identity as Information}

\begin{definition}[Identity Information]
\label{def:identity_info}
The \emph{identity information} $I_{\text{id}}$ of a system $S$ is the minimum information required to distinguish $S$ from all other systems:
\begin{equation}
    I_{\text{id}}(S) = \min_{D} H(D(S))
\end{equation}
where $D$ ranges over all distinguishing functions and $H$ is Shannon entropy.
\end{definition}

\begin{theorem}[Identity Information is Finite]
\label{thm:identity_finite}
For any physical system $S$, the identity information is finite:
\begin{equation}
    I_{\text{id}}(S) < \infty
\end{equation}
\end{theorem}

\begin{proof}
A physical system occupies a bounded region of phase space with finite volume $V$. Distinguishing $S$ requires specifying its location within this volume to precision $\delta$. The number of distinguishable locations is $V/\delta^d$ where $d$ is the dimensionality.

The identity information is at most:
\begin{equation}
    I_{\text{id}}(S) \leq \ln\left(\frac{V}{\delta^d}\right) < \infty
\end{equation}
for any finite precision $\delta > 0$.
\end{proof}

\begin{definition}[Identity Entropy]
\label{def:identity_entropy}
The \emph{identity entropy} of a system is:
\begin{equation}
    S_{\text{id}} = \kB I_{\text{id}}
\end{equation}
This is the thermodynamic entropy associated with the system's distinguishability.
\end{definition}

\subsection{Component Replacement as Partition-Composition}

\begin{definition}[Component Replacement]
\label{def:replacement}
A \emph{component replacement} operation on system $S$ consists of:
\begin{enumerate}[(i)]
    \item Partition: Remove component $c_{\text{old}}$ from $S$, creating $S' = S \setminus \{c_{\text{old}}\}$
    \item Composition: Add component $c_{\text{new}}$ to $S'$, creating $S'' = S' \cup \{c_{\text{new}}\}$
\end{enumerate}
\end{definition}

\begin{theorem}[Replacement Generates Entropy]
\label{thm:replacement_entropy}
Each component replacement generates entropy:
\begin{equation}
    \Delta S_{\text{replacement}} = S_{\text{partition}} + S_{\text{composition}} > 0
\end{equation}
where $S_{\text{partition}}$ is the entropy of removing the old component and $S_{\text{composition}}$ is the entropy of adding the new component.
\end{theorem}

\begin{proof}
By Theorem~\ref{thm:irreversibility}, partition generates undetermined residue with positive entropy:
\begin{equation}
    S_{\text{partition}} = \kB H_{\text{boundary}}(\text{removal})
\end{equation}

The composition operation does not generate new entropy but also does not recover the partition entropy. The new component $c_{\text{new}}$ is not identical to $c_{\text{old}}$, so additional distinguishing information is required:
\begin{equation}
    S_{\text{composition}} = \kB I(c_{\text{new}} \neq c_{\text{old}})
\end{equation}

The total entropy generated per replacement is:
\begin{equation}
    \Delta S_{\text{replacement}} = S_{\text{partition}} + S_{\text{composition}} > 0
\end{equation}
\end{proof}

\subsection{Cumulative Identity Loss}

\begin{theorem}[Cumulative Entropy from Sequential Replacements]
\label{thm:cumulative}
After $n$ component replacements, the cumulative entropy generated is:
\begin{equation}
    S_{\text{cumulative}}(n) = \sum_{i=1}^{n} \Delta S_i = n \cdot \langle \Delta S \rangle
\end{equation}
where $\langle \Delta S \rangle$ is the average entropy per replacement.
\end{theorem}

\begin{proof}
Each replacement $i$ generates entropy $\Delta S_i$. These contributions are additive because each replacement is an independent operation on the current state of the system. The cumulative entropy is the sum over all replacements.

If all replacements are statistically similar, then $\langle \Delta S \rangle = \Delta S_{\text{replacement}}$ is constant, and:
\begin{equation}
    S_{\text{cumulative}}(n) = n \cdot \Delta S_{\text{replacement}}
\end{equation}
\end{proof}

\begin{theorem}[Identity Dissipation Threshold]
\label{thm:threshold}
The original identity of system $S$ is thermodynamically dissipated when the cumulative replacement entropy exceeds the identity entropy:
\begin{equation}
    S_{\text{cumulative}}(n^*) \geq S_{\text{id}}(S)
\end{equation}
The threshold number of replacements is:
\begin{equation}
    n^* = \frac{S_{\text{id}}(S)}{\langle \Delta S \rangle} = \frac{I_{\text{id}}(S)}{\langle \Delta I \rangle}
\end{equation}
\end{theorem}

\begin{proof}
Identity information $I_{\text{id}}(S)$ is the total information required to distinguish the original system $S$ from all others. Each replacement dissipates some of this information—the information about the original components is lost to undetermined residue.

When the cumulative information loss equals the total identity information:
\begin{equation}
    n \cdot \langle \Delta I \rangle = I_{\text{id}}(S)
\end{equation}
the system no longer contains sufficient information to be identified as the original $S$. From a thermodynamic perspective, the identity has been completely dissipated.

Solving for $n$:
\begin{equation}
    n^* = \frac{I_{\text{id}}(S)}{\langle \Delta I \rangle}
\end{equation}
\end{proof}

\subsection{The Vagueness of Identity Boundaries}

\begin{theorem}[Edge Indeterminacy of Identity]
\label{thm:identity_edge}
The boundary at which original identity is lost is fundamentally vague, with uncertainty:
\begin{equation}
    \Delta n \geq \frac{\kB T}{|\Delta S_{\text{replacement}}|}
\end{equation}
\end{theorem}

\begin{proof}
By Theorem~\ref{thm:edge_indeterminacy}, partition boundaries have irreducible uncertainty due to partition lag. The identity threshold $n^*$ is itself determined by a partition process—the conceptual division between ``same identity'' and ``different identity.''

The uncertainty in this boundary is proportional to thermal fluctuations:
\begin{equation}
    \Delta n \cdot \Delta S_{\text{replacement}} \geq \kB T
\end{equation}

Solving:
\begin{equation}
    \Delta n \geq \frac{\kB T}{|\Delta S_{\text{replacement}}|}
\end{equation}

This uncertainty is irreducible—there is no sharp boundary between ``same system'' and ``different system.''
\end{proof}

\subsection{Case Study: Sequential Plank Replacement}

Consider a wooden structure (e.g., a vessel) composed of $N$ planks. Each plank is sequentially replaced over time.

\begin{theorem}[Vessel Identity Analysis]
\label{thm:vessel}
For a vessel with $N$ planks, each carrying identity fraction $1/N$:
\begin{enumerate}[(i)]
    \item After replacing $k$ planks, the fractional identity remaining is $(N-k)/N$
    \item The identity entropy remaining is $S_{\text{id}} \cdot (N-k)/N$
    \item Complete replacement ($k = N$) dissipates all original identity entropy
\end{enumerate}
\end{theorem}

\begin{proof}
Assume identity is uniformly distributed among planks (each plank carries $I_{\text{id}}/N$ identity information). After replacing $k$ planks:
\begin{itemize}
    \item $N - k$ original planks remain, carrying $(N-k)/N \cdot I_{\text{id}}$ identity
    \item $k$ new planks carry zero original identity
\end{itemize}

The fractional identity is:
\begin{equation}
    f(k) = \frac{N - k}{N}
\end{equation}

When $k = N$ (all planks replaced):
\begin{equation}
    f(N) = 0
\end{equation}

All original identity has been dissipated.
\end{proof}

\begin{theorem}[Gradual vs. Sudden Replacement]
\label{thm:gradual}
Gradual replacement (one plank at a time) and sudden replacement (all planks at once) yield the same final identity entropy, but gradual replacement distributes the identity loss over time while sudden replacement concentrates it.
\end{theorem}

\begin{proof}
Let $S_0 = S_{\text{id}}(S)$ be the original identity entropy.

\textbf{Gradual replacement}: After $k$ replacements, identity entropy remaining is:
\begin{equation}
    S_k = S_0 \cdot \frac{N - k}{N}
\end{equation}
At $k = N$, $S_N = 0$.

\textbf{Sudden replacement}: All $N$ planks are replaced simultaneously. Identity entropy immediately drops to:
\begin{equation}
    S_{\text{sudden}} = 0
\end{equation}

Both processes result in zero remaining identity entropy. The difference is the time profile of the loss, not the final state.
\end{proof}

\subsection{The Two-Vessel Problem}

Consider a scenario where the replaced planks are preserved and reassembled into a second structure.

\begin{theorem}[Conservation of Identity Entropy]
\label{thm:conservation}
When replaced components are preserved and reassembled, the total identity entropy is conserved:
\begin{equation}
    S_{\text{id}}(\text{modified}) + S_{\text{id}}(\text{reassembled}) + S_{\text{dissipated}} = S_{\text{id}}(\text{original})
\end{equation}
\end{theorem}

\begin{proof}
The original identity entropy $S_{\text{id}}(\text{original})$ cannot be created or destroyed—only redistributed.

After complete replacement:
\begin{itemize}
    \item Modified structure: Contains new components, zero original identity
    \item Reassembled structure: Contains original components, partial original identity
    \item Environment: Contains dissipated entropy from partition boundaries
\end{itemize}

The sum is conserved:
\begin{equation}
    0 + S_{\text{reassembled}} + S_{\text{dissipated}} = S_{\text{id}}(\text{original})
\end{equation}

The reassembled structure has identity $S_{\text{reassembled}} = S_{\text{id}}(\text{original}) - S_{\text{dissipated}}$, which is less than the original due to partition entropy losses during removal and reassembly.
\end{proof}

\begin{corollary}[Neither Vessel is Identical to Original]
\label{cor:neither}
After complete replacement with reassembly:
\begin{enumerate}[(i)]
    \item Modified vessel: $S_{\text{id}} = 0$ (contains no original identity)
    \item Reassembled vessel: $S_{\text{id}} < S_{\text{original}}$ (contains most but not all)
\end{enumerate}
Neither vessel is fully identical to the original.
\end{corollary}

\begin{figure*}[htbp]
\centering
\includegraphics[width=0.95\textwidth]{figures/ship_theseus_panel.png}
\caption{\textbf{Identity Persistence Under Sequential Component Exchange.} \textbf{(A)} Sequential component exchange: each replacement cycle consists of partition (remove old component) and composition (add new component), each generating entropy. \textbf{(B)} Hardware-measured identity dissipation: identity remaining fraction decays exponentially with number of exchanges; 50\% threshold crossed after sufficient replacements. \textbf{(C)} Cumulative entropy from exchanges: monotonically increasing, verifying Second Law—each cycle irreversibly dissipates identity information. \textbf{(D)} Entropy per exchange cycle: each replacement generates measurable positive entropy. \textbf{(E)} Identity-entropy relationship: identity decays as $I \propto e^{-S/k_B}$, showing thermodynamic relationship between information and entropy. \textbf{(F)} Connection to Ship of Theseus: the paradox dissolves when identity is recognized as finite information that is progressively dissipated through sequential partition-composition cycles—no sharp boundary exists, only gradual transition governed by $S_{\text{cumulative}}/S_{\text{id}}$.}
\label{fig:ship_theseus}
\end{figure*}

\subsection{Resolution of the Traditional Puzzle}

\begin{remark}[Historical Note]
This analysis provides the thermodynamic structure underlying the Ship of Theseus paradox. The traditional question—``If every plank is replaced, is it the same ship?''—dissolves under thermodynamic analysis:
\begin{itemize}
    \item Identity is information, which has finite quantity $I_{\text{id}}$
    \item Each replacement dissipates identity information (entropy to environment)
    \item After sufficient replacements, no original identity remains
    \item The boundary is fuzzy due to edge indeterminacy at partition boundaries
\end{itemize}
The question ``is it the same ship?'' presupposes a sharp identity boundary. The thermodynamic answer: identity is progressively dissipated through sequential partition-composition cycles, with no sharp threshold but a gradual transition from ``same'' to ``different'' governed by the ratio $S_{\text{cumulative}}/S_{\text{id}}$.
\end{remark}

