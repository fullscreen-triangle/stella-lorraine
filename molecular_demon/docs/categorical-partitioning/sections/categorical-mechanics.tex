\section{Entropy from Categorical Mechanics}
\label{sec:categorical}

We derive entropy from first principles of categorical structure, making no reference to oscillatory dynamics or partition operations. The derivation rests solely on the mathematics of distinguishable states in structured spaces.

\subsection{Axioms of Categorical Spaces}

\begin{axiom}[Categorical Distinguishability]
\label{axiom:distinguishable}
A \emph{categorical state} is a configuration that can be distinguished from all other configurations by an observer with access to the relevant observables. Two states $C$ and $C'$ are categorically distinct if and only if there exists an observable $\mathcal{O}$ such that $\mathcal{O}(C) \neq \mathcal{O}(C')$.
\end{axiom}

\begin{axiom}[Dimensional Structure]
\label{axiom:dimensional}
Categorical space admits decomposition into $M$ orthogonal dimensions. Each dimension represents an independent axis along which categorical distinctions can be made:
\begin{equation}
    \mathcal{C} = \mathcal{C}_1 \times \mathcal{C}_2 \times \cdots \times \mathcal{C}_M
\end{equation}
where $\times$ denotes the Cartesian product.
\end{axiom}

\begin{axiom}[Finite Resolution]
\label{axiom:resolution}
Each dimension $\mathcal{C}_i$ admits a finite number $n_i$ of distinguishable levels. This finiteness reflects the physical limitation that infinite precision is impossible in any physical measurement or observation.
\end{axiom}

\begin{definition}[Categorical Space]
\label{def:cat_space}
A \emph{categorical space} is the tuple $(\mathcal{C}, M, \{n_i\})$ where:
\begin{itemize}
    \item $\mathcal{C}$ is the set of all categorical states
    \item $M$ is the number of categorical dimensions
    \item $n_i$ is the number of distinguishable levels in dimension $i$
\end{itemize}
\end{definition}

\subsection{Structure of Categorical State Space}

\begin{theorem}[Cardinality of Categorical Space]
\label{thm:cardinality}
For a categorical space with $M$ dimensions, each with $n$ distinguishable levels, the total number of categorical states is:
\begin{equation}
    |\mathcal{C}| = n^M
\end{equation}
\end{theorem}

\begin{proof}
By Axiom~\ref{axiom:dimensional}, categorical space is the Cartesian product of $M$ factor spaces. By Axiom~\ref{axiom:resolution}, each factor space $\mathcal{C}_i$ has cardinality $n_i = n$ (assuming uniform resolution). The cardinality of a Cartesian product is the product of the cardinalities:
\begin{equation}
    |\mathcal{C}| = |\mathcal{C}_1| \times |\mathcal{C}_2| \times \cdots \times |\mathcal{C}_M| = n \times n \times \cdots \times n = n^M
\end{equation}
\end{proof}

\begin{definition}[Tri-Dimensional Categorical Space]
\label{def:tri_dim}
A categorical space is \emph{tri-dimensional} if it admits decomposition into exactly three orthogonal factor spaces:
\begin{equation}
    \mathcal{C} = \mathcal{C}_k \times \mathcal{C}_t \times \mathcal{C}_e
\end{equation}
where:
\begin{itemize}
    \item $\mathcal{C}_k$ is the \emph{knowledge dimension}, parametrising distinctions based on informational content
    \item $\mathcal{C}_t$ is the \emph{temporal dimension}, parametrising distinctions based on causal ordering
    \item $\mathcal{C}_e$ is the \emph{entropy dimension}, parametrising distinctions based on configurational multiplicity
\end{itemize}
\end{definition}

The tri-dimensional structure is not arbitrary but reflects the three-dimensionality of physical space. Categorical distinctions are ultimately grounded in spatial distinctions, and spatial distinctionens can be made along three independent axes.

\subsection{Recursive Self-Similarity}

\begin{axiom}[Recursive Decomposition]
\label{axiom:recursive}
Every categorical space admits recursive decomposition: each factor space $\mathcal{C}_i$ is itself a categorical space admitting the same dimensional structure.
\end{axiom}

\begin{theorem}[Recursive Self-Similarity]
\label{thm:recursive}
Under Axiom~\ref{axiom:recursive}, categorical space at depth $k$ has cardinality:
\begin{equation}
    |\mathcal{C}^{(k)}| = n^{Mk}
\end{equation}
where $M$ is the number of dimensions and $n$ is the branching factor per dimension.
\end{theorem}

\begin{proof}
At depth $k = 1$, the categorical space has cardinality $|\mathcal{C}^{(1)}| = n^M$ by Theorem~\ref{thm:cardinality}.

At depth $k = 2$, each of the $n^M$ states at level 1 admits decomposition into $n^M$ sub-states. The total cardinality is:
\begin{equation}
    |\mathcal{C}^{(2)}| = (n^M)^M = n^{M \cdot M} = n^{2M}
\end{equation}

By induction, at depth $k$:
\begin{equation}
    |\mathcal{C}^{(k)}| = n^{kM}
\end{equation}
\end{proof}

For tri-dimensional space ($M = 3$) with ternary branching ($n = 3$), this yields the characteristic $3^{3k} = 27^k$ growth.

\subsection{Derivation of Categorical Entropy}

\begin{theorem}[Categorical Entropy]
\label{thm:cat_entropy}
For a categorical space with $M$ dimensions and $n$ distinguishable levels per dimension, the entropy is:
\begin{equation}
    \boxed{\Scat = \kB M \ln n}
\end{equation}
\end{theorem}

\begin{proof}
The total number of distinguishable categorical states is $|\mathcal{C}| = n^M$ (Theorem~\ref{thm:cardinality}). If all categorical states are equally accessible—the condition of maximum categorical entropy—then the probability of occupying any particular state is:
\begin{equation}
    p_i = \frac{1}{|\mathcal{C}|} = \frac{1}{n^M}
\end{equation}

The Shannon entropy of this uniform distribution is:
\begin{equation}
    H = -\sum_{i=1}^{|\mathcal{C}|} p_i \ln p_i = -\sum_{i=1}^{n^M} \frac{1}{n^M} \ln \frac{1}{n^M} = \ln(n^M) = M \ln n
\end{equation}

Converting to thermodynamic entropy:
\begin{equation}
    \Scat = \kB H = \kB M \ln n
\end{equation}
\end{proof}

\begin{remark}[Physical Interpretation]
The entropy $\Scat = \kB M \ln n$ has the following interpretation:
\begin{itemize}
    \item $M$ counts the number of independent categorical dimensions
    \item $n$ counts the number of distinguishable levels per dimension
    \item $\ln n$ is the information capacity (in nats) per dimension
    \item $\kB$ converts to thermodynamic units (J/K)
\end{itemize}
\end{remark}

\subsection{Categorical Completion and Entropy Increase}

\begin{definition}[Categorical Completion]
\label{def:completion}
A categorical state $C$ is \emph{completed} at time $t$ if it has been distinguished from all other states by some observation prior to $t$. The set of completed states at time $t$ is denoted $\gamma(t)$.
\end{definition}

\begin{theorem}[Entropy Increases with Completion]
\label{thm:entropy_completion}
Categorical entropy increases monotonically with the number of completed categorical states:
\begin{equation}
    \frac{d\Scat}{d|\gamma|} > 0
\end{equation}
\end{theorem}

\begin{proof}
The categorical entropy of the completed portion of categorical space is:
\begin{equation}
    \Scat(t) = \kB \ln |\gamma(t)|
\end{equation}
Since $|\gamma(t)|$ is monotonically increasing (completed states cannot be "un-completed"), and $\ln$ is a monotonically increasing function:
\begin{equation}
    \frac{d\Scat}{dt} = \kB \frac{1}{|\gamma(t)|} \frac{d|\gamma(t)|}{dt} > 0
\end{equation}
provided that categorical completion continues ($d|\gamma|/dt > 0$).
\end{proof}

\subsection{Independence from Oscillatory and Partition Concepts}

The derivation of $\Scat = \kB M \ln n$ relies solely on:
\begin{enumerate}
    \item Categorical distinguishability (Axiom~\ref{axiom:distinguishable})
    \item Dimensional structure (Axiom~\ref{axiom:dimensional})
    \item Finite resolution (Axiom~\ref{axiom:resolution})
    \item Boltzmann-Shannon entropy relation $S = \kB \ln W$
\end{enumerate}

No reference has been made to oscillatory dynamics, phase space trajectories, or partition operations. The entropy arises purely from counting distinguishable categorical configurations.

