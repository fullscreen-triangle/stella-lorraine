\section{Finite Geometric Partitioning of Aggregate Properties}
\label{sec:aggregate}

We analyse the thermodynamics of partitioning systems that possess \emph{aggregate properties}—properties of the whole that are not distributed among the parts. The key result is that partition operations generate entropy that accounts for the ``disappearance'' of aggregate properties when wholes are divided into parts.

\subsection{Aggregate Properties}

\begin{definition}[Aggregate Property]
\label{def:aggregate}
A property $P$ is an \emph{aggregate property} of system $W$ if:
\begin{enumerate}[(i)]
    \item $P(W) \neq 0$ (the whole possesses the property)
    \item $P(w_i) = 0$ for all parts $w_i$ when $W$ is partitioned (no part possesses the property)
    \item $\sum_i P(w_i) \neq P(W)$ (the property is not additive)
\end{enumerate}
\end{definition}

\begin{example}[Examples of Aggregate Properties]
\label{ex:aggregate}
\begin{enumerate}
    \item \textbf{Acoustic intensity}: A mass $M$ produces sound intensity $I$ upon impact. Individual grains produce negligible sound.
    \item \textbf{Structural integrity}: A bridge supports load $L$. Individual atoms cannot support macroscopic loads.
    \item \textbf{Collective behaviour}: A flock exhibits coordinated motion. Individual birds do not exhibit ``flocking.''
    \item \textbf{Threshold properties}: A heap of sand is a ``heap.'' Individual grains are not ``heaps.''
\end{enumerate}
\end{example}

\subsection{Partition of Systems with Aggregate Properties}

\begin{theorem}[Aggregate Property Loss]
\label{thm:aggregate_loss}
When a system $W$ with aggregate property $P$ is partitioned into $n$ parts $\{w_1, \ldots, w_n\}$, the property $P$ is transferred to the undetermined residue:
\begin{equation}
    P(W) = \sum_{i=1}^{n} P(w_i) + P(\mathcal{U})
\end{equation}
where $P(\mathcal{U})$ is the property content of the undetermined residue.
\end{theorem}

\begin{proof}
By conservation, the property $P$ cannot be destroyed—only redistributed. Before partition, $P$ resides entirely in the whole $W$: $P_{\text{total}} = P(W)$.

After partition, $P$ must be distributed among:
\begin{itemize}
    \item The parts: $\sum_i P(w_i)$
    \item The undetermined residue: $P(\mathcal{U})$
\end{itemize}

By Definition~\ref{def:aggregate}, $P(w_i) = 0$ for all parts. Therefore:
\begin{equation}
    P(W) = \sum_{i=1}^{n} 0 + P(\mathcal{U}) = P(\mathcal{U})
\end{equation}

The entire property $P$ has been transferred to the undetermined residue.
\end{proof}

\subsection{Entropy of Aggregate Property Loss}

\begin{theorem}[Entropy Cost of Aggregate Property Loss]
\label{thm:entropy_aggregate}
The entropy generated when partitioning a system with aggregate property $P$ is:
\begin{equation}
    \Delta S_P = \kB \ln\left( \frac{W_P}{W_0} \right)
\end{equation}
where $W_P$ is the number of configurations consistent with possessing $P$, and $W_0$ is the number of configurations of parts lacking $P$.
\end{theorem}

\begin{proof}
Before partition, the system occupies one of $W_P$ configurations that collectively possess property $P$. After partition, the parts occupy one of $W_0$ configurations, none of which possess $P$.

The entropy change is:
\begin{equation}
    \Delta S = S_{\text{after}} - S_{\text{before}} = \kB \ln W_0 - \kB \ln W_P = -\kB \ln\left(\frac{W_P}{W_0}\right)
\end{equation}

But the Second Law requires $\Delta S_{\text{total}} \geq 0$. The resolution is that the ``missing'' entropy resides in the undetermined residue:
\begin{equation}
    S_{\text{residue}} = \kB \ln\left(\frac{W_P}{W_0}\right)
\end{equation}

The total entropy increases by $\Delta S_P = S_{\text{residue}} > 0$.
\end{proof}

\subsection{Case Study: Mass and Acoustic Intensity}

Consider a mass $M$ that produces acoustic intensity $I$ when dropped from height $h$. We partition $M$ into $N$ grains of mass $m_i = M/N$.

\begin{theorem}[Acoustic Intensity as Aggregate Property]
\label{thm:acoustic}
The acoustic intensity $I(M)$ produced by mass $M$ is an aggregate property:
\begin{equation}
    I(M) > \sum_{i=1}^{N} I(m_i)
\end{equation}
The difference is accounted for by partition entropy.
\end{theorem}

\begin{proof}
Acoustic intensity scales with the coherence of the impact. A unified mass $M$ produces a single coherent pressure wave. When partitioned into $N$ grains:
\begin{itemize}
    \item Each grain impacts at slightly different times (temporal decoherence)
    \item Each grain impacts at slightly different locations (spatial decoherence)
    \item The pressure waves partially cancel through destructive interference
\end{itemize}

The acoustic intensity of a coherent impact is:
\begin{equation}
    I_{\text{coherent}} \propto M^2
\end{equation}

The acoustic intensity of $N$ incoherent impacts is:
\begin{equation}
    I_{\text{incoherent}} \propto N \cdot \left(\frac{M}{N}\right)^2 = \frac{M^2}{N}
\end{equation}

The ratio is:
\begin{equation}
    \frac{I_{\text{coherent}}}{I_{\text{incoherent}}} = N
\end{equation}

The ``missing'' intensity corresponds to entropy:
\begin{equation}
    \Delta S_{\text{acoustic}} = \kB \ln N
\end{equation}

This entropy is generated by the partition operation—it resides in the temporal and spatial decoherence introduced when the unified mass is divided into grains.
\end{proof}

\subsection{Case Study: Threshold Properties}

Consider a collection of $N$ elements that collectively possesses a threshold property $P$ (such as ``being a heap'') that no individual element possesses.

\begin{theorem}[Threshold Property Entropy]
\label{thm:threshold}
The entropy cost of eliminating a threshold property through partition is:
\begin{equation}
    \Delta S_{\text{threshold}} = \kB \ln\left(\frac{W_{\text{above}}}{W_{\text{below}}}\right)
\end{equation}
where $W_{\text{above}}$ is the number of configurations above threshold and $W_{\text{below}}$ is the number below.
\end{theorem}

\begin{proof}
The threshold property $P$ exists when the system is in one of $W_{\text{above}}$ configurations—those with sufficient elements, organisation, or coherence to exceed the threshold. Below threshold, there are $W_{\text{below}}$ configurations.

Partition reduces the system from above-threshold to below-threshold configurations. The entropy change is:
\begin{equation}
    \Delta S = \kB \ln W_{\text{below}} - \kB \ln W_{\text{above}}
\end{equation}

If $W_{\text{below}} > W_{\text{above}}$ (more ways to be disorganised than organised), then $\Delta S > 0$: partition increases entropy, as required by the Second Law.

The threshold property is not destroyed but transferred to undetermined residue—it becomes part of the boundary entropy that cannot be recovered.
\end{proof}

\subsection{Non-Recovery of Aggregate Properties}

\begin{theorem}[Composition Cannot Recover Aggregate Properties]
\label{thm:non_recovery}
Composition of parts cannot recover aggregate properties lost to partition:
\begin{equation}
    P(\text{Compose}(\{w_1, \ldots, w_n\})) < P(W)
\end{equation}
The inequality is strict whenever $P(\mathcal{U}) > 0$.
\end{theorem}

\begin{proof}
Let $W$ have aggregate property $P(W)$. Partition creates parts $\{w_1, \ldots, w_n\}$ with $P(w_i) = 0$ and residue $\mathcal{U}$ with $P(\mathcal{U}) = P(W)$.

Compose the parts: $W' = \text{Compose}(\{w_1, \ldots, w_n\})$.

The composed system $W'$ is constructed only from the parts $\{w_i\}$. The residue $\mathcal{U}$ is not included—it was lost during partition and is thermodynamically inaccessible.

Since $P$ was entirely in $\mathcal{U}$ and $\mathcal{U} \not\subseteq W'$:
\begin{equation}
    P(W') = P(\text{Compose}(\{w_i\})) = \sum_i P(w_i) = 0 < P(W)
\end{equation}

The aggregate property cannot be recovered.
\end{proof}

\begin{figure*}[htbp]
\centering
\includegraphics[width=0.95\textwidth]{figures/heap_paradox_panel.png}
\caption{\textbf{Finite Geometric Partitioning of Aggregate Properties.} \textbf{(A)} Collective property $P(\text{Whole})$: an aggregate (heap) produces measurable property (sound) that exists only for the whole. \textbf{(B)} After partition: $P(\text{Unit}) = 0$—individual units lack the collective property entirely. \textbf{(C)} Hardware-measured partition entropy: entropy generated scales with number of units, matching theory $S \propto k_B \ln(N)$. \textbf{(D)} Thermodynamic equation: partition transfers collective property to undetermined residue; composition cannot decrease entropy and therefore cannot recover property. \textbf{(E)} Why composition fails: the missing information ($\Delta S_{\text{lag}}$ worth of coherence) was dissipated during partition; Second Law forbids recovery. \textbf{(F)} Connection to classical paradox: the Sorites/Heap paradox dissolves when ontological direction is corrected—heaps are primary, grains are derived by partition, ``heap-ness'' is dissipated as entropy.}
\label{fig:heap_paradox}
\end{figure*}

\subsection{Resolution of the Traditional Puzzle}

The analysis above resolves a traditional puzzle in natural philosophy. Consider the following:

\begin{quote}
\emph{A single grain produces no sound upon falling. Adding one grain to a soundless collection does not create sound. Yet a thousand grains produce sound. How can sound emerge from the accumulation of individually soundless elements?}
\end{quote}

The puzzle assumes composition: starting from grains (no sound), combining them to form a mass (sound), asking how sound ``emerges.''

The thermodynamic resolution reverses the direction:
\begin{enumerate}
    \item The mass with sound exists \emph{first}—it is the primordial entity
    \item Partition creates the individual grains
    \item Sound is transferred to undetermined residue during partition
    \item Composition cannot recover the sound because residue is inaccessible
\end{enumerate}

Sound does not ``emerge'' from grains. Rather, \emph{silence} is created from sound by partition. The question ``how do silences combine to make sound?'' is malformed—silences do not combine to make sound; partition creates silence from sound.

\begin{remark}[Historical Note]
This analysis provides the thermodynamic structure underlying the classical Millet Paradox, attributed to Zeno of Elea. The paradox dissolves when the ontological direction is corrected: wholes with aggregate properties are primary; parts lacking those properties are derived through partition, with the property lost to undetermined residue.
\end{remark}

\begin{remark}[The Sorites Paradox]
The same analysis applies to the Sorites Paradox (Paradox of the Heap). The question ``when do grains become a heap?'' presupposes that grains are primary and heaps are composed. The thermodynamic resolution: heaps are primary categorical entities; grains are created by partition; the ``heap'' property is lost to boundary entropy. The vagueness of ``heap'' reflects edge indeterminacy at partition boundaries.
\end{remark}

