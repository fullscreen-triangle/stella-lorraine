\section{Entropy from Oscillatory Mechanics}
\label{sec:oscillatory}

We derive entropy from first principles of oscillatory dynamics, making no reference to categorical structure or partition operations. The derivation rests solely on the physics of bounded oscillating systems.

\subsection{Axioms of Oscillatory Systems}

\begin{axiom}[Boundedness]
\label{axiom:bounded}
Physical systems occupy bounded regions of phase space. For any system with generalised coordinates $\{q_i\}$ and momenta $\{p_i\}$, there exist finite bounds:
\begin{equation}
    |q_i| \leq Q_{\max}, \quad |p_i| \leq P_{\max}
\end{equation}
for all degrees of freedom $i$.
\end{axiom}

\begin{axiom}[Nonlinear Coupling]
\label{axiom:nonlinear}
Physical systems exhibit nonlinear coupling between degrees of freedom. The Hamiltonian contains interaction terms:
\begin{equation}
    H = \sum_i H_i(q_i, p_i) + \sum_{i < j} V_{ij}(q_i, q_j)
\end{equation}
where $V_{ij} \neq 0$ for at least some pairs $(i, j)$.
\end{axiom}

\begin{theorem}[Bounded Systems Oscillate]
\label{thm:bounded_oscillate}
Every dynamical system satisfying Axioms~\ref{axiom:bounded} and~\ref{axiom:nonlinear} exhibits oscillatory behaviour in phase space.
\end{theorem}

\begin{proof}
Let $(X, d)$ be the bounded phase space with finite diameter $\text{diam}(X) = R < \infty$, and let $T: X \to X$ be the time evolution map. By boundedness, any orbit $\{T^n(x_0)\}_{n=0}^{\infty}$ is contained within $X$. By the Bolzano-Weierstrass theorem, every bounded sequence in finite-dimensional space possesses at least one accumulation point.

For the system to possess stable fixed points, we require solutions to $x^* = T(x^*)$. In systems with nonlinear coupling, such fixed points are generically unstable or absent. The absence of stable fixed points, combined with boundedness, precludes monotonic convergence to equilibrium.

By Poincaré's recurrence theorem, for any measurable set $A \subset X$ with positive measure $\mu(A) > 0$, almost every point in $A$ returns to $A$ infinitely often under $T$. Combined with the absence of stable fixed points, this necessitates oscillatory behaviour: the system repeatedly traverses regions of phase space without settling into static configurations.
\end{proof}

\subsection{Oscillatory Mode Structure}

\begin{definition}[Oscillatory Mode]
\label{def:mode}
An \emph{oscillatory mode} is an independent degree of freedom characterised by a frequency $\omega_i$ and amplitude $A_i$. For a system with $M$ modes, the state is specified by the vector of mode amplitudes:
\begin{equation}
    \mathbf{A} = (A_1, A_2, \ldots, A_M)
\end{equation}
\end{definition}

\begin{definition}[Quantum Oscillator States]
\label{def:quantum_states}
For quantum mechanical systems, each oscillatory mode $i$ admits discrete energy levels:
\begin{equation}
    E_{n_i} = \hbar \omega_i \left( n_i + \frac{1}{2} \right)
\end{equation}
where $n_i \in \{0, 1, 2, \ldots, n_{\max}\}$ is the quantum number and $n_{\max}$ is determined by the energy available to the mode.
\end{definition}

At temperature $T$, the equipartition theorem yields average energy per mode:
\begin{equation}
    \langle E_i \rangle = \kB T
\end{equation}
The maximum quantum number accessible at temperature $T$ is therefore:
\begin{equation}
    n_{\max} \approx \frac{\kB T}{\hbar \omega}
\end{equation}
for modes with characteristic frequency $\omega$.

\subsection{Derivation of Oscillatory Entropy}

\begin{theorem}[Oscillatory Entropy]
\label{thm:osc_entropy}
For a system with $M$ oscillatory modes, each admitting $n$ distinguishable states, the entropy is:
\begin{equation}
    \boxed{\Sosc = \kB M \ln n}
\end{equation}
\end{theorem}

\begin{proof}
The state of the system is specified by the vector of quantum numbers $\mathbf{n} = (n_1, n_2, \ldots, n_M)$, where each $n_i \in \{0, 1, \ldots, n-1\}$. The total number of distinguishable configurations is:
\begin{equation}
    W_{\text{osc}} = \prod_{i=1}^{M} n = n^M
\end{equation}

By Boltzmann's relation, the entropy is:
\begin{equation}
    \Sosc = \kB \ln W_{\text{osc}} = \kB \ln(n^M) = \kB M \ln n
\end{equation}
\end{proof}

\begin{remark}[Physical Interpretation]
The entropy $\Sosc = \kB M \ln n$ has the following interpretation:
\begin{itemize}
    \item $M$ counts the number of independent oscillatory degrees of freedom
    \item $n$ counts the number of distinguishable states per degree of freedom
    \item $\ln n$ is the information content (in natural units) per mode
    \item $\kB$ converts to thermodynamic units (J/K)
\end{itemize}
The entropy increases linearly with the number of modes and logarithmically with the number of states per mode.
\end{remark}

\subsection{Temperature Dependence}

\begin{corollary}[Temperature Scaling]
\label{cor:temp_scaling}
For harmonic oscillators at temperature $T$ with characteristic frequency $\omega$, the oscillatory entropy scales as:
\begin{equation}
    \Sosc = \kB M \ln\left( \frac{\kB T}{\hbar \omega} \right)
\end{equation}
in the high-temperature limit $\kB T \gg \hbar \omega$.
\end{corollary}

\begin{proof}
At temperature $T$, the number of accessible states per mode is $n \approx \kB T / \hbar \omega$. Substituting into Theorem~\ref{thm:osc_entropy}:
\begin{equation}
    \Sosc = \kB M \ln\left( \frac{\kB T}{\hbar \omega} \right)
\end{equation}
This recovers the classical result for the entropy of $M$ harmonic oscillators.
\end{proof}

\begin{remark}[Quantum Corrections]
At low temperatures $\kB T \lesssim \hbar \omega$, quantum corrections become significant. The full quantum expression is:
\begin{equation}
    \Sosc = \kB \sum_{i=1}^{M} \left[ \frac{\hbar \omega_i / \kB T}{e^{\hbar \omega_i / \kB T} - 1} - \ln\left(1 - e^{-\hbar \omega_i / \kB T}\right) \right]
\end{equation}
However, the functional form $S \propto M$ persists across all temperature regimes.
\end{remark}

\subsection{Independence from Categorical and Partition Concepts}

The derivation of $\Sosc = \kB M \ln n$ relies solely on:
\begin{enumerate}
    \item Boundedness of phase space (Axiom~\ref{axiom:bounded})
    \item Nonlinear coupling (Axiom~\ref{axiom:nonlinear})
    \item Quantum discretisation of energy levels (Definition~\ref{def:quantum_states})
    \item Boltzmann's entropy relation $S = \kB \ln W$
\end{enumerate}

No reference has been made to categorical structure, partition operations, or information-theoretic concepts. The entropy arises purely from counting distinguishable oscillatory configurations.

