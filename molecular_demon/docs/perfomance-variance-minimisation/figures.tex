\begin{figure*}[htbp]
\centering
\includegraphics[width=\textwidth]{figures/chartset3_mechanism.png}
\caption{
\textbf{Mechanism revealed: From \ce{O2} consumption to consciousness through oscillatory hole completion.}
\textbf{(Panel A)} \ce{O2} configuration around hole showing 3D distribution of $\sim$50 oxygen molecules (spheres) surrounding central hole (red star) in space (X, Y, Z in Ångströms, $-4$ to $+4$ Å range). Color indicates distance from hole (1--6 Å scale, purple to yellow). Molecules cluster in shell at $\sim$3 Å (teal-green, $\sim$30 molecules) with annotation ``Completion frequency: $\sim$5--6 Hz'', indicating oxygen binding/unbinding cycles create oscillatory holes at this rate.
\textbf{(Panel B)} \ce{VO2} $\rightarrow$ Completion Frequency showing linear relationship (blue fitted line with shaded confidence interval): $f = k \times \text{\ce{VO2}}$ where $k = 0.24$ Hz per \%. Baseline conditions (Benzos, red circle at 100\% \ce{VO2}, 20 Hz) anchor the relationship. Cocaine (red circle at $\sim$130\% \ce{VO2}, 40 Hz) and Exercise (red circle at 400\% \ce{VO2}, 95 Hz) demonstrate that completion frequency scales linearly with oxygen consumption. The tight linear fit validates the metabolic-oscillatory coupling.
\textbf{(Panel C)} Frequency $\rightarrow$ Subjective Time showing inverse relationship between completion frequency and perceived time duration. High Frequency (240 Hz, green ticks): Many ``ticks'' $\rightarrow$ Time feels SLOWER $\rightarrow$ 60s feels like 240s. Normal Frequency (60 Hz, yellow ticks): Normal ``ticks'' $\rightarrow$ Time feels NORMAL $\rightarrow$ 60s feels like 60s. Low Frequency (15 Hz, red ticks): Few ``ticks'' $\rightarrow$ Time feels FASTER $\rightarrow$ 60s feels like 15s. Mechanism annotation: ``Each completion = one `tick' of subjective time. More completions/second = slower perceived time.''
\textbf{(Panel D)} Multi-Scale Integration showing hierarchical cascade from physical to phenomenal: Molecular level (blue box): \ce{O2} consumption 250--1000 mL/min drives $\rightarrow$ Cellular level (green box): Completion frequency 60--240 Hz $\rightarrow$ Neural level (tan box): CFF / RT 60--240 Hz / 2--6 ms $\rightarrow$ Perceptual level (pink box): Subjective time 60--240s perceived $\rightarrow$ Behavioral level (purple box): Reports / Actions (variable). Bottom annotation: ``Complete Causal Chain: \ce{O2} $\rightarrow$ Frequency $\rightarrow$ Perception.'' Arrows show unidirectional causation from physical to phenomenal.
}
\label{fig:mechanism}
\end{figure*}


\begin{figure*}[htbp]
\centering
\includegraphics[width=\textwidth]{figures/figure3_oscillatory_coupling.png}
\caption{
\textbf{Multi-scale oscillatory coupling integrates biochemical, neural, mechanical, and biomechanical systems.}
\textbf{(A)} Biochemical scale ($0.1$--$10~\text{s}$): ATP-PCr (orange), glycolytic (yellow), total energy (red) normalized over $10~\text{s}$. Glycolytic onset at $\sim 6~\text{s}$.
\textbf{(B)} Neural scale ($40$--$50~\text{Hz}$ firing): Oscillations at $45~\text{Hz}$, zoom $5.0$--$5.5~\text{s}$.
\textbf{(C)} Mechanical scale ($4.5~\text{Hz}$ stride): Ground contact (green) and vertical oscillation (orange) over $4.0$--$5.2~\text{s}$.
\textbf{(D)} Biomechanical scale ($25~\text{Hz}$ contraction): Muscle force at $25~\text{Hz}$, zoom $5.0$--$5.2~\text{s}$.
\textbf{(E)} Coupled system output showing performance envelope over $10~\text{s}$ with optimal coupling zone (yellow) at $0$--$4~\text{s}$.
\textbf{(F)} Horizontal velocity profile stable at mean $= 12.0~\text{m/s}$ over $10~\text{s}$.
\textbf{(G)} Multi-scale frequency spectrum (log scale) showing peaks at biochemical ($0.1~\text{Hz}$), mechanical ($4.5~\text{Hz}$), biomechanical ($25~\text{Hz}$), neural ($45~\text{Hz}$).
\textbf{(H)} Phase coupling ($5:1$ ratio) between stride ($4.5~\text{Hz}$, orange) and muscle ($25~\text{Hz}$, green) over $5.0$--$6.0~\text{s}$.
\textbf{(I)} Distance-time profile reaching $100~\text{m}$ at finish time $9.86~\text{s}$.
\textbf{(J)} Oscillatory coupling efficiency maintaining $0.4$--$0.5$ over $10~\text{s}$.
\textbf{(K)} Architecture diagram showing four scales converging to coupled performance $= 9.57 \pm 0.03~\text{s}$.
}
\label{fig:oscillatory_coupling}
\end{figure*}


\begin{figure*}[htbp]
\centering
\includegraphics[width=\textwidth]{figures/figure_resonance_quality_analysis.png}
\caption{
\textbf{Resonance quality as a quantitative measure of consciousness states.}
\textbf{(Panel A)} 3D resonance space showing heart rate ($2.1$--$2.6~\text{Hz}$), restoration time ($0.0$--$1.0~\text{ms}$), and resonance quality ($0.3$--$1.0$) axes. Green points indicate high resonance (optimal coupling), transitioning through yellow/orange to red (low resonance). Annotation: ``High resonance $=$ Green points (optimal coupling).''
\textbf{(Panel B)} Resonance quality time series over $120$ beats showing oscillations with mean $= 0.574$, high resonance $= 5.6\%$, std $= 0.199$. Blue trace oscillates $0.3$--$1.0$, red trend line (window $n = 20$) stable at $\sim 0.6$. Three regions: high resonance ($> 0.9$, green zone), medium ($> 0.5$, yellow), low ($< 0.1$, red).
\textbf{(Panel C)} Resonance quality distribution across consciousness states showing violin plots for six states: Coma ($\sim 0.05$), Deep Sleep ($\sim 0.1$), Light Sleep ($\sim 0.25$), Drowsy ($\sim 0.5$), Alert ($\sim 0.65$), Peak Focus ($\sim 0.9$). Annotation: ``Resonance quality distribution defines consciousness state.'' Orange dashed line at $0.5$ marks medium resonance threshold.
\textbf{(Panel D)} 2D resonance landscape showing heart rate ($2.1$--$2.6~\text{Hz}$, x-axis) vs. restoration time ($0.2$--$1.0~\text{ms}$, y-axis) colored by mean resonance quality ($0.0$--$1.0$, red to green). Green region (upper-right) marks optimal coupling zone. Blue star indicates optimal point at $(\sim 2.5~\text{Hz}, \sim 0.5~\text{ms})$ with resonance $\sim 0.9$.
}
\label{fig:resonance_quality}
\end{figure*}


\begin{figure*}[htbp]
\centering
\includegraphics[width=\textwidth]{figures/master_figure_2_consciousness_geometry.png}
\caption{
\textbf{Consciousness geometry: Structure, states, and topological complexity.}
\textbf{(Panel A)} Consciousness manifold $|\mathbf{C}(x,y)| = ||\mathbf{P}(x,y) - \mathbf{T}(x,y)||$ showing 3D surface colored by intensity ($0.0$--$3.0$, purple to yellow). High intensity (red/yellow) indicates large P-T separation (strong consciousness). Annotation: ``High intensity (red) $=$ Large P-T separation $=$ Strong consciousness.''
\textbf{(Panel B)} Consciousness state space from coma to peak focus showing 3D scatter plot with axes: resonance quality ($0.0$--$1.0$), manifold distance ($0.0$--$1.0$), heartbeat variability ($0.0$--$1.0$). Six states: Coma (dark red, bottom-left), Deep Sleep (red-orange), Light Sleep (orange), Drowsy (yellow), Alert (green), Peak Focus (bright green, top-right). Black line shows state trajectory.
\textbf{(Panel C)} Multi-scale consciousness structure showing log-log plot: consciousness complexity ($10^{-1}$--$10^9$ bits) vs. spatial scale ($10^{-32}$--$10^{-2}~\text{m}$). Purple line connects six scales: Planck ($10^{-35}~\text{m}$), Femtometer ($1~\text{fm}$), Picometer ($1~\text{pm}$), Nanometer ($1~\text{nm}$), Micrometer ($1~\mu\text{m}$), Millimeter ($1~\text{mm}$), GPS ($5~\text{m}$). Three colored regions: quantum (pink), molecular (blue), macro (green). Annotation: ``Same geometric structure at all scales. Complexity increases with precision.''
\textbf{(Panel D)} Consciousness topology showing Betti numbers ($\beta^0$ components, $\beta^1$ loops, $\beta^2$ voids) across six states. Peak Focus has highest topology: $\beta^0 = 15$, $\beta^1 = 12$, $\beta^2 = 15$. Annotation: ``Higher consciousness $=$ Richer topology.''
}
\label{fig:consciousness_geometry}
\end{figure*}


\begin{figure*}[htbp]
\centering
\includegraphics[width=\textwidth]{figures/figure_neural_resonance_1_bands.png}
\caption{
\textbf{Neural oscillatory bands and cardiac coupling during running.}
\textbf{(Panel A)} Neural frequency bands (log scale) showing six bands: Delta ($2.2~\text{Hz}$, red), Theta ($6.0~\text{Hz}$, orange), Alpha ($10.5~\text{Hz}$, yellow), Beta ($21.5~\text{Hz}$, green), Gamma ($65.0~\text{Hz}$, blue), High-$\gamma$ ($150.0~\text{Hz}$, purple). Annotation: ``Neural oscillatory bands during running.'' Frequency axis spans $10^0$--$10^2~\text{Hz}$.
\textbf{(Panel B)} Resonance quality by band showing bar chart: Delta ($0.65$, red), Theta ($0.78$, orange), Alpha ($0.85$, yellow), Beta ($0.92$, green), Gamma ($0.88$, blue), High-$\gamma$ ($0.75$, purple). Red dashed line at $0.8$ marks high resonance threshold. Green annotation: ``Beta band shows highest resonance (motor control).''
\textbf{(Panel C)} Cardiac coupling strength vs. neural frequency (log scale) showing six bands with cardiac frequency $= 2.32~\text{Hz}$ (red vertical lines mark harmonics). Gamma shows strongest coupling ($\sim 1.0$), followed by Beta ($\sim 0.6$). Annotation: ``Cardiac freq: $2.32~\text{Hz}$. Red lines: Harmonics.''
\textbf{(Panel D)} Temporal synchronization over $2~\text{s}$ showing four waveforms: Cardiac ($2.3~\text{Hz}$, red), Alpha ($10~\text{Hz}$, yellow), Beta ($20~\text{Hz}$, green), Gamma ($40~\text{Hz}$, blue). Yellow box annotation: ``All neural bands synchronize to cardiac rhythm.''
}
\label{fig:neural_resonance}
\end{figure*}
\begin{figure*}[htbp]
\centering
\includegraphics[width=\textwidth]{figures/figure_muscle_timing.png}
\caption{
\textbf{Muscle activation timing and integrated work patterns.}
\textbf{(Panel A)} Activation threshold detection over $60~\text{s}$: Quadriceps (red), Hamstrings (blue), Gastrocnemius (green) with threshold at $0.3$ (gray dashed line). Colored circles mark activations above threshold. Annotation: ``Quad Activations: $20$, Ham Activations: $20$, Gastro Activations: $19$.''
\textbf{(Panel B)} Activation timeline showing red bars for five muscles (Tibialis, Glutes, Hip Flexors, Gastrocnemius, Hamstrings, Quadriceps) over $60~\text{s}$. Annotation: ``Red bars $=$ Active periods ($>$ threshold).'' All muscles show $\sim 20$ activation events.
\textbf{(Panel C)} Activation duration violin plots for three muscles showing distributions centered at $\sim 1500~\text{ms}$: Quadriceps (red), Hamstrings (blue), Gastrocnemius (green). Median lines at $1500~\text{ms}$, range $1500$--$1600~\text{ms}$.
\textbf{(Panel D)} Integrated activation (work) showing horizontal bars for six muscles: Tibialis Anterior ($19.97$, yellow), Quadriceps ($19.97$, green), Hamstrings ($19.97$, cyan), Glutes ($19.97$, blue), Gastrocnemius ($19.97$, purple), Hip Flexors ($19.97$, dark purple). Annotation: ``Total work $= 19.97$. Integrated activation over time.''
}
\label{fig:muscle_timing}
\end{figure*}
\begin{figure*}[htbp]
\centering
\includegraphics[width=\textwidth]{figures/figure_muscle_activation.png}
\caption{
\textbf{Muscle activation dynamics and synergy patterns during running.}
\textbf{(Panel A)} Lower limb muscle activation over $60~\text{s}$: Quadriceps (red), Hamstrings (blue), Gastrocnemius (green). All oscillate $0.0$--$0.7$ with mean $= 0.333$. Annotation: ``Quad Mean: $0.333$, Hamstring Mean: $0.333$, Gastro Mean: $0.333$, Duration: $59.9~\text{s}$.''
\textbf{(Panel B)} Upper limb and core muscles: Hip Flexors (red), Glutes (orange), Tibialis Anterior (green). All show periodic activation $0.0$--$0.7$ with mean $= 0.333$.
\textbf{(Panel C)} Muscle correlation matrix showing six muscles (Quadriceps, Hamstrings, Gastrocnemius, Hip Flexors, Glutes, Tibialis) with correlation values ($-1.00$ to $+1.00$, blue to red). Strong positive correlations (red, $+1.00$) between synergistic pairs; strong negative (blue, $-1.00$) between antagonists.
\textbf{(Panel D)} Synergy activation over $60~\text{s}$: Extensor (red), Flexor (blue), Co-activation (purple fill). Annotation: ``Extensor Mean: $0.333$, Flexor Mean: $0.333$, Co-activation: $0.221$. Higher co-activation $=$ Greater stability.''
}
\label{fig:muscle_activation}
\end{figure*}
\begin{figure*}[htbp]
\centering
\includegraphics[width=\textwidth]{figures/figure_joint_kinematics.png}
\caption{
\textbf{Joint angle kinematics during running reveal cyclic coordination patterns.}
\textbf{(Panel A)} Lower limb joint angles over $60~\text{s}$: Hip (blue, $0$--$130^\circ$), Knee (red, $-25$--$50^\circ$), Ankle (green, $0$--$25^\circ$). Annotation: ``Hip Range: $77.0^\circ$, Knee Range: $71.5^\circ$, Ankle Range: $33.0^\circ$, Duration: $59.9~\text{s}$.'' All three show periodic oscillations with $\sim 20$ cycles.
\textbf{(Panel B)} Upper limb joint angles: Shoulder (orange, $80$--$110^\circ$), Elbow (purple, $-40$--$60^\circ$). Annotation: ``Shoulder Range: $89.0^\circ$, Elbow Range: $33.0^\circ$, Arm Swing Amplitude.'' Both oscillate synchronously over $60~\text{s}$.
\textbf{(Panel C)} Angular velocity time series for three joints over $60~\text{s}$: Hip (red), Knee (blue), Ankle (green) spanning $-150$ to $+150~\text{deg/s}$. Annotation: ``Hip Max Vel: $80.2~\text{deg/s}$, Knee Max Vel: $148.9~\text{deg/s}$, Ankle Max Vel: $34.4~\text{deg/s}$.''
\textbf{(Panel D)} Phase space plot showing knee angle ($70$--$140^\circ$, x-axis) vs. angular velocity ($-150$ to $+150~\text{deg/s}$, y-axis) colored by time ($0$--$60~\text{s}$, purple to yellow). Green circle marks start, red square marks end. Annotation: ``Phase space shows knee joint dynamics (cyclic pattern).''
}
\label{fig:joint_kinematics}
\end{figure*}
\begin{figure*}[htbp]
\centering
\includegraphics[width=\textwidth]{figures/figure_joint_angles_3_frequency.png}
\caption{
\textbf{Joint angle frequency analysis reveals temporal modulation patterns.}
\textbf{(Panel A)} Power spectral density showing three joints: Knee (blue), Elbow (orange), Ankle (purple) over $0$--$10~\text{Hz}$. Knee shows dominant peak at $\sim 1~\text{Hz}$ with power $\sim 10^5$. Annotation: ``Frequency Resolution: $0.01667~\text{Hz}$. Nyquist Frequency: $5.00~\text{Hz}$.'' Power decays from $10^5$ to $10^{-13}$ across spectrum.
\textbf{(Panel B)} Time-frequency evolution heatmap for knee joint over $50~\text{s}$ showing frequency ($0$--$5~\text{Hz}$, y-axis) vs. time (x-axis) colored by power ($-60$ to $+20~\text{dB}$, blue to yellow). Annotation: ``Knee joint time-frequency evolution.'' Strong bands at $\sim 0.5~\text{Hz}$ and $\sim 2.5~\text{Hz}$.
\textbf{(Panel C)} Instantaneous frequency showing Hilbert transform for knee (cyan) and ankle (dark blue) over $60~\text{s}$. Both oscillate $0.5$--$1.0~\text{Hz}$ with periodic modulation. Annotation: ``Hilbert transform reveals temporal frequency modulation.''
\textbf{(Panel D)} Coherence spectrum showing purple envelope with red circle at peak: $0.60~\text{Hz}$, coherence $= 0.021$. Yellow annotation box. Red dashed line marks threshold ($0.5$).
}
\label{fig:joint_frequency}
\end{figure*}
\begin{figure*}[htbp]
\centering
\includegraphics[width=\textwidth]{figures/figure_heartbeat_unified_framework.png}
\caption{
\textbf{Heartbeat-gas-BMD unified framework: Equilibrium restoration drives perception.}
\textbf{(Panel A)} Gas molecular equilibrium time series over $8~\text{s}$ showing oscillations between $0.65$--$1.05$ with perfect equilibrium at $1.00$ (green line). Red dashed vertical lines mark heartbeats. Blue shaded region shows equilibrium envelope. Annotation: ``Heart Rate: $2.32~\text{Hz}$, RR Interval: $431.1~\text{ms}$, Restoration: $0.502~\text{ms}$, Red lines $=$ Heartbeats.''
\textbf{(Panel B)} Restoration time distribution showing histogram with mean $= 0.502~\text{ms}$, max $= 0.999~\text{ms}$, Gaussian fit. Peak frequency $\sim 7$ at $0.5~\text{ms}$. Black curve shows distribution envelope spanning $0.0$--$1.0~\text{ms}$.
\textbf{(Panel C)} Log-scale comparison showing three bars: Heart Rate ($2.32~\text{Hz}$, red, $\sim 10^1$), Perception Rate ($1993~\text{Hz}$, blue, $\sim 10^3$), Frames per Heartbeat ($859.3$, purple, $\sim 10^3$). Green annotation: ``KEY INSIGHT: $859$ perception frames between heartbeats. Resonance Quality: $1.000$.''
\textbf{(Panel D)} Restoration time variability over $120$ beats showing scatter plot colored by restoration time ($0.0$--$1.0~\text{ms}$, purple to yellow). Red line shows rolling average (window $n = 10$) oscillating $0.3$--$0.6~\text{ms}$ around mean $= 0.502~\text{ms}$ (blue dashed line). Annotation: ``Restoration time varies with each heartbeat.''
}
\label{fig:heartbeat_framework}
\end{figure*}
\begin{figure*}[htbp]
\centering
\includegraphics[width=\textwidth]{figures/figure_gps_thought_geometry.png}
\caption{
\textbf{Multi-scale thought geometry from GPS to attosecond precision.}
\textbf{(Panel A)} Thought manifold at GPS scale showing 3D space with planning (acceleration, $0.0$--$1.0$), prediction (jerk, $0.0$--$1.0$), and decision (direction change, $0.0$--$1.0$) axes. Points colored by jerk intensity ($-2.5$ to $+1.0$, blue to red). Annotation: ``Thought $=$ Planning $+$ Prediction $+$ Decision.'' Green cluster indicates stable cognitive state; red points mark high-intensity decisions.
\textbf{(Panel B)} Thought manifold at attosecond scale showing same 3D structure with enhanced precision. Dark purple cluster reveals quantum thought structure. Annotation: ``Attosecond precision reveals quantum thought structure.''
\textbf{(Panel C)} Thought complexity time series over normalized time ($0.0$--$1.0$) showing oscillations with mean $= 0.704$, max $= 1.414$, decisions $= 19$. Purple envelope with red trend line (window $n = 20$). Black stars mark complexity peaks at decision moments. Annotation: ``Complexity peaks $=$ Decision moments.''
\textbf{(Panel D)} Thought volume across spatial scales showing bar chart: GPS ($0.25$, red), ns ($0.25$, orange), ps ($0.25$, yellow), fs ($0.25$, green), as ($0.25$, cyan), zs ($0.25$, blue), Planck ($0.25$, purple), Trans-Planck ($0.20$, pink). Right axis shows mean complexity ($0.0$--$0.7$). Annotation: ``Thought volume expands with precision. More precision $=$ Richer cognitive structure.''
}
\label{fig:thought_geometry}
\end{figure*}


\begin{figure*}[htbp]
\centering
\includegraphics[width=\textwidth]{figures/figure_gait_cycle_analysis.png}
\caption{
\textbf{Comprehensive gait cycle analysis from running biomechanics.}
\textbf{(Panel A)} Knee angle oscillations over time showing $20$ gait cycles at stride frequency $f = 0.33~\text{Hz}$ (period $T = 3.0~\text{s}$). Blue trace oscillates between $20^\circ$--$90^\circ$ with regular periodicity. Red dashed line indicates mean knee angle $= 55.0^\circ$. Annotation: ``Knee angle oscillations: $20$ cycles, $f = 0.33~\text{Hz}$.''
\textbf{(Panel B)} Joint angle trajectories through normalized gait cycle ($0.0$--$1.0$) showing hip (blue, range $30^\circ$--$107^\circ$), knee (orange, range $20^\circ$--$91^\circ$), and ankle (green, range $75^\circ$--$108^\circ$). Vertical gray line at $0.5$ marks mid-cycle. Annotation: ``Hip-Knee-Ankle coordination through gait cycle.''
\textbf{(Panel C)} Hip-knee coordination plot showing cyclic coupling pattern. Hip angle ($30^\circ$--$107^\circ$, x-axis) vs. knee angle ($20^\circ$--$91^\circ$, y-axis) colored by gait cycle phase ($0.0$--$1.0$, purple to yellow). Elliptical trajectory indicates coordinated joint motion. Annotation: ``Hip-Knee coordination: Cyclic coupling pattern.''
\textbf{(Panel D)} Range of motion comparison across joints showing bar chart: Hip ($77.0^\circ$, blue), Knee ($71.5^\circ$, orange), Ankle ($33.0^\circ$, green). Annotation: ``ROM: Hip $> $ Knee $> $ Ankle.''
}
\label{fig:gait_cycle}
\end{figure*}

\begin{figure*}[htbp]
\centering
\includegraphics[width=\textwidth]{figures/figure_atmospheric_garmin_analysis.png}
\caption{
\textbf{Atmospheric displacement and energy transfer from running activity.}
\textbf{(Panel A)} Cumulative volume and mass over time ($0$--$60~\text{min}$) showing volume (blue, left axis, $0$--$800~\text{m}^3$) and mass (red, right axis, $0$--$1000~\text{kg}$). Final values: volume $= 686.36~\text{m}^3$, mass $= 840.79~\text{kg}$. Annotation: ``Total displaced: $686.36~\text{m}^3$, $840.79~\text{kg}$.''
\textbf{(Panel B)} Molecular scale comparison showing runner silhouette (height $1.75~\text{m}$) with magnified inset ($4000\times$ enhancement) revealing molecular-scale air displacement ($\sim 0.4~\text{mm}$ region). Blue dots represent air molecules. Annotation: ``$4000\times$ enhancement reveals molecular displacement.''
\textbf{(Panel C)} Wake boundary analysis showing runner profile with turbulent wake region (blue shading). Reynolds number $\text{Re} = 376{,}384$ (turbulent regime). Wake extends $823.1~\text{m}$ behind runner. Annotation: ``Reynolds $= 376{,}384$, Wake $= 823.1~\text{m}$.''
\textbf{(Panel D)} Energy transfer calculation showing kinetic energy ($7378.5~\text{J}$, blue bar) converting to thermal energy with temperature rise $\Delta T = 8.75~\text{mK}$ (red bar, right axis $0$--$10~\text{mK}$). Annotation: ``$7378.5~\text{J} \rightarrow 8.75~\text{mK}$ temperature rise.''
}
\label{fig:atmospheric_analysis}
\end{figure*}

\begin{figure*}[htbp]
\centering
\includegraphics[width=\textwidth]{figures/figure_consciousness_geometry.png}
\caption{
\textbf{Consciousness as geometric residual between perception and thought manifolds.}
\textbf{(Panel A)} 3D perception-thought curves showing perception manifold (blue surface, spatial $0.0$--$1.0$ and motion $0.0$--$1.0$ axes) and thought manifold (red surface, planning $0.0$--$1.0$ and prediction $0.0$--$1.0$ axes). Vertical gap represents consciousness as geometric residual. Annotation: ``Consciousness $=$ Geometric residual between perception and thought.''
\textbf{(Panel B)} Consciousness manifold visualization showing 3D surface colored by consciousness intensity ($0.0$--$1.0$, blue to yellow). Spatial awareness ($0.0$--$1.0$, x-axis), motion awareness ($0.0$--$1.0$, y-axis), consciousness intensity (z-axis, $0.0$--$1.0$). Yellow peaks indicate high consciousness regions. Annotation: ``Consciousness manifold: Intensity varies with perception-thought coupling.''
\textbf{(Panel C)} Temporal consciousness metrics over normalized time ($0.0$--$1.0$) showing consciousness intensity (purple trace, oscillating $0.0$--$1.0$) with mean $= 0.500$, max $= 1.000$, peaks $= 8$. Red trend line (window $n = 20$) stable at $\sim 0.5$. Black stars mark consciousness peaks. Annotation: ``Peaks $=$ Moments of acute awareness.''
\textbf{(Panel D)} Volume-intensity relationship showing consciousness volume across precision scales: GPS ($0.25$, red), ns ($0.25$, orange), ps ($0.25$, yellow), fs ($0.25$, green), as ($0.25$, cyan), zs ($0.25$, blue), Planck ($0.25$, purple), Trans-Planck ($0.20$, pink). Right axis shows mean intensity ($0.0$--$0.5$). Annotation: ``Volume constant, intensity varies with precision.''
}
\label{fig:consciousness_geometry}
\end{figure*}

\begin{figure*}[htbp]
\centering
\includegraphics[width=\textwidth]{figures/figure_perception_geometry.png}
\caption{
\textbf{Multi-scale perception geometry from GPS to femtosecond precision.}
\textbf{(Panel A)} GPS-level perception manifold showing 3D space with spatial awareness ($0.0$--$1.0$), motion awareness ($0.0$--$1.0$), and perception intensity ($0.0$--$1.0$) axes. Points colored by intensity ($0.0$--$1.0$, blue to red). Red cluster indicates high perception intensity. Annotation: ``Perception $=$ Spatial awareness $+$ Motion awareness.''
\textbf{(Panel B)} Femtosecond-precision perception structure showing same 3D manifold with enhanced resolution. Deep purple cluster reveals quantum-scale perception structure. Annotation: ``Femtosecond precision reveals quantum perception structure.''
\textbf{(Panel C)} Curvature analysis showing perception curvature over normalized time ($0.0$--$1.0$). Purple trace oscillates $0.0$--$1.4$ with mean $= 0.704$, max $= 1.414$, peaks $= 19$. Red trend line (window $n = 20$) at $\sim 0.7$. Black stars mark curvature peaks. Annotation: ``High curvature $=$ Attention focus points.''
\textbf{(Panel D)} Perception volume across scales showing bar chart: GPS ($0.25$, red), ns ($0.25$, orange), ps ($0.25$, yellow), fs ($0.25$, green), as ($0.25$, cyan), zs ($0.25$, blue), Planck ($0.25$, purple), Trans-Planck ($0.20$, pink). Right axis shows mean curvature ($0.0$--$0.7$). Annotation: ``Perception volume expands with precision. More precision $=$ Richer perceptual structure.''
}
\label{fig:perception_geometry}
\end{figure*}

\begin{figure*}[htbp]
\centering
\includegraphics[width=\textwidth]{figures/figure_gps_precision_cascade.png}
\caption{
\textbf{GPS precision cascade: Same physical path across four temporal scales.}
\textbf{(Panel A)} Millisecond precision showing trajectory in 3D space (x: $-50$--$50~\text{m}$, y: $-50$--$50~\text{m}$, z: $0$--$100~\text{m}$) colored by time ($0$--$60~\text{min}$, purple to yellow). Uncertainty $\sim \text{mm}$ scale. Annotation: ``Millisecond precision: $\sim \text{mm}$ uncertainty.''
\textbf{(Panel B)} Picosecond precision showing same trajectory with enhanced detail. Uncertainty $\sim \text{pm}$ scale. Path structure reveals finer oscillations. Annotation: ``Picosecond precision: $\sim \text{pm}$ uncertainty.''
\textbf{(Panel C)} Attosecond precision showing trajectory with quantum-scale resolution. Uncertainty $\sim \text{am}$ scale. Deep purple coloring indicates early time points. Annotation: ``Attosecond precision: $\sim \text{am}$ uncertainty.''
\textbf{(Panel D)} Trans-Planckian precision showing trajectory beyond Planck scale. Uncertainty $< $ Planck length. Maximum resolution reveals fundamental structure. Annotation: ``Trans-Planckian: Sub-Planckian uncertainty.''
All panels share same spatial extent but reveal progressively finer structure with increasing temporal precision.
}
\label{fig:gps_cascade}
\end{figure*}

\begin{figure*}[htbp]
\centering
\includegraphics[width=\textwidth]{figures/figure_brain_wave_oscillatory_analysis.png}
\caption{
\textbf{Comprehensive brain wave oscillatory analysis with cross-frequency coupling.}
\textbf{(Panel A)} EEG signal over $10$ seconds showing raw amplitude oscillations ($-100$--$+100~\mu\text{V}$). Black trace exhibits high-frequency components with regular envelope modulation. Annotation: ``EEG Signal (10 seconds).''
\textbf{(Panel B)} Power spectral density showing frequency content ($0$--$100~\text{Hz}$, x-axis) with PSD ($10^{-1}$--$10^3~\mu\text{V}^2/\text{Hz}$, log scale, y-axis). Blue trace shows dominant peaks at low frequencies with harmonics. Legend indicates six bands: delta (purple), theta (green), alpha (pink), beta (red), gamma (cyan), high\_gamma (yellow).
\textbf{(Panel C)} Brain wave band powers showing delta dominance ($42.8\%$, purple bar), followed by beta ($21.9\%$, teal), theta ($15.5\%$, dark blue), alpha ($13.3\%$, cyan), gamma ($5.7\%$, green), high\_gamma ($0.3\%$, yellow). Annotation: ``Brain Wave Band Powers.''
\textbf{(Panel D)} Frequency components over $5$ seconds showing decomposed bands: delta (blue, $0$--$50~\mu\text{V}$), theta (orange, $50$--$100~\mu\text{V}$), alpha (green, $100$--$150~\mu\text{V}$), beta (red, $150$--$200~\mu\text{V}$). Each band shows characteristic oscillation frequency.
\textbf{(Panel E)} Cross-frequency coupling strength showing four coupling types: theta\_gamma\_pac ($0.012$), alpha\_beta\_coupling ($0.011$), delta\_theta\_coupling ($0.016$), gamma\_coherence ($0.053$, dominant, red bar). Annotation: ``Cross-Frequency Coupling, $0.053$.''
\textbf{(Panel F)} Gamma oscillations over $2$ seconds showing high-frequency activity ($-30$--$+30~\mu\text{V}$, red trace) with rapid oscillations ($\sim 40~\text{Hz}$). Annotation: ``Gamma Oscillations (2 seconds).''
\textbf{(Panel G)} Alpha-beta interaction showing envelope dynamics over $10$ seconds. Orange trace (alpha envelope, $35$--$47$) and blue trace (beta envelope, $10$--$28$) exhibit anti-phase relationship. Annotation: ``Alpha envelope, Beta envelope.''
\textbf{(Panel H)} Theta-gamma phase-amplitude coupling with modulation index MI $= 0.012$. Histogram shows mean gamma amplitude ($0$--$25$) vs. theta phase ($-3$--$+3$ radians). Blue bars show weak but consistent coupling with peak at $\sim -1$ radian.
\textbf{(Panel I)} Validation summary showing red box with status: FAIL. Three criteria: Alpha Dominance $= 13.3\%$ (Expected: $20$--$40\%$), Theta-Gamma PAC $= 0.012$ (Threshold: $\geq 0.1$), Alpha-Beta Coupling $= 0.011$ (Expected: $-0.7$ to $-0.2$). Annotation: ``BRAIN WAVE VALIDATION.''
}
\label{fig:brain_wave_analysis}
\end{figure*}

\begin{figure*}[htbp]
\centering
\includegraphics[width=\textwidth]{figures/figure_muscle_comparison.png}
\caption{
\textbf{Multi-scale coupling effects on muscle force generation and activation dynamics.}
\textbf{(Panel A)} Muscle force comparison over $3.5$ seconds showing with coupling (blue, peak $\sim 6000~\text{N}$) vs. classical (red dashed, peak $\sim 7000~\text{N}$). Coupling model shows gradual rise ($0.5$--$1.0~\text{s}$) and plateau ($1.0$--$2.5~\text{s}$) with smooth decay. Classical model exhibits sharper transitions. Annotation: ``Muscle Force.''
\textbf{(Panel B)} Muscle activation comparison showing activation level ($0.0$--$1.0$) over time. Blue trace (with coupling) and red dashed (classical) nearly overlap, both showing rapid rise at $0.5~\text{s}$, plateau at $1.0$, and decay at $2.5~\text{s}$. Annotation: ``Muscle Activation.''
\textbf{(Panel C)} Inter-scale coupling evolution showing average coupling strength ($0.0$--$0.6$) over time. Green trace shows rapid rise to $0.6$ at $0.5~\text{s}$, plateau at $0.55$ until $1.0~\text{s}$, then gradual decay to $\sim 0.05$ by $3.5~\text{s}$. Annotation: ``Avg Coupling Strength.''
\textbf{(Panel D)} State space trajectory in 3D showing knowledge dimension ($0.0$--$1.0$, x-axis), time dimension ($0.0$--$1.3$, y-axis), and entropy ($0.000$--$0.025$, z-axis). Blue trajectory starts at green dot (origin), spirals upward through middle region, ends at red square (upper-right). Black dots mark intermediate states. Annotation: ``Start, End.''
\textbf{(Panel E)} Final coupling matrix heatmap showing coupling strength between five scales: Tiss, Neur, Neur, Card, Loco (both axes). Black horizontal band at Neur-Neur shows strong coupling ($\sim 0.030$). Other regions show weak coupling ($\sim 0.010$, yellow). Color scale: yellow ($0.010$) to black ($0.030$). Annotation: ``Coupling Strength.''
\textbf{(Panel F)} Coupling effect on force showing force difference ($0$ to $-2000~\text{N}$) over time. Purple trace shows sharp drop to $-2300~\text{N}$ at $0.5~\text{s}$, gradual recovery to $-1200~\text{N}$ at $1.0$--$2.5~\text{s}$, then return to $0~\text{N}$ by $3.0~\text{s}$. Gray dashed line at $0~\text{N}$. Annotation: ``Force Difference (N).''
}
\label{fig:muscle_comparison}
\end{figure*}

\begin{figure*}[htbp]
\centering
\includegraphics[width=\textwidth]{figures/figure_frequency_decomposition.png}
\caption{
\textbf{Multi-scale frequency decomposition of muscle force signal across physiological bands.}
\textbf{(Panel A)} Original force signal over $5$ seconds showing complex oscillatory pattern. Black trace oscillates $0$--$6000~\text{N}$ with decreasing amplitude envelope and multiple frequency components. Annotation: ``Original Force Signal.''
\textbf{(Panel B)} Rhythmic excitation pattern showing normalized excitation ($0.0$--$1.0$) over $5$ seconds. Blue trace exhibits regular oscillations with period $\sim 0.5~\text{s}$ (frequency $\sim 2~\text{Hz}$). Peaks reach $1.0$, troughs reach $0.0$. Annotation: ``Rhythmic Excitation.''
\textbf{(Panel C)} Tissue-scale component ($0.0$--$10.0~\text{Hz}$) showing slow oscillations. Light blue trace ranges $-2000$--$+3000$ amplitude with period $\sim 1~\text{s}$. Largest amplitude at $t = 0$. Annotation: ``Tissue (0.0-10.0 Hz).''
\textbf{(Panel D)} Neural-scale component ($1.0$--$100.0~\text{Hz}$) showing faster oscillations. Light blue trace ranges $-3000$--$+3000$ amplitude with multiple frequency components. Dominant period $\sim 0.5~\text{s}$. Annotation: ``Neural (1.0-100.0 Hz).''
\textbf{(Panel E)} Neuromuscular component ($0.0$--$20.0~\text{Hz}$) showing intermediate-frequency oscillations. Light blue trace ranges $-2000$--$+3000$ amplitude. Large initial transient at $t = 0$, followed by regular oscillations. Annotation: ``Neuromuscular (0.0-20.0 Hz).''
\textbf{(Panel F)} Cardiovascular component ($0.0$--$5.0~\text{Hz}$) showing slow modulation. Light blue trace ranges $-1000$--$+3000$ amplitude with period $\sim 1~\text{s}$. Smooth envelope with decreasing amplitude. Annotation: ``Cardiovascular (0.0-5.0 Hz).''
\textbf{(Panel G)} Locomotor component ($0.5$--$3.0~\text{Hz}$) showing gait-related oscillations. Light blue trace ranges $-1000$--$+3000$ amplitude with period $\sim 1~\text{s}$. Dominant low-frequency component with largest amplitude at $t = 0$. Annotation: ``Locomotor (0.5-3.0 Hz).''
}
\label{fig:frequency_decomposition}
\end{figure*}

\begin{figure*}[htbp]
\centering
\includegraphics[width=\textwidth]{figures/figure_dynamic_coupling.png}
\caption{
\textbf{Dynamic coupling evolution and phase space analysis of force-activation system.}
\textbf{(Panel A)} Force and coupling dynamics over $4$ seconds showing dual y-axes. Blue trace (left axis, $0$--$6000~\text{N}$) shows force rising to $6000~\text{N}$ at $1.0~\text{s}$, plateau until $2.5~\text{s}$, then decay. Red trace (right axis, $0.0$--$0.6$ coupling strength) shows peak at $0.6$ during rise phase ($0.5$--$1.0~\text{s}$), then gradual decay to $\sim 0.05$. Annotation: ``Force and Coupling Dynamics.''
\textbf{(Panel B)} 3D state space trajectory showing knowledge ($0.0$--$1.0$, x-axis), time ($0.00$--$1.50$, y-axis), and entropy ($0.000$--$0.025$, z-axis). Blue trajectory with black dots starts at green circle (origin), spirals through middle region, ends at red square (upper-right). Grid surface shows coordinate planes. Annotation: ``Start, End.''
\textbf{(Panel C)} Force-activation phase space showing force ($0$--$6000~\text{N}$, y-axis) vs. activation ($0.0$--$1.0$, x-axis). Blue trajectory forms loop starting at origin, rising vertically to peak (marked with red star at $\sim 6000~\text{N}$, activation $\sim 1.0$), then descending. Annotation: ``* Peak.''
\textbf{(Panel D)} Force vs. coupling relationship showing force ($0$--$6000~\text{N}$, y-axis) vs. coupling strength ($0.0$--$0.6$, x-axis). Trajectory colored by time ($0.0$--$4.0~\text{s}$, yellow to purple). Two distinct branches: rising phase (green-cyan, coupling $0.05$--$0.15$, force $0$--$6000~\text{N}$) and active phase (purple, coupling $\sim 0.5$, force $2000$--$6000~\text{N}$). Small purple cluster at high coupling indicates transient high-force state. Annotation: ``Force vs Coupling.''
}
\label{fig:dynamic_coupling}
\end{figure*}

\begin{figure*}[htbp]
\centering
\includegraphics[width=\textwidth]{figures/Figure9_Electron_Cascade.png}
\caption{
\textbf{Electron cascade mechanism for instantaneous molecular identification.}
\textbf{(Panel A)} Electron cascade mechanism showing membrane with embedded molecules. Top: membrane cross-section (pink ovals) with electron (blue circle with $e^-$) at $t = 0.01~\text{ns}$. Yellow arrow points to electron injection site. Three molecules identified: O$_2^-$ (pink box, left), ATP (orange box, center), Glucose (green box, right). Arrows indicate categorical signature identification pathways. Annotation: ``Electron Cascade: Instantaneous Molecular Identification $t = 0.01~\text{ns}$, Categorical Signature Identification.''
\textbf{(Panel B)} Cascade propagation time distribution showing histogram of propagation times ($0.007$--$0.014~\text{ns}$, x-axis) with frequency ($0$--$100$, y-axis). Blue bars form Gaussian distribution centered at mean $= 0.010~\text{ns}$ (red dashed line) with $\pm 1\sigma = 0.001~\text{ns}$ (orange dotted lines). Peak frequency $\sim 100$ at $0.010~\text{ns}$. Annotation: ``Cascade Propagation Time Distribution.''
\textbf{(Panel C)} Speed comparison showing two bars on log scale. Left: Electron Cascade (blue, $10^6~\text{m/s}$, labeled ``1e+06 m/s''). Right: Molecular Diffusion (gray, $10^{-6}~\text{m/s}$, labeled ``1e-06 m/s''). Red line with yellow box annotation: ``$10^{12} \times$ faster''. Speed ratio spans $12$ orders of magnitude. Annotation: ``Speed Comparison: $10^{12} \times$ Faster!''
\textbf{(Panel D)} Phase synchronization with O$_2^-$ cycle showing amplitude ($-1.0$--$+1.0~\text{a.u.}$, y-axis) vs. O$_2^-$ cycle phase ($0$--$2\pi~\text{rad}$, x-axis). Red sinusoidal curve shows O$_2^-$ oscillation. Blue circles mark electron injection events, clustered in two green-shaded optimal windows: $0$--$\pi/2$ (rising phase) and $3\pi/2$--$2\pi$ (falling phase). Annotation: ``O$_2^-$ Oscillation, Electron Injection, Optimal Window.''
\textbf{(Panel E)} Circuit completion success rate over $1000$ events showing success rate ($90$--$105\%$, y-axis) vs. event number ($0$--$100$, x-axis). Green trace maintains constant $100\%$ success (matches red dashed line labeled ``Perfect Success''). Green-shaded region ($95$--$105\%$) indicates high performance zone. Cyan box annotation: ``$100\%$ Success Rate''. Annotation: ``Circuit Completion Success Rate (1000 events).''
}
\label{fig:electron_cascade}
\end{figure*}

\begin{figure*}[htbp]
\centering
\includegraphics[width=\textwidth]{figures/Figure20_Maxwell_Demon.png}
\caption{
\textbf{Biological Maxwell demon: Information-driven energy extraction from ion gradients.}
\textbf{(Panel A)} Ion concentration gradients showing inside (blue bars) vs. outside (orange bars) concentrations (log scale, $10^{-4}$--$10^2~\text{mM}$) for five ion species. Na$^+$: inside $= 10$, outside $= 145$ (14.5$\times$ gradient). K$^+$: inside $= 140$, outside $= 5$ (0.03$\times$ gradient, reversed). Ca$^{2+}$: inside $= 0.0001$, outside $= 1.8$ (18000$\times$ gradient, largest). Mg$^{2+}$: inside $= 0.5$, outside $= 0.8$ (1.6$\times$ gradient). Cl$^-$: inside $= 10$, outside $= 110$ (11$\times$ gradient). Annotation: ``Ion Concentration Gradients - Maxwell Demon Substrate.''
\textbf{(Panel B)} Gradient strength showing concentration ratio (out/in, log scale $10^{-1}$--$10^4$) for each ion. Ca$^{2+}$ dominant (red bar, $18000.0\times$). Na$^+$ (purple, $14.5\times$), Cl$^-$ (purple, $11.0\times$), Mg$^{2+}$ (cyan, $1.6\times$), K$^+$ (yellow, $0.0\times$, reversed gradient). Annotation: ``Gradient Strength.''
\textbf{(Panel C)} Maxwell demon mechanism schematic showing membrane (gray vertical bar) with pump (yellow oval labeled ``PUMP (Demon)'') containing ATP (green oval). Left side (``INSIDE''): scattered K$^+$ (blue ovals) and Na$^+$ (red ovals). Right side (``OUTSIDE''): scattered K$^+$ and Na$^+$ ions. Green arrow indicates ATP-driven pumping. Annotation: ``Maxwell Demon Mechanism - Information-Driven Sorting.''
\textbf{(Panel D)} Pump rate distribution showing histogram of pump rates ($75$--$200~\text{ions/sec}$, x-axis) with count ($0$--$10$, y-axis). Green bars form distribution centered at mean $= 146.9~\text{ions/sec}$ (red dashed line). Peak count $= 10$ at $\sim 150~\text{ions/sec}$. Annotation: ``Pump Rate Distribution.''
\textbf{(Panel E)} Gradient energies showing $\Delta G$ ($-10$--$+30~\text{kJ/mol}$, y-axis) for five ions. Na$^+$ (purple, $\sim 7~\text{kJ/mol}$), K$^+$ (yellow, $\sim -5~\text{kJ/mol}$, negative), Ca$^{2+}$ (orange, $\sim 25~\text{kJ/mol}$, largest), Mg$^{2+}$ (cyan, $\sim 2~\text{kJ/mol}$), Cl$^-$ (purple, $\sim 6~\text{kJ/mol}$). Gray dashed line at ATP $= 30.5~\text{kJ/mol}$ shows energy source. Annotation: ``Gradient Energies, ATP: 30.5 kJ/mol.''
\textbf{(Panel F)} H$^+$ framework analogy showing two purple boxes connected to yellow central box. Top left: ``BIOLOGICAL MAXWELL DEMON - Sorts ions by type, Uses ATP energy, Creates gradients''. Top right: ``H-OSCILLATOR MAXWELL DEMON - Sorts by frequency, Uses oscillation energy, Creates patterns''. Center: ``SAME PRINCIPLE: Information $\rightarrow$ Energy Conversion, Apparent 2nd Law Violation (but not really)''. Right: white box with summary text listing physical parameters (Temperature $= 37.0^\circ\text{C}$, ATP $= 30.5~\text{kJ/mol}$), ion gradients, demon function, H$^+$ framework connection, and validation checkmarks. Annotation: ``H$^+$ Framework Analogy, MAXWELL DEMON SUMMARY.''
}
\label{fig:maxwell_demon}
\end{figure*}

\begin{figure*}[htbp]
\centering
\includegraphics[width=\textwidth]{figures/figure_hierarchical_bmd_system.png}
\caption{
\textbf{Hierarchical BMD (Biological Molecular Demon) system with observer structure and reconstruction quality.}
\textbf{(Panel A)} Observer hierarchy showing tree structure. Top: ``System Observer'' (cyan box). Three branches to: ``visual process'' (green), ``motion process'' (green), ``thermal process'' (green). Visual process branches further to: ``edge detection'' and ``color analysis''. Annotation: ``Observer Hierarchy.''
\textbf{(Panel B)} Average BMD ambiguity by level showing three bars: System (blue, $\sim 0.95$, highest), Process (avg) (green, $\sim 0.80$), Subprocess (avg) (orange, $\sim 0.80$). Y-axis: Ambiguity ($0.0$--$1.0$). Annotation: ``Average BMD Ambiguity by Level.''
\textbf{(Panel C)} Completion space size showing single purple bar for System BMD with size $= 1.00 \times 10^{10}$ (log scale, y-axis $10^9$--$10^{11}$). Annotation: ``Completion Space Size, $1.00\text{e}{+}10$.''
\textbf{(Panel D)} System BMD characteristics showing three bars: Signature Length (blue, $\sim 28$), Ambiguity (orange, $\sim 90$, dominant), Completion Space (log) (green, $\sim 10$). Y-axis: Value ($0$--$100$). Annotation: ``System BMD Characteristics.''
\textbf{(Panel E)} Bottom-up reconstruction quality showing pie chart. Green section: ``Preserved $73.5\%$'' (dominant, $\sim 260^\circ$). Gray section: ``Lost $26.5\%$'' ($\sim 100^\circ$). Annotation: ``Bottom-Up Reconstruction Quality.''
\textbf{(Panel F)} Hierarchical coherence showing single cyan bar for level\_1 with coherence score $\sim 0.07$. Y-axis: Coherence Score ($0.00$--$0.07$). X-axis shows level\_1 and level\_2 (level\_2 has no visible bar). Annotation: ``Hierarchical Coherence.''
}
\label{fig:hierarchical_bmd}
\end{figure*}

\begin{figure*}[htbp]
\centering
\includegraphics[width=\textwidth]{figures/figure_mixing_process.png}
\caption{
\textbf{Irreversible mixing process showing phase-lock network formation and entropy increase.}
\textbf{(Panel A)} Physical configuration after mixing showing 2D spatial distribution ($0.0$--$1.0$ for both x and y axes). Purple lines connect molecules forming complex network. Blue circles (``Originally from A'') and red circles (``Originally from B'') distributed throughout space with complete spatial mixing. Multiple geometric clusters visible. Purple box annotation: ``Purple lines = NEW A-B phase-lock interactions.'' Annotation: ``Physical Configuration - MIXED.''
\textbf{(Panel B)} Categorical state progression showing original container (y-axis) vs. categorical state ID ($-0.04$--$+0.04$, x-axis). Yellow background fills entire plot. Blue circles (``Originally A'') and red squares (``Originally B'') scattered across narrow horizontal band. Yellow box annotation: ``ALL states are NEW (yellow backg), C\_initial $\rightarrow$ C\_mixed''. Annotation: ``Categorical State Progression.''
\textbf{(Panel C)} Phase-lock network with A-B edges showing U-shaped network structure. Blue circles (originally A) and red circles (originally B) connected by purple lines forming dense network with $70$ edges. Network spans full spatial range with symmetric distribution. Purple box annotation: ``Purple lines = NEW A-B interactions (70 edges), These did NOT exist in separated state!'' Annotation: ``Phase-Lock Network with A-B Edges.''
\textbf{(Panel D)} New A-B interactions showing bar chart comparing before (white bars) vs. after mixing (purple bars). A-A: before $= 0$, after $= 0$. B-B: before $= 0$, after $= 0$. A-B: before $= 0$, after $= 70$ (dominant purple bar). Red box annotation: ``NEW! +70 edges''. Y-axis: Number of Phase-Lock Edges ($0$--$70$). Annotation: ``New A-B Interactions.''
\textbf{(Panel E)} Entropy increase from mixing showing text box with calculations. Before mixing: Total edges $= 0$, A-B edges $= 0$, $S_{\text{initial}} = 0.000\text{e}{+}00~\text{J/K}$. After mixing: Total edges $= 70$, A-B edges $= 70$ (NEW!), $S_{\text{mixed}} = 1.208\text{e}{-}23~\text{J/K}$. Entropy increase: $\Delta S = S_{\text{mixed}} - S_{\text{initial}} = 1.208\text{e}{-}23~\text{J/K}$, $\Delta S/k_B = 0.88$. Key insight: ``NEW phase-lock edges between originally-separated molecules create denser topological network. This is IRREVERSIBLE: once A-B phase correlations form, they persist!'' Annotation: ``Entropy Increase from Mixing.''
\textbf{(Panel F)} Mixing summary showing red box with comprehensive system state. System configuration: Molecules from A $= 0$, from B $= 0$, Partition: REMOVED, Spatial mixing: Complete. Categorical state: Previous: C\_initial ($0$ states), Current: C\_mixed ($2$ states), NEW states created: $2$, Axiom: C\_initial CANNOT be re-occupied. Phase-lock network: A-A edges $= 0$, B-B edges $= 0$, A-B edges $= 70$ (NEW!), Total edges $= 70$, Network densification: $70/0 = 7000.0\%$. Critical insight: ``The 70 new A-B phase-lock edges represent IRREVERSIBLE categorical state completion. These phase correlations persist even if we re-separate spatially!'' Annotation: ``Mixing Summary, MIXED STATE.''
}
\label{fig:mixing_process}
\end{figure*}

\begin{figure*}[htbp]
\centering
\includegraphics[width=\textwidth]{figures/chartset1_universal_law.png}
\caption{
\textbf{The universal law of temporal perception: VO$_2^-$ oscillation frequency determines subjective time across all physiological states.}
\textbf{(Panel A)} Master relationship showing perceived duration ($50$--$250$ s, y-axis) vs. VO$_2^-$ percentage of baseline ($100$--$400\%$, x-axis). Colored circles represent different conditions: cyan (resting/normal), orange (fever states), red (post-exercise, top right at $\sim 400\%$, $\sim 240$ s). Black dashed line shows linear fit with $R^2 = 1.000$, $p < 0.001$. Orange dotted horizontal line marks actual 60 s. Yellow box annotation: ``All conditions collapse onto single relationship.'' Perfect linear correlation demonstrates universal law. Annotation: ``Linear fit $R^2 = 1.000$ $p < 0.001$, Actual 60s.''
\textbf{(Panel B)} Mechanism schematic showing oscillatory hole (pink circle) with arrows pointing to ``Completions'' below. Seven blue circles with $e^-$ symbols arranged above hole. Green wave below shows $\sim 5$--$6$ Hz oscillation frequency. Yellow box annotation: ``Each completion = one 'tick' of subjective time.'' Demonstrates electron cascade completion mechanism. Annotation: ``Oscillatory Hole, Completions, $\sim 5$--$6$ Hz.''
\textbf{(Panel C)} Dynamic range showing horizontal bars for 13 conditions with VO$_2^-$ values ($0$--$400\%$ baseline). Post-Exercise (red, $\sim 380\%$, longest) marked with yellow box: ``4.7$\times$ range.'' Other conditions: Cocaine ($\sim 120\%$), High Fever 40$^\circ$C ($\sim 120\%$), Caffeine (green, $\sim 110\%$), Fever 38.5$^\circ$C (orange, $\sim 110\%$), Resting (gray, $100\%$, red dashed baseline), Age 20 (green, $\sim 105\%$), Normal 37$^\circ$C (green, $\sim 105\%$), Age 30--70 (cyan/gray, $\sim 95$--$100\%$), Hypothermia 36$^\circ$C (cyan, $\sim 95\%$), Alcohol (cyan, $\sim 90\%$), Benzos (purple, $\sim 85\%$). Annotation: ``Dynamic Range, Baseline (100\%).''
\textbf{(Panel D)} Perceptual consequences showing three measures vs. VO$_2^-$ ($100$--$400\%$). Left y-axis: Perceived Duration (red circles with line, $50$--$225$ s). Right y-axis: CFF (blue squares with dashed line, $50$--$250$ Hz) and RT scaled (green triangles with dotted line, $200$--$50$, inverted scale). All three measures scale together linearly. Yellow box annotation: ``All measures scale together.'' Demonstrates unified perceptual effects. Annotation: ``Time Perception, CFF, RT (scaled).''
}
\label{fig:universal_law}
\end{figure*}

\begin{figure*}[htbp]
\centering
\includegraphics[width=\textwidth]{figures/demo1_muscle_comparison.png}
\caption{
\textbf{Multi-scale coupling effects on muscle force generation and activation dynamics.}
\textbf{(Panel A)} Muscle force comparison over $3.5$ seconds showing with coupling (blue solid line, peak $\sim 6000$ N) vs. classical (red dashed line, peak $\sim 7000$ N). Coupling model shows gradual rise ($0.5$--$1.0$ s) and plateau ($1.0$--$2.5$ s) with smooth decay. Classical model exhibits sharper transitions and higher peak force. Annotation: ``Muscle Force, With Coupling, Classical.''
\textbf{(Panel B)} Muscle activation comparison showing activation level ($0.0$--$1.0$) over time. Blue solid trace (with coupling) and red dashed trace (classical) nearly overlap, both showing rapid rise at $0.5$ s, plateau at $1.0$, and decay at $2.5$ s. Minimal difference between models. Annotation: ``Muscle Activation, With Coupling, Classical.''
\textbf{(Panel C)} Inter-scale coupling evolution showing average coupling strength ($0.0$--$0.6$) over time. Green trace shows rapid rise to $0.6$ at $0.5$ s, plateau at $0.55$ until $1.0$ s, then gradual decay to $\sim 0.05$ by $3.5$ s. Annotation: ``Inter-Scale Coupling Evolution, Avg Coupling Strength.''
\textbf{(Panel D)} State space trajectory in 3D showing knowledge dimension ($0.0$--$1.0$, x-axis), time dimension ($0.0$--$1.5$, y-axis), and entropy ($0.000$--$0.025$, z-axis). Blue trajectory with black dots starts at green circle (origin), spirals upward through middle region, ends at red square (upper-right). Grid surface shows coordinate planes. Annotation: ``State Space Trajectory, Start, End.''
\textbf{(Panel E)} Final coupling matrix heatmap showing coupling strength between five scales: Tiss, Neur, Neur, Card, Loco (both axes). Black horizontal band at Neur-Neur shows strong coupling ($\sim 0.030$). Other regions show weak coupling ($\sim 0.010$, yellow). Color scale: yellow ($0.010$) to black ($0.030$). Annotation: ``Final Coupling Matrix, Coupling Strength.''
\textbf{(Panel F)} Coupling effect on force showing force difference ($0$ to $-2000$ N) over time. Purple trace shows sharp drop to $-2300$ N at $0.5$ s, gradual recovery to $-1200$ N at $1.0$--$2.5$ s, then return to $0$ N by $3.0$ s. Gray dashed line at $0$ N. Annotation: ``Coupling Effect on Force, Force Difference (N).''
}
\label{fig:muscle_comparison_demo}
\end{figure*}

\begin{figure*}[htbp]
\centering
\includegraphics[width=\textwidth]{figures/demo2_frequency_decomposition.png}
\caption{
\textbf{Multi-scale frequency decomposition of muscle force signal across physiological bands.}
\textbf{(Panel A)} Original force signal over $5$ seconds showing complex oscillatory pattern. Black trace oscillates $0$--$6000$ N with decreasing amplitude envelope and multiple frequency components. Annotation: ``Original Force Signal.''
\textbf{(Panel B)} Rhythmic excitation pattern showing normalized excitation ($0.0$--$1.0$) over $5$ seconds. Blue trace exhibits regular oscillations with period $\sim 0.5$ s (frequency $\sim 2$ Hz). Peaks reach $1.0$, troughs reach $0.0$. Annotation: ``Rhythmic Excitation.''
\textbf{(Panel C)} Tissue-scale component ($0.0$--$10.0$ Hz) showing slow oscillations. Light blue trace ranges $-2000$--$+3000$ amplitude with period $\sim 1$ s. Largest amplitude at $t = 0$. Annotation: ``Tissue (0.0-10.0 Hz).''
\textbf{(Panel D)} Neural-scale component ($1.0$--$100.0$ Hz) showing faster oscillations. Light blue trace ranges $-3000$--$+3000$ amplitude with multiple frequency components. Dominant period $\sim 0.5$ s. Annotation: ``Neural (1.0-100.0 Hz).''
\textbf{(Panel E)} Neuromuscular component ($0.0$--$20.0$ Hz) showing intermediate-frequency oscillations. Light blue trace ranges $-2000$--$+3000$ amplitude. Large initial transient at $t = 0$, followed by regular oscillations. Annotation: ``Neuromuscular (0.0-20.0 Hz).''
\textbf{(Panel F)} Cardiovascular component ($0.0$--$5.0$ Hz) showing slow modulation. Light blue trace ranges $-1000$--$+3000$ amplitude with period $\sim 1$ s. Smooth envelope with decreasing amplitude. Annotation: ``Cardiovascular (0.0-5.0 Hz).''
\textbf{(Panel G)} Locomotor component ($0.5$--$3.0$ Hz) showing gait-related oscillations. Light blue trace ranges $-1000$--$+3000$ amplitude with period $\sim 1$ s. Dominant low-frequency component with largest amplitude at $t = 0$. Annotation: ``Locomotor (0.5-3.0 Hz).''
}
\label{fig:frequency_decomposition_demo}
\end{figure*}

\begin{figure*}[htbp]
\centering
\includegraphics[width=\textwidth]{figures/demo3_dynamic_coupling.png}
\caption{
\textbf{Dynamic coupling evolution and phase space analysis of force-activation system.}
\textbf{(Panel A)} Force and coupling dynamics over $4$ seconds showing dual y-axes. Blue trace (left axis, $0$--$6000$ N) shows force rising to $6000$ N at $1.0$ s, plateau until $2.5$ s, then decay. Red trace (right axis, $0.0$--$0.6$ coupling strength) shows peak at $0.6$ during rise phase ($0.5$--$1.0$ s), then gradual decay to $\sim 0.05$. Annotation: ``Force and Coupling Dynamics.''
\textbf{(Panel B)} 3D state space trajectory showing knowledge ($0.0$--$1.0$, x-axis), time ($0.00$--$1.50$, y-axis), and entropy ($0.000$--$0.025$, z-axis). Blue trajectory with black dots starts at green circle (origin), spirals through middle region, ends at red square (upper-right). Grid surface shows coordinate planes. Annotation: ``3D State Space Trajectory, Start, End.''
\textbf{(Panel C)} Force-activation phase space showing force ($0$--$6000$ N, y-axis) vs. activation ($0.0$--$1.0$, x-axis). Blue trajectory forms loop starting at origin, rising vertically to peak (marked with red star at $\sim 6000$ N, activation $\sim 1.0$), then descending. Annotation: ``Force-Activation Phase Space, * Peak.''
\textbf{(Panel D)} Force vs. coupling relationship showing force ($0$--$6000$ N, y-axis) vs. coupling strength ($0.0$--$0.6$, x-axis). Trajectory colored by time ($0.0$--$4.0$ s, yellow to purple). Two distinct branches: rising phase (green-cyan, coupling $0.05$--$0.15$, force $0$--$6000$ N) and active phase (purple, coupling $\sim 0.5$, force $2000$--$6000$ N). Small purple cluster at high coupling indicates transient high-force state. Annotation: ``Force vs Coupling.''
}
\label{fig:dynamic_coupling_demo}
\end{figure*}

\begin{figure*}[htbp]
\centering
\includegraphics[width=\textwidth]{figures/figure_muscle_timing.png}
\caption{
\textbf{Muscle activation timing and patterns during locomotion showing synchronized recruitment across muscle groups.}
\textbf{(Panel A)} Muscle activation over $60$ seconds showing three muscles. Quadriceps (red trace), Hamstrings (cyan trace), Gastrocnemius (green trace) oscillate between $0.0$--$1.0$ activation with regular periodicity. Gray dashed horizontal line at threshold $= 0.3$. Colored circles at peaks indicate activation events: Quadriceps (red, 20 activations), Hamstrings (cyan, 20 activations), Gastrocnemius (green, 19 activations). White box annotation: ``Quad Activations: 20, Ham Activations: 20, Gastro Activations: 19.'' Annotation: ``Muscle Activation (0-1), Quadriceps, Hamstrings, Gastrocnemius, Threshold (0.3).''
\textbf{(Panel B)} Activity periods showing six muscle groups (Tibialis, Glutes, Hip Flexors, Gastrocnemius, Hamstrings, Quadriceps) over $60$ seconds. Red horizontal bars indicate active periods (above threshold). All muscles show regular, synchronized activation patterns with $\sim 20$ activation cycles. White box annotation: ``Red bars = Active periods ($>$ threshold).'' Annotation: ``Time (s).''
\textbf{(Panel C)} Activation duration distribution showing violin plots for three muscles. Quadriceps (red), Hamstrings (blue), Gastrocnemius (green) all centered at $\sim 1500$--$1600$ ms with narrow distributions. Black horizontal lines show median and quartiles. Minimal variation between muscles indicates consistent timing. Annotation: ``Activation Duration (ms).''
\textbf{(Panel D)} Integrated activation (work) showing horizontal bars for six muscles. All muscles show nearly identical integrated activation $\sim 19.97$ work units: Tibialis Anterior (yellow, 19.97), Quadriceps (green, 19.97), Hamstrings (cyan, 19.97), Glutes (teal, 19.97), Gastrocnemius (blue, 19.97), Hip Flexors (purple, 19.97). Yellow box annotation: ``Total work = 19.97, Integrated activation over time.'' Demonstrates balanced muscle recruitment. Annotation: ``Integrated Activation (work).''
}
\label{fig:muscle_timing}
\end{figure*}


\begin{figure*}[htbp]
\centering
\includegraphics[width=\textwidth]{figures/figure2_drug_hole_matching.png}
\caption{
\textbf{Pharmacological oscillatory hole matching across three neurotransmitter pathways.}
\textbf{(Panel A)} Inositol metabolism pathway showing three drugs with overall score (blue bars), frequency match (red bars), and pathway match (green bars). Lithium shows perfect matching: overall score $= 0.99$, frequency match $= 1.00$, pathway match $= 1.00$ (all bars at maximum). Valproate shows moderate overall score $= 0.62$ with poor frequency match $= 0.10$ but perfect pathway match $= 1.00$. Aripiprazole shows moderate overall score $= 0.59$ with poor frequency match $= 0.13$ but perfect pathway match $= 1.00$. Annotation: ``Inositol Metabolism.''
\textbf{(Panel B)} Serotonin signaling pathway showing three drugs. Citalopram demonstrates near-perfect matching: overall score $= 0.97$, frequency match $= 1.00$, pathway match $= 1.00$. Valproate shows good overall score $= 0.77$ with excellent frequency match $= 0.93$ but moderate pathway match $= 0.40$. Lorazepam shows moderate overall score $= 0.71$ with excellent frequency match $= 0.93$ but moderate pathway match $= 0.40$. Legend shows blue (Overall Score), red (Frequency Match), green (Pathway Match). Annotation: ``Serotonin Signaling, Overall Score, Frequency Match, Pathway Match.''
\textbf{(Panel C)} Dopamine signaling pathway showing three drugs with only overall scores (blue bars). Valproate shows high score $= 0.88$. Aripiprazole shows excellent score $= 0.93$. Lithium shows near-perfect score $= 0.99$ (highest). No frequency or pathway match data shown. Annotation: ``Dopamine Signaling.''
}
\label{fig:drug_hole_matching}
\end{figure*}

\begin{figure*}[htbp]
\centering
\includegraphics[width=\textwidth]{figures/master_figure_1_framework_integration.png}
\caption{
\textbf{Three-paper framework integration: From biomechanics to consciousness geometry through perception quantization.}
\textbf{(Panel A)} Thought manifold (Paper 1) showing biomechanical signatures in 3D state space. Axes: Acceleration ($-0.08$--$+0.08$ m/s$^2$, x-axis), Jerk ($-0.04$--$+0.06$ m/s$^3$, y-axis), Velocity ($0.00$--$0.20$ m/s, z-axis), Time ($0$--$10$ s, color scale purple to yellow). Green and blue dots show trajectory through state space. Two black stars mark thought events (decision moments). Yellow box annotation: ``Red stars = Thought events (Decision moments).'' Grid surface shows coordinate planes. Annotation: ``A: Thought Manifold (Paper 1) Biomechanical Signatures.''
\textbf{(Panel B)} Perception quantization (Paper 2) showing heartbeat-driven perception frames over $5$ seconds. Red sinusoidal curve shows normalized heartbeat signal ($-1.0$--$+1.0$). Thin green vertical lines mark perception frames at $2000$ Hz. Thick red vertical lines mark heartbeats at $1.2$ Hz (period $= 833$ ms). Blue trace (right y-axis, $0.91$--$0.95$ O$_2$ saturation) shows oscillatory equilibrium. Cyan box annotation: ``Heart Rate: 1.2 Hz, RR Interval: 833 ms, Restoration: 0.50 ms, Perception Rate: 2000 Hz, Red lines = Heartbeats, Green lines = Perception frames.'' Demonstrates temporal quantization mechanism. Annotation: ``B: Perception Quantization (Paper 2) Heartbeat Creates Perception Frames, Heartbeat, O Equilibrium, Heartbeat Signal, O Saturation.''
\textbf{(Panel C)} Consciousness manifold (Paper 3) showing geometric residual in 3D. Axes: Spatial Y ($-4$--$+4$), Spatial Dimension 2 ($-4$--$+4$), Velocity ($-1.0$--$+1.5$). Large red-orange hemisphere (perception manifold) overlays smaller blue hemisphere (thought manifold). Purple base region shows consciousness as residual. Green arrows point outward indicating consciousness vectors $|\mathbf{C}| = ||\mathbf{P} - \mathbf{T}||$. Color scale: purple ($0.0$) to yellow ($3.0$) consciousness intensity. Green box annotation: ``Green vectors = Consciousness $[\mathbf{C}] = ||\mathbf{P} - \mathbf{T}||$.'' Annotation: ``C: Consciousness Manifold (Paper 3) Geometric Residual, Perception Manifold, Thought Manifold, Consciousness (Residual).''
\textbf{(Panel D)} Complete framework integration flowchart. Top: Red box ``PAPER 1: Thought Validation, Sprint Running, Thought $>$ Biomechanics.'' Blue box ``PAPER 2: Anthropometric-Cardiac Oscillations, Heartbeat $\rightarrow$ Perception.'' White boxes show ``Thought Manifold'' and ``Perception Quantization'' connecting to green box ``PAPER 3: Geometry of Consciousness.'' White box ``Geometric Residual'' connects to purple box at bottom: ``UNIFIED FRAMEWORK: Consciousness = Geometric Residual between Perception and Thought, Quantized by Heartbeat | Measured via Movement.'' Annotation: ``D: Complete Framework Integration.''
}
\label{fig:framework_integration}
\end{figure*}

\begin{figure*}[htbp]
\centering
\includegraphics[width=\textwidth]{figures/master_figure_2_consciousness_geometry.png}
\caption{
\textbf{Consciousness geometry: Multi-scale structure, state space topology, and topological complexity across awareness levels.}
\textbf{(Panel A)} Consciousness manifold intensity showing $|\mathbf{C}(x,y)| = ||\mathbf{P}(x,y) - \mathbf{T}(x,y)||$ in 3D. Axes: Spatial Dimension 1 ($-4$--$+4$), Spatial Dimension 2 ($-4$--$+4$), Consciousness Intensity ($0.00$--$2.00$). Surface forms dome with peak at center (red-orange, intensity $\sim 2.0$) transitioning to purple edges (intensity $\sim 0.5$). Color scale: purple ($0.0$) to yellow ($3.0$). Yellow box annotation: ``High intensity (red) = Large P-T separation = Strong consciousness.'' Grid surface shows base plane. Annotation: ``A: Consciousness Manifold $|\mathbf{C}(x,y)| = ||\mathbf{P}(x,y) - \mathbf{T}(x,y)||$.''
\textbf{(Panel B)} Consciousness state space from coma to peak focus in 3D. Axes: Resonance Quality ($0.0$--$1.0$), Manifold Distance ($0.0$--$1.0$), Heartbeat Variability ($0.0$--$1.0$). Six consciousness states shown as colored point clouds: Coma (dark red, bottom-left, low on all dimensions), Deep Sleep (red-orange), Light Sleep (orange), Drowsy (yellow), Alert (green), Peak Focus (bright green, top-right corner, high on all dimensions). Black dashed line shows state trajectory ascending from coma to peak focus. Legend lists all states. Annotation: ``B: Consciousness State Space From Coma to Peak Focus, Coma, Deep Sleep, Light Sleep, Drowsy, Alert, Peak Focus, State Trajectory.''
\textbf{(Panel C)} Multi-scale consciousness structure showing log-log plot. Y-axis: Consciousness Complexity/Information Content ($10^{-1}$--$10^9$). X-axis: Spatial Scale ($10^{-32}$--$10^{-2}$ meters). Purple diagonal line descends from top-left (Planck scale $10^{-35}$ m) through six labeled points: Femtometer/1fm ($10^5$), Picometer/1pm ($10^3$), Nanometer/1nm ($10^1$), Micrometer/1$\mu$m ($10^{-1}$), Millimeter/1mm ($10^{-3}$), GPS/5m ($10^{-5}$). Three colored regions: purple (Quantum Scale), pink (Molecular Scale), blue (Macro Scale), green (rightmost). Green box annotation: ``Same geometric structure at all scales, Complexity increases with precision.'' Annotation: ``C: Multi-Scale Consciousness Structure, Planck ($10^{-35}$m).''
\textbf{(Panel D)} Consciousness topology showing Betti numbers across six states. Y-axis: Betti Number/Topological Features ($0$--$25$). X-axis: Consciousness State (Coma, Deep Sleep, Light Sleep, Drowsy, Alert, Peak Focus). Three bar types per state: $\beta^0$ Components (red), $\beta^1$ Loops (blue), $\beta^2$ Voids (green). Coma shows minimal topology ($\beta^0 = 1$, $\beta^1 = 2$, $\beta^2 = 1$). Peak Focus shows maximal complexity ($\beta^0 = 15$, $\beta^1 = 12$, $\beta^2 = 15$). Progressive increase through intermediate states. Yellow box annotation: ``$\beta^0$ = Connected components, $\beta^1$ = Loops/cycles, $\beta^2$ = Voids/cavities, Higher consciousness = Richer topology.'' Annotation: ``D: Consciousness Topology Topological Complexity Increases with Awareness.''
}
\label{fig:consciousness_geometry}
\end{figure*}

\begin{figure*}[htbp]
\centering
\includegraphics[width=\textwidth]{figures/maxwell_demon_results.png}
\caption{
\textbf{Maxwell's demon prisoner parable: Temperature sorting, entropy evolution, and information processing dynamics.}
\textbf{(Panel A)} Temperature evolution over $20$ time units showing two compartments. Blue trace (Compartment A) starts at $1.0$, drops sharply to $\sim 0.3$ by $t = 2$, then oscillates around $0.3$ with small fluctuations. Orange trace (Compartment B) starts at $1.0$, rises to $\sim 1.8$ by $t = 2$, then maintains plateau at $\sim 1.75$ with minor oscillations. Demonstrates successful temperature gradient creation. Annotation: ``Temperature Evolution, Compartment A, Compartment B, Temperature.''
\textbf{(Panel B)} Entropy evolution showing four components. Blue line (Compartment A, barely visible near zero), orange line (Compartment B, near zero), green line (Demon cost, near zero), and thick black line (Total) rising linearly from $0$ to $\sim 90$ entropy units. Total entropy increases monotonically, satisfying second law. Annotation: ``Entropy Evolution, Compartment A, Compartment B, Demon cost, Total, Entropy.''
\textbf{(Panel C)} Particle distribution showing number of particles ($75$--$125$) over time. Blue trace (Compartment A) starts at $\sim 100$, drops to minimum $\sim 78$ at $t = 2$, then gradually recovers to $\sim 105$ by $t = 20$. Orange trace (Compartment B) starts at $\sim 100$, rises to maximum $\sim 125$ at $t = 2$, then gradually decreases to $\sim 98$ by $t = 20$. Annotation: ``Particle Distribution, Compartment A, Compartment B, Number of Particles.''
\textbf{(Panel D)} Demon information processing showing total bits processed ($0$--$4500$) over time. Purple trace rises monotonically with slight upward curvature, reaching $\sim 4400$ bits by $t = 20$. Demonstrates continuous information acquisition. Annotation: ``Demon Information Processing, Total Bits Processed.''
\textbf{(Panel E)} Demon performance showing classification accuracy ($0.0$--$1.0$) over time. Green trace starts at $\sim 0.88$, rises sharply to $\sim 0.98$ by $t = 1$, then maintains plateau at $\sim 0.96$ throughout remaining time. High accuracy demonstrates effective demon operation. Annotation: ``Demon Performance, Classification Accuracy.''
\textbf{(Panel F)} Gradient vs. information cost showing two measures over time. Blue solid line (Temperature gradient) remains constant at $\sim 1.5$ throughout. Orange dashed line (Demon entropy cost) rises linearly from $0$ to $\sim 90$, matching total entropy increase. Demonstrates thermodynamic cost of maintaining gradient. Annotation: ``Gradient vs Information Cost, Temperature gradient, Demon entropy cost, Value.''
}
\label{fig:maxwell_demon_results}
\end{figure*}

\begin{figure*}[htbp]
\centering
\includegraphics[width=\textwidth]{figures/maxwell_demon_validation_maxwell_demon_20250919_000336_5df7dfda.png}
\caption{
\textbf{Biological Maxwell demon validation: Ion gradients, work extraction, ATP synthesis, and thermodynamic compliance.}
\textbf{(Panel A)} Ion electrochemical gradients showing five ion species. Y-axis: Electrochemical Gradient ($-15$--$+15$ kJ/mol). Na$^+$ (small red bar, $\sim 0$ kJ/mol), K$^+$ (large blue bar, $\sim -15$ kJ/mol, most negative), Ca$^{2+}$ (large red bar, $\sim +12$ kJ/mol, most positive), Mg$^{2+}$ (small blue bar, $\sim -5$ kJ/mol), Cl$^-$ (large red bar, $\sim +13$ kJ/mol). Red bars indicate outward gradients, blue bars indicate inward gradients. Annotation: ``Ion Electrochemical Gradients, Electrochemical Gradient (kJ/mol).''
\textbf{(Panel B)} Distribution of work extraction showing histogram. X-axis: Work Extracted ($-9000$--$+2500$ zJ). Y-axis: Frequency ($0$--$25$). Distribution shows three peaks: negative work cluster at $-8000$ zJ (frequency $\sim 10$), small peak at $-7000$ zJ (frequency $\sim 18$), tiny peak at $-6000$ zJ (frequency $\sim 6$), gap in middle range, positive work cluster at $0$--$500$ zJ (frequency $\sim 23$), small peak at $1000$ zJ (frequency $\sim 16$), final peak at $2000$ zJ (frequency $\sim 25$, highest). Bimodal distribution indicates two distinct work extraction regimes. Annotation: ``Distribution of Work Extraction, Frequency, Work Extracted (zJ).''
\textbf{(Panel C)} ATP energy synthesis vs. theoretical showing two bars. Y-axis: Energy ($0$ to $-800$ kJ/mol). Mean ATP Energy (green bar, $\sim -800$ kJ/mol, extends to bottom). Theoretical ATP Energy (red bar, $\sim +50$ kJ/mol, small positive bar at top). Large discrepancy indicates energy accounting issue or different reference states. Annotation: ``ATP Energy Synthesis vs Theoretical, Energy (kJ/mol), Mean ATP Energy, Theoretical ATP Energy.''
\textbf{(Panel D)} Thermodynamic validation success rates showing two bars. Y-axis: Success Rate ($0.0$--$1.0$). Energy Conservation (blue bar, $\sim 0.4$, fails to meet 95\% threshold shown by red dashed line). Second Law $\Delta S \geq 0$ (purple bar, $\sim 1.0$, perfect compliance, exceeds threshold). Red dashed horizontal line at $0.95$ marks threshold. Second law consistently satisfied while energy conservation shows partial compliance. Annotation: ``Thermodynamic Validation Success Rates, --- 95\% Threshold, Success Rate, Energy Conservation, Second Law ($\Delta S \geq 0$).''
}
\label{fig:maxwell_demon_validation}
\end{figure*}

\begin{figure*}[htbp]
\centering
\includegraphics[width=\textwidth]{figures/oscillatory_muscle_simulation.png}
\caption{
\textbf{Oscillatory muscle simulation: Multi-scale coupling effects on force generation, fiber dynamics, and state space evolution.}
\textbf{(Panel A)} Muscle force comparison over $3$ seconds showing with coupling (blue solid line) vs. without coupling (orange dashed line). Both traces show similar profiles: baseline at $0$ N until $0.5$ s, rapid rise to peak ($\sim 6000$--$7000$ N) at $1.0$ s, plateau until $2.0$ s, then decay to baseline by $2.5$ s. Without coupling achieves slightly higher peak force ($\sim 7000$ N) compared to with coupling ($\sim 6000$ N). Annotation: ``Muscle Force, With Coupling, Without Coupling, Force (N).''
\textbf{(Panel B)} Muscle activation showing nearly identical traces for both conditions. Blue solid line (with coupling) and orange dashed line (without coupling) overlap almost completely. Both show: baseline at $0.0$ until $0.5$ s, rapid rise to $1.0$ at $0.7$ s, plateau at $1.0$ until $2.0$ s, rapid decay to $0.0$ by $2.3$ s. Minimal coupling effect on activation timing. Annotation: ``Muscle Activation, With Coupling, Without Coupling, Activation.''
\textbf{(Panel C)} Muscle fiber length showing blue trace over time. Y-axis: Length ($0.078$--$0.092$ m). Length starts at $\sim 0.093$ m, remains constant until $0.5$ s, drops sharply to minimum $\sim 0.078$ m at $1.0$ s, maintains short length until $2.0$ s, then returns to $\sim 0.086$ m by $2.5$ s. Fiber shortening corresponds to force generation phase. Annotation: ``Muscle Fiber Length, Length (m).''
\textbf{(Panel D)} Average coupling strength over time. Y-axis: Coupling Strength ($0.0$--$0.6$). Blue trace shows: baseline near $0.0$ until $0.5$ s, rapid rise to peak $\sim 0.57$ at $0.7$ s, brief plateau at $\sim 0.55$ until $0.9$ s, gradual decay to $\sim 0.1$ by $2.0$ s, slow decline to $\sim 0.05$ by $3.0$ s. Coupling strength peaks during force rise phase. Annotation: ``Average Coupling Strength, Coupling Strength.''
\textbf{(Panel E)} State space coordinates showing three dimensions over time. Blue trace (Knowledge) shows step-like increases from $\sim 0.65$ to $\sim 0.95$, with major transitions at $0.5$ s and $2.0$ s. Orange trace (Time) rises monotonically from $0.0$ to $1.0$ in staircase pattern. Green trace (Entropy) remains constant at $0.0$ throughout. Knowledge and time show coordinated evolution. Annotation: ``State Space Coordinates, Knowledge, Time, Entropy, State Coordinate.''
\textbf{(Panel F)} Coupling matrix heatmap showing coupling strength between five scales: Tis, Neur, Neur, Card, Loc (both axes). Dominant feature: black horizontal band at Neur-Neur intersection indicating strong coupling ($\sim 0.045$). All other regions show weak coupling ($\sim 0.010$--$0.015$, cream/yellow). Color scale: black ($0.010$) to yellow ($0.045$). Neural scale shows strongest self-coupling. Annotation: ``Coupling Matrix, Scale Index, Tis, Neur, Neur, Card, Loc.''
}
\label{fig:oscillatory_muscle_simulation}
\end{figure*}
\begin{figure*}[htbp]
\centering
\includegraphics[width=\textwidth]{figures/cardiac_master_clock_panel.png}
\caption{
\textbf{Cardiac cycle as master clock of consciousness: Heartbeat-Gas-BMD unified framework linking cardiac rhythm to perception quantization.}
\textbf{(Panel A)} Cardiac cycle master clock showing normalized amplitude ($0.0$--$1.0$) over $5$ seconds. Red sinusoidal trace shows cardiac signal with regular peaks (period $\sim 0.43$ s, frequency $= 2.32$ Hz). Yellow box annotation: ``Heart Rate: 2.32 Hz, RR Interval: 431.1 ms.'' Demonstrates fundamental timing signal. Annotation: ``A. Cardiac Cycle: The Master Clock, Cardiac Signal, Amplitude (normalized).''
\textbf{(Panel B)} Heart rate variability histogram showing RR interval distribution. X-axis: RR Interval ($390$--$450$ ms). Y-axis: Frequency ($0.0$--$2.0$). Pink bars show distribution centered at mean $= 431.1$ ms (red dashed vertical line). Narrow distribution indicates stable rhythm. Annotation: ``B. Heart Rate Variability, --- Mean: 431.1 ms, Frequency, RR Interval (ms).''
\textbf{(Panel C)} Gas molecular equilibrium perturbation showing perturbation magnitude ($0.0$--$1.0$) over $5$ seconds. Blue trace exhibits sharp spikes to $1.0$ at each heartbeat, followed by exponential decay to baseline. Black arrow with text box: ``Each heartbeat perturbs molecular equilibrium.'' Regular perturbations every $\sim 0.43$ s match cardiac cycle. Annotation: ``C. Gas Molecular Equilibrium Perturbation, Gas Perturbation, Perturbation Magnitude.''
\textbf{(Panel D)} Equilibrium restoration times histogram showing frequency distribution. X-axis: Restoration Time ($0.0$--$1.0$ ms). Y-axis: Frequency ($0$--$6 \times 10^8$). Purple bars show distribution with mean $= 0.502$ ms (red dashed line). Multiple peaks indicate complex restoration dynamics. Annotation: ``D. Equilibrium Restoration Times, --- Mean: 0.502 ms, Frequency, Restoration Time (ms).''
\textbf{(Panel E)} BMD variance minimization process showing variance ($0.0$--$1.0$) over $5$ seconds. Green trace exhibits sharp spikes to $1.0$ at each heartbeat, followed by exponential decay. Black arrow with text box: ``BMD selects frames to minimize variance.'' Pattern matches perturbation timing. Annotation: ``E. BMD Variance Minimization Process, BMD Variance, Variance.''
\textbf{(Panel F)} Rate hierarchy showing three bars on log scale. Y-axis: Frequency ($10^0$--$10^3$ Hz). Heart Rate (red bar, $2.3$ Hz, shortest), Restoration Time (purple bar, $1993.2$ Hz, tallest), Perception Rate (orange bar, $1993.2$ Hz, same height as restoration). Demonstrates $859.3\times$ coupling ratio. Annotation: ``F. Rate Hierarchy, 2.3 Hz, 1993.2 Hz, 1993.2 Hz, Heart Rate, Restoration Time, Perception Rate, Frequency (Hz).''
\textbf{(Panel G)} Consciousness resonance quality showing resonance quality ($0.990$--$1.000$) vs. beat number ($0$--$100$). Orange trace with black dots oscillates around mean $= 0.999$ (red dashed line) with minimal variation ($\pm 0.002$). High, stable resonance indicates strong cardiac-perception coupling. Annotation: ``G. Consciousness Resonance Quality, --- Mean: 0.999, Resonance Quality, Beat Number.''
\textbf{(Panel H)} Cardiac-perception coupling showing normalized RR interval vs. normalized restoration time. Scatter plot shows dense cloud of points forming diagonal band from $(0, 0)$ to $(1000, 0.8)$, indicating strong positive correlation between cardiac cycle and molecular restoration dynamics. Annotation: ``H. Cardiac-Perception Coupling, RR Interval (normalized), Restoration Time (normalized).''
\textbf{(Panel I)} Text box summary with yellow background containing key parameters and insights: ``CARDIAC CYCLE AS MASTER CLOCK. CARDIAC PARAMETERS: Heart Rate: 2.320 Hz, RR Interval: 431.10 ms, HRV (std): 19.94 ms. PERTURBATION DYNAMICS: Restoration Time: 0.5017 ms, Restoration Rate: 1993.2 Hz. PERCEPTION: Perception Rate: 1993.2 Hz, Resonance Quality: 1.0000. COUPLING RATIO: Perception/Cardiac: 859.3$\times$. KEY INSIGHT: Each heartbeat perturbs molecular equilibrium. BMD minimizes variance by selecting frames during restoration. Rate of perception = Rate of equilibrium restoration after heartbeat perturbation. Consciousness = Ability to resonate with cardiac cycle.''
}
\label{fig:cardiac_master_clock}
\end{figure*}

\begin{figure*}[htbp]
\centering
\includegraphics[width=\textwidth]{figures/complementarity_analysis_lower_half.png}
\caption{
\textbf{Complementarity analysis of numerical and CV methods: Feature space projections, cross-method correlations, and method performance across spectra.}
\textbf{(Panel G)} Feature space (PCA) showing PC2 ($-3.5$--$+1.5$, 23.1\% variance) vs. PC1 ($-2$--$+7$, 75.5\% variance). Six spectra labeled S100--S105 shown as purple circles. Cluster of four spectra (S100--S103) at left (PC1 $\sim -1$ to $0$, PC2 $\sim -3$ to $+1$). S105 isolated at right (PC1 $\sim +6$, PC2 $\sim -0.2$). Legend shows orange (Numerical Better), purple (CV Better), gray (Equal). All spectra purple-coded indicating CV method superiority. Annotation: ``G. Feature Space (PCA), Numerical Better, CV Better, Equal, PC2 (23.1\%), PC1 (75.5\%).''
\textbf{(Panel H)} Feature space (t-SNE) showing t-SNE Dimension 2 ($-30$--$+60$) vs. Dimension 1 ($-70$--$+30$). Six spectra distributed: S105 (bottom-left, $\sim -27, -27$), S100 (top-center, $\sim -40, +55$), S101 (center, $\sim -20, 0$), S102 (upper-right, $\sim 0, +13$), S104 (right, $\sim +20, -2$), S103 (bottom-right, $\sim +10, -27$). Greater separation than PCA indicates nonlinear structure. Annotation: ``H. Feature Space (t-SNE), t-SNE Dimension 2, t-SNE Dimension 1.''
\textbf{(Panel I)} Feature importance (PCA) showing horizontal bars for 11 features. Top features: Shannon Entropy (orange, $\sim 0.095$, longest), S\_knowledge ($\mu$) (pink, $\sim 0.092$), Velocity ($\mu$) (pink, $\sim 0.090$), Peak Count (orange, $\sim 0.088$). Bottom features: Gini Coeff (orange, $\sim 0.025$, shortest). Orange bars indicate numerical features, pink bars indicate CV features. Legend at right. CV features dominate top importance. Annotation: ``I. Feature Importance (PCA), Shannon Entropy, S\_knowledge ($\mu$), Velocity ($\mu$), Peak Count, S\_time ($\mu$), S\_knowledge ($\sigma$), Radius ($\mu$), S\_entropy ($\mu$), S\_entropy ($\sigma$), S\_time ($\sigma$), Gini Coeff, Numerical Features, CV Features, Feature Importance.''
\textbf{(Panel J)} Cross-method feature correlation showing two bars. Left bar (Peak Count vs. Droplet Count): teal, $r = 1.000$, perfect correlation. Right bar (Shannon Entropy vs. Mean S\_entropy): teal, $r = 0.951$, strong correlation. Both exceed moderate correlation threshold (gray dashed line at $\sim 0.6$). Text annotation: ``Strong correlation, Moderate correlation.'' Demonstrates high inter-method agreement. Annotation: ``J. Cross-Method Feature Correlation, $r = 1.000$, $r = 0.951$, Pearson Correlation ($r$).''
\textbf{(Panel K)} Method complementarity by spectrum showing horizontal bars for six spectra. X-axis: Complementarity Score ($-0.35$--$0.00$). All bars salmon-colored, extending leftward (negative scores). S104 shows highest complementarity (shortest bar, $\sim -0.05$). S101 shows lowest (longest bar, $\sim -0.33$). Green box annotation at top: ``High score = methods complement well.'' Negative scores indicate weak complementarity overall. Annotation: ``K. Method Complementarity by Spectrum, High score = methods complement well, S104, S105, S103, S102, S100, S101, Complementarity Score.''
\textbf{(Panel L)} Summary and recommendations text box with salmon background: ``COMPLEMENTARITY ANALYSIS SUMMARY. METHOD PERFORMANCE: Numerical better: 0/6 spectra (0.0\%), CV better: 6/6 spectra (100.0\%), Equal performance: 0/6 spectra (0.0\%). MEAN CONFIDENCE SCORES: Numerical method: 0.269, CV method: 0.805, Combined method: 0.537, Improvement: -33.3\%. COMPLEMENTARITY: Mean complementarity score: -0.330, Methods show weak complementarity. RECOMMENDATIONS: $\checkmark$ Use NUMERICAL method for: Simple spectra, high-throughput. $\checkmark$ Use CV method for: Complex spectra, isobaric compounds. $\checkmark$ Use COMBINED approach for: Maximum confidence, novel compounds.'' Annotation: ``L. Summary and Recommendations.''
}
\label{fig:complementarity_analysis}
\end{figure*}

\begin{figure*}[htbp]
\centering
\includegraphics[width=\textwidth]{figures/figure_1_perception_rate_foundation.png}
\caption{
\textbf{Perception rate foundation: Molecular restoration time distribution, calculation, frequency comparison, and experimental validation.}
\textbf{(Panel A)} Restoration time distribution histogram with KDE overlay. X-axis: Restoration Time ($0$--$1000$ $\mu$s). Y-axis: Probability Density ($0.00000$--$0.00200$). Blue bars show bimodal distribution with peaks at $\sim 100$ $\mu$s and $\sim 500$ $\mu$s. Red curve shows kernel density estimate. Red dashed vertical line marks mean $= 501.7$ $\mu$s. White box annotation: ``KDE, $n = 108$, $\bar{x} = 501.7$ $\mu$s.'' Sample size $n = 108$ measurements. Annotation: ``A, Probability Density, Restoration Time ($\mu$s), KDE, $n = 108$, $\bar{x} = 501.7$ $\mu$s.''
\textbf{(Panel B)} Perception rate calculation showing mathematical derivation in text box. Formula: ``Perception Rate $=$ $\frac{1}{\text{Restoration Time}}$ $= 1 / 501.7$ $\mu$s $= 1993.2$ Hz.'' Yellow highlight box emphasizes final result: ``$= 1993.2$ Hz.'' Demonstrates inverse relationship between restoration time and perception frequency. Annotation: ``B, Perception Rate Calculation, Perception Rate, $=$, $\frac{1}{\text{Restoration Time}}$, $= 1 / 501.7$ $\mu$s, $= 1993.2$ Hz.''
\textbf{(Panel C)} Frequency comparison showing three bars. Left y-axis: Frequency ($10^1$--$10^3$ Hz, log scale). Right y-axis: Fold Increase ($0$--$35$). Traditional Estimate/Neural (gray bar, $60$ Hz, short). Measured/Molecular (green bar, $1993$ Hz, tall, labeled ``1993 Hz''). Ratio (salmon bar, right axis, $\sim 33.2\times$ fold increase, labeled ``33.2$\times$''). Molecular measurement $33.2\times$ higher than traditional neural estimate. Annotation: ``C, 1993 Hz, 33.2$\times$, Frequency (Hz), Fold Increase, Traditional Estimate (Neural), Measured (Molecular), Ratio.''
\textbf{(Panel D)} Experimental validation showing text box with green border. Title: ``Resonance Quality: 1.00, Experimental Validation.'' Three sections: ``Running requires: Perception $\Box$ Thought $\Box$ Action'' (checkboxes). ``If desynchronized: Perception $\neq$ Thought $\rightarrow$ Fall'' (red text). ``Observed: No falls during 400m run'' (green text). Bottom conclusion in green box: ``$\Box$ Perception = Thought.'' Perfect resonance quality ($1.00$) validated by successful running without falls. Annotation: ``D, Resonance Quality: 1.00, Experimental Validation, Running requires:, Perception $\Box$ Thought $\Box$ Action, If desynchronized:, Perception $\neq$ Thought $\rightarrow$ Fall, Observed:, No falls during 400m run, $\Box$ Perception = Thought.''
}
\label{fig:perception_rate_foundation}
\end{figure*}

\begin{figure*}[htbp]
\centering
\includegraphics[width=\textwidth]{figures/master_figure_4_multiscale_atlas.png}
\caption{
\textbf{Multi-scale consciousness atlas: From GPS to Planck scale showing consciousness signatures across $37$ orders of magnitude in spatial precision.}
\textbf{(Panel A)} GPS scale (5m precision) showing macro consciousness from decisions and attention. X-axis: X Position ($0$--$400$ m). Y-axis: Y Position ($-0.6$--$+0.6$ m). Sparse trajectory with color indicating consciousness intensity (purple $0.0$ to yellow $1.0$). Most points show low intensity (purple-blue, $\sim 0.2$). Demonstrates decision-making during locomotion. Annotation: ``A: GPS Scale (5m precision) Macro Consciousness - Decisions \& Attention, Y Position (m), X Position (m), Consciousness Intensity.''
\textbf{(Panel B)} Nanosecond scale (neural) showing consciousness from spike timing over $1000$ nanoseconds. Y-axis: Neuron ID ($0$--$4$). Blue bars show spike synchrony/consciousness (right y-axis, $0.00$--$2.00$). Red vertical tick marks indicate individual spike times. Four neurons show coordinated firing patterns with synchrony peaks reaching $\sim 2.0$ at multiple timepoints ($\sim 200$, $400$, $600$, $800$ ns). Annotation: ``B: Nanosecond Scale (Neural) Consciousness from Spike Timing, Neuron ID, Time (nanoseconds), Spike Synchrony (Consciousness).''
\textbf{(Panel C)} Femtosecond scale (molecular) showing quantum coherence as consciousness in 3D. Axes: X ($-1.0$--$+1.0$ nm), Y ($-1.0$--$+1.0$ nm), Z ($-1.5$--$+1.5$ nm). Point cloud colored by quantum coherence (purple $0.25$ to yellow $0.60$). Spherical distribution with higher coherence (green-yellow, $\sim 0.50$--$0.55$) at periphery, lower coherence (purple-blue, $\sim 0.30$--$0.40$) near center. Demonstrates molecular-scale consciousness substrate. Annotation: ``C: Femtosecond Scale (Molecular) Quantum Coherence = Consciousness, Z (nm), Y (nm), X (nm), Quantum Coherence.''
\textbf{(Panel D)} Planck scale ($10^{-35}$ m) showing spacetime geometry as consciousness. X-axis: Planck Length Units ($-4$--$+4$). Y-axis: Planck Length Units ($-4$--$+4$). Heatmap shows spacetime curvature (purple $-1.6$ to red $+1.6$). Pattern exhibits cellular structure with alternating high-curvature (red-orange, $\sim +1.2$) and low-curvature (blue-purple, $\sim -0.8$) regions. Fundamental geometric structure of consciousness at quantum gravity scale. Annotation: ``D: Planck Scale ($10^{-35}$m) Spacetime Geometry = Consciousness, Planck Length Units, Planck Length Units, Spacetime Curvature.''
\textbf{(Panel E)} Multi-scale consciousness complexity showing log-log plot. Y-axis: Consciousness Complexity/Information Bits ($10^1$--$10^9$). X-axis: Spatial Precision ($10^{-32}$--$10^{-2}$ meters). Purple line with circles descends following power law $C \sim L^{-0.28}$ (red dashed line). Seven labeled points: Planck ($10^{-32}$ m, $10^9$ bits), Picometer ($10^{-12}$ m, $10^5$ bits), Nanometer ($10^{-9}$ m, $10^3$ bits), Micrometer ($10^{-6}$ m, $10^2$ bits), Millimeter ($10^{-3}$ m, $10^1$ bits), GPS ($10^{-2}$ m, $10^1$ bits). Yellow box annotation: ``Consciousness complexity scales as power law, Finer precision = Richer structure.'' Annotation: ``E: Multi-Scale Consciousness Complexity Same Geometry, Increasing Information, --- Power Law: $C \sim L^{-0.28}$, Planck, Picometer, Micrometer, Nanometer, Millimeter, GPS, Consciousness Complexity (Information Bits), Spatial Precision (meters).''
\textbf{(Panel F)} Unified multi-scale consciousness equations in text box with purple border. ``Consciousness Framework: $C(x, t, \varepsilon) = ||P(x, t, \varepsilon) - T(x, t, \varepsilon)||$ where: $P =$ Perception manifold, $T =$ Thought manifold, $\varepsilon =$ Precision scale, $x =$ Spatial coordinates, $t =$ Time.'' Three boxed principles: ``Scale Invariance: $C(x, t, \varepsilon) \sim C(x, t, \lambda\varepsilon)$.'' ``Complexity Scaling: $\mathcal{I}(\varepsilon) \sim \log_2(\varepsilon) | C_0$.'' ``Heartbeat Quantization: $\Delta t \sim$ RRInterval, $f_{\text{perception}} = 1/T_{\text{restoration}}$.'' ``Consciousness Measure: $Q = \frac{|\omega_{\text{heart}} - \omega_{\text{perception}}|}{\text{HRV}}$.'' Annotation: ``F: Unified Multi-Scale Consciousness Equations, Consciousness Framework, Scale Invariance, Complexity Scaling, Heartbeat Quantization, Consciousness Measure.''
}
\label{fig:multiscale_consciousness_atlas}
\end{figure*}


\begin{figure*}[htbp]
    \centering
    \includegraphics[width=\textwidth]{figures/cardiac_master_clock_panel.png}
    \caption{
    \textbf{Cardiac cycle as master clock of consciousness: Heartbeat-Gas-BMD unified framework linking cardiac rhythm to perception quantization.}
    \textbf{(Panel A)} Cardiac cycle master clock showing normalized amplitude ($0.0$--$1.0$) over $5$ seconds. Red sinusoidal trace shows cardiac signal with regular peaks (period $\sim 0.43$ s, frequency $= 2.32$ Hz). Yellow box annotation: ``Heart Rate: 2.32 Hz, RR Interval: 431.1 ms.'' Demonstrates fundamental timing signal. Annotation: ``A. Cardiac Cycle: The Master Clock, Cardiac Signal, Amplitude (normalized).''
    \textbf{(Panel B)} Heart rate variability histogram showing RR interval distribution. X-axis: RR Interval ($390$--$450$ ms). Y-axis: Frequency ($0.0$--$2.0$). Pink bars show distribution centered at mean $= 431.1$ ms (red dashed vertical line). Narrow distribution indicates stable rhythm. Annotation: ``B. Heart Rate Variability, --- Mean: 431.1 ms, Frequency, RR Interval (ms).''
    \textbf{(Panel C)} Gas molecular equilibrium perturbation showing perturbation magnitude ($0.0$--$1.0$) over $5$ seconds. Blue trace exhibits sharp spikes to $1.0$ at each heartbeat, followed by exponential decay to baseline. Black arrow with text box: ``Each heartbeat perturbs molecular equilibrium.'' Regular perturbations every $\sim 0.43$ s match cardiac cycle. Annotation: ``C. Gas Molecular Equilibrium Perturbation, Gas Perturbation, Perturbation Magnitude.''
    \textbf{(Panel D)} Equilibrium restoration times histogram showing frequency distribution. X-axis: Restoration Time ($0.0$--$1.0$ ms). Y-axis: Frequency ($0$--$6 \times 10^8$). Purple bars show distribution with mean $= 0.502$ ms (red dashed line). Multiple peaks indicate complex restoration dynamics. Annotation: ``D. Equilibrium Restoration Times, --- Mean: 0.502 ms, Frequency, Restoration Time (ms).''
    \textbf{(Panel E)} BMD variance minimization process showing variance ($0.0$--$1.0$) over $5$ seconds. Green trace exhibits sharp spikes to $1.0$ at each heartbeat, followed by exponential decay. Black arrow with text box: ``BMD selects frames to minimize variance.'' Pattern matches perturbation timing. Annotation: ``E. BMD Variance Minimization Process, BMD Variance, Variance.''
    \textbf{(Panel F)} Rate hierarchy showing three bars on log scale. Y-axis: Frequency ($10^0$--$10^3$ Hz). Heart Rate (red bar, $2.3$ Hz, shortest), Restoration Time (purple bar, $1993.2$ Hz, tallest), Perception Rate (orange bar, $1993.2$ Hz, same height as restoration). Demonstrates $859.3\times$ coupling ratio. Annotation: ``F. Rate Hierarchy, 2.3 Hz, 1993.2 Hz, 1993.2 Hz, Heart Rate, Restoration Time, Perception Rate, Frequency (Hz).''
    \textbf{(Panel G)} Consciousness resonance quality showing resonance quality ($0.990$--$1.000$) vs. beat number ($0$--$100$). Orange trace with black dots oscillates around mean $= 0.999$ (red dashed line) with minimal variation ($\pm 0.002$). High, stable resonance indicates strong cardiac-perception coupling. Annotation: ``G. Consciousness Resonance Quality, --- Mean: 0.999, Resonance Quality, Beat Number.''
    \textbf{(Panel H)} Cardiac-perception coupling showing normalized RR interval vs. normalized restoration time. Scatter plot shows dense cloud of points forming diagonal band from $(0, 0)$ to $(1000, 0.8)$, indicating strong positive correlation between cardiac cycle and molecular restoration dynamics. Annotation: ``H. Cardiac-Perception Coupling, RR Interval (normalized), Restoration Time (normalized).''
    \textbf{(Panel I)} Text box summary with yellow background containing key parameters and insights: ``CARDIAC CYCLE AS MASTER CLOCK. CARDIAC PARAMETERS: Heart Rate: 2.320 Hz, RR Interval: 431.10 ms, HRV (std): 19.94 ms. PERTURBATION DYNAMICS: Restoration Time: 0.5017 ms, Restoration Rate: 1993.2 Hz. PERCEPTION: Perception Rate: 1993.2 Hz, Resonance Quality: 1.0000. COUPLING RATIO: Perception/Cardiac: 859.3$\times$. KEY INSIGHT: Each heartbeat perturbs molecular equilibrium. BMD minimizes variance by selecting frames during restoration. Rate of perception = Rate of equilibrium restoration after heartbeat perturbation. Consciousness = Ability to resonate with cardiac cycle.''
    }
    \label{fig:cardiac_master_clock}
    \end{figure*}

    \begin{figure*}[htbp]
    \centering
    \includegraphics[width=\textwidth]{figures/complementarity_analysis_lower_half.png}
    \caption{
    \textbf{Complementarity analysis of numerical and CV methods: Feature space projections, cross-method correlations, and method performance across spectra.}
    \textbf{(Panel G)} Feature space (PCA) showing PC2 ($-3.5$--$+1.5$, 23.1\% variance) vs. PC1 ($-2$--$+7$, 75.5\% variance). Six spectra labeled S100--S105 shown as purple circles. Cluster of four spectra (S100--S103) at left (PC1 $\sim -1$ to $0$, PC2 $\sim -3$ to $+1$). S105 isolated at right (PC1 $\sim +6$, PC2 $\sim -0.2$). Legend shows orange (Numerical Better), purple (CV Better), gray (Equal). All spectra purple-coded indicating CV method superiority. Annotation: ``G. Feature Space (PCA), Numerical Better, CV Better, Equal, PC2 (23.1\%), PC1 (75.5\%).''
    \textbf{(Panel H)} Feature space (t-SNE) showing t-SNE Dimension 2 ($-30$--$+60$) vs. Dimension 1 ($-70$--$+30$). Six spectra distributed: S105 (bottom-left, $\sim -27, -27$), S100 (top-center, $\sim -40, +55$), S101 (center, $\sim -20, 0$), S102 (upper-right, $\sim 0, +13$), S104 (right, $\sim +20, -2$), S103 (bottom-right, $\sim +10, -27$). Greater separation than PCA indicates nonlinear structure. Annotation: ``H. Feature Space (t-SNE), t-SNE Dimension 2, t-SNE Dimension 1.''
    \textbf{(Panel I)} Feature importance (PCA) showing horizontal bars for 11 features. Top features: Shannon Entropy (orange, $\sim 0.095$, longest), S\_knowledge ($\mu$) (pink, $\sim 0.092$), Velocity ($\mu$) (pink, $\sim 0.090$), Peak Count (orange, $\sim 0.088$). Bottom features: Gini Coeff (orange, $\sim 0.025$, shortest). Orange bars indicate numerical features, pink bars indicate CV features. Legend at right. CV features dominate top importance. Annotation: ``I. Feature Importance (PCA), Shannon Entropy, S\_knowledge ($\mu$), Velocity ($\mu$), Peak Count, S\_time ($\mu$), S\_knowledge ($\sigma$), Radius ($\mu$), S\_entropy ($\mu$), S\_entropy ($\sigma$), S\_time ($\sigma$), Gini Coeff, Numerical Features, CV Features, Feature Importance.''
    \textbf{(Panel J)} Cross-method feature correlation showing two bars. Left bar (Peak Count vs. Droplet Count): teal, $r = 1.000$, perfect correlation. Right bar (Shannon Entropy vs. Mean S\_entropy): teal, $r = 0.951$, strong correlation. Both exceed moderate correlation threshold (gray dashed line at $\sim 0.6$). Text annotation: ``Strong correlation, Moderate correlation.'' Demonstrates high inter-method agreement. Annotation: ``J. Cross-Method Feature Correlation, $r = 1.000$, $r = 0.951$, Pearson Correlation ($r$).''
    \textbf{(Panel K)} Method complementarity by spectrum showing horizontal bars for six spectra. X-axis: Complementarity Score ($-0.35$--$0.00$). All bars salmon-colored, extending leftward (negative scores). S104 shows highest complementarity (shortest bar, $\sim -0.05$). S101 shows lowest (longest bar, $\sim -0.33$). Green box annotation at top: ``High score = methods complement well.'' Negative scores indicate weak complementarity overall. Annotation: ``K. Method Complementarity by Spectrum, High score = methods complement well, S104, S105, S103, S102, S100, S101, Complementarity Score.''
    \textbf{(Panel L)} Summary and recommendations text box with salmon background: ``COMPLEMENTARITY ANALYSIS SUMMARY. METHOD PERFORMANCE: Numerical better: 0/6 spectra (0.0\%), CV better: 6/6 spectra (100.0\%), Equal performance: 0/6 spectra (0.0\%). MEAN CONFIDENCE SCORES: Numerical method: 0.269, CV method: 0.805, Combined method: 0.537, Improvement: -33.3\%. COMPLEMENTARITY: Mean complementarity score: -0.330, Methods show weak complementarity. RECOMMENDATIONS: $\checkmark$ Use NUMERICAL method for: Simple spectra, high-throughput. $\checkmark$ Use CV method for: Complex spectra, isobaric compounds. $\checkmark$ Use COMBINED approach for: Maximum confidence, novel compounds.'' Annotation: ``L. Summary and Recommendations.''
    }
    \label{fig:complementarity_analysis}
    \end{figure*}

    \begin{figure*}[htbp]
    \centering
    \includegraphics[width=\textwidth]{figures/figure_1_perception_rate_foundation.png}
    \caption{
    \textbf{Perception rate foundation: Molecular restoration time distribution, calculation, frequency comparison, and experimental validation.}
    \textbf{(Panel A)} Restoration time distribution histogram with KDE overlay. X-axis: Restoration Time ($0$--$1000$ $\mu$s). Y-axis: Probability Density ($0.00000$--$0.00200$). Blue bars show bimodal distribution with peaks at $\sim 100$ $\mu$s and $\sim 500$ $\mu$s. Red curve shows kernel density estimate. Red dashed vertical line marks mean $= 501.7$ $\mu$s. White box annotation: ``KDE, $n = 108$, $\bar{x} = 501.7$ $\mu$s.'' Sample size $n = 108$ measurements. Annotation: ``A, Probability Density, Restoration Time ($\mu$s), KDE, $n = 108$, $\bar{x} = 501.7$ $\mu$s.''
    \textbf{(Panel B)} Perception rate calculation showing mathematical derivation in text box. Formula: ``Perception Rate $=$ $\frac{1}{\text{Restoration Time}}$ $= 1 / 501.7$ $\mu$s $= 1993.2$ Hz.'' Yellow highlight box emphasizes final result: ``$= 1993.2$ Hz.'' Demonstrates inverse relationship between restoration time and perception frequency. Annotation: ``B, Perception Rate Calculation, Perception Rate, $=$, $\frac{1}{\text{Restoration Time}}$, $= 1 / 501.7$ $\mu$s, $= 1993.2$ Hz.''
    \textbf{(Panel C)} Frequency comparison showing three bars. Left y-axis: Frequency ($10^1$--$10^3$ Hz, log scale). Right y-axis: Fold Increase ($0$--$35$). Traditional Estimate/Neural (gray bar, $60$ Hz, short). Measured/Molecular (green bar, $1993$ Hz, tall, labeled ``1993 Hz''). Ratio (salmon bar, right axis, $\sim 33.2\times$ fold increase, labeled ``33.2$\times$''). Molecular measurement $33.2\times$ higher than traditional neural estimate. Annotation: ``C, 1993 Hz, 33.2$\times$, Frequency (Hz), Fold Increase, Traditional Estimate (Neural), Measured (Molecular), Ratio.''
    \textbf{(Panel D)} Experimental validation showing text box with green border. Title: ``Resonance Quality: 1.00, Experimental Validation.'' Three sections: ``Running requires: Perception $\Box$ Thought $\Box$ Action'' (checkboxes). ``If desynchronized: Perception $\neq$ Thought $\rightarrow$ Fall'' (red text). ``Observed: No falls during 400m run'' (green text). Bottom conclusion in green box: ``$\Box$ Perception = Thought.'' Perfect resonance quality ($1.00$) validated by successful running without falls. Annotation: ``D, Resonance Quality: 1.00, Experimental Validation, Running requires:, Perception $\Box$ Thought $\Box$ Action, If desynchronized:, Perception $\neq$ Thought $\rightarrow$ Fall, Observed:, No falls during 400m run, $\Box$ Perception = Thought.''
    }
    \label{fig:perception_rate_foundation}
    \end{figure*}

    \begin{figure*}[htbp]
    \centering
    \includegraphics[width=\textwidth]{figures/master_figure_4_multiscale_atlas.png}
    \caption{
    \textbf{Multi-scale consciousness atlas: From GPS to Planck scale showing consciousness signatures across $37$ orders of magnitude in spatial precision.}
    \textbf{(Panel A)} GPS scale (5m precision) showing macro consciousness from decisions and attention. X-axis: X Position ($0$--$400$ m). Y-axis: Y Position ($-0.6$--$+0.6$ m). Sparse trajectory with color indicating consciousness intensity (purple $0.0$ to yellow $1.0$). Most points show low intensity (purple-blue, $\sim 0.2$). Demonstrates decision-making during locomotion. Annotation: ``A: GPS Scale (5m precision) Macro Consciousness - Decisions \& Attention, Y Position (m), X Position (m), Consciousness Intensity.''
    \textbf{(Panel B)} Nanosecond scale (neural) showing consciousness from spike timing over $1000$ nanoseconds. Y-axis: Neuron ID ($0$--$4$). Blue bars show spike synchrony/consciousness (right y-axis, $0.00$--$2.00$). Red vertical tick marks indicate individual spike times. Four neurons show coordinated firing patterns with synchrony peaks reaching $\sim 2.0$ at multiple timepoints ($\sim 200$, $400$, $600$, $800$ ns). Annotation: ``B: Nanosecond Scale (Neural) Consciousness from Spike Timing, Neuron ID, Time (nanoseconds), Spike Synchrony (Consciousness).''
    \textbf{(Panel C)} Femtosecond scale (molecular) showing quantum coherence as consciousness in 3D. Axes: X ($-1.0$--$+1.0$ nm), Y ($-1.0$--$+1.0$ nm), Z ($-1.5$--$+1.5$ nm). Point cloud colored by quantum coherence (purple $0.25$ to yellow $0.60$). Spherical distribution with higher coherence (green-yellow, $\sim 0.50$--$0.55$) at periphery, lower coherence (purple-blue, $\sim 0.30$--$0.40$) near center. Demonstrates molecular-scale consciousness substrate. Annotation: ``C: Femtosecond Scale (Molecular) Quantum Coherence = Consciousness, Z (nm), Y (nm), X (nm), Quantum Coherence.''
    \textbf{(Panel D)} Planck scale ($10^{-35}$ m) showing spacetime geometry as consciousness. X-axis: Planck Length Units ($-4$--$+4$). Y-axis: Planck Length Units ($-4$--$+4$). Heatmap shows spacetime curvature (purple $-1.6$ to red $+1.6$). Pattern exhibits cellular structure with alternating high-curvature (red-orange, $\sim +1.2$) and low-curvature (blue-purple, $\sim -0.8$) regions. Fundamental geometric structure of consciousness at quantum gravity scale. Annotation: ``D: Planck Scale ($10^{-35}$m) Spacetime Geometry = Consciousness, Planck Length Units, Planck Length Units, Spacetime Curvature.''
    \textbf{(Panel E)} Multi-scale consciousness complexity showing log-log plot. Y-axis: Consciousness Complexity/Information Bits ($10^1$--$10^9$). X-axis: Spatial Precision ($10^{-32}$--$10^{-2}$ meters). Purple line with circles descends following power law $C \sim L^{-0.28}$ (red dashed line). Seven labeled points: Planck ($10^{-32}$ m, $10^9$ bits), Picometer ($10^{-12}$ m, $10^5$ bits), Nanometer ($10^{-9}$ m, $10^3$ bits), Micrometer ($10^{-6}$ m, $10^2$ bits), Millimeter ($10^{-3}$ m, $10^1$ bits), GPS ($10^{-2}$ m, $10^1$ bits). Yellow box annotation: ``Consciousness complexity scales as power law, Finer precision = Richer structure.'' Annotation: ``E: Multi-Scale Consciousness Complexity Same Geometry, Increasing Information, --- Power Law: $C \sim L^{-0.28}$, Planck, Picometer, Micrometer, Nanometer, Millimeter, GPS, Consciousness Complexity (Information Bits), Spatial Precision (meters).''
    \textbf{(Panel F)} Unified multi-scale consciousness equations in text box with purple border. ``Consciousness Framework: $C(x, t, \varepsilon) = ||P(x, t, \varepsilon) - T(x, t, \varepsilon)||$ where: $P =$ Perception manifold, $T =$ Thought manifold, $\varepsilon =$ Precision scale, $x =$ Spatial coordinates, $t =$ Time.'' Three boxed principles: ``Scale Invariance: $C(x, t, \varepsilon) \sim C(x, t, \lambda\varepsilon)$.'' ``Complexity Scaling: $\mathcal{I}(\varepsilon) \sim \log_2(\varepsilon) | C_0$.'' ``Heartbeat Quantization: $\Delta t \sim$ RRInterval, $f_{\text{perception}} = 1/T_{\text{restoration}}$.'' ``Consciousness Measure: $Q = \frac{|\omega_{\text{heart}} - \omega_{\text{perception}}|}{\text{HRV}}$.'' Annotation: ``F: Unified Multi-Scale Consciousness Equations, Consciousness Framework, Scale Invariance, Complexity Scaling, Heartbeat Quantization, Consciousness Measure.''
    }
    \label{fig:multiscale_consciousness_atlas}
    \end{figure*}

    \begin{figure*}[htbp]
        \centering
        \includegraphics[width=\textwidth]{figures/brain_wave_oscillatory_analysis.png}
        \caption{
        \textbf{Brain wave oscillatory analysis: EEG signal decomposition, spectral characteristics, cross-frequency coupling, and validation metrics.}
        \textbf{(Panel A)} EEG signal (10 seconds) showing amplitude ($-100$--$+100$ $\mu$V) over time. Black trace exhibits high-frequency oscillations with regular amplitude modulation. Signal shows consistent periodic structure throughout recording period. Raw neural activity demonstrates complex multi-frequency composition. Annotation: ``EEG Signal (10 seconds), Amplitude ($\mu$V), Time (s).''
        \textbf{(Panel B)} Power spectral density showing PSD ($10^{-1}$--$10^3$ $\mu$V$^2$/Hz, log scale) vs. frequency ($0$--$100$ Hz). Blue trace shows multiple prominent peaks: dominant peak at $\sim 10$ Hz (alpha band, $\sim 10^3$ $\mu$V$^2$/Hz), secondary peaks at $\sim 5$ Hz (theta), $\sim 20$ Hz, $\sim 40$ Hz (gamma), and $\sim 80$ Hz. Baseline noise floor at $\sim 10^{-1}$ $\mu$V$^2$/Hz. Legend shows delta, theta, alpha, beta, gamma, high\_gamma bands (color-coded). Annotation: ``Power Spectral Density, delta, theta, alpha, beta, gamma, high\_gamma, PSD ($\mu$V$^2$/Hz), Frequency (Hz).''
        \textbf{(Panel C)} Frequency components (5 seconds) showing amplitude ($-50$--$+200$ $\mu$V) for four bands. Blue trace (delta, $\sim 100$ $\mu$V baseline), orange trace (theta, $\sim 75$ $\mu$V baseline), green trace (alpha, $\sim 50$ $\mu$V baseline), red trace (beta, $\sim 180$ $\mu$V baseline, highest amplitude). All bands show sinusoidal oscillations with different frequencies and phases. Legend identifies delta, theta, alpha, beta. Annotation: ``Frequency Components (5 seconds), delta, theta, alpha, beta, Amplitude ($\mu$V), Time (s).''
        \textbf{(Panel D)} Cross-frequency coupling showing coupling strength for four interactions. Y-axis: Coupling Strength ($0.000$--$0.053$). Salmon-colored bars: theta\_gamma\_pac ($\sim 0.012$), alpha\_beta\_coupling ($\sim 0.011$), delta\_theta\_coupling ($\sim 0.016$), gamma\_coherence ($\sim 0.053$, tallest bar). Gamma coherence shows strongest coupling. Annotation: ``Cross-Frequency Coupling, 0.053, Coupling Strength, theta\_gamma\_pac, alpha\_beta\_coupling, delta\_theta\_coupling, gamma\_coherence.''
        \textbf{(Panel E)} Alpha-beta interaction showing envelope amplitudes over 10 seconds. Orange trace (Alpha envelope) oscillates between $\sim 35$--$45$ with slow modulation (period $\sim 5$ s). Blue trace (Beta envelope) oscillates between $\sim 10$--$28$ with inverse phase relationship. Demonstrates cross-frequency amplitude modulation. Legend shows Alpha envelope, Beta envelope. Annotation: ``Alpha-Beta Interaction, Alpha envelope, Beta envelope, Envelope Amplitude, Time (s).''
        \textbf{(Panel F)} Brain wave band powers showing percentage distribution. Y-axis: Power ($0$--$45$\%). Six bars: delta (purple, $42.8$\%, dominant), theta (purple, $15.5$\%), alpha (teal, $13.3$\%), beta (green, $21.9$\%), gamma (cyan, $5.7$\%), high\_gamma (cyan, $0.3$\%, smallest). Delta dominates power spectrum. Legend shows color coding. Annotation: ``Brain Wave Band Powers, 42.8\%, 21.9\%, 15.5\%, 13.3\%, 5.7\%, 0.3\%, delta, theta, alpha, beta, gamma, high\_gamma, Power (\%).''
        \textbf{(Panel G)} Gamma oscillations (2 seconds) showing amplitude ($-30$--$+30$ $\mu$V). Red trace exhibits rapid oscillations ($\sim 40$ Hz) with amplitude modulation envelope. High-frequency activity shows consistent periodicity with varying amplitude ($\pm 25$ $\mu$V peaks). Annotation: ``Gamma Oscillations (2 seconds), Amplitude ($\mu$V), Time (s).''
        \textbf{(Panel H)} Theta-gamma PAC showing mean gamma amplitude vs. theta phase. X-axis: Theta Phase ($-3$--$+3$ radians). Y-axis: Mean Gamma Amplitude ($0$--$25$). Blue bars form histogram with peak at phase $\sim -1$ radian (amplitude $\sim 24$), indicating preferred phase coupling. Distribution shows phase-amplitude modulation (MI=0.012). Annotation: ``Theta-Gamma PAC (MI=0.012), Mean Gamma Amplitude, Theta Phase (radians).''
        \textbf{(Panel I)} Validation summary in salmon-colored box: ``$\square$ BRAIN WAVE VALIDATION. $\checkmark$ Status: FAIL. Alpha Dominance: 13.3\% Expected: (20, 40). Theta-Gamma PAC: 0.012 Threshold: $\geq 0.1$. Alpha-Beta Coupling: 0.011 Expected: (-0.7, -0.2).'' Three validation criteria fail to meet thresholds. Annotation: ``Validation Summary, $\square$ BRAIN WAVE VALIDATION, $\checkmark$ Status: FAIL, Alpha Dominance: 13.3\% Expected: (20, 40), Theta-Gamma PAC: 0.012 Threshold: $\geq 0.1$, Alpha-Beta Coupling: 0.011 Expected: (-0.7, -0.2).''
        }
        \label{fig:brain_wave_analysis}
        \end{figure*}

        \begin{figure*}[htbp]
        \centering
        \includegraphics[width=\textwidth]{figures/cardiac_master_comprehensive.png}
        \caption{
        \textbf{Cardiac cycle as master clock: Comprehensive analysis of heartbeat-gas-BMD unified framework showing synchronized multi-scale coupling.}
        \textbf{(Panel A)} Synchronized multi-scale view showing three normalized signals ($0.0$--$1.0$) over 5 seconds. Red trace (Cardiac Cycle) shows sinusoidal pattern with period $\sim 0.43$ s. Blue trace (Gas Perturbation) shows sharp spikes to $1.0$ at each cardiac peak, followed by exponential decay. Green trace (BMD Variance) shows similar spike pattern with slightly delayed timing. All three signals phase-locked to cardiac rhythm. Annotation: ``Synchronized Multi-Scale View: Cardiac Cycle Drives All Processes, Cardiac Cycle, Gas Perturbation, BMD Variance, Cardiac Signal, Gas Perturbation, BMD Variance, Time (s).''
        \textbf{(Panel B)} Phase relationships (cardiac leads) showing polar plot. Three traces: red (Cardiac), blue (Gas), green (BMD). Cardiac trace forms largest circle (radius $\sim 2.5$) centered at origin. Gas trace (radius $\sim 1.5$) shows phase lag. BMD trace (radius $\sim 1.0$, innermost) shows further phase lag. Angles marked: $0°$, $45°$, $90°$, $135°$, $180°$, $225°$, $270°$, $315°$. Demonstrates cardiac leads all processes. Annotation: ``Phase Relationships (Cardiac Leads), $90°$, $135°$, $45°$, Cardiac, Gas, BMD, $180°$, $0°$, $225°$, $315°$, $270°$.''
        \textbf{(Panel C)} Coupling strength matrix showing heatmap. Y-axis: Influencing Process (Cardiac, Gas, BMD, Perception). X-axis: Influenced Process (Cardiac, Gas, BMD, Perception). Color scale: dark red ($1.0$, strongest) to dark blue ($0.0$, weakest). Cardiac row shows strong coupling to all processes: Cardiac-Gas ($0.30$, orange), Cardiac-BMD ($0.20$, orange), Cardiac-Perception ($0.10$, orange). Diagonal shows self-coupling ($1.00$, dark red). Gas-BMD ($0.40$, orange), BMD-Perception ($0.25$, orange). Cardiac dominates coupling structure. Annotation: ``Coupling Strength Matrix (Cardiac Dominates), Cardiac, Gas, BMD, Perception, Influencing Process, Cardiac, Gas, BMD, Perception, Influenced Process, Coupling Strength, $1.00$, $1.00$, $0.30$, $1.00$, $0.85$, $0.75$, $0.20$, $0.40$, $1.00$, $0.90$, $0.10$, $0.25$, $0.50$, $4.00$.''
        \textbf{(Panel D)} Energy/information flow diagram showing flowchart from cardiac to perception. Red box (Cardiac Contraction) $\rightarrow$ blue box (Mechanical Perturbation) $\rightarrow$ teal box (Gas Equilibrium). Parallel path: green box (BMD Processing) $\rightarrow$ orange box (Conscious Perception). Demonstrates information cascade from cardiac cycle through molecular equilibrium to conscious perception. Annotation: ``Energy/Information Flow (Cardiac $\rightarrow$ Perception), Cardiac Contraction, Mechanical Perturbation, Gas Equilibrium, BMD Processing, Conscious Perception.''
        \textbf{(Panel E)} Frequency spectrum (cardiac as reference) showing log-scale plot. X-axis: Frequency (Hz, $10^0$--$10^3$). Four labeled points: Respiration ($0.2$ Hz, red circle), Cardiac ($2.3$ Hz, red circle), Neural ($40.0$ Hz, teal circle), Perception ($1993.2$ Hz, orange box). Red dashed vertical line marks cardiac reference. Perception frequency $859\times$ higher than cardiac. Annotation: ``Frequency Spectrum (Cardiac as Reference), Cardiac Reference, Perception 1993.2 Hz, Cardiac 2.3 Hz, Respiration 0.2 Hz, Neural Y 40.0 Hz, 1993 2 Hz, Frequency (Hz, log scale), $10^0$, $10^1$, $10^2$, $10^3$.''
        \textbf{(Panel F)} Temporal precision showing bar chart. Y-axis: Temporal Jitter (CV\%, $0$--$60$). Four bars: Cardiac (red, $4.68$\%, lowest, most precise), Gas Restore (blue, $57.95$\%, highest), BMD Sample (green, $5.00$\%), Perception (orange, $3.00$\%, second lowest).
        }
        \label{fig:cardiac_master_comprehensive}
        \end{figure*}

        \begin{figure*}[htbp]
        \centering
        \includegraphics[width=\textwidth]{figures/cognitive_processing_analysis.png}
        \caption{
        \textbf{Cognitive processing analysis: State dynamics, neural oscillations, network coupling, and performance validation across cognitive domains.}
        \textbf{(Panel A)} Cognitive state dynamics showing four state levels ($0$--$7$) over 175 seconds. Green trace (Attention) shows sharp peaks to $\sim 7$ at regular intervals ($\sim 50$ s period), baseline at $\sim 3$. Orange trace (Working Memory) remains constant at $\sim 1$. Red trace (Executive) shows small oscillations around $\sim 0.5$. Purple trace (Consciousness) shows periodic peaks to $\sim 3$ synchronized with attention peaks. Legend identifies all four states. Annotation: ``Cognitive State Dynamics, Attention, Working Memory, Executive, Consciousness, State Level, Time (s).''
        \textbf{(Panel B)} Neural oscillations (10 seconds) showing four stacked bands with offset. Blue band (Working Memory, $0$--$50$, bottom), green band (Executive, $50$--$100$), red band (Attention, $100$--$150$), purple band (Consciousness, $150$--$200$, top). All bands show dense oscillatory activity. High-frequency fluctuations throughout all cognitive states. Annotation: ``Neural Oscillations (10 seconds), Working Memory, Executive, Attention, Consciousness, Neural Activity (offset), Time (s).''
        \textbf{(Panel C)} Cognitive performance over time showing two metrics. Blue trace (Reaction Time, left y-axis, $360$--$440$ ms) oscillates with period $\sim 50$ s, peaks at $\sim 430$ ms, troughs at $\sim 350$ ms. Red trace (Accuracy, right y-axis, $0.6$--$1.3$) shows inverse relationship, peaks when RT is low. Demonstrates performance oscillations synchronized with cognitive states. Annotation: ``Cognitive Performance Over Time, Reaction Time, Accuracy, Reaction Time (ms), Accuracy, Time (s).''
        \textbf{(Panel D)} Cognitive network coupling heatmap. Y-axis: attention, working memory, executive, consciousness. X-axis: attention, working memory, executive, consciousness. Color scale: dark red ($1.0$) to dark blue ($0.0$). Diagonal shows self-coupling ($1.0$, dark red). Attention-working memory shows strong coupling ($\sim 0.8$, red). Working memory-executive moderate coupling ($\sim 0.6$, orange). Executive-consciousness weak coupling ($\sim 0.2$, blue). Off-diagonal asymmetry indicates directional influences. Annotation: ``Cognitive Network Coupling, attention, working memory, executive, consciousness, attention, working memory, executive, consciousness, $1.0$, $0.8$, $0.6$, $0.4$, $0.2$, $0.0$.''
        \textbf{(Panel E)} Key coupling relationships showing three bars. Y-axis: Coupling Strength ($0.0000$--$0.0040$). Cyan bar (RT-Executive, $\sim 0.0037$, tallest), green bar (WM-Consciousness, $\sim 0.0039$), yellow bar (Cognitive Coherence, $\sim 0.0030$, shortest). All coupling strengths very weak ($< 0.004$). Annotation: ``Key Coupling Relationships, Coupling Strength, RT-Executive, WM-Consciousness, Cognitive Coherence.''
        \textbf{(Panel F)} Processing efficiency showing efficiency ($0.0$--$0.8$) over 175 seconds. Orange trace with yellow shading oscillates with period $\sim 50$ s. Peaks reach $\sim 0.75$ at $t \sim 25, 75, 125$ s. Troughs drop to $\sim 0.3$ at $t \sim 50, 100, 150$ s. Efficiency varies $2.5\times$ across cognitive cycle. Annotation: ``Processing Efficiency, Efficiency, Time (s).''
        \textbf{(Panel G)} Cognitive resources showing resource level ($0.00$--$1.75$) over 175 seconds. Maroon trace with pink shading shows sinusoidal oscillation. Peaks at $\sim 1.8$ occur at $t \sim 25, 75, 125$ s. Troughs at $\sim 0.2$ occur at $t \sim 0, 50, 100, 150$ s. Resource availability cycles with $\sim 50$ s period. Annotation: ``Cognitive Resources, Resource Level, Time (s).''
        \textbf{(Panel H)} RT-neural correlation scatter plot showing reaction time ($360$--$440$ ms) vs. attention neural activity ($0.0$--$20.0$). Red dots ($n \sim 500$) form diffuse cloud with weak positive trend. Orange dashed line shows linear fit with $r = 0.329$ (weak correlation). Wide scatter indicates poor predictive relationship. Annotation: ``RT-Neural Correlation ($r=0.329$), Reaction Time (ms), Attention Neural Activity, Validation Results.''
        \textbf{(Panel I)} Validation summary in salmon-colored box: ``$\square$ COGNITIVE VALIDATION. $\checkmark$ Status: FAIL. Attention-Executive: 0.004 Required: $\geq 0.4$. WM-Consciousness: 0.004 Required: $\geq 0.35$. RT-Neural Correlation: 0.329 Expected: (-0.8, -0.2). Cognitive Coherence: 0.003 Required: $\geq 0.3$.'' All four validation criteria fail to meet thresholds. Annotation: ``$\square$ COGNITIVE VALIDATION, $\checkmark$ Status: FAIL, Attention-Executive: 0.004 Required: $\geq 0.4$, WM-Consciousness: 0.004 Required: $\geq 0.35$, RT-Neural Correlation: 0.329 Expected: (-0.8, -0.2), Cognitive Coherence: 0.003 Required: $\geq 0.3$.''
        }
        \label{fig:cognitive_processing}
        \end{figure*}

        \begin{figure*}[htbp]
        \centering
        \includegraphics[width=\textwidth]{figures/figure_gps_consciousness_geometry.png}
        \caption{
        \textbf{Paper 3: The geometry of consciousness as residual of perception-thought confluence across multiple scales.}
        \textbf{(Panel A)} Perception and thought curves in 3D state space. Axes: Dimension 1 ($0.0$--$1.0$), Dimension 2 ($0.0$--$1.0$), Dimension 3 ($0.0$--$1.0$). Blue trajectory (perception manifold) forms loop with points colored by dimension value (purple to yellow). Red trajectory (thought manifold) forms smaller inner loop. Red arrows connect corresponding points showing geometric separation. Green box annotation: ``Green lines = Consciousness (the gap between perception \& thought).'' Consciousness emerges as residual distance between manifolds. Annotation: ``A: Perception \& Thought Curves, Green lines = Consciousness (the gap between perception \& thought), Dimension 3, Dimension 2, Dimension 1.''
        \textbf{(Panel B)} Consciousness manifold in 3D showing residual magnitude. Axes: Consciousness X ($0.0$--$1.0$), Consciousness Y ($0.3$--$0.8$), Consciousness Z ($0.0$--$0.8$). Trajectory colored by consciousness intensity (purple $0.6$ to yellow $1.4$). Path forms complex loop with varying intensity. Peak intensity (yellow-green, $\sim 1.2$--$1.4$) at top-right. Lower intensity (purple-blue, $\sim 0.6$--$0.8$) at bottom-left. Yellow center point marks reference. Red box annotation: ``Consciousness = Geometric residual between perception and thought.'' Annotation: ``B: Consciousness Manifold, Consciousness = Geometric residual between perception and thought, Consciousness Z, Consciousness Y, Consciousness X, Consciousness Intensity (residual magnitude).''
        \textbf{(Panel C)} Consciousness trend over normalized time showing intensity ($0.0$--$1.4$) vs. time ($0.0$--$1.0$). Red trace with purple shading oscillates around mean $= 0.733$. Multiple sharp peaks (black stars) reach $\sim 1.0$ at $t \sim 0.05, 0.3, 0.6, 0.9$. Troughs drop to $\sim 0.5$ between peaks. Green box annotation: ``Consciousness Metrics: Mean: 0.733, Max: 1.454, Std: 0.166. High peaks = Moments of acute awareness.'' Annotation: ``C, Consciousness Metrics: Mean: 0.733, Max: 1.454, Std: 0.166, High peaks = Moments of acute awareness, Consciousness Trend, Consciousness Intensity, Normalized Time.''
        \textbf{(Panel D)} Multi-scale consciousness volume and intensity showing paired bars for eight precision levels. Left y-axis: Consciousness Volume ($0.00$--$0.12$, normalized). Right y-axis: Mean Intensity ($0.0$--$0.7$). Each level shows two bars: left bar (volume, colored by level), right bar (intensity, purple). GPS (red, volume $\sim 0.25$, intensity $\sim 0.6$), ns (orange, $\sim 0.25, 0.6$), ps (yellow, $\sim 0.25, 0.6$), fs (yellow, $\sim 0.25, 0.6$), as (green, $\sim 0.25, 0.6$), zs (blue, $\sim 0.02, 0.6$), Planck (purple, $\sim 0.12, 0.6$), Trans-P (pink, $\sim 0.00, 0.6$). Volume decreases at finer scales. Blue box annotation: ``Volume, Consciousness emerges from the residual. Finer precision = Richer consciousness structure.'' Annotation: ``D, Volume, Consciousness emerges from the residual, Finer precision = Richer consciousness structure, Consciousness Volume, Mean Intensity, GPS, ns, ps, fs, as, zs, Planck, Trans-P.''
        }
        \label{fig:consciousness_geometry}
        \end{figure*}

        \begin{figure*}[htbp]
        \centering
        \includegraphics[width=\textwidth]{figures/figure_gps_perception_geometry.png}
        \caption{
        \textbf{Paper 1: The geometry of perception showing manifold structure at GPS and femtosecond scales with curvature analysis.}
        \textbf{(Panel A)} Perception manifold (GPS) in 3D. Axes: Spatial X ($0.0$--$1.0$, normalized), Spatial Y ($0.0$--$1.0$, normalized), Velocity ($0.0$--$1.0$, normalized). Trajectory forms closed loop colored by velocity (purple $0$ to yellow $8$ m/s). Path shows higher velocity (yellow-green, $\sim 6$--$8$ m/s) at top-right, lower velocity (purple-blue, $\sim 0$--$2$ m/s) at bottom-left. Green box annotation: ``Perception = Spatial awareness + Motion awareness.'' Demonstrates perceptual state space at macroscopic scale. Annotation: ``A: Perception Manifold (GPS), Perception = Spatial awareness + Motion awareness, Velocity (normalized), Spatial Y (normalized), Spatial X (normalized), Velocity (m/s).''
        \textbf{(Panel B)} Perception manifold (Femtosecond) in 3D showing enhanced precision. Same axis structure as Panel A. Purple-magenta trajectory forms similar loop topology but reveals finer structure. Velocity gradient (purple $0$ to yellow $8$ m/s) shows smoother transitions. Yellow box annotation: ``Enhanced precision reveals finer perceptual structure.'' Higher temporal resolution exposes microscopic perception dynamics. Annotation: ``B: Perception Manifold (Femtosecond), Enhanced precision reveals finer perceptual structure, Velocity (normalized), Spatial Y (normalized), Spatial X (normalized), Velocity (m/s).''
        \textbf{(Panel C)} Perception curvature along path showing curvature ($0$--$140$) vs. normalized path distance ($0.0$--$3.0$). Red dots with pink trace show sparse high-curvature events. Most points cluster near zero ($< 10$). Sharp spike to $\sim 140$ at distance $\sim 0.4$. Secondary peaks ($\sim 20$) at distances $\sim 1.6, 2.0, 2.4$. Red circle marks highest curvature point. Green box annotation: ``High curvature = Sharp turns = Attention focus. Mean: 4.289, Max: 146.619.'' High curvature indicates decision points. Annotation: ``C, High curvature = Sharp turns = Attention focus, Mean: 4.289, Max: 146.619, High Curvature (Turns), Perception Curvature, Path Distance (normalized).''
        \textbf{(Panel D)} Perception volume vs. precision level showing eight colored bars. Y-axis: Perception Volume ($0.00$--$0.25$, normalized). X-axis: Precision Level (GPS, ns, ps, fs, as, zs, Planck, Trans-P). All bars show identical height ($\sim 0.25$) and color progression: red (GPS), orange (ns), yellow (ps), yellow (fs), green (as), blue (zs), purple (Planck), pink (Trans-P). Green box annotation: ``Perception volume changes with precision scale. Finer precision = Richer perceptual structure.'' Volume remains constant across scales, indicating scale-invariant geometry. Annotation: ``D, $0.2563$, $0.2563$, $0.2563$, $0.2563$, $0.2563$, $0.2563$, $0.2564$, $564$, Perception volume changes with precision scale, Finer precision = Richer perceptual structure, Perception Volume (normalized), GPS, ns, ps, fs, as, zs, Planck, Trans-P, Precision Level.''
        }
        \label{fig:perception_geometry}
        \end{figure*}

        \begin{figure*}[htbp]
        \centering
        \includegraphics[width=\textwidth]{figures/figure_gps_precision_cascade_2.png}
        \caption{
        \textbf{GPS track at multiple precision levels showing same physical path measured across four temporal scales spanning 47 orders of magnitude.}
        \textbf{(Panel A)} GPS level (ms precision) showing latitude ($0.0022$--$0.0034°$, $+4.818 \times 10^1$) vs. longitude ($0.00525$--$0.00675°$, $+1.135 \times 10^1$). Red dots ($n = 93$) form elliptical loop. Salmon box annotation: ``Points: 93, Precision: 1 ms, Uncertainty: $\sim$mm.'' Trajectory shows smooth path with uniform point spacing. Standard GPS measurement at millisecond temporal resolution. Annotation: ``$+4.818$e$1$, A: GPS Level (ms precision), Points: 93, Precision: 1 ms, Uncertainty: $\sim$mm, Latitude ($°$), Longitude ($°$), $+1.135$e$1$.''
        \textbf{(Panel B)} Picosecond level (ps precision) showing identical axes and coordinate range. Orange dots ($n = 93$) form same elliptical loop as Panel A. Yellow box annotation: ``Points: 93, Precision: 1 ps, Uncertainty: $\sim$pm.'' Path topology preserved at picometer spatial uncertainty. Temporal precision increased $10^9\times$ from GPS level. Annotation: ``$+4.818$e$1$, B: Picosecond Level (ps precision), Points: 93, Precision: 1 ps, Uncertainty: $\sim$pm, Latitude ($°$), Longitude ($°$), $+1.135$e$1$.''
        \textbf{(Panel C)} Attosecond level (as precision) showing same coordinate system. Green dots ($n = 93$) maintain elliptical loop structure. Green box annotation: ``Points: 93, Precision: 1 as, Uncertainty: $\sim$am.'' Attometer spatial resolution achieved. Temporal precision $10^{18}\times$ finer than GPS, $10^9\times$ finer than picosecond. Annotation: ``$+4.818$e$1$, C: Attosecond Level (as precision), Points: 93, Precision: 1 as, Uncertainty: $\sim$am, Latitude ($°$), Longitude ($°$), $+1.135$e$1$.''
        \textbf{(Panel D)} Trans-Planckian level ($< t_P$) showing same axes. Purple dots ($n = 93$) preserve loop geometry. Purple box annotation: ``Points: 93, Precision: $7.5 \times 10^{-60}$ s, Uncertainty: Sub-Planckian.'' Temporal precision exceeds Planck time ($5.4 \times 10^{-44}$ s) by $10^{16}$ orders. Spatial uncertainty below Planck length. Same physical path measured at quantum gravity scale. Annotation: ``$+4.818$e$1$, D: Trans-Planckian Level ($< t_P$), Points: 93, Precision: $7.5 \times 10^{-60}$ s, Uncertainty: Sub-Planckian, Latitude ($°$), Longitude ($°$), $+1.135$e$1$.''
        }
        \label{fig:gps_precision_cascade}
        \end{figure*}

        \begin{figure*}[htbp]
        \centering
        \includegraphics[width=\textwidth]{figures/figure_gps_precision_cascade_3.png}
        \caption{
        \textbf{The most measured 400m run in history: Dual-watch validation and 7-layer precision cascade from GPS to trans-Planckian scales.}
        \textbf{(Panel A)} Dual-watch GPS comparison showing latitude ($0.0022$--$0.0036°$, $+4.818 \times 10^1$) vs. longitude ($0.00500$--$0.00700°$, $+1.135 \times 10^1$). Blue circles (Watch 1, GARMIN, $n = 93$ pts) and red squares (Watch 2, COROS, $n = 48$ pts) trace same elliptical path. Watch 1 shows denser sampling. Both devices capture identical trajectory, validating physical consistency. Annotation: ``A, $+4.818$e$1$, Watch 1 (93 pts), Watch 2 (48 pts), Latitude ($°$), Longitude ($°$), $+1.135$e$1$.''
        \textbf{(Panel B)} Point count and velocity comparison. Left bars (blue, Number of Points, left y-axis $0$--$100$): Watch 1 ($93$ points), Watch 2 ($48$ points). Right bars (salmon, Mean Velocity, right y-axis $0$--$12$ m/s): Watch 1 ($4.32$ m/s), Watch 2 ($10.45$ m/s). Watch 2 shows $2.4\times$ higher velocity estimate despite $1.9\times$ fewer points. Annotation: ``B, $93$, $48$, $4.32$, $10.45$, Number of Points, Mean Velocity (m/s), Watch 1, Watch 2.''
        \textbf{(Panel C)} Precision cascade showing eight bars with constant point count. Y-axis: Number of Points ($0$--$100$). All bars show $93$ points (labeled at top). Colors progress: red (GPS), orange (ns), yellow (ps), yellow (fs), green (as), blue (zs), purple (Planck), pink (Trans-P). Demonstrates same physical event measured at eight temporal scales spanning $10^{60}$ orders of magnitude. Annotation: ``C, $93$, $93$, $93$, $93$, $93$, $93$, $93$, $93$, Number of Points, GPS, ns, ps, fs, as, zs, Planck, Trans-P, Precision Level.''
        \textbf{(Panel D)} Methodology summary in yellow box: ``THE MOST MEASURED 400m RUN IN HISTORY. DUAL-WATCH RECORDING: Watch 1 (GARMIN): 93 points, Watch 2 (COROS): 48 points, Same physical event, independent sensors. 7-LAYER PRECISION CASCADE: 1. GPS (millisecond) $\rightarrow$ mm uncertainty, 2. Nanosecond $\rightarrow$ nm uncertainty, 3. Picosecond $\rightarrow$ pm uncertainty, 4. Femtosecond $\rightarrow$ fm uncertainty, 5. Attosecond $\rightarrow$ am uncertainty, 6. Zeptosecond $\rightarrow$ zm uncertainty, 7. Planck $\rightarrow$ Planck length, 8. Trans-Planckian $\rightarrow$ Sub-Planckian. METHODOLOGY: Harmonic cascade refinement, Oscillatory-categorical equivalence, Molecular equilibrium restoration, Neural-cardiac-atmospheric coupling. VALIDATION: Dual-watch cross-validation, Multi-scale coherence, Physical consistency checks, No falls $\rightarrow$ Perception-action sync. This is not simulation. This is measured reality at unprecedented precision. Created: 2025-10-13T05:34:45.396686.'' Annotation: ``D, THE MOST MEASURED 400m RUN IN HISTORY.''
        }
        \label{fig:dual_watch_precision}
        \end{figure*}

        \begin{figure*}[htbp]
        \centering
        \includegraphics[width=\textwidth]{figures/figure_muscle_activation.png}
        \caption{
        \textbf{Muscle activation dynamics during running: Temporal patterns, correlation structure, and synergy coordination across six muscle groups.}
        \textbf{(Panel A)} Lower limb muscle activation showing three muscles over 60 seconds. Y-axis: Muscle Activation ($0.0$--$1.0$). Red trace (Quadriceps), blue trace (Hamstrings), green trace (Gastrocnemius) all oscillate between $0.0$--$0.6$ with period $\sim 1$ s (stride frequency). All three muscles show synchronized rhythmic activation. White box annotation: ``Quad Mean: 0.333, Hamstring Mean: 0.333, Gastro Mean: 0.333, Duration: 59.9 s.'' Perfect mean symmetry across muscle groups. Annotation: ``A, Quad Mean: 0.333, Hamstring Mean: 0.333, Gastro Mean: 0.333, Duration: 59.9 s, Quadriceps, Hamstrings, Gastrocnemius, Muscle Activation (0-1), Time (s).''
        \textbf{(Panel B)} Upper limb and stabilizer activation showing three muscles. Red trace (Hip Flexors), orange trace (Glutes), blue trace (Tibialis Anterior) oscillate between $0.0$--$0.6$ with same $\sim 1$ s period. Phase relationships differ from lower limb. White box annotation: ``Hip Flexor Mean: 0.333, Glute Mean: 0.333, Tibialis Mean: 0.334, Tibialis Anterior.'' Consistent mean activation across all muscle groups. Annotation: ``B, Hip Flexor Mean: 0.333, Glute Mean: 0.333, Tibialis Mean: 0.334, Tibialis Anterior, Muscle Activation (0-1), Time (s).''
        \textbf{(Panel C)} Muscle correlation matrix heatmap showing $6 \times 6$ structure. Rows/columns: Quadriceps, Hamstrings, Gastrocnemius, Hip Flexors, Glutes, Tibialis. Color scale: dark red ($1.00$, perfect correlation) to dark blue ($-1.00$, perfect anti-correlation). Diagonal shows self-correlation ($1.00$, dark red). Quadriceps-Hamstrings shows strong negative correlation ($-1.00$, dark blue, antagonistic). Quadriceps-Glutes positive ($\sim 1.00$, red, synergistic). Hamstrings-Hip Flexors negative ($-1.00$, blue). Complex pattern reveals coordinated muscle control. Numerical values labeled in cells. Annotation: ``C, Quadriceps, Hamstrings, Gastrocnemius, Hip Flexors, Glutes, Tibialis, Quadriceps, Hamstrings, Gastrocnemius, Hip Flexors, Glutes, Tibialis, $1.00$, $1.00$, $0.00$, $0.00$, $1.00$, $0.00$, $0.00$, $1.00$, $1.00$, $0.00$, $-1.00$, $-0.00$, $0.00$, $0.00$, $1.00$, $1.00$, $0.00$, $-1.00$, $0.00$, $0.00$, $1.00$, $1.00$, $0.00$, $1.00$, $0.00$, $-0.00$, $-1.00$, $1.00$, $0.00$, $1.00$, Correlation, $0.75$, $0.50$, $0.25$, $0.00$, $-0.25$, $-0.50$, $-0.75$, $-1.00$.''
        \textbf{(Panel D)} Synergy activation showing three traces over 60 seconds. Red trace (Extensor Synergy) oscillates between $0.2$--$0.6$ with peaks reaching $\sim 0.55$. Blue trace (Flexor Synergy) oscillates between $0.2$--$0.5$ with inverse phase. Pink shading (Co-activation) fills region between traces, indicating overlap. White box annotation: ``Extensor Mean: 0.333, Flexor Mean: 0.333, Co-activation: 0.221. Higher co-activation = Greater stability.'' Synergies show coordinated but distinct activation patterns. Annotation: ``D, Extensor Synergy, Flexor Synergy, Co-activation, Synergy Activation (0-1), Extensor Mean: 0.333, Flexor Mean: 0.333, Co-activation: 0.221, Higher co-activation = Greater stability, Time (s).''
        }
        \label{fig:muscle_activation}
        \end{figure*}


        \begin{figure*}[htbp]
            \centering
            \includegraphics[width=\textwidth]{figures/demo3_dynamic_coupling.png}
            \caption{
            \textbf{Force and coupling dynamics: Multi-dimensional analysis of force-activation relationships and state space trajectories.}
            \textbf{(Panel A)} Force and coupling dynamics showing two traces over 4 seconds. Blue trace (Force, left y-axis, $0$--$6000$ N) shows rapid rise to $\sim 6000$ N at $t \sim 0.5$ s, plateau until $t \sim 2.5$ s, then exponential decay to baseline by $t = 4$ s. Red trace (Coupling Strength, right y-axis, $0.0$--$0.6$) shows sharp peak to $\sim 0.6$ at $t \sim 0.3$ s, followed by gradual decline to $\sim 0.1$ during plateau, then drop to $\sim 0.05$ during force decay. Demonstrates temporal relationship between force generation and system coupling. Annotation: ``Force and Coupling Dynamics, Force (N), Coupling Strength, Time (s).''
            \textbf{(Panel B)} 3D state space trajectory showing path through knowledge-time-entropy space. Axes: Knowledge ($0.0$--$1.0$), Time ($0.00$--$1.50$), Entropy ($0.000$--$0.025$). Trajectory colored by time (purple $0.5$ to yellow $3.5$ s). Path starts at green circle (Start, low knowledge/entropy), spirals upward through teal-blue region (increasing knowledge/entropy), reaches peak at yellow region (high knowledge, $\sim 0.8$, entropy $\sim 0.020$), terminates at red square (End, knowledge $\sim 0.9$, entropy $\sim 0.010$). Black dots mark intermediate states. Demonstrates evolution through cognitive state space. Annotation: ``3D State Space Trajectory, Start, End, Entropy, Time, Knowledge, Time (s).''
            \textbf{(Panel C)} Force-activation phase space showing force ($0$--$6000$ N) vs. activation ($0.0$--$1.0$). Blue trajectory forms loop starting at origin, rising linearly to $\sim 2500$ N at activation $0.4$, continuing to peak at $\sim 6000$ N and activation $1.0$ (marked with red star, labeled ``Peak''), then descending along parallel path with brief dip to $\sim 1800$ N at activation $0.4$, returning to origin. Hysteresis loop indicates path-dependent force-activation relationship. Annotation: ``Force-Activation Phase Space, $\star$ Peak, Force (N), Activation.''
            \textbf{(Panel D)} Force vs. coupling showing force ($0$--$6000$ N) vs. coupling strength ($0.0$--$0.6$). Trajectory colored by time (yellow $0$ to purple $4$ s). Two distinct clusters: dense vertical band at coupling $\sim 0.1$ spanning force $1000$--$6000$ N (blue-purple, $t \sim 1$--$3$ s), and sparse vertical band at coupling $\sim 0.55$ spanning force $0$--$2500$ N (green-yellow, $t \sim 0$--$1$ s). Small isolated cluster at coupling $\sim 0.05$, force $\sim 100$ N (purple, $t \sim 4$ s). Bimodal coupling distribution suggests distinct dynamical regimes. Annotation: ``Force vs Coupling, Force (N), Coupling Strength, Time (s).''
            }
            \label{fig:dynamic_coupling}
            \end{figure*}

            \begin{figure*}[htbp]
            \centering
            \includegraphics[width=\textwidth]{figures/figure_dual_watch_comparison.png}
            \caption{
            \textbf{Dual-watch validation summary: Cross-device comparison of normalized metrics, absolute measurements, device ratios, and agreement analysis.}
            \textbf{(Panel A)} Normalized value comparison showing four metrics. Y-axis: Normalized Value ($0.0$--$2.5$). Four paired bars (COROS blue, GARMIN salmon): Frame Rate (Hz) both $2.00$, Perception Bandwidth (COROS $2.36$, GARMIN $2.30$, nearly equal), Neural Efficiency (COROS $1.33$, GARMIN $1.38$), Total Frames (COROS $0.94$, GARMIN $1.84$, largest difference). Values labeled above bars. GARMIN shows higher total frames, other metrics comparable. Annotation: ``A, $2.36$, $2.30$, $2.00$, $2.00$, $1.84$, $1.38$, $1.33$, $0.94$, COROS, GARMIN, Normalized Value, Frame Rate (Hz), Perception Bandwidth, Neural Efficiency, Total Frames.''
            \textbf{(Panel B)} Measurement value comparison showing four absolute metrics. Y-axis: Measurement Value ($0$--$7500$). Four paired bars: Air Mass (kg) both $\sim 500$ kg (COROS green $\sim 400$, GARMIN orange $\sim 800$), Wake Volume (m³) both $\sim 1500$ m³, Energy (J) shows large difference (COROS green $\sim 3700$ J, GARMIN orange $\sim 7300$ J, tallest bars), Reynolds Number both $\sim 350$. Energy shows $2\times$ difference between devices. Annotation: ``B, COROS, GARMIN, Measurement Value, Air Mass (kg), Wake Volume (m³), Energy (J), Reynolds Number.''
            \textbf{(Panel C)} GARMIN/COROS ratio showing five metrics. Y-axis: GARMIN / COROS Ratio ($0.0$--$2.5$). Red dashed line marks perfect agreement at $1.0$. Green shaded region shows $\pm 10$\% range ($0.9$--$1.1$). Five bars (salmon, except two green): Duration Ratio ($1.957$, above range), Frame Rate Ratio ($1.000$, perfect, green), Neural Eff. Ratio ($1.042$, within range, green), Air Mass Ratio ($1.966$, above range), Energy Ratio ($1.978$, above range). Three metrics exceed $10$\% tolerance. Values labeled above bars. Annotation: ``C, $1.957$, $2.0$, $1.966$, $1.978$, $1.000$, $1.042$, GARMIN / COROS Ratio, -- Perfect Agreement, $\pm 10$\% Range, Duration Ratio, Frame Rate Ratio, Neural Eff. Ratio, Air Mass Ratio, Energy Ratio.''
            \textbf{(Panel D)} Measurement agreement analysis text box with blue background: ``MEASUREMENT AGREEMENT ANALYSIS. Dual-Watch Validation Summary. Mean Ratio: 1.589, Std Deviation: 0.464, Coefficient of Var: 29.19\%. Agreement Score: 0.40 / 1.00. COROS Watch: Duration: 47.0 s, Datapoints: 48, Focus: Consciousness metrics. GARMIN Watch: Duration: 92.0 s, Datapoints: 93, Focus: Atmospheric dynamics. VALIDATION STATUS: $\triangle$ REVIEW. Both watches measured the same physical event with complementary sensor arrays. High agreement validates methodology.'' Moderate agreement score indicates systematic differences between devices. Annotation: ``D, MEASUREMENT AGREEMENT ANALYSIS, Dual-Watch Validation Summary, Mean Ratio: 1.589, Std Deviation: 0.464, Coefficient of Var: 29.19\%, Agreement Score: 0.40 / 1.00, COROS Watch: Duration: 47.0 s, Datapoints: 48, Focus: Consciousness metrics, GARMIN Watch: Duration: 92.0 s, Datapoints: 93, Focus: Atmospheric dynamics, VALIDATION STATUS: $\triangle$ REVIEW, Both watches measured the same physical event with complementary sensor arrays. High agreement validates methodology.''
            }
            \label{fig:dual_watch_comparison}
            \end{figure*}

            \begin{figure*}[htbp]
            \centering
            \includegraphics[width=\textwidth]{figures/figure_neural_resonance_1_bands.png}
            \caption{
            \textbf{Neural resonance analysis: Oscillatory band frequencies, resonance quality, cardiac coupling, and multi-band synchronization during running.}
            \textbf{(Panel A)} Neural oscillatory bands during running showing frequency distribution on log scale. Y-axis: Band labels (Delta, Theta, Alpha, Beta, Gamma, High-$\gamma$). X-axis: Frequency (Hz, $10^0$--$10^2$, log scale). Six horizontal bars span frequency ranges: Delta (maroon, $2.2$ Hz, labeled), Theta (orange, $6.0$ Hz), Alpha (yellow, $10.5$ Hz), Beta (green, $21.5$ Hz), Gamma (blue, $65.0$ Hz), High-$\gamma$ (purple, $150.0$ Hz). Frequencies span $2.2$--$150$ Hz range ($68\times$ span). Blue box annotation: ``Neural oscillatory bands during running.'' Annotation: ``A, Neural oscillatory bands during running, $150.0$ Hz, $65.0$ Hz, $21.5$ Hz, $10.5$ Hz, $6.0$ Hz, $2.2$ Hz, High-$\gamma$, Gamma, Beta, Alpha, Theta, Delta, Frequency (Hz), $10^0$, $10^1$, $10^2$.''
            \textbf{(Panel B)} Resonance quality showing six bars. Y-axis: Resonance Quality ($0.0$--$1.0$). Bars with values: Delta (maroon, $0.65$), Theta (orange, $0.78$), Alpha (yellow, $0.85$), Beta (green, $0.92$, highest), Gamma (blue, $0.88$), High-$\gamma$ (purple, $0.75$). Red dashed line marks threshold at $0.8$. Yellow dashed line at $0.8$. Blue dashed line at $0.75$. Green box annotation: ``Beta band shows highest resonance (motor control).'' Beta exceeds threshold, indicating strongest motor coupling. Annotation: ``B, $0.92$, $0.88$, $0.85$, $0.78$, $0.65$, $0.75$, Beta band shows highest resonance (motor control), Resonance Quality, Delta, Theta, Alpha, Beta, Gamma, High-$\gamma$, High Resonance Threshold.''
            \textbf{(Panel C)} Cardiac coupling strength showing scatter plot with harmonic structure. Y-axis: Cardiac Coupling Strength ($0.0$--$1.0$). X-axis: Neural Frequency (Hz, $10^0$--$10^2$, log scale). Vertical red dashed lines mark harmonics of cardiac frequency ($2.32$ Hz). Five labeled circles: Theta (orange, $\sim 6$ Hz, strength $\sim 0.45$), Alpha (yellow, $\sim 10$ Hz, $\sim 0.40$), Beta (green, $\sim 20$ Hz, $\sim 0.60$), Gamma (blue, $\sim 65$ Hz, $\sim 0.95$), High-$\gamma$ (purple, $\sim 150$ Hz, $\sim 0.50$). Gamma shows strongest cardiac coupling. Blue box annotation: ``Cardiac freq: 2.32 Hz. Red lines: Harmonics.'' Annotation: ``C, Cardiac freq: 2.32 Hz, Red lines: Harmonics, Gamma, Beta, High-$\gamma$, Theta, Alpha, Cardiac Coupling Strength, Neural Frequency (Hz), $10^0$, $10^1$, $10^2$.''
            \textbf{(Panel D)} Multi-band synchronization showing four oscillating traces over 2 seconds. Y-axis: Amplitude (offset). Four colored traces: Red (Cardiac, $2.3$ Hz, period $\sim 0.43$ s, largest amplitude, slowest), Yellow (Alpha, $10$ Hz, period $\sim 0.1$ s, medium amplitude), Green (Beta, $20$ Hz, period $\sim 0.05$ s, smaller amplitude), Blue (Gamma, $40$ Hz, period $\sim 0.025$ s, smallest amplitude, fastest). Red dashed vertical lines mark cardiac cycle boundaries. Yellow box annotation: ``All neural bands synchronize to cardiac rhythm.'' Phase alignment at cardiac peaks demonstrates cross-frequency coupling. Annotation: ``D, Cardiac ($2.3$ Hz), Alpha ($10$ Hz), Beta ($20$ Hz), Gamma ($40$ Hz), Amplitude (offset), All neural bands synchronize to cardiac rhythm, Time (s).''
            }
            \label{fig:neural_resonance_bands}
            \end{figure*}

            \begin{figure*}[htbp]
            \centering
            \includegraphics[width=\textwidth]{figures/figure_neural_resonance_2_integration.png}
            \caption{
            \textbf{Multi-scale neural resonance integration: Frequency hierarchy, phase-lock matrix, resonance quality dynamics, and consciousness gauge.}
            \textbf{(Panel A)} Frequency span showing eight horizontal bars on log scale. Y-axis: Scale labels (Breathing, Stride, Cardiac, Neural $\alpha$, Neural $\beta$, Neural $\gamma$, Cellular, Molecular). X-axis: $\log_{10}$(Frequency) [Hz] ($0$--$12$). Bars span: Breathing (maroon, $\sim 0.2$ Hz, $\log \sim 0$), Stride (orange, $\sim 1.88$ Hz, labeled, $\log \sim 0.3$), Cardiac (salmon, $2.30$ Hz, $\log \sim 0.4$), Neural $\alpha$ (pink, $10.00$ Hz, $\log \sim 1$), Neural $\beta$ (purple, $20.00$ Hz, $\log \sim 1.3$), Neural $\gamma$ (purple, $40.00$ Hz, $\log \sim 1.6$), Cellular (purple, $1$ MHz, $\log \sim 6$), Molecular (dark blue, $1$ THz, $\log \sim 12$, longest). Green box annotation: ``Frequency Span: 12.6 orders of magnitude. All synchronized!'' Demonstrates coherent coupling across $10^{12}$ frequency range. Annotation: ``A, Frequency Span: 12.6 orders of magnitude, All synchronized!, Breathing, Stride, Cardiac, Neural $\alpha$, Neural $\beta$, Neural $\gamma$, Cellular, Molecular, $1.88$ Hz, $2.30$ Hz, $10.00$ Hz, $20.00$ Hz, $40.00$ Hz, $1$ MHz, $1$ THz, $\log_{10}$(Frequency) [Hz].''
            \textbf{(Panel B)} Phase-lock strength matrix heatmap showing $8 \times 8$ structure. Rows/columns: Molecular, Cellular, Neural $\gamma$, Neural $\beta$, Neural $\alpha$, Cardiac, Stride, Breathing. Color scale: dark red ($1.0$, strong phase-lock) to white ($0.0$, no phase-lock). Diagonal shows self-locking ($1.0$, dark red). Strong off-diagonal coupling: Molecular-Cellular ($\sim 0.9$, red), Neural bands mutually coupled ($\sim 0.8$, red-orange), Cardiac-Stride ($\sim 0.7$, orange), Breathing-Stride ($\sim 0.9$, red). Hierarchical structure visible with stronger coupling between adjacent scales. Annotation: ``B, Molecular, Cellular, Neural $\gamma$, Neural $\beta$, Neural $\alpha$, Cardiac, Stride, Breathing, Molecular, Cellular, Neural $\gamma$, Neural $\beta$, Neural $\alpha$, Cardiac, Stride, Breathing, Phase-Lock Strength, $1.0$, $0.8$, $0.6$, $0.4$, $0.2$, $0.0$.''
            \textbf{(Panel C)} Neural resonance quality over time showing quality ($0.0$--$1.0$) vs. time ($0$--$60$ s). Blue trace with cyan shading shows three phases: Initialization (pink background, $0$--$10$ s, rapid rise from $\sim 0.5$ to $\sim 0.75$), Steady State (green background, $10$--$50$ s, oscillations around $\sim 0.85$ with two broad peaks at $t \sim 20$ s and $t \sim 40$ s), Fatigue Onset (beige background, $50$--$60$ s, gradual decline to $\sim 0.80$). Red dashed line marks Target Resonance at $0.8$. Quality remains above target throughout. Legend shows phase labels. Annotation: ``C, Neural Resonance Quality, Initialization, Steady State, Fatigue Onset, Target Resonance, Time (s).''
            \textbf{(Panel D)} Consciousness gauge showing semicircular dial. Arc spans from $0.0$ (Coma, left, gray) through $0.5$ (Sleep, top-left) to $1.0$ (Peak, right, green). Red needle points to $\sim 0.92$ (upper-right, green zone). Large yellow box displays ``$0.92$'' with label ``Consciousness Level (Running)'' below. High consciousness level indicates optimal neural integration during running. Annotation: ``D, $0.5$ Sleep, $0.0$ Coma, $1.0$ Peak, $0.92$, Consciousness Level (Running).''
            }
            \label{fig:neural_resonance_integration}
            \end{figure*}

            \begin{figure*}[htbp]
            \centering
            \includegraphics[width=\textwidth]{figures/master_figure_3_empirical_validation.png}
            \caption{
            \textbf{Empirical validation: Real data supports consciousness framework through thought signatures, heartbeat-perception coupling, consciousness intensity timeline, and biomechanical correlations.}
            \textbf{(Panel A)} Thought signatures from real biomechanics showing thought complexity ($0.0$--$1.0$, normalized) over 60 seconds. Red bars with purple shading show periodic spikes. Black stars mark 19 thought events (labeled ``Thought Events ($n=19$)'') with varying complexity. Peaks reach $\sim 1.0$ at $t \sim 5, 35, 45, 55$ s. Yellow box annotation: ``Duration: 59.9 s, Thought Events: 19, Mean Complexity: 0.213, Peak Complexity: 1.000.'' Formula: $C = \sqrt{a^2 + j^2}$. Demonstrates quantifiable thought signatures from biomechanical data. Annotation: ``A: Thought Signatures from Real Biomechanics $C = \sqrt{a^2 + j^2}$, Duration: 59.9 s, Thought Events: 19, Mean Complexity: 0.213, Peak Complexity: 1.000, Thought Complexity, Thought Events ($n=19$), Thought Complexity (normalized), Time (s).''
            \textbf{(Panel B)} Heartbeat-perception coupling equilibrium restoration showing gas molecular equilibrium ($0.65$--$1.05$) over 10 seconds. Blue shading with red dashed vertical lines marking heartbeats (period $\sim 0.43$ s). Green dashed horizontal line marks Perfect Equilibrium at $1.0$. Signal oscillates between $\sim 0.75$--$1.00$ with rapid restoration after each heartbeat perturbation. Yellow box annotation: ``Heart Rate: 2.32 Hz, RR Interval: 431.1 ms, Restoration: 0.502 ms, Perception Rate: 1993 Hz.'' Demonstrates molecular equilibrium restoration coupling to cardiac cycle. Annotation: ``B: Heartbeat-Perception Coupling Equilibrium Restoration, Heart Rate: 2.32 Hz, RR Interval: 431.1 ms, Restoration: 0.502 ms, Perception Rate: 1993 Hz, Gas Molecular Equilibrium, Perfect Equilibrium, Time (s).''
            \textbf{(Panel C)} Consciousness intensity timeline showing intensity ($0.0$--$1.0$) over 60 seconds. Purple bars with pink shading show high-frequency oscillations. Red trace shows trend oscillating around $\sim 0.35$ with peaks at $t \sim 5, 35, 55$ s reaching $\sim 0.40$. Yellow box annotation: ``Mean Intensity: 0.354, Peak Intensity: 1.000, Std Dev: 0.204.'' Formula: $|C| = ||P - T||$. Legend shows Consciousness Intensity and Trend. High-frequency fluctuations indicate moment-to-moment consciousness dynamics. Annotation: ``C: Consciousness Intensity Timeline $|C| = ||P - T||$, Mean Intensity: 0.354, Peak Intensity: 1.000, Std Dev: 0.204, Consciousness Intensity, Trend, Consciousness Intensity, Time (s).''
            \textbf{(Panel D)} Variable correlation matrix showing consciousness correlates with biomechanics. Heatmap: $7 \times 7$ structure. Rows/columns: Hip, Knee, Ankle, Quad, Ham, Gastro, Consciousness. Color scale: dark red ($1.00$, perfect positive) to dark blue ($-1.00$, perfect negative). Diagonal shows self-correlation ($1.00$, dark red). Strong correlations: Hip-Quad ($0.71$, orange), Ankle-Hip ($0.71$, orange), Quad-Ham ($-1.00$, dark blue, antagonistic). Consciousness row shows: Hip ($-0.00$), Knee ($0.98$, strong positive, red), Ankle ($0.03$), Quad ($-0.00$), Ham ($0.00$), Gastro ($0.04$). Consciousness strongly correlates with knee angle, indicating biomechanical coupling. Values labeled in cells. Yellow vertical band highlights consciousness column. Annotation: ``D: Variable Correlation Matrix Consciousness Correlates with Biomechanics, Hip, Knee, Ankle, Quad, Ham, Gastro, Consciousness, Hip, Knee, Ankle, Quad, Ham, Gastro, Consciousness, $1.00$, $-0.00$, $0.71$, $1.00$, $-1.00$, $0.00$, $-0.00$, $-0.00$, $1.00$, $0.00$, $-0.00$, $0.00$, $0.00$, $-0.98$, $0.71$, $0.00$, $1.00$, $0.71$, $-0.71$, $0.71$, $0.03$, $1.00$, $-0.00$, $0.71$, $1.00$, $-1.00$, $0.00$, $-0.00$, $-1.00$, $0.00$, $-0.71$, $-1.00$, $1.00$, $0.00$, $0.00$, $0.00$, $0.00$, $0.71$, $0.00$, $0.00$, $1.00$, $0.04$, $-0.00$, $0.98$, $0.03$, $-0.00$, $0.00$, $0.04$, $1.00$, Pearson Correlation, $1.00$, $0.75$, $0.50$, $0.25$, $0.00$, $-0.25$, $-0.50$, $-0.75$, $-1.00$.''
            }
            \label{fig:empirical_validation}
            \end{figure*}
