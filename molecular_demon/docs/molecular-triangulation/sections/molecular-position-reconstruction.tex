%==============================================================================
% Molecular Position Reconstruction via Inverse S-Entropy Mapping
%==============================================================================

\subsection{The Inverse Problem: From Partition to Position}

\begin{principle}[Position-Partition Duality]
\label{prin:position_partition_duality}
Spatial position and partition state are dual descriptions: position determines partition state through atmospheric physics; partition state determines position through inverse mapping.
\end{principle}

The forward mapping (position to partition):
\begin{equation}
\Pi: \mathbf{r} = (x, y, z) \mapsto \sigma = (S_k, S_t, S_e)
\end{equation}

The inverse mapping (partition to position):
\begin{equation}
\Pi^{-1}: \sigma = (S_k, S_t, S_e) \mapsto \mathbf{r} = (x, y, z)
\end{equation}

\begin{theorem}[Inverse Mapping Existence]
\label{thm:inverse_existence}
The inverse mapping $\Pi^{-1}$ exists and is unique almost everywhere under standard atmospheric conditions.
\end{theorem}

\begin{proof}
The forward mapping $\Pi$ is determined by atmospheric physics:
\begin{align}
S_k(\mathbf{r}) &= f_k(T(\mathbf{r}), P(\mathbf{r}), \text{composition}(\mathbf{r})) \\
S_t(\mathbf{r}) &= f_t(\mathbf{v}(\mathbf{r}), T(\mathbf{r})) \\
S_e(\mathbf{r}) &= f_e(E(\mathbf{r}), T(\mathbf{r}))
\end{align}

These functions are smooth (infinitely differentiable) under standard conditions. The Jacobian:
\begin{equation}
J_\Pi = \frac{\partial(S_k, S_t, S_e)}{\partial(x, y, z)}
\end{equation}

By the inverse function theorem, $\Pi^{-1}$ exists locally wherever $\det(J_\Pi) \neq 0$.

Atmospheric gradients ensure non-degeneracy:
\begin{itemize}
\item Vertical: $\partial T/\partial z \approx -6.5$ K/km (lapse rate)
\item Horizontal: $|\nabla_H T| \approx 10^{-5}$-$10^{-3}$ K/m (weather systems)
\item Composition: $\partial X_{\text{H}_2\text{O}}/\partial z \neq 0$ (humidity gradient)
\end{itemize}

These gradients guarantee $\det(J_\Pi) \neq 0$ except at isolated singular points (atmospheric fronts, inversions).

By Theorem \ref{thm:signature_uniqueness}, signatures are unique with probability $> 1 - 10^{-15}$, ensuring global invertibility almost everywhere.
\end{proof}

\subsection{Explicit Inverse Mapping Construction}

\begin{definition}[Inverse S-Entropy Mapping]
\label{def:inverse_mapping}
The inverse mapping is constructed through iterative refinement:
\begin{equation}
\mathbf{r}^{(n+1)} = \mathbf{r}^{(n)} - J_\Pi^{-1}(\Pi(\mathbf{r}^{(n)}) - \sigma_{\text{target}})
\end{equation}
where $\sigma_{\text{target}}$ is the measured S-entropy state.
\end{definition}

\begin{algorithm}[H]
\caption{Inverse S-Entropy Mapping}
\label{alg:inverse_mapping}
\begin{algorithmic}[1]
\State \textbf{Input:} Target S-entropy $\sigma_{\text{target}} = (S_k, S_t, S_e)$
\State \textbf{Output:} Spatial position $\mathbf{r} = (x, y, z)$
\State
\State \textbf{Phase 1: Initial Guess from Lookup Table}
\State Query precomputed table: $\mathbf{r}^{(0)} = \text{LUT}(\sigma_{\text{target}})$
\State
\State \textbf{Phase 2: Newton-Raphson Refinement}
\For{$n = 0, 1, 2, \ldots$ until convergence}
    \State Compute forward mapping: $\sigma^{(n)} = \Pi(\mathbf{r}^{(n)})$
    \State Compute residual: $\delta\sigma = \sigma_{\text{target}} - \sigma^{(n)}$
    \If{$\|\delta\sigma\| < \epsilon_{\text{tol}}$}
        \State \textbf{break}
    \EndIf
    \State Compute Jacobian: $J = J_\Pi(\mathbf{r}^{(n)})$
    \State Update: $\mathbf{r}^{(n+1)} = \mathbf{r}^{(n)} + J^{-1} \delta\sigma$
\EndFor
\State
\Return $\mathbf{r} = \mathbf{r}^{(n)}$
\end{algorithmic}
\end{algorithm}

Convergence is typically achieved in 3-5 iterations due to smoothness of atmospheric fields.

\subsection{Atmospheric Jacobian Computation}

The Jacobian matrix encodes how partition state varies with position:
\begin{equation}
J_\Pi = \begin{pmatrix}
\partial S_k/\partial x & \partial S_k/\partial y & \partial S_k/\partial z \\
\partial S_t/\partial x & \partial S_t/\partial y & \partial S_t/\partial z \\
\partial S_e/\partial x & \partial S_e/\partial y & \partial S_e/\partial z
\end{pmatrix}
\end{equation}

\textbf{Vertical component} ($z$-derivatives):
\begin{align}
\frac{\partial S_k}{\partial z} &\approx \frac{1}{\omega_{\max} - \omega_{\min}} \frac{\partial \omega}{\partial T} \frac{\partial T}{\partial z} \approx -10^{-5} \text{ m}^{-1} \\
\frac{\partial S_t}{\partial z} &\approx \frac{1}{v_{\max}} \frac{\partial v}{\partial T} \frac{\partial T}{\partial z} \approx -10^{-6} \text{ m}^{-1} \\
\frac{\partial S_e}{\partial z} &\approx \frac{1}{E_{\max}} \frac{\partial E}{\partial T} \frac{\partial T}{\partial z} \approx -10^{-5} \text{ m}^{-1}
\end{align}

\textbf{Horizontal components} ($x$, $y$-derivatives):

Typically $10^2$-$10^3$ times smaller than vertical, but non-zero due to weather systems:
\begin{equation}
|\nabla_H \sigma| \approx 10^{-7} \text{-} 10^{-5} \text{ m}^{-1}
\end{equation}

\subsection{Molecular Ensemble Reconstruction}

\begin{definition}[Molecular Ensemble]
\label{def:molecular_ensemble}
An atmospheric volume $V$ contains $N = n V$ molecules, where $n = 2.5 \times 10^{25}$ m$^{-3}$ is number density at sea level.
\end{definition}

Full atmospheric reconstruction would require tracking $N_{\text{atm}} \approx 10^{44}$ molecules---computationally impossible.

\begin{theorem}[Representative Sampling Sufficiency]
\label{thm:representative_sampling}
Macroscopic thermodynamic properties can be reconstructed from $N_{\text{rep}} \sim 10^6$ representative molecules, reducing computational requirements by factor $\sim 10^{38}$.
\end{theorem}

\begin{proof}
Thermodynamic properties are ensemble averages:
\begin{align}
T &= \frac{2}{3\kB} \langle E_{\text{kin}} \rangle = \frac{2}{3\kB} \frac{1}{N} \sum_{i=1}^N \frac{1}{2}m v_i^2 \\
P &= n \kB T \\
\rho &= n m
\end{align}

By central limit theorem, sample mean converges to ensemble mean:
\begin{equation}
\langle E_{\text{kin}} \rangle_{\text{sample}} = \langle E_{\text{kin}} \rangle_{\text{true}} \pm \frac{\sigma}{\sqrt{N_{\text{sample}}}}
\end{equation}

For temperature accuracy $\Delta T/T = 10^{-3}$ (0.3 K at 300 K):
\begin{equation}
N_{\text{sample}} \geq \left(\frac{\sigma/\langle E_{\text{kin}} \rangle}{10^{-3}}\right)^2 \approx 10^6
\end{equation}

Therefore $N_{\text{rep}} \sim 10^6$ molecules suffice for 0.1\% thermodynamic accuracy.
\end{proof}

\subsection{Molecular Position Reconstruction Algorithm}

\begin{algorithm}[H]
\caption{Atmospheric Molecular Ensemble Reconstruction}
\label{alg:ensemble_reconstruction}
\begin{algorithmic}[1]
\State \textbf{Input:} Atmospheric column S-entropy profile $\Sigma(z)$, volume $V$
\State \textbf{Output:} Representative molecular ensemble $\{(\mathbf{r}_i, \mathbf{v}_i, E_i)\}_{i=1}^{N_{\text{rep}}}$
\State
\State \textbf{Phase 1: Altitude Stratification}
\State Divide column into $N_z$ altitude layers
\For{each layer $j = 1$ to $N_z$}
    \State Extract layer S-entropy: $\sigma_j = (S_{k,j}, S_{t,j}, S_{e,j})$
    \State Compute thermodynamic state: $(T_j, P_j, \rho_j) = \mathcal{T}(\sigma_j)$
\EndFor
\State
\State \textbf{Phase 2: Molecular Sampling}
\For{each layer $j$}
    \State Compute layer molecules: $N_j = \rho_j V_j / m$
    \State Sample $N_{\text{rep},j} = N_{\text{rep}} \times (N_j / N_{\text{total}})$ representatives
    \For{$i = 1$ to $N_{\text{rep},j}$}
        \State Sample position: $\mathbf{r}_i \sim \text{Uniform}(V_j)$
        \State Sample velocity: $\mathbf{v}_i \sim \text{Maxwell}(T_j)$
        \State Sample internal energy: $E_i \sim \text{Boltzmann}(T_j)$
    \EndFor
\EndFor
\State
\State \textbf{Phase 3: Consistency Verification}
\State Compute reconstructed S-entropy: $\hat{\sigma} = \Pi(\{\mathbf{r}_i, \mathbf{v}_i, E_i\})$
\State Verify: $\|\hat{\sigma} - \sigma_{\text{measured}}\| < \epsilon$
\State
\Return $\{(\mathbf{r}_i, \mathbf{v}_i, E_i)\}_{i=1}^{N_{\text{rep}}}$
\end{algorithmic}
\end{algorithm}

\subsection{From S-Entropy to Thermodynamic State}

The thermodynamic reconstruction operator $\mathcal{T}$:
\begin{equation}
\mathcal{T}: (S_k, S_t, S_e) \mapsto (T, P, \rho, \mathbf{v}, X_i)
\end{equation}

\begin{theorem}[Thermodynamic Reconstruction]
\label{thm:thermo_reconstruction}
Given S-entropy coordinates, thermodynamic state variables are uniquely determined through:
\begin{align}
T &= T_0 \exp\left(\frac{S_e}{c_V/\kB}\right) \\
P &= P_0 \exp\left(\frac{S_k + S_e}{R/\kB}\right) \\
\rho &= \frac{P M}{R T} \\
|\mathbf{v}| &= v_{\text{max}} \cdot S_t \\
X_i &= f_{\text{composition}}(S_k, T, P)
\end{align}
where $T_0$, $P_0$ are reference values and $M$ is mean molecular mass.
\end{theorem}

\begin{proof}
The S-entropy coordinates encode thermodynamic information through:

\textbf{$S_e$ (evolution entropy)}: Encodes internal energy distribution
\begin{equation}
S_e = \frac{E - E_{\min}}{E_{\max} - E_{\min}} \propto \ln(T/T_0)
\end{equation}

Inverting:
\begin{equation}
T = T_0 \exp(S_e / \alpha_e)
\end{equation}
where $\alpha_e = c_V/(E_{\max} - E_{\min})$.

\textbf{$S_k$ (kinetic entropy)}: Encodes vibrational state, hence composition and temperature
\begin{equation}
S_k = f(\{\omega_i\}, T) = f(\text{composition}, T)
\end{equation}

Combined with $T$ from $S_e$, composition is determined.

\textbf{$S_t$ (temporal entropy)}: Encodes velocity distribution
\begin{equation}
S_t \propto \langle v \rangle / v_{\text{max}}
\end{equation}

Given Maxwell distribution at temperature $T$:
\begin{equation}
\langle v \rangle = \sqrt{\frac{8\kB T}{\pi m}}
\end{equation}

The mapping is invertible because $(T, P, \rho, \mathbf{v}, X_i)$ uniquely determine $(S_k, S_t, S_e)$ through the forward definitions, and the inverse relations are explicitly constructible.
\end{proof}

\subsection{Spatial Resolution of Molecular Reconstruction}

\begin{theorem}[Reconstruction Spatial Resolution]
\label{thm:reconstruction_resolution}
Molecular ensemble reconstruction achieves spatial resolution:
\begin{equation}
\delta r_{\text{recon}} = \left(\frac{V}{N_{\text{rep}}}\right)^{1/3} \approx 1 \text{ m}
\end{equation}
for $N_{\text{rep}} = 10^6$ molecules in $V = 1$ km$^3$ volume.
\end{theorem}

Higher resolution requires larger $N_{\text{rep}}$:
\begin{center}
\begin{tabular}{ccc}
\toprule
$\boldsymbol{N_{\text{rep}}}$ & \textbf{Resolution} & \textbf{Computational Cost} \\
\midrule
$10^6$ & 1 m & 1 s \\
$10^9$ & 10 cm & 1000 s \\
$10^{12}$ & 1 cm & $10^6$ s \\
\bottomrule
\end{tabular}
\end{center}

For weather prediction, 1 m resolution suffices. For local atmospheric phenomena (turbulence, microclimate), 10 cm resolution is achievable with moderate computational resources.

\subsection{Verification Through Forward Consistency}

\begin{definition}[Forward Consistency Check]
\label{def:forward_consistency}
Reconstructed ensemble $\{\mathbf{r}_i, \mathbf{v}_i, E_i\}$ is verified by computing its S-entropy signature and comparing to measured values.
\end{definition}

Forward computation:
\begin{align}
\hat{S}_k &= \frac{1}{N_{\text{rep}}} \sum_i f_k(\omega_i, E_i) \\
\hat{S}_t &= \frac{1}{N_{\text{rep}}} \sum_i f_t(\mathbf{v}_i) \\
\hat{S}_e &= \frac{1}{N_{\text{rep}}} \sum_i f_e(E_i)
\end{align}

Consistency criterion:
\begin{equation}
\|\hat{\sigma} - \sigma_{\text{measured}}\| < \epsilon_{\text{consistency}}
\end{equation}

If violated, reconstruction is refined through iterative adjustment of molecular parameters.

\subsection{Connection to Weather Prediction}

The reconstructed molecular ensemble provides the complete microstate necessary for weather prediction:

\begin{enumerate}
\item \textbf{Initial conditions}: Molecular positions and velocities at $t = 0$
\item \textbf{Dynamics}: Partition evolution equations (Section \ref{sec:weather_prediction})
\item \textbf{Observables}: Macroscopic properties from ensemble averaging
\end{enumerate}

This establishes the bridge from categorical GPS (position from partition) to weather prediction (atmospheric evolution from partition dynamics), unified through the molecular reconstruction framework.
