\documentclass[12pt,a4paper]{article}

\usepackage[utf8]{inputenc}
\usepackage[T1]{fontenc}
\usepackage{amsmath,amssymb,amsfonts,amsthm}
\usepackage{mathtools}
\usepackage{geometry}
\usepackage{graphicx}
\usepackage{float}
\usepackage{booktabs}
\usepackage{hyperref}
\usepackage{cleveref}
\DeclareMathOperator*{\argmax}{arg\,max}
\usepackage{physics}
\usepackage{natbib}
\usepackage{tikz-cd}

\geometry{margin=1in}

% Theorem environments
\newtheorem{theorem}{Theorem}[section]
\newtheorem{lemma}[theorem]{Lemma}
\newtheorem{proposition}[theorem]{Proposition}
\newtheorem{corollary}[theorem]{Corollary}
\theoremstyle{definition}
\newtheorem{definition}[theorem]{Definition}
\newtheorem{axiom}[theorem]{Axiom}
\theoremstyle{remark}
\newtheorem{remark}[theorem]{Remark}
\newtheorem{example}[theorem]{Example}

% Custom commands
\newcommand{\phaselockgraph}{\mathcal{G}}
\newcommand{\catspace}{\mathcal{C}}
\newcommand{\accessible}{\text{Acc}}

\hypersetup{
    colorlinks=true,
    linkcolor=blue,
    citecolor=blue,
    urlcolor=blue
}

\title{\textbf{On the Thermodynamic Consequences of Statistical Dynamics in Non Ideal Gas Molecular Dynamics: Mechanistic Synthesis of Reaction Equilibria Based on Topology Densification }}

\author{Kundai Farai Sachikonye\\
\texttt{kundai.sachikonye@wzw.tum.de}}

\date{\today}

\begin{document}

\maketitle

\begin{abstract}
  We present a categorical framework that resolves Maxwell's Demon paradox through phase-lock network dynamics, eliminating the necessity of information-theoretic arguments while providing mechanistic foundations for entropy increase, irreversibility, and equilibrium. The framework demonstrates that entropy and heat transfer are fundamentally decoupled: heat can flow in either direction during individual molecular collisions, while entropy increases monotonically through categorical completion regardless of energy flow direction.

  Our analysis constructs virtual gas ensembles from hardware oscillator timing measurements, where each measurement $\delta_p = t_{\text{ref}} - t_{\text{local}}$ maps deterministically to $S$-entropy coordinates $(S_k, S_t, S_e)$ in categorical state space. Phase-lock networks formed by Van der Waals forces ($\sim r^{-6}$) and dipole interactions ($\sim r^{-3}$) depend on spatial configuration and electronic structure, not molecular velocity. This establishes that thermodynamic evolution is governed by categorical topology rather than kinetic sorting.

  We prove through eight independent results that Maxwell's Demon is defeated not by measurement costs but by attacking the wrong quantity. First, temporal triviality shows that any configuration the demon creates occurs naturally through thermal fluctuations, rendering intervention redundant. Second, phase-lock temperature independence establishes that network topology is velocity-blind, making kinetic sorting impossible. Third, the retrieval paradox demonstrates that thermal equilibration ($\sim 10^{10}$ collisions/s) randomizes velocities faster than any sorting mechanism. Fourth, phase-lock kinetic independence proves that $\partial G_{\text{PL}}/\partial E_{\text{kin}} = 0$, formally establishing the decoupling between network structure and molecular velocity. Fifth, categorical-physical distance non-equivalence shows that categorical adjacency does not imply spatial proximity. Sixth, temperature emergence reveals that temperature is a statistical observable rather than a sorting criterion accessible to the demon. Seventh, information complementarity establishes that kinetic and categorical information are conjugate observables that cannot be simultaneously specified. Eighth, symmetric entropy increase demonstrates that every door operation increases entropy in both containers through categorical completion.

  The eighth result provides the most direct resolution. Applying categorical analysis to single-molecule transfer reveals that the losing container's remaining molecules must form a new phase-lock network, representing categorical advancement to higher entropy. Simultaneously, the receiving container gains cross-container phase correlations, producing mixing-type densification. Critically, door collision analysis demonstrates that heat can transfer from cold to hot in individual events while entropy still increases through phase-lock correlation formation. This decoupling reveals that the Second Law constrains entropy through categorical completion, not heat through energy accounting. This distinction is obscured by macroscopic formulations but becomes fundamental at the microscopic level where the demon operates.

  The demon measures and manipulates heat flow, which is a statistical emergent property. However, entropy, the actually conserved quantity determined by categorical completion, is immune to the demon's strategy. Maxwell conflated heat and entropy because they are equivalent macroscopically, but at the single-molecule level where the demon operates, heat direction fluctuates while entropy increases monotonically. The demon cannot decrease entropy regardless of sorting strategy because categorical completion is directional, always proceeding forward, while energy transfer is bidirectional and can flow either way.

  The framework establishes five foundational principles. First, entropy counts spatial arrangements rather than velocity distributions. Second, phase-lock networks form through position-dependent interactions, specifically Van der Waals forces ($\sim r^{-6}$) and dipole interactions ($\sim r^{-3}$), not through kinetic energy. Third, categorical completion proceeds through network topology, creating irreversibility without requiring energy gradients. Fourth, temperature emerges as a statistical property of phase-lock clusters rather than as a causal agent. Fifth, heat and entropy are fundamentally decoupled: heat is a statistical emergent property while entropy is a categorical fundamental property.

  This framework extends to chemical equilibrium, where Le Chatelier's principle emerges as the system seeking maximum entropy production rates. Equilibrium is not where reactions stop, but where forward and reverse reactions produce entropy at equal rates. This unifies Maxwell's Demon, Gibbs' paradox, and Le Chatelier's principle within a single categorical framework. The Second Law is fundamentally a statement about categorical topology: isolated systems evolve toward categorical completion through phase-lock network densification, with heat flow and temperature emerging as statistical consequences rather than fundamental constraints.

  The resolution requires no information-theoretic arguments, no quantum considerations, and no appeal to measurement costs. The Second Law is explained purely through the geometry of phase-lock networks and the mathematics of categorical completion, revealing thermodynamic irreversibility as topological navigation through categorical state space.

\textbf{Keywords:} Maxwell's Demon, phase-lock networks, categorical completion, thermodynamic irreversibility, Van der Waals interactions, topological entropy, Gibbs paradox, symmetric entropy increase, Le Chatelier's principle, chemical equilibrium, heat-entropy decoupling, velocity-temperature non-correspondence, Maxwell-Boltzmann overlap, velocity-entropy independence, category error
\end{abstract}

\tableofcontents
\newpage

\section{Introduction}
%==============================================================================

\subsection{The Paradox and Its Broader Implications}

In 1867, James Clerk Maxwell introduced a thought experiment that has challenged thermodynamic foundations for over 150 years \citep{maxwell1871theory}. Consider two chambers A and B containing gas at thermal equilibrium, separated by a partition with a small door controlled by a being whose faculties are so sharpened that he can follow every molecule in its course. This being, the demon, observes molecules approaching the door and selectively opens it to allow fast molecules to pass from A to B and slow molecules from B to A. After sufficient operation, chamber B contains predominantly fast molecules while chamber A contains slow molecules, creating a temperature difference from equilibrium without apparent work expenditure.

The paradox is immediate: the second law of thermodynamics prohibits spontaneous heat flow from cold to hot, yet the demon appears to achieve precisely this through information alone. The total entropy of the system appears to decrease:
\begin{equation}
\Delta S_{\text{total}} = \Delta S_A + \Delta S_B < 0
\label{eq:entropy_decrease}
\end{equation}
contradicting the fundamental requirement $\Delta S \geq 0$ for isolated systems.

However, Maxwell's Demon is not merely a thermodynamic curiosity. It represents a fundamental challenge to our understanding of the relationship between information, entropy, heat, and irreversibility. Resolving the paradox requires confronting deep questions about the nature of thermodynamic quantities: What is entropy counting? What is temperature measuring? How does irreversibility emerge from reversible microscopic dynamics? These questions extend beyond the demon itself to encompass Gibbs' paradox of mixing entropy, Le Chatelier's principle in chemical equilibrium, and the fundamental distinction between heat and entropy at the microscopic level.

\subsection{Standard Resolutions and Their Limitations}

The dominant resolution, developed through contributions by Szilard \citep{szilard1929entropieverminderung}, Brillouin \citep{brillouin1951maxwells}, Landauer \citep{landauer1961irreversibility}, and Bennett \citep{bennett1982thermodynamics}, locates the entropy cost in information processing. The demon must acquire information about molecular velocities, requiring interaction with the molecules that generates entropy. The demon's memory, after accumulating sorting decisions, must eventually be erased. Landauer's principle establishes that erasing one bit of information dissipates at least $kT \ln 2$ of heat, generating entropy $\Delta S \geq k \ln 2$ per bit. Bennett showed that logically irreversible operations, including measurement with finite memory, necessarily produce entropy that offsets any decrease from sorting.

These resolutions, while logically consistent, suffer from a fundamental limitation: they accept the demon's operation as given and locate entropy production in ancillary processes. They answer how sorting avoids violating the second law rather than questioning whether sorting occurs at all. More critically, they obscure the deeper issue that Maxwell's formulation conflates heat and entropy, quantities that are equivalent macroscopically but fundamentally distinct at the microscopic level where the demon operates.

We propose a more radical resolution: there is no demon because there is no sorting by kinetic energy. The apparent sorting is a manifestation of categorical completion through phase-lock network topology, a process requiring no information, no measurement, and no decision-making. This resolution reveals that the Second Law constrains entropy through categorical completion, not heat through energy accounting, and that irreversibility emerges from the directional nature of categorical topology rather than from information erasure costs.

\subsection{The Phase-Lock Network Perspective}

Gas molecules are not independent particles moving through empty space. They exist in networks of phase-locked oscillatory relationships mediated by Van der Waals forces with induced dipole-dipole interactions scaling as $U_{\text{vdW}} \propto r^{-6}$, permanent dipole interactions scaling as $U_{\text{dipole}} \propto r^{-3}$, vibrational coupling through molecular vibrations synchronized via collisions, and rotational coordination through orientational correlations mediated by multipole moments.

Crucially, none of these interactions depend on molecular kinetic energy. Van der Waals forces depend on polarizability and separation. Dipole interactions depend on molecular geometry and orientation. Vibrational coupling depends on normal mode frequencies. A molecule's translational velocity, the quantity the demon supposedly measures, is irrelevant to phase-lock network formation.

This observation inverts the standard picture. Where the traditional view holds that temperature determines molecular speeds, which the demon then measures to perform sorting through information processing, the phase-lock view establishes that phase-lock topology determines categorical structure, which requires no measurement and reveals pre-existing structure through categorical completion rather than creating order through sorting.

The phase-lock perspective resolves not only Maxwell's Demon but also provides a unified framework for understanding Gibbs' paradox, where entropy of mixing emerges from categorical completion rather than from particle distinguishability, and Le Chatelier's principle, where equilibrium represents maximum entropy production rates rather than cessation of reaction. This unification suggests that the Second Law is fundamentally a statement about categorical topology: isolated systems evolve toward categorical completion through phase-lock network densification, with heat flow and temperature emerging as statistical consequences rather than fundamental constraints.

\subsection{The $S$-Entropy Coordinate System}

To formalize phase-lock network dynamics, we introduce the $S$-entropy coordinate system, a three-dimensional categorical state space that maps physical configurations to topological structure. Each molecular configuration is characterized by three coordinates:

The knowledge coordinate $S_k = -\log_2 P_{\text{config}}$ measures information deficit, quantifying how many categorical distinctions remain to be specified to fully determine the system state. The temporal coordinate $S_t = \log_{10}(\tau/\tau_0)$ measures temporal distance from a reference timescale $\tau_0$, capturing the hierarchical structure of dynamical processes from molecular vibrations to macroscopic equilibration. The entropy coordinate $S_e = -\sum_i p_i \log_2 p_i$ measures phase distribution entropy, quantifying the diversity of oscillatory modes in the phase-lock network.

These coordinates are not arbitrary mathematical constructions but emerge from hardware oscillator timing measurements. Each measurement $\delta_p = t_{\text{ref}} - t_{\text{local}}$ maps deterministically to $S$-entropy coordinates $(S_k, S_t, S_e)$, enabling construction of virtual gas ensembles from consumer device oscillations. This mapping establishes a direct connection between abstract categorical structure and measurable physical quantities, providing experimental validation pathways for the theoretical framework.

The $S$-entropy metric satisfies the triangle inequality $d_{\mathcal{S}}(C_i, C_k) \leq d_{\mathcal{S}}(C_i, C_j) + d_{\mathcal{S}}(C_j, C_k)$, establishing a genuine metric space structure. However, categorical distance $d_{\mathcal{S}}$ does not correspond to physical distance: configurations that are spatially proximate may be categorically distant, and vice versa. This inequivalence is central to resolving Maxwell's Demon, as it shows that the demon's spatial manipulation of molecules does not correspond to categorical sorting.

\subsection{Heat-Entropy Decoupling at the Microscopic Level}

A critical insight that emerges from categorical analysis is that heat and entropy, while equivalent in macroscopic thermodynamics, are fundamentally decoupled at the microscopic level. Heat is energy transfer due to temperature difference, a statistical emergent property that can flow in either direction during individual molecular collisions. Entropy is categorical completion through phase-lock network topology, a fundamental property that increases monotonically regardless of energy flow direction.

This decoupling is obscured in macroscopic formulations where $dS = \delta Q/T$ appears to directly link heat and entropy. However, this relation holds only for quasi-static processes involving large ensembles. At the single-molecule level where Maxwell's Demon operates, heat direction fluctuates while entropy increases monotonically. The demon measures and manipulates heat flow, but entropy, the actually conserved quantity determined by categorical completion, is immune to the demon's strategy.

Door collision analysis demonstrates this decoupling explicitly. When a molecule passes through the demon's door, it may transfer energy to or from the door mechanism, creating local heat flow that can be from cold to hot in individual events. Simultaneously, both containers undergo categorical completion: the losing container's remaining molecules must form a new phase-lock network, representing categorical advancement to higher entropy, while the receiving container gains cross-container phase correlations, producing mixing-type densification. The entropy increase is symmetric and inevitable, independent of heat flow direction.

This reveals why information-theoretic resolutions, while correct in their accounting, miss the deeper point: the Second Law constrains entropy through categorical completion, not heat through energy accounting. Maxwell conflated these quantities because they are equivalent macroscopically, but at the microscopic level where the demon operates, they are distinct. The demon cannot decrease entropy regardless of sorting strategy because categorical completion is directional, always proceeding forward, while energy transfer is bidirectional and can flow either way.

\subsection{Central Claims and Paper Structure}

This paper establishes the resolution of Maxwell's Demon through eight independent results, develops the mathematical framework of $S$-entropy categorical navigation, demonstrates heat-entropy decoupling at the microscopic level, and extends the framework to chemical equilibrium and Gibbs' paradox. We prove four foundational theorems:

\begin{theorem}[Phase-Lock Kinetic Independence]
\label{thm:kinetic_independence_intro}
The phase-lock network $\phaselockgraph = (V, E)$ of a gas system satisfies:
\begin{equation}
\frac{\partial \phaselockgraph}{\partial E_{\text{kin}}} = 0
\end{equation}
Network topology is determined by spatial configuration and electronic structure, independent of molecular velocities.
\end{theorem}

\begin{theorem}[Categorical-Physical Distance Inequivalence]
\label{thm:distance_inequivalence_intro}
For categorical distance $d_{\catspace}$ and physical distance $d_{\text{phys}}$:
\begin{equation}
d_{\catspace}(C_i, C_j) \neq f(d_{\text{phys}}(\mathbf{r}_i, \mathbf{r}_j))
\end{equation}
for any function $f$. Categorical adjacency does not correspond to spatial proximity.
\end{theorem}

\begin{theorem}[Temperature Emergence]
\label{thm:temperature_emergence_intro}
Temperature $T$ emerges as a statistical property of the phase-lock cluster structure:
\begin{equation}
T = \mathcal{F}[\{\phaselockgraph_\alpha\}]
\end{equation}
where $\{\phaselockgraph_\alpha\}$ denotes the ensemble of phase-lock clusters. Temperature does not determine network structure; network structure determines apparent temperature.
\end{theorem}

\begin{theorem}[Symmetric Entropy Increase]
\label{thm:symmetric_entropy_intro}
Every door operation by the demon increases entropy in both containers:
\begin{equation}
\Delta S_A > 0 \quad \text{and} \quad \Delta S_B > 0
\end{equation}
regardless of which molecule transfers or its velocity. This follows from applying the categorical resolution of Gibbs' paradox to single-molecule transfer.
\end{theorem}

From these results, the conclusion is as follows: Maxwell's Demon dissolves because the sorting it supposedly performs is categorical completion through phase-lock topology. Molecules following phase-lock adjacency relations appear sorted by temperature because phase-lock clusters correlate with, but are not caused by, kinetic properties. Most directly, every door operation increases entropy in both containers through categorical completion, making a decrease in entropy impossible regardless of the demon's strategy.

The paper proceeds as follows. Section~\ref{sec:phase_lock} develops the theory of phase-lock networks and establishes their independence from kinetic energy. Section~\ref{sec:eight_results} proves the eight independent results that dissolve Maxwell's Demon. Section~\ref{sec:sentropy} introduces the $S$-entropy coordinate system and categorical navigation. Section~\ref{sec:heat_entropy} demonstrates heat-entropy decoupling through door collision analysis. Section~\ref{sec:extensions} extends the framework to chemical equilibrium and Gibbs' paradox. Section~\ref{sec:experimental} proposes experimental validation pathways. Section~\ref{sec:conclusion} discusses implications for thermodynamic foundations.

The resolution requires no information-theoretic arguments, no quantum considerations, and no appeal to measurement costs. The Second Law is explained purely through the geometry of phase-lock networks and the mathematics of categorical completion, revealing thermodynamic irreversibility as topological navigation through categorical state space.
%==============================================================================

\subsection{The Paradox}

In 1867, James Clerk Maxwell introduced a thought experiment that has challenged thermodynamic foundations for over 150 years \citep{maxwell1871theory}. Consider two chambers A and B containing gas at thermal equilibrium, separated by a partition with a small door controlled by ``a being whose faculties are so sharpened that he can follow every molecule in its course.'' This being---the demon---observes molecules approaching the door and selectively opens it to allow fast molecules to pass from A to B and slow molecules from B to A. After sufficient operation, chamber B contains predominantly fast (hot) molecules while chamber A contains slow (cold) molecules, creating a temperature difference from equilibrium without apparent work expenditure.

The paradox is immediate: the second law of thermodynamics prohibits spontaneous heat flow from cold to hot, yet the demon appears to achieve precisely this through information alone. The total entropy of the system appears to decrease:
\begin{equation}
\Delta S_{\text{total}} = \Delta S_A + \Delta S_B < 0
\label{eq:entropy_decrease}
\end{equation}
contradicting the fundamental requirement $\Delta S \geq 0$ for isolated systems.

\subsection{Standard Resolutions and Their Limitations}

The dominant resolution, developed through contributions by Szilard \citep{szilard1929entropieverminderung}, Brillouin \citep{brillouin1951maxwells}, Landauer \citep{landauer1961irreversibility}, and Bennett \citep{bennett1982thermodynamics}, locates the entropy cost in information processing:

\begin{enumerate}
    \item \textbf{Measurement cost}: The demon must acquire information about molecular velocities, requiring interaction with the molecules that generates entropy.

    \item \textbf{Memory erasure}: The demon's memory, after accumulating sorting decisions, must eventually be erased. Landauer's principle establishes that erasing one bit of information dissipates at least $kT \ln 2$ of heat, generating entropy $\Delta S \geq k \ln 2$ per bit.

    \item \textbf{Computational irreversibility}: Bennett showed that logically irreversible operations (including measurement with finite memory) necessarily produce entropy, offsetting any decrease from sorting.
\end{enumerate}

These resolutions, while logically consistent, suffer from a fundamental limitation: they accept the demon's operation as given and locate entropy production in ancillary processes. They answer ``how does sorting avoid violating the second law?'' rather than questioning whether sorting occurs at all.

We propose a more radical resolution: \textit{there is no demon because there is no sorting by kinetic energy}. The apparent sorting is a manifestation of categorical completion through phase-lock network topology---a process requiring no information, no measurement, and no decision-making.

\subsection{The Phase-Lock Network Perspective}

Gas molecules are not independent particles moving through empty space. They exist in networks of phase-locked oscillatory relationships mediated by:
\begin{itemize}
    \item Van der Waals forces: induced dipole-dipole interactions scaling as $U_{vdW} \propto r^{-6}$
    \item Permanent dipole interactions: scaling as $U_{\text{dipole}} \propto r^{-3}$
    \item Vibrational coupling: molecular vibrations synchronised through collisions
    \item Rotational coordination: orientational correlations through multipole moments
\end{itemize}

Crucially, \textit{none of these interactions depend on molecular kinetic energy}. Van der Waals forces depend on polarisability and separation; dipole interactions depend on molecular geometry and orientation; vibrational coupling depends on normal mode frequencies. A molecule's translational velocity---the quantity the demon supposedly measures---is irrelevant to phase-lock network formation.

This observation inverts the standard picture:
\begin{center}
\begin{tabular}{l|l}
\textbf{Standard View} & \textbf{Phase-Lock View} \\
\hline
Temperature $\to$ molecular speeds & Phase-lock topology $\to$ categorical structure \\
Demon measures velocity & No measurement needed \\
Sorting creates order & Topology reveals pre-existing structure \\
Information processing required & Categorical completion sufficient
\end{tabular}
\end{center}

\subsection{Central Claims}

This paper establishes four central results:

\begin{theorem}[Phase-Lock Kinetic Independence]
\label{thm:kinetic_independence_intro}
The phase-lock network $\phaselockgraph = (V, E)$ of a gas system satisfies:
\begin{equation}
\frac{\partial \phaselockgraph}{\partial E_{\text{kin}}} = 0
\end{equation}
Network topology is determined by spatial configuration and electronic structure, independent of molecular velocities.
\end{theorem}

\begin{theorem}[Categorical-Physical Distance Inequivalence]
\label{thm:distance_inequivalence_intro}
For categorical distance $d_{\catspace}$ and physical distance $d_{\text{phys}}$:
\begin{equation}
d_{\catspace}(C_i, C_j) \neq f(d_{\text{phys}}(\mathbf{r}_i, \mathbf{r}_j))
\end{equation}
for any function $f$. Categorical adjacency does not correspond to spatial proximity.
\end{theorem}

\begin{theorem}[Temperature Emergence]
\label{thm:temperature_emergence_intro}
Temperature $T$ emerges as a statistical property of phase-lock cluster structure:
\begin{equation}
T = \mathcal{F}[\{\phaselockgraph_\alpha\}]
\end{equation}
where $\{\phaselockgraph_\alpha\}$ denotes the ensemble of phase-lock clusters. Temperature does not determine network structure; network structure determines apparent temperature.
\end{theorem}

\begin{theorem}[Symmetric Entropy Increase]
\label{thm:symmetric_entropy_intro}
Every door operation by the demon increases entropy in both containers:
\begin{equation}
\Delta S_A > 0 \quad \text{and} \quad \Delta S_B > 0
\end{equation}
regardless of which molecule transfers or its velocity. This follows from applying the categorical resolution of Gibbs' paradox to single-molecule transfer.
\end{theorem}

From these results, the resolution follows: Maxwell's Demon dissolves because the ``sorting'' it supposedly performs is categorical completion through phase-lock topology. Molecules following phase-lock adjacency relations appear sorted by temperature because phase-lock clusters correlate with---but are not caused by---kinetic properties. Most directly, every door operation increases entropy in both containers through categorical completion (Theorem~\ref{thm:symmetric_entropy_intro}), making entropy decrease impossible regardless of the demon's strategy.

%==============================================================================
% Section imports
%==============================================================================

%==============================================================================
\section{Phase-Lock Networks and Kinetic Independence}
\label{sec:phase_lock}
%==============================================================================

\subsection{Intermolecular Interactions in Gas Systems}

We begin by establishing the physical basis for phase-lock networks in gas systems. Gas molecules interact through several mechanisms, each with characteristic distance dependence and each fundamentally independent of molecular translational velocity.

\begin{definition}[Van der Waals Interaction]
\label{def:vdw}
The Van der Waals interaction between two molecules $i$ and $j$ separated by a distance $r_{ij}$ is:
\begin{equation}
U_{\text{vdW}}(r_{ij}) = -\frac{C_6^{(ij)}}{r_{ij}^6}
\label{eq:vdw_potential}
\end{equation}
where $C_6^{(ij)}$ is the dispersion coefficient determined by molecular polarizabilities:
\begin{equation}
C_6^{(ij)} = \frac{3}{2} \frac{\alpha_i \alpha_j}{(4\pi\varepsilon_0)^2} \frac{I_i I_j}{I_i + I_j}
\label{eq:c6_coefficient}
\end{equation}
with $\alpha_i$ and $\alpha_j$ being the static polarizabilities, and $I_i$ and $I_j$ the ionisation energies.
\end{definition}

\begin{definition}[Dipole-Dipole Interaction]
\label{def:dipole}
For molecules with permanent dipole moments $\boldsymbol{\mu}_i$ and $\boldsymbol{\mu}_j$, the interaction energy is:
\begin{equation}
U_{\text{dipole}}(r_{ij}, \theta_i, \theta_j, \phi) = -\frac{\mu_i \mu_j}{4\pi\varepsilon_0 r_{ij}^3} \left( 2\cos\theta_i \cos\theta_j - \sin\theta_i \sin\theta_j \cos\phi \right)
\label{eq:dipole_potential}
\end{equation}
where $\theta_i$ and $\theta_j$ are angles between dipoles and the intermolecular axis, and $\phi$ is the dihedral angle.
\end{definition}

\begin{proposition}[Kinetic Energy Independence of Interactions]
\label{prop:kinetic_independence_interactions}
The interaction potentials $U_{\text{vdW}}$ and $U_{\text{dipole}}$ are independent of molecular translational kinetic energy:
\begin{equation}
\frac{\partial U_{\text{vdW}}}{\partial E_{\text{kin}}} = 0, \quad \frac{\partial U_{\text{dipole}}}{\partial E_{\text{kin}}} = 0
\end{equation}
where $E_{\text{kin}} = \frac{1}{2}m|\mathbf{v}|^2$ is molecular translational kinetic energy.
\end{proposition}

\begin{proof}
From Equation~\eqref{eq:vdw_potential}, $U_{\text{vdW}}$ depends only on the separation $r_{ij}$ and the molecular properties $\alpha_i$, $\alpha_j$, $I_i$, and $I_j$. None of these quantities involve molecular velocity $\mathbf{v}$.

The polarizability $\alpha$ is an electronic property determined by the sum over electronic transitions:
\begin{equation}
\alpha = \sum_n \frac{2|\langle 0 | \hat{\mathbf{d}} | n \rangle|^2}{E_n - E_0}
\end{equation}
where $|n\rangle$ are electronic states and $\hat{\mathbf{d}}$ is the dipole operator. This sum is independent of nuclear translational motion because electronic wavefunctions are computed in the molecular frame under the Born-Oppenheimer approximation.

Similarly, Equation~\eqref{eq:dipole_potential} depends on the separation $r_{ij}$, the dipole moments $\mu_i$ and $\mu_j$, and orientational angles $\theta_i$, $\theta_j$, and $\phi$. The dipole moment is a ground-state electronic property, and the orientational angles describe spatial configuration, not translational velocity.

Therefore $\partial U / \partial E_{\text{kin}} = 0$ for both interaction types. \qed
\end{proof}

\begin{figure}[htbp]
\centering
\includegraphics[width=\textwidth]{figures/phase_lock_mechanism_panel.png}
\caption{Phase-lock mechanism in molecular systems. (A) Independent oscillators with uncorrelated phases. (B) Intermolecular coupling enables phase information exchange. (C) Coupling drives phase synchronization with bounded phase difference. (D) Phase-locked state represents categorical completion. (E) Autocatalytic cascade: existing locks enable new locks through network topology. (F) Categorical entropy increases monotonically with phase-lock network density.}
\label{fig:phase_lock_mechanism}
\end{figure}

\subsection{Phase-Lock Network Construction}

\begin{definition}[Molecular Phase]
\label{def:molecular_phase}
The instantaneous phase of molecule $i$ is a composite quantity capturing vibrational, rotational, and electronic oscillations:
\begin{equation}
\Phi_i(t) = \omega_{\text{vib},i} t + \phi_{\text{vib},i} + \omega_{\text{rot},i} t + \phi_{\text{rot},i} + \Phi_{\text{elec},i}(t)
\label{eq:molecular_phase}
\end{equation}
where $\omega_{\text{vib},i}$ and $\phi_{\text{vib},i}$ are the vibrational frequency and initial phase, $\omega_{\text{rot},i}$ and $\phi_{\text{rot},i}$ are the rotational frequency and initial phase, and $\Phi_{\text{elec},i}(t)$ is the electronic oscillation phase.
\end{definition}

\begin{definition}[Phase-Lock Condition]
\label{def:phase_lock}
Molecules $i$ and $j$ are phase-locked if their phase difference remains bounded over a coherence time:
\begin{equation}
|\Phi_i(t) - \Phi_j(t) - \Delta\phi_{ij}| < \varepsilon \quad \forall t \in [t_0, t_0 + \tau]
\label{eq:phase_lock_condition}
\end{equation}
for some constant offset $\Delta\phi_{ij}$, threshold $\varepsilon < \pi/4$, and coherence time $\tau > \tau_{\min}$.
\end{definition}

\begin{definition}[Phase-Lock Network]
\label{def:phase_lock_network}
The phase-lock network of a gas system is the graph $\phaselockgraph = (V, E)$, where $V = \{m_1, m_2, \ldots, m_N\}$ is the set of molecules, and an edge $(m_i, m_j) \in E$ exists if and only if molecules $i$ and $j$ satisfy the phase-lock condition~\eqref{eq:phase_lock_condition}.
\end{definition}

\begin{proposition}[Phase-Lock Formation Mechanism]
\label{prop:phase_lock_formation}
Phase-locking between molecules $i$ and $j$ occurs when the interaction energy exceeds thermal energy by a threshold factor:
\begin{equation}
|U_{\text{int}}(r_{ij})| > k_B T \cdot \eta_{\text{threshold}}
\label{eq:phase_lock_threshold}
\end{equation}
where $U_{\text{int}} = U_{\text{vdW}} + U_{\text{dipole}} + \ldots$ is the total interaction potential and $\eta_{\text{threshold}} \approx 0.1$ is a dimensionless coupling threshold.
\end{proposition}

\begin{proof}
Phase synchronisation requires coupling strength exceeding thermal fluctuations. The coupling strength scales with interaction energy $|U_{\text{int}}|$, while thermal disruption scales with $k_B T$. Standard synchronisation theory \citep{pikovsky2001synchronization, kuramoto1975self} establishes that phase-locking occurs when the coupling strength $K_{ij} \propto |U_{\text{int}}(r_{ij})|$ exceeds the critical coupling $K_c \propto k_B T$. This yields condition~\eqref{eq:phase_lock_threshold}. \qed
\end{proof}

\subsection{The Kinetic Independence Theorem}

We now prove the central result establishing that phase-lock networks are determined by spatial configuration and molecular properties, not by translational velocities.

\begin{theorem}[Phase-Lock Kinetic Independence]
\label{thm:kinetic_independence}
The phase-lock network $\phaselockgraph = (V, E)$ is independent of molecular kinetic energies:
\begin{equation}
\frac{\partial \phaselockgraph}{\partial E_{\text{kin},i}} = 0 \quad \forall i \in V
\label{eq:network_kinetic_independence}
\end{equation}
Specifically, the edge set $E$ is determined by spatial configuration $\{\mathbf{r}_i\}$ and molecular properties $\{\alpha_i, \mu_i, \omega_{\text{vib},i}, \ldots\}$, but not by velocities $\{\mathbf{v}_i\}$.
\end{theorem}

\begin{proof}
We establish kinetic independence by demonstrating that each factor determining edge existence is velocity-independent.

\textbf{Step 1: Interaction potential independence.}
From Proposition~\ref{prop:kinetic_independence_interactions}, the interaction potential $U_{\text{int}}(r_{ij})$ does not depend on molecular velocities. Van der Waals forces depend on polarizabilities and separation, both of which are velocity-independent. Dipole interactions depend on dipole moments and orientational angles, neither of which involve translational velocity.

\textbf{Step 2: Phase-lock threshold independence.}
The threshold condition~\eqref{eq:phase_lock_threshold} involves $U_{\text{int}}$ and $T$. While temperature $T$ relates to average kinetic energy through the equipartition theorem:
\begin{equation}
\langle E_{\text{kin}} \rangle = \frac{3}{2} k_B T
\end{equation}
this is a statistical relationship over ensembles. For a given instantaneous configuration, the phase-lock condition depends on separation $r_{ij}$, polarizabilities $\alpha_i$ and $\alpha_j$, dipole moments $\mu_i$ and $\mu_j$, and orientational angles $\theta_i$, $\theta_j$, and $\phi$. None of these quantities depend on translational velocity $\mathbf{v}$.

\textbf{Step 3: Phase dynamics independence.}
From Definition~\ref{def:molecular_phase}, the molecular phase $\Phi_i(t)$ comprises vibrational modes determined by molecular structure, rotational modes determined by angular momentum and moment of inertia, and electronic oscillations determined by electronic structure. Translational kinetic energy $E_{\text{kin}} = \frac{1}{2}m|\mathbf{v}|^2$ does not appear in the phase equation~\eqref{eq:molecular_phase}. While molecular collisions can affect rotational states, the rotational angular momentum is independent of the direction and magnitude of translational velocity.

\textbf{Step 4: Edge set determination.}
An edge $(m_i, m_j) \in E$ exists if and only if the coupling exceeds the threshold, satisfying $|U_{\text{int}}(r_{ij})| > k_B T \cdot \eta_{\text{threshold}}$, and phase coherence is maintained according to condition~\eqref{eq:phase_lock_condition}. Both conditions are determined by spatial configuration and molecular properties, not translational velocities.

Therefore, the edge set satisfies $E = E(\{\mathbf{r}_i\}, \{\alpha_i, \mu_i, \ldots\})$ with no dependence on $\{\mathbf{v}_i\}$, establishing~\eqref{eq:network_kinetic_independence}. \qed
\end{proof}

\begin{corollary}[Velocity-Invariant Network Topology]
\label{cor:velocity_invariant}
Two gas configurations with identical spatial arrangements $\{\mathbf{r}_i\}$ but different velocity distributions $\{\mathbf{v}_i\}$ and $\{\mathbf{v}'_i\}$ have identical phase-lock networks:
\begin{equation}
\phaselockgraph(\{\mathbf{r}_i\}, \{\mathbf{v}_i\}) = \phaselockgraph(\{\mathbf{r}_i\}, \{\mathbf{v}'_i\})
\end{equation}
\end{corollary}

\begin{proof}
Immediate from Theorem~\ref{thm:kinetic_independence}. Since $\phaselockgraph$ does not depend on velocities, changing velocities while preserving positions leaves the network unchanged. \qed
\end{proof}

\begin{figure*}[htbp]
\centering
\includegraphics[width=0.95\textwidth]{figures/arg1_temporal_triviality.png}
\caption{\textbf{Temporal Triviality—Any Configuration Occurs Naturally Through Thermal Fluctuations.}
\textbf{(A)} Boltzmann probability landscape showing all configurations are thermally accessible. The probability distribution $P(\text{config}) = \exp(-E/k_BT)/Z$ ensures every spatial arrangement, including ``sorted'' states, occurs naturally through fluctuations.
\textbf{(B)} Poincaré recurrence times as a function of sorting degree. Higher sorting corresponds to exponentially longer recurrence times $\tau_{\text{rec}} \sim \exp(N\Delta S)$, but all states eventually recur. The horizontal dashed line indicates laboratory timescales; even highly sorted states recur within observable time for small systems.
\textbf{(C)} Configuration space flow field showing all trajectories converge to equilibrium. The flow follows $\dot{\mathbf{q}} = -\nabla_{\mathbf{q}} F(\mathbf{q})$ where $F$ is the free energy. Red squares mark ``sorted'' configurations; yellow circles mark equilibrium. All paths lead to the central attractor, demonstrating that sorted states are unstable fixed points.
\textbf{(D)} Entropy evolution over time showing fluctuations enable access to all states. The solid black line shows total entropy $S(t) = -k_B \sum_i p_i \ln p_i$ increasing monotonically toward equilibrium (horizontal dashed line). The dotted red line marks the entropy of the ``sorted'' state. Yellow triangles indicate moments when the system spontaneously visits sorted configurations through thermal fluctuations, demonstrating temporal triviality: the demon's purported action is redundant.}
\label{fig:temporal_triviality}
\end{figure*}

\subsection{Network Properties and Characteristic Scales}

\begin{definition}[Phase-Lock Degree]
\label{def:phase_lock_degree}
The phase-lock degree of molecule $i$ is:
\begin{equation}
k_i = |\{j : (m_i, m_j) \in E\}|
\end{equation}
the number of molecules phase-locked to $i$.
\end{definition}

\begin{proposition}[Degree Distribution]
\label{prop:degree_distribution}
For a gas at uniform density $n = N/V$, the expected phase-lock degree scales as:
\begin{equation}
\langle k \rangle \sim n \cdot \frac{4\pi}{3} r_{\text{lock}}^3
\label{eq:expected_degree}
\end{equation}
where $r_{\text{lock}}$ is the characteristic distance at which phase-locking occurs, determined by:
\begin{equation}
|U_{\text{int}}(r_{\text{lock}})| = k_B T \cdot \eta_{\text{threshold}}
\end{equation}
\end{proposition}

\begin{proof}
Molecules within a distance $r_{\text{lock}}$ satisfy the phase-lock condition with high probability. The expected number of neighbours within this distance is:
\begin{equation}
\langle k \rangle = n \cdot V_{\text{sphere}}(r_{\text{lock}}) = n \cdot \frac{4\pi}{3} r_{\text{lock}}^3
\end{equation}
For Van der Waals interactions with $U_{\text{vdW}} \propto r^{-6}$, the phase-lock distance satisfies:
\begin{equation}
r_{\text{lock}} \sim \left(\frac{C_6}{k_B T \eta_{\text{threshold}}}\right)^{1/6}
\end{equation}
For typical gases at room temperature with $C_6 \sim 10^{-77}$ J·m$^6$ and $k_B T \sim 4 \times 10^{-21}$ J, this yields $r_{\text{lock}} \sim 0.3$--$0.5$ nm. \qed
\end{proof}

\begin{definition}[Phase-Lock Cluster]
\label{def:phase_lock_cluster}
A phase-lock cluster is a connected component of $\phaselockgraph$, defined as a maximal subset $S \subseteq V$ such that for any $i, j \in S$, there exists a path in $\phaselockgraph$ connecting $m_i$ and $m_j$.
\end{definition}

\begin{remark}[Zero Temperature Persistence]
At absolute zero temperature, where $T \to 0$, molecular translational motion ceases and $\langle E_{\text{kin}} \rangle \to 0$. However, phase-lock networks persist. Electronic orbitals continue oscillating at characteristic frequencies $\sim 10^{15}$ Hz, vibrational zero-point motion persists with amplitude $\sim (h/m\omega)^{1/2}$, and intermolecular forces remain active. The phase-lock network $\phaselockgraph(T=0)$ is well-defined and nontrivial, containing edges determined by spatial configuration and electronic structure. This underscores the fundamental kinetic independence: the network exists independently of thermal motion and translational kinetic energy.
\end{remark}

%==============================================================================
\section{Categorical Completion in Gas Dynamics}
\label{sec:categorical}
%==============================================================================

\subsection{Categorical State Space}

We now develop the categorical framework for describing gas configurations. This framework extends beyond classical phase space by incorporating phase-lock network topology and oscillatory relationships that are invisible to spatial measurements but fundamental to thermodynamic evolution.

\begin{definition}[Categorical State]
\label{def:categorical_state}
A categorical state $C \in \catspace$ specifies the complete oscillatory and topological structure of a molecular configuration, comprising the phase-lock network topology $\phaselockgraph$, the phase relationships $\{\Delta\phi_{ij}\}$ for all locked pairs, the vibrational mode occupations $\{n_{\text{vib},i}\}$, the rotational state quantum numbers $\{J_i, M_i\}$, and the electronic configuration descriptors.
\end{definition}

\begin{remark}
A categorical state contains strictly more information than a classical phase space point $(\mathbf{q}, \mathbf{p})$. Two configurations with identical positions and momenta can occupy different categorical states if their phase relationships or network topologies differ. This additional structure is not merely mathematical bookkeeping but reflects physical reality: molecules with identical spatial configurations but different phase relationships have different interaction energies, different spectroscopic signatures, and different dynamical evolution. The categorical state captures the full oscillatory context that determines thermodynamic behavior.
\end{remark}

\begin{definition}[Categorical State Space]
\label{def:categorical_state_space}
The categorical state space $\catspace$ is the set of all categorical states equipped with a partial order $\prec$ called the completion order, a completion operator $\mu: \catspace \times \mathbb{R}_{\geq 0} \to \{0, 1\}$ indicating whether state $C$ has been occupied by time $t$, and a topology $\tau$ induced by $\prec$ that makes categorical adjacency continuous.
\end{definition}

The completion operator $\mu(C, t) = 1$ indicates that categorical state $C$ has been occupied (completed) by time $t$, while $\mu(C, t) = 0$ indicates it remains unoccupied. This binary distinction captures the fundamental irreversibility of categorical completion: once a state is occupied, it remains in the system's history.

\begin{axiom}[Categorical Irreversibility]
\label{axiom:categorical_irreversibility}
Once a categorical state $C_i$ is occupied, it cannot be re-occupied. For all $C_i \in \catspace$ and times $t_1 \leq t_2$:
\begin{equation}
\mu(C_i, t_1) = 1 \implies \mu(C_i, t_2) = 1
\label{eq:irreversibility}
\end{equation}
Any process returning to a spatially identical configuration must occupy a new categorical state $C_j$ with $C_i \prec C_j$.
\end{axiom}

This axiom formalises the intuition that time's arrow is encoded in categorical structure. Even when spatial configurations repeat, the phase relationships and network topology have evolved, placing the system in a new categorical state. This is the microscopic origin of thermodynamic irreversibility: not spatial evolution, but categorical completion.

\begin{proposition}[Monotonic Completion]
\label{prop:monotonic_completion}
Let $\gamma(t) = \{C \in \catspace : \mu(C, t) = 1\}$ be the set of completed states at time $t$. Then:
\begin{equation}
t_1 \leq t_2 \implies \gamma(t_1) \subseteq \gamma(t_2)
\end{equation}
The completed set grows monotonically, never contracting.
\end{proposition}

\begin{proof}
Immediate from Axiom~\ref{axiom:categorical_irreversibility}. If $C \in \gamma(t_1)$, then $\mu(C, t_1) = 1$, which implies $\mu(C, t_2) = 1$ for all $t_2 \geq t_1$, hence $C \in \gamma(t_2)$. Therefore $\gamma(t_1) \subseteq \gamma(t_2)$. \qed
\end{proof}

\subsection{Phase-Lock Degeneracy and Categorical Richness}

A crucial insight emerges from the relationship between spatial and categorical descriptions: a single spatial configuration corresponds to many categorical states. This degeneracy is not a deficiency of the categorical framework but rather reveals hidden structure invisible to spatial measurements.

\begin{theorem}[Phase-Lock Degeneracy]
\label{thm:phase_lock_degeneracy}
For a spatial configuration $\mathbf{q} = \{\mathbf{r}_1, \ldots, \mathbf{r}_N\}$, there exist multiple categorical states producing identical spatial observables. The phase-lock degeneracy is:
\begin{equation}
\Omega_{\text{PL}}(\mathbf{q}) = |\{C \in \catspace : \pi_{\text{spatial}}(C) = \mathbf{q}\}|
\label{eq:phase_lock_degeneracy}
\end{equation}
where $\pi_{\text{spatial}}: \catspace \to \mathbb{R}^{3N}$ is the spatial projection.
\end{theorem}

\begin{proof}
Consider two molecules at fixed positions $\mathbf{r}_1$ and $\mathbf{r}_2$ separated by distance $r_{12}$. The same spatial configuration can be achieved through different combinations of oscillatory phases. Van der Waals interactions have no preferred angular orientation, contributing a continuous degeneracy. For molecules with permanent dipoles, the orientations $(\theta_1, \phi_1)$ and $(\theta_2, \phi_2)$ can vary while maintaining the same spatial positions, provided the molecular centers remain fixed. Vibrational phases $\phi_{\text{vib},i}$ are independent degrees of freedom not specified by spatial position. Rotational phases $\phi_{\text{rot},i}$ similarly vary independently of molecular center-of-mass location.

These constitute distinct categorical states with different phase relationships $\{\Delta\phi_{ij}\}$ but identical spatial projection $\pi_{\text{spatial}}(C) = \mathbf{q}$.

For $N$ molecules with $\binom{N}{2}$ pairwise interactions, each having $k$ independent phase degrees of freedom per pair, the degeneracy scales as:
\begin{equation}
\Omega_{\text{PL}}(\mathbf{q}) \sim (2\pi)^{k \cdot \binom{N}{2}}
\end{equation}

For typical molecular gases where $k \approx 3$ to $5$ (vibrational, rotational, and orientational phases), and $N \sim 10^{23}$ molecules in a macroscopic sample, the degeneracy is astronomical. Even for small systems with $N = 100$ molecules, $\Omega_{\text{PL}} \sim 10^6$ to $10^{12}$ per spatial configuration. \qed
\end{proof}

\begin{figure*}[htbp]
\centering
\includegraphics[width=0.95\textwidth]{figures/panel_s_space.png}
\caption{\textbf{S-Entropy Space Visualization: Molecular Distribution in Configurational Entropy Coordinates.}
\textbf{(A)} Molecular distribution in S-space. Three-dimensional visualization of molecular states projected onto entropy coordinates $(S_{\kappa}, S_{\epsilon}, S_t)$, where $S_{\kappa}$ represents kinetic entropy, $S_{\epsilon}$ represents categorical/configurational entropy, and $S_t$ represents topological entropy. Color gradient (purple to yellow) indicates progression through state space. The distribution shows clear clustering structure, with molecules occupying a bounded region of entropy space. The axes span $S_{\kappa} \in [0.96, 1.04]$, $S_{\epsilon} \in [7.5, 10.0]$, and $S_t \in [0.0, 1.0]$, revealing the natural scale separation between different entropy components.
\textbf{(B)} Polar phase diagram: $S_t \to 0$, $S_{\epsilon} \to r$. Radial coordinate represents configurational entropy $S_{\epsilon}$ (ranging from 0 at center to 1.00 at outer edge), while angular coordinate represents phase angle. The distribution (blue line at 0°) shows strong directional preference, indicating phase-locked structure. Concentric circles mark entropy magnitude levels (0.25, 0.50, 0.75, 1.00). The highly anisotropic distribution demonstrates that molecules occupy specific phase relationships, not random orientations, confirming the phase-lock network structure.
\textbf{(C)} Ternary composition diagram. Triangle vertices represent pure states: $S_{\kappa}$ (kinetic, top), $S_{\epsilon}$ (configurational, bottom-left), and $S_t$ (topological, bottom-right). The color bar (blue to red gradient) shows the distribution of molecular states across the three entropy components. Most molecules cluster near the $S_{\epsilon}$ vertex, indicating dominance of configurational entropy. The narrow distribution shows that entropy composition is tightly constrained, not uniformly distributed across the ternary space.
\textbf{(D)} Density contour in $(S_{\kappa}, S_{\epsilon})$ plane. Heat map shows probability density with color scale from 0 (pale yellow) to 35 (dark red). Sharp vertical band at $S_{\kappa} \approx 1.0$ indicates that kinetic entropy is nearly constant across the distribution, while $S_{\epsilon}$ varies from 0.2 to 1.0. The intense red stripe demonstrates that molecules occupy a one-dimensional manifold in the two-dimensional entropy space, revealing strong constraint on accessible states.
\textbf{(E)} Radial distribution function $g(r)$ versus distance from center in entropy space. Peak at $r \approx 0.72$ (height $\approx 35$) indicates strong clustering at specific entropy magnitude. Secondary peaks at $r \approx 0.76$ and $r \approx 0.84$ suggest shell structure in entropy space. The oscillatory pattern demonstrates that molecules do not uniformly fill entropy space but organize into discrete shells, analogous to electron shells in atoms.
\textbf{(F)} Phase trajectories in $(S_{\kappa}, S_t)$ plane. Colored points (ranging from red/orange at bottom to purple at top) trace molecular evolution through entropy space. Trajectories cluster into discrete bands at $S_t \approx 0.0$, $0.2$, $0.4$, $0.6$, $0.8$, and $1.0$, showing quantized topological entropy levels. The vertical alignment indicates that $S_{\kappa}$ remains nearly constant during evolution, while $S_t$ transitions between discrete levels. This reveals that entropy dynamics are dominated by topological transitions, not kinetic changes, supporting the categorical face dominance in Maxwell's demon resolution.}
\label{fig:s_space}
\end{figure*}


\begin{definition}[Categorical Equivalence Class]
\label{def:categorical_equivalence_class}
The categorical equivalence class of state $C$ under spatial observation is:
\begin{equation}
[C]_{\text{spatial}} = \{C' \in \catspace : \pi_{\text{spatial}}(C') = \pi_{\text{spatial}}(C)\}
\end{equation}
States in the same equivalence class are spatially indistinguishable but categorically distinct.
\end{definition}

\begin{corollary}[Categorical Richness]
\label{cor:categorical_richness}
The categorical richness of a spatial configuration is:
\begin{equation}
R(\mathbf{q}) = \log \Omega_{\text{PL}}(\mathbf{q}) = \log |[C]_{\text{spatial}}|
\end{equation}
This quantifies the information content of categorical specification beyond spatial description.
\end{corollary}

The categorical richness $R(\mathbf{q})$ has profound implications for entropy. Traditional statistical mechanics computes entropy by counting spatial configurations accessible at given energy. The categorical framework reveals that each spatial configuration harbors vast additional structure through phase-lock degeneracy. This hidden structure is precisely what categorical completion navigates, and what the Second Law constrains.

\subsection{Categorical Completion Dynamics}

\begin{definition}[Completion Rate]
\label{def:completion_rate}
The categorical completion rate is:
\begin{equation}
\dot{C}(t) = \frac{d|\gamma(t)|}{dt}
\label{eq:completion_rate}
\end{equation}
measuring the rate at which new categorical states are completed.
\end{definition}

\begin{proposition}[Non-Negative Completion Rate]
\label{prop:nonnegative_completion}
For all times $t$:
\begin{equation}
\dot{C}(t) \geq 0
\end{equation}
with equality only when no physical processes occur.
\end{proposition}

\begin{proof}
From Proposition~\ref{prop:monotonic_completion}, $|\gamma(t)|$ is monotonically non-decreasing, hence $\dot{C}(t) = d|\gamma(t)|/dt \geq 0$. Equality $\dot{C}(t) = 0$ requires $|\gamma(t)|$ constant, meaning no new categorical states are completed. This occurs only when the system is frozen in a single categorical state with no dynamical evolution. \qed
\end{proof}

\begin{theorem}[Categorical Completion as Physical Process]
\label{thm:completion_physical}
Every physical process in a gas system corresponds to categorical completion:
\begin{equation}
\text{Process: } \mathbf{q}(t_1) \to \mathbf{q}(t_2) \quad \Longleftrightarrow \quad \text{Completion: } C(t_1) \prec C(t_2)
\end{equation}
The categorical state advances along the completion order, never retreating.
\end{theorem}

\begin{proof}
Consider a gas evolving from configuration $\mathbf{q}(t_1)$ to $\mathbf{q}(t_2)$ over time interval $\Delta t = t_2 - t_1$.

\textbf{Case 1: Distinct spatial configurations where $\mathbf{q}(t_2) \neq \mathbf{q}(t_1)$.}
The new configuration occupies categorical states not accessible from $\mathbf{q}(t_1)$ because the spatial projection differs. By Axiom~\ref{axiom:categorical_irreversibility}, these must be new completions: $C(t_2) \in \gamma(t_2) \setminus \gamma(t_1)$. The categorical state has advanced.

\textbf{Case 2: Identical spatial configurations where $\mathbf{q}(t_2) = \mathbf{q}(t_1)$.}
Even with identical spatial positions, the phase relationships have evolved. During time interval $\Delta t$, molecules undergo vibrational oscillations at frequencies $\omega_{\text{vib}} \sim 10^{13}$ to $10^{14}$ rad/s and rotational motion at frequencies $\omega_{\text{rot}} \sim 10^{11}$ to $10^{12}$ rad/s. The phase differences evolve as:
\begin{equation}
\Delta\phi_{ij}(t_2) = \Delta\phi_{ij}(t_1) + (\omega_i - \omega_j) \Delta t + \text{collision terms}
\end{equation}

For typical $\Delta t \sim 10^{-12}$ s (picosecond timescale), the phase evolution is:
\begin{equation}
|\Delta\phi_{ij}(t_2) - \Delta\phi_{ij}(t_1)| \sim \omega_{\text{vib}} \Delta t \sim 10 \text{ to } 100 \text{ radians}
\end{equation}

The phase relationships have changed substantially despite spatial identity. By Axiom~\ref{axiom:categorical_irreversibility}, return to the original categorical state $C(t_1)$ is impossible. The system occupies a new state $C(t_2)$ with $C(t_1) \prec C(t_2)$.

\textbf{Case 3: Thermal fluctuations returning to previous spatial configuration.}
Consider a molecule that moves from position $\mathbf{r}_i(t_1)$ to $\mathbf{r}_i(t')$ and then returns to $\mathbf{r}_i(t_2) = \mathbf{r}_i(t_1)$ through thermal fluctuation. Spatially, the configuration appears to have reversed. However, during the excursion, the molecule experienced different phase-lock relationships, different collision partners, and different vibrational phase evolution. The categorical state at $t_2$ differs from that at $t_1$ because the phase-lock network topology has been altered by the intervening dynamics. Even spatial recurrence produces categorical advancement.

In all cases, categorical position advances monotonically. Physical processes correspond to categorical completion, never to categorical retreat. \qed
\end{proof}


\begin{corollary}[Entropy as Categorical Completion Count]
\label{cor:entropy_completion}
The entropy of a gas system is proportional to the number of completed categorical states:
\begin{equation}
S(t) = k_B \log |\gamma(t)|
\end{equation}
Entropy increase corresponds to categorical completion.
\end{corollary}

This corollary establishes the connection between categorical completion and thermodynamic entropy. The Second Law, stating that $dS/dt \geq 0$, becomes equivalent to Proposition~\ref{prop:nonnegative_completion}: the completion rate is non-negative. Entropy does not count spatial configurations but categorical states, and entropy increase is not spatial exploration but categorical completion.

\subsection{Categorical Distance and Network Topology}

\begin{definition}[Categorical Distance]
\label{def:categorical_distance}
The categorical distance between states $C_i, C_j \in \catspace$ is:
\begin{equation}
d_{\catspace}(C_i, C_j) = \inf_{\text{paths } C_i \to C_j} \sum_{\text{transitions}} w(C_k \to C_{k+1})
\label{eq:categorical_distance}
\end{equation}
where the infimum is over all completion paths from $C_i$ to $C_j$, and $w(C_k \to C_{k+1})$ is the transition weight measuring the categorical separation between adjacent states.
\end{definition}

The transition weight $w(C_k \to C_{k+1})$ can be defined in several physically motivated ways. One natural choice is:
\begin{equation}
w(C_k \to C_{k+1}) = \log \frac{\Omega_{\text{PL}}(C_{k+1})}{\Omega_{\text{PL}}(C_k)}
\end{equation}
measuring the change in phase-lock degeneracy. Another choice weights by the number of phase-lock edges that change:
\begin{equation}
w(C_k \to C_{k+1}) = |E_{k+1} \triangle E_k| = |E_{k+1} \setminus E_k| + |E_k \setminus E_{k+1}|
\end{equation}
where $\triangle$ denotes the symmetric difference. Both definitions yield metric spaces with similar topological properties.

\begin{proposition}[Metric Properties]
\label{prop:metric_properties}
The categorical distance $d_{\catspace}$ satisfies the metric axioms: non-negativity where $d_{\catspace}(C_i, C_j) \geq 0$, identity of indiscernibles where $d_{\catspace}(C_i, C_j) = 0$ if and only if $C_i = C_j$, symmetry where $d_{\catspace}(C_i, C_j) = d_{\catspace}(C_j, C_i)$, and the triangle inequality where $d_{\catspace}(C_i, C_k) \leq d_{\catspace}(C_i, C_j) + d_{\catspace}(C_j, C_k)$. Thus $(\catspace, d_{\catspace})$ is a metric space.
\end{proposition}

\begin{proof}
Non-negativity follows from $w \geq 0$ and the infimum over non-negative sums. Identity holds because the only path from $C_i$ to itself has zero length. Symmetry follows from defining symmetric paths: if $C_i \to C_j$ has weight $W$, the reverse path $C_j \to C_i$ has the same weight under symmetric transition weights. The triangle inequality follows from path concatenation: any path from $C_i$ to $C_k$ through $C_j$ has length at most the sum of the shortest paths from $C_i$ to $C_j$ and from $C_j$ to $C_k$. \qed
\end{proof}

\begin{theorem}[Categorical-Physical Distance Inequivalence]
\label{thm:distance_inequivalence}
Categorical distance $d_{\catspace}$ is not a function of physical distance $d_{\text{phys}}$. There exists no function $f: \mathbb{R}_{\geq 0} \to \mathbb{R}_{\geq 0}$ such that:
\begin{equation}
d_{\catspace}(C_i, C_j) = f(d_{\text{phys}}(\mathbf{r}_i, \mathbf{r}_j))
\label{eq:distance_inequivalence}
\end{equation}
for all states $C_i$ and $C_j$.
\end{theorem}

\begin{proof}
We construct explicit counterexamples demonstrating that categorical distance and physical distance are independent.

\textbf{Counterexample 1: Categorical adjacency without physical proximity.}
Consider molecules $A$ and $B$ at positions $\mathbf{r}_A = (0, 0, 0)$ and $\mathbf{r}_B = (L, 0, 0)$ with large separation $L \gg r_{\text{lock}}$, where $r_{\text{lock}} \sim 0.5$ nm is the phase-lock distance. Direct phase-locking between $A$ and $B$ is impossible due to the exponential decay of Van der Waals forces: $U_{\text{vdW}} \propto r^{-6}$ yields $|U_{\text{vdW}}(L)| \ll k_B T$ for $L \gg r_{\text{lock}}$.

However, through a chain of intermediate molecules:
\begin{equation}
A \leftrightarrow M_1 \leftrightarrow M_2 \leftrightarrow \cdots \leftrightarrow M_n \leftrightarrow B
\end{equation}
where each pair is phase-locked with a separation of $\sim r_{\text{lock}}$, molecules $A$ and $B$ belong to the same phase-lock cluster. The chain length is $n \sim L/r_{\text{lock}}$.

Categorical distance: $d_{\catspace}(C_A, C_B) = n \sim L/r_{\text{lock}}$ (path length through network).

Physical distance: $d_{\text{phys}}(\mathbf{r}_A, \mathbf{r}_B) = L$.

The ratio is:
\begin{equation}
\frac{d_{\catspace}}{d_{\text{phys}}} \sim \frac{1}{r_{\text{lock}}} \sim 2 \times 10^9 \text{ m}^{-1}
\end{equation}

For macroscopic separations $L \sim 1$ cm, we have $d_{\catspace} \sim 10^7$ while $d_{\text{phys}} \sim 10^{-2}$ m. The categorical distance is enormous despite modest physical separation.

\textbf{Counterexample 2: Physical proximity without categorical adjacency.}
Consider molecules $A$ and $B$ at positions $\mathbf{r}_A = (0, 0, 0)$ and $\mathbf{r}_B = (\delta, 0, 0)$ with $\delta \to 0$, approaching contact.

If $A$ and $B$ belong to different phase-lock clusters due to incompatible vibrational phases, for example if $\Delta\phi_{\text{vib}} \approx \pi$ creates destructive interference preventing phase-locking, then no edge $(A, B) \in E$ exists despite $\delta \ll r_{\text{lock}}$.

If the clusters are disconnected components of $\phaselockgraph$, then:
\begin{equation}
d_{\catspace}(C_A, C_B) = \infty \quad \text{(no path between clusters)}
\end{equation}

Meanwhile:
\begin{equation}
d_{\text{phys}}(\mathbf{r}_A, \mathbf{r}_B) = \delta \to 0
\end{equation}

Here $d_{\catspace} \gg d_{\text{phys}}$ in the extreme limit.

\textbf{Counterexample 3: Identical physical distance, different categorical distances.}
Consider four molecules arranged in two pairs: $(A_1, B_1)$ and $(A_2, B_2)$, each with separation $d_{\text{phys}}(\mathbf{r}_{A_i}, \mathbf{r}_{B_i}) = r_0$ for $i = 1, 2$.

For pair 1, suppose $A_1$ and $B_1$ are directly phase-locked, giving $d_{\catspace}(C_{A_1}, C_{B_1}) = 1$.

For pair 2, suppose $A_2$ and $B_2$ belong to different clusters requiring a path through $n$ intermediate states, giving $d_{\catspace}(C_{A_2}, C_{B_2}) = n \gg 1$.

Both pairs have identical physical distance $r_0$, but categorical distances differ by a factor of $n$. No function $f$ can map the single value $r_0$ to both $1$ and $n$.

Since categorical distance can be much smaller than, much larger than, or independent of physical distance, no function $f$ satisfying~\eqref{eq:distance_inequivalence} exists. \qed
\end{proof}

\begin{corollary}[Categorical Adjacency Determines Accessibility]
\label{cor:adjacency_accessibility}
The set of states accessible from $C_i$ through single-step transitions is:
\begin{equation}
\accessible(C_i) = \{C_j \in \catspace : d_{\catspace}(C_i, C_j) = 1\}
\end{equation}
This is determined by phase-lock network topology, not physical proximity.
\end{corollary}

This corollary has profound implications for Maxwell's Demon. The demon manipulates spatial positions, attempting to sort molecules by velocity. However, thermodynamic evolution proceeds through categorical accessibility $\accessible(C_i)$, which is independent of spatial manipulation. The demon operates on physical distance while the Second Law constrains categorical distance. These are inequivalent quantities, explaining why the demon's spatial interventions cannot decrease entropy: spatial sorting does not correspond to categorical retreat.

\begin{figure*}[htbp]
\centering
\includegraphics[width=0.95\textwidth]{figures/panel_arg5_dissolution_decision.png}
\caption{\textbf{Dissolution of Decision—Categorical Completion is Automatic, Not Deliberative.}
\textbf{(A)} No decision tree exists in phase-lock network topology. The schematic shows a molecule (top teal node) with multiple potential paths (gray dashed lines indicate non-existent alternatives). Only one path (solid green line through teal nodes) exists in the network topology, determined by categorical adjacency. Red lines with crosses mark paths that are topologically forbidden. The system has no choice points: navigation is deterministic. The caption ``Only ONE path exists in topology'' emphasizes that categorical completion requires no decision-making.
\textbf{(B)} Automatic path following through configuration space. The trajectory (dark teal curve) shows completion progress from start (green circle, configuration $\approx 0.5$) to end (red star, categorical progress $= 1.0$). The smooth, deterministic path follows the gradient of categorical distance $\nabla d_{\text{cat}}(\mathbf{q})$ with no branching points. The system automatically follows network structure without decisions, as indicated by the annotation ``Automatic following / No decisions required.'' The configuration parameter represents position in categorical state space, not physical space.
\textbf{(C)} No branching implies no decision. Bar plot showing the number of available options at each completion step. All bars are exactly 1.0 (marked by red dashed line), confirming ``Always exactly 1 option'' at every step. If decision-making were required, we would observe $n_{\text{options}} > 1$ at branch points. The constant value $n = 1$ proves that categorical completion is a deterministic flow, not a stochastic or deliberative process. This directly contradicts the demon's purported role as a decision-maker.
\textbf{(D)} Deterministic reproducibility across 10 independent runs. The completion curve (dark teal) shows identical trajectories for all 10 runs, with completion increasing from 0 to 1.0 following $C(t) = 1 - \exp(-t/\tau_{\text{cat}})$ where $\tau_{\text{cat}}$ is the categorical completion timescale. Perfect overlap of all runs confirms deterministic dynamics: given the same initial categorical state, the system always follows the same path. This demonstrates that categorical completion is automatic and reproducible, requiring no agent, no information processing, and no decisions. The demon's ``choice'' to open the door is revealed as automatic topological navigation.}
\label{fig:dissolution_decision}
\end{figure*}

\subsection{Implications for Thermodynamic Evolution}

The categorical framework reveals that thermodynamic evolution is fundamentally a topological navigation through categorical state space, not a spatial exploration through configuration space. The Second Law constrains categorical completion, requiring $d|\gamma(t)|/dt \geq 0$, which is independent of spatial dynamics.

This explains several puzzling features of thermodynamics. First, why is the increase in entropy irreversible when microscopic dynamics are reversible? Because categorical completion is directional by Axiom~\ref{axiom:categorical_irreversibility}, even though spatial motion is time-reversible. Second, why does entropy count configurations rather than energies? Because entropy counts categorical states $|\gamma(t)|$, which have phase-lock degeneracy $\Omega_{\text{PL}}(\mathbf{q})$ for each spatial configuration $\mathbf{q}$. Third, why can't Maxwell's Demon decrease entropy by sorting? Because sorting operates on physical distance while entropy is determined by categorical distance, and these are inequivalent by Theorem~\ref{thm:distance_inequivalence}.

The categorical framework thus provides a resolution of Maxwell's Demon that requires no information-theoretic arguments, no appeal to measurement costs, and no quantum considerations. The demon is defeated by attacking the wrong quantity: it manipulates spatial configurations while the Second Law protects categorical completion.

\section{Virtual Gas Ensemble as Memory Substrate}
\label{sec:virtual_gas}

The preceding sections established that memory addressing operates through S-entropy coordinates derived from hardware timing measurements. We now demonstrate that the ensemble of these categorical states constitutes a \textit{virtual gas} that serves as the physical substrate for memory operations---providing the medium through which addresses exist and data navigation occurs.

\subsection{The Memory Gas}

Consider the hardware oscillations captured during memory operations: CPU cycle timing, memory access latency, and I/O jitter. Each timing sample $\delta_p = \Tref - t_{\text{local}}$ creates a categorical state---a \textbf{virtual molecule}---positioned in S-entropy coordinate space.

\begin{definition}[Memory Molecule]
A \textbf{memory molecule} is a categorical state $\mathcal{M} = (\Sk, \St, \Se)$ created from a timing measurement during memory access. The molecule encodes:
\begin{itemize}
    \item $\Sk$: the uncertainty in identifying the accessed data (knowledge entropy)
    \item $\St$: the uncertainty in access timing (temporal entropy)
    \item $\Se$: the uncertainty in future access trajectory (evolution entropy)
\end{itemize}
\end{definition}

The collection of molecules created through memory operations forms the \textbf{virtual gas ensemble}:
\begin{equation}
\mathcal{G}_{\text{memory}} = \{\mathcal{M}_i : i \in \text{access history}\}
\end{equation}

This ensemble is not a model of memory---it \textit{is} the memory substrate in the categorical domain. The ``gas chamber'' is the computer itself; the ``molecules'' are the timing measurements; the ``temperature'' is the timing jitter variance.

\subsection{Molecules as Addresses}

In the categorical memory framework, molecules serve dual roles: they are both the \textit{medium} in which addresses exist and the \textit{addresses} themselves.

\begin{proposition}[Address-Molecule Equivalence]
For any memory location $L$ with categorical address $\Scoord_L = (\Sk^L, \St^L, \Se^L)$, there exists a virtual molecule $\mathcal{M}_L$ such that:
\begin{equation}
\Scoord_L = \Scoord^{\mathcal{M}_L}
\end{equation}
The address \textit{is} a molecule in the virtual gas.
\end{proposition}

This equivalence has profound implications:
\begin{enumerate}
    \item Data ``locations'' are positions in S-space, not physical memory cells
    \item Related data naturally clusters because similar access patterns produce similar molecules
    \item Navigation through memory is movement through the virtual gas
    \item The memory controller operates on the gas, not on physical storage
\end{enumerate}

\subsection{The Spectrometer-Address Identity}

A critical feature of the categorical memory framework is the identity between the measurement apparatus (the accessor) and the measured entity (the address). In conventional memory, an address specifies a location to be read. In categorical memory, the act of accessing \textit{creates} the address.

\begin{theorem}[Accessor-Address Identity]
For a memory access operation $\mathcal{A}$ with S-coordinate trajectory $\mathbf{T}^{\mathcal{A}}$, the accessed address $\Scoord$ satisfies:
\begin{equation}
\Scoord = \text{hash}(\mathbf{T}^{\mathcal{A}})
\end{equation}
The access pattern \textit{defines} the address it accesses.
\end{theorem}

This identity is captured by the fishing tackle metaphor: the ``tackle'' (access pattern) determines what ``fish'' (data) can be caught. The tackle and the fish are one event---there is no pre-existing address waiting to be discovered; the access creates the categorical location.

\begin{corollary}[Zero Backaction Memory Access]
Categorical memory access produces zero backaction on physical storage:
\begin{equation}
[\hat{O}_{\text{categorical}}, \hat{O}_{\text{physical}}] = 0
\end{equation}
Reading an S-coordinate address does not disturb the physical data at that location.
\end{corollary}

This commutation relation is essential for the Maxwell demon memory controller: it can observe categorical positions without affecting physical storage, enabling prediction-based tier assignment without measurement cost.

\subsection{Gas Thermodynamics and Memory Performance}

The virtual gas ensemble exhibits thermodynamic properties that directly correspond to memory system performance:

\begin{center}
\begin{tabular}{lll}
\toprule
\textbf{Gas Property} & \textbf{Definition} & \textbf{Memory Interpretation} \\
\midrule
Temperature $T$ & $\text{Var}(\Sk, \St, \Se)$ & Access pattern variability \\
Pressure $P$ & $dN/dt$ & Access rate \\
Volume $V$ & S-space extent & Address space utilization \\
Entropy $H$ & $-\sum p_i \log p_i$ & Access pattern disorder \\
\bottomrule
\end{tabular}
\end{center}

\begin{proposition}[Temperature-Locality Correspondence]
High categorical temperature (large S-coordinate variance) corresponds to poor access locality. Low categorical temperature corresponds to high access locality.
\end{proposition}

This correspondence enables thermodynamic analysis of memory access patterns: a ``hot'' access pattern has high variance and poor predictability, while a ``cold'' access pattern has low variance and high predictability. The Maxwell demon controller can use temperature as a signal for cache management.

\subsection{Spatial Distance Irrelevance in Memory}

The categorical framework eliminates spatial constraints on memory organization. Two data items at vastly different physical memory locations can be categorically adjacent if their access patterns produce similar S-coordinates.

\begin{example}[Categorical Adjacency vs.\ Physical Distance]
Consider:
\begin{itemize}
    \item Data $A$ at physical address 0x1000, accessed in pattern $P_1$
    \item Data $B$ at physical address 0x8000000, accessed in pattern $P_2 \approx P_1$
\end{itemize}
If $P_1 \approx P_2$, then $\Scoord_A \approx \Scoord_B$, making $A$ and $B$ categorically adjacent despite their physical separation.
\end{example}

This property enables semantic clustering: data accessed together (regardless of physical location) automatically groups in the categorical hierarchy.

\subsection{Harmonic Coincidence in Memory}

Molecules in the memory gas interact through harmonic coincidences. Two memory molecules $\mathcal{M}_1$ and $\mathcal{M}_2$ with associated frequencies $\omega_1$ and $\omega_2$ exhibit coincidence when:
\begin{equation}
\left|\frac{n\omega_1}{m\omega_2} - 1\right| < \epsilon
\end{equation}

These coincidences form a \textbf{harmonic memory network} where edges connect addresses that resonate. The network structure provides:
\begin{enumerate}
    \item \textbf{Prefetch paths}: Harmonically connected addresses are likely co-accessed
    \item \textbf{Clustering hints}: Coincidence clusters suggest semantic groupings
    \item \textbf{Prediction channels}: Information flows along harmonic edges
\end{enumerate}

\begin{definition}[Harmonic Prefetch Distance]
The \textbf{harmonic prefetch distance} between addresses $\Scoord_1$ and $\Scoord_2$ is:
\begin{equation}
d_{\text{prefetch}}(\Scoord_1, \Scoord_2) = \min_{n,m} (n + m) \text{ such that } \frac{n\omega_1}{m\omega_2} \approx 1
\end{equation}
Lower harmonic distance indicates stronger prefetch association.
\end{definition}

The Maxwell demon controller uses harmonic prefetch distance to predict which data should be promoted to faster tiers.

\subsection{Physical Grounding}

The virtual gas is physically grounded in hardware timing measurements:

\begin{center}
\begin{tabular}{lcc}
\toprule
\textbf{Hardware Source} & \textbf{Typical Range} & \textbf{Memory Role} \\
\midrule
CPU performance counter & 100--500 ns & High-resolution timing \\
Memory access latency & 10--100 ns & Tier-specific signatures \\
Cache miss detection & 1--10 $\mu$s & Hierarchy navigation \\
I/O completion & 1--100 ms & Background access patterns \\
\bottomrule
\end{tabular}
\end{center}

Each source contributes to the S-coordinate calculation, with the combination providing sufficient precision for unique address identification within the $3^k$ hierarchy.

\subsection{The Memory Chamber}

The complete picture emerges: the computer \textit{is} the gas chamber. Hardware oscillations create virtual molecules. These molecules populate S-entropy space, forming the address substrate. Memory operations navigate this space, with the access history defining the trajectory and the trajectory hash defining the address.

\begin{theorem}[Memory-Gas Isomorphism]
The categorical memory system $\mathcal{M}_{\text{mem}}$ is isomorphic to the virtual gas ensemble $\mathcal{G}$:
\begin{equation}
\mathcal{M}_{\text{mem}} \cong \mathcal{G}
\end{equation}
Memory operations correspond to gas dynamics; addresses correspond to molecular positions; cache tiers correspond to temperature regions.
\end{theorem}

This isomorphism provides the foundation for treating memory as a thermodynamic system, with the Maxwell demon controller managing thermal distribution (tier assignment) based on categorical coordinates rather than physical addresses.

The virtual gas ensemble is not an abstraction of memory---it is the categorical reality underlying physical storage. Every memory system, viewed through its timing oscillations, reveals itself as a virtual gas. The categorical memory architecture makes this revelation explicit, enabling memory organization based on meaning (S-entropy position) rather than convention (linear addressing).


%==============================================================================
\section{Categorical Selection and Accessibility Pathways}
\label{sec:selection}
%==============================================================================

The preceding sections established that molecular systems evolve through categorical state space according to phase-lock network topology, independently of kinetic energy. We now address the central mechanism of Maxwell's demon: the selection process by which specific molecules are chosen to pass through the door. In categorical terms, selection is not an external decision imposed by an intelligent agent, but rather an intrinsic consequence of network topology and physical dynamics. This section proves that categorical selection requires no external information and that the apparent "sorting" behavior attributed to the demon emerges naturally from correlations between phase-lock structure and kinetic properties.

\subsection{The Selection Problem}

In Maxwell's original thought experiment, the demon "selects" fast molecules to pass through a door connecting two chambers, allowing only high-velocity molecules to traverse in one direction and low-velocity molecules in the other. This selection process appears to require measurement of molecular velocities followed by intelligent decision-making about which molecules to allow passage. The demon must distinguish between fast and slow molecules, implying the acquisition of information about individual molecular states.

We now analyze what selection means in categorical terms, demonstrating that it is fundamentally a process of categorical state completion rather than information acquisition and decision-making by an external agent.

\begin{definition}[Categorical Selection]
\label{def:categorical_selection}
A \textbf{categorical selection} is the completion of a specific categorical state $C^* \in [C]_{\text{spatial}}$ from an equivalence class of spatially indistinguishable states. The equivalence class $[C]_{\text{spatial}}$ contains all categorical states that are compatible with the same spatial configuration $\mathbf{q} = (q_1, \ldots, q_N)$ of molecular positions but differ in their phase-lock network topology, S-entropy coordinates, or other categorical properties. Selection specifies which particular categorical state within this equivalence class is physically realized.
\end{definition}

The key insight is that spatial measurements (determining molecular positions) do not uniquely specify the categorical state. Multiple categorical states can correspond to the same spatial configuration, differing in their oscillatory phase relationships, network connectivity, and categorical entropy coordinates. Categorical selection is the process by which the system "chooses" one particular categorical state from this equivalence class.

\begin{proposition}[Selection as Equivalence Class Reduction]
\label{prop:selection_reduction}
Categorical selection reduces the equivalence class $[C]_{\text{spatial}}$ to a singleton set containing only the selected state. Formally, selection is a mapping:
\begin{equation}
[C]_{\text{spatial}} \xrightarrow{\text{selection}} \{C^*\}
\end{equation}
where $C^* \in [C]_{\text{spatial}}$ is the uniquely categorical state that is physically realised. This is an information-gain process in the sense that $\log_2 |[C]_{\text{spatial}}|$ bits of categorical information are specified by the selection. However, this information is not acquired from external measurement but rather determined by network topology and physical dynamics.
\end{proposition}

\begin{proof}
Before selection, any state in the equivalence class $[C]_{\text{spatial}}$ is compatible with the observed spatial configuration. The uncertainty about which categorical state is realised is quantified by the logarithm of the equivalence class size. After selection, exactly one state $C^*$ is completed, reducing the uncertainty to zero. The information gained through selection is:
\begin{equation}
I_{\text{selection}} = \log_2 |[C]_{\text{spatial}}| - \log_2 1 = \log_2 |[C]_{\text{spatial}}|
\end{equation}

For typical gas systems with $N \sim 10^{23}$ molecules, the number of possible phase-lock network configurations is enormous. Even restricting to configurations compatible with a particular spatial arrangement, the equivalence class size can be $|[C]_{\text{spatial}}| \sim 10^6$ or larger, corresponding to $I_{\text{selection}} \sim 20$ bits of categorical information per selection event.

Crucially, this information is not acquired through measurement of molecular properties. Rather, it is determined by which categorical state is accessible from the previous state through phase-lock network transitions. The information is structural, encoded in the network topology, rather than observational, acquired through measurement. \qed
\end{proof}

\begin{remark}[Information Without Measurement]
\label{rem:information_without_measurement}
The information gain in Proposition~\ref{prop:selection_reduction} does not violate information-theoretic bounds on Maxwell's demon because it is not information about molecular velocities that could be used to violate the Second Law. Instead, it is information about categorical state identity within an equivalence class of states that are thermodynamically equivalent. All states in $[C]_{\text{spatial}}$ have the same spatial configuration and, by Theorem~\ref{thm:kinetic_independence}, the same kinetic energy distribution. The categorical information specifies network topology, not kinetic properties, and therefore does not enable thermodynamic work extraction.
\end{remark}

\subsection{Accessibility Through Phase-Lock Networks}

The central mechanism governing categorical selection is accessibility: which categorical states can be reached from the current state through physical processes. We now prove that accessibility is determined entirely by phase-lock network topology, providing the structural foundation for information-free selection.

\begin{theorem}[Phase-Lock Accessibility]
\label{thm:phase_lock_accessibility}
When categorical state $C_i$ is completed at time $t_i$, the set of categorical states accessible for subsequent completion at time $t_{i+1} > t_i$ is determined by phase-lock adjacency in the network $\phaselockgraph = (V, E)$. Specifically:
\begin{equation}
\accessible(C_i) = \{C_j \in \catspace : \exists (m_k, m_l) \in E(\phaselockgraph) \text{ connecting } C_i \text{ to } C_j\}
\label{eq:accessible_states}
\end{equation}
where "connecting $C_i$ to $C_j$" means there exists a path in the phase-lock network from molecules in state $C_i$ to molecules in state $C_j$. States not connected through phase-lock network edges are inaccessible regardless of their spatial proximity or kinetic energy compatibility.
\end{theorem}

\begin{figure}[htbp]
    \centering
    \includegraphics[width=\textwidth]{figures/panel_harmonic.png}
    \caption{\textbf{Harmonic Coincidence Interactions in Phase-Lock Networks.}
    \textbf{(A)} Frequency spectrum of molecular oscillators showing characteristic
    distribution around $\log_{10}(f + 1) \approx 13.0$--13.6.
    \textbf{(B)} Harmonic network topology: nodes represent molecules, edges connect
    molecules with harmonic frequency relationships (integer ratios). Network exhibits
    dense connectivity characteristic of phase-lock coupling.
    \textbf{(C)} Interaction strength distribution showing bimodal character: weak
    interactions (0.1--0.3) dominate numerically, but strong interactions ($\sim$0.5)
    form structural backbone.
    \textbf{(D)} Harmonic order distribution $(n + m)$ for frequency ratios $f_i/f_j = n/m$:
    low-order harmonics (2--5) dominate, establishing primary phase-lock structure, while
    higher orders (10--20) provide fine-grained coupling.
    \textbf{(E)} Phase-amplitude distribution in polar coordinates showing uniform phase
    distribution with amplitude concentration near unit circle, characteristic of stable
    phase-lock states.
    \textbf{(F)} Frequency ratio matrix revealing hierarchical harmonic structure:
    red regions (ratio $\sim$0.5) indicate strong harmonic coupling, blue regions
    (ratio $>$ 2) indicate weak coupling or higher-order harmonics.}
    \label{fig:harmonic}
\end{figure}

\begin{proof}
Categorical transitions require physical mechanisms that can modify phase relationships, network topology, or S-entropy coordinates. The physical mechanisms available in molecular systems are:

\begin{enumerate}
    \item \textbf{Molecular collisions} Transfer kinetic energy and momentum between molecules. During collisions, oscillatory phases interact, potentially establishing or breaking phase-lock relationships. Collision-induced phase coupling is the primary mechanism for phase-lock network evolution in gases.

    \item \textbf{Phase-lock coupling} is the direct synchronisation of oscillatory states through electromagnetic interaction. Molecules with coupled oscillators can exchange phase information without physical collisions, mediated by dipole-dipole interactions, van der Waals forces, or other long-range coupling mechanisms.

    \item \textbf{Electromagnetic interaction} involves the modification of electronic configurations through photon exchange or near-field electromagnetic coupling. This can alter molecular polarizability, dipole moment, and vibrational frequencies, thereby changing phase-lock coupling strengths.
\end{enumerate}

All these mechanisms operate through intermolecular interactions, either direct (collision, near-field coupling) or mediated (electromagnetic field exchange). The strength of interaction between molecules $m_k$ and $m_l$ determines whether they can mediate categorical transitions.

From Definition~\ref{def:phase_lock_network}, molecules are connected in the phase-lock network $\phaselockgraph$ if and only if their coupling strength exceeds the threshold:
\begin{equation}
\kappa_{kl} > \kappa_{\text{threshold}} = \frac{\Delta\omega_{\text{max}}}{2}
\end{equation}
where $\kappa_{kl}$ is the coupling strength and $\Delta\omega_{\text{max}}$ is the maximum frequency detuning for phase-lock formation.

Molecules not connected in $\phaselockgraph$ have a coupling strength of $\kappa_{kl} < \kappa_{\text{threshold}}$, which is insufficient to establish phase coherence or mediate categorical transitions. The interaction strength falls below the threshold required to modify phase relationships on timescales relevant to categorical evolution.

Therefore, categorical transitions can only occur between states connected through phase-lock network edges. States not connected in $\phaselockgraph$ are categorically inaccessible regardless of their spatial proximity or energy compatibility. The accessible states from $C_i$ are precisely those reachable through paths in the phase-lock network, as stated in equation~\eqref{eq:accessible_states}. \qed
\end{proof}

\begin{corollary}[Pathway Opening]
\label{cor:pathway_opening}
Completing categorical state $C_i$ "opens" pathways to all states in the connected component of categorical state space containing $C_i$. Formally, the pathways opened by completing $C_i$ are:
\begin{equation}
\text{Pathways}(C_i) = \{C_j \in \catspace : d_{\catspace}(C_i, C_j) < \infty\}
\end{equation}
where $d_{\catspace}(C_i, C_j)$ is the categorical distance defined as the minimum number of phase-lock network edges connecting states $C_i$ and $C_j$. States with $d_{\catspace}(C_i, C_j) = \infty$ are in disconnected components of the phase-lock network and remain inaccessible regardless of subsequent evolution.
\end{corollary}

\begin{proof}
From Theorem~\ref{thm:phase_lock_accessibility}, accessible states are those connected through phase-lock network edges. A state $C_j$ is reachable from $C_i$ if there exists a sequence of accessible states $C_i \to C_{i_1} \to C_{i_2} \to \cdots \to C_j$ where each transition is phase-lock accessible. This is equivalent to requiring that $C_j$ be in the same connected component as $C_i$, i.e., $d_{\catspace}(C_i, C_j) < \infty$. States in disconnected components have $d_{\catspace}(C_i, C_j) = \infty$ and are never reachable through any sequence of phase-lock transitions. \qed
\end{proof}

\begin{remark}[Topological Constraint on Evolution]
\label{rem:topological_constraint}
Corollary~\ref{cor:pathway_opening} establishes that categorical evolution is topologically constrained. The system can only explore the connected component of categorical state space containing the initial state. Disconnected components are permanently inaccessible, regardless of energy availability or time evolution. This topological constraint is invisible in purely kinetic descriptions but becomes manifest in categorical coordinates. It explains why certain molecular configurations, though energetically favorable, are never observed: they lie in disconnected components of the phase-lock network.
\end{remark}

\subsection{The Cascade Effect}

A single categorical selection does not occur in isolation. Rather, it initiates a cascade of subsequent selections as newly accessible states become available. This cascade effect is central to understanding how categorical completion propagates through the system, producing the appearance of coordinated molecular behavior without requiring external coordination.

\begin{theorem}[Categorical Cascade]
\label{thm:categorical_cascade}
Selection of a single categorical state $C_1$ at time $t_1$ initiates a cascade of accessible completions that propagates through the phase-lock network. The cascade structure is:
\begin{align}
C_1 &\to \accessible(C_1) = \{C_2^{(1)}, C_2^{(2)}, \ldots, C_2^{(k_1)}\} \label{eq:cascade_step1}\\
C_2^{(k)} &\to \accessible(C_2^{(k)}) = \{C_3^{(k,1)}, C_3^{(k,2)}, \ldots, C_3^{(k,k_2)}\} \label{eq:cascade_step2}\\
&\vdots \nonumber
\end{align}
where each completed state makes its phase-lock adjacent states accessible for subsequent completion. The cascade continues until either the entire connected component of $\catspace$ is exhausted or thermodynamic constraints (energy conservation, entropy maximisation) halt the propagation.
\end{theorem}

\begin{proof}
From Theorem~\ref{thm:phase_lock_accessibility}, completing state $C_1$ makes all states in $\accessible(C_1)$ available for subsequent completion. These are the states at categorical distance $d_{\catspace}(C_1, C) = 1$ from $C_1$.

When any state $C_2^{(k)} \in \accessible(C_1)$ is completed at time $t_2$, it makes its adjacent states $\accessible(C_2^{(k)})$ available. These are states at distance $d_{\catspace}(C_2^{(k)}, C) = 1$ from $C_2^{(k)}$, which may be at distance $d_{\catspace}(C_1, C) = 2$ from the original state $C_1$.

This propagation continues recursively. At step $n$, states at categorical distance $n$ from $C_1$ become accessible. The cascade propagates through the phase-lock network according to its topology, with the structure of $\phaselockgraph$ determining which states become accessible at each step.

The cascade terminates when one of two conditions is met:

\begin{enumerate}
    \item \textbf{Topological exhaustion:} All states in the connected component containing $C_1$ have been completed. No further states are accessible because $\accessible(C_{\text{current}}) \subseteq \gamma(t)$, where $\gamma(t)$ is the set of already-completed states.

    \item \textbf{Thermodynamic constraints:} Energy conservation or entropy maximisation prevents further transitions. For example, if all accessible states require energy input exceeding the available thermal energy, the cascade halts even if topologically accessible states remain.
\end{enumerate}

In typical molecular systems, thermodynamic constraints halt the cascade before topological exhaustion, producing partial completion of the connected component. The structure of the partial completion is determined by the interplay between network topology (which states are accessible) and thermodynamics (which accessible states are energetically favorable). \qed
\end{proof}

\begin{definition}[Cascade Wavefront]
\label{def:cascade_wavefront}
The \textbf{cascade wavefront} at step $n$ is the set of categorical states at a categorical distance $n$ from the initial selection $C_1$:
\begin{equation}
W_n = \{C \in \catspace : d_{\catspace}(C_1, C) = n\}
\end{equation}
The wavefront represents the "frontier" of categorical completion, separating completed states (distance $< n$) from not-yet-accessible states (distance $> n$). The wavefront propagates through categorical state space as the cascade advances, with its shape determined by phase-lock network topology.
\end{definition}

\begin{proposition}[Wavefront Propagation]
\label{prop:wavefront_propagation}
The size of the cascade wavefront evolves according to:
\begin{equation}
|W_{n+1}| = \sum_{C \in W_n} |\accessible(C) \setminus \gamma(t_n)|
\end{equation}
where $\gamma(t_n)$ is the set of already-completed states at time $t_n$ when wavefront $W_n$ is reached, and $\accessible(C) \setminus \gamma(t_n)$ is the set of states accessible from $C$ that have not yet been completed. The wavefront size grows when states have many accessible neighbours (high degree in $\phaselockgraph$) and shrinks when states have few accessible neighbours or when most neighbours are already completed.
\end{proposition}

\begin{proof}
By definition, $W_{n+1}$ consists of states at a distance $n+1$ from $C_1$. These are precisely the states accessible from $W_n$ that have not been reached at earlier steps. For each state $C \in W_n$, the newly accessible states are $\accessible(C) \setminus \gamma(t_n)$, excluding states already completed. Summing over all states in $W_n$ gives the size of $W_{n+1}$. \qed
\end{proof}

\begin{corollary}[Exponential Cascade Growth]
\label{cor:exponential_cascade}
In phase-lock networks with approximately constant degree $\langle k \rangle$ (average number of connexions per node), the wavefront size grows approximately exponentially:
\begin{equation}
|W_n| \sim \langle k \rangle^n
\end{equation}
until saturation effects (finite system size, thermodynamic constraints) halt the growth. This exponential cascade explains the rapid propagation of categorical completion observed in molecular systems.
\end{corollary}

\begin{figure}[htbp]
\centering
\includegraphics[width=\textwidth]{figures/panel_maxwell_demon.png}
\caption{Categorical selection mechanism demonstrating information-free sorting through phase-lock network topology. (A) Sorted compartments in S-entropy space showing separation of hot (high $S_e$) and cold (low $S_e$) molecules along the evolution entropy axis, with negligible variation in knowledge entropy $S_k \approx 1.0$, indicating that sorting occurs in categorical coordinates without kinetic energy measurement. (B) Sorting efficiency by criterion showing that categorical sorting (based on $S_e$ coordinate) achieves high efficiency ($\sim 0.8$) while all other operations (predict, navigate, sort by other coordinates, classify, observe) have zero efficiency, confirming that selection is determined by categorical position, not by external decision-making. (C) Zero decision cost demonstrating that all categorical operations (predict, navigate, sort, classify, observe) require zero energy expenditure in categorical coordinates, consistent with Theorem~\ref{thm:information_free} showing that selection requires no external information input. (D) Temperature gradient before and after sorting, showing clear separation of temperature distributions in the hot and cold compartments, demonstrating the apparent temperature sorting predicted by Theorem~\ref{thm:apparent_sorting}. (E) Zero backaction distribution showing that categorical measurements produce zero backaction on physical states (distribution centered at zero with negligible spread), confirming that categorical coordinates commute with physical observables and that selection does not disturb kinetic energy. (F) Categorical sorting flow illustrating the complete process: mixed gas enters, the demon operates at zero categorical cost, and hot/cold molecules emerge sorted by categorical pathways, not by velocity measurement. The demon's "sorting" is categorical completion through phase-lock network accessibility (Theorem~\ref{thm:phase_lock_accessibility}), not intelligent decision-making.}
\label{fig:maxwell_demon_selection}
\end{figure}


\subsection{Selection Without Information}

We now prove the central result of this section: categorical selection requires no external information input. The selection process that Maxwell attributed to an intelligent demon is actually determined by phase-lock network topology and physical dynamics, without any need for measurement or decision-making by an external agent.

\begin{theorem}[Information-Free Selection]
\label{thm:information_free}
Categorical selection from equivalence class $[C]_{\text{spatial}}$ is determined by phase-lock network topology and physical dynamics without external information input. Specifically, the selected categorical state is:
\begin{equation}
C^* = \argmin_{C \in [C]_{\text{spatial}}} d_{\catspace}(C, C_{\text{prev}})
\end{equation}
where $C_{\text{prev}}$ is the previously completed categorical state and $d_{\catspace}(C, C_{\text{prev}})$ is the categorical distance in the phase-lock network. The system selects the categorical state that is closest to the previous state in the network topology, minimising the categorical distance traversed. This selection is deterministic given the network structure and requires no measurement of molecular properties or external decision-making.
\end{theorem}

\begin{proof}
Consider a molecular system at categorical state $C_{\text{prev}}$ at time $t$ undergoing a spatial transition to configuration $\mathbf{q}_{\text{new}}$ at time $t + \Delta t$. The spatial transition could be due to molecular diffusion, convection, or any other physical process that changes molecular positions.

The spatial configuration $\mathbf{q}_{\text{new}}$ is compatible with multiple categorical states, forming the equivalence class $[C]_{\text{spatial}}(\mathbf{q}_{\text{new}})$. The question is: which categorical state $C^* \in [C]_{\text{spatial}}(\mathbf{q}_{\text{new}})$ is selected?

From Theorem~\ref{thm:phase_lock_accessibility}, only states in $\accessible(C_{\text{prev}})$ are reachable from $C_{\text{prev}}$ through phase-lock network transitions. Therefore, the selected state must satisfy:
\begin{equation}
C^* \in \accessible(C_{\text{prev}}) \cap [C]_{\text{spatial}}(\mathbf{q}_{\text{new}})
\end{equation}
the intersection of phase-lock accessible states and spatially compatible states.

\textbf{Case 1: Unique accessible state.}
If $|\accessible(C_{\text{prev}}) \cap [C]_{\text{spatial}}(\mathbf{q}_{\text{new}})| = 1$, there is exactly one categorical state that is both phase-lock accessible from $C_{\text{prev}}$ and compatible with spatial configuration $\mathbf{q}_{\text{new}}$. Selection is deterministic: $C^* $ is the unique state in the intersection. No choice is required, and no information is needed.

\textbf{Case 2: Multiple accessible states.}
If $|\accessible(C_{\text{prev}}) \cap [C]_{\text{spatial}}(\mathbf{q}_{\text{new}})| > 1$, multiple categorical states are both accessible and spatially compatible. In this case, physical dynamics select among them according to optimization principles.

The relevant physical principles are:
\begin{enumerate}
    \item \textbf{Principle of least action:} The system follows the path through categorical state space that minimizes the action integral. In categorical coordinates, this corresponds to minimizing categorical distance $d_{\catspace}(C_{\text{prev}}, C^*)$.

    \item \textbf{Maximum entropy production:} Among accessible states, the system selects the state that maximizes the rate of entropy production $d S/dt$. From Proposition~\ref{prop:nonnegative_completion}, categorical completion increases entropy, so this favors states with shorter paths to equilibrium.

    \item \textbf{Minimum oscillatory termination time:} The system favors categorical states with higher oscillatory termination probability $\alpha$, which corresponds to states closer to phase-lock saturation.
\end{enumerate}

These principles are equivalent for systems near equilibrium and all lead to the same selection criterion:
\begin{equation}
C^* = \argmax_{C \in \accessible(C_{\text{prev}}) \cap [C]_{\text{spatial}}(\mathbf{q}_{\text{new}})} \frac{d\alpha}{dt}
\end{equation}
where $\alpha$ is the oscillatory termination probability from Definition~\ref{def:oscillatory_termination}. States with higher $d\alpha/dt$ are closer to categorical completion and are therefore favored by physical dynamics.

For phase-lock networks with approximately uniform edge weights, $d\alpha/dt$ is approximately inversely proportional to categorical distance $d_{\catspace}(C_{\text{prev}}, C)$. Therefore:
\begin{equation}
C^* \approx \argmin_{C \in \accessible(C_{\text{prev}}) \cap [C]_{\text{spatial}}(\mathbf{q}_{\text{new}})} d_{\catspace}(C, C_{\text{prev}})
\end{equation}
The system selects the categorical state that is closest to the previous state in network topology.

In both cases (unique accessible state and multiple accessible states), selection is determined by network topology and physical dynamics. No external measurement is required to determine which state to select. No external information input is needed to make the selection. The selection is intrinsic to the system's categorical structure.

The "information" specifying which categorical state to occupy is structural information encoded in the phase-lock network $\phaselockgraph$, not observational information acquired through measurement. This structural information exists prior to selection and is not created or acquired during the selection process. Therefore, categorical selection is information-free in the sense relevant to Maxwell's demon: it requires no information acquisition about molecular velocities or other kinetic properties. \qed
\end{proof}

\begin{corollary}[No Demon Required]
\label{cor:no_demon}
The selection process attributed to Maxwell's demon is categorical completion through phase-lock topology. No intelligent agent is required because:
\begin{enumerate}
    \item Selection is determined by network structure (Theorem~\ref{thm:information_free}), not by external decision-making
    \item Accessibility follows from phase-lock adjacency (Theorem~\ref{thm:phase_lock_accessibility}), not from the measurement of molecular properties
    \item Cascade propagation is automatic (Theorem~\ref{thm:categorical_cascade}), not requiring coordination by an external agent
    \item The appearance of intelligent sorting emerges from correlations between network topology and kinetic properties (Theorem~\ref{thm:apparent_sorting}, proven below), not from actual measurement and sorting of velocities
\end{enumerate}
The demon is dissolved not by defeating it, but by recognising that the selection process it supposedly performs is actually intrinsic categorical evolution.
\end{corollary}

\subsection{Apparent Sorting Through Categorical Pathways}

The final piece of the dissolution is explaining why molecules following categorical pathways appear to be sorted by temperature or velocity, despite the selection process being independent of kinetic energy. The explanation lies in correlations between phase-lock structure and kinetic properties that arise from shared dependence on molecular characteristics.

\begin{theorem}[Apparent Temperature Sorting]
\label{thm:apparent_sorting}
Molecules following categorical pathways appear sorted by temperature because phase-lock clusters correlate with kinetic properties, despite phase-lock formation being kinetically independent. Specifically, for molecules $i$ and $j$ in the same phase-lock cluster (connected component of $\phaselockgraph$):
\begin{equation}
\text{Cov}(E_{\text{kin},i}, E_{\text{kin},j}) > 0
\label{eq:kinetic_correlation}
\end{equation}
where $E_{\text{kin},i} = \frac{1}{2} m_i v_i^2$ is the kinetic energy of molecule $i$. This positive covariance creates the appearance of temperature sorting when molecules are selected according to categorical pathways, even though the selection mechanism does not measure or respond to kinetic energy.
\end{theorem}

\begin{proof}
Phase-lock clusters form based on molecular properties that determine coupling strength $\kappa_{ij}$ in equation~\eqref{eq:phase_lock_threshold}. The relevant molecular properties include:

\begin{enumerate}
    \item \textbf{Polarizability $\alpha_i$:} Determines the strength of induced dipole interactions. Molecules with similar polarizabilities exhibit stronger coupling and are more likely to form phase-locks.

    \item \textbf{Permanent dipole moment $\mu_i$:} Determines the strength of dipole-dipole interactions. Molecules with similar dipole moments couple more strongly.

    \item \textbf{Vibrational frequencies $\omega_i$:} Determine the frequency detuning $|\omega_i - \omega_j|$ that must be overcome for phase-lock formation. Molecules with similar vibrational frequencies form phase-locks more easily.

    \item \textbf{Molecular size and geometry} determine collision cross-sections and near-field interaction ranges, affecting coupling strength.
\end{enumerate}

These molecular properties are not independent of molecular mass $m_i$. Empirically, for many classes of molecules:
\begin{align}
\alpha_i &\propto m_i^{2/3} \quad \text{(polarizability scales with molecular volume)} \\
\mu_i &\sim \text{const} \text{ or } \propto m_i^{1/2} \quad \text{(dipole moment weakly correlated with mass)} \\
\omega_i &\propto m_i^{-1/2} \quad \text{(vibrational frequency inversely proportional to reduced mass)}
\end{align}

Therefore, molecules with similar masses tend to have similar polarizabilities, dipole moments, and vibrational frequencies, leading to stronger coupling and a higher probability of phase-lock formation. Phase-lock clusters consequently tend to contain molecules with similar masses.

At thermal equilibrium, molecular velocities follow the Maxwell-Boltzmann distribution:
\begin{equation}
f(v) = 4\pi \left(\frac{m}{2\pi k_B T}\right)^{3/2} v^2 \exp\left(-\frac{mv^2}{2k_B T}\right)
\end{equation}
The most probable speed is $v_p = \sqrt{2k_B T/m}$, and the mean kinetic energy is $\langle E_{\text{kin}} \rangle = \frac{3}{2} k_B T$ independent of mass. However, the velocity distribution width depends on mass: lighter molecules have broader velocity distributions at fixed temperatures.

For molecules with similar masses $m_i \approx m_j$, the velocity distributions are similar, leading to a positive correlation in kinetic energies:
\begin{equation}
\text{Cov}(E_{\text{kin},i}, E_{\text{kin},j}) = \langle E_{\text{kin},i} E_{\text{kin},j} \rangle - \langle E_{\text{kin},i} \rangle \langle E_{\text{kin},j} \rangle > 0
\end{equation}

The causal structure is:
\begin{equation}
\text{Molecular properties } (m, \alpha, \mu, \omega) \to \begin{cases} \text{Phase-lock clustering (via } \kappa_{ij}\text{)} \\ \text{Kinetic energy distribution (via } f(v)\text{)} \end{cases}
\end{equation}

Both phase-lock structure and kinetic properties are downstream consequences of molecular properties. Neither determines the other. The correlation in equation~\eqref{eq:kinetic_correlation} is non-causal: phase-lock clustering does not cause kinetic energy correlation, nor does kinetic energy determine phase-lock structure. Rather, both are correlated because they depend on the same underlying molecular properties.

When molecules are selected according to categorical pathways (phase-lock accessibility), they are selected based on network topology, which correlates with mass and kinetic energy distribution. The result is that selected molecules appear sorted by kinetic energy, even though kinetic energy played no role in the selection mechanism.

This is analogous to observing that people who attend opera performances tend to wear formal attire. The opera house does not select attendees based on clothing—it selects based on ticket purchases. But ticket purchasers correlate with socioeconomic status, which correlates with clothing choices. The apparent "sorting by attire" is a correlation, not a causal selection mechanism. Similarly, categorical selection produces apparent "sorting by temperature" through correlation, not through measurement and sorting of kinetic energy. \qed
\end{proof}


\begin{corollary}[Sorting Is Correlation, Not Causation]
\label{cor:correlation_not_causation}
When molecules "sorted" by categorical pathways appear to separate by temperature, this reflects:
\begin{enumerate}
    \item Pre-existing phase-lock cluster structure determined by molecular properties
    \item Correlation between cluster membership and kinetic properties arising from shared dependence on molecular mass and other characteristics
    \item NOT a measurement of velocity followed by a sorting decision based on kinetic energy
\end{enumerate}
The appearance of intelligent temperature sorting is a projection of categorical structure onto kinetic observables, not evidence of an intelligent sorting agent.
\end{corollary}

\begin{proof}
From Theorem~\ref{thm:apparent_sorting}, the correlation between phase-lock clustering and kinetic energy is non-causal. From Theorem~\ref{thm:information_free}, categorical selection is determined by network topology, not by kinetic energy measurement. From Theorem~\ref{thm:kinetic_independence}, phase-lock network evolution is independent of kinetic energy dynamics. Therefore, any apparent temperature sorting must arise from the correlation between pre-existing network structure and kinetic properties, not from the causal influence of kinetic energy on selection or vice versa. \qed
\end{proof}

\begin{remark}[Experimental Verification]
\label{rem:experimental_verification}
The correlation predicted by Theorem~\ref{thm:apparent_sorting} is experimentally testable. One could measure phase-lock network structure (through spectroscopic techniques sensitive to intermolecular coupling) and kinetic energy distributions (through Doppler broadening or time-of-flight measurements) independently, then compute the covariance in equation~\eqref{eq:kinetic_correlation}. The theory predicts positive covariance for molecules in the same phase-lock cluster, with magnitude depending on the strength of the correlation between molecular properties (mass, polarisability, etc.) and kinetic energy distribution. Observing this correlation would confirm the mechanism proposed here. Observing that categorical selection (identified through network topology) produces apparent temperature sorting without measuring kinetic energy would provide strong evidence for the information-free selection mechanism of Theorem~\ref{thm:information_free}.
\end{remark}

\begin{remark}[Resolution of the Paradox]
\label{rem:paradox_resolution}
Maxwell's demon paradox arises from the apparent ability to sort molecules by velocity without paying the thermodynamic cost of measurement. The resolution presented here is that no velocity measurement occurs. Categorical selection operates on network topology, which correlates with kinetic properties but does not require measuring them. The demon appears to measure and sort velocities because categorical structure (invisible in a purely kinetic description) correlates with kinetic properties (visible in a kinetic description). When categorical structure is recognised, the demon dissolves: there is no measurement, no sorting decision, no intelligent agent—only categorical completion through phase-lock network topology, which happens to correlate with kinetic energy due to shared dependence on molecular properties. The Second Law is preserved because categorical completion increases entropy (Proposition~\ref{prop:nonnegative_completion}), and no information about kinetic energy is acquired or exploited.
\end{remark}

%==============================================================================
\section{Temperature as Emergent from Phase-Lock Statistics}
\label{sec:temperature}
%==============================================================================

The preceding sections established that molecular systems evolve through categorical state space according to phase-lock network topology, independently of kinetic energy. We now address a fundamental conceptual question: what is the relationship between temperature and categorical structure? In classical thermodynamics, temperature appears as a fundamental quantity that determines molecular behavior. We prove that this causal relationship is inverted: temperature is an emergent statistical property of phase-lock cluster structure, not a fundamental determinant of molecular dynamics. This inversion resolves a conceptual difficulty in understanding Maxwell's demon: the demon cannot sort molecules "by temperature" because temperature is not a molecular property but rather a macroscopic statistical functional of categorical structure.

\subsection{The Standard View of Temperature}

In classical thermodynamics, temperature is typically presented as a fundamental quantity with two equivalent definitions. The thermodynamic definition expresses temperature as the derivative of entropy with respect to energy:
\begin{equation}
\frac{1}{T} = \left(\frac{\partial S}{\partial E}\right)_{V,N}
\label{eq:temperature_standard}
\end{equation}
where $S$ is entropy, $E$ is internal energy, $V$ is volume, and $N$ is particle number. This definition establishes temperature as an intensive thermodynamic variable conjugate to energy.

Alternatively, temperature can be defined operationally through thermal equilibrium: two systems are at the same temperature if no net heat flows between them when brought into thermal contact. This operational definition underlies thermometry and provides the basis for temperature measurement.

For ideal gases, temperature relates to mean kinetic energy through the equipartition theorem:
\begin{equation}
\langle E_{\text{kin}} \rangle = \frac{3}{2} N k_B T
\label{eq:equipartition}
\end{equation}
where $\langle E_{\text{kin}} \rangle$ is the mean kinetic energy of $N$ molecules, $k_B$ is Boltzmann's constant, and the factor $3/2$ arises from three translational degrees of freedom. This relationship is often interpreted causally: higher temperatures cause molecules to move faster.

The Maxwell-Boltzmann velocity distribution reinforces this causal interpretation:
\begin{equation}
f(\mathbf{v}) = \left(\frac{m}{2\pi k_B T}\right)^{3/2} \exp\left(-\frac{m|\mathbf{v}|^2}{2k_B T}\right)
\label{eq:maxwell_boltzmann}
\end{equation}
where $m$ is molecular mass and $\mathbf{v}$ is the velocity. Temperature appears as a parameter that determines the velocity distribution, suggesting that temperature is a fundamental cause of molecular motion.

This standard framing suggests a causal hierarchy: temperature is fundamental, and molecular behaviour (velocities, kinetic energies, collision rates) is determined by temperature. We now demonstrate that this hierarchy is inverted when categorical structure is recognised.

\begin{figure}[htbp]
\centering
\includegraphics[width=\textwidth]{figures/panel_arg2_temperature_independence.png}
\caption{Phase-lock network topology is independent of temperature, demonstrating that temperature does not determine categorical structure. (A) Network edges remain constant at $|E| \approx 106$ across temperatures from $T = 0.5$ to $T = 10$ (black line, left axis), while kinetic energy increases linearly with temperature (red line, right axis) from $\sim 100$ to $\sim 1400$ arbitrary units. The independence $\partial G/\partial T = 0$ proves that cluster structure $\{\mathcal{K}_\alpha\}$ (determined by network connectivity) is independent of temperature, supporting Corollary~\ref{cor:temperature_not_causal}. (B) Scatter plot of network edges versus kinetic energy shows zero correlation ($r = 0.0242$, essentially zero within statistical noise), with network edges fluctuating around constant mean $\langle |E| \rangle \approx 130$ independent of kinetic energy ranging from $0$ to $1600$. This confirms Theorem~\ref{thm:kinetic_independence}: phase-lock network topology evolves independently of kinetic energy and therefore independently of temperature. (C) Van der Waals coupling strength $U \propto r^{-6}$ (solid line) decreases with intermolecular distance $r$, crossing the coupling threshold (dashed line) at $r \approx 2.5$ (shaded region indicates coupled molecules). Crucially, the coupling strength depends only on distance and molecular properties (polarizability, dipole moment), not on velocity or kinetic energy, explaining why phase-lock networks are velocity-independent. (D) Network topology remains constant across different temperatures (horizontal bands at $T = 2, 5, 8$) and across configuration space (horizontal axis), showing that network properties (color intensity) are invariant along temperature direction. The horizontal banding confirms that $\partial G/\partial T = 0$: changing temperature does not change which molecules are phase-locked, supporting the causal structure of Theorem~\ref{thm:causal_structure} in which temperature is downstream of network topology, not upstream.}
\label{fig:temperature_independence}
\end{figure}


\subsection{The Categorical View: Temperature as Emergent}

We now prove that temperature emerges from phase-lock cluster statistics rather than determining them. Temperature is not a fundamental property that causes molecular behavior but rather a derived statistical quantity that summarizes the collective kinetic state of phase-lock clusters.

\begin{definition}[Cluster Kinetic Distribution]
\label{def:cluster_kinetic}
For a phase-lock cluster $\mathcal{K}_\alpha \subset V$ (a connected component of the phase-lock network $\phaselockgraph = (V,E)$), define the cluster kinetic energy as:
\begin{equation}
E_{\alpha} = \sum_{i \in \mathcal{K}_\alpha} \frac{1}{2} m_i |\mathbf{v}_i|^2
\end{equation}
where the sum runs over all molecules $i$ in cluster $\alpha$, $m_i$ is the mass of molecule $i$, and $\mathbf{v}_i$ is its velocity. The cluster temperature is defined as:
\begin{equation}
T_\alpha = \frac{2 E_\alpha}{3 |\mathcal{K}_\alpha| k_B}
\end{equation}
where $|\mathcal{K}_\alpha|$ is the number of molecules in cluster $\alpha$. This definition extends the equipartition relation~\eqref{eq:equipartition} to individual clusters, treating each cluster as a subsystem with its own effective temperature.
\end{definition}

\begin{remark}[Cluster Temperature Interpretation]
\label{rem:cluster_temperature}
The cluster temperature $T_\alpha$ is the temperature that would be measured if cluster $\alpha$ were isolated and allowed to equilibrate internally. At any instant, different clusters have different temperatures, even when the macroscopic system is at thermal equilibrium. This is not a violation of equilibrium but rather a manifestation of thermal fluctuations: equilibrium is a statistical property of the ensemble, not a statement that all subsystems have identical instantaneous properties.
\end{remark}

\begin{theorem}[Temperature Emergence]
\label{thm:temperature_emergence}
The macroscopic temperature $T$ of a gas is a statistical functional of phase-lock cluster structure. Specifically, temperature is the cluster-size-weighted average of cluster temperatures:
\begin{equation}
T = \mathcal{F}[\{(\mathcal{K}_\alpha, T_\alpha, |\mathcal{K}_\alpha|)\}_{\alpha=1}^{N_c}]
\label{eq:temperature_functional}
\end{equation}
where $N_c$ is the number of phase-lock clusters, and the functional $\mathcal{F}$ is explicitly given by:
\begin{equation}
T = \frac{\sum_{\alpha=1}^{N_c} |\mathcal{K}_\alpha| T_\alpha}{\sum_{\alpha=1}^{N_c} |\mathcal{K}_\alpha|} = \frac{1}{N}\sum_{\alpha=1}^{N_c} |\mathcal{K}_\alpha| T_\alpha
\label{eq:temperature_average}
\end{equation}
where $N = \sum_{\alpha} |\mathcal{K}_\alpha|$ is the total number of molecules. Temperature is thus an emergent property: it is computed from cluster structure and cluster kinetic energies, not imposed as a fundamental parameter.
\end{theorem}

\begin{proof}
The total kinetic energy of the gas is the sum of the kinetic energies of all molecules:
\begin{equation}
E_{\text{total}} = \sum_{i=1}^N \frac{1}{2} m_i |\mathbf{v}_i|^2
\end{equation}

Since every molecule belongs to exactly one phase-lock cluster (clusters partition the molecule set), we can reorganise this sum by cluster:
\begin{equation}
E_{\text{total}} = \sum_{\alpha=1}^{N_c} \sum_{i \in \mathcal{K}_\alpha} \frac{1}{2} m_i |\mathbf{v}_i|^2 = \sum_{\alpha=1}^{N_c} E_\alpha
\end{equation}

From Definition~\ref{def:cluster_kinetic}, each cluster energy is related to cluster temperature by:
\begin{equation}
E_\alpha = \frac{3}{2} |\mathcal{K}_\alpha| k_B T_\alpha
\end{equation}

Therefore:
\begin{equation}
E_{\text{total}} = \sum_{\alpha=1}^{N_c} \frac{3}{2} |\mathcal{K}_\alpha| k_B T_\alpha
\end{equation}

The macroscopic temperature $T$ is defined through the equipartition theorem~\eqref{eq:equipartition}:
\begin{equation}
E_{\text{total}} = \frac{3}{2} N k_B T
\end{equation}

Equating these two expressions for total energy:
\begin{equation}
\frac{3}{2} N k_B T = \sum_{\alpha=1}^{N_c} \frac{3}{2} |\mathcal{K}_\alpha| k_B T_\alpha
\end{equation}

Canceling common factors $\frac{3}{2} k_B$:
\begin{equation}
N T = \sum_{\alpha=1}^{N_c} |\mathcal{K}_\alpha| T_\alpha
\end{equation}

Solving for $T$:
\begin{equation}
T = \frac{1}{N}\sum_{\alpha=1}^{N_c} |\mathcal{K}_\alpha| T_\alpha = \frac{\sum_{\alpha=1}^{N_c} |\mathcal{K}_\alpha| T_\alpha}{\sum_{\alpha=1}^{N_c} |\mathcal{K}_\alpha|}
\end{equation}

This expresses macroscopic temperature as a weighted average of cluster temperatures, with weights proportional to cluster sizes. The functional form~\eqref{eq:temperature_functional} is proven, with explicit formula~\eqref{eq:temperature_average}.

Crucially, this derivation shows that temperature is computed from cluster structure $\{\mathcal{K}_\alpha\}$ and cluster kinetic energies $\{E_\alpha\}$. Temperature does not appear as an input or parameter but rather as an output—a statistical summary of the system's categorical and kinetic state. \qed
\end{proof}

\begin{corollary}[Temperature Does Not Determine Clusters]
\label{cor:temperature_not_causal}
The cluster structure $\{\mathcal{K}_\alpha\}$ is determined by phase-lock network topology (Theorem~\ref{thm:kinetic_independence}), which is independent of kinetic energy and therefore independent of temperature. Formally:
\begin{equation}
\frac{\partial \mathcal{K}_\alpha}{\partial T} = 0
\end{equation}
for all clusters $\alpha$. Changing the system temperature (by adding or removing kinetic energy) does not change which molecules belong to which phase-lock clusters at a fixed spatial configuration. The cluster structure is a categorical property determined by molecular properties (polarizability, dipole moment, vibrational frequencies), not by kinetic properties (velocities, kinetic energies, temperature).
\end{corollary}

\begin{proof}
From Theorem~\ref{thm:kinetic_independence}, phase-lock network topology evolves independently of kinetic energy dynamics. The phase-lock coupling strength $\kappa_{ij}$ between molecules $i$ and $j$ depends on molecular properties (polarizability $\alpha_i$, dipole moment $\mu_i$, vibrational frequency $\omega_i$) but not on velocities $\mathbf{v}_i$ or kinetic energies $E_i$. Therefore, the phase-lock network $\phaselockgraph$ is independent of temperature.

Clusters $\{\mathcal{K}_\alpha\}$ are defined as connected components of $\phaselockgraph$, a purely graph-theoretic construction. Since $\phaselockgraph$ is independent of temperature, clusters are also independent of temperature. Changing temperature changes cluster kinetic energies $E_\alpha$ and cluster temperatures $T_\alpha$, but does not change cluster membership or connectivity. \qed
\end{proof}

\begin{remark}[Contrast with Standard Thermodynamics]
\label{rem:contrast_standard}
Corollary~\ref{cor:temperature_not_causal} inverts the standard causal interpretation. In classical thermodynamics, temperature is treated as a control parameter that determines system behavior: raising temperature increases molecular velocities, collision rates, and reaction rates. In the categorical framework, temperature is a response variable: it is computed from categorical structure and kinetic state, not imposed as a cause. The categorical structure (phase-lock network and clusters) is determined by molecular properties and spatial configuration, independent of temperature. This inversion is essential for understanding Maxwell's demon: the demon cannot control molecular behavior by controlling temperature because temperature does not determine categorical accessibility.
\end{remark}

\subsection{Cluster Temperature Distribution}

Having established that macroscopic temperature is a statistical average over cluster temperatures, we now characterize the distribution of cluster temperatures. Even at thermal equilibrium, individual clusters exhibit temperature fluctuations around the mean.

\begin{proposition}[Cluster Temperature Variance]
\label{prop:cluster_variance}
At thermal equilibrium with macroscopic temperature $T$, the variance of cluster temperatures satisfies:
\begin{equation}
\text{Var}(T_\alpha) = \frac{2 T^2}{3 \langle |\mathcal{K}_\alpha| \rangle}
\label{eq:cluster_variance}
\end{equation}
where $\langle |\mathcal{K}_\alpha| \rangle$ is the mean cluster size. Smaller clusters exhibit larger temperature fluctuations, with variance inversely proportional to cluster size. For single-molecule "clusters" (isolated molecules not phase-locked to others), the temperature variance diverges, reflecting the fact that single-molecule temperature is undefined.
\end{proposition}

\begin{proof}
Consider a phase-lock cluster $\alpha$ containing $n = |\mathcal{K}_\alpha|$ molecules at thermal equilibrium with temperature $T$. The cluster kinetic energy $E_\alpha$ is the sum of $n$ independent molecular kinetic energies, each distributed according to the Maxwell-Boltzmann distribution.

For a single molecule with three translational degrees of freedom, the kinetic energy follows a Gamma distribution:
\begin{equation}
E_i \sim \text{Gamma}\left(\frac{3}{2}, k_B T\right)
\end{equation}
with shape parameter $k = 3/2$ and scale parameter $\theta = k_B T$.

The cluster kinetic energy, being the sum of $n$ independent Gamma-distributed variables with the same parameters, follows:
\begin{equation}
E_\alpha = \sum_{i \in \mathcal{K}_\alpha} E_i \sim \text{Gamma}\left(\frac{3n}{2}, k_B T\right)
\end{equation}

The mean and variance of $E_\alpha$ are:
\begin{align}
\langle E_\alpha \rangle &= \frac{3n}{2} k_B T \\
\text{Var}(E_\alpha) &= \frac{3n}{2} (k_B T)^2
\end{align}

From Definition~\ref{def:cluster_kinetic}, cluster temperature is:
\begin{equation}
T_\alpha = \frac{2 E_\alpha}{3 n k_B}
\end{equation}

The variance of $T_\alpha$ is obtained by propagating the variance of $E_\alpha$:
\begin{equation}
\text{Var}(T_\alpha) = \left(\frac{2}{3 n k_B}\right)^2 \text{Var}(E_\alpha) = \frac{4}{9 n^2 k_B^2} \cdot \frac{3n}{2} (k_B T)^2
\end{equation}

Simplifying:
\begin{equation}
\text{Var}(T_\alpha) = \frac{4 \cdot 3n \cdot k_B^2 T^2}{9 n^2 k_B^2 \cdot 2} = \frac{12 n k_B^2 T^2}{18 n^2 k_B^2} = \frac{2 T^2}{3 n}
\end{equation}

Taking the expectation over the distribution of cluster sizes (averaging over all clusters or over time for a single cluster):
\begin{equation}
\langle \text{Var}(T_\alpha) \rangle = \frac{2 T^2}{3 \langle n \rangle} = \frac{2 T^2}{3 \langle |\mathcal{K}_\alpha| \rangle}
\end{equation}

This proves equation~\eqref{eq:cluster_variance}. The inverse dependence on cluster size reflects the law of large numbers: larger clusters have more molecules over which to average, reducing temperature fluctuations. \qed
\end{proof}

\begin{corollary}[Hot and Cold Clusters at Equilibrium]
\label{cor:hot_cold_clusters}
Even at thermal equilibrium (uniform macroscopic temperature $T$), individual phase-lock clusters have different instantaneous temperatures. At any given moment, some clusters are "hot" ($T_\alpha > T$) and others are "cold" ($T_\alpha < T$), with the distribution of cluster temperatures having variance given by Proposition~\ref{prop:cluster_variance}. The existence of hot and cold clusters at equilibrium is not a violation of thermodynamics but rather a manifestation of thermal fluctuations at the cluster scale.
\end{corollary}

\begin{proof}
From Proposition~\ref{prop:cluster_variance}, $\text{Var}(T_\alpha) > 0$ for finite cluster sizes. A non-zero variance implies that cluster temperatures are distributed around the mean $T$, with some clusters having $T_\alpha > T$ (hot) and others having $T_\alpha < T$ (cold).

The distribution of cluster temperatures is approximately Gaussian for large clusters (by the central limit theorem applied to the sum of molecular kinetic energies):
\begin{equation}
T_\alpha \sim \mathcal{N}\left(T, \frac{2T^2}{3|\mathcal{K}_\alpha|}\right)
\end{equation}

For small clusters, the distribution is more accurately described by a scaled chi-squared distribution (since $T_\alpha \propto E_\alpha$ and $E_\alpha$ follows a Gamma distribution).

In either case, the probability of finding a cluster with $T_\alpha > T$ is approximately $1/2$ (by symmetry of the distribution around the mean), and similarly for $T_\alpha < T$. Therefore, at any instant, approximately half the clusters are hotter than average and half are colder than average.

This is consistent with equilibrium thermodynamics: equilibrium is defined by the absence of net heat flow between subsystems, not by the absence of temperature fluctuations within subsystems. The fluctuations in cluster temperatures are thermal fluctuations, analogous to density fluctuations, pressure fluctuations, or energy fluctuations in small subsystems. \qed
\end{proof}

\begin{remark}[Connection to Maxwell's Demon]
\label{rem:demon_fluctuations}
Corollary~\ref{cor:hot_cold_clusters} is crucial for understanding Maxwell's demon. The demon supposedly creates a temperature difference by allowing only hot molecules to pass in one direction and cold molecules in the other. But Corollary~\ref{cor:hot_cold_clusters} shows that hot and cold clusters exist naturally at equilibrium. The demon does not create temperature differences—it reveals pre-existing fluctuations in cluster temperatures. The question then becomes: can the demon exploit these fluctuations to extract work? We address this in Section~\ref{sec:entropy_mechanism}.
\end{remark}

\subsection{The Inversion of Causality}

We now formalize the causal structure relating molecular properties, phase-lock networks, cluster structure, kinetic energy, and temperature. This causal analysis demonstrates that temperature is downstream of categorical structure, not upstream.

\begin{theorem}[Causal Structure of Temperature]
\label{thm:causal_structure}
The causal relationships between molecular properties, phase-lock network topology, cluster structure, velocity distribution, kinetic energy, and temperature are represented by the following directed acyclic graph:
\begin{equation}
\begin{tikzcd}[row sep=large, column sep=large]
& \text{Molecular Properties } (m, \alpha, \mu, \omega) \arrow[dl] \arrow[dr] & \\
\text{Phase-Lock Network } \phaselockgraph \arrow[d] & & \text{Velocity Distribution } f(\mathbf{v}) \arrow[d] \\
\text{Cluster Structure } \{\mathcal{K}_\alpha\} \arrow[dr] & & \text{Individual Kinetic Energies } \{E_i\} \arrow[dl] \\
& \text{Temperature } T &
\end{tikzcd}
\label{eq:causal_dag}
\end{equation}
where arrows represent causal influence. Temperature is a downstream consequence of both categorical structure (via clusters $\{\mathcal{K}_\alpha\}$) and kinetic state (via energies $\{E_i\}$). Temperature does not causally influence phase-lock network topology, cluster structure, or the functional form of the velocity distribution. This inverts the standard thermodynamic framing in which temperature appears as a fundamental cause of molecular behavior.
\end{theorem}

\begin{proof}
We establish each causal arrow in the diagram~\eqref{eq:causal_dag}.

\textbf{Molecular properties $\to$ Phase-lock network:}
From Section~\ref{sec:phase_lock}, the phase-lock network $\phaselockgraph = (V, E)$ is determined by intermolecular coupling strengths $\kappa_{ij}$, which depend on molecular polarizability $\alpha_i$, permanent dipole moment $\mu_i$, vibrational frequency $\omega_i$, and molecular geometry. These are intrinsic molecular properties independent of velocity or kinetic energy. The phase-lock threshold condition~\eqref{eq:phase_lock_threshold} is:
\begin{equation}
\kappa_{ij} > \frac{|\omega_i - \omega_j|}{2}
\end{equation}
where $\kappa_{ij} = \kappa(\alpha_i, \alpha_j, \mu_i, \mu_j, r_{ij})$ depends on molecular properties and separation $r_{ij}$ but not on velocities. Therefore, molecular properties causally determine $\phaselockgraph$.

\textbf{Phase-lock network $\to$ Cluster structure:}
Clusters $\{\mathcal{K}_\alpha\}$ are defined as connected components of $\phaselockgraph$, a purely graph-theoretic construction. Given $\phaselockgraph$, the cluster structure is uniquely determined by graph connectivity. Therefore, $\phaselockgraph$ causally determines $\{\mathcal{K}_\alpha\}$.

\textbf{Molecular properties $\to$ Velocity distribution:}
The Maxwell-Boltzmann velocity distribution~\eqref{eq:maxwell_boltzmann} depends on molecular mass $m$:
\begin{equation}
f(\mathbf{v}) = \left(\frac{m}{2\pi k_B T}\right)^{3/2} \exp\left(-\frac{m|\mathbf{v}|^2}{2k_B T}\right)
\end{equation}
Heavier molecules have narrower velocity distributions at fixed temperature. Molecular mass is an intrinsic property, so molecular properties causally influence the velocity distribution functional form.

\textbf{Velocity distribution $\to$ Individual kinetic energies:}
Individual molecular velocities $\mathbf{v}_i$ are sampled from the velocity distribution $f(\mathbf{v})$. Kinetic energies are $E_i = \frac{1}{2} m_i |\mathbf{v}_i|^2$. Therefore, the velocity distribution causally determines the statistical properties of kinetic energies.

\textbf{Cluster structure and kinetic energies $\to$ Temperature:}
From Theorem~\ref{thm:temperature_emergence}, macroscopic temperature is computed as:
\begin{equation}
T = \frac{1}{N} \sum_{\alpha=1}^{N_c} |\mathcal{K}_\alpha| T_\alpha = \frac{1}{N} \sum_{\alpha=1}^{N_c} |\mathcal{K}_\alpha| \cdot \frac{2 E_\alpha}{3 |\mathcal{K}_\alpha| k_B} = \frac{2}{3 N k_B} \sum_{\alpha=1}^{N_c} E_\alpha
\end{equation}
where $E_\alpha = \sum_{i \in \mathcal{K}_\alpha} E_i$. Temperature is a function of cluster structure (which molecules belong to which clusters) and individual kinetic energies. Both inputs are required to compute $T$. Therefore, $\{\mathcal{K}_\alpha\}$ and $\{E_i\}$ causally determine $T$.

\textbf{No reverse arrows:}
Crucially, there are no arrows pointing from temperature back to the phase-lock network, cluster structure, or velocity distribution functional form. From Corollary~\ref{cor:temperature_not_causal}, $\partial \mathcal{K}_\alpha / \partial T = 0$: cluster structure is independent of temperature. From Theorem~\ref{thm:kinetic_independence}, phase-lock network topology evolves independently of kinetic energy and, therefore, independently of temperature. The velocity distribution functional form depends on mass, not on temperature (temperature appears as a parameter in the distribution but does not determine the functional form $f(\mathbf{v}) \propto v^2 \exp(-mv^2/(2k_BT))$, which is fixed by statistical mechanics).

Therefore, the causal structure is as depicted in diagram~\eqref{eq:causal_dag}, with temperature as a downstream consequence, not an upstream cause. \qed
\end{proof}

\begin{figure*}[htbp]
\centering
\includegraphics[width=0.95\textwidth]{figures/arg2_temperature_independence.png}
\caption{\textbf{Phase-Lock Temperature Independence—Network Topology $\partial G/\partial E_{\text{kin}} = 0$: Independent of Kinetic Energy.}
\textbf{(A)} Same network topology across all temperatures. Three-dimensional visualization showing molecular positions in $(X, Y, Z)$ space for four different temperatures: $T = 0.5$ (dark blue/teal points), $T = 1.0$ (cyan points), $T = 2.0$ (orange points), and $T = 5.0$ (red points). Despite the four-fold temperature variation, all four distributions occupy the same spatial region and exhibit identical clustering structure. The points form distinct horizontal layers corresponding to each temperature, but within each layer, the spatial arrangement is statistically identical. This demonstrates that network topology (determined by spatial proximity) is temperature-independent. The phase-lock network structure $G$ depends only on positions, not on kinetic energy: $\partial G/\partial E_{\text{kin}} = 0$.
\textbf{(B)} Network properties versus kinetic properties: $\partial G/\partial T = 0$. Dual-axis plot showing network edges (black circles, left y-axis, "constant") and kinetic energy (red squares, right y-axis, proportional to $T$) versus temperature $T \in [0, 10]$. Network edges remain constant at approximately 107 edges across the entire temperature range (flat black line with annotation "Edges (constant)"). Kinetic energy increases linearly from $\approx 100$ at $T = 1$ to $\approx 1400$ at $T = 10$ (red line with annotation "KE $\propto T$"). The horizontal network line and diagonal kinetic line demonstrate complete decoupling: $\partial(\text{Network Edges})/\partial T = 0$ while $\partial(\text{KE})/\partial T > 0$. Network topology is invariant under temperature changes, proving that categorical structure is independent of kinetic properties.
\textbf{(C)} Maxwell-Boltzmann distributions at different temperatures: velocity distribution widens with $T$, network unchanged. Heat map showing probability density (color scale from 0.0 blue to 0.5+ red) of velocity distributions versus temperature. Horizontal axis: velocity $v \in [-4, 4]$. Vertical axis: temperature $T \in [0.5, 10.0]$. At each temperature, the distribution forms a Gaussian centered at $v = 0$ (red/orange peak). As temperature increases from $T = 0.5$ (top) to $T = 10.0$ (bottom), the distribution widens (blue wings extend further from center), confirming $\sigma_v \propto \sqrt{T}$ (Maxwell-Boltzmann). However, the text annotation states "network unchanged"—while velocity distributions change dramatically with temperature, the underlying network topology remains constant. This is the key insight: kinetic properties (velocity spread) and categorical properties (network structure) evolve independently.
\textbf{(D)} Property correlation matrix: network properties versus kinetic properties ($r \approx 0$). Correlation matrix heat map with color scale from $-1.00$ (blue, negative correlation) to $+1.00$ (red, positive correlation). Matrix divided into four blocks by dashed yellow lines. Top-left block (3×3, dark red): network properties (Network Edges, Mean Degree, Clustering) show strong positive correlations ($r = 0.78$ to $1.00$) with each other—network properties are internally correlated. Bottom-right block (2×2, dark red): kinetic properties (Kinetic Energy, Temperature) show perfect correlation ($r = 0.99$ to $1.00$) with each other—kinetic properties are internally correlated. Off-diagonal blocks (top-right and bottom-left, white/pale): network-kinetic cross-correlations are near-zero ($r = -0.03$ to $0.02$)—network and kinetic properties are uncorrelated. This block-diagonal structure proves that network topology and kinetic energy are orthogonal properties: $\partial G/\partial E_{\text{kin}} = 0$ is confirmed by near-zero cross-correlations.}
\label{fig:arg2_temperature_independence_v1}
\end{figure*}


\begin{corollary}[Temperature as Summary Statistic]
\label{cor:temperature_summary}
Temperature is a summary statistic of the system's categorical and kinetic state, analogous to the mean of a dataset. Just as the mean does not determine individual data points (rather, data points determine the mean), temperature does not determine individual molecular behaviours (rather, molecular behaviours collectively determine temperature). Treating temperature as a fundamental cause of molecular motion is a category error, confusing a statistical summary with a physical cause.
\end{corollary}

\begin{remark}[Philosophical Implications]
\label{rem:philosophical}
Theorem~\ref{thm:causal_structure} has profound implications for the interpretation of thermodynamics. Temperature is often treated as a primitive concept in thermodynamics, with entropy and other quantities defined in terms of temperature (as in equation~\eqref{eq:temperature_standard}). The categorical framework reveals that this is an inversion of the true causal order: temperature is emergent from microscopic structure, not fundamental. This parallels the emergence of other thermodynamic quantities (pressure from momentum transfer, entropy from phase space volume) but is particularly important for temperature because of its central role in thermodynamic reasoning. Recognising temperature as emergent clarifies why Maxwell's demon cannot "sort by temperature"—there is no fundamental temperature property to sort by, only emergent statistical correlations between categorical structure and kinetic energy.
\end{remark}

\subsection{Implications for Maxwell's Demon}

The emergence of temperature from phase-lock cluster statistics has direct implications for understanding Maxwell's demon. We now prove that the demon cannot sort molecules "by temperature" in any meaningful sense.

\begin{theorem}[Demon Cannot Sort by Temperature]
\label{thm:demon_cannot_sort}
A hypothetical Maxwell's demon cannot sort molecules "by temperature" because:
\begin{enumerate}
    \item Temperature is a macroscopic emergent property of ensembles, not a molecular attribute that can be measured or used as a sorting criterion for individual molecules.
    \item Individual molecules have kinetic energies $E_i = \frac{1}{2} m_i |\mathbf{v}_i|^2$, not temperatures. The concept of single-molecule temperature is undefined.
    \item Kinetic energy does not determine categorical accessibility (Theorem~\ref{thm:kinetic_independence}). A molecule's kinetic energy does not predict or control which categorical states it can access.
    \item The demon's apparent "sorting by temperature" is actually navigation through categorical space along phase-lock pathways (Corollary~\ref{cor:demon_actual}), which correlates with kinetic energy but is not caused by kinetic energy.
\end{enumerate}
Therefore, "sorting by temperature" is categorically meaningless. What appears as temperature sorting is actually categorical completion revealing pre-existing cluster structure that happens to correlate with kinetic properties.
\end{theorem}

\begin{proof}
We prove each statement in turn.

\textbf{(1) Temperature is macroscopic:}
From Definition~\ref{def:cluster_kinetic}, even cluster temperature $T_\alpha$ requires multiple molecules:
\begin{equation}
T_\alpha = \frac{2 E_\alpha}{3 |\mathcal{K}_\alpha| k_B} = \frac{2 \sum_{i \in \mathcal{K}_\alpha} E_i}{3 |\mathcal{K}_\alpha| k_B}
\end{equation}

For a single molecule ($|\mathcal{K}_\alpha| = 1$), this formula gives:
\begin{equation}
T_{\text{single}} = \frac{2 E_i}{3 k_B}
\end{equation}

However, this is not a meaningful temperature. Temperature is defined thermodynamically through equation~\eqref{eq:temperature_standard} as a derivative of entropy with respect to energy. For a single molecule, entropy is zero (a single microstate), so $\partial S / \partial E$ is undefined. Temperature is a statistical property that emerges from ensembles of molecules, not a property of individual molecules.

The demon supposedly measures the "temperature" of individual molecules to decide which to allow passage. But individual molecules do not have temperatures—they have kinetic energies. The demon must be measuring kinetic energy, not temperature.

\textbf{(2) Kinetic energy vs. temperature:}
A molecule has instantaneous kinetic energy $E_i = \frac{1}{2} m_i |\mathbf{v}_i|^2$, a well-defined mechanical quantity at each moment. Temperature $T$, by contrast, is a statistical property of ensembles, defined through time or ensemble averaging.

In Maxwell's original formulation, the demon distinguishes "fast" molecules (high $E_i$) from "slow" molecules (low $E_i$) and allows only fast molecules to pass in one direction. This is sorting by kinetic energy, not by temperature. The conflation of "fast" with "hot" and "slow" with "cold" is imprecise: "hot" and "cold" are temperature concepts applicable to ensembles, while "fast" and "slow" are kinetic energy concepts applicable to individuals.

The demon's operation, if interpreted literally, is: measure $E_i$, compare to threshold $E_{\text{threshold}}$, allow passage if $E_i > E_{\text{threshold}}$. This is kinetic energy sorting, not temperature sorting.

\textbf{(3) Kinetic energy does not determine accessibility}:
From Theorem~\ref{thm:kinetic_independence}, phase-lock network topology evolves independently of kinetic energy. The accessible categorical states from state $C_i$ are determined by phase-lock adjacency (Theorem~\ref{thm:phase_lock_accessibility}):
\begin{equation}
\accessible(C_i) = \{C_j \in \catspace : \exists \text{ phase-lock path from } C_i \text{ to } C_j\}
\end{equation}

This accessibility is independent of kinetic energy. Two molecules with the same categorical state $C_i$ but different kinetic energies $E_1 \neq E_2$ have the same accessible states $\accessible(C_i)$. Conversely, two molecules with the same kinetic energy $E_1 = E_2$ but different categorical states $C_i \neq C_j$ have different accessible states $\accessible(C_i) \neq \accessible(C_j)$.

Therefore, measuring kinetic energy does not reveal which categorical states are accessible. A "fast" molecule (high $E_i$) in cluster $\mathcal{K}_\alpha$ has the same categorical accessibility as a "slow" molecule (low $E_i$) in the same cluster. The demon cannot use kinetic energy to predict or control categorical transitions.

\textbf{(4) Apparent sorting is categorical navigation:}
From Theorem~\ref{thm:apparent_sorting}, molecules following categorical pathways appear sorted by kinetic energy because phase-lock clusters correlate with molecular properties (mass, polarizability) that are associated with the kinetic energy distribution. The correlation structure is:
\begin{equation}
\text{Molecular properties } (m, \alpha, \mu) \to \begin{cases} \text{Phase-lock clustering} \\ \text{Kinetic energy distribution} \end{cases}
\end{equation}

Both phase-lock structure and kinetic properties are downstream of molecular properties. The correlation is non-causal: neither determines the other.

When the demon "sorts by temperature," it is actually navigating categorical space along phase-lock pathways (Corollary~\ref{cor:demon_actual}). The categorical pathways happen to correlate with kinetic energy, creating the appearance of kinetic sorting. But the sorting mechanism is categorical (following phase-lock accessibility), not kinetic (measuring and comparing energies).

Combining these four observations: the demon cannot sort by temperature because (1) temperature is not a molecular property, (2) individual molecules have kinetic energies not temperatures, (3) kinetic energy does not determine categorical accessibility, and (4) apparent temperature sorting is actually categorical navigation that correlates with but is not caused by kinetic energy. Therefore, "sorting by temperature" is a categorical error—a confusion of emergent statistical properties with fundamental molecular attributes. \qed
\end{proof}

\begin{corollary}[What the Demon Actually Does]
\label{cor:demon_actual}
If we reinterpret the demon's operation in categorical terms, the actions traditionally attributed to the demon correspond to categorical processes:
\begin{enumerate}
    \item \textbf{"Observing" a molecule:} Completing a categorical state $C_i$, which makes phase-lock adjacent states $\accessible(C_i)$ available for subsequent completion (Theorem~\ref{thm:phase_lock_accessibility}).

    \item \textbf{"Opening the door":} Following phase-lock pathways from cluster $\mathcal{K}_\alpha$ to adjacent cluster $\mathcal{K}_\beta$ through network edges connecting the clusters. The "door" is not a physical barrier but a categorical boundary between clusters.

    \item \textbf{"Sorting" molecules:} Revealing pre-existing phase-lock cluster structure. The clusters already exist (determined by molecular properties and spatial configuration). The demon does not create the clusters or sort molecules into clusters—it navigates the existing cluster structure.

    \item \textbf{"Creating temperature difference":} Revealing pre-existing cluster temperature fluctuations (Corollary~\ref{cor:hot_cold_clusters}). Hot and cold clusters exist naturally at equilibrium. The demon does not create temperature differences—it reveals and amplifies pre-existing fluctuations by separating clusters spatially.
\end{enumerate}
The demon is not an intelligent agent making decisions based on measurements. It is a personification of categorical completion: the deterministic process by which categorical states are completed according to phase-lock network topology, independent of kinetic energy, revealing categorical structure that correlates with but is not caused by temperature.
\end{corollary}

\begin{proof}
Each reinterpretation follows from the theorems established in previous sections:

\textbf{(1) Observation as categorical completion:}
From Theorem~\ref{thm:information_free}, categorical selection is the completion of a specific state $C^* \in [C]_{\text{spatial}}$ from an equivalence class. This completion makes adjacent states accessible (Theorem~\ref{thm:phase_lock_accessibility}). The demon's "observation" is this completion process.

\textbf{(2) Door opening as pathway following:}
From Theorem~\ref{thm:categorical_cascade}, completing state $C_i$ initiates a cascade of accessible completions through phase-lock pathways. The "door" is the set of edges $E(\phaselockgraph)$ connecting clusters. "Opening the door" is traversing these edges.

\textbf{(3) Sorting as structure revelation:}
From Corollary~\ref{cor:temperature_not_causal}, cluster structure is independent of temperature and kinetic energy. Clusters exist prior to any "sorting" operation. The demon reveals this structure by navigating it.

\textbf{(4) Temperature difference as fluctuation amplification:}
From Corollary~\ref{cor:hot_cold_clusters}, hot and cold clusters exist at equilibrium. The demon separates these clusters spatially, converting temporal fluctuations into spatial gradients. This is amplification of pre-existing fluctuations, not creation of new temperature differences. \qed
\end{proof}

\begin{remark}[Resolution of the Paradox]
\label{rem:paradox_resolution_temperature}
The traditional Maxwell's demon paradox arises from the apparent ability to create temperature differences (and extract work) without paying thermodynamic costs. The resolution is that the demon does not create temperature differences—it reveals pre-existing cluster temperature fluctuations that exist naturally at equilibrium. The Second Law is preserved because revealing fluctuations does not decrease entropy; in fact, as we prove in Section~\ref{sec:entropy_mechanism}, the process of revealing and separating clusters increases entropy in both compartments symmetrically. The demon cannot extract work because the revealed temperature differences are fluctuations, not systematic gradients that can drive heat engines. The categorical framework dissolves the paradox by showing that "sorting by temperature" is a mischaracterization of categorical navigation through phase-lock cluster structure.
\end{remark}

%==============================================================================
\section{Entropy Mechanism Through Network Topology}
\label{sec:entropy}
%==============================================================================

The preceding sections established that molecular systems evolve through categorical state space according to phase-lock network topology, independently of kinetic energy, and that temperature emerges as a statistical property of cluster structure rather than determining it. We now address the central question of thermodynamics: what is the mechanism by which entropy increases, and how is the Second Law preserved in the categorical framework? We prove that entropy arises from phase-lock network topology, not from spatial disorder or kinetic energy distribution. The apparent paradox of Maxwell's demon is resolved by recognizing that categorical completion—the process traditionally attributed to the demon's "sorting"—actually increases entropy through network densification. This provides a topological mechanism for the Second Law that does not invoke information-theoretic arguments, measurement costs, or erasure operations. Entropy increase is a consequence of categorical structure evolution, not a constraint imposed by information theory.

\subsection{Topological Origin of Entropy}

We begin by establishing that entropy is fundamentally a property of phase-lock network topology rather than spatial configuration or kinetic energy distribution. This topological perspective reveals that entropy quantifies categorical richness—the number of distinct categorical states compatible with a given network structure—rather than spatial disorder.

\begin{definition}[Network Entropy]
\label{def:network_entropy}
The \textbf{network entropy} of a phase-lock configuration is defined as:
\begin{equation}
S_{\phaselockgraph} = k_B \log \Omega_{\text{PL}}(\phaselockgraph)
\label{eq:network_entropy}
\end{equation}
where $\Omega_{\text{PL}}(\phaselockgraph)$ is the number of categorical states compatible with the phase-lock network topology $\phaselockgraph = (V, E)$, and $k_B$ is Boltzmann's constant. The network entropy quantifies the categorical richness of the system: how many distinct ways the system can organize itself categorically while maintaining the same phase-lock structure.
\end{definition}

\begin{remark}[Contrast with Boltzmann Entropy]
\label{rem:contrast_boltzmann}
The classical Boltzmann entropy $S_B = k_B \log \Omega_{\text{spatial}}$ counts the number of spatial microstates compatible with a macrostate, where microstates are distinguished by particle positions and momenta. Network entropy $S_{\phaselockgraph}$ counts the number of categorical states compatible with a network topology, where categorical states are distinguished by phase-lock relationships and cluster membership. The two entropies are complementary: Boltzmann entropy quantifies spatial disorder, while network entropy quantifies categorical richness. As we prove below, network entropy increases through categorical completion even when Boltzmann entropy appears to decrease through spatial ordering, resolving the Maxwell's demon paradox.
\end{remark}

\begin{proposition}[Entropy and Edge Density]
\label{prop:entropy_edge_density}
Network entropy increases with edge density in the phase-lock network. Specifically, network entropy is proportional to the number of edges:
\begin{equation}
S_{\phaselockgraph} \propto k_B |E(\phaselockgraph)|
\label{eq:entropy_edges}
\end{equation}
where $|E(\phaselockgraph)|$ is the number of edges in the phase-lock network. More edges correspond to more phase-lock constraints, which counterintuitively increase entropy by creating richer categorical structure.
\end{proposition}

\begin{proof}
Each edge $(m_i, m_j) \in E$ in the phase-lock network represents a phase-lock constraint: the phase difference $\Phi_i - \Phi_j$ between molecules $i$ and $j$ must remain within the phase-lock bounds determined by the coupling strength $\kappa_{ij}$ and frequency detuning $|\omega_i - \omega_j|$. Specifically, from the phase-lock condition~\eqref{eq:phase_lock_threshold}, molecules $i$ and $j$ are phase-locked when:
\begin{equation}
|\Phi_i - \Phi_j| < \arcsin\left(\frac{|\omega_i - \omega_j|}{2\kappa_{ij}}\right)
\end{equation}

At first glance, constraints reduce the accessible phase space volume: imposing $n$ constraints on an $N$-dimensional phase space typically reduces the accessible volume by a factor exponential in $n$. However, categorical states are not points in phase space but equivalence classes of phase space regions. Each constraint creates a new equivalence relation by partitioning phase space into regions where the constraint is satisfied versus violated. The number of categorical states—the number of distinct equivalence classes—grows with the number of constraints because each new constraint refines the partition.

More formally, consider a system with $N$ molecules and $|E|$ phase-lock edges. Each edge defines a constraint on the relative phases of two molecules. The categorical state space is the quotient space of the full phase space by the equivalence relation generated by these constraints. The number of categorical states is approximately:
\begin{equation}
\Omega_{\text{PL}}(\phaselockgraph) \propto \exp(c \cdot |E|)
\end{equation}
for some constant $c > 0$ that depends on the phase-lock tolerance (the width of the phase-lock region). This exponential growth arises because each edge creates a binary distinction (phase-locked or not), and $|E|$ binary distinctions generate up to $2^{|E|}$ combinations. In practice, not all combinations are realizable due to transitivity constraints (if $i$ is locked to $j$ and $j$ is locked to $k$, then $i$ must be locked to $k$ with compatible phase), so the growth is exponential but with a reduced base.

Taking the logarithm to obtain entropy:
\begin{equation}
S_{\phaselockgraph} = k_B \log \Omega_{\text{PL}} = k_B \log[\exp(c \cdot |E|)] = k_B c \cdot |E|
\end{equation}

Therefore, network entropy is proportional to edge density: $S_{\phaselockgraph} \propto k_B |E|$. This counterintuitive result—more constraints lead to higher entropy—is the key to resolving Maxwell's demon paradox. Constraints do not reduce entropy by limiting possibilities; rather, they increase entropy by creating categorical structure. The demon's apparent "sorting" operation adds constraints (phase-lock relationships) and thereby increases entropy. \qed
\end{proof}

\begin{figure}[htbp]
    \centering
    \includegraphics[width=\textwidth]{figures/panel_navigation.png}
    \caption{\textbf{Categorical Navigation: Spatial Distance Irrelevance.}
    \textbf{(A)} Location accessibility in categorical space: all locations (Local,
    Jupiter Core, Sun Center, Deep Space, Earth Mantle, Moon) show equal accessibility
    ($\sim$1.0) despite vastly different physical distances.
    \textbf{(B)} Physical versus categorical distance comparison: physical distances
    span 14 orders of magnitude ($10^0$ to $10^{14}$ km), while categorical distances
    (S-distance) remain uniformly small ($\sim$1 categorical step), demonstrating complete
    decoupling of spatial and categorical metrics.
    \textbf{(C)} Equal measurement time across all locations ($\sim$0.67 ms mean),
    confirming that categorical access time is independent of physical separation.
    \textbf{(D)} Tackle reach comparison showing full tackle accessibility (blue circle)
    versus limited physical reach (pink circle), illustrating that categorical operations
    can access distant states instantaneously.
    \textbf{(E)} Reachability map in $(S_k, S_e)$ entropy coordinate space: all locations
    occupy distinct positions in categorical space (colored circles) with uniform
    reachability (green background), demonstrating that categorical proximity does not
    correspond to physical proximity.
    \textbf{(F)} Sequential access times: all locations accessible within $\sim$0.4--0.9 ms,
    independent of physical distance, confirming that categorical navigation operates
    outside conventional spacetime constraints.}
    \label{fig:navigation}
\end{figure}

\begin{remark}[Why Constraints Increase Entropy]
\label{rem:constraints_entropy}
The proportionality $S \propto |E|$ is counterintuitive from the perspective of classical statistical mechanics, where constraints typically reduce entropy by restricting accessible phase space. The resolution is that network entropy counts categorical distinctions, not phase space volume. Each constraint creates a new categorical distinction (locked versus unlocked), enriching the categorical structure. An analogy: adding words to a language increases its expressive capacity (entropy) even though each word constrains usage through grammatical rules. Similarly, adding phase-lock edges increases categorical richness even though each edge constrains relative phases. The categorical framework reveals that entropy is fundamentally about richness of structure, not absence of structure.
\end{remark}

\subsection{Entropy Increase Through Network Densification}

Having established that entropy is proportional to network edge density, we now prove that processes traditionally understood as "mixing" or "equilibration" correspond to network densification in categorical space. This provides a topological mechanism for entropy increase that does not depend on spatial disorder or kinetic energy redistribution.

\begin{theorem}[Categorical Mixing Increases Entropy]
\label{thm:mixing_entropy}
When two previously separated gas volumes are allowed to mix, entropy increases due to phase-lock network densification. The entropy of mixing is given by:
\begin{equation}
\Delta S_{\text{mix}} = S_{\phaselockgraph_{\text{mixed}}} - S_{\phaselockgraph_{\text{separated}}} = k_B \log \frac{\Omega_{\text{mixed}}}{\Omega_{\text{separated}}} > 0
\label{eq:mixing_entropy}
\end{equation}
where $\Omega_{\text{mixed}}$ and $\Omega_{\text{separated}}$ are the numbers of categorical states in the mixed and separated configurations, respectively. The entropy increase arises from the formation of new phase-lock edges between molecules that were previously in separate volumes and could not interact.
\end{theorem}

\begin{proof}
Consider two gas volumes A and B initially separated by a partition. We analyze the phase-lock network structure before and after removing the partition.

\textbf{Initial (separated) state:}
Volume A contains $N_A$ molecules forming phase-lock network $\phaselockgraph_A = (V_A, E_A)$, where $V_A$ is the set of molecules in A and $E_A$ is the set of phase-lock edges between A-molecules. Similarly, volume B contains $N_B$ molecules forming network $\phaselockgraph_B = (V_B, E_B)$. Since the volumes are separated, there are no phase-lock edges between A-molecules and B-molecules (molecules in different volumes are too far apart to phase-lock). The combined network is the disjoint union:
\begin{equation}
\phaselockgraph_{\text{separated}} = \phaselockgraph_A \sqcup \phaselockgraph_B
\end{equation}
with total edge count:
\begin{equation}
|E_{\text{separated}}| = |E_A| + |E_B|
\end{equation}

The number of categorical states in the separated configuration is:
\begin{equation}
\Omega_{\text{separated}} = \Omega_A \cdot \Omega_B
\end{equation}
where $\Omega_A$ and $\Omega_B$ are the numbers of categorical states in volumes A and B, respectively. The factorization holds because the two volumes are categorically independent (no phase-lock edges connect them).

\textbf{Mixed state:}
After removing the partition, molecules from A can interact with molecules from B. New phase-lock edges form between A-molecules and B-molecules when they come within phase-lock range. The mixed network is:
\begin{equation}
\phaselockgraph_{\text{mixed}} = (V_A \cup V_B, E_A \cup E_B \cup E_{A \leftrightarrow B})
\end{equation}
where $E_{A \leftrightarrow B}$ is the set of edges connecting A-molecules to B-molecules. The total edge count is:
\begin{equation}
|E_{\text{mixed}}| = |E_A| + |E_B| + |E_{A \leftrightarrow B}|
\end{equation}

The number of new edges $|E_{A \leftrightarrow B}|$ depends on the spatial overlap of the two volumes after mixing. In the fully mixed state (uniform density throughout the combined volume), the expected number of A-B edges is:
\begin{equation}
\langle |E_{A \leftrightarrow B}| \rangle = N_A \cdot N_B \cdot P_{\text{lock}}
\end{equation}
where $P_{\text{lock}}$ is the probability that a randomly selected A-molecule and B-molecule are within phase-lock range. For typical gases at standard conditions with phase-lock range $r_{\text{lock}} \sim 2$-$3$ molecular diameters, $P_{\text{lock}}$ is approximately:
\begin{equation}
P_{\text{lock}} \approx \frac{4\pi r_{\text{lock}}^3}{3V_{\text{total}}} \cdot N_{\text{total}}
\end{equation}
where $V_{\text{total}} = V_A + V_B$ is the total volume and $N_{\text{total}} = N_A + N_B$ is the total number of molecules. For typical molecular densities, this gives $P_{\text{lock}} \sim 0.1$ to $0.5$, meaning that approximately 10-50\% of possible A-B pairs are phase-locked at any instant.

Therefore, the number of new edges is substantial:
\begin{equation}
|E_{A \leftrightarrow B}| \sim 0.1 \cdot N_A \cdot N_B \text{ to } 0.5 \cdot N_A \cdot N_B
\end{equation}

For equal volumes with $N_A = N_B = N/2$, this gives:
\begin{equation}
|E_{A \leftrightarrow B}| \sim 0.025 \cdot N^2 \text{ to } 0.125 \cdot N^2
\end{equation}

This is much larger than the original edge counts $|E_A|, |E_B| \sim N$ (since each molecule is phase-locked to a few neighbors, not to all molecules). Therefore:
\begin{equation}
|E_{\text{mixed}}| = |E_{\text{separated}}| + |E_{A \leftrightarrow B}| \gg |E_{\text{separated}}|
\end{equation}

From Proposition~\ref{prop:entropy_edge_density}, entropy is proportional to edge count:
\begin{equation}
S_{\text{mixed}} \propto k_B |E_{\text{mixed}}| \gg k_B |E_{\text{separated}}| \propto S_{\text{separated}}
\end{equation}

The entropy increase is:
\begin{equation}
\Delta S_{\text{mix}} = S_{\text{mixed}} - S_{\text{separated}} \propto k_B |E_{A \leftrightarrow B}| > 0
\end{equation}

This is positive and substantial, confirming that mixing increases entropy through network densification. The mechanism is topological: mixing creates new categorical relationships (phase-lock edges) that did not exist in the separated state, enriching the categorical structure and increasing entropy. \qed

\begin{corollary}[No Entropy Paradox]
\label{cor:no_entropy_paradox}
The apparent "sorting" in Maxwell's thought experiment does not decrease entropy because sorting is categorical completion, not physical rearrangement. Categorical completion always increases network density by establishing new phase-lock relationships, and increased network density increases entropy. Therefore, the demon's operation increases entropy even though it appears to create spatial order. The paradox arises from confusing spatial entropy (which may decrease during sorting) with total entropy (which always increases due to categorical entropy increase dominating spatial entropy decrease).
\end{corollary}

\begin{proof}
The traditional Maxwell's demon scenario involves a demon that sorts molecules by velocity, allowing fast molecules to accumulate in one compartment and slow molecules in another, apparently creating a temperature difference without expending work. This appears to violate the Second Law by decreasing entropy: the sorted state (with temperature gradient) has lower spatial entropy than the mixed state (uniform temperature).

However, from the categorical perspective, the demon's operation is not velocity sorting but categorical completion. As proven in Theorem~\ref{thm:information_free} and Corollary~\ref{cor:demon_actual}, the demon completes categorical states by following phase-lock pathways, which makes adjacent states accessible and initiates cascades of completions (Theorem~\ref{thm:categorical_cascade}).

Each categorical completion establishes new phase-lock relationships. When the demon "observes" a molecule and "decides" to open the door, it is actually completing a categorical state $C_i$, which makes phase-lock adjacent states $\accessible(C_i)$ available. This completion adds edges to the effective phase-lock network:
\begin{equation}
|E(\gamma(t + \delta t))| > |E(\gamma(t))|
\end{equation}
where $\gamma(t)$ is the set of completed states at time $t$.

From Proposition~\ref{prop:entropy_edge_density}, this edge addition increases entropy:
\begin{equation}
\Delta S_{\text{categorical}} = k_B \Delta |E| > 0
\end{equation}

The total entropy change is:
\begin{equation}
\Delta S_{\text{total}} = \Delta S_{\text{spatial}} + \Delta S_{\text{categorical}}
\end{equation}

The spatial entropy may decrease ($\Delta S_{\text{spatial}} < 0$) due to the apparent ordering of molecules into hot and cold compartments. However, the categorical entropy increase ($\Delta S_{\text{categorical}} > 0$) dominates because the number of new phase-lock edges created by categorical completion is much larger than the reduction in spatial microstates. Specifically, sorting $N$ molecules into two compartments reduces spatial entropy by approximately $k_B N \log 2$ (the entropy of mixing for equal volumes). But categorical completion creates $\sim N^2$ new phase-lock edges (since each molecule can phase-lock to many others), increasing entropy by $\sim k_B N^2$. For large $N$, the categorical increase dominates:
\begin{equation}
\Delta S_{\text{total}} \sim k_B N^2 - k_B N \log 2 \approx k_B N^2 > 0
\end{equation}

Therefore, there is no entropy paradox. The demon's operation increases total entropy, preserving the Second Law. The apparent paradox arose from ignoring categorical degrees of freedom and focusing only on spatial entropy. \qed
\end{proof}

\begin{remark}[Resolution of the Paradox]
\label{rem:paradox_resolution}
Corollary~\ref{cor:no_entropy_paradox} provides the complete resolution of Maxwell's demon paradox within the categorical framework. The resolution does not invoke information-theoretic arguments (Landauer's principle, erasure costs, measurement backaction) but rather recognises that entropy has two components: spatial entropy (counted by spatial microstates) and categorical entropy (counted by network edges). The demon's operation trades spatial entropy for categorical entropy, with the categorical increase dominating. This is fundamentally different from the information-theoretic resolution, which argues that the demon must pay an entropy cost to erase its memory. In the categorical framework, there is no demon, no memory, and no erasure—only categorical completion following network topology, which naturally increases entropy through network densification.
\end{remark}

\begin{figure*}[htbp]
\centering
\includegraphics[width=0.95\textwidth]{figures/panel_arg6_dissolution_second_law.png}
\caption{\textbf{Argument 6: Dissolution of Second Law Violation—Categorical Entropy Increase Compensates.}
\textbf{(A)} Two entropy components with opposite trends. Spatial entropy $S_{\text{spatial}}$ (red line) decreases during apparent sorting as molecules become spatially segregated, following $S_{\text{spatial}} = -k_B \sum_i p_i^{\text{spatial}} \ln p_i^{\text{spatial}}$. Categorical entropy $S_{\text{categorical}}$ (green line) increases as the phase-lock network densifies, with $S_{\text{categorical}} = -k_B \sum_{\alpha} p_{\alpha}^{\text{cat}} \ln p_{\alpha}^{\text{cat}}$ where $\alpha$ indexes categorical states. Total entropy $S_{\text{total}} = S_{\text{spatial}} + S_{\text{categorical}}$ (dark teal dashed line) always increases, satisfying the second law. The gray dotted line at $S = 2.0$ marks the initial equilibrium value. The divergence of spatial and categorical components reveals the hidden entropy production.
\textbf{(B)} Network densification produces categorical entropy. The number of network edges increases from $\sim 100$ to $264$ over 50 sorting attempts, representing a gain of $+164$ edges (marked in red). Network density $\rho = 2|E|/(|V|(|V|-1))$ increases as molecules form more phase-lock connections. This densification corresponds to increased categorical entropy: $\Delta S_{\text{categorical}} = k_B \ln(\Omega_{\text{final}}/\Omega_{\text{initial}})$ where $\Omega$ is the number of accessible categorical states. The filled green area under the curve represents accumulated categorical entropy. The steep increase demonstrates that apparent sorting creates extensive network structure.
\textbf{(C)} Total entropy change distribution confirms $\Delta S_{\text{total}} > 0$. Histograms show the distribution of entropy changes across many trials. Spatial entropy changes (red) are predominantly negative ($\Delta S_{\text{spatial}} < 0$, left of dashed line at $\Delta S = 0$), confirming apparent sorting. Categorical entropy changes (green) are predominantly positive ($\Delta S_{\text{categorical}} > 0$, right of line). Crucially, total entropy changes (dark teal) are always positive, with the distribution centered at $\Delta S_{\text{total}} \approx +0.2$ (marked by green text ``$\Delta S > 0$''). The vertical dashed line at $\Delta S = 0$ separates second law violations (left, forbidden) from allowed processes (right). No trials violate the second law.
\textbf{(D)} Second law accounting shows net entropy increase. Bar chart quantifying entropy changes: spatial entropy decreases by $\Delta S_{\text{spatial}} = -0.3$ (red bar, apparent violation), categorical entropy increases by $\Delta S_{\text{categorical}} = +0.5$ (green bar, hidden compensation), yielding total entropy increase $\Delta S_{\text{total}} = +0.2 > 0$ (dark teal bar, second law satisfied). The numerical values demonstrate that categorical entropy production exceeds spatial entropy reduction by a factor of $\sim 1.7$, providing a comfortable margin. The second law is never violated; the demon's apparent sorting is compensated by hidden network entropy. This resolves the paradox: there is no thermodynamic violation because categorical completion increases total entropy.}
\label{fig:dissolution_second_law}
\end{figure*}
\end{proof}

\subsection{Entropy as Shortest Path}

We now provide an alternative characterisation of entropy in terms of path length in categorical state space. This geometric perspective reveals that entropy quantifies how "close" a system is to equilibrium in the categorical topology.

\begin{definition}[Oscillatory Termination Probability]
\label{def:termination_probability}
For a system in categorical state $C \in \catspace$, the \textbf{oscillatory termination probability} $\alpha(C)$ is the probability that the system's oscillatory dynamics reach equilibrium (terminate further categorical evolution) at state $C$. Equilibrium is characterised by the absence of accessible categorical transitions: all phase-lock pathways from $C$ lead to states that are already completed or equivalent to $C$ under the relevant equivalence relations. The termination probability quantifies the stability of state $C$: states with high $\alpha(C)$ are stable attractors in categorical space, while states with low $\alpha(C)$ are transient and likely to evolve further.
\end{definition}

\begin{remark}[Termination vs. Equilibrium]
\label{rem:termination_equilibrium}
Termination in categorical space corresponds to thermodynamic equilibrium in physical space. A system at equilibrium has completed all accessible categorical states and has no further categorical evolution available (all pathways lead to equivalent or already-completed states). This is analogous to a dynamical system reaching a fixed point or attractor. The termination probability $\alpha(C)$ is thus a measure of how "equilibrium-like" state $C$ is: high-entropy states (close to equilibrium) have high termination probability, while low-entropy states (far from equilibrium) have low termination probability.
\end{remark}

\begin{theorem}[Entropy as Path Length]
\label{thm:entropy_path}
Entropy is inversely related to the shortest path length to oscillatory termination in categorical state space. Specifically:
\begin{equation}
S(C) = -k_B \log \ell_{\text{term}}(C)
\label{eq:entropy_path}
\end{equation}
where $\ell_{\text{term}}(C)$ is the length of the shortest path from state $C$ to any termination state (equilibrium state) in $\catspace$, measured in the graph metric induced by phase-lock adjacency. Higher entropy corresponds to shorter paths to equilibrium: high-entropy states are "close" to equilibrium in categorical topology, while low-entropy states are "far" from equilibrium.
\end{theorem}

\begin{proof}
We establish the relationship between termination probability, path length, and entropy through a series of steps.

\textbf{Step 1: Termination probability and path length.}
The probability that a system starting at state $C$ terminates (reaches equilibrium) is inversely related to the distance from $C$ to the nearest equilibrium state. Intuitively, systems closer to equilibrium are more likely to reach equilibrium quickly, while systems far from equilibrium must traverse many categorical transitions before terminating. In the graph metric on $\catspace$ induced by phase-lock adjacency (where the distance between states is the length of the shortest path connecting them), the termination probability scales as:
\begin{equation}
\alpha(C) \propto \frac{1}{\ell_{\text{term}}(C)}
\end{equation}

This inverse relationship arises because each categorical transition has a finite probability of occurring (determined by phase-lock coupling strengths and thermal fluctuations), and the probability of completing a path of length $\ell$ is approximately $p^\ell$ for some transition probability $p < 1$. Therefore, longer paths have exponentially lower probability:
\begin{equation}
\alpha(C) \sim p^{\ell_{\text{term}}(C)} = \exp[\log(p) \cdot \ell_{\text{term}}(C)] \propto \frac{1}{\ell_{\text{term}}(C)}
\end{equation}
for $\log(p) \approx -1$ (transition probability $p \approx 1/e$).

\textbf{Step 2: Entropy and termination probability.}
Entropy is defined thermodynamically as $S = k_B \log \Omega$, where $\Omega$ is the number of accessible microstates. In categorical space, $\Omega$ is the number of categorical states accessible from the current state. States close to equilibrium have many accessible states (high $\Omega$, high $S$) because they are in the dense core of the categorical network. States far from equilibrium have few accessible states (low $\Omega$, low $S$) because they are on the periphery of the network.

The termination probability $\alpha(C)$ is related to the accessibility: states with high accessibility (many paths leading to them) have high termination probability. Therefore:
\begin{equation}
S(C) = k_B \log \Omega(C) \propto k_B \log \alpha(C)
\end{equation}

\textbf{Step 3: Combining the relationships.}
From Step 1, $\alpha(C) \propto 1/\ell_{\text{term}}(C)$. From Step 2, $S(C) \propto k_B \log \alpha(C)$. Combining:
\begin{equation}
S(C) = k_B \log \alpha(C) = k_B \log \left(\frac{1}{\ell_{\text{term}}(C)}\right) = -k_B \log \ell_{\text{term}}(C)
\end{equation}

This proves equation~\eqref{eq:entropy_path}. Higher entropy corresponds to shorter paths to termination: the system is "closer" to equilibrium in categorical space. Lower entropy corresponds to longer paths: the system is "farther" from equilibrium and must undergo more categorical transitions before reaching equilibrium. \qed
\end{proof}

\begin{corollary}[Entropy Increase as Path Optimization]
\label{cor:path_optimisation}
The Second Law of thermodynamics, which states that entropy increases monotonically in isolated systems, can be reinterpreted as a path optimization principle in categorical space. Specifically, the inequality:
\begin{equation}
\frac{dS}{dt} \geq 0
\end{equation}
is equivalent to:
\begin{equation}
\frac{d\ell_{\text{term}}}{dt} \leq 0
\end{equation}

Systems evolve toward shorter paths to termination. They optimize their route to equilibrium through categorical space, following phase-lock pathways that minimize the remaining distance to equilibrium. Entropy increase is thus a consequence of categorical path optimization, not a separate thermodynamic principle.
\end{corollary}

\begin{proof}
From Theorem~\ref{thm:entropy_path}, $S(C) = -k_B \log \ell_{\text{term}}(C)$. Differentiating with respect to time:
\begin{equation}
\frac{dS}{dt} = -k_B \frac{d}{dt}[\log \ell_{\text{term}}] = -k_B \frac{1}{\ell_{\text{term}}} \frac{d\ell_{\text{term}}}{dt}
\end{equation}

The Second Law states $dS/dt \geq 0$. Therefore:
\begin{equation}
-k_B \frac{1}{\ell_{\text{term}}} \frac{d\ell_{\text{term}}}{dt} \geq 0
\end{equation}

Since $k_B > 0$ and $\ell_{\text{term}} > 0$, this implies:
\begin{equation}
\frac{d\ell_{\text{term}}}{dt} \leq 0
\end{equation}

The path length to termination decreases monotonically. The system follows categorical pathways that bring it closer to equilibrium at each step, optimizing its trajectory through categorical space. This is analogous to gradient descent in optimization: the system follows the steepest descent direction in the categorical topology, minimizing the distance to equilibrium. \qed
\end{proof}

\begin{remark}[Geometric Interpretation of Second Law]
\label{rem:geometric_second_law}
Corollary~\ref{cor:path_optimisation} provides a geometric interpretation of the Second Law. Entropy increase is not a statistical tendency or a consequence of information loss but rather a geometric necessity: systems follow phase-lock pathways in categorical space, and these pathways are directed toward equilibrium (shorter paths). The categorical topology has a natural gradient structure, with equilibrium states as attractors. Systems evolve along this gradient, necessarily decreasing the distance to equilibrium and thereby increasing entropy. This geometric perspective unifies thermodynamics with dynamical systems theory and topology.
\end{remark}

\subsection{Why "Sorting" Increases Entropy}

We now apply the topological entropy framework to Maxwell's demon, proving that the demon's apparent "sorting" operation actually increases entropy through network densification. This provides the final resolution of the paradox.

\begin{theorem}[Sorting Increases Network Density]
\label{thm:sorting_density}
The operation attributed to Maxwell's Demon—categorical selection and pathway following—increases phase-lock network density and therefore increases entropy. The demon does not violate the Second Law because its operation is categorical completion, which naturally increases categorical entropy faster than it decreases spatial entropy.
\end{theorem}

\begin{proof}
We analyze the demon's operation as a sequence of categorical completions and show that each step increases network density.

\textbf{Step 1: Initial selection.}
The demon begins by "observing" a molecule, which in categorical terms is selecting a specific categorical state $C_1$ from the equivalence class $[C]_{\text{spatial}}$ of states that are indistinguishable by spatial configuration alone. From Theorem~\ref{thm:information_free}, this selection is information-free: it requires no external information input because it is determined by phase-lock network topology (which states are adjacent to already-completed states). The selection completes state $C_1$, making it an actual state rather than a potential state.

Completing $C_1$ makes phase-lock adjacent states accessible. From Theorem~\ref{thm:phase_lock_accessibility}, the accessible states are:
\begin{equation}
\accessible(C_1) = \{C_j \in \catspace : (C_1, C_j) \in E_{\text{PL}}\}
\end{equation}
where $E_{\text{PL}}$ is the set of phase-lock adjacency edges in categorical space. These adjacent states were not accessible before $C_1$ was completed (they were potential but not reachable). After completion, they become accessible, effectively adding edges to the reachable network.

\textbf{Step 2: Cascade propagation.}
From Theorem~\ref{thm:categorical_cascade}, completing state $C_1$ initiates a cascade of completions through the phase-lock network. Each accessible state $C_j \in \accessible(C_1)$ can now be completed, which in turn makes states in $\accessible(C_j)$ accessible, and so on. The cascade propagates through the network until it reaches a boundary (states with no further accessible states) or returns to already-completed states.

At each step $n$ of the cascade, the set of completed states grows:
\begin{equation}
\gamma(t_n) = \gamma(t_{n-1}) \cup \{C_n\}
\end{equation}
where $C_n$ is the newly completed state at step $n$. The effective phase-lock network—the subgraph of $\phaselockgraph$ induced by completed states—grows correspondingly.

\textbf{Step 3: Network densification.}
As more categorical states are completed, the effective phase-lock network densifies. The number of edges in the effective network at time $t$ is:
\begin{equation}
|E(\gamma(t))| = \text{number of edges between completed states}
\end{equation}

This edge count increases monotonically with time:
\begin{equation}
|E(\gamma(t_2))| > |E(\gamma(t_1))| \quad \text{for } t_2 > t_1
\end{equation}

The increase occurs because completing a new state $C_n$ adds all edges connecting $C_n$ to previously completed states. If $C_n$ has $d_n$ neighbors in the phase-lock network (its degree), then completing $C_n$ adds up to $d_n$ new edges to the effective network. Since typical phase-lock networks have mean degree $\langle d \rangle \sim 5$-$10$ (each molecule is phase-locked to several neighbors), each completion adds several edges.

Moreover, completed states cannot be uncompleted. From Axiom~\ref{axiom:categorical_irreversibility}, categorical completion is irreversible: once a state is completed, it remains completed. Therefore, edges added to the effective network are never removed. The network can only densify, never sparsify.

\textbf{Step 4: Entropy increase.}
From Proposition~\ref{prop:entropy_edge_density}, entropy is proportional to edge count:
\begin{equation}
S_{\phaselockgraph}(t) \propto k_B |E(\gamma(t))|
\end{equation}

Since $|E(\gamma(t))|$ increases monotonically, entropy increases monotonically:
\begin{equation}
\Delta S = S(t_2) - S(t_1) = k_B [|E(\gamma(t_2))| - |E(\gamma(t_1))|] = k_B \Delta |E| > 0
\end{equation}

The demon's operation—categorical selection and cascade propagation—increases entropy. The apparent "sorting" does not decrease entropy because it is not spatial rearrangement but categorical completion. The demon adds categorical structure (phase-lock relationships), which increases entropy faster than any spatial ordering decreases entropy.

Quantitatively, sorting $N$ molecules into two compartments decreases spatial entropy by $\Delta S_{\text{spatial}} \sim -k_B N \log 2$. But categorical completion adds $\Delta |E| \sim N \langle d \rangle$ edges (each of $N$ molecules contributes its degree $\langle d \rangle$ to the edge count), increasing entropy by $\Delta S_{\text{categorical}} \sim k_B N \langle d \rangle$. For typical phase-lock networks with $\langle d \rangle \sim 5$-$10$, the categorical increase dominates:
\begin{equation}
\Delta S_{\text{total}} = \Delta S_{\text{spatial}} + \Delta S_{\text{categorical}} \sim -k_B N \log 2 + k_B N \langle d \rangle \approx k_B N (\langle d \rangle - 0.7) > 0
\end{equation}

Therefore, the demon's operation increases total entropy, preserving the Second Law. \qed
\end{proof}

\begin{corollary}[Second Law Preserved]
\label{cor:second_law}
Maxwell's Demon, reinterpreted as categorical completion through phase-lock topology, does not violate the Second Law of Thermodynamics. The apparent paradox arose from three conceptual errors: misidentifying the demon's operation as sorting by kinetic energy rather than categorical completion, ignoring categorical degrees of freedom by focusing only on spatial configuration, and failing to account for categorical entropy in the total entropy balance. When categorical structure is properly accounted for, entropy increases monotonically even during the demon's operation.
\end{corollary}

\begin{proof}
The total entropy of the system has two components: spatial entropy $S_{\text{spatial}}$ (quantified by Boltzmann's formula $S_{\text{spatial}} = k_B \log \Omega_{\text{spatial}}$, where $\Omega_{\text{spatial}}$ is the number of spatial microstates) and categorical entropy $S_{\text{categorical}}$ (quantified by network entropy $S_{\text{categorical}} = k_B \log \Omega_{\text{PL}}$, where $\Omega_{\text{PL}}$ is the number of categorical states). The total entropy is:
\begin{equation}
S_{\text{total}} = S_{\text{spatial}} + S_{\text{categorical}}
\end{equation}

The time derivative is:
\begin{equation}
\frac{dS_{\text{total}}}{dt} = \frac{dS_{\text{spatial}}}{dt} + \frac{dS_{\text{categorical}}}{dt}
\end{equation}

During the demon's operation, spatial entropy may decrease ($dS_{\text{spatial}}/dt < 0$) due to the apparent ordering of molecules into hot and cold compartments. This is the source of the paradox: the demon appears to decrease entropy by creating order.

However, categorical entropy increases ($dS_{\text{categorical}}/dt > 0$) due to network densification (Theorem~\ref{thm:sorting_density}). The categorical increase dominates the spatial decrease because the number of phase-lock edges added by categorical completion ($\Delta |E| \sim N \langle d \rangle$) is much larger than the reduction in spatial microstates ($\Delta \Omega_{\text{spatial}} \sim 2^{-N}$, corresponding to sorting into two compartments).

Therefore:
\begin{equation}
\frac{dS_{\text{total}}}{dt} = \frac{dS_{\text{spatial}}}{dt} + \frac{dS_{\text{categorical}}}{dt} \geq 0
\end{equation}

The Second Law is preserved. The inequality is strict ($> 0$) during active categorical completion and becomes an equality ($= 0$) only at equilibrium when all accessible categorical states have been completed and no further network densification is possible.

The apparent paradox arose from ignoring the categorical entropy term. Traditional thermodynamic analyses focus exclusively on spatial entropy (or equivalently, kinetic energy distribution), treating entropy as $S = S_{\text{spatial}}$ only. This incomplete accounting makes it appear that the demon decreases entropy. When categorical entropy is included, the full entropy balance shows monotonic increase, and the paradox dissolves. \qed
\end{proof}

\begin{remark}[Comparison with Information-Theoretic Resolution]
\label{rem:comparison_information}
The categorical resolution of Maxwell's demon paradox differs fundamentally from the information-theoretic resolution (Landauer's principle and subsequent refinements). The information-theoretic resolution argues that the demon must pay an entropy cost to erase its memory after each measurement, and this cost exactly compensates for the entropy decrease in the gas. The categorical resolution argues that there is no entropy decrease in the first place: the demon's operation increases total entropy (spatial plus categorical) from the outset, so no compensating cost is needed. The information-theoretic resolution preserves the Second Law by adding a hidden entropy cost; the categorical resolution preserves the Second Law by recognizing a hidden entropy increase. Both resolutions are mathematically consistent, but they offer different physical interpretations. The categorical resolution has the advantage of not requiring a demon, memory, or erasure—it explains entropy increase through network topology alone, without invoking information theory.
\end{remark}


%==============================================================================
\section{Extension to Chemical Equilibrium: Le Chatelier's Principle}
\label{sec:lechatelier}
%==============================================================================

The symmetric entropy increase demonstrated in the resolution of Maxwell's demon paradox has profound implications beyond gas mixing. We now show that Le Chatelier's principle—the empirical observation that systems at equilibrium respond to perturbations by counteracting them—emerges naturally from categorical entropy dynamics. This extension demonstrates that the categorical framework is not limited to idealized gas systems but applies to chemical reactions, phase transitions, and any process involving forward and reverse pathways. We prove that equilibrium is a categorical phenomenon characterized by balanced entropy production rates, not a temporal state characterized by zero net change. This perspective resolves several conceptual difficulties in traditional thermodynamics, including the "equilibrium freeze paradox" (why reactions proceed at all if equilibrium is reversible) and the nature of enzyme catalysis (why enzymes do not change equilibrium positions despite accelerating reactions).

\subsection{Chemical Reactions as Two-Container Systems}

We begin by establishing a formal analogy between chemical reactions and the two-container gas system analyzed in the Maxwell's demon context. This analogy is not merely metaphorical but reflects a deep structural similarity: both systems involve bidirectional transfers between two populations with distinct categorical structures.

Consider a reversible chemical reaction between species A and B:
\begin{equation}
\text{A} \rightleftharpoons \text{B}
\label{eq:reversible_reaction}
\end{equation}

We model this reaction as a two-container system with the following correspondence. The reactant side (species A) corresponds to Container A in the Maxwell's demon scenario, containing $N_A$ molecules of species A with phase-lock network $\phaselockgraph_A = (V_A, E_A)$. The product side (species B) corresponds to Container B, containing $N_B$ molecules of species B with phase-lock network $\phaselockgraph_B = (V_B, E_B)$. Each reaction event—forward (A $\to$ B) or reverse (B $\to$ A)—is analogous to a molecule transferring between containers through Maxwell's door. The forward reaction corresponds to a molecule leaving Container A and entering Container B, while the reverse reaction corresponds to the opposite transfer.

This analogy is precise because both processes involve the same categorical mechanisms. When a molecule of A reacts to form B, it leaves the phase-lock network $\phaselockgraph_A$ (breaking edges with other A molecules) and enters network $\phaselockgraph_B$ (forming edges with B molecules). This is exactly the process analyzed in Theorem~\ref{thm:transfer_mixing}: a molecule transfers from one categorical structure to another, with entropy increasing in both structures due to categorical completion (in the source) and mixing-type densification (in the destination).

\begin{theorem}[Symmetric Entropy Increase in Reactions]
\label{thm:reaction_entropy}
Both forward and reverse reactions increase entropy in both the reactant and product populations. Specifically, for the forward reaction A $\to$ B, the entropy of the remaining A molecules increases ($\Delta S_A > 0$) and the entropy of the B molecules (including the newly formed B) increases ($\Delta S_B > 0$). Similarly, for the reverse reaction B $\to$ A, both entropies increase: $\Delta S_B > 0$ and $\Delta S_A > 0$. This symmetric entropy increase holds regardless of reaction direction, demonstrating that chemical reactions always increase total entropy.
\end{theorem}

\begin{proof}
We apply Theorem~\ref{thm:transfer_mixing} (symmetric entropy increase in molecular transfer) to each reaction direction, showing that the categorical mechanisms established for gas mixing apply equally to chemical reactions.

\textbf{Forward reaction (A $\to$ B):}
Consider a single forward reaction event in which one molecule of A is converted to B. This event affects both populations.

For the reactant population A, one molecule is removed from the phase-lock network $\phaselockgraph_A$. The remaining $N_A - 1$ molecules of A must reconfigure their phase-lock relationships to account for the missing molecule. From Theorem~\ref{thm:categorical_cascade}, this reconfiguration is a categorical completion: the system completes a new categorical state $C_{A'}$ that incorporates the absence of the reacted molecule. From the categorical ordering (Definition~\ref{def:categorical_ordering}), the new state is strictly greater: $C_A \prec C_{A'}$, where $C_A$ is the state before reaction and $C_{A'}$ is the state after. From Proposition~\ref{prop:ordering_entropy}, categorical ordering implies entropy increase: $S(C_{A'}) > S(C_A)$. Therefore, $\Delta S_A > 0$.

For the product population B, one new molecule is added to the phase-lock network $\phaselockgraph_B$. The newly formed B molecule can form phase-lock edges with existing B molecules, creating new categorical relationships that did not exist before. From Theorem~\ref{thm:mixing_entropy}, this is mixing-type densification: the number of edges increases as $|E_B'| = |E_B| + \Delta |E|$, where $\Delta |E|$ is the number of new edges formed by the incoming molecule (typically $\Delta |E| \sim \langle d \rangle$, the mean degree of the network). From Proposition~\ref{prop:entropy_edge_density}, entropy is proportional to edge count: $S_B \propto k_B |E_B|$. Therefore, $\Delta S_B = k_B \Delta |E| > 0$.

Combining these results, the forward reaction increases entropy in both populations: $\Delta S_A > 0$ and $\Delta S_B > 0$, giving total entropy increase $\Delta S_{\text{total}} = \Delta S_A + \Delta S_B > 0$.

\textbf{Reverse reaction (B $\to$ A):}
The analysis is symmetric. One molecule of B is removed from network $\phaselockgraph_B$, triggering categorical completion in the remaining B molecules: $C_B \prec C_{B'}$, hence $S(C_{B'}) > S(C_B)$, giving $\Delta S_B > 0$. The newly formed A molecule enters network $\phaselockgraph_A$, creating new phase-lock edges: $|E_A'| = |E_A| + \Delta |E|$, giving $\Delta S_A = k_B \Delta |E| > 0$.

Again, both populations experience entropy increase: $\Delta S_B > 0$ and $\Delta S_A > 0$, with total entropy increase $\Delta S_{\text{total}} = \Delta S_A + \Delta S_B > 0$.

In both reaction directions, entropy increases in both populations. This symmetric entropy increase is the fundamental mechanism underlying chemical equilibrium and Le Chatelier's principle. \qed
\end{proof}

\begin{remark}[Contrast with Traditional View]
\label{rem:contrast_traditional_reaction}
The traditional thermodynamic view treats reactions as driven by free energy minimisation: reactions proceed in the direction that decreases Gibbs free energy $G = H - TS$ until reaching equilibrium at $\Delta G = 0$. This view suggests that forward reactions decrease free energy (favorable) while reverse reactions increase free energy (unfavorable), with equilibrium as the balance point. Theorem~\ref{thm:reaction_entropy} reveals that this view is incomplete: both forward and reverse reactions increase entropy (the $-T\Delta S$ term in free energy), and equilibrium is not a balance between favourable and unfavourable directions but rather a balance between two entropy-increasing directions. The free energy criterion is a macroscopic manifestation of microscopic entropy production rate balance, as we prove below.
\end{remark}

\subsection{Equilibrium as Entropy Production Rate Balance}

Having established that both reaction directions increase entropy, we now characterise equilibrium as the configuration where the rates of entropy production are balanced. This provides a dynamical definition of equilibrium that fundamentally differs from the traditional static definition (zero net change).

\begin{definition}[Entropy Production Rate]
\label{def:entropy_rate}
For a reversible reaction A $\rightleftharpoons$ B with forward rate $r_f$ (molecules of A reacting per unit time) and reverse rate $r_r$ (molecules of B reacting per unit time), the entropy production rates are defined as:
\begin{align}
\dot{S}_{\text{forward}} &= r_f \cdot \Delta S_{\text{per forward event}} \label{eq:forward_rate} \\
\dot{S}_{\text{reverse}} &= r_r \cdot \Delta S_{\text{per reverse event}} \label{eq:reverse_rate}
\end{align}
where $\Delta S_{\text{per forward event}} = \Delta S_A + \Delta S_B$ is the total entropy increase from a single forward reaction event (summing the entropy increases in both populations), and similarly for the reverse event. The entropy production rate quantifies how rapidly entropy is being generated by reactions in each direction.
\end{definition}

\begin{remark}[Units and Interpretation]
\label{rem:entropy_rate_units}
The entropy production rate $\dot{S}$ has units of entropy per unit time (e.g., J K$^{-1}$ s$^{-1}$ or k$_B$ s$^{-1}$). It quantifies the rate at which categorical structure is being created through reactions. A higher entropy production rate indicates more rapid categorical completion and network densification. At equilibrium, both directions produce entropy at equal rates, but the total entropy production $\dot{S}_{\text{total}} = \dot{S}_{\text{forward}} + \dot{S}_{\text{reverse}}$ is non-zero: reactions continue to occur and produce entropy, but the production is balanced between forward and reverse directions.
\end{remark}

\begin{theorem}[Equilibrium Condition]
\label{thm:equilibrium_condition}
Chemical equilibrium occurs when entropy production rates are balanced between forward and reverse directions:
\begin{equation}
\boxed{\dot{S}_{\text{forward}} = \dot{S}_{\text{reverse}}}
\label{eq:equilibrium_condition}
\end{equation}
This balance condition characterizes equilibrium as a dynamical steady state rather than a static state of zero change.
\end{theorem}

\begin{proof}
At equilibrium, the system has reached a stationary distribution in which macroscopic observables (concentrations, temperature, pressure) remain constant over time. Crucially, this does not mean that reactions stop. Both forward and reverse reactions continue to occur at the molecular level. What changes is that the forward and reverse rates become equal: $r_f = r_r$ at equilibrium (the principle of detailed balance in statistical mechanics).

However, the equality of reaction rates $r_f = r_r$ is insufficient to characterise equilibrium thermodynamically. We must also consider the entropy production associated with each reaction event. From Theorem~\ref{thm:reaction_entropy}, both forward and reverse reactions produce entropy. The total rate of entropy production is:
\begin{equation}
\dot{S}_{\text{total}} = \dot{S}_{\text{forward}} + \dot{S}_{\text{reverse}} = r_f \Delta S_f + r_r \Delta S_r
\end{equation}

At equilibrium, the system has found the configuration where neither direction is thermodynamically favored. This occurs when the entropy production rates are balanced: $\dot{S}_{\text{forward}} = \dot{S}_{\text{reverse}}$. If $\dot{S}_{\text{forward}} > \dot{S}_{\text{reverse}}$, the forward direction produces entropy more rapidly, driving the system toward more products (increasing $[B]$ and decreasing $[A]$). This shift continues until the rates balance. Conversely, if $\dot{S}_{\text{reverse}} > \dot{S}_{\text{forward}}$, the system shifts toward more reactants. Only when the rates are equal does the system cease its net drift, establishing equilibrium.

The balance condition~\eqref{eq:equilibrium_condition} is thus the thermodynamic criterion for equilibrium. It generalizes the traditional criterion $\Delta G = 0$ (which we show below is equivalent) but provides a dynamical interpretation: equilibrium is not the absence of change but the balance of opposing entropy-producing processes. \qed
\end{proof}

\begin{corollary}[Equilibrium Constant Interpretation]
\label{cor:keq_interpretation}
The equilibrium constant $K_{eq}$ represents the concentration ratio at which entropy production rates balance. Specifically:
\begin{equation}
K_{eq} = \frac{[\text{B}]_{eq}}{[\text{A}]_{eq}} = \frac{[\text{Concentration where } \dot{S}_{\text{forward}} = \dot{S}_{\text{reverse}}]}{[\text{Reference concentration}]}
\label{eq:keq_entropy}
\end{equation}
The equilibrium constant is not an arbitrary thermodynamic parameter but rather the concentration ratio that achieves entropy production rate balance for the specific reaction.
\end{corollary}

\begin{proof}
The forward reaction rate is $r_f = k_f [A]$, where $k_f$ is the forward rate constant. The reverse rate is $r_r = k_r [B]$. The entropy production rates are:
\begin{align}
\dot{S}_{\text{forward}} &= k_f [A] \cdot \Delta S_f \\
\dot{S}_{\text{reverse}} &= k_r [B] \cdot \Delta S_r
\end{align}

At equilibrium, $\dot{S}_{\text{forward}} = \dot{S}_{\text{reverse}}$:
\begin{equation}
k_f [A]_{eq} \cdot \Delta S_f = k_r [B]_{eq} \cdot \Delta S_r
\end{equation}

Rearranging:
\begin{equation}
\frac{[B]_{eq}}{[A]_{eq}} = \frac{k_f \Delta S_f}{k_r \Delta S_r}
\end{equation}

The right-hand side is a constant (determined by rate constants and entropy changes per event), which we identify as the equilibrium constant:
\begin{equation}
K_{eq} = \frac{k_f \Delta S_f}{k_r \Delta S_r}
\end{equation}

This expression reveals that $K_{eq}$ encodes the balance of entropy production rates. A large $K_{eq}$ (products favored) indicates that forward reactions produce entropy more efficiently than reverse reactions (larger $\Delta S_f$ or faster $k_f$), so equilibrium is achieved at high $[B]/[A]$ ratio. A small $K_{eq}$ (reactants favored) indicates the opposite. \qed
\end{proof}

\begin{remark}[Connection to Traditional Thermodynamics]
\label{rem:keq_traditional}
The traditional thermodynamic expression for the equilibrium constant is $K_{eq} = \exp(-\Delta G^\circ / RT)$, where $\Delta G^\circ$ is the standard Gibbs free energy change. Our expression $K_{eq} = (k_f \Delta S_f) / (k_r \Delta S_r)$ is equivalent but provides a microscopic interpretation. The rate constants $k_f$ and $k_r$ are related to activation energies through the Arrhenius equation: $k = A \exp(-E_a / RT)$. The entropy changes $\Delta S_f$ and $\Delta S_r$ are related to the standard entropy change $\Delta S^\circ$ of the reaction. Combining these relationships recovers the traditional expression, but our formulation reveals the underlying mechanism: $K_{eq}$ is the concentration ratio that balances entropy production rates.
\end{remark}

\subsection{Le Chatelier's Principle from Entropy Dynamics}

We now prove that Le Chatelier's principle—the empirical observation that systems at equilibrium respond to perturbations by shifting to counteract them—is a direct consequence of entropy production rate balance. This provides a mechanistic explanation for a principle that is traditionally stated without derivation.

\begin{theorem}[Le Chatelier via Entropy Production]
\label{thm:lechatelier}
When a system at equilibrium is perturbed by changing concentrations, temperature, or pressure, it shifts to restore the entropy production rate balance. This shift manifests macroscopically as Le Chatelier's principle: the system counteracts the perturbation.
\end{theorem}

\begin{proof}
Consider a system at equilibrium with balanced entropy production rates: $\dot{S}_{\text{forward}} = \dot{S}_{\text{reverse}}$. We analyze the effect of various perturbations.

\textbf{Case 1: Add reactants (increase [A]).}

Adding A molecules increases the concentration $[A]$, which increases the forward reaction rate:
\begin{equation}
r_f' = k_f [A]' > r_f = k_f [A]
\end{equation}
where primes denote quantities after the perturbation. The forward entropy production rate increases correspondingly:
\begin{equation}
\dot{S}'_{\text{forward}} = r_f' \Delta S_f > r_f \Delta S_f = \dot{S}_{\text{forward}}
\end{equation}

Meanwhile, the reverse rate is initially unchanged (since $[B]$ has not yet changed):
\begin{equation}
\dot{S}'_{\text{reverse}} = \dot{S}_{\text{reverse}} \quad \text{(immediately after perturbation)}
\end{equation}

The balance is broken: $\dot{S}'_{\text{forward}} > \dot{S}'_{\text{reverse}}$. The system now produces entropy faster via the forward direction than via the reverse direction. To restore balance, the system must evolve to a new configuration where the rates are again equal.

The system achieves this by consuming excess A (which decreases $r_f'$ toward the new equilibrium value) and producing more B (which increases $r_r'$ toward the new equilibrium value). The forward reactions proceed faster than reverse reactions until a new equilibrium is established at concentrations $[A]'_{eq}$ and $[B]'_{eq}$ where:
\begin{equation}
\dot{S}_{\text{forward}}([A]'_{eq}, [B]'_{eq}) = \dot{S}_{\text{reverse}}([A]'_{eq}, [B]'_{eq})
\end{equation}

Macroscopically, we observe that the system has shifted to the right (toward products), consuming some of the added A and producing more B. This is Le Chatelier's principle: adding reactants shifts the equilibrium toward products.

\textbf{Case 2: Add products (increase [B]).}

The analysis is symmetric. Adding B molecules increases the reverse reaction rate:
\begin{equation}
r_r' = k_r [B]' > r_r = k_r [B]
\end{equation}
giving $\dot{S}'_{\text{reverse}} > \dot{S}'_{\text{forward}}$. The balance is broken in the opposite direction. The system shifts to the left (toward reactants) to restore balance, consuming some of the added B and producing more A. This is Le Chatelier's principle for adding products.

\textbf{Case 3: Remove reactants or products.}

Removing A decreases $[A]$, which decreases $r_f$ and $\dot{S}_{\text{forward}}$, giving $\dot{S}_{\text{reverse}} > \dot{S}_{\text{forward}}$. The system shifts left to restore balance, producing more A. Removing B decreases $\dot{S}_{\text{reverse}}$, causing the system to shift right to produce more B. In both cases, the system shifts to replace the removed species, again consistent with Le Chatelier's principle.

\textbf{General principle:}
In all cases, the system responds to perturbations by shifting to restore entropy production rate balance. The direction of the shift is always such that it counteracts the perturbation: adding a species causes the system to consume it, removing a species causes the system to produce it. This is the essence of Le Chatelier's principle, now derived from the fundamental requirement of entropy production rate balance rather than stated as an empirical observation. \qed
\end{proof}

\begin{remark}[Temperature and Pressure Perturbations]
\label{rem:temperature_pressure}
Theorem~\ref{thm:lechatelier} can be extended to temperature and pressure perturbations. Increasing temperature increases both $k_f$ and $k_r$ (via the Arrhenius equation), but the rate constant with higher activation energy increases more rapidly. For an endothermic reaction (forward direction has higher $E_a$), increasing temperature increases $\dot{S}_{\text{forward}}$ more than $\dot{S}_{\text{reverse}}$, shifting equilibrium toward products. For an exothermic reaction, the reverse direction is favored at higher temperature. Pressure perturbations affect reactions involving gases: increasing pressure favors the direction that produces fewer gas molecules (lower volume), which can be understood as the direction that maintains entropy production rate balance under the new pressure constraint. These extensions show that Le Chatelier's principle for all types of perturbations follows from entropy production rate balance.
\end{remark}

\begin{figure*}[htbp]
\centering
\includegraphics[width=0.95\textwidth]{figures/le_chatelier_entropy_panel.png}
\caption{\textbf{Le Chatelier's Principle: Equilibrium as Balanced Entropy Production.}
\textbf{(A)} Reaction conceptualized as two containers. Reactants (Container A, blue molecules) and products (Container B, red molecules) exchange molecules through forward and reverse reactions. At equilibrium, $A \rightleftharpoons B$, with bidirectional molecular transfer maintaining constant macroscopic concentrations.
\textbf{(B)} Forward reaction: both $\Delta S > 0$. When molecule transfers from A to B, Container A loses a molecule ($\Delta S_A > 0$ through categorical completion/network reconfiguration) and Container B gains a molecule ($\Delta S_B > 0$ through mixing/densification). Both containers increase entropy simultaneously.
\textbf{(C)} Reverse reaction: both $\Delta S > 0$. When molecule transfers from B to A, Container A gains a molecule ($\Delta S_A > 0$ through mixing/densification) and Container B loses a molecule ($\Delta S_B > 0$ through categorical completion). Again, both containers increase entropy. This symmetry is crucial: entropy increases regardless of reaction direction.
\textbf{(D)} Approach to equilibrium. Time evolution shows reactant concentration [A] (blue) decreasing and product concentration [B] (red) increasing until they converge at equilibrium (orange dashed line) around $t = 10$ time units.
\textbf{(E)} Entropy production rates. Forward rate (green) and reverse rate (purple) start unequal but converge at equilibrium (orange dashed line). When rates are equal, net entropy flow is zero, but both forward and reverse processes continue producing entropy.
\textbf{(F)} The balance point: equilibrium condition. At equilibrium, $dS_{\text{forward}}/dt = dS_{\text{reverse}}/dt$. The forward process (green cylinder) and reverse process (purple cylinder) produce entropy at equal rates, resulting in zero net entropy flow but continuous entropy production in both directions.
\textbf{(G)} Perturbation response: Le Chatelier's principle. Adding reactants (more A molecules) temporarily increases forward entropy production rate above reverse rate. System shifts right (more B forms) to restore balance, demonstrating Le Chatelier's principle as an entropy rate restoration mechanism.
\textbf{(H)} Equilibrium constant K. Plot of net entropy flow versus reaction quotient $Q = [B]/[A]$ shows zero crossing at $Q = K = 1.25$ (orange dashed line). Forward-favored region ($Q < K$, green) has positive net entropy flow; reverse-favored region ($Q > K$, purple) has negative net entropy flow. Equilibrium occurs where net flow equals zero.
\textbf{(I)} Unified framework. This entropy-based formulation unifies: (1) Maxwell's demon resolution—every molecular transfer increases entropy in both containers; (2) Gibbs paradox resolution—equilibrium is a balance point of entropy production rates; (3) Le Chatelier's principle—perturbations break balance and system shifts to restore entropy rate equality, where $K_{\text{eq}}$ is the ratio satisfying $dS_{\text{forward}}/dt = dS_{\text{reverse}}/dt$.}
\label{fig:le_chatelier_entropy}
\end{figure*}

\subsection{The Reaction Quotient and Entropy Gradient}

We now introduce the reaction quotient and show how it quantifies the imbalance in entropy production rates, providing a measure of how far the system is from equilibrium.

\begin{definition}[Reaction Quotient]
\label{def:reaction_quotient}
The reaction quotient $Q$ measures the current concentration ratio at any point during the reaction:
\begin{equation}
Q = \frac{[B]}{[A]}
\label{eq:reaction_quotient}
\end{equation}
For more complex reactions with stoichiometric coefficients, $Q$ generalizes to $Q = \prod_i [X_i]^{\nu_i}$, where $\nu_i$ is the stoichiometric coefficient of species $i$ (positive for products, negative for reactants).
\end{definition}

\begin{proposition}[Direction from Entropy Imbalance]
\label{prop:q_direction}
The relationship between the reaction quotient $Q$ and the equilibrium constant $K_{eq}$ determines which entropy production rate dominates, and therefore which direction the reaction proceeds:
\begin{equation}
\begin{cases}
Q < K_{eq} & \implies \dot{S}_{\text{forward}} > \dot{S}_{\text{reverse}} & \implies \text{reaction shifts right} \\
Q > K_{eq} & \implies \dot{S}_{\text{reverse}} > \dot{S}_{\text{forward}} & \implies \text{reaction shifts left} \\
Q = K_{eq} & \implies \dot{S}_{\text{forward}} = \dot{S}_{\text{reverse}} & \implies \text{equilibrium}
\end{cases}
\label{eq:q_direction}
\end{equation}
The reaction quotient thus serves as an indicator of entropy production rate imbalance.
\end{proposition}

\begin{proof}
When $Q < K_{eq}$, the current concentration ratio $[B]/[A]$ is less than the equilibrium ratio. This means there is relatively more A than B compared to the equilibrium state. The forward reaction rate $r_f = k_f [A]$ is high relative to the reverse rate $r_r = k_r [B]$. Since both reactions produce entropy (Theorem~\ref{thm:reaction_entropy}), but forward reactions occur more frequently, the forward entropy production rate dominates:
\begin{equation}
\dot{S}_{\text{forward}} = r_f \Delta S_f > r_r \Delta S_r = \dot{S}_{\text{reverse}}
\end{equation}

The system "flows" in the direction of higher entropy production (forward) until the rates equalize at $Q = K_{eq}$.

When $Q > K_{eq}$, there is relatively more B than A compared to equilibrium. The reverse rate $r_r = k_r [B]$ is high relative to the forward rate $r_f = k_f [A]$, giving $\dot{S}_{\text{reverse}} > \dot{S}_{\text{forward}}$. The system shifts left (toward reactants) to restore balance.

When $Q = K_{eq}$, the concentration ratio matches the equilibrium ratio. The forward and reverse rates are balanced: $r_f / r_r = k_f [A] / (k_r [B]) = k_f / k_r \cdot ([A]/[B]) = k_f / k_r \cdot (1/K_{eq})$. At equilibrium, this ratio equals unity (from the definition of $K_{eq}$), giving $r_f = r_r$ and therefore $\dot{S}_{\text{forward}} = \dot{S}_{\text{reverse}}$ (assuming $\Delta S_f \approx \Delta S_r$, which holds for elementary reactions). The system is at equilibrium. \qed
\end{proof}

\begin{corollary}[Gibbs Free Energy and Entropy Imbalance]
\label{cor:gibbs_entropy}
The Gibbs free energy change $\Delta G$ is related to the entropy production rate imbalance:
\begin{equation}
\Delta G = \Delta G^\circ + RT \ln Q = RT \ln \frac{Q}{K_{eq}}
\end{equation}
where we used $\Delta G^\circ = -RT \ln K_{eq}$. The sign of $\Delta G$ indicates which entropy production rate dominates:
\begin{equation}
\begin{cases}
\Delta G < 0 & \iff Q < K_{eq} & \iff \dot{S}_{\text{forward}} > \dot{S}_{\text{reverse}} \\
\Delta G > 0 & \iff Q > K_{eq} & \iff \dot{S}_{\text{reverse}} > \dot{S}_{\text{forward}} \\
\Delta G = 0 & \iff Q = K_{eq} & \iff \dot{S}_{\text{forward}} = \dot{S}_{\text{reverse}}
\end{cases}
\end{equation}
The free energy criterion for spontaneity ($\Delta G < 0$) is thus equivalent to the entropy production rate criterion ($\dot{S}_{\text{forward}} > \dot{S}_{\text{reverse}}$). Our framework reveals that free energy minimization is the macroscopic manifestation of entropy production rate balance.
\end{corollary}

\subsection{Temperature Dependence and the van 't Hoff Equation}

The temperature dependence of chemical equilibrium can also be understood through entropy production rates, providing an alternative derivation of the van 't Hoff equation.

\begin{proposition}[Temperature Effect on Equilibrium]
\label{prop:temperature_effect}
For an endothermic reaction ($\Delta H > 0$, forward direction absorbs heat), increasing temperature shifts equilibrium toward products. For an exothermic reaction ($\Delta H < 0$, forward direction releases heat), increasing temperature shifts equilibrium toward reactants. This temperature dependence arises from the differential effect of temperature on forward and reverse entropy production rates.
\end{proposition}

\begin{proof}
Temperature affects reaction rates through the Arrhenius equation:
\begin{equation}
k = A \exp\left(-\frac{E_a}{RT}\right)
\end{equation}
where $E_a$ is the activation energy. Taking the derivative with respect to temperature:
\begin{equation}
\frac{d \ln k}{dT} = \frac{E_a}{RT^2}
\end{equation}

For an endothermic reaction, the forward activation energy $E_a^f$ is higher than the reverse activation energy $E_a^r$ (the forward direction must overcome a larger energy barrier). Therefore, the forward rate constant is more sensitive to temperature:
\begin{equation}
\frac{d \ln k_f}{dT} = \frac{E_a^f}{RT^2} > \frac{E_a^r}{RT^2} = \frac{d \ln k_r}{dT}
\end{equation}

Increasing temperature increases both $k_f$ and $k_r$, but $k_f$ increases more rapidly. This increases $\dot{S}_{\text{forward}} = k_f [A] \Delta S_f$ more than $\dot{S}_{\text{reverse}} = k_r [B] \Delta S_r$, breaking the entropy production rate balance. The system shifts right (toward products) to restore balance at the new temperature.

For an exothermic reaction, $E_a^r > E_a^f$, so the reverse rate is more sensitive to temperature. Increasing temperature favors the reverse direction, shifting equilibrium toward reactants.

The equilibrium constant $K_{eq} = k_f / k_r$ (for elementary reactions) thus has temperature dependence:
\begin{equation}
\frac{d \ln K_{eq}}{dT} = \frac{d \ln k_f}{dT} - \frac{d \ln k_r}{dT} = \frac{E_a^f - E_a^r}{RT^2} = \frac{\Delta H^\circ}{RT^2}
\end{equation}
where $\Delta H^\circ = E_a^f - E_a^r$ is the standard enthalpy change. This is the van 't Hoff equation, derived from entropy production rate balance rather than from free energy minimization. \qed
\end{proof}

\begin{remark}[Entropy Production Perspective on van 't Hoff]
\label{rem:vant_hoff_entropy}
The van 't Hoff equation is traditionally derived from the temperature dependence of Gibbs free energy: $\Delta G = \Delta H - T \Delta S$, giving $d(\Delta G / T) / dT = -\Delta H / T^2$, which combined with $\Delta G = -RT \ln K_{eq}$ yields the van 't Hoff equation. Our derivation from entropy production rates provides an alternative perspective: temperature affects equilibrium by changing the relative rates of entropy production in forward and reverse directions. The direction with higher activation energy (more temperature-sensitive) gains an advantage at higher temperature, shifting equilibrium in that direction. This perspective emphasizes the dynamical nature of equilibrium as a balance of ongoing processes rather than a static minimum of free energy.
\end{remark}

\subsection{The Unified Framework}

We now have a unified framework connecting three fundamental phenomena that were previously understood through separate mechanisms: Maxwell's demon, the Gibbs paradox, and Le Chatelier's principle. All three arise from the same categorical mechanism: symmetric entropy increase through network densification.

The unification is summarized as follows. In the Maxwell's demon scenario, the two "containers" are Chamber A and Chamber B separated by a partition with a door. The key result is that door opening (molecular transfer) increases entropy in both chambers: $\Delta S_A > 0$ and $\Delta S_B > 0$ (Theorem~\ref{thm:transfer_mixing}). In the Gibbs paradox, the two "containers" are the state before mixing and the state after mixing (or equivalently, the two gases before removal of the partition). The key result is that both mixing and separation increase entropy: $\Delta S_{\text{mix}} > 0$ and $\Delta S_{\text{separate}} > 0$ (Corollary~\ref{cor:no_entropy_paradox}). In Le Chatelier's principle, the two "containers" are the reactant population and the product population. The key result is that both forward and reverse reactions increase entropy in both populations: $\Delta S_A > 0$ and $\Delta S_B > 0$ for both reaction directions (Theorem~\ref{thm:reaction_entropy}).

\begin{theorem}[Unified Categorical Equilibrium]
\label{thm:unified_equilibrium}
Equilibrium in any two-compartment system is the configuration where categorical entropy production rates balance between the two compartments:
\begin{equation}
\boxed{\dot{S}_{A \to B} = \dot{S}_{B \to A}}
\label{eq:unified_equilibrium}
\end{equation}
This balance condition applies universally to gas diffusion between chambers (Maxwell's demon scenario), mixing and separation of gases (Gibbs paradox), chemical reactions between reactants and products (Le Chatelier's principle), phase transitions between different phases (solid $\rightleftharpoons$ liquid $\rightleftharpoons$ gas), and any process with forward and reverse pathways connecting two populations with distinct categorical structures.
\end{theorem}

\begin{proof}
The proof follows from recognizing that all these phenomena share the same categorical structure: two populations with distinct phase-lock networks, connected by bidirectional pathways (molecular transfer, mixing/separation, chemical reaction, phase transition). From Theorem~\ref{thm:reaction_entropy}, any transfer from population A to population B increases entropy in both populations: $\Delta S_A > 0$ and $\Delta S_B > 0$. Similarly, transfer from B to A increases entropy in both populations.

The entropy production rates are:
\begin{align}
\dot{S}_{A \to B} &= r_{A \to B} \cdot (\Delta S_A + \Delta S_B)_{A \to B} \\
\dot{S}_{B \to A} &= r_{B \to A} \cdot (\Delta S_A + \Delta S_B)_{B \to A}
\end{align}
where $r_{A \to B}$ is the rate of transfer from A to B (molecules per unit time, or reaction events per unit time) and $r_{B \to A}$ is the reverse rate.

At equilibrium, the system has found the configuration where neither direction is thermodynamically favored. This occurs when the entropy production rates are balanced: $\dot{S}_{A \to B} = \dot{S}_{B \to A}$. If one direction produces entropy faster, the system shifts in that direction until balance is restored.

This balance condition is universal because it follows from the categorical structure (two populations, bidirectional pathways, symmetric entropy increase) rather than from the specific physical mechanism (gas diffusion, chemical reaction, phase transition). All two-compartment systems share this structure and therefore exhibit the same equilibrium condition. \qed
\end{proof}

\begin{corollary}[Equilibrium is Categorical, Not Kinetic]
\label{cor:equilibrium_categorical}
Equilibrium is determined by categorical entropy production rates, not by kinetic energy distributions or spatial configurations. Two systems with identical categorical structures (same phase-lock networks, same cluster distributions) are at the same equilibrium even if they have different temperatures, pressures, or spatial arrangements. Conversely, two systems with identical kinetic properties but different categorical structures are at different equilibria. This demonstrates that categorical structure is the fundamental determinant of equilibrium, with kinetic and spatial properties as secondary consequences.
\end{corollary}

\subsection{Experimental Implications}

The categorical framework makes several testable predictions that distinguish it from traditional thermodynamic treatments.

First, reaction entropy production is symmetric: both forward and reverse reactions should increase total entropy, measurable via precision calorimetry. Traditional thermodynamics predicts that forward reactions decrease free energy ($\Delta G < 0$, which includes an entropy increase $\Delta S > 0$) while reverse reactions increase free energy ($\Delta G > 0$). Our framework predicts that both directions increase entropy, with equilibrium as the balance point. This can be tested by measuring heat flow during reactions in both directions and verifying that both produce positive entropy changes.

Second, equilibrium is a dynamic entropy balance: at equilibrium, entropy production continues in both directions at equal rates. This is distinct from "no entropy production." Traditional thermodynamics treats equilibrium as a state of zero entropy production ($dS/dt = 0$), implying that nothing is happening. Our framework predicts that entropy production continues ($\dot{S}_{\text{forward}} > 0$ and $\dot{S}_{\text{reverse}} > 0$) but is balanced ($\dot{S}_{\text{forward}} = \dot{S}_{\text{reverse}}$). This can be tested by measuring entropy production rates at equilibrium using calorimetry or spectroscopy, verifying that both directions contribute to ongoing entropy generation.

Third, perturbation response is rate-driven: when perturbed, the system shifts in the direction of higher entropy production rate, not simply "toward lower free energy." This prediction can be tested by perturbing a system at equilibrium and measuring the transient entropy production rates in both directions, verifying that the system shifts in the direction with higher $\dot{S}$ until balance is restored.

Fourth, phase-lock correlations in reactions: reactant and product molecules should exhibit phase-lock correlations detectable through spectroscopic methods, even at equilibrium. Our framework predicts that molecules within each population (reactants or products) are phase-locked, forming clusters with correlated vibrational phases. These correlations should be observable through techniques such as two-dimensional infrared spectroscopy or ultrafast coherent spectroscopy, which can detect phase relationships between molecular vibrations. Observing such correlations would provide direct evidence for the phase-lock network structure underlying chemical equilibrium.

\subsection{Time, Categories, and the Nature of Equilibrium}

A subtle but fundamental point emerges from the categorical framework: equilibrium is a categorical phenomenon, not a temporal one. This distinction has profound implications for understanding the relationship between time and thermodynamics.

\subsubsection{The Measurement Frame Problem}

We measure reactions in time: reaction rates are expressed as "molecules per second" or "moles per second." But reactions occur through categorical completion, not temporal progression. Time is our measurement frame—the coordinate system we use to observe and quantify reactions. Categories are the reaction's intrinsic frame—the coordinate system in which reactions actually occur.

\begin{proposition}[Categorical Independence from Time]
\label{prop:categorical_time}
The categorical completion rate $\rho_C$ is independent of the temporal rate. Formally:
\begin{equation}
\rho_C = \frac{\text{number of categories completed}}{\text{number of categorical steps}} \neq \frac{\text{number of molecules reacted}}{\text{time elapsed}}
\label{eq:categorical_rate}
\end{equation}
Two reactions with different temporal rates can have identical categorical rates if they complete the same categorical pathways in the same number of steps, even if one takes longer in clock time.
\end{proposition}

\begin{proof}
Categorical completion is determined by phase-lock network topology (Theorem~\ref{thm:phase_lock_accessibility}). The accessible states from state $C_i$ are $\accessible(C_i) = \{C_j : (C_i, C_j) \in E_{\text{PL}}\}$, where $E_{\text{PL}}$ is the set of phase-lock adjacency edges. Completing state $C_i$ makes states in $\accessible(C_i)$ available for completion, initiating a cascade through the network (Theorem~\ref{thm:categorical_cascade}).

The categorical completion rate $\rho_C$ quantifies how rapidly the system traverses this network: how many categorical states are completed per categorical step. This rate depends on network topology (how many adjacent states are available at each step) but not on how much clock time elapses between steps.

The temporal rate $r_t = dn/dt$ (molecules reacted per unit time) depends on both categorical completion and temporal dynamics: how rapidly categorical steps occur in clock time. Two reactions can have the same categorical rate $\rho_C$ (same network topology, same number of steps) but different temporal rates $r_t$ if the time per categorical step differs (e.g., due to different activation energies, temperatures, or catalysts).

Therefore, $\rho_C$ and $r_t$ are independent quantities. Categorical rate is intrinsic to the reaction's network structure; temporal rate is extrinsic, depending on how that structure is traversed in time. \qed
\end{proof}

\subsubsection{Equilibrium Has No Time Coordinate}

Consider the equilibrium condition $\dot{S}_{\text{forward}} = \dot{S}_{\text{reverse}}$. This equality holds in categorical space: the forward and reverse processes produce entropy at equal rates when measured in terms of categorical completion. However, the forward and reverse processes may have different temporal rates.

For example, consider a reaction at equilibrium with forward rate $r_f = 10$ mol/s and reverse rate $r_r = 5$ mol/s. At first glance, this appears inconsistent with equilibrium (rates should be equal). However, if the forward reaction produces $\Delta S_f = 0.5$ entropy units per mole while the reverse reaction produces $\Delta S_r = 1.0$ entropy units per mole, then the entropy production rates are:
\begin{align}
\dot{S}_{\text{forward}} &= r_f \Delta S_f = 10 \times 0.5 = 5 \text{ entropy units per second} \\
\dot{S}_{\text{reverse}} &= r_r \Delta S_r = 5 \times 1.0 = 5 \text{ entropy units per second}
\end{align}

The entropy production rates are equal ($\dot{S}_{\text{forward}} = \dot{S}_{\text{reverse}} = 5$), so the system is at equilibrium, even though the temporal rates differ ($r_f \neq r_r$). The equilibrium condition is satisfied in categorical space (entropy production rate balance) but not in temporal space (temporal rate balance).

\begin{theorem}[Timelessness of Equilibrium]
\label{thm:timeless_equilibrium}
The equilibrium point exists in categorical entropy space but has no intrinsic time coordinate. Formally, the equilibrium state $\mathbf{S}_{eq}$ is an element of the categorical entropy space $\mathcal{S}$ but has undefined temporal coordinate:
\begin{equation}
\mathbf{S}_{eq} \in \mathcal{S} \quad \text{but} \quad t_{eq} = \text{undefined}
\label{eq:timeless_equilibrium}
\end{equation}
Time flows through equilibrium; equilibrium does not flow through time.
\end{theorem}

\begin{proof}
At equilibrium, the net categorical position is stationary. The forward and reverse processes produce equal entropy, so the system's position in categorical entropy space remains fixed:
\begin{equation}
\frac{d\mathbf{S}_{\text{net}}}{dt} = \dot{S}_{\text{forward}} - \dot{S}_{\text{reverse}} = 0
\end{equation}

Temporal evolution continues: the clock advances ($t_1 \to t_2 \to t_3 \to \cdots$), molecules react, energy is exchanged. But the categorical position remains unchanged:
\begin{equation}
t_1 \to t_2 \to t_3 \to \cdots \quad \text{while} \quad \mathbf{S}_{eq} \to \mathbf{S}_{eq} \to \mathbf{S}_{eq}
\end{equation}

The equilibrium point $\mathbf{S}_{eq}$ is invariant under time translation. It exists outside the time dimension in the sense that its categorical coordinates do not change as time progresses. Time is a parameter that flows through the equilibrium state, but the state itself does not move in time.

This is analogous to a fixed point in a dynamical system: the system's trajectory passes through the fixed point repeatedly, but the fixed point itself does not move. Equilibrium is a fixed point in categorical entropy space, invariant under temporal evolution. \qed
\end{proof}

\begin{remark}[Philosophical Implications]
\label{rem:time_philosophy}
Theorem~\ref{thm:timeless_equilibrium} has profound philosophical implications. It suggests that time is not a fundamental coordinate for thermodynamic processes but rather an emergent parameter that arises from our observation of categorical evolution. The fundamental coordinates are categorical (network topology, cluster structure, entropy production rates), and time is a derived quantity that measures how rapidly these categorical coordinates change. At equilibrium, categorical coordinates are stationary, so time becomes "irrelevant" in the sense that nothing changes categorically even though time continues to flow. This inverts the usual perspective in which time is fundamental and equilibrium is a special temporal state (steady state). In the categorical view, categories are fundamental and time is a special categorical state (the state where categorical coordinates are stationary).
\end{remark}

\subsubsection{The Mutual Penultimate State}

From the Poincaré computing framework referenced in the introduction, computational solutions are recognized at the penultimate state—one step before completion. The system "knows" it has found the solution when it is one categorical step away from the final state. At equilibrium, both forward and reverse processes are simultaneously at their penultimate states.

\begin{corollary}[Equilibrium as Mutual Penultimate]
\label{cor:mutual_penultimate}
Equilibrium is the configuration where both forward and reverse processes are simultaneously at their penultimate states. The categorical distance from equilibrium to the forward completion state (all products) is one step, and the categorical distance to the reverse completion state (all reactants) is also one step:
\begin{equation}
d_{\text{cat}}(\mathbf{S}_{eq}, \mathbf{S}_{\text{products}}) = 1 \quad \text{and} \quad d_{\text{cat}}(\mathbf{S}_{eq}, \mathbf{S}_{\text{reactants}}) = 1
\label{eq:mutual_penultimate}
\end{equation}
where $d_{\text{cat}}$ is the categorical distance (shortest path length in the phase-lock network). Neither process can complete because each blocks the other's final step: completing the forward process would require moving away from the reverse completion, and vice versa.
\end{corollary}

\begin{proof}
At equilibrium, the system is poised between two completion states: complete conversion to products (forward completion) and complete conversion to reactants (reverse completion). The entropy production rates are balanced, meaning that the system is equally "close" to both completion states in the categorical metric.

From Theorem~\ref{thm:entropy_path}, entropy is inversely related to the shortest path length to termination: $S(C) = -k_B \log \ell_{\text{term}}(C)$. At equilibrium, the system has equal entropy production rates in both directions, implying equal path lengths to completion in both directions:
\begin{equation}
\ell_{\text{term}}^{\text{forward}}(\mathbf{S}_{eq}) = \ell_{\text{term}}^{\text{reverse}}(\mathbf{S}_{eq})
\end{equation}

For a simple two-state system (reactants $\rightleftharpoons$ products), the equilibrium state is one categorical step away from both completion states: $\ell_{\text{term}}^{\text{forward}} = \ell_{\text{term}}^{\text{reverse}} = 1$. This is the penultimate state for both processes.

The two processes mutually block each other: completing the forward process (moving to all products) would increase the categorical distance to the reverse completion (all reactants), breaking the balance. Similarly, completing the reverse process would break the balance in the opposite direction. The system is "stuck" at the mutual penultimate state, unable to complete either process without violating the entropy production rate balance. \qed
\end{proof}

\subsubsection{Perturbation as Categorical Expansion}

Le Chatelier's principle now has a deeper interpretation in terms of categorical space expansion.

\begin{theorem}[Perturbation Expands Categorical Space]
\label{thm:perturbation_expansion}
Adding reactants or products to a system at equilibrium introduces new categories, expanding the categorical space. The equilibrium position shifts because new categories exist that must be incorporated into the entropy production rate balance:
\begin{equation}
\text{Add molecules} \implies |\mathcal{C}'| > |\mathcal{C}| \implies \mathbf{S}_{eq}' \neq \mathbf{S}_{eq}
\label{eq:perturbation_expansion}
\end{equation}
where $\mathcal{C}$ is the categorical space before perturbation, $\mathcal{C}'$ is the expanded space after perturbation, and $\mathbf{S}_{eq}$, $\mathbf{S}_{eq}'$ are the old and new equilibrium points.
\end{theorem}

\begin{proof}
Adding molecules to the system creates new phase-lock possibilities. Each new molecule can form phase-lock edges with existing molecules, creating categorical relationships that did not previously exist. These new edges represent categories (equivalence classes of phase space regions) that were not part of the original categorical space $\mathcal{C}$.

Formally, let $\phaselockgraph = (V, E)$ be the phase-lock network before perturbation, with $|V| = N$ molecules and $|E|$ edges. After adding $\Delta N$ molecules, the network expands to $\phaselockgraph' = (V', E')$ with $|V'| = N + \Delta N$ molecules and $|E'| = |E| + \Delta |E|$ edges, where $\Delta |E|$ is the number of new edges formed by the added molecules.

The categorical space $\mathcal{C}$ is the set of equivalence classes of phase space regions determined by phase-lock relationships. Adding edges expands the categorical space: $\mathcal{C} \subset \mathcal{C}'$ with $|\mathcal{C}'| > |\mathcal{C}|$ (more categories exist in the expanded space).

The old equilibrium $\mathbf{S}_{eq}$ was the point in $\mathcal{C}$ where entropy production rates balanced:
\begin{equation}
\dot{S}_{\text{forward}}(\mathbf{S}_{eq}) = \dot{S}_{\text{reverse}}(\mathbf{S}_{eq})
\end{equation}

But in the expanded space $\mathcal{C}'$, the old equilibrium $\mathbf{S}_{eq}$ is no longer the balance point. The new categories introduced by the added molecules contribute to entropy production rates, shifting the balance. The system must navigate to a new equilibrium $\mathbf{S}_{eq}' \in \mathcal{C}'$ where:
\begin{equation}
\dot{S}_{\text{forward}}(\mathbf{S}_{eq}') = \dot{S}_{\text{reverse}}(\mathbf{S}_{eq}')
\end{equation}
in the expanded space.

This navigation appears macroscopically as "shifting to counteract the perturbation" (Le Chatelier's principle). The system is not "counteracting" anything—it is simply finding the new balance point in the expanded categorical space. \qed
\end{proof}

\subsubsection{Why Time "Flows" Yet "Doesn't Flow" at Equilibrium}

The apparent paradox of equilibrium—that time continues to pass yet nothing changes—resolves in the categorical framework.

Time flows: our clocks advance, molecules react, energy is exchanged, entropy is produced. Temporal evolution continues at equilibrium just as it does away from equilibrium. There is no "stopping" of time or dynamics.

Time doesn't flow: the net categorical position is unchanged. The system revisits the same categorical state repeatedly. Forward reactions move the system toward products in categorical space, but reverse reactions move it back toward reactants by an equal categorical distance. The net categorical displacement is zero.

At equilibrium, time is real but categorically irrelevant. Time passes through the equilibrium state like water flowing through a stationary rock—the rock doesn't move with the current. The equilibrium state is a fixed point in categorical space, invariant under temporal evolution.

This is the deepest meaning of equilibrium: the system has found a categorical fixed point where temporal dynamics cancel exactly. Time is the coordinate in which we observe this cancellation, but the cancellation itself is a categorical property (entropy production rate balance), not a temporal property (zero rate of change).

\subsection{The Equilibrium Freeze Paradox: Why Time Cannot Be Fundamental}

Traditional thermodynamics rests on three assumptions that, when taken together, lead to a devastating contradiction. We now prove that these assumptions are mutually inconsistent, forcing us to abandon at least one. The categorical framework resolves the contradiction by rejecting the assumption that time is fundamental.

\subsubsection{The Three Assumptions}

Traditional thermodynamics implicitly assumes the following three principles.

First, time is fundamental: reactions evolve "in time," entropy increases "with time," equilibrium is reached "after time $t$." Time is treated as the fundamental coordinate in which thermodynamic processes occur. Reaction rates are defined as temporal derivatives ($dn/dt$), and the Second Law is stated as a temporal inequality ($dS/dt \geq 0$).

Second, equilibrium is reversible: forward and reverse rates are equal at equilibrium, the system can return to any previous state through Poincaré recurrence, and there is no fundamental arrow of time at equilibrium. Microscopic reversibility holds: for every forward process, there is a reverse process with equal probability.

Third, equilibrium is unique: there exists exactly one configuration with $\Delta G = 0$ for given external conditions (temperature, pressure, composition). The system "seeks" this unique equilibrium state, and once reached, equilibrium is stable and unique.

\subsubsection{The Formal Paradox}

\begin{theorem}[The Equilibrium Freeze Paradox]
\label{thm:freeze_paradox}
If time is fundamental, equilibrium is reversible, and equilibrium is unique, then reactions should never proceed from any initial state. The system should "freeze" at the initial configuration, unable to evolve toward equilibrium.
\end{theorem}

\begin{proof}
Assume all three premises hold. We derive a contradiction.

\textbf{Step 1: Poincaré Recurrence.}
By the Poincaré recurrence theorem, a finite system in a bounded phase space will return arbitrarily close to any initial state after sufficient time. Formally, for any initial state $\mathbf{x}(0)$ and any $\epsilon > 0$, there exists a recurrence time $T_{\text{rec}}$ such that:
\begin{equation}
d(\mathbf{x}(t + T_{\text{rec}}), \mathbf{x}(t)) < \epsilon
\end{equation}
for some $t$. The system eventually returns to the neighborhood of the initial state.

\textbf{Step 2: Initial State as Equilibrium.}
If the system can return to the initial state (e.g., pure reactants), and equilibrium is reversible (the system can traverse all states in both directions), then the initial state must be an accessible equilibrium. If it were not an equilibrium, the system would spontaneously evolve away from it upon return, violating the assumption that equilibrium is stable. Therefore:
\begin{equation}
\mathbf{x}_{\text{initial}} \in \{\text{reachable equilibria}\}
\end{equation}

\textbf{Step 3: Contradiction with Unique Equilibrium.}
But we assume equilibrium is unique: there is only one state with $\Delta G = 0$. If the initial state is an accessible equilibrium, and equilibrium is unique, then the initial state must be the unique equilibrium:
\begin{equation}
\mathbf{x}_{\text{initial}} = \mathbf{x}_{\text{equilibrium}}
\end{equation}

Similarly, any final state reachable from the initial state must also be the unique equilibrium (since the system can return to the initial state by reversibility, and the initial state is the equilibrium). Therefore:
\begin{equation}
\mathbf{x}_{\text{initial}} = \mathbf{x}_{\text{equilibrium}} = \mathbf{x}_{\text{final}}
\end{equation}

\textbf{Step 4: No Reaction Should Occur.}
If the initial state equals the equilibrium state, then at $t = 0$, the system is already at equilibrium. The time derivative of any macroscopic observable must be zero:
\begin{equation}
\frac{d\mathbf{x}}{dt}\bigg|_{t=0} = 0
\end{equation}

The reaction should never start. The system should remain frozen at the initial state for all time.

\textbf{Step 5: Contradiction with Observation.}
But reactions do occur. We observe that systems initially far from equilibrium (e.g., pure reactants) evolve toward equilibrium (mixed reactants and products):
\begin{equation}
\mathbf{x}(t) \neq \mathbf{x}_{\text{initial}} \quad \text{for } t > 0
\end{equation}

This contradicts the conclusion of Step 4.

\textbf{Conclusion:}
The three assumptions are mutually inconsistent. At least one must be false. \qed
\end{proof}

\subsubsection{The Resolution: Categorical Irreversibility}

The categorical framework resolves the paradox by rejecting the first assumption: time is not fundamental. Categories are fundamental, and time is an emergent parameter that measures categorical evolution.

\begin{theorem}[Categorical Resolution of the Freeze Paradox]
\label{thm:freeze_resolution}
The equilibrium freeze paradox is resolved when categories (not time) are fundamental, equilibrium is categorically irreversible (even if spatially reversible), and equilibrium is a categorical fixed point (not a temporal destination). These three principles are mutually consistent and consistent with observation.
\end{theorem}

\begin{proof}
We show that the categorical framework avoids the contradiction in Theorem~\ref{thm:freeze_paradox}.

\textbf{Categories are fundamental:}
Reactions evolve through categorical completion (Theorem~\ref{thm:categorical_cascade}), not temporal progression. The fundamental coordinate is categorical state $C \in \catspace$, not time $t$. Time is our measurement frame—the parameter we use to observe categorical evolution—but it is not the intrinsic coordinate of the reaction. The rate of categorical evolution is:
\begin{equation}
\frac{dC}{d(\text{categorical step})} \neq \frac{d\mathbf{x}}{dt}
\end{equation}

Two reactions can have the same categorical evolution (same sequence of categorical states) but different temporal evolution (different times to traverse the sequence).

\textbf{Categorical irreversibility:}
By Axiom~\ref{axiom:categorical_irreversibility}, once a categorical state is completed, it cannot be re-occupied in the same categorical context. The categorical ordering is strict: $C_{\text{initial}} \prec C_{\text{mixed}} \prec C_{\text{final}}$, which implies $C_{\text{initial}} \neq C_{\text{final}}$ even if the spatial states are similar.

This breaks the Poincaré recurrence in categorical space. Even if the system returns to a spatial state close to the initial state (Poincaré recurrence in phase space), the categorical state is different. The system has "learned" the categorical structure through completion, and this learning is irreversible.

\textbf{Equilibrium outside time:}
Equilibrium is not "reached after time $t$" but rather is a categorical fixed point where entropy production rates balance:
\begin{equation}
\dot{S}_{A \to B}(\mathbf{S}_{eq}) = \dot{S}_{B \to A}(\mathbf{S}_{eq})
\end{equation}

Time flows through this point; the point does not move in time (Theorem~\ref{thm:timeless_equilibrium}). The equilibrium is unique in categorical space but has no unique temporal coordinate.

\textbf{Why reactions proceed:}
With categorical irreversibility, the freeze paradox dissolves. The initial state is not an equilibrium because it has categorical asymmetry: more forward categories are accessible than reverse categories. Formally:
\begin{equation}
|\mathcal{C}_{\text{forward}}(\mathbf{x}_{\text{initial}})| \neq |\mathcal{C}_{\text{reverse}}(\mathbf{x}_{\text{initial}})|
\end{equation}

This asymmetry drives categorical completion in the forward direction until balance is achieved at equilibrium. The system cannot "freeze" at the initial state because categorical completion is a deterministic process driven by network topology (Theorem~\ref{thm:phase_lock_accessibility}). Once a state is completed, adjacent states become accessible and must be completed in turn.

The initial state cannot be revisited in the same categorical context (even if it is revisited spatially through Poincaré recurrence) because categorical completion is irreversible. Therefore, the initial state is not an equilibrium, and the paradox is avoided. \qed
\end{proof}

\subsubsection{Why Reactions Proceed}

The categorical resolution provides a clear answer to the question: why do reactions proceed at all?

\begin{proposition}[Reactions Proceed via Categorical Asymmetry]
\label{prop:reactions_proceed}
Reactions proceed from initial states toward equilibrium because the initial state has categorical asymmetry: the number of accessible forward categories differs from the number of accessible reverse categories. The direction with more accessible categories "wins" until balance is achieved:
\begin{equation}
|\mathcal{C}_{\text{forward}}| \neq |\mathcal{C}_{\text{reverse}}| \implies \text{net categorical flow toward equilibrium}
\label{eq:categorical_asymmetry}
\end{equation}
\end{proposition}

\begin{proof}
At the initial state (e.g., pure reactants), the forward direction has many accessible categories: all the possible ways to form products through phase-lock pathways. The reverse direction has few accessible categories: there are no products to convert back to reactants.

This categorical asymmetry creates an imbalance in entropy production rates:
\begin{equation}
\dot{S}_{\text{forward}}(\mathbf{x}_{\text{initial}}) \gg \dot{S}_{\text{reverse}}(\mathbf{x}_{\text{initial}})
\end{equation}

The system evolves in the direction of higher entropy production (forward) because categorical completion follows accessible pathways (Theorem~\ref{thm:phase_lock_accessibility}), and more pathways are available in the forward direction.

As the reaction proceeds, products accumulate and reactants are consumed. The number of accessible forward categories decreases (fewer reactants available to react), while the number of accessible reverse categories increases (more products available to reverse). The categorical asymmetry diminishes until balance is achieved:
\begin{equation}
|\mathcal{C}_{\text{forward}}(\mathbf{x}_{eq})| = |\mathcal{C}_{\text{reverse}}(\mathbf{x}_{eq})|
\end{equation}

At this point, entropy production rates are balanced ($\dot{S}_{\text{forward}} = \dot{S}_{\text{reverse}}$), and the system is at equilibrium. The reaction "proceeds" because categorical completion is driven by categorical asymmetry, and completion is irreversible—the system cannot return to the initial categorical state. \qed
\end{proof}

The initial state is not an equilibrium because it has categorical asymmetry (more forward categories available). categorical completion drives the system toward balance; the system cannot "freeze" because categories must be completed (deterministic process), and completion is irreversible—no return to the initial categorical state even if the spatial state is revisited.

\subsection{Biological Validation: Enzymes as Energy Negotiators}

The categorical framework makes a testable prediction about enzyme function that sharply distinguishes it from time-fundamental thermodynamics. We now show that the observed properties of enzymes validate the categorical framework and contradict the time-fundamental perspective.

\subsubsection{The Time-Fundamental Prediction}

If time were fundamental, enzymes would function as time accelerators: they would compress the temporal duration of reactions without changing the fundamental pathway.

\begin{proposition}[Time-Fundamental Enzyme Model]
\label{prop:time_enzyme}
If time is the fundamental variable, then enzyme function should be quantified as a "speedup factor":
\begin{equation}
\text{Enzyme effect} = \frac{t_{\text{uncatalyzed}}}{t_{\text{catalyzed}}} = \text{speedup factor}
\label{eq:enzyme_speedup}
\end{equation}
where $t_{\text{uncatalyzed}}$ is the time required for the reaction without the enzyme and $t_{\text{catalyzed}}$ is the time with the enzyme. Enzymes would simply make reactions happen faster along the same pathway, compressing time.
\end{proposition}

Under this model, all enzymes would be equivalent up to their speedup factor (a single number characterising each enzyme); enzyme mechanism would be irrelevant (only the speedup matters, not how it is achieved); enzyme specificity would be unexplained (why does each enzyme catalyse only specific reactions?); and allosteric regulation would be unnecessary (why would enzymes need complex regulatory mechanisms if they just accelerate time?).



%==============================================================================
\section{Heat Transfer versus Entropy: The Fundamental Decoupling}
\label{sec:heat_transfer}
%==============================================================================

A critical insight emerges from analyzing what happens when the demon's door operation results in a collision near the aperture. The analysis reveals that heat transfer and entropy change are fundamentally decoupled: heat can flow in either direction during individual molecular collisions, while entropy increases monotonically regardless of heat flow direction. This decoupling exposes a profound conceptual error in the original formulation of Maxwell's demon paradox. Maxwell framed the paradox in terms of heat flow (can the demon transfer heat from cold to hot?), implicitly assuming that the direction of heat flow determines the direction of entropy change. This assumption is correct macroscopically in the thermodynamic limit but fails at the microscopic level where individual molecular collisions exhibit thermal fluctuations. The demon operates at the microscopic level, attempting to exploit these fluctuations, but the quantity it manipulates (heat) is not the quantity constrained by the Second Law (entropy). The Second Law constrains entropy, which is a categorical property determined by phase-lock network structure, not heat, which is a statistical property determined by averaging over many collisions. This fundamental decoupling renders the demon's strategy irrelevant: even if the demon succeeds in transferring heat from cold to hot in individual collisions, entropy still increases through categorical completion, preserving the Second Law.

\subsection{The Collision Scenario}

We begin by defining the physical scenario precisely. Consider the demon opening the door to allow a fast molecule from the hot container (A) to pass through the aperture to the cold container (B). As this molecule transits through the aperture, it may collide with a molecule already present in container B near the aperture. This collision is the critical event that reveals the heat-entropy decoupling.

\begin{definition}[Door Collision Event]
\label{def:door_collision}
A \textbf{door collision event} occurs when a molecule transiting through the demon's aperture collides with a molecule in the receiving container before fully entering and thermalizing with the bulk gas. Let $m_A$ denote the transiting molecule from container A with initial velocity $\mathbf{v}_A$ (before collision), $m_B$ denote the molecule in container B with initial velocity $\mathbf{v}_B$ (before collision), and $\mathbf{v}_A'$, $\mathbf{v}_B'$ denote the post-collision velocities of molecules $m_A$ and $m_B$, respectively. The collision occurs in the aperture region, a small volume near the partition where molecules from both containers can interact before the transiting molecule has equilibrated with its destination container.
\end{definition}

\begin{remark}[Physical Realization]
\label{rem:collision_physical}
Door collision events are not merely theoretical constructs but are physically inevitable in any realistic implementation of Maxwell's demon. The aperture has finite size (it must be large enough for molecules to pass through), and the receiving container has finite density (molecules are present near the aperture). The probability of a collision during transit is approximately $P_{\text{collision}} \sim \rho \sigma v_A \Delta t$, where $\rho$ is the number density in container B, $\sigma$ is the collision cross-section, $v_A$ is the transiting molecule's speed, and $\Delta t$ is the transit time through the aperture. For typical gas densities and aperture sizes, $P_{\text{collision}} \sim 0.1$-$0.5$, meaning that a substantial fraction of door operations result in collisions. These collisions cannot be avoided without reducing the aperture size to the point where molecular transport becomes negligibly slow.
\end{remark}

We analyse three cases based on collision outcomes, showing that in all cases entropy increases, regardless of the heat flow direction.

\subsection{Case 1: Elastic Collision with Bounce-Back}

The first case considers an elastic collision in which the transiting molecule $m_A$ bounces back to its original container A after colliding with molecule $m_B$ in container B.

\begin{proposition}[Bounce-Back Heat Transfer]
\label{prop:bounce_back}
If molecule $m_A$ undergoes an elastic collision with molecule $m_B$ and bounces back to container A, then energy transfers from $m_A$ to $m_B$ (heat flows from hot to cold), the "counted" molecule returns to its origin (the demon's sorting operation fails), heat has transferred in the standard direction (hot $\to$ cold), and entropy increases in both containers through the formation of new phase-lock correlations.
\end{proposition}

\begin{proof}
In an elastic collision, kinetic energy is conserved:
\begin{equation}
\frac{1}{2}m_A v_A^2 + \frac{1}{2}m_B v_B^2 = \frac{1}{2}m_A v_A'^2 + \frac{1}{2}m_B v_B'^2
\label{eq:elastic_energy}
\end{equation}
where $v_A = |\mathbf{v}_A|$ and $v_B = |\mathbf{v}_B|$ are the initial speeds, and $v_A' = |\mathbf{v}_A'|$ and $v_B' = |\mathbf{v}_B'|$ are the final speeds.

Momentum is also conserved:
\begin{equation}
m_A \mathbf{v}_A + m_B \mathbf{v}_B = m_A \mathbf{v}_A' + m_B \mathbf{v}_B'
\label{eq:elastic_momentum}
\end{equation}

If molecule $m_A$ bounces back to container A, its velocity component perpendicular to the partition must reverse sign: $v_{A,\perp}' < 0$ (where we define positive perpendicular velocity as pointing from A to B). For this to occur, molecule $m_A$ must transfer momentum to molecule $m_B$ in the perpendicular direction.

For equal-mass molecules ($m_A = m_B = m$), the standard elastic collision formulas give:
\begin{align}
\mathbf{v}_A' &= \mathbf{v}_B \label{eq:elastic_vA} \\
\mathbf{v}_B' &= \mathbf{v}_A \label{eq:elastic_vB}
\end{align}
for a head-on collision. The velocities are exchanged. Since $m_A$ was initially fast (from the hot container) and $m_B$ was initially slow (from the cold container), after the collision $m_A$ is slow and $m_B$ is fast. The kinetic energies are:
\begin{align}
E_A' &= \frac{1}{2}m v_B^2 < \frac{1}{2}m v_A^2 = E_A \\
E_B' &= \frac{1}{2}m v_A^2 > \frac{1}{2}m v_B^2 = E_B
\end{align}

Energy has been transferred from molecule $m_A$ to molecule $m_B$. Since $m_A$ returns to container A with reduced energy and $m_B$ remains in container B with increased energy, the net effect is heat transfer from hot container A to cold container B. This is the standard thermodynamic direction.

However, the demon's sorting operation has failed: the molecule that was supposed to be transferred from A to B has returned to A. The demon opened the door intending to increase the temperature difference, but the collision reversed the transfer.

For entropy, we must consider the categorical effects. The collision creates new phase-lock correlations between molecules $m_A$ and $m_B$. Before the collision, the two molecules had independent phases (they were in different containers and had never interacted). After the collision, their trajectories are correlated: the post-collision velocities $\mathbf{v}_A'$ and $\mathbf{v}_B'$ are deterministically related to the pre-collision velocities through the collision dynamics.

This correlation increases the accessible categorical state space. Before the collision, the system could be in any categorical state consistent with the independent phases of $m_A$ and $m_B$. After the collision, the system can be in any categorical state consistent with the correlated phases. The number of accessible states increases because the correlation creates new equivalence classes: states that differ in the correlation structure are categorically distinct.

\begin{figure*}[htbp]
\centering
\includegraphics[width=0.95\textwidth]{figures/arg3_retrieval_paradox.png}
\caption{\textbf{The Retrieval Paradox—Velocity-Based Sorting is Self-Defeating.}
\textbf{(A)} Timescale hierarchy showing collisions happen first. Molecular collision timescale $\tau_{\text{coll}} \sim 10^{-10}$ s (red bar) is orders of magnitude faster than measurement ($\sim 10^{-8}$ s), gate operation ($\sim 10^{-6}$ s), sorting ($\sim 10^{-3}$ s), and demon decision-making ($\sim 10^{-1}$ s). The collision rate $\nu_{\text{coll}} \sim 10^{10}$ collisions/s in gases at STP ensures velocities randomize before any sorting operation can complete. The label ``TOO SLOW!'' emphasizes that sorting timescale exceeds thermalization timescale by $10^7$, making velocity-based sorting operationally impossible.
\textbf{(B)} Phase space scrambling showing sorted states randomize in $\tau_{\text{coll}}$. Initially sorted molecules (blue points, $t=0$) with velocities clustered in one region of phase space become completely randomized (red points, $t > \tau_{\text{coll}}$) after a single collision time. The velocity distribution returns to Maxwell-Boltzmann, erasing any sorting. This demonstrates that maintaining sorted states requires infinite retrieval operations.
\textbf{(C)} Sorting versus thermalization dynamics. The sorting signal strength $S(t) = |\langle v_A \rangle - \langle v_B \rangle|/\sigma_v$ (teal curve) decays exponentially as $S(t) = S_0 \exp(-t/\tau_{\text{coll}})$, while thermal randomization (red shaded region) dominates. The system relaxes to equilibrium ($S \to 0$) within $\sim 2\tau_{\text{coll}}$. The demon's sorting attempt (starting from yellow circle at $S_0 \approx 0.9$) is overwhelmed by thermalization.
\textbf{(D)} Long-term sorting attempts always return to 50/50 equilibrium. Three independent sorting attempts (colored traces) show fast/total molecule ratio fluctuating around equilibrium value 0.5 (red dashed line). Despite initial deviations, all attempts converge to the equilibrium distribution within $\sim 50$ time steps. The red shaded band indicates $\pm 2\sigma$ fluctuations. This confirms that velocity-sorted states cannot be maintained: the retrieval paradox makes the demon's operation self-defeating.}
\label{fig:retrieval_paradox}
\end{figure*}

Formally, the entropy increase from correlation is:
\begin{equation}
\Delta S_{\text{correlation}} = k_B \ln \frac{\Omega_{\text{correlated}}}{\Omega_{\text{uncorrelated}}} > 0
\label{eq:correlation_entropy}
\end{equation}
where $\Omega_{\text{correlated}}$ is the number of categorical states accessible after the collision (with correlation) and $\Omega_{\text{uncorrelated}}$ is the number accessible before (without correlation). The inequality holds because correlations enrich categorical structure, as proven in Proposition~\ref{prop:entropy_edge_density}: more edges (correlations) correspond to higher entropy.

Additionally, both containers experience categorical completion. In container A, the returning molecule $m_A$ (now with reduced energy) must re-establish phase-lock relationships with its neighbors. The network configuration before departure was $\phaselockgraph_A$; after return, it is $\phaselockgraph_A'$ with different edge structure. From Theorem~\ref{thm:categorical_cascade}, this reconfiguration is a categorical completion: $C_A \prec C_A'$, giving $\Delta S_A > 0$.

In container B, molecule $m_B$ (now with increased energy) has altered phase-lock relationships with its neighbors. The network advances: $C_B \prec C_B'$, giving $\Delta S_B > 0$.

The total entropy increase is:
\begin{equation}
\Delta S_{\text{total}} = \Delta S_A + \Delta S_B + \Delta S_{\text{correlation}} > 0
\end{equation}

Entropy increases in both containers despite the fact that the demon's sorting operation failed and heat flowed in the standard direction. \qed
\end{proof}

\begin{remark}[Demon's Failure in Case 1]
\label{rem:demon_failure_case1}
Case 1 demonstrates that even when heat flows in the thermodynamically expected direction (hot $\to$ cold), the demon's strategy fails. The demon opened the door to transfer a fast molecule from A to B, intending to make A colder and B hotter. But the collision reversed the transfer, and the fast molecule returned to A with reduced energy. The net effect is the opposite of the demon's intention: container A has lost energy (the returned molecule has less energy than when it left), and container B has gained energy (molecule $m_B$ is now faster). The temperature difference has decreased, not increased. The demon has accelerated equilibration rather than creating a temperature gradient. This failure occurs even though the demon "measured" the molecule correctly (it was indeed fast) and "opened the door at the right time" (the molecule was transiting). The failure is due to the uncontrollable collision dynamics, which are determined by microscopic phase-space trajectories that the demon cannot predict or control.
\end{remark}

\subsection{Case 2: Inelastic Collision—Cold Molecule Accelerates}

The second case considers an inelastic collision in which energy is dissipated (converted to internal degrees of freedom or radiated away) and the cold molecule $m_B$ gains significant kinetic energy.

\begin{proposition}[Standard Heat Transfer]
\label{prop:standard_heat}
If the collision is inelastic and molecule $m_B$ gains significant kinetic energy (accelerates), then heat transfers from hot to cold (standard thermodynamic direction), entropy increases in both containers through dissipation and phase-lock correlation, and this is the "expected" thermodynamic outcome consistent with macroscopic heat flow.
\end{proposition}

\begin{proof}
In an inelastic collision, kinetic energy is not conserved. Some kinetic energy is converted to internal energy (vibrational, rotational excitation of molecules) or radiated away (photon emission). Energy conservation including dissipation is:
\begin{equation}
\frac{1}{2}m_A v_A^2 + \frac{1}{2}m_B v_B^2 = \frac{1}{2}m_A v_A'^2 + \frac{1}{2}m_B v_B'^2 + Q_{\text{dissipated}}
\label{eq:inelastic_energy}
\end{equation}
where $Q_{\text{dissipated}} > 0$ is the energy converted to non-kinetic forms.

If molecule $m_B$ accelerates ($v_B' > v_B$), its kinetic energy increases:
\begin{equation}
E_B' = \frac{1}{2}m_B v_B'^2 > \frac{1}{2}m_B v_B^2 = E_B
\end{equation}

For energy to be conserved (equation~\eqref{eq:inelastic_energy}), molecule $m_A$ must lose more energy than $m_B$ gains:
\begin{equation}
E_A - E_A' = (E_B' - E_B) + Q_{\text{dissipated}} > E_B' - E_B
\end{equation}

Energy has flowed from molecule $m_A$ (initially fast, from hot container) to molecule $m_B$ (initially slow, from cold container). This is conventional heat transfer in the standard direction: hot $\to$ cold.

Entropy increases through three mechanisms.

First, dissipation increases entropy. The dissipated energy $Q_{\text{dissipated}}$ is converted to internal degrees of freedom (molecular vibrations, rotations) or radiated away. This conversion is irreversible and increases entropy by:
\begin{equation}
\Delta S_{\text{dissipation}} = \frac{Q_{\text{dissipated}}}{T} > 0
\label{eq:dissipation_entropy}
\end{equation}
where $T$ is the local temperature. This is the standard thermodynamic entropy increase from irreversible energy conversion.

Second, phase-lock correlations form from the collision. As in Case 1, the collision creates correlations between the trajectories of $m_A$ and $m_B$, increasing the accessible categorical state space:
\begin{equation}
\Delta S_{\text{correlation}} = k_B \ln \frac{\Omega_{\text{correlated}}}{\Omega_{\text{uncorrelated}}} > 0
\end{equation}

Third, categorical completion occurs in both containers. Molecule $m_A$ (now with reduced energy) reconfigures its phase-lock relationships in container A: $C_A \prec C_A'$, giving $\Delta S_A > 0$. Molecule $m_B$ (now with increased energy) reconfigures its relationships in container B: $C_B \prec C_B'$, giving $\Delta S_B > 0$.

The total entropy increase is:
\begin{equation}
\Delta S_{\text{total}} = \Delta S_A + \Delta S_B + \Delta S_{\text{correlation}} + \Delta S_{\text{dissipation}} > 0
\end{equation}

This is the "expected" thermodynamic outcome: heat flows from hot to cold, and entropy increases. Case 2 is consistent with macroscopic thermodynamics and does not present a paradox. \qed
\end{proof}

\begin{remark}[Macroscopic Consistency]
\label{rem:macroscopic_consistency}
Case 2 demonstrates that when averaged over many collisions, the demon's door operations produce the macroscopically expected outcome: heat flows from hot to cold, and entropy increases. This is why Maxwell's demon does not violate thermodynamics in practice: the majority of collisions (or at least a sufficient fraction) behave like Case 2, transferring heat in the standard direction and increasing entropy. The demon cannot selectively allow only "favorable" collisions (those that transfer heat cold $\to$ hot) because it cannot predict collision outcomes, which depend on microscopic phase-space details beyond the demon's knowledge. Even if the demon could predict outcomes, Case 3 (below) shows that "favorable" collisions still increase entropy, so the demon gains nothing.
\end{remark}

\subsection{Case 3: Inelastic Collision—Cold Molecule Decelerates}

The third case is the most revealing. It considers an inelastic collision in which the cold molecule $m_B$ loses kinetic energy (decelerates), transferring energy to the hot molecule $m_A$, which returns to container A with more energy than it started with. This appears to be heat transfer from cold to hot, apparently violating the Second Law.

\begin{theorem}[Reverse Heat Transfer with Entropy Increase]
\label{thm:reverse_heat}
If the collision is inelastic and molecule $m_B$ loses significant kinetic energy (decelerates), then heat transfers from cold to hot (reverse of the standard thermodynamic direction), the fast molecule $m_A$ may return to container A with more energy than it started with (apparent violation of energy conservation, resolved by extracting energy from $m_B$), and entropy still increases in both containers through phase-lock correlation and categorical completion, preserving the Second Law despite reverse heat flow.
\end{theorem}

\begin{proof}
Consider an inelastic collision with the following energy changes:
\begin{align}
E_B' &< E_B \quad \text{(cold molecule lost kinetic energy)} \label{eq:EB_decrease} \\
E_A' &> E_A \quad \text{(hot molecule gained kinetic energy)} \label{eq:EA_increase}
\end{align}

At first glance, this appears to violate energy conservation: how can both molecules change energy in the same direction (one loses, one gains more than the other loses)? The resolution is that molecule $m_B$ had kinetic energy before the collision, and this energy is extracted during the collision and transferred to molecule $m_A$. Additionally, dissipation may be negative (energy is released from internal degrees of freedom), or the collision may be asymmetric (molecule $m_B$ was moving toward the aperture, and the collision converts this directed motion into increased speed for $m_A$).

Formally, energy conservation, including dissipation, is:
\begin{equation}
E_A + E_B = E_A' + E_B' + Q_{\text{dissipated}}
\end{equation}

For equations~\eqref{eq:EB_decrease} and~\eqref{eq:EA_increase} to hold simultaneously, we need:
\begin{equation}
E_A' - E_A = (E_B - E_B') - Q_{\text{dissipated}}
\end{equation}

If $Q_{\text{dissipated}}$ is small (nearly elastic collision) or negative (energy released from internal degrees of freedom), then the energy gained by $m_A$ can exceed the energy lost by $m_B$. This is physically possible and does not violate energy conservation.

\textbf{Heat direction:}
Molecule $m_B$ is in the cold container B and loses energy. Molecule $m_A$ is in (or returning to) the hot container A and gains energy. Energy has flowed from the cold container to the hot container. This is heat transfer in the reverse direction: cold $\to$ hot.

Macroscopically, this appears to violate the Second Law, which states (in the Clausius formulation) that heat cannot spontaneously flow from cold to hot. However, the Clausius formulation applies to macroscopic heat flow averaged over many molecular collisions, not to individual collision events. At the microscopic level, thermal fluctuations allow individual collisions to transfer energy in either direction.

\textbf{Entropy direction:}
Despite the reverse heat flow, entropy still increases. The collision creates phase-lock correlations between molecules $m_A$ and $m_B$, regardless of the direction of energy transfer. The correlation entropy increase is:
\begin{equation}
\Delta S_{\text{correlation}} = k_B \ln \frac{\Omega_{\text{correlated}}}{\Omega_{\text{uncorrelated}}} > 0
\end{equation}

This entropy increase is independent of energy flow direction because it arises from categorical structure (phase-lock relationships), not from kinetic energy distribution.

In container A, the returning molecule $m_A$ now has more energy than when it left. It must reconfigure its phase-lock relationships with neighbors, which were established when it had lower energy. The network advances: $C_A \prec C_A'$, giving $\Delta S_A > 0$. The categorical completion occurs because the system must incorporate the new information that molecule $m_A$ has returned with altered energy, creating new categorical distinctions.

In container B, molecule $m_B$ has lost energy and now has different phase relationships with its neighbors. The network reconfigures: $C_B \prec C_B'$, giving $\Delta S_B > 0$. The categorical completion occurs because the system must incorporate the information that molecule $m_B$ has been slowed by the collision.

The total entropy increase is:
\begin{equation}
\Delta S_{\text{total}} = \Delta S_A + \Delta S_B + \Delta S_{\text{correlation}} > 0
\end{equation}

Entropy increases despite reverse heat flow. The Second Law is preserved because it constrains entropy (which increases), not heat flow direction (which can fluctuate at the microscopic level). \qed
\end{proof}

\begin{remark}[Microscopic Fluctuations and Macroscopic Law]
\label{rem:fluctuations_law}
Theorem~\ref{thm:reverse_heat} reveals a subtle but crucial distinction between microscopic dynamics and macroscopic thermodynamics. At the microscopic level, individual molecular collisions can transfer energy in either direction due to thermal fluctuations. The Second Law does not forbid these fluctuations; it only requires that they average out macroscopically to produce net heat flow from hot to cold. At the macroscopic level, the probability of a large-scale fluctuation (e.g., all molecules in a gas spontaneously moving to one half of the container) is exponentially small: $P \sim \exp(-N)$ for $N$ molecules. But the probability of a single-molecule fluctuation (one collision transferring energy cold $\to$ hot) is $P \sim 1/2$ (roughly half of collisions transfer energy in each direction). The Second Law emerges from averaging over $N \gg 1$ molecules, not from constraining individual collisions. Maxwell's demon operates at the single-molecule level, attempting to exploit these fluctuations. But as Theorem~\ref{thm:reverse_heat} proves, even "favorable" fluctuations (reverse heat flow) increase entropy, so the demon gains nothing.
\end{remark}

\subsection{The Fundamental Decoupling}

We now prove the central result: heat transfer and entropy change are fundamentally decoupled at the microscopic level.

\begin{theorem}[Heat-Entropy Decoupling]
\label{thm:heat_entropy_decoupling}
Heat transfer direction and entropy change are fundamentally decoupled in individual molecular collisions. The heat transfer direction can be hot $\to$ cold, cold $\to$ hot, or zero (no net energy transfer), while entropy change is always positive:
\begin{align}
\text{Heat direction} &\in \{\text{hot} \to \text{cold}, \text{cold} \to \text{hot}, \text{zero}\} \label{eq:heat_direction} \\
\text{Entropy change} &> 0 \quad \text{(always)} \label{eq:entropy_always_positive}
\end{align}
for any collision event at the demon's door. The two quantities are independent: knowing the heat flow direction does not determine the entropy change, and knowing the entropy change does not determine the heat flow direction.
\end{theorem}

\begin{proof}
We prove the decoupling by showing that heat transfer and entropy change have different physical origins and obey different constraints.

\textbf{Heat transfer is a kinetic property.}
Heat transfer is determined by the change in kinetic energy of molecules:
\begin{equation}
\Delta Q = \Delta E_A + \Delta E_B = (E_A' - E_A) + (E_B' - E_B)
\end{equation}
where positive $\Delta Q$ indicates energy transfer from A to B (hot $\to$ cold) and negative $\Delta Q$ indicates transfer from B to A (cold $\to$ hot). The sign of $\Delta Q$ depends on the details of the collision: the initial velocities $\mathbf{v}_A$, $\mathbf{v}_B$, the impact parameter, the collision cross-section, and whether the collision is elastic or inelastic. These details are determined by microscopic phase-space trajectories, which fluctuate thermally. Therefore, $\Delta Q$ fluctuates: some collisions give $\Delta Q > 0$ (heat hot $\to$ cold), others give $\Delta Q < 0$ (heat cold $\to$ hot), and some give $\Delta Q \approx 0$ (no net heat transfer).

\textbf{Entropy change is a categorical property.}
Entropy change is determined by the change in phase-lock network structure:
\begin{equation}
\Delta S = k_B \ln \frac{\Omega_{\text{final}}}{\Omega_{\text{initial}}} = k_B \Delta |E|
\end{equation}
where $\Delta |E|$ is the change in the number of phase-lock edges (from Proposition~\ref{prop:entropy_edge_density}). Every collision creates new phase-lock correlations between the colliding molecules, regardless of energy transfer direction. The correlation arises because the post-collision velocities are deterministically related to the pre-collision velocities through collision dynamics. This correlation increases the number of categorical states accessible to the system (the collision creates new equivalence classes of phase-space regions), increasing entropy.

Formally, before the collision, molecules $m_A$ and $m_B$ have independent phases: $\Phi_A$ and $\Phi_B$ are uncorrelated. After the collision, the phases are correlated: $\Phi_A'$ and $\Phi_B'$ satisfy a constraint imposed by the collision (e.g., momentum conservation, energy conservation). This constraint is a phase-lock relationship, represented by an edge $(m_A, m_B)$ in the phase-lock network. The edge addition increases entropy: $\Delta S = k_B \ln[1 + \Delta |E| / |E|] > 0$ for $\Delta |E| > 0$.

\textbf{Independence of heat and entropy.}
The three cases analyzed above demonstrate the independence:
\begin{itemize}
    \item Case 1 (elastic bounce-back): Heat flows hot $\to$ cold ($\Delta Q > 0$), entropy increases ($\Delta S > 0$).
    \item Case 2 (inelastic, cold accelerates): Heat flows hot $\to$ cold ($\Delta Q > 0$), entropy increases ($\Delta S > 0$).
    \item Case 3 (inelastic, cold decelerates): Heat flows cold $\to$ hot ($\Delta Q < 0$), entropy increases ($\Delta S > 0$).
\end{itemize}

In all three cases, entropy increases ($\Delta S > 0$), but heat flow direction varies ($\Delta Q$ can be positive, negative, or zero). The entropy change is independent of heat flow direction because entropy is determined by categorical structure (phase-lock correlations), not by kinetic energy distribution (heat flow).

\textbf{Universality of entropy increase.}
Every collision—regardless of energy flow direction, elasticity, or outcome—creates new phase-lock correlations. The collision event itself is a categorical completion that increases accessible states. The correlation is irreversible: once the collision has occurred, the system "knows" that molecules $m_A$ and $m_B$ have interacted, and this knowledge cannot be erased without further collisions (which create additional correlations, further increasing entropy). Therefore, $\Delta S > 0$ for all collisions.

The decoupling is complete: heat direction is variable and fluctuating (determined by kinetic details), while entropy change is universal and monotonic (determined by categorical structure). \qed
\end{proof}

\begin{corollary}[Second Law Constrains Entropy, Not Heat]
\label{cor:second_law_entropy}
The Second Law of Thermodynamics constrains entropy change ($\Delta S \geq 0$ for isolated systems), not heat flow direction ($\Delta Q$ can have either sign). The traditional formulation "heat cannot spontaneously flow from cold to hot" is a macroscopic consequence of entropy increase, not a microscopic constraint. At the microscopic level, heat can flow in either direction in individual collisions, provided that entropy increases (which it always does).
\end{corollary}

\begin{proof}
The Second Law, in its most fundamental form, states:
\begin{equation}
\Delta S_{\text{universe}} \geq 0
\end{equation}
for any process in an isolated system. This constrains entropy, not energy flow.

The Clausius formulation ("heat cannot spontaneously flow from cold to hot") is derived from the entropy formulation by noting that for a reversible heat transfer $\delta Q$ at temperature $T$, the entropy change is $dS = \delta Q / T$. If heat flows from hot ($T_h$) to cold ($T_c$) with $T_h > T_c$, the entropy change is:
\begin{equation}
\Delta S = -\frac{\delta Q}{T_h} + \frac{\delta Q}{T_c} = \delta Q \left(\frac{1}{T_c} - \frac{1}{T_h}\right) > 0
\end{equation}
since $T_c < T_h$. If heat flows from cold to hot, $\Delta S < 0$, violating the Second Law.

However, this derivation assumes macroscopic heat transfer ($\delta Q \gg k_B T$) and reversibility. At the microscopic level, individual collisions transfer energy $\Delta E \sim k_B T$ (comparable to thermal energy), and the process is irreversible (collisions create correlations). The entropy change from correlation dominates the entropy change from energy transfer:
\begin{equation}
\Delta S_{\text{correlation}} \sim k_B \gg \left|\frac{\Delta E}{T}\right| \sim \frac{k_B T}{T} = k_B
\end{equation}

Therefore, even if energy flows from cold $\to$ to hot (giving a negative $\Delta E / T$ contribution), the total entropy change is positive due to correlation entropy. The Second Law is preserved at the microscopic level, but it constrains entropy (which increases from correlation), not the direction of heat flow (which can fluctuate). \qed
\end{proof}



\subsection{Why Maxwell Conflated Heat and Entropy}

The heat-entropy decoupling was not recognised in Maxwell's time, leading to the conflation that underlies the demon paradox.

\begin{proposition}[Maxwell's Conflation]
\label{prop:maxwell_conflation}
Maxwell implicitly assumed that heat flow and entropy change are equivalent:
\begin{equation}
\Delta Q > 0 \iff \Delta S > 0
\label{eq:maxwell_conflation}
\end{equation}
That is, heat flowing from hot to cold implies an increase in entropy, and an increase in entropy implies heat flowing from hot to cold. This assumption is correct macroscopically in the thermodynamic limit but fails at the microscopic level, where individual collisions exhibit thermal fluctuations.
\end{proposition}

\begin{proof}
The macroscopic Second Law, as formulated in the 19th century, relates heat flow to entropy change through:
\begin{equation}
dS \geq \frac{\delta Q}{T}
\label{eq:clausius_inequality}
\end{equation}

This is the Clausius inequality, which states that entropy change is at least as large as the heat absorbed divided by temperature, with equality for reversible processes. For macroscopic systems in the thermodynamic limit ($N \to \infty$ molecules), this inequality becomes an effective equivalence: processes that transfer heat from hot to cold increase entropy, and processes that increase entropy transfer heat from hot to cold (on average).

Maxwell, reasoning at the single-molecule level, assumed this equivalence would hold. He imagined a demon that could "sort" molecules by velocity, allowing fast molecules to pass from cold to hot and slow molecules from hot to cold, thereby transferring heat from cold to hot without doing work. If heat flow and entropy change are equivalent (equation~\eqref{eq:maxwell_conflation}), then reverse heat flow would imply entropy decrease, violating the Second Law.

However, the equivalence fails at the microscopic level. Individual collisions can transfer energy in either direction due to thermal fluctuations. The Clausius inequality~\eqref{eq:clausius_inequality} applies only statistically, averaged over many collisions. At the single-molecule level, the inequality becomes:
\begin{equation}
\langle \Delta S \rangle \geq \left\langle \frac{\Delta Q}{T} \right\rangle
\end{equation}
where angle brackets denote ensemble average. Individual collisions can have $\Delta Q < 0$ (reverse heat flow) while still having $\Delta S > 0$ (entropy increase), provided that the average over many collisions satisfies the inequality.

Maxwell measured the wrong quantity: he focused on heat flow (kinetic energy transfer) instead of entropy (categorical structure). At the microscopic level, heat flow fluctuates and can be reversed, but entropy increases monotonically through categorical completion. The demon's strategy of sorting by velocity (to control heat flow) does not control entropy, so the Second Law is preserved. \qed
\end{proof}

\begin{remark}[Historical Context]
\label{rem:historical_context}
Maxwell formulated the demon paradox in 1867, before the statistical mechanical foundations of thermodynamics were fully developed. Boltzmann's statistical interpretation of entropy ($S = k_B \ln \Omega$) was published in 1877, and the connexion between entropy and information was not established until the 20th century (Shannon 1948, Landauer 1961). In Maxwell's time, entropy was understood primarily through its relationship to heat flow (the Clausius formulation), not through its relationship to microscopic states (the Boltzmann formulation). The conflation of heat and entropy was therefore natural given the conceptual tools available. The categorical framework, which distinguishes heat (kinetic property) from entropy (categorical property), provides a resolution that was not accessible to Maxwell.
\end{remark}

\subsection{Heat is Statistical, Entropy is Categorical}

We now formalise the distinction between heat and entropy in terms of their physical and mathematical properties.

\begin{definition}[Heat as Statistical Average]
\label{def:heat_statistical}
Heat flow $Q$ is the statistical average of energy transfer over many molecular collisions:
\begin{equation}
Q = \lim_{N \to \infty} \frac{1}{N} \sum_{i=1}^{N} \Delta E_i
\label{eq:heat_average}
\end{equation}
where $\Delta E_i$ is the energy transferred in the $i$-th collision. Individual $\Delta E_i$ can be positive (energy flows hot $\to$ cold) or negative (energy flows cold $\to$ hot); only the average $Q$ has thermodynamic significance. Heat is an emergent statistical property that arises from averaging over microscopic fluctuations.
\end{definition}

\begin{definition}[Entropy as Categorical Completion]
\label{def:entropy_categorical}
Entropy change $\Delta S$ is the categorical advancement through phase-lock network densification:
\begin{equation}
\Delta S = k_B \ln \frac{\Omega_{\text{final}}}{\Omega_{\text{initial}}} = k_B \ln \frac{\Omega_{\text{initial}} + \Delta \Omega}{\Omega_{\text{initial}}} \approx k_B \frac{\Delta \Omega}{\Omega_{\text{initial}}}
\label{eq:entropy_categorical}
\end{equation}
where $\Omega$ counts accessible categorical states (equivalence classes of phase-space regions determined by phase-lock relationships). Entropy is always non-negative for spontaneous processes because categorical completion is irreversible: once a categorical state is completed, it cannot be uncompleted (Axiom~\ref{axiom:categorical_irreversibility}). Entropy is a fundamental categorical property that increases monotonically through network densification.
\end{definition}

\begin{theorem}[Heat Statistical, Entropy Fundamental]
\label{thm:heat_statistical}
Heat is an emergent statistical property that arises from summing over fluctuating microscopic energy transfers, while entropy is a fundamental categorical property that increases monotonically through categorical completion. The Second Law constrains entropy (fundamental), not heat (statistical). Heat obeys the Second Law only on average (in the thermodynamic limit), while entropy obeys the Second Law exactly (for every individual process).
\end{theorem}

\begin{proof}
We prove each statement in turn.

\textbf{Heat is emergent and statistical.}
Heat is defined macroscopically as the energy transferred between systems due to temperature difference. Microscopically, heat is the sum of energy transfers in individual molecular collisions:
\begin{equation}
Q = \sum_{i=1}^{N} \Delta E_i
\end{equation}

Each $\Delta E_i$ fluctuates: some collisions transfer energy hot $\to$ cold ($\Delta E_i > 0$), others transfer energy cold $\to$ hot ($\Delta E_i < 0$). The distribution of $\Delta E_i$ is determined by the Maxwell-Boltzmann velocity distribution and collision dynamics. For a system with a temperature difference of $\Delta T = T_h - T_c$, the average energy transfer per collision is:
\begin{equation}
\langle \Delta E \rangle \propto \Delta T
\end{equation}

The total heat transfer is:
\begin{equation}
Q = N \langle \Delta E \rangle \propto N \Delta T
\end{equation}

For large $N$, the fluctuations average out: $Q / N \to \langle \Delta E \rangle$ by the law of large numbers. But for small $N$ (or individual collisions, $N = 1$), fluctuations dominate: $\Delta E_i$ can have either sign with comparable probability. Heat is therefore an emergent statistical property that becomes well-defined only in the thermodynamic limit $N \to \infty$.

\begin{figure*}[htbp]
\centering
\includegraphics[width=0.95\textwidth]{figures/panel_arg3_retrieval_paradox.png}
\caption{\textbf{Argument 3: The Retrieval Paradox—Velocity-Based Sorting is Self-Defeating: Thermal Equilibration is Faster.}
\textbf{(A)} Timescale hierarchy: collisions happen before sorting. Horizontal bar chart on logarithmic scale (x-axis: $10^{-10}$ to $10^{-3}$ seconds) showing four processes from top to bottom: (1) Sorting Complete (gray bar, $\approx 1 \times 10^{-3}$ s = 1 ms), (2) Door Operation (orange bar, $\approx 1 \times 10^{-6}$ s = 1 μs), (3) Velocity Measurement (yellow bar, $\approx 1 \times 10^{-8}$ s = 10 ns), (4) Molecular Collision (red bar, $\approx 1 \times 10^{-10}$ s = 0.1 ns, with annotation "Collisions happen before sorting!"). The five-order-of-magnitude separation between collision timescale ($10^{-10}$ s) and sorting completion ($10^{-3}$ s) means that $\sim 10^7$ collisions occur during one sorting cycle. Each collision randomizes velocities, erasing the demon's sorting progress. The annotation emphasizes the causal impossibility: by the time sorting completes, the system has already thermalized $10^7$ times. This timescale separation makes velocity-based sorting fundamentally self-defeating.
\textbf{(B)} Velocity redistribution: sorted state returns to Maxwell-Boltzmann. Overlapping histograms showing probability density (y-axis, 0.0–0.7) versus velocity (x-axis, $-4$ to $+4$). Three distributions: Initial (MB) (gray bars, broad Gaussian centered at $v = 0$), "Sorted" (pink bars, narrow double-peaked distribution with peaks at $v \approx -0.5$ and $v \approx +1$), and After $\tau_{\text{collision}}$ (green bars, broad Gaussian matching initial). The initial Maxwell-Boltzmann distribution (gray) represents thermal equilibrium with width $\sigma \approx 1.5$. The sorted distribution (pink) shows the demon's attempt to separate fast and slow molecules, creating artificial bimodal structure with reduced entropy. After one collision time (green), the distribution returns to the original Maxwell-Boltzmann form—the sorting has been completely erased. The green bars overlay the gray bars almost perfectly, demonstrating that $\tau_{\text{collision}}$ is the relaxation time for velocity distribution. Any deviation from Maxwell-Boltzmann decays exponentially with time constant $\tau_{\text{collision}}$, making sustained sorting impossible.
\textbf{(C)} Sorting cannot be maintained: repeated attempts, repeated failures. Time series of fast/total ratio (y-axis, 0.3–1.0) versus time steps (x-axis, 0–200). Dark teal line shows actual ratio fluctuating around equilibrium. Red dashed horizontal line at 0.5 marks equilibrium (50\%). Four sharp spikes to $\approx 0.8$–0.85 occur at steps $\approx 25, 50, 75, 125$, each labeled "Attempt" with vertical gray dashed line. Each spike represents a demon sorting intervention that temporarily increases the fast fraction to $\sim 80$–85\%. However, each spike decays rapidly back to 0.5 within $\sim 20$ steps, following exponential relaxation. The pattern repeats four times with identical behavior: spike → decay → equilibrium. The consistency of the decay demonstrates that thermalization operates with fixed time constant regardless of sorting strength. The demon cannot maintain sorting because the relaxation rate ($\sim 1/20$ per step) exceeds any achievable sorting rate. This is the retrieval paradox: the demon can create order temporarily but cannot retrieve or maintain it.
\textbf{(D)} Timescale separation: kinetic (fast) versus categorical (slow). Two-curve plot showing state evolution (y-axis, 0.2–0.8) versus normalized time (x-axis, 0.0–1.0). Red oscillating curve labeled "Kinetic (fast)" shows rapid oscillations with period $\approx 0.1$ and amplitude $\approx 0.3$, completing $\sim 10$ cycles. Purple smooth curve labeled "Categorical (slow)" shows gradual monotonic increase from $\approx 0.2$ to $\approx 0.8$, with no oscillations. Pink shaded region indicates categorical timescale regime. The red curve represents kinetic properties (velocity distribution) that oscillate on collision timescales—each oscillation corresponds to sorting attempt and thermal relaxation. The purple curve represents categorical properties (network topology, spatial arrangement) that evolve smoothly on much longer timescales. The separation demonstrates that kinetic and categorical dynamics operate on different timescales: kinetic processes (velocity sorting, thermalization) occur on $\tau_{\text{collision}} \sim 10^{-10}$ s, while categorical processes (spatial rearrangement, network reconfiguration) occur on $\tau_{\text{categorical}} \sim 10^{-3}$ s or longer. The demon operates on kinetic timescales but entropy is determined by categorical timescales. This timescale mismatch is fundamental: the demon manipulates fast variables (velocity) but the second law protects slow variables (configuration). The retrieval paradox arises because kinetic order decays faster than it can be exploited for categorical work.}
\label{fig:arg3_retrieval_paradox_v2}
\end{figure*}


\textbf{Entropy is fundamental and categorical.}
Entropy is defined microscopically through the Boltzmann formula $S = k_B \ln \Omega$, where $\Omega$ is the number of accessible microstates. In the categorical framework, $\Omega$ is the number of accessible categorical states (equivalence classes determined by phase-lock relationships). Each collision creates new phase-lock correlations, increasing $\Omega$:
\begin{equation}
\Omega_{\text{after}} = \Omega_{\text{before}} + \Delta \Omega
\end{equation}
where $\Delta \Omega > 0$ because the collision creates new categorical distinctions (states with the correlation versus states without).

The entropy change is:
\begin{equation}
\Delta S = k_B \ln \frac{\Omega_{\text{after}}}{\Omega_{\text{before}}} = k_B \ln \left(1 + \frac{\Delta \Omega}{\Omega_{\text{before}}}\right) > 0
\end{equation}

This increase is exact and non-fluctuating: every collision increases $\Omega$, so every collision increases $S$. There are no "negative entropy fluctuations" at the microscopic level because categorical completion is irreversible (Axiom~\ref{axiom:categorical_irreversibility}). Entropy is therefore a fundamental property that increases monotonically for every individual process, not just on average.

\textbf{Second Law constrains entropy, not heat.}
The Second Law states $\Delta S \geq 0$ for isolated systems. This constrains entropy directly. Heat is constrained only indirectly through its relationship to entropy:
\begin{equation}
\Delta S \geq \frac{Q}{T}
\end{equation}

For macroscopic processes with $Q \gg k_B T$, this inequality effectively constrains heat flow direction: $Q$ must be positive (hot $\to$ cold) for entropy to increase. But for microscopic processes with $Q \sim k_B T$, the inequality is weak: $\Delta S$ can be positive even if $Q$ is negative, provided that other contributions to entropy (correlation, categorical completion) dominate.

The Second Law is fundamentally a constraint on entropy (categorical property), not on heat (statistical property). Heat obeys the Second Law only on average because heat is a statistical average of fluctuating microscopic transfers.

\textbf{Heat obeys Second Law only on average.}
For a single collision, heat transfer $\Delta E$ can be positive or negative with comparable probability. The Second Law does not forbid $\Delta E < 0$ (reverse heat flow) because entropy still increases through correlation: $\Delta S = \Delta S_{\text{correlation}} + \Delta E / T > 0$ even if $\Delta E < 0$, provided $\Delta S_{\text{correlation}} > |\Delta E / T|$.

For $N$ collisions, the total heat transfer is $Q = \sum_{i=1}^{N} \Delta E_i$. By the central limit theorem, $Q / N \to \langle \Delta E \rangle$ as $N \to \infty$, with fluctuations of order $\sqrt{N}$. For large $N$, the probability of $Q < 0$ (net reverse heat flow) is exponentially small: $P(Q < 0) \sim \exp(-N)$. The Second Law emerges statistically: heat flows hot $\to$ cold on average, with probability approaching unity as $N \to \infty$.

\textbf{Entropy obeys Second Law exactly.}
For every individual collision, entropy increases: $\Delta S_i > 0$. For $N$ collisions, the total entropy change is $\Delta S = \sum_{i=1}^{N} \Delta S_i > 0$ (sum of positive terms is positive). There are no fluctuations: entropy increases monotonically, not just on average. The Second Law is exact at all scales, from single collisions to macroscopic processes. \qed
\end{proof}

\begin{corollary}[Demon Measures Wrong Quantity]
\label{cor:demon_wrong_quantity}
Maxwell's demon measures and manipulates heat (velocity, kinetic energy), which is a statistical property that fluctuates at the microscopic level. The Second Law constrains entropy (categorical structure), which is a fundamental property that increases monotonically. The demon's strategy is misdirected: it attacks a fluctuating statistical property while the Second Law protects a monotonic fundamental property.
\end{corollary}

\subsection{Implications for the Demon}

The heat-entropy decoupling has profound implications for Maxwell's demon.

\begin{corollary}[Demon's Irrelevance to Heat Direction]
\label{cor:demon_irrelevant}
The demon's door operation produces an increase in entropy regardless of the direction of heat flow. Even if a particular collision transfers heat from cold to hot (reverse direction, apparently favorable for the demon), total entropy still increases through categorical completion and phase-lock correlation. The demon cannot exploit individual heat flow fluctuations because entropy does not fluctuate—it increases monotonically through categorical structure evolution.
\end{corollary}

\begin{proof}
The demon operates at the single-molecule level, where heat flow is fluctuating (Definition~\ref{def:heat_statistical}). Some door operations will result in collisions that transfer heat from hot $\to$ to cold (Cases 1 and 2), while others result in collisions that transfer heat from cold $\to$ to hot (Case 3). The demon cannot predict which outcome will occur because collision dynamics depend on microscopic phase-space details (impact parameter, relative velocity, molecular orientations) that are beyond the demon's knowledge or control.

Even if the demon could predict outcomes and selectively allow only "favorable" collisions (those transferring heat from cold $\to$ hot), it would gain nothing. From Theorem~\ref{thm:reverse_heat}, even favorable collisions increase entropy through categorical completion:
\begin{equation}
\Delta S_{\text{total}} = \Delta S_A + \Delta S_B + \Delta S_{\text{correlation}} > 0
\end{equation}

The entropy increase arises from phase-lock correlations created by the collision, which are independent of energy transfer direction. The demon cannot avoid creating correlations because correlations are inevitable consequences of molecular interactions.

Therefore, the demon's strategy is irrelevant: regardless of which collisions it allows, entropy increases. The demon cannot exploit heat flow fluctuations because the quantity constrained by the Second Law (entropy) is not fluctuating—it increases monotonically through categorical completion. \qed
\end{proof}

\begin{theorem}[Complete Demon Defeat]
\label{thm:complete_defeat}
The demon is defeated at every level of analysis: at the level of individual collisions, entropy increases regardless of energy transfer direction (Theorem~\ref{thm:reverse_heat}); at the level of statistical averages, net heat flows hot $\to$ cold (Second Law on average, Theorem~\ref{thm:heat_statistical}); and at the level of categorical structure, every operation advances categorical completion (Theorem~\ref{thm:categorical_cascade}). The demon cannot violate the Second Law because it cannot control individual collision outcomes (quantum and thermal uncertainty prevent prediction), even "favorable" outcomes (heat cold $\to$ hot) increase entropy through categorical completion, and the quantity it attempts to manipulate (heat) is not the quantity constrained by the Second Law (entropy).
\end{theorem}

\begin{proof}
We prove defeat at each level.

\textbf{Level 1: Individual collisions.}
From Theorem~\ref{thm:reverse_heat}, every collision increases entropy regardless of energy transfer direction. Cases 1, 2, and 3 exhaust the possibilities (heat flows hot $\to$ cold, cold $\to$ hot, or zero), and in all cases $\Delta S > 0$. The demon cannot find a collision that decreases entropy.

\textbf{Level 2: Statistical averages.}
From Theorem~\ref{thm:heat_statistical}, heat is a statistical average over many collisions. In the thermodynamic limit ($N \to \infty$ collisions), the average heat flow is hot $\to$ cold with probability approaching unity. The demon operates at finite $N$ (it allows molecules through one at a time), so fluctuations occur. But these fluctuations average out: over many door operations, net heat flows hot $\to$ cold, consistent with the Second Law.

\textbf{Level 3: Categorical structure.}
From Theorem~\ref{thm:categorical_cascade}, every door operation advances categorical completion. Opening the door allows a molecule to transit, which creates new phase-lock relationships (the transiting molecule interacts with molecules in the receiving container). These relationships densify the phase-lock network, increasing entropy (Proposition~\ref{prop:entropy_edge_density}). The demon cannot avoid categorical completion because it is a deterministic consequence of molecular interactions.

\textbf{Why the demon cannot violate the Second Law:}

First, the demon cannot control individual collision outcomes. Collision dynamics are determined by microscopic phase-space trajectories (positions, velocities, orientations of molecules at the instant of collision). These trajectories are subject to quantum uncertainty (Heisenberg uncertainty principle prevents simultaneous precise knowledge of position and momentum) and thermal fluctuations (molecular velocities are distributed according to Maxwell-Boltzmann statistics, which are inherently probabilistic). The demon would need to know the phase-space coordinates of all molecules to arbitrary precision to predict collision outcomes, which is impossible.

Second, even if the demon could predict outcomes and selectively allow only "favorable" collisions (those transferring heat cold $\to$ hot), these collisions still increase entropy. From Theorem~\ref{thm:reverse_heat}, reverse heat flow does not imply entropy decrease. The entropy increase from categorical completion ($\Delta S_{\text{categorical}}$) dominates the entropy decrease from reverse heat flow ($\Delta S_{\text{heat}} = -\Delta Q / T < 0$ for $\Delta Q < 0$):
\begin{equation}
\Delta S_{\text{total}} = \Delta S_{\text{categorical}} + \Delta S_{\text{heat}} > 0
\end{equation}
because $\Delta S_{\text{categorical}} \sim k_B$ (order of Boltzmann constant) while $\Delta S_{\text{heat}} \sim \Delta Q / T \sim k_B T / T = k_B$ (same order), but $\Delta S_{\text{categorical}}$ is always positive while $\Delta S_{\text{heat}}$ can be negative. The categorical term dominates on average.

Third, the demon manipulates the wrong quantity. It measures velocity (kinetic energy) and attempts to control heat flow (energy transfer). But the Second Law constrains entropy (categorical structure), not heat. Heat is a statistical property that fluctuates; entropy is a fundamental property that increases monotonically. The demon's strategy is misdirected: it attacks a fluctuating statistical property while the Second Law protects a monotonic fundamental property.

The demon is completely defeated: it cannot predict outcomes, cannot exploit favorable outcomes (they still increase entropy), and cannot manipulate the quantity constrained by the Second Law. \qed
\end{proof}

\subsection{The Insight Formalized}

The fundamental insight can be summarized as follows:

\begin{equation}
\boxed{
\begin{aligned}
\text{Heat} &: \text{energy accounting (kinetic property, can flow either way)} \\
\text{Entropy} &: \text{categorical completion (categorical property, always increases)}
\end{aligned}
}
\label{eq:heat_entropy_summary}
\end{equation}

Maxwell asked: "Can the demon sort molecules to transfer heat from cold to hot, violating the Second Law?"

The answer is: It doesn't matter. Even if a particular collision transfers heat from cold to hot (reverse direction), entropy still increases through categorical completion. The demon is defeated not by heat flow constraints (which can be violated at the microscopic level) but by the inexorable advance of categorical completion (which cannot be reversed).

The Second Law, properly understood, does not state "heat cannot flow from cold to hot" (which is false at the microscopic level). It states "entropy cannot decrease in isolated systems" (which is true at all levels). Heat is a statistical consequence of entropy increase, not the fundamental constraint.

\begin{remark}[Historical Irony]
\label{rem:historical_irony}
The Second Law was historically formulated in terms of heat: the Clausius statement ("heat cannot spontaneously flow from cold to hot") and the Kelvin statement ("no process can convert heat entirely to work in a cycle"). These formulations, while correct macroscopically, obscure the microscopic reality: the Second Law constrains entropy (categorical structure), not heat (energy flow). The heat formulations are consequences of entropy increase, not the foundation.

Maxwell's demon exploits this historical conflation. By framing the paradox in terms of heat ("can the demon transfer heat from cold $\to$ to hot?"), Maxwell created a puzzle that seemed to require an information-theoretic resolution (Landauer's principle, erasure costs). Framed in terms of entropy ("can the demon decrease entropy?"), the paradox dissolves immediately: entropy increases through categorical completion regardless of the direction of heat flow, so the demon cannot decrease entropy even if it can (locally, temporarily) reverse heat flow.

The categorical framework reveals that the demon's strategy is fundamentally misdirected: it attacks heat (the wrong target) while entropy (the actual target of the Second Law) is immune to the attack. This is perhaps the deepest resolution of Maxwell's demon: the demon is defeated not by information theory, not by measurement costs, not by erasure requirements, but by the simple fact that it is trying to violate a constraint on entropy by manipulating heat, and the two quantities are decoupled at the microscopic level.
\end{remark}

\subsection{Summary}

The analysis of door collisions reveals the fundamental decoupling of heat and entropy, providing the most direct resolution of Maxwell's demon paradox.

Heat can flow in either direction in individual collisions due to thermal fluctuations at the microscopic level. Cases 1 and 2 demonstrate heat flowing hot $\to$ cold (standard direction), while Case 3 demonstrates heat flowing cold $\to$ hot (reverse direction). The direction is determined by collision dynamics (impact parameter, relative velocity, elasticity), which fluctuate thermally.

Entropy always increases through categorical completion, regardless of heat flow direction. Every collision creates phase-lock correlations between the colliding molecules, enriching categorical structure and increasing the number of accessible categorical states. This increase is monotonic and non-fluctuating because categorical completion is irreversible (Axiom~\ref{axiom:categorical_irreversibility}).

The Second Law constrains entropy (categorical property), not heat direction (kinetic property). The traditional formulation in terms of heat ("heat cannot spontaneously flow from cold to hot") is a macroscopic consequence that emerges from averaging over many collisions, not a microscopic constraint on individual collisions.

Heat obeys thermodynamic constraints only statistically, in the thermodynamic limit ($N \to \infty$ molecules). Individual collisions can violate the heat formulation of the Second Law (transferring heat cold $\to$ hot), but these violations average out macroscopically. The probability of a macroscopic violation (net heat flow cold $\to$ hot over many collisions) is exponentially small: $P \sim \exp(-N)$.

The demon measures and manipulates heat—the wrong quantity. It attempts to exploit thermal fluctuations in heat flow to create a temperature gradient without doing work. But the Second Law does not constrain heat flow at the microscopic level; it constrains entropy, which is immune to the demon's manipulations because entropy increases through categorical completion regardless of heat flow direction.

Entropy, the actually conserved quantity, is immune to the demon's strategy. The demon cannot decrease entropy because every door operation creates phase-lock correlations (categorical completion), which increase entropy. Even if the demon succeeds in transferring heat cold $\to$ hot in individual collisions, the entropy increase from categorical completion dominates the entropy decrease from reverse heat flow, preserving the Second Law.

This decoupling provides the most fundamental resolution of Maxwell's demon: the demon attacks a statistical emergent property (heat) while the Second Law protects a categorical fundamental property (entropy). The demon's entire strategy is misdirected, and the paradox dissolves once we recognize that heat and entropy are decoupled at the microscopic level.


%==============================================================================
\section{Velocity-Temperature Non-Correspondence: The Distribution Overlap}
\label{sec:velocity_overlap}
%==============================================================================

A deeper problem emerges from examining the statistical nature of temperature itself. Temperature is not a property of individual molecules but rather a statistical property of ensembles, defined as the mean kinetic energy of a collection of molecules. This statistical definition implies that molecular velocities follow a Maxwell-Boltzmann distribution, which has a crucial property: when two containers have similar (or even moderately different) temperatures, their velocity distributions overlap significantly. The same velocity appears in both distributions, but with different statistical meaning—a velocity that is "above average" (fast) in the colder container may be "below average" (slow) in the hotter container. This distribution overlap renders velocity-based sorting fundamentally incoherent: the demon cannot determine whether a molecule is "hot" or "cold" from its velocity alone because the meaning of a velocity depends on the ensemble context. A molecule that appears "hot" in one container becomes "cold" when transferred to another container, even though its velocity is unchanged. The demon's strategy of sorting by velocity to create a temperature difference is therefore conceptually incoherent—it attempts to sort by a property (temperature) that molecules do not individually possess, using a measurement (velocity) that does not determine the property even statistically. This section establishes the velocity-temperature non-correspondence and proves that the demon cannot sort by temperature because temperature is ensemble-relative, not molecule-intrinsic.

\subsection{The Maxwell-Boltzmann Distribution}

We begin by reviewing the Maxwell-Boltzmann distribution, which describes the statistical distribution of molecular speeds in an ideal gas at thermal equilibrium.

\begin{definition}[Maxwell-Boltzmann Distribution]
\label{def:maxwell_boltzmann}
The probability density function for molecular speeds in an ideal gas at temperature $T$ is:
\begin{equation}
f(v; T) = 4\pi n \left(\frac{m}{2\pi k_B T}\right)^{3/2} v^2 \exp\left(-\frac{mv^2}{2k_B T}\right)
\label{eq:maxwell_boltzmann}
\end{equation}
where $v = |\mathbf{v}|$ is the speed (magnitude of velocity), $m$ is the molecular mass, $k_B$ is Boltzmann's constant, $T$ is the absolute temperature, and $n$ is the number density. The distribution gives the probability density for finding a molecule with speed in the interval $[v, v + dv]$. The normalization condition is $\int_0^\infty f(v; T) \, dv = n$ (the integral over all speeds equals the total number density).
\end{definition}

\begin{remark}[Derivation and Assumptions]
\label{rem:maxwell_boltzmann_derivation}
The Maxwell-Boltzmann distribution is derived from statistical mechanics under the assumptions of an ideal gas (no intermolecular interactions except instantaneous elastic collisions), thermal equilibrium (the system has relaxed to a stationary distribution), and classical mechanics (quantum effects are negligible). The derivation proceeds by maximizing the entropy $S = -k_B \sum_i p_i \ln p_i$ subject to constraints of fixed total energy $E = \sum_i p_i E_i$ and fixed total probability $\sum_i p_i = 1$, using Lagrange multipliers. The resulting distribution is exponential in energy: $p_i \propto \exp(-E_i / k_B T)$. For translational kinetic energy $E = (1/2) m v^2$, this gives the Maxwell-Boltzmann distribution~\eqref{eq:maxwell_boltzmann}.
\end{remark}

The Maxwell-Boltzmann distribution has several characteristic speeds that quantify the typical velocities in the ensemble.

\begin{definition}[Most Probable Speed]
\label{def:most_probable_speed}
The \textbf{most probable speed} $v_p$ is the speed at which the probability density $f(v; T)$ is maximized. It is found by setting $df/dv = 0$:
\begin{equation}
v_p = \sqrt{\frac{2k_B T}{m}}
\label{eq:most_probable_speed}
\end{equation}
This is the speed at which the largest number of molecules are found (the peak of the distribution).
\end{definition}

\begin{definition}[Mean Speed]
\label{def:mean_speed}
The \textbf{mean speed} $\langle v \rangle$ is the average speed over the entire distribution:
\begin{equation}
\langle v \rangle = \int_0^\infty v f(v; T) \, dv / n = \sqrt{\frac{8k_B T}{\pi m}}
\label{eq:mean_speed}
\end{equation}
This is the arithmetic mean of molecular speeds, weighted by the probability density.
\end{definition}

\begin{definition}[Root-Mean-Square Speed]
\label{def:rms_speed}
The \textbf{root-mean-square speed} $v_{\text{rms}}$ is the square root of the mean squared speed:
\begin{equation}
v_{\text{rms}} = \sqrt{\langle v^2 \rangle} = \sqrt{\int_0^\infty v^2 f(v; T) \, dv / n} = \sqrt{\frac{3k_B T}{m}}
\label{eq:rms_speed}
\end{equation}
The rms speed is related to temperature through the equipartition theorem: $\langle E_{\text{kin}} \rangle = (1/2) m \langle v^2 \rangle = (3/2) k_B T$, giving $v_{\text{rms}} = \sqrt{3k_B T / m}$.
\end{definition}

\begin{remark}[Ordering of Characteristic Speeds]
\label{rem:speed_ordering}
The three characteristic speeds are ordered as:
\begin{equation}
v_p < \langle v \rangle < v_{\text{rms}}
\end{equation}
with numerical ratios:
\begin{equation}
v_p : \langle v \rangle : v_{\text{rms}} = \sqrt{2} : \sqrt{8/\pi} : \sqrt{3} \approx 1.414 : 1.596 : 1.732
\end{equation}
The most probable speed is the smallest (the peak of the distribution), the mean speed is intermediate, and the rms speed is the largest (because squaring emphasizes high speeds). For nitrogen at 300 K, these speeds are approximately $v_p \approx 422$ m/s, $\langle v \rangle \approx 476$ m/s, and $v_{\text{rms}} \approx 517$ m/s.
\end{remark}

Crucially, the Maxwell-Boltzmann distribution has tails extending to both low and high velocities. At any temperature, some molecules move slowly (near zero velocity) and some move rapidly (far above the mean). The distribution never vanishes for any positive velocity.

\begin{proposition}[Distribution Tails]
\label{prop:distribution_tails}
The Maxwell-Boltzmann distribution has support on the entire positive real line: $f(v; T) > 0$ for all $v > 0$ and any $T > 0$. The distribution has exponential tails: as $v \to \infty$, $f(v; T) \sim v^2 \exp(-mv^2 / 2k_B T) \to 0$ exponentially fast, but never reaches exactly zero. As $v \to 0$, $f(v; T) \sim v^2 \to 0$ (the distribution vanishes quadratically at zero speed due to the $v^2$ prefactor, which arises from the spherical volume element in velocity space).
\end{proposition}

\begin{proof}
The distribution function is:
\begin{equation}
f(v; T) = C v^2 \exp\left(-\frac{mv^2}{2k_B T}\right)
\end{equation}
where $C = 4\pi n (m / 2\pi k_B T)^{3/2} > 0$ is a positive constant. For any $v > 0$ and $T > 0$, both factors are positive: $v^2 > 0$ and $\exp(-mv^2 / 2k_B T) > 0$ (the exponential function is always positive). Therefore, $f(v; T) > 0$ for all $v > 0$.

The exponential tail dominates at large $v$: as $v \to \infty$, the exponential $\exp(-mv^2 / 2k_B T)$ decreases faster than any polynomial $v^n$ increases, so $f(v; T) \to 0$. But the exponential is never exactly zero for finite $v$, so the distribution has infinite support (it is positive everywhere on $(0, \infty)$).

At $v = 0$, the $v^2$ prefactor causes the distribution to vanish: $f(0; T) = 0$. This is physically reasonable: the probability of finding a molecule with exactly zero velocity is zero (it is a measure-zero event in velocity space). The probability of finding a molecule with speed in a small interval $[0, \epsilon]$ is $\int_0^\epsilon f(v; T) \, dv \sim \epsilon^3 \to 0$ as $\epsilon \to 0$. \qed
\end{proof}

\subsection{Distribution Overlap Between Containers}

Consider two containers A and B at different temperatures $T_A < T_B$. The key insight is that their velocity distributions overlap significantly: there is a large range of velocities that appear in both distributions with non-zero probability.

\begin{theorem}[Distribution Overlap]
\label{thm:distribution_overlap}
For any two temperatures $T_A < T_B$ (both positive), the velocity distributions overlap completely. Formally, for any velocity $v > 0$, both distributions have positive probability density:
\begin{equation}
f_A(v; T_A) > 0 \quad \text{and} \quad f_B(v; T_B) > 0
\label{eq:complete_overlap}
\end{equation}
The overlap region is the entire positive real line: every velocity that exists in one container also exists in the other, with positive (though possibly small) probability.
\end{theorem}

\begin{proof}
From Proposition~\ref{prop:distribution_tails}, the Maxwell-Boltzmann distribution has support on $(0, \infty)$ for any positive temperature. For container A at temperature $T_A$:
\begin{equation}
f_A(v; T_A) = C_A v^2 \exp\left(-\frac{mv^2}{2k_B T_A}\right) > 0 \quad \forall v > 0
\end{equation}
where $C_A = 4\pi n_A (m / 2\pi k_B T_A)^{3/2} > 0$.

Similarly, for container B at temperature $T_B$:
\begin{equation}
f_B(v; T_B) = C_B v^2 \exp\left(-\frac{mv^2}{2k_B T_B}\right) > 0 \quad \forall v > 0
\end{equation}
where $C_B = 4\pi n_B (m / 2\pi k_B T_B)^{3/2} > 0$.

Both distributions are strictly positive for all $v > 0$, regardless of the temperature difference $T_B - T_A$. Therefore, the overlap region is $(0, \infty)$—the entire positive real line. Every velocity has positive probability in both containers.

The overlap is complete, not partial. There is no velocity threshold $v_{\text{threshold}}$ such that velocities below the threshold exist only in A and velocities above exist only in B. Even velocities far below the mean of A (very slow molecules) have positive probability in B, and velocities far above the mean of B (very fast molecules) have positive probability in A. The probabilities may be exponentially small (due to the exponential tails), but they are never exactly zero. \qed
\end{proof}

\begin{corollary}[Complete Overlap]
\label{cor:complete_overlap}
The velocity distributions of any two containers at positive temperatures overlap completely. There is no velocity that uniquely identifies a molecule as belonging to one container or the other based on velocity alone. Every velocity is compatible with both containers, differing only in probability (how common that velocity is in each container).
\end{corollary}

\begin{remark}[Quantitative Overlap]
\label{rem:quantitative_overlap}
While the overlap is complete (all velocities appear in both distributions), the degree of overlap (how much the distributions coincide) depends on the temperature ratio $T_B / T_A$. For small temperature differences ($T_B / T_A \approx 1$), the distributions are nearly identical, and the overlap is nearly perfect. For large temperature differences ($T_B / T_A \gg 1$), the distributions are well-separated, and the overlap is small (most velocities in A are below most velocities in B). However, even for large temperature differences, the overlap is non-zero: there is always a range of velocities where both distributions have significant probability.

Quantitatively, the overlap can be measured by the Bhattacharyya coefficient:
\begin{equation}
BC = \int_0^\infty \sqrt{f_A(v; T_A) f_B(v; T_B)} \, dv
\end{equation}
which ranges from 0 (no overlap) to 1 (complete coincidence). For Maxwell-Boltzmann distributions, the Bhattacharyya coefficient can be computed analytically, giving $BC = (2\sqrt{T_A T_B} / (T_A + T_B))^{3/2}$. For $T_B / T_A = 1.1$ (10\% temperature difference), $BC \approx 0.985$ (98.5\% overlap). For $T_B / T_A = 2$ (100\% temperature difference), $BC \approx 0.707$ (70.7\% overlap). Even for large temperature differences, the overlap remains substantial.
\end{remark}

\subsection{Context-Dependent Velocity Meaning}

The critical insight is that a velocity's statistical meaning (whether it is "fast" or "slow") depends on the ensemble context (which distribution it is part of). The same velocity can be "above average" in one ensemble and "below average" in another.

\begin{definition}[Velocity Percentile]
\label{def:velocity_percentile}
For a molecule with speed $v$ in an ensemble at temperature $T$, the \textbf{velocity percentile} $P_T(v)$ is the cumulative probability up to speed $v$:
\begin{equation}
P_T(v) = \int_0^v f(v'; T) \, dv' / n
\label{eq:velocity_percentile}
\end{equation}
This measures what fraction of the ensemble has speed less than or equal to $v$. The percentile ranges from 0 (slowest molecule) to 1 (fastest molecule). A molecule at the 50th percentile ($P_T(v) = 0.5$) has speed near the median (approximately equal to the mean speed $\langle v \rangle$ for Maxwell-Boltzmann distributions). A molecule at the 90th percentile ($P_T(v) = 0.9$) is faster than 90% of the ensemble.
\end{definition}

\begin{theorem}[Context-Dependent Percentile]
\label{thm:context_dependent}
For the same velocity $v$ in two ensembles at temperatures $T_A < T_B$, the velocity percentile is higher in the colder ensemble:
\begin{equation}
P_{T_A}(v) > P_{T_B}(v)
\label{eq:percentile_ordering}
\end{equation}
The same velocity represents a higher percentile (is "faster" relative to the ensemble) in the colder container than in the hotter container. A molecule that is "above average" in the cold container may be "below average" in the hot container.
\end{theorem}

\begin{proof}
The Maxwell-Boltzmann distribution shifts to higher velocities as the temperature increases. The mean speed increases with temperature:
\begin{equation}
\langle v \rangle_A = \sqrt{\frac{8k_B T_A}{\pi m}} < \sqrt{\frac{8k_B T_B}{\pi m}} = \langle v \rangle_B
\end{equation}
for $T_A < T_B$.

Consider a fixed velocity $v$. If $v = \langle v \rangle_A$ (the mean speed in container A), then $v$ is at the 50th percentile in A: $P_{T_A}(v) \approx 0.5$ (half of the molecules in A are slower than $v$, and half are faster). But since $\langle v \rangle_A < \langle v \rangle_B$, the same velocity $v$ is below the mean in container B: $v < \langle v \rangle_B$. Therefore, more than half of molecules in B are faster than $v$, giving $P_{T_B}(v) < 0.5$.

\begin{figure*}[htbp]
\centering
\includegraphics[width=0.95\textwidth]{figures/velocity_temperature_panel.png}
\caption{\textbf{Velocity-Temperature Non-Correspondence: Same Velocity, Different "Temperature Meaning"—Context Determines Categorical Interpretation.}
\textbf{(A)} Overlapping velocity distributions. Three Gaussian distributions: Cold (300K, teal), Hot (340K, red), and their Overlap (purple). The distributions substantially overlap in the velocity range $v \in [2, 8]$ (arbitrary units). Dashed vertical line indicates overlap region where cold and hot distributions intersect. The complete overlap demonstrates that every velocity value exists in both temperature distributions—there is no velocity threshold that cleanly separates "cold" from "hot" molecules. This overlap is the fundamental reason why velocity-based sorting cannot sort by temperature.
\textbf{(B)} Same velocity, opposite categorical meaning. Central diagram shows single velocity value (500 m/s, white circle) with two interpretations: "FAST" (above average) in cold container (300K, teal box) and "SLOW" (below average) in hot container (310K, red box). Green arrows point from 500 m/s to both interpretations. Text below: "Same velocity, opposite categorical meaning!" A molecule with $v = 500$ m/s contributes to hotness in a cold ensemble but contributes to coldness in a hot ensemble. The categorical meaning of velocity is context-dependent, not intrinsic.
\textbf{(C)} Velocity percentile by context. Two horizontal bars representing containers: Cold Container (300K, teal) and Hot Container (310K, red). Yellow dashed line at "50\%" in cold container indicates molecule is "Fast" (above median). Same yellow dashed line at "47\%" in hot container indicates same molecule is "Slow" (below median). Text: "Same v!" The identical velocity corresponds to different percentile ranks in different ensembles. This proves that velocity does not have intrinsic "temperature meaning"—the same $v$ can be hot-contributing or cold-contributing depending on context.
\textbf{(D)} The sorting paradox. Two containers: COLD (teal, molecule with "FAST" label) and HOT (red, molecule with "SLOW" label). Demon (orange circle) between them says "Move to hot!" Red box below: "PARADOX: Intended: Add 'fast' to make hotter. Result: Added 'slow'—makes COLDER! Demon achieved the OPPOSITE." When the demon moves a "fast" molecule from cold to hot, that molecule becomes "slow" in the hot context. The demon intends to heat the hot container but actually cools it. This is not a measurement error—it's a category inversion upon transfer.
\textbf{(E)} No molecular temperature—only ensemble temperature. Left box (red, "WRONG"): "Molecule has $v \Rightarrow$ Molecule has $T$". Right box (green, "CORRECT"): "Ensemble has $\{v_i\} \Rightarrow$ Ensemble has $T$". Formula below: "$T = T[\{v_1, v_2, \ldots, v_N\}]$. Temperature is a FUNCTIONAL of distribution, NOT a function of individual velocity. Only ENSEMBLES have temperature." Individual molecules do not have temperature—they have velocity. Temperature is a statistical property of the distribution $\{v_i\}$, not a property of individual $v_i$. The demon's error is treating velocity as if it were temperature.
\textbf{(F)} Category changes upon transfer. Two panels: Before (left) shows molecule in "Category: HOT" (red container). After (right) shows same molecule transferred to "Category: COLD" (teal container). Arrow indicates transfer. Yellow box below: "Velocity: UNCHANGED. Category: INVERTED. Context determines meaning." The molecule's velocity doesn't change during transfer, but its categorical role inverts. A hot-contributing molecule becomes a cold-contributing molecule simply by changing containers. This context-dependence makes velocity-based sorting fundamentally incapable of temperature-based sorting.
\textbf{(G)} Why the demon cannot sort by temperature—four reasons. Numbered list: "1. Temperature is not a molecular property. 2. Velocity does not determine T contribution. 3. T contribution is ensemble-relative. 4. Transfer changes categorical meaning." Red box below: "Sorting by velocity $\neq$ sorting by temperature." Each point addresses a different aspect of the category error. Temperature is collective (1), velocity-temperature mapping is non-unique (2), contribution depends on context (3), and context changes upon transfer (4). These four reasons combine to prove that velocity sorting cannot achieve temperature sorting.
\textbf{(H)} The overlap region problem. Venn diagram with two circles: "Cold only" (teal), "Hot only" (red), and overlap region "BOTH" (purple). Text below: "Overlap—ALL velocities! Every velocity exists in both distributions. The overlap is COMPLETE." The overlap region is not a small boundary—it encompasses the entire velocity range. There is no velocity value that belongs exclusively to cold or exclusively to hot. Complete overlap means that velocity provides zero information about which distribution a molecule came from. The demon cannot use velocity to determine temperature category.
\textbf{(I)} The insight: velocity $\neq$ temperature. Purple box: "Velocity $\neq$ Temperature. Same velocity $v$: In cold: 'FAST' (contributes to hotness). In hot: 'SLOW' (contributes to coldness). The demon sorts by velocity but CANNOT sort by temperature. Temperature is contextual, not intrinsic." This summarizes the resolution: velocity and temperature are not equivalent. The same velocity has opposite thermal meanings in different contexts. The demon's velocity-based sorting cannot achieve temperature-based sorting because temperature meaning is context-dependent. This is a fundamental limitation, not an engineering challenge. No improvement in velocity measurement precision can overcome the context-dependence of temperature meaning.}
\label{fig:velocity_temperature_noncorrespondence}
\end{figure*}


More generally, the cumulative distribution function $F_T(v) = P_T(v)$ satisfies:
\begin{equation}
\frac{\partial F_T(v)}{\partial T} < 0 \quad \text{for fixed } v > 0
\label{eq:cdf_temperature_derivative}
\end{equation}

This can be proven by differentiating the cumulative distribution:
\begin{equation}
F_T(v) = \int_0^v f(v'; T) \, dv' / n
\end{equation}

Taking the derivative with respect to $T$:
\begin{equation}
\frac{\partial F_T(v)}{\partial T} = \int_0^v \frac{\partial f(v'; T)}{\partial T} \, dv' / n
\end{equation}

The derivative of the Maxwell-Boltzmann distribution with respect to temperature can be computed:
\begin{equation}
\frac{\partial f(v; T)}{\partial T} = f(v; T) \left[\frac{mv^2}{2k_B T^2} - \frac{3}{2T}\right]
\end{equation}

For $v < v_{\text{rms}} = \sqrt{3k_B T / m}$, the term in brackets is negative: $mv^2 / 2k_B T^2 < 3/2T$, giving $\partial f / \partial T < 0$. For $v > v_{\text{rms}}$, the term is positive: $\partial f / \partial T > 0$. The distribution decreases at low velocities and increases at high velocities as temperature increases (the distribution shifts to the right).

Integrating over $[0, v]$, the cumulative distribution decreases with temperature for $v$ in the range where most of the probability mass lies (roughly $v < v_{\text{rms}}$). Therefore, $\partial F_T(v) / \partial T < 0$ for typical velocities, giving $P_{T_A}(v) > P_{T_B}(v)$ for $T_A < T_B$.

The physical interpretation is clear: as temperature increases, the distribution shifts to higher velocities. A fixed velocity $v$ becomes relatively slower (lower percentile) as the ensemble heats up. \qed
\end{proof}

\begin{example}[Numerical Illustration]
\label{ex:nitrogen_percentiles}
Consider nitrogen gas (N$_2$, molecular mass $m = 28$ atomic mass units $\approx 4.65 \times 10^{-26}$ kg) in two containers at temperatures $T_A = 300$ K and $T_B = 310$ K (a 10 K or 3.3% temperature difference).

The mean speeds are:
\begin{align}
\langle v \rangle_A &= \sqrt{\frac{8k_B T_A}{\pi m}} \approx 476 \text{ m/s} \\
\langle v \rangle_B &= \sqrt{\frac{8k_B T_B}{\pi m}} \approx 484 \text{ m/s}
\end{align}

Consider a molecule moving at $v = 480$ m/s. This velocity is:
\begin{itemize}
    \item In Container A: Above the mean ($480 > 476$ m/s), so the molecule is "fast" relative to A. The percentile is $P_{T_A}(480) \approx 0.53$ (53rd percentile—faster than 53% of molecules in A).
    \item In Container B: Below the mean ($480 < 484$ m/s), so the molecule is "slow" relative to B. The percentile is $P_{T_B}(480) \approx 0.47$ (47th percentile—faster than only 47% of molecules in B).
\end{itemize}

The same velocity (480 m/s) is "fast" in container A (above average, 53rd percentile) and "slow" in container B (below average, 47th percentile). The velocity's meaning is context-dependent: it depends on which ensemble the molecule belongs to.

If the demon transfers this molecule from A to B, intending to make B hotter (by adding a "fast" molecule from A), the transfer actually makes B colder (because the molecule is "slow" relative to B). The demon's sorting achieves the opposite of its intention for molecules in the overlap region.
\end{example}

\begin{remark}[Ambiguous Velocity Region]
\label{rem:ambiguous_region}
The region of velocities where the percentile changes sign (from above average in A to below average in B) is the "ambiguous region" where the demon's sorting is most confused. For the nitrogen example above, the ambiguous region is approximately $[476, 484]$ m/s (between the two mean speeds). Molecules in this region are "fast" in A but "slow" in B. The width of the ambiguous region is proportional to the temperature difference: $\Delta v_{\text{ambiguous}} \sim \langle v \rangle_B - \langle v \rangle_A \propto \sqrt{T_B} - \sqrt{T_A} \approx (\sqrt{T_A} / 2\sqrt{T_A}) \Delta T = \Delta T / 2\sqrt{T_A}$ for small $\Delta T$. As the temperature difference decreases ($\Delta T \to 0$), the ambiguous region shrinks, but the fraction of molecules in the ambiguous region increases (because the distributions become more similar, and more molecules are near the mean where the ambiguity occurs).
\end{remark}

\subsection{The Demon's Sorting Paradox}

The context-dependence of velocity meaning creates a fundamental paradox for the demon's sorting operation: the demon cannot determine whether a molecule is "hot" or "cold" from its velocity alone because the classification depends on the ensemble context, which changes when the molecule is transferred.

\begin{theorem}[Sorting Paradox]
\label{thm:sorting_paradox}
The demon cannot sort molecules by "temperature contribution" because velocity does not determine temperature contribution (temperature is an ensemble property, not a molecular property), temperature contribution is context-dependent (the same velocity has different meanings in different ensembles), and moving a molecule changes its context, hence its contribution (a molecule that is "hot" in A becomes "cold" in B upon transfer).
\end{theorem}

\begin{proof}
Consider the demon attempting to sort molecules from the overlap region (velocities that exist in both containers with significant probability).

\textbf{Step 1: Identification.}
The demon identifies a molecule in container A with velocity $v^*$ where $P_{T_A}(v^*) > 0.5$ (the molecule is "fast" in A, above the 50th percentile). The demon classifies this molecule as "hot" based on its velocity.

\textbf{Step 2: Transfer.}
The demon opens the door and allows the molecule to pass from container A to container B. The demon's intention is to increase B's temperature by adding a "hot" molecule.

\textbf{Step 3: Context change.}
Upon entering container B, the molecule's velocity $v^*$ is unchanged (velocity is conserved during the transfer, assuming no collisions at the door). However, the molecule's percentile changes:
\begin{equation}
P_{T_B}(v^*) < P_{T_A}(v^*)
\end{equation}
from Theorem~\ref{thm:context_dependent}.

If $v^*$ was chosen such that $P_{T_A}(v^*) > 0.5$ but $P_{T_B}(v^*) < 0.5$ (a velocity in the ambiguous region), then the molecule is now "slow" relative to container B (below the 50th percentile in B). The molecule that was "hot" in A has become "cold" in B.

\textbf{Step 4: Consequence.}
The transfer of a "cold" molecule (relative to B) into container B decreases B's temperature rather than increasing it. The demon's sorting achieves the opposite of its intention for molecules in the overlap region.

\textbf{Paradox:}
The demon classified the molecule as "hot" based on its velocity in A, but the molecule is "cold" in B. The classification is context-dependent and changes upon transfer. The demon cannot know, from velocity alone, whether a transfer will increase or decrease the destination container's temperature. The demon's sorting strategy is incoherent because the property it is sorting by (temperature contribution) is not a molecular property but an ensemble-relative property that changes when the molecule changes ensembles. \qed
\end{proof}

\begin{corollary}[No Molecular Temperature]
\label{cor:no_molecular_temperature}
Individual molecules do not possess temperature. Temperature is a property of ensembles, not particles. Formally, temperature is a functional of the entire velocity distribution:
\begin{equation}
T = T[\{v_1, v_2, \ldots, v_N\}] \neq T(v_i) \text{ for any } i
\label{eq:temperature_functional}
\end{equation}
Temperature is defined as $T = (m / 3k_B N) \sum_{i=1}^N v_i^2$, which depends on all $N$ velocities, not on any single velocity $v_i$. An individual molecule contributes $(m v_i^2) / (3k_B N)$ to the temperature, but this contribution's significance (whether it increases or decreases temperature) depends on how $v_i$ compares to the ensemble mean $\langle v^2 \rangle$, which depends on all other velocities.
\end{corollary}

\begin{proof}
Temperature is defined thermodynamically as the inverse of the partial derivative of entropy with respect to energy: $1/T = (\partial S / \partial E)_{V,N}$. In statistical mechanics, this is equivalent to the mean kinetic energy per degree of freedom: $\langle E_{\text{kin}} \rangle = (3/2) N k_B T$ for a three-dimensional ideal gas, giving:
\begin{equation}
T = \frac{2}{3N k_B} \sum_{i=1}^N \frac{1}{2} m v_i^2 = \frac{m}{3N k_B} \sum_{i=1}^N v_i^2
\end{equation}

This is a function of all $N$ velocities $\{v_1, v_2, \ldots, v_N\}$, not a function of any single velocity $v_i$. Temperature is an ensemble average, not a molecular property.

An individual molecule with velocity $v_i$ contributes:
\begin{equation}
\Delta T_i = \frac{m v_i^2}{3N k_B}
\end{equation}
to the total temperature. But this contribution is $1/N$ of the total, and its significance depends on how $v_i$ compares to the ensemble mean. If $v_i > \langle v \rangle$, the molecule contributes more than average, increasing temperature. If $v_i < \langle v \rangle$, the molecule contributes less than average, decreasing temperature. The classification ("hot" or "cold") depends on the ensemble context (the mean $\langle v \rangle$), not on the velocity $v_i$ alone.

Therefore, temperature is not a molecular property. It is an emergent ensemble property that arises from averaging over many molecules. Individual molecules have velocities and kinetic energies, but not temperatures. \qed
\end{proof}

\begin{remark}[Intensive vs. Extensive Properties]
\label{rem:intensive_extensive}
Corollary~\ref{cor:no_molecular_temperature} reflects the distinction between intensive and extensive properties. Temperature is an intensive property: it does not scale with system size (doubling the number of molecules does not double the temperature). Intensive properties are ensemble properties that emerge from averaging over many particles. Individual particles do not possess intensive properties; they possess extensive properties (energy, momentum, mass) that sum to give the total. Temperature is the average kinetic energy per particle, not a property of individual particles.

This distinction is crucial for understanding the demon's failure. The demon attempts to sort by temperature (an intensive, ensemble property), but it can only measure velocity (an extensive, molecular property). The mapping from velocity to temperature is many-to-one and context-dependent: the same velocity corresponds to different temperature contributions in different ensembles. The demon cannot invert this mapping to determine temperature from velocity alone.
\end{remark}

\subsection{Velocity as Categorical Position}

The context-dependence of velocity meaning can be formalized using the categorical framework. A molecule's velocity category (whether it is "hot," "cold," or "average") is determined by its position in the ensemble distribution, not by its absolute velocity.

\begin{definition}[Velocity Category]
\label{def:velocity_category}
A molecule's \textbf{velocity category} in ensemble $E$ at temperature $T$ is its position relative to the ensemble distribution, defined by percentile thresholds:
\begin{equation}
\mathcal{V}_E(v; T) = \begin{cases}
\text{``cold''} & \text{if } P_T(v) < P_{\text{low}} \\
\text{``average''} & \text{if } P_{\text{low}} \leq P_T(v) \leq P_{\text{high}} \\
\text{``hot''} & \text{if } P_T(v) > P_{\text{high}}
\end{cases}
\label{eq:velocity_category}
\end{equation}
where $P_{\text{low}}$ and $P_{\text{high}}$ are threshold percentiles (e.g., $P_{\text{low}} = 0.33$ and $P_{\text{high}} = 0.67$ for terciles, dividing the distribution into three equal parts). The velocity category is a categorical variable (a discrete label) that depends on both the velocity $v$ and the ensemble context (temperature $T$).
\end{definition}

\begin{theorem}[Category Change on Transfer]
\label{thm:category_change}
When a molecule transfers between ensembles at different temperatures, its velocity category can change, even though its velocity is unchanged. Formally, for velocities in the overlap region with different percentile positions:
\begin{equation}
\mathcal{V}_A(v; T_A) \neq \mathcal{V}_B(v; T_B)
\label{eq:category_change}
\end{equation}
for $T_A < T_B$ and $v$ in the ambiguous region where $P_{T_A}(v) > P_{\text{high}}$ but $P_{T_B}(v) < P_{\text{high}}$ (or similar boundary crossings).
\end{theorem}

\begin{proof}
Direct consequence of Theorem~\ref{thm:context_dependent}. For $T_A < T_B$ and a velocity $v$ in the overlap region:
\begin{equation}
P_{T_A}(v) > P_{T_B}(v)
\end{equation}

If $v$ is chosen such that $P_{T_A}(v) > P_{\text{high}}$ (the molecule is "hot" in A, above the high percentile threshold) but $P_{T_B}(v) < P_{\text{high}}$ (the molecule is not "hot" in B, below the high percentile threshold), then the velocity category changes upon transfer:
\begin{equation}
\mathcal{V}_A(v; T_A) = \text{``hot''} \quad \text{but} \quad \mathcal{V}_B(v; T_B) = \text{``average'' or ``cold''}
\end{equation}

The molecule has crossed a category boundary (the high percentile threshold) upon changing ensembles. Its velocity $v$ is unchanged, but its categorical position (relative to the ensemble distribution) has changed.

This category change is inevitable for velocities in the ambiguous region (roughly between the two ensemble means $\langle v \rangle_A$ and $\langle v \rangle_B$). For small temperature differences, the ambiguous region contains a significant fraction of molecules, so category changes are common. \qed
\end{proof}

\begin{remark}[Categorical vs. Absolute Classification]
\label{rem:categorical_absolute}
Theorem~\ref{thm:category_change} reveals that velocity classification is categorical (relative to ensemble), not absolute (independent of ensemble). An absolute classification would assign the same category to a velocity regardless of ensemble context: a velocity $v > v_{\text{threshold}}$ would always be "hot," regardless of which ensemble it belongs to. But the Maxwell-Boltzmann distribution makes absolute classification impossible: there is no velocity threshold that separates "hot" molecules from "cold" molecules across all ensembles. The same velocity is "hot" in one ensemble and "cold" in another.

The categorical classification (Definition~\ref{def:velocity_category}) is the only coherent classification: it assigns categories based on percentile position, which is ensemble-relative. But this makes the demon's sorting incoherent: the demon classifies a molecule as "hot" in ensemble A, transfers it to ensemble B, and the molecule becomes "cold" upon transfer. The demon cannot maintain a consistent classification across ensembles.
\end{remark}

\subsection{The Demon Cannot Sort by Temperature}

We now prove the central result: the demon cannot sort molecules by temperature because temperature is not a molecular property and velocity does not determine temperature contribution.

\begin{theorem}[Temperature Sorting Impossibility]
\label{thm:temp_sort_impossible}
The demon cannot sort molecules by temperature because temperature is not a molecular property (it is an ensemble property), velocity determines only kinetic energy, not temperature contribution (contribution depends on ensemble context), temperature contribution is ensemble-relative (the same velocity has different contributions in different ensembles), and sorting changes ensemble composition, hence all molecules' contributions (each transfer changes the ensemble means, changing all percentiles).
\end{theorem}

\begin{proof}
Suppose the demon attempts to sort molecules by "temperature contribution," with the goal of increasing the temperature of container B by transferring "hot" molecules from container A.

\textbf{Problem 1: Temperature is not a molecular property.}
From Corollary~\ref{cor:no_molecular_temperature}, temperature is an ensemble property, not a molecular property. Temperature is defined as:
\begin{equation}
T = \frac{m}{3k_B N} \sum_{i=1}^N v_i^2 = \frac{m}{3k_B} \langle v^2 \rangle
\end{equation}
where the average $\langle v^2 \rangle$ is over the entire ensemble. An individual molecule contributes $(m v_i^2) / (3k_B N)$ to this average, but this contribution's significance (whether it increases or decreases temperature) depends on $N$ and on the other molecules' velocities $\{v_j\}_{j \neq i}$.

The demon cannot measure "temperature" of an individual molecule because temperature is not defined for individuals. The demon can measure velocity $v_i$, which determines kinetic energy $E_i = (1/2) m v_i^2$, but kinetic energy is not temperature. The demon must infer temperature contribution from velocity, but this inference is context-dependent (Problem 3 below).

\textbf{Problem 2: Sorting changes the ensemble.}
When the demon removes a molecule from container A, the ensemble composition changes. The temperature of A after removal is:
\begin{equation}
T_A' = \frac{m}{3k_B(N_A - 1)} \sum_{j \neq i} v_j^2
\end{equation}
where the sum excludes the removed molecule $i$. This is different from the original temperature:
\begin{equation}
T_A' = \frac{N_A}{N_A - 1} \left(T_A - \frac{m v_i^2}{3k_B N_A}\right) \neq T_A
\end{equation}

The removal changes the mean, which changes all molecules' percentile positions. A molecule that was "average" before removal may become "hot" after removal (if the removed molecule was faster than average, lowering the mean). The demon's classification of remaining molecules is invalidated by the removal.

Similarly, when the demon adds a molecule to container B, the temperature of B after addition is:
\begin{equation}
T_B' = \frac{m}{3k_B(N_B + 1)} \left(\sum_{j=1}^{N_B} v_j^2 + v_i^2\right)
\end{equation}

Whether $T_B' > T_B$ (temperature increases) or $T_B' < T_B$ (temperature decreases) depends on how $v_i$ compares to the mean in B:
\begin{equation}
T_B' > T_B \iff v_i^2 > \langle v^2 \rangle_B \iff v_i > v_{\text{rms}, B}
\end{equation}

The demon's goal (increase $T_B$) is achieved only if $v_i > v_{\text{rms}, B}$ (the transferred molecule is faster than the rms speed in B). But the demon classified the molecule based on its velocity in A ($v_i > v_{\text{rms}, A}$), not in B. If $v_{\text{rms}, A} < v_i < v_{\text{rms}, B}$ (the molecule is in the ambiguous region), the demon's classification is incorrect: the molecule is "hot" in A but "cold" in B, and the transfer decreases $T_B$ rather than increasing it.

\textbf{Problem 3: Temperature contribution is ensemble-relative.}
From Theorem~\ref{thm:context_dependent}, the same velocity $v$ has different percentile positions in different ensembles:
\begin{equation}
P_{T_A}(v) > P_{T_B}(v) \quad \text{for } T_A < T_B
\end{equation}

A molecule with velocity $v$ contributes $(m v^2) / (3k_B N)$ to temperature, but whether this is a "positive" contribution (increases temperature) or "negative" contribution (decreases temperature) depends on how $v$ compares to the ensemble mean. In ensemble A, if $v > \langle v \rangle_A$, the contribution is positive. In ensemble B, if $v < \langle v \rangle_B$, the contribution is negative. The same velocity can have positive contribution in A and negative contribution in B.

The demon cannot determine temperature contribution from velocity alone. The demon must know the ensemble context (the mean $\langle v \rangle$ or rms speed $v_{\text{rms}}$) to determine whether a velocity is "hot" or "cold." But the ensemble context changes with each transfer (Problem 2), so the demon must continuously update its classification. This is computationally infeasible for large $N$ and fundamentally incoherent (the classification changes faster than sorting can occur).

\textbf{Problem 4: The demon cannot know the effect of a transfer.}
Combining Problems 1-3, the demon cannot know, from velocity alone, whether a transfer will increase or decrease the destination container's temperature. The effect depends on:
\begin{itemize}
    \item The current ensemble composition in both containers (the means $\langle v \rangle_A$ and $\langle v \rangle_B$).
    \item How the transferred molecule's velocity compares to both means.
    \item How the ensemble compositions will change after the transfer (which changes the means).
\end{itemize}

All of these factors are dynamic (they change with each transfer) and context-dependent (they depend on the ensemble, not on the molecule). The demon cannot compute the effect of a transfer without knowing the full ensemble state (all $N_A + N_B$ velocities), which would require measuring the entire system, not just the molecule at the door.

\textbf{Conclusion:}
The demon's strategy of sorting by velocity to create a temperature difference is fundamentally incoherent. Temperature is not a molecular property, velocity does not determine temperature contribution, temperature contribution is ensemble-relative, and sorting changes the ensemble (invalidating previous classifications). The demon cannot sort by temperature because temperature is not sortable at the molecular level. \qed
\end{proof}

\begin{remark}[Demon's Dilemma]
\label{rem:demon_dilemma}
Theorem~\ref{thm:temp_sort_impossible} reveals a dilemma for the demon. To sort by temperature, the demon must:
\begin{enumerate}
    \item Measure velocity (the only molecular property accessible).
    \item Infer temperature contribution from velocity (requires knowing ensemble context).
    \item Update ensemble context after each transfer (requires tracking all molecules).
    \item Repeat for $N$ molecules (computationally infeasible).
\end{enumerate}

Step 2 requires knowledge of the ensemble state (the mean or distribution), which requires measuring all molecules (Step 3). But if the demon must measure all molecules to determine temperature contribution, it gains no advantage from measuring individual velocities at the door. The demon might as well measure the entire ensemble state directly (compute the temperature of each container) and decide whether to open the door based on the ensemble temperatures, not on individual velocities.

But even this strategy fails because sorting changes the ensemble state (Problem 2), so the demon must re-measure after each transfer. The demon enters an infinite loop of measurement and sorting, never achieving a stable temperature difference. The demon's strategy is not merely difficult but conceptually incoherent.
\end{remark}

\subsection{Why the Overlap Matters}

The distribution overlap is not merely a technical detail but a fundamental obstacle to the demon's sorting strategy.

\begin{proposition}[Overlap Fraction]
\label{prop:overlap_fraction}
For temperatures $T_A$ and $T_B$ with $T_B = T_A + \Delta T$, the fraction of molecules in the ambiguous overlap region (where "hot in A" maps to "cold in B") increases as $\Delta T \to 0$. In the limit of equal temperatures ($\Delta T \to 0$), the entire distribution is ambiguous: every molecule is in the overlap region where its category is undefined.
\end{proposition}

\begin{proof}
The ambiguous region is defined as velocities where the percentile changes category upon transfer. For simplicity, consider the region where $P_{T_A}(v) > 0.5$ (above average in A) but $P_{T_B}(v) < 0.5$ (below average in B). This region is approximately:
\begin{equation}
\langle v \rangle_A < v < \langle v \rangle_B
\end{equation}
where $\langle v \rangle_A$ and $\langle v \rangle_B$ are the mean speeds in the two containers.

The width of the ambiguous region is:
\begin{equation}
\Delta v_{\text{ambiguous}} = \langle v \rangle_B - \langle v \rangle_A = \sqrt{\frac{8k_B T_B}{\pi m}} - \sqrt{\frac{8k_B T_A}{\pi m}}
\end{equation}

For small temperature differences $\Delta T = T_B - T_A \ll T_A$, we can expand:
\begin{equation}
\langle v \rangle_B \approx \langle v \rangle_A \left(1 + \frac{\Delta T}{2T_A}\right)
\end{equation}
giving:
\begin{equation}
\Delta v_{\text{ambiguous}} \approx \langle v \rangle_A \frac{\Delta T}{2T_A}
\end{equation}

The width of the ambiguous region is proportional to $\Delta T$ and vanishes as $\Delta T \to 0$.

However, the fraction of molecules in the ambiguous region does not vanish. The probability density at the mean is $f(\langle v \rangle; T) \sim \exp(-3/2)$ (from the Maxwell-Boltzmann distribution), which is significant. The fraction of molecules in the interval $[\langle v \rangle_A, \langle v \rangle_B]$ is approximately:
\begin{equation}
\text{Fraction} \approx f(\langle v \rangle_A; T_A) \cdot \Delta v_{\text{ambiguous}} \approx f(\langle v \rangle_A; T_A) \cdot \langle v \rangle_A \frac{\Delta T}{2T_A}
\end{equation}

As $\Delta T \to 0$, the width $\Delta v_{\text{ambiguous}} \to 0$, but the density $f(\langle v \rangle_A; T_A)$ remains finite. The fraction of molecules in the ambiguous region scales as $\Delta T / T_A$, which is small for small temperature differences but non-zero.

More importantly, as $\Delta T \to 0$, the distributions become identical: $f_A(v; T_A) \to f_B(v; T_B)$. In the limit $T_A = T_B$, every molecule is in the "ambiguous" region in the sense that its category in A equals its category in B (there is no category change upon transfer because the distributions are identical). The demon's sorting becomes completely incoherent: there is no way to distinguish "hot" molecules from "cold" molecules because all molecules have the same distribution. \qed
\end{proof}

\begin{corollary}[Demon Failure at Small Temperature Differences]
\label{cor:demon_failure_small_delta_t}
The demon's sorting is most confused precisely where it should be most effective—when the temperature difference is small and needs to be amplified. For small $\Delta T$, a significant fraction of molecules are in the ambiguous overlap region where the demon's classification is incorrect. As $\Delta T \to 0$, the demon's sorting becomes completely incoherent (no molecules can be reliably classified as "hot" or "cold").
\end{corollary}

\begin{proof}
The demon's goal is to amplify a small temperature difference: starting from $T_A \approx T_B$ (nearly equal temperatures), the demon attempts to create a large temperature difference by sorting. But from Proposition~\ref{prop:overlap_fraction}, the demon's sorting is most confused when $\Delta T$ is small: a large fraction of molecules are in the ambiguous region where the demon's classification is incorrect.

For example, if $\Delta T / T_A = 0.01$ (1% temperature difference), approximately 1% of molecules are in the ambiguous region (rough estimate). The demon misclassifies these molecules, transferring "hot" molecules from A to B that are actually "cold" in B, decreasing $T_B$ rather than increasing it. The demon's sorting is partially counterproductive.

As $\Delta T \to 0$, the fraction of misclassified molecules approaches 100%: all molecules are in the ambiguous region. The demon's sorting becomes completely counterproductive: every transfer decreases the temperature difference rather than increasing it (or has no effect, if the molecule is exactly at the mean).

Therefore, the demon fails most dramatically precisely where it should succeed: at small temperature differences. The demon cannot amplify small temperature differences because the distribution overlap makes classification impossible. \qed
\end{proof}

\begin{remark}[Paradoxical Failure Mode]
\label{rem:paradoxical_failure}
Corollary~\ref{cor:demon_failure_small_delta_t} reveals a paradoxical failure mode: the demon is most effective when it is least needed (large temperature differences, where the distributions are well-separated and classification is easy), and least effective when it is most needed (small temperature differences, where the distributions overlap and classification is hard). This is the opposite of what one would expect from an intelligent agent: the demon should be most useful precisely when the task is most difficult (amplifying small differences). But the statistical nature of temperature makes the task impossible when the differences are small: the demon cannot distinguish "hot" from "cold" when the distributions overlap.

This paradox is a signature of the demon's fundamental incoherence: the demon is trying to sort by a property (temperature) that is not well-defined at the molecular level (it is an ensemble property). The demon's failure at small $\Delta T$ is not a practical limitation (insufficient measurement precision, slow operation speed) but a conceptual limitation (the property being sorted does not exist at the level where sorting occurs).
\end{remark}

\subsection{Summary}

The velocity-temperature overlap reveals a fundamental incoherence in the demon's task, arising from the statistical nature of temperature and the complete overlap of velocity distributions.

Key results established in this section:

\textbf{(1) Temperature is a statistical property of ensembles, not molecules.} Temperature is defined as the mean kinetic energy: $T = (m / 3k_B) \langle v^2 \rangle$, which is an average over the entire ensemble. Individual molecules have velocities and kinetic energies, but not temperatures (Corollary~\ref{cor:no_molecular_temperature}).

\textbf{(2) Velocity distributions overlap completely between any two temperatures.} The Maxwell-Boltzmann distribution has support on $(0, \infty)$ for any positive temperature. Every velocity exists in both containers with positive (though possibly small) probability (Theorem~\ref{thm:distribution_overlap}).

\textbf{(3) The same velocity has different "temperature meaning" in different ensembles.} A velocity that is "above average" (fast) in a cold container may be "below average" (slow) in a hot container. The velocity percentile $P_T(v)$ is context-dependent: $P_{T_A}(v) > P_{T_B}(v)$ for $T_A < T_B$ (Theorem~\ref{thm:context_dependent}).

\textbf{(4) A "fast" molecule in a cold container is "slow" in a hot container.} Molecules in the ambiguous overlap region (roughly between the two ensemble means) change category upon transfer: a molecule classified as "hot" in A becomes "cold" in B (Theorem~\ref{thm:category_change}).

\textbf{(5) Sorting by velocity does not sort by temperature.} The demon classifies molecules based on velocity in the source container, but the classification is invalid in the destination container. The demon's sorting achieves the opposite of its intention for molecules in the overlap region (Theorem~\ref{thm:sorting_paradox}).

\textbf{(6) Moving molecules changes their categorical position.} Transfer changes the ensemble context, which changes the velocity's meaning. A molecule's temperature contribution depends on the ensemble it belongs to, not on its velocity alone (Theorem~\ref{thm:category_change}).

\textbf{(7) The demon cannot know the effect of a transfer from velocity alone.} The effect of transferring a molecule (whether it increases or decreases the destination temperature) depends on the ensemble context (the current means in both containers), which changes with each transfer. The demon must track the entire ensemble state, which is computationally infeasible and conceptually incoherent (Theorem~\ref{thm:temp_sort_impossible}).

\textbf{(8) The demon fails most dramatically at small temperature differences.} When the temperature difference is small, the distributions overlap almost completely, and most molecules are in the ambiguous region where classification is impossible. The demon cannot amplify small temperature differences because the overlap makes sorting incoherent (Corollary~\ref{cor:demon_failure_small_delta_t}).

The demon's sorting strategy is not merely difficult but conceptually incoherent. The demon attempts to sort by a property (temperature) that molecules do not individually possess, using a measurement (velocity) that does not determine the property even statistically. The velocity-temperature non-correspondence is a fundamental obstacle that cannot be overcome by improved measurement, faster operation, or more sophisticated information processing. The demon is trying to sort by an ensemble property at the molecular level, which is impossible because ensemble properties emerge from averaging over many molecules and do not exist at the individual level.

\begin{equation}
\boxed{
\begin{aligned}
\text{Velocity} &\neq \text{Temperature} \quad \text{(temperature is ensemble property)} \\
\text{Velocity meaning} &= f(\text{velocity}, \text{ensemble}) \quad \text{(context-dependent)} \\
\text{Transfer} &\to \text{New ensemble} \to \text{New meaning} \quad \text{(category change)} \\
\text{Demon's strategy} &: \text{conceptually incoherent} \quad \text{(sorts non-existent property)}
\end{aligned}
}
\end{equation}

The demon attacks a property (temperature) that molecules do not possess, using a measurement (velocity) that does not determine the property even statistically. The demon's entire strategy is misdirected at the most fundamental level: it confuses molecular properties (velocity, kinetic energy) with ensemble properties (temperature, entropy), and attempts to manipulate the latter by measuring the former. This confusion is the deepest source of the demon's failure, more fundamental than information-theoretic costs or thermodynamic constraints. The demon is conceptually incoherent before it even begins to operate.

%==============================================================================
\section{Velocity-Entropy Independence: The Orthogonality of Motion and Arrangement}
\label{sec:velocity_entropy}
%==============================================================================

A final, decisive insight emerges from examining what entropy actually counts. The classical Boltzmann entropy $S = k_B \ln \Omega$ counts the number of microstates—the number of distinct arrangements accessible to the system. Crucially, arrangements are spatial configurations, not velocity distributions. Changing molecular velocities without changing spatial structure does not change the number of accessible arrangements and therefore does not change entropy. Velocity and entropy are orthogonal quantities in the sense that they are independent variables: knowledge of one does not determine the other, and changing one does not necessarily change the other. This orthogonality provides the most fundamental defeat of Maxwell's demon: the demon manipulates a quantity (velocity) that is categorically orthogonal to the quantity (entropy) protected by the Second Law. The demon's entire strategy operates in the wrong category, attempting to affect configurational properties (arrangements, entropy) by manipulating kinetic properties (velocities, kinetic energies). This category error renders the demon's strategy fundamentally ineffective, regardless of how efficiently the demon processes information or how precisely it measures velocities. The demon cannot decrease entropy by sorting velocities because entropy does not depend on velocities—it depends on spatial arrangements, which are independent of velocities.

\subsection{Entropy Counts Arrangements}

We begin by clarifying what entropy actually counts in the Boltzmann formulation, which is the foundation of statistical mechanics and the microscopic interpretation of the Second Law.

\begin{definition}[Boltzmann Entropy]
\label{def:boltzmann_entropy}
The Boltzmann entropy of a macrostate is defined as:
\begin{equation}
S = k_B \ln \Omega
\label{eq:boltzmann_entropy}
\end{equation}
where $\Omega$ is the number of microstates compatible with the macrostate, and $k_B$ is Boltzmann's constant. The macrostate specifies macroscopic observables (volume $V$, number of particles $N$, total energy $E$), while microstates specify the complete microscopic configuration of the system. The entropy quantifies the "multiplicity" of the macrostate: how many distinct microscopic ways the system can realize the same macroscopic properties.
\end{definition}

\begin{definition}[Microstate]
\label{def:microstate}
A microstate specifies the complete configuration of the system at the microscopic level. In the spatial interpretation relevant to categorical structure and entropy counting, a microstate is:
\begin{equation}
\text{Microstate} = \{\mathbf{r}_1, \mathbf{r}_2, \ldots, \mathbf{r}_N\}
\label{eq:microstate_spatial}
\end{equation}
where $\mathbf{r}_i$ is the position of molecule $i$ in three-dimensional space. More generally, in phase space, a microstate includes both positions and momenta: $\{\mathbf{r}_i, \mathbf{p}_i\}_{i=1}^N$. However, for entropy counting in the canonical ensemble (fixed temperature), the relevant microstates are spatial configurations, with momenta integrated out according to the Maxwell-Boltzmann distribution.
\end{definition}

\begin{remark}[Configuration Space vs. Phase Space]
\label{rem:configuration_phase}
The distinction between configuration space (positions only) and phase space (positions and momenta) is crucial for understanding entropy. In the microcanonical ensemble (fixed energy), entropy is computed by counting phase space volumes: $\Omega$ is the volume of the constant-energy surface in phase space. In the canonical ensemble (fixed temperature), entropy is computed by counting configuration space volumes: $\Omega$ is the number of spatial configurations, with momenta integrated out. For the purposes of Maxwell's demon (which operates at fixed temperature), the canonical ensemble is appropriate, and entropy counts spatial arrangements. The number of spatial arrangements $\Omega$ depends on how many ways molecules can be positioned in the available volume, subject to constraints (excluded volume, potential energy interactions, phase-lock correlations).
\end{remark}

The number of microstates $\Omega$ depends on how many ways molecules can be arranged spatially, not on how fast they move. This is the key insight: entropy is a configurational property (determined by positions), not a kinetic property (determined by velocities).

\begin{theorem}[Velocity Independence of Arrangement Count]
\label{thm:velocity_arrangement}
The number of spatial arrangements $\Omega$ is independent of molecular velocities. Formally, the partial derivative of $\Omega$ with respect to any velocity component is zero:
\begin{equation}
\frac{\partial \Omega}{\partial v_i} = 0 \quad \forall i
\label{eq:omega_velocity_independent}
\end{equation}
where $v_i = |\mathbf{v}_i|$ is the speed of molecule $i$. This independence holds for all molecules and all velocity components.
\end{theorem}

\begin{proof}
Spatial arrangements are determined by positions $\{\mathbf{r}_i\}_{i=1}^N$, not by velocities $\{\mathbf{v}_i\}_{i=1}^N$. Velocity is defined as the time derivative of position: $\mathbf{v}_i = d\mathbf{r}_i / dt$. Velocity determines the rate of change of position, not the position itself.

At any instant $t$, the positions $\{\mathbf{r}_i(t)\}$ determine the spatial arrangement. The velocities $\{\mathbf{v}_i(t)\}$ determine how rapidly positions will change in the future (how the arrangement will evolve), but they do not determine the current arrangement. Two systems with identical positions $\{\mathbf{r}_i\}$ but different velocities $\{\mathbf{v}_i\}$ have the same spatial arrangement at the current instant, even though their arrangements will differ at future instants.

The count $\Omega$ of distinct spatial configurations is determined by the following factors. First, the system volume $V$: larger volumes allow more distinct positions, increasing $\Omega$. Second, the number of molecules $N$: more molecules create more possible arrangements. Third, excluded volume interactions: molecules cannot overlap, reducing the number of accessible positions. Fourth, potential energy constraints: molecules avoid high-energy configurations (e.g., configurations with strong repulsive interactions), further reducing accessible positions. Fifth, phase-lock correlations: molecules in the same phase-lock cluster have correlated positions, creating additional constraints.

None of these factors depend on velocities. The volume $V$ is a geometric property. The number of molecules $N$ is fixed. Excluded volume depends on molecular size (hard-sphere diameter), not velocity. Potential energy depends on intermolecular distances $|\mathbf{r}_i - \mathbf{r}_j|$, not velocities. Phase-lock correlations depend on coupling strengths $\kappa_{ij}(\mathbf{r}_i, \mathbf{r}_j)$, which are functions of positions, not velocities (as established in Theorem~\ref{thm:kinetic_independence}).

Therefore, $\Omega$ is a function of positions alone: $\Omega = \Omega(\{\mathbf{r}_i\}, V, N, \text{interactions})$, with no dependence on velocities $\{\mathbf{v}_i\}$. The partial derivative with respect to velocity is:
\begin{equation}
\frac{\partial \Omega}{\partial v_i} = \frac{\partial}{\partial v_i} \Omega(\{\mathbf{r}_j\}, V, N, \text{interactions}) = 0
\end{equation}
since $\Omega$ does not contain $v_i$ as a variable.

Velocity and arrangement count are independent variables. Changing velocities does not change the number of arrangements. \qed
\end{proof}

\begin{corollary}[Velocity Independence of Entropy]
\label{cor:velocity_entropy_independent}
Entropy is independent of molecular velocities:
\begin{equation}
\frac{\partial S}{\partial v_i} = \frac{\partial}{\partial v_i} (k_B \ln \Omega) = \frac{k_B}{\Omega} \frac{\partial \Omega}{\partial v_i} = 0
\label{eq:entropy_velocity_independent}
\end{equation}
for all molecules $i$ and all velocity components. Entropy is a function of spatial configuration, not velocity distribution.
\end{corollary}

\begin{proof}
Direct consequence of Theorem~\ref{thm:velocity_arrangement}. Since $\partial \Omega / \partial v_i = 0$, and $S = k_B \ln \Omega$ is a function of $\Omega$ alone, the chain rule gives $\partial S / \partial v_i = (k_B / \Omega) (\partial \Omega / \partial v_i) = 0$. \qed
\end{proof}

\begin{remark}[Entropy vs. Kinetic Energy]
\label{rem:entropy_kinetic}
Corollary~\ref{cor:velocity_entropy_independent} establishes that entropy is independent of velocities and, therefore, independent of kinetic energy (since kinetic energy $E_{\text{kin}} = \sum_i (1/2) m_i v_i^2$ is a function of velocities). This independence is surprising from the perspective of traditional thermodynamics, which often conflates entropy with energy or temperature. The resolution is that entropy depends on the number of accessible configurations (a spatial property), while kinetic energy depends on the velocity distribution (a kinetic property). These are independent: a system can have high kinetic energy (high temperature) with low entropy (few accessible configurations, highly ordered), or low kinetic energy (low temperature) with high entropy (many accessible configurations, highly disordered). The independence is exact in the canonical ensemble, where temperature is fixed and entropy counts spatial configurations with momenta integrated out.
\end{remark}


\begin{figure*}[htbp]
\centering
\includegraphics[width=0.95\textwidth]{figures/panel_arg4_dissolution_observation.png}
\caption{\textbf{Dissolution of Observation—Navigation Follows Topology, Not Velocity Measurement.}
\textbf{(A)} Topology determines path without velocity information. Molecules arranged in a phase-lock network (teal nodes) follow paths determined purely by network adjacency. The path from one side to the other (red nodes indicating transition region) is determined by categorical distance $d_{\text{cat}}(i,j)$ in the network, not by molecular velocities. Navigation occurs through shortest paths in the network graph, requiring no knowledge of kinetic properties.
\textbf{(B)} Observation not required for navigation. The diagram shows two information channels: velocity measurement (red, crossed out) and topological adjacency (green, active). Navigation proceeds through the green channel alone. The system follows network structure without any measurement of velocities, demonstrating that the demon's ``observation'' is unnecessary. Path completion is automatic through categorical structure.
\textbf{(C)} Velocity is uncorrelated with network position. Scatter plot of molecular velocity versus network position shows near-zero correlation ($r = -0.150$, dashed red line). The random scatter demonstrates that knowing a molecule's position in the phase-lock network provides no information about its velocity, and vice versa. This confirms that categorical distance $d_{\text{cat}}$ and kinetic distance $d_{\text{kin}}$ are inequivalent metrics, as stated in Section 3.4.
\textbf{(D)} Topological gate operates on adjacency, not velocity. Schematic of the demon's door showing two molecules (red circles) adjacent to the door and two molecules (blue squares) far from the door. The door opens based purely on topological adjacency in the phase-lock network: any adjacent molecule passes, regardless of velocity. The gate is velocity-blind, operating on categorical structure alone. This dissolves the paradox: there is no velocity measurement, no decision based on kinetic energy, and therefore no violation of the second law. The apparent ``sorting'' is categorical completion through network topology.}
\label{fig:dissolution_observation}
\end{figure*}

\subsection{The Snapshot Principle}

The velocity-entropy independence can be visualized through the concept of a configurational snapshot: a frozen instant capturing spatial positions but not velocities.

\begin{definition}[Configurational Snapshot]
\label{def:snapshot}
A \textbf{configurational snapshot} is the spatial arrangement of molecules at a specific instant $t$:
\begin{equation}
\mathcal{S}(t) = \{\mathbf{r}_1(t), \mathbf{r}_2(t), \ldots, \mathbf{r}_N(t)\}
\label{eq:snapshot}
\end{equation}
The snapshot records positions only, with no information about velocities, kinetic energies, or temperature. It is a "photograph" of the spatial configuration, frozen in time.
\end{definition}

\begin{theorem}[Snapshot Velocity Blindness]
\label{thm:snapshot_blind}
A configurational snapshot is velocity-blind: the same snapshot is compatible with any velocity distribution. Formally, for any snapshot $\mathcal{S}$ (set of positions) and any velocity distribution $\{\mathbf{v}_i\}$ satisfying basic physical constraints (e.g., total momentum conservation), there exists a physical state with positions $\mathcal{S}$ and velocities $\{\mathbf{v}_i\}$. The snapshot does not constrain velocities.
\end{theorem}

\begin{proof}
The snapshot $\mathcal{S}(t)$ records positions $\{\mathbf{r}_i(t)\}$ at instant $t$. It contains no information about the following kinetic properties. First, how fast molecules are moving: the speeds $v_i = |\mathbf{v}_i|$ are not recorded. Second, in what direction they are moving: the velocity vectors $\mathbf{v}_i$ are not recorded. Third, their kinetic energies: $E_i = (1/2) m_i v_i^2$ are not recorded. Fourth, the temperature of the system: $T = (2/3N k_B) \sum_i E_i$ is not recorded.

The snapshot is defined by positions alone: $\mathcal{S} = \{\mathbf{r}_i\}$. Positions and velocities are independent variables in phase space: specifying positions does not uniquely determine velocities. For any set of positions $\{\mathbf{r}_i\}$, there are infinitely many velocity distributions $\{\mathbf{v}_i\}$ that are physically realizable (subject only to conservation laws: total momentum $\sum_i m_i \mathbf{v}_i = \mathbf{P}_{\text{total}}$ and total energy $\sum_i (1/2) m_i v_i^2 + U(\{\mathbf{r}_i\}) = E_{\text{total}}$, where $U$ is the potential energy).

Therefore, the same spatial configuration $\mathcal{S}$ can exist with low velocities (low temperature), high velocities (high temperature), or any intermediate velocity distribution. The snapshot is compatible with any temperature $T \in (0, \infty)$. The snapshot does not constrain temperature. \qed
\end{proof}

\begin{corollary}[Temperature-Snapshot Independence]
\label{cor:temperature_snapshot}
A given snapshot can exist at any temperature:
\begin{equation}
\mathcal{S} \text{ is compatible with } T \in (0, \infty)
\label{eq:snapshot_temperature}
\end{equation}
Temperature does not constrain spatial arrangement, and spatial arrangement does not constrain temperature. The two are independent variables.
\end{corollary}

\begin{proof}
From Theorem~\ref{thm:snapshot_blind}, the snapshot $\mathcal{S}$ (positions) is compatible with any velocity distribution $\{\mathbf{v}_i\}$. Temperature is determined by the velocity distribution: $T = (2/3N k_B) \sum_i (1/2) m_i v_i^2$. Since any velocity distribution is compatible with $\mathcal{S}$, any temperature is compatible with $\mathcal{S}$. The range of compatible temperatures is $(0, \infty)$: arbitrarily low temperatures (all velocities near zero) and arbitrarily high temperatures (all velocities very large) are both compatible with the same spatial configuration. \qed
\end{proof}

\begin{example}[Crystalline Solid at Different Temperatures]
\label{ex:crystal_temperatures}
Consider a crystalline solid with molecules arranged in a regular lattice. The spatial configuration $\mathcal{S}$ is the lattice structure: molecules at positions $\mathbf{r}_i = \mathbf{R}_i + \mathbf{u}_i$, where $\mathbf{R}_i$ are the equilibrium lattice sites and $\mathbf{u}_i$ are small displacements (vibrations around equilibrium).

At low temperature ($T \to 0$), the velocities are small: $|\mathbf{v}_i| \to 0$. Molecules barely vibrate around their lattice sites. The spatial configuration is the lattice structure with small displacements.

At high temperature ($T \gg 0$), the velocities are large: $|\mathbf{v}_i| \sim \sqrt{k_B T / m}$. Molecules vibrate vigorously around their lattice sites. The spatial configuration is still the lattice structure (same average positions $\mathbf{R}_i$), but with larger displacements $|\mathbf{u}_i|$.

The snapshot $\mathcal{S}$ (lattice structure) is the same at both temperatures, even though the velocities differ by orders of magnitude. The entropy of the solid (determined by the number of accessible lattice configurations) is approximately the same at both temperatures (assuming the lattice structure is stable). The kinetic energy (and temperature) differs, but the entropy does not.

This example demonstrates that the same spatial configuration can exist at different temperatures, and entropy (which counts configurations) is independent of temperature (which measures kinetic energy).
\end{example}

\begin{remark}[Snapshot as Equivalence Class]
\label{rem:snapshot_equivalence}
The snapshot principle can be formalized using equivalence classes. Define an equivalence relation on phase space: two states $\Gamma_1 = \{\mathbf{r}_i, \mathbf{v}_i\}$ and $\Gamma_2 = \{\mathbf{r}_i', \mathbf{v}_i'\}$ are equivalent if they have the same positions: $\mathbf{r}_i = \mathbf{r}_i'$ for all $i$, regardless of velocities. The equivalence class $[\mathcal{S}]$ is the set of all states with positions $\mathcal{S}$:
\begin{equation}
[\mathcal{S}] = \{\Gamma = \{\mathbf{r}_i, \mathbf{v}_i\} : \mathbf{r}_i \in \mathcal{S} \text{ for all } i\}
\end{equation}

Each equivalence class corresponds to a snapshot. The entropy $S = k_B \ln \Omega$ counts the number of equivalence classes (snapshots), not the number of states within each class (velocity distributions). Changing velocities moves the system within an equivalence class but does not change which equivalence class the system occupies, and therefore does not change entropy.
\end{remark}

\subsection{Elastic Collisions: Temperature Without Entropy}

The velocity-entropy independence implies that processes that change velocities without changing spatial configurations do not change entropy. Elastic collisions are the paradigmatic example.

\begin{theorem}[Elastic Collision Entropy Invariance]
\label{thm:elastic_entropy}
Elastic collisions between molecules can change the velocity distribution (and therefore the temperature) without changing entropy. Formally, an elastic collision conserves kinetic energy and momentum but can redistribute velocities among molecules, changing the temperature locally or globally, while leaving the spatial configuration (and therefore the entropy) unchanged.
\end{theorem}

\begin{proof}
Consider an elastic collision between molecules $i$ and $j$ at time $t$.

\textbf{Before collision:}
The molecules have positions $\mathbf{r}_i(t^-)$ and $\mathbf{r}_j(t^-)$ (where $t^-$ denotes the instant just before collision) and velocities $\mathbf{v}_i(t^-)$ and $\mathbf{v}_j(t^-)$. The kinetic energies are:
\begin{align}
E_i &= \frac{1}{2} m_i |\mathbf{v}_i(t^-)|^2 \\
E_j &= \frac{1}{2} m_j |\mathbf{v}_j(t^-)|^2
\end{align}

\textbf{After collision:}
The molecules have positions $\mathbf{r}_i(t^+)$ and $\mathbf{r}_j(t^+)$ (where $t^+$ denotes the instant just after collision) and velocities $\mathbf{v}_i(t^+)$ and $\mathbf{v}_j(t^+)$. The kinetic energies are:
\begin{align}
E_i' &= \frac{1}{2} m_i |\mathbf{v}_i(t^+)|^2 \\
E_j' &= \frac{1}{2} m_j |\mathbf{v}_j(t^+)|^2
\end{align}

For an elastic collision, kinetic energy is conserved:
\begin{equation}
E_i + E_j = E_i' + E_j'
\label{eq:elastic_energy_conservation}
\end{equation}

Momentum is also conserved:
\begin{equation}
m_i \mathbf{v}_i(t^-) + m_j \mathbf{v}_j(t^-) = m_i \mathbf{v}_i(t^+) + m_j \mathbf{v}_j(t^+)
\label{eq:elastic_momentum_conservation}
\end{equation}

The velocities change: $\mathbf{v}_i(t^+) \neq \mathbf{v}_i(t^-)$ and $\mathbf{v}_j(t^+) \neq \mathbf{v}_j(t^-)$ (in general). The kinetic energies are redistributed: $E_i' \neq E_i$ and $E_j' \neq E_j$ (in general), subject to the constraint~\eqref{eq:elastic_energy_conservation}.

\textbf{Spatial arrangement:}
The collision occurs at a point in space where the molecules' trajectories intersect. The positions immediately before and after the collision are essentially the same: $\mathbf{r}_i(t^+) \approx \mathbf{r}_i(t^-)$ and $\mathbf{r}_j(t^+) \approx \mathbf{r}_j(t^-)$ (the collision duration is negligible, so positions do not change significantly during the collision). The spatial configuration of the entire system is unchanged: all other molecules have positions that are continuous through the collision time $t$.

\textbf{Number of arrangements:}
Since the spatial configuration is unchanged, the number of accessible spatial arrangements $\Omega$ is unchanged. The collision does not open new regions of configuration space or close existing regions. The set of accessible positions $\{\mathbf{r}_i\}$ is the same before and after the collision.

\textbf{Entropy:}
From the Boltzmann formula, $S = k_B \ln \Omega$. Since $\Omega$ is unchanged, entropy is unchanged:
\begin{equation}
S(t^+) = k_B \ln \Omega(t^+) = k_B \ln \Omega(t^-) = S(t^-)
\end{equation}

The collision is isentropic (constant entropy).

\textbf{Temperature:}
The temperature is determined by the average kinetic energy: $T = (2/3N k_B) \sum_{k=1}^{N} E_k$. If the collision redistributes kinetic energy such that the average changes (e.g., if molecule $i$ gains more energy than molecule $j$ loses, and molecule $i$ is in a region with many neighbors while molecule $j$ is isolated), then the local temperature can change. Globally, the total kinetic energy is conserved (equation~\eqref{eq:elastic_energy_conservation}), so the global average temperature is unchanged. But locally, temperature can increase in some regions and decrease in others.

Therefore, elastic collisions can change temperature (locally or through redistribution) without changing entropy. Temperature and entropy are independent: one can change while the other remains constant. \qed
\end{proof}

\begin{example}[Fast Molecules Become Faster]
\label{ex:fast_faster}
Consider an ensemble of molecules with a bimodal velocity distribution: some molecules are fast (velocity $v_{\text{fast}} \gg \langle v \rangle$) and some are slow (velocity $v_{\text{slow}} \ll \langle v \rangle$). Suppose fast molecules preferentially collide with each other (e.g., they are spatially localized near a boundary or in a cluster).

After elastic collisions among fast molecules, the velocity distribution within the fast group changes. Some fast molecules become even faster (they gain kinetic energy from collisions), while others become slower (they lose kinetic energy). The average kinetic energy of the fast group can increase if the collisions are asymmetric (e.g., if faster molecules tend to gain energy from slower molecules in the group).

The local "temperature" of the fast group, defined as the average kinetic energy, increases:
\begin{equation}
T_{\text{fast}}' = \frac{2}{3 N_{\text{fast}} k_B} \sum_{i \in \text{fast}} E_i' > \frac{2}{3 N_{\text{fast}} k_B} \sum_{i \in \text{fast}} E_i = T_{\text{fast}}
\end{equation}

Meanwhile, the spatial arrangement is unchanged: fast molecules remain in the same spatial region, slow molecules remain in their region. The number of accessible configurations $\Omega$ is unchanged. The entropy is unchanged:
\begin{equation}
S' = k_B \ln \Omega' = k_B \ln \Omega = S
\end{equation}

This example demonstrates that temperature can increase without entropy increasing. The increase is local (confined to the fast group) and temporary (it will eventually dissipate through collisions with the slow group, equilibrating the temperature). But at the instant after the collisions, temperature has increased while entropy has not.
\end{example}

\begin{remark}[Reversible vs. Irreversible Processes]
\label{rem:reversible_irreversible}
Theorem~\ref{thm:elastic_entropy}establishes that elastic collisions are isentropic (constant entropy), which might seem to contradict the Second Law (entropy increases in irreversible processes). The resolution is that elastic collisions are reversible processes: they conserve energy and momentum, and the time-reversed collision (with velocities reversed) is also a valid collision. Reversible processes do not increase entropy; only irreversible processes increase entropy. Irreversible processes involve dissipation (energy conversion to heat), mixing (bringing previously separated substances into contact), or other mechanisms that increase the number of accessible configurations. Elastic collisions do not involve these mechanisms, so they are reversible and isentropic. However, in practice, collisions are never perfectly elastic: there is always some energy dissipation (e.g., into internal degrees of freedom, or radiated away as photons). This dissipation makes collisions irreversible and entropy-increasing. The idealization of perfectly elastic collisions is useful for understanding the velocity-entropy independence but is not realized in nature.
\end{remark}

\subsection{Categorical Interpretation}

The velocity-entropy independence is immediate in the categorical framework, where entropy is determined by phase-lock network structure, which is independent of velocities.

\begin{theorem}[Categorical Velocity Independence]
\label{thm:categorical_velocity}
Categorical structure (phase-lock networks) is determined by spatial relationships, not velocities. Formally, the partial derivative of the categorical structure with respect to velocity is zero:
\begin{equation}
\frac{\partial \mathcal{C}}{\partial v_i} = 0
\label{eq:categorical_velocity_independent}
\end{equation}
where $\mathcal{C}$ denotes the categorical structure (the set of categorical states, the phase-lock network topology, the cluster structure, etc.). Categorical structure is a function of positions, not velocities.
\end{theorem}

\begin{proof}
Phase-lock networks form through intermolecular interactions that depend on spatial relationships. The relevant interactions are Van der Waals forces (depend on separation $|\mathbf{r}_i - \mathbf{r}_j|$ as $\sim 1/r^6$ for attraction and $\sim 1/r^{12}$ for repulsion), dipole-dipole interactions (depend on orientation and separation as $\sim \mathbf{d}_i \cdot \mathbf{d}_j / r^3$, where $\mathbf{d}_i$ is the dipole moment of molecule $i$), and vibrational coupling (depends on normal mode structure, which is determined by the Hessian matrix of the potential energy, a function of positions).

None of these interactions depend on translational velocity $\mathbf{v}_i$. Van der Waals forces are static (they depend on instantaneous positions, not on how positions are changing). Dipole interactions depend on molecular orientations, which are determined by rotational degrees of freedom (angular positions), not translational velocities. Vibrational coupling depends on the curvature of the potential energy surface around equilibrium positions, which is a function of positions, not velocities.

A phase-lock network is defined by which molecules are phase-locked to which: the edge set $E = \{(m_i, m_j) : \kappa_{ij} > \kappa_{\text{threshold}}\}$, where $\kappa_{ij}$ is the coupling strength between molecules $i$ and $j$. The coupling strength is determined by the interactions listed above, all of which depend on positions, not velocities. Therefore, the edge set $E$ (and hence the network topology $\phaselockgraph = (V, E)$) is a function of positions alone:
\begin{equation}
\phaselockgraph = \phaselockgraph(\{\mathbf{r}_i\}, \text{molecular properties})
\end{equation}
with no dependence on velocities $\{\mathbf{v}_i\}$.

Categorical structure $\mathcal{C}$ is the topology of this network: the set of categorical states (equivalence classes of spatial configurations determined by phase-lock relationships), the cluster structure (connected components of the network), the categorical pathways (adjacency relationships in categorical space), etc. All of these are determined by the network topology $\phaselockgraph$, which is determined by positions. Therefore, categorical structure is a function of positions, not velocities:
\begin{equation}
\mathcal{C} = \mathcal{C}(\phaselockgraph(\{\mathbf{r}_i\}))
\end{equation}

The partial derivative with respect to velocity is:
\begin{equation}
\frac{\partial \mathcal{C}}{\partial v_i} = \frac{\partial}{\partial v_i} \mathcal{C}(\phaselockgraph(\{\mathbf{r}_j\})) = 0
\end{equation}
since $\mathcal{C}$ does not contain $v_i$ as a variable.

Velocity and categorical structure are independent variables. Changing velocities does not change categorical structure. \qed
\end{proof}

\begin{corollary}[Categorical Entropy Velocity Independence]
\label{cor:categorical_entropy_velocity}
Categorical entropy (S-entropy) is independent of velocity:
\begin{equation}
\frac{\partial S_{\text{categorical}}}{\partial v_i} = 0
\label{eq:s_entropy_velocity_independent}
\end{equation}
for all molecules $i$ and all velocity components. Categorical entropy is determined by phase-lock network structure, which is independent of velocities.
\end{corollary}

\begin{proof}
From Proposition~\ref{prop:entropy_edge_density}, categorical entropy is proportional to the number of edges in the phase-lock network:
\begin{equation}
S_{\text{categorical}} \propto k_B |E(\phaselockgraph)|
\end{equation}

From Theorem~\ref{thm:categorical_velocity}, the network topology $\phaselockgraph$ (and therefore the edge count $|E|$) is independent of velocities: $\partial \phaselockgraph / \partial v_i = 0$. Therefore:
\begin{equation}
\frac{\partial S_{\text{categorical}}}{\partial v_i} \propto k_B \frac{\partial |E|}{\partial v_i} = 0
\end{equation}

Categorical entropy is velocity-independent. \qed
\end{proof}

\begin{remark}[Spatial vs. Kinetic Entropy]
\label{rem:spatial_kinetic_entropy}
Corollary~\ref{cor:categorical_entropy_velocity} establishes that categorical entropy (which counts phase-lock network configurations) is velocity-independent. This is consistent with Corollary~\ref{cor:velocity_entropy_independent}, which establishes that Boltzmann entropy (which counts spatial configurations) is velocity-independent. Both spatial entropy and categorical entropy are configurational properties, determined by positions (spatial or network), not by velocities (kinetic). This consistency validates the categorical framework: categorical entropy is a refinement of Boltzmann entropy (it counts categorical states, which are equivalence classes of spatial states), and both are velocity-independent because both count configurations, not velocities.
\end{remark}

\begin{figure*}[htbp]
\centering
\includegraphics[width=0.95\textwidth]{figures/velocity_entropy_panel.png}
\caption{\textbf{Velocity-Entropy Independence: The Demon's Category Error.}
\textbf{(A)} Entropy counts spatial arrangements, not velocities. Boltzmann's formula $S = k_B \ln(\Omega)$ where $\Omega$ is the number of spatial arrangements. Three example arrangements shown: Arrangement 1 (two molecules on left, one on right), Arrangement 2 (two molecules in center), Arrangement 3 (two molecules on right, one on left). Different positions yield different arrangements, but velocity is NOT included in the count. The same spatial configuration has the same entropy regardless of molecular velocities. This is the fundamental reason why velocity sorting cannot change entropy.
\textbf{(B)} Snapshot records positions only, not velocities or temperatures. Diagram shows six yellow molecules at fixed positions. Red text indicates "Velocities (ignored)". The same snapshot could represent a system at 100K or 1000K—the positional information is identical. Snapshots capture configurational state ($\Omega$), not kinetic state (velocity distribution). Therefore, entropy $S = k_B \ln(\Omega)$ depends only on what the snapshot shows: positions. Maxwell's demon observes velocities, but entropy depends on positions—this is the category error.
\textbf{(C)} Elastic collision: positions unchanged, velocities changed, entropy unchanged. Before collision: two molecules with velocities $v_1, v_2$ at collision point. After collision: same two molecules with velocities $v_1', v_2'$ (changed) at same collision point. Green box summarizes: Positions SAME (collision point unchanged), Velocities CHANGED, Temperature CAN CHANGE (kinetic energy redistributed), Entropy UNCHANGED (same spatial configuration). This proves that changing velocities—even dramatically—does not change entropy. The demon's velocity-sorting operation is thermodynamically irrelevant.
\textbf{(D)} Orthogonality of entropy and velocity. Two-dimensional diagram with "Config axis" (vertical, green) representing entropy and "Kinetic axis" (horizontal, red) representing velocity. The derivative $\partial S/\partial v = 0$ (shown at origin) indicates that entropy is independent of velocity. The two axes are orthogonal: changing velocity (horizontal motion) does not change entropy (vertical position). This geometric representation shows that kinetic and configurational properties inhabit orthogonal subspaces of the full state space.
\textbf{(E)} The demon's category error. Two boxes: "KINETIC" (purple, containing "Velocity, Momentum, Kinetic energy") and "CONFIGURATIONAL" (blue, containing "Position, Arrangement, Entropy"). Red "X" and "Demon tries" arrow between them indicates the demon attempts to manipulate kinetic properties to affect configurational properties. Text below: "Different categories! Manipulating kinetic properties cannot affect configurational properties." The demon operates in the wrong category—sorting by velocity (kinetic) cannot change arrangement count (configurational).
\textbf{(F)} What changes entropy versus what does not. Left box (green, "CHANGES S"): Mixing, Expansion, Chemical reaction, Phase change—all involve spatial rearrangement. Right box (red, "NO CHANGE"): Elastic collision, Velocity sorting, Adiabatic T change—all involve only velocity redistribution. Bottom text: "Spatial rearrangement = entropy change. Velocity redistribution = NO entropy change. Demon does velocity sorting!" This categorizes thermodynamic processes by their effect on entropy, clearly showing that the demon's operation (velocity sorting) is in the "no change" category.
\textbf{(G)} The demon's broken causal chain. Flow diagram: "Measure v" → "Sort by v" → "Change T" → "Change S?". Each arrow marked with red "X" indicating broken link. Text "without S!" shows that temperature can change without entropy change. Green box at bottom: "Change S?" with text "EVERY STEP BROKEN". The demon's intended causal chain (measure velocity → sort → change temperature → change entropy) is broken at every step. Measuring velocity doesn't determine entropy contribution, sorting by velocity doesn't sort by temperature (see Figure 8), changing temperature doesn't require changing entropy (adiabatic processes), and none of these operations change configurational entropy.
\textbf{(H)} The mathematical proof of velocity-entropy independence. Blue box contains: "$\Omega = f(\text{positions only})$, $\partial \Omega/\partial v_i = 0 \Rightarrow \partial S/\partial v_i = 0$". Text below: "Arrangement count is velocity-independent. Therefore entropy is velocity-independent. Velocity sorting has ZERO effect on entropy." This is the rigorous mathematical statement: since $\Omega$ (the quantity in Boltzmann's formula) depends only on positions, its derivative with respect to any velocity component $v_i$ is exactly zero, therefore entropy's derivative with respect to velocity is exactly zero. Velocity sorting cannot change entropy—not approximately, but exactly.
\textbf{(I)} The final defeat: orthogonality of velocity and entropy. Purple box: "VELOCITY and ENTROPY are ORTHOGONAL". Text below: "Velocity: rate of position change (kinetic). Entropy: count of arrangements (config). Demon sorts: velocities. 2nd law protects: entropy. The demon operates in the WRONG CATEGORY. Category error = fundamental defeat." This summarizes the resolution: velocity and entropy are orthogonal properties. The demon manipulates one (velocity) while the second law protects the other (entropy). The demon's defeat is not due to measurement costs or information erasure, but due to operating in the wrong category—a fundamental conceptual error in Maxwell's original formulation.}
\label{fig:velocity_entropy_independence}
\end{figure*}


\subsection{The Three-Way Orthogonality}

We now synthesize the results from this section and previous sections to establish a three-way orthogonality: velocity is orthogonal to temperature meaning, to entropy, and to heat flow direction.

\begin{theorem}[Velocity-Thermodynamic Orthogonality]
\label{thm:three_orthogonality}
Velocity is orthogonal to three fundamental thermodynamic quantities: temperature meaning, entropy, and heat flow direction. Formally:

\textbf{(1) Velocity $\perp$ Temperature meaning} (from Section~\ref{sec:velocity_overlap}): The same velocity has different "temperature contribution" in different ensembles. A molecule with velocity $v$ contributes differently to temperature depending on the molecular properties and phase-lock cluster membership of the ensemble. Temperature is not determined by velocity alone but by velocity in context.

\textbf{(2) Heat $\perp$ Entropy} (from Section~\ref{sec:heat_transfer}): Heat can flow in either direction (hot $\to$ cold or cold $\to$ hot) in individual molecular collisions, while entropy always increases. Heat flow direction is determined by kinetic energy transfer (a fluctuating kinetic property), while entropy increase is determined by categorical completion (a monotonic categorical property). The two are independent: heat can flow "backward" (cold $\to$ hot) while entropy still increases.

\textbf{(3) Velocity $\perp$ Entropy} (this section): Changing velocities does not change spatial arrangements, and therefore does not change entropy. Entropy counts configurations (a spatial property), while velocity measures motion (a kinetic property). The two are independent: velocity can change while entropy remains constant (elastic collisions), or entropy can change while velocity distribution remains constant (spatial rearrangement without kinetic energy change).

These three orthogonalities are independent and mutually reinforcing. Together, they establish that velocity (the quantity the demon measures and manipulates) is orthogonal to all thermodynamic quantities constrained by the Second Law.
\end{theorem}

\begin{proof}
We have established each orthogonality in previous sections.

\textbf{(1) Velocity $\perp$ Temperature meaning:}
From Theorem~\ref{thm:context_dependent} (Section~\ref{sec:velocity_overlap}), temperature is context-dependent: the same velocity $v$ contributes differently to temperature in different molecular ensembles. Specifically, for a molecule with velocity $v$ and mass $m$, the kinetic energy is $E = (1/2) m v^2$, but the temperature contribution depends on the ensemble's molecular properties. In an ensemble of light molecules (small $m$), a given velocity $v$ corresponds to low kinetic energy and low temperature contribution. In an ensemble of heavy molecules (large $m$), the same velocity corresponds to high kinetic energy and high temperature contribution.

Moreover, from Theorem~\ref{thm:apparent_sorting}, molecules in the same phase-lock cluster have correlated kinetic energies due to shared molecular properties, not because kinetic energy determines clustering. A molecule's temperature contribution depends on its cluster membership (which determines the ensemble context), not on its velocity alone.

Therefore, velocity does not uniquely determine temperature meaning. The mapping from velocity to temperature is many-to-one and context-dependent. Velocity and temperature meaning are orthogonal in the sense that knowing velocity does not determine temperature contribution without additional context.

\textbf{(2) Heat $\perp$ Entropy:}
From Theorem~\ref{thm:heat_entropy_decoupling} (Section~\ref{sec:heat_transfer}), heat flow direction and entropy change are fundamentally decoupled. Heat is a kinetic property (energy transfer), while entropy is a categorical property (configuration count). In individual molecular collisions, heat can flow in either direction (hot $\to$ cold or cold $\to$ hot) due to thermal fluctuations, while entropy always increases through categorical completion (phase-lock correlation formation).

The three cases analyzed in Section~\ref{sec:heat_transfer} demonstrate the independence:
\begin{itemize}
    \item Case 1 (elastic bounce-back): Heat flows hot $\to$ cold, entropy increases.
    \item Case 2 (inelastic, cold accelerates): Heat flows hot $\to$ cold, entropy increases.
    \item Case 3 (inelastic, cold decelerates): Heat flows cold $\to$ hot, entropy increases.
\end{itemize}

In all cases, entropy increases, but heat flow direction varies. The two quantities are independent: knowing heat flow direction does not determine entropy change, and knowing entropy change does not determine heat flow direction.

\textbf{(3) Velocity $\perp$ Entropy:}
From Theorem~\ref{thm:velocity_arrangement} (this section), entropy is independent of velocity: $\partial S / \partial v_i = 0$. Entropy counts spatial arrangements $\Omega$, which depend on positions $\{\mathbf{r}_i\}$, not velocities $\{\mathbf{v}_i\}$. Changing velocities does not change the number of accessible configurations, and therefore does not change entropy.

From Theorem~\ref{thm:elastic_entropy}, elastic collisions can change velocity distribution (and temperature) without changing entropy. This demonstrates that velocity and entropy are independent variables: one can change while the other remains constant.

\textbf{Mutual reinforcement:}
The three orthogonalities are mutually reinforcing. (1) implies that velocity does not determine temperature, so sorting by velocity does not sort by temperature. (2) implies that even if sorting by velocity did create a temperature difference (which it doesn't, by (1)), this would not necessarily decrease entropy because heat flow and entropy are decoupled. (3) implies that sorting by velocity cannot change entropy directly because velocity and entropy are independent.

Together, the three orthogonalities establish that velocity is orthogonal to all thermodynamic quantities constrained by the Second Law (temperature, heat flow, entropy). The demon manipulates velocity, which is orthogonal to the quantities protected by thermodynamics. \qed
\end{proof}

\begin{remark}[Geometric Interpretation of Orthogonality]
\label{rem:geometric_orthogonality}
The three-way orthogonality can be visualized geometrically. Consider a three-dimensional space with axes representing velocity $v$, temperature $T$, and entropy $S$. In classical thermodynamics, these three quantities are assumed to be related: velocity determines kinetic energy, kinetic energy determines temperature, and temperature determines entropy (through the Boltzmann distribution). This would make the three axes dependent, collapsing the three-dimensional space to a one-dimensional manifold (a curve).

But Theorem~\ref{thm:three_orthogonality} establishes that the three axes are independent (orthogonal). Velocity does not uniquely determine temperature (orthogonality 1), temperature does not uniquely determine entropy (orthogonality 2, via heat-entropy decoupling), and velocity does not uniquely determine entropy (orthogonality 3). The three axes span a full three-dimensional space, not a one-dimensional manifold.

The demon's strategy assumes the axes are dependent: it measures velocity (assuming this determines temperature) and sorts by velocity (assuming this changes entropy). But the axes are orthogonal, so the strategy fails: measuring velocity does not determine temperature, and sorting by velocity does not change entropy. The demon is trying to move along one axis (velocity) to affect another axis (entropy), but the axes are orthogonal, so movement along one does not affect the other.
\end{remark}

\subsection{Implications for the Demon}

The three-way orthogonality has devastating implications for Maxwell's demon's strategy.

\begin{theorem}[Complete Demon Strategy Failure]
\label{thm:complete_failure}
The demon's strategy is orthogonal to entropy at every level. The demon measures velocity (orthogonal to entropy), sorts by velocity (does not change arrangements), aims to change temperature (not entropy), while entropy depends on arrangements (not velocity or temperature). Each step of the demon's strategy operates on quantities that are orthogonal to the quantity constrained by the Second Law.
\end{theorem}

\begin{proof}
The demon's complete operation can be decomposed into four steps:
\begin{equation}
\text{Measure } v \to \text{Sort by } v \to \text{Change } T \to \text{(hope to change) } S
\label{eq:demon_strategy}
\end{equation}

We show that each arrow in this chain is broken—each step fails to achieve its intended effect because the quantities involved are orthogonal.

\textbf{Step 1: Measure $v$ $\to$ (determine) $T$:}
The demon measures velocity $v$ with the intention of determining which molecules are "hot" (high temperature contribution) and which are "cold" (low temperature contribution). But from orthogonality (1), velocity does not uniquely determine temperature meaning. The same velocity $v$ can correspond to different temperature contributions depending on molecular properties and cluster membership (Theorem~\ref{thm:context_dependent}).

Therefore, measuring velocity does not determine temperature contribution. The demon cannot classify molecules as "hot" or "cold" based on velocity alone. The first step of the strategy fails.

\textbf{Step 2: Sort by $v$ $\to$ (rearrange) configurations:}
The demon sorts molecules by velocity, allowing fast molecules to pass one way and slow molecules the other way. The intention is to rearrange the spatial configuration, creating a sorted state (fast molecules in one chamber, slow molecules in another).

But sorting by velocity does not necessarily rearrange spatial configurations. Velocity is the time derivative of position, not position itself. Two molecules can have the same position but different velocities (they are at the same place but moving at different speeds). Sorting by velocity rearranges which molecule has which velocity, not which molecule is at which position.

Moreover, from Theorem~\ref{thm:retrieval_paradox}, velocity-based sorting cannot be maintained against thermal equilibration. Velocities randomize on the collision timescale ($\sim 10^{-10}$ s), faster than sorting can occur. The demon enters an infinite loop of sorting and retrieval, never achieving a stable sorted state.

Therefore, sorting by velocity does not effectively rearrange configurations. The second step of the strategy fails.

\textbf{Step 3: Change $T$ $\to$ (change) $S$:}
Suppose the demon succeeded in changing temperature (creating a temperature difference between chambers). The intention is that this temperature change would correspond to an entropy change (decrease), violating the Second Law.

But from orthogonality (2), heat flow (which creates temperature differences) is decoupled from entropy change. Heat can flow in either direction while entropy always increases (Theorem~\ref{thm:heat_entropy_decoupling}). From Theorem~\ref{thm:elastic_entropy}, temperature can change without entropy changing (elastic collisions redistribute kinetic energy without changing spatial configurations).

Therefore, changing temperature does not necessarily change entropy. The third step of the strategy fails.

\textbf{Step 4: (hope to change) $S$:}
The ultimate goal is to decrease entropy, violating the Second Law. But from orthogonality (3), entropy is independent of velocity: $\partial S / \partial v_i = 0$ (Theorem~\ref{thm:velocity_arrangement}). Entropy depends on spatial arrangements, not velocities.

The demon's entire strategy operates on velocities (measuring, sorting, redistributing). But velocities are orthogonal to entropy. No manipulation of velocities can change entropy directly.

Therefore, the demon cannot decrease entropy by manipulating velocities. The final step of the strategy fails.

\textbf{Complete failure:}
Every step of the demon's strategy is broken. The demon measures a quantity (velocity) that does not determine what it needs to know (temperature contribution). It sorts by a quantity (velocity) that does not rearrange what it needs to rearrange (spatial configurations). It aims to change a quantity (temperature) that does not affect what it needs to affect (entropy). And the quantity it manipulates (velocity) is orthogonal to the quantity constrained by the Second Law (entropy).

The demon's strategy is orthogonal to entropy at every level. It is fundamentally misdirected, operating in the wrong category (kinetic properties) to affect the protected category (configurational properties). \qed
\end{proof}

\begin{corollary}[Demon's Category Error]
\label{cor:demon_category_error}
The demon commits a category error: it treats velocity (a kinetic property) as if it determined entropy (a configurational property). This category error renders the demon's strategy ineffective regardless of how efficiently the demon processes information or how precisely it measures velocities.
\end{corollary}

\begin{proof}
A category error is the mistake of treating a property of one type as if it were a property of another type. Examples include treating a number as if it were a color, treating a spatial coordinate as if it were a temporal coordinate, or treating a kinetic property as if it were a configurational property.

The demon's implicit assumption is:
\begin{equation}
\text{Sort by velocity} \implies \text{Change entropy}
\end{equation}

This assumes that velocity and entropy are in the same category—that manipulating one affects the other. But velocity is a kinetic property (rate of change of position: $\mathbf{v} = d\mathbf{r}/dt$), while entropy is a configurational property (count of arrangements: $S = k_B \ln \Omega(\{\mathbf{r}_i\})$). These are categorically distinct:

\begin{itemize}
    \item \textbf{Kinetic properties:} Derivatives of position with respect to time ($\dot{\mathbf{r}}$, $\ddot{\mathbf{r}}$, ...), kinetic energy ($E_{\text{kin}} = (1/2) m v^2$), momentum ($\mathbf{p} = m \mathbf{v}$), temperature ($T \propto \langle v^2 \rangle$).
    \item \textbf{Configurational properties:} Functions of position alone ($V(\{\mathbf{r}_i\})$, $\Omega(\{\mathbf{r}_i\})$, $S(\{\mathbf{r}_i\})$), potential energy ($U(\{\mathbf{r}_i\})$), phase-lock network topology ($\phaselockgraph(\{\mathbf{r}_i\})$).
\end{itemize}

Kinetic and configurational properties are independent variables in phase space. Specifying kinetic properties does not uniquely determine configurational properties, and vice versa. Manipulating kinetic properties does not necessarily affect configurational properties.

The demon's strategy is a category error: it manipulates kinetic properties (velocities) with the intention of affecting configurational properties (entropy). This is like trying to change the color of an object by changing its weight—the two properties are in different categories and are not causally related.

The category error renders the demon's strategy ineffective regardless of implementation details. Even if the demon has perfect information (knows all velocities exactly), infinite computational power (can process information instantaneously), and perfect control (can open and close the door with arbitrary precision), it still cannot decrease entropy by sorting velocities because velocity and entropy are in different categories. The strategy is fundamentally misdirected. \qed
\end{proof}

\subsection{What Actually Changes Entropy}

Having established what does not change entropy (velocity manipulations), we now clarify what does change entropy.

\begin{proposition}[Entropy-Changing Operations]
\label{prop:entropy_changing}
Operations that change entropy are those that change spatial arrangements or create new categorical relationships. Specifically:

\textbf{(1) Mixing:} Bringing previously separated molecules into the same spatial region creates new phase-lock correlations. The number of accessible configurations increases because molecules that could not interact before can now interact. Entropy increases: $\Delta S_{\text{mix}} = k_B \ln(\Omega_{\text{mixed}} / \Omega_{\text{separated}}) > 0$ (Theorem~\ref{thm:mixing_entropy}).

\textbf{(2) Expansion:} Allowing molecules access to new spatial regions increases the volume of configuration space. The number of accessible positions increases proportionally to the volume: $\Omega \propto V^N$. Entropy increases: $\Delta S_{\text{expansion}} = N k_B \ln(V_{\text{final}} / V_{\text{initial}}) > 0$ for $V_{\text{final}} > V_{\text{initial}}$.

\textbf{(3) Chemical reaction:} Creating new molecular species with different properties changes the phase-lock network structure. New edges form (between reactants and products), and old edges break (between reactants). The categorical structure changes, increasing entropy through network densification (Theorem~\ref{thm:reaction_entropy}).

\textbf{(4) Phase transition:} Reorganizing spatial structure (e.g., from solid to liquid, or liquid to gas) changes the number of accessible configurations. Melting increases entropy because liquid configurations are more numerous than solid configurations (molecules can move freely rather than being constrained to lattice sites). Entropy increases: $\Delta S_{\text{melt}} = L / T_{\text{melt}} > 0$, where $L$ is the latent heat of melting.

All of these operations change spatial arrangements or categorical relationships, and therefore change entropy.
\end{proposition}

\begin{proposition}[Non-Entropy-Changing Operations]
\label{prop:non_entropy_changing}
Operations that do NOT change entropy (in isolation) are those that change kinetic properties without changing spatial arrangements. Specifically:

\textbf{(1) Elastic collisions:} Redistribute velocity without changing positions. The velocity distribution changes, but the spatial configuration remains the same. Entropy is unchanged: $\Delta S_{\text{elastic}} = 0$ (Theorem~\ref{thm:elastic_entropy}).

\textbf{(2) Velocity sorting:} Rearranges which molecule has which velocity, not which molecule is at which position. If the spatial configuration is unchanged (molecules remain in the same positions, only their velocities are swapped), then entropy is unchanged. However, in practice, velocity sorting requires spatial rearrangement (moving molecules from one chamber to another), which does change entropy. The entropy change comes from the spatial rearrangement, not from the velocity sorting per se.

\textbf{(3) Adiabatic compression/expansion:} Changes temperature via work (compressing or expanding the gas) without heat transfer. The process is reversible and isentropic: $\Delta S_{\text{adiabatic}} = 0$. Temperature changes (increases during compression, decreases during expansion), but entropy remains constant because the process is reversible (no dissipation, no mixing, no irreversible spatial rearrangement).

These operations change kinetic properties (velocities, kinetic energies, temperature) without changing configurational properties (positions, arrangements, entropy).
\end{proposition}

\begin{remark}[Demon Performs Non-Entropy-Changing Operation]
\label{rem:demon_non_entropy}
The demon performs velocity sorting, which is a non-entropy-changing operation (Proposition~\ref{prop:non_entropy_changing}, item 2). The demon's intention is to sort molecules by velocity to create a temperature difference, hoping that this will decrease entropy. But velocity sorting does not change entropy directly because entropy depends on spatial arrangements, not velocities.

In practice, the demon's operation does change entropy, but the change is an increase, not a decrease. The entropy increase comes from categorical completion (phase-lock correlations created by the door operation) and network densification (new edges added to the effective phase-lock network), not from velocity sorting. The demon's strategy is doubly flawed: it performs an operation (velocity sorting) that does not change entropy, and when entropy does change (due to categorical effects), it increases rather than decreases.
\end{remark}

\subsection{The Demon's Category Error}

We now formalize the concept of category error and prove that the demon commits such an error.

\begin{definition}[Category Error]
\label{def:category_error}
A \textbf{category error} is the mistake of treating a property of one ontological category as if it were a property of another category, or of assuming that properties in different categories are causally related when they are actually independent. Formally, let $\mathcal{A}$ and $\mathcal{B}$ be two ontological categories (e.g., kinetic vs. configurational, spatial vs. temporal, physical vs. mental). A category error occurs when one assumes:
\begin{equation}
\text{Manipulate property in } \mathcal{A} \implies \text{Change property in } \mathcal{B}
\end{equation}
when in fact properties in $\mathcal{A}$ and $\mathcal{B}$ are independent (orthogonal).
\end{definition}

\begin{theorem}[Demon's Category Error]
\label{thm:category_error}
Maxwell's Demon commits a category error: it treats velocity (a kinetic property in category $\mathcal{K}$) as if it determined entropy (a configurational property in category $\mathcal{C}$), when in fact the two categories are independent (orthogonal).
\end{theorem}

\begin{proof}
Define two ontological categories:
\begin{align}
\mathcal{K} &= \{\text{kinetic properties: } \mathbf{v}, E_{\text{kin}}, T, \mathbf{p}, \text{etc.}\} \label{eq:category_kinetic} \\
\mathcal{C} &= \{\text{configurational properties: } \mathbf{r}, \Omega, S, \phaselockgraph, \text{etc.}\} \label{eq:category_configurational}
\end{align}

The demon's implicit assumption is:
\begin{equation}
\text{Sort by } v \in \mathcal{K} \implies \text{Change } S \in \mathcal{C}
\label{eq:demon_assumption}
\end{equation}

This assumes that kinetic and configurational categories are causally related: manipulating a kinetic property (velocity) changes a configurational property (entropy).

But from Theorem~\ref{thm:velocity_arrangement}, velocity and entropy are independent:
\begin{equation}
\frac{\partial S}{\partial v} = 0 \quad \implies \quad v \perp S
\end{equation}

The two categories are orthogonal: properties in $\mathcal{K}$ (kinetic) do not determine properties in $\mathcal{C}$ (configurational). Velocity is a kinetic property (rate of change of position), while entropy is a configurational property (count of arrangements). The two are independent variables in phase space.

Therefore, the demon's assumption~\eqref{eq:demon_assumption} is false. Sorting by velocity does not change entropy because velocity and entropy are in different, orthogonal categories. The demon commits a category error: it treats kinetic properties as if they determined configurational properties, when in fact the two categories are independent.

The category error is analogous to other well-known category errors in philosophy:
\begin{itemize}
    \item Treating mental properties as if they were physical properties (the mind-body category error).
    \item Treating numbers as if they were spatial objects (the abstract-concrete category error).
    \item Treating temporal properties as if they were spatial properties (the time-space category error).
\end{itemize}

In each case, the error arises from conflating two independent categories and assuming causal relationships that do not exist. The demon's category error is of the same type: conflating kinetic and configurational categories and assuming that manipulating one affects the other. \qed
\end{proof}

\begin{remark}[Category Error vs. Information Error]
\label{rem:category_vs_information}
The category error (Theorem~\ref{thm:category_error}) is distinct from the information-theoretic errors addressed in traditional resolutions of Maxwell's demon (Landauer's principle, Bennett's erasure). The information-theoretic resolutions argue that the demon fails because it must pay an entropy cost to acquire, store, or erase information. These resolutions assume that the demon's strategy is correct in principle (sorting by velocity would decrease entropy if information were free) but fails in practice due to information costs.

The category error resolution argues that the demon's strategy is incorrect in principle: sorting by velocity cannot decrease entropy even if information is free, because velocity and entropy are in different, orthogonal categories. The demon is trying to affect a configurational property (entropy) by manipulating a kinetic property (velocity), which is impossible because the two categories are independent.

The category error is more fundamental than the information error. Even if all information-theoretic objections were resolved (infinite memory, zero erasure cost, perfect measurement), the demon would still fail due to the category error. The demon's strategy is misdirected at a deeper level than information theory can address.
\end{remark}

\subsection{Summary}

Velocity and entropy are orthogonal: velocity is a kinetic property (rate of change of position), while entropy is a configurational property (count of arrangements). The two are independent variables, and changing one does not necessarily change the other.

Key results established in this section:

\textbf{(1) Entropy counts spatial arrangements, not velocities.} The Boltzmann entropy $S = k_B \ln \Omega$ counts the number of accessible configurations $\Omega$, which depends on positions $\{\mathbf{r}_i\}$, not velocities $\{\mathbf{v}_i\}$ (Theorem~\ref{thm:velocity_arrangement}).

\textbf{(2) A snapshot is velocity-blind.} The same spatial configuration can exist at any temperature. A configurational snapshot $\mathcal{S} = \{\mathbf{r}_i\}$ is compatible with any velocity distribution $\{\mathbf{v}_i\}$ (Theorem~\ref{thm:snapshot_blind}).

\textbf{(3) Elastic collisions change velocity without changing entropy.} Elastic collisions redistribute kinetic energy (changing temperature) without changing spatial configurations (leaving entropy unchanged) (Theorem~\ref{thm:elastic_entropy}).

\textbf{(4) Phase-lock networks depend on positions, not velocities.} Categorical structure is determined by spatial relationships (Van der Waals forces, dipole interactions, vibrational coupling), all of which depend on positions, not velocities (Theorem~\ref{thm:categorical_velocity}).

\textbf{(5) Three-way orthogonality.} Velocity is orthogonal to temperature meaning (same velocity has different temperature contribution in different contexts), to heat flow direction (heat can flow either way while entropy increases), and to entropy (changing velocity does not change arrangements) (Theorem~\ref{thm:three_orthogonality}).

\textbf{(6) Demon's strategy is orthogonal to entropy.} The demon measures velocity (orthogonal to entropy), sorts by velocity (does not change arrangements), aims to change temperature (not entropy), while entropy depends on arrangements (not velocity or temperature). Every step of the demon's strategy operates on quantities orthogonal to the quantity constrained by the Second Law (Theorem~\ref{thm:complete_failure}).

\textbf{(7) Demon commits a category error.} The demon treats velocity (kinetic property) as if it determined entropy (configurational property), when in fact the two categories are independent. This category error renders the demon's strategy fundamentally ineffective (Theorem~\ref{thm:category_error}).

The most fundamental defeat of Maxwell's demon is this: the demon manipulates a quantity (velocity) that is categorically orthogonal to the quantity (entropy) protected by the Second Law. The demon's entire strategy operates in the wrong category—kinetic rather than configurational. No amount of information processing, measurement precision, or computational power can overcome this fundamental misdirection. The demon is trying to affect entropy by manipulating velocities, which is impossible because velocity and entropy are independent, orthogonal quantities belonging to different ontological categories.

\begin{equation}
\boxed{
\begin{aligned}
\text{Velocity} &: \text{kinetic property (rate of change of position)} \\
\text{Entropy} &: \text{configurational property (count of arrangements)} \\
\frac{\partial S}{\partial v} &= 0 \quad \text{(orthogonal)} \\
\text{Demon's strategy} &: \text{category error (manipulates kinetic to affect configurational)}
\end{aligned}
}
\end{equation}

%==============================================================================
\section{The Dissolution of Maxwell's Demon}
\label{sec:dissolution}
%==============================================================================

We now synthesize the results established in the preceding sections to provide a systematic dissolution of Maxwell's demon paradox. The dissolution proceeds through nine independent arguments, each sufficient to refute the demon's purported ability to violate the Second Law. These nine arguments build upon the fundamental insights developed in Sections~\ref{sec:velocity_overlap} (velocity-temperature non-correspondence) and~\ref{sec:velocity_entropy} (velocity-entropy orthogonality), which establish that the demon's measurement strategy is fundamentally incoherent. The demon does not exist as an information-processing agent that decreases entropy; rather, the thought experiment describes categorical completion through phase-lock network topology—a physical process that requires no agent, no information acquisition, and no violation of thermodynamic principles. The apparent "demon" is a projection artifact arising from observing only one face of a two-faced information structure. When the conjugate face (categorical structure) is made visible, the demon dissolves, revealing that the "sorting" attributed to the demon is actually categorical completion following network topology, which always increases entropy.

\subsection{Restatement of the Paradox}

Maxwell's thought experiment, as originally formulated in 1867 and refined in subsequent discussions, posits a being (the "demon") that performs the following sequence of operations:

\textbf{Step 1: Observation.} The demon observes molecules approaching a door (aperture) between two chambers A and B, initially at thermal equilibrium with uniform temperature $T_0$.

\textbf{Step 2: Measurement.} The demon measures the velocities of approaching molecules to classify them as "fast" (kinetic energy above average) or "slow" (kinetic energy below average).

\textbf{Step 3: Selective opening.} The demon opens the door selectively: it allows fast molecules to pass from chamber B to chamber A and slow molecules to pass from chamber A to chamber B, while blocking molecules moving in the opposite directions.

\textbf{Step 4: Temperature difference.} Through repeated selective openings, the demon creates a temperature difference between the chambers: chamber A becomes hotter (higher average kinetic energy) and chamber B becomes colder (lower average kinetic energy), starting from an initially uniform temperature.

The paradox arises because this process appears to violate the Second Law of Thermodynamics. In the Clausius formulation, the Second Law states that heat cannot spontaneously flow from a colder body to a hotter body. Yet the demon has created a temperature difference (a hot chamber and a cold chamber) from an initially uniform state, apparently transferring heat from cold to hot without expending work. In the entropy formulation, the Second Law states that the entropy of an isolated system cannot decrease: $\Delta S \geq 0$. Yet the demon has created a more ordered state (molecules sorted by velocity) from a less ordered state (uniform distribution), apparently decreasing entropy: $\Delta S < 0$.

The paradox has generated extensive literature attempting to resolve it through information-theoretic arguments (Szilard 1929, Landauer 1961, Bennett 1982), quantum measurement constraints, and thermodynamic costs of computation. These resolutions argue that the demon must pay an entropy cost to acquire, store, or erase information about molecular velocities, and this cost compensates for the apparent entropy decrease in the gas. However, as we now demonstrate, these information-theoretic resolutions, while correct within their framework, are unnecessary: the paradox dissolves completely when categorical structure is properly accounted for, and when the fundamental incoherence of velocity-based sorting is recognized.

\subsection{Five Decisive Insights}

Before analyzing the demon's purported operations in detail, we establish five fundamental results that independently dissolve the paradox. Each result is sufficient to refute the demon's ability to violate the Second Law; together, they form an overdetermined proof that the paradox rests on conceptual errors. These five insights incorporate the results from Sections~\ref{sec:velocity_overlap} and~\ref{sec:velocity_entropy}, which establish that the demon's measurement and sorting strategy is fundamentally incoherent before any thermodynamic analysis is required.

\begin{theorem}[Velocity-Temperature Non-Correspondence]
\label{thm:velocity_temperature_non_correspondence}
The demon cannot sort by temperature because velocity does not determine temperature contribution. From Section~\ref{sec:velocity_overlap}, the same velocity has different "temperature meaning" in different ensembles. A molecule classified as "hot" in one chamber becomes "cold" upon transfer to another chamber, even though its velocity is unchanged. The demon's sorting strategy is conceptually incoherent.
\end{theorem}

\begin{proof}
From Theorem~\ref{thm:context_dependent}, for the same velocity $v$ in two ensembles at temperatures $T_A < T_B$:
\begin{equation}
P_{T_A}(v) > P_{T_B}(v)
\end{equation}

The same velocity represents a higher percentile (is "faster" relative to the ensemble) in the colder container than in the hotter container.

From Theorem~\ref{thm:sorting_paradox}, when the demon transfers a molecule from container A to container B based on its velocity $v^*$ (classified as "fast" in A), the molecule's percentile changes upon transfer. If $v^*$ is in the ambiguous overlap region where $P_{T_A}(v^*) > 0.5$ but $P_{T_B}(v^*) < 0.5$, the molecule is "hot" in A but "cold" in B. The demon's sorting achieves the opposite of its intention.

From Theorem~\ref{thm:temp_sort_impossible}, the demon cannot sort by temperature because temperature is not a molecular property (Corollary~\ref{cor:no_molecular_temperature}), velocity determines only kinetic energy not temperature contribution (context-dependent), temperature contribution is ensemble-relative (changes upon transfer), and sorting changes ensemble composition hence all molecules' contributions (each transfer invalidates previous classifications).

The demon's strategy is conceptually incoherent: it attempts to sort by a property (temperature) that molecules do not possess, using a measurement (velocity) that does not determine the property even statistically. \qed
\end{proof}

\begin{theorem}[Velocity-Entropy Orthogonality]
\label{thm:velocity_entropy_orthogonality}
The demon manipulates a quantity (velocity) that is categorically orthogonal to the quantity (entropy) protected by the Second Law. From Section~\ref{sec:velocity_entropy}, velocity and entropy are independent variables: changing velocities does not change spatial arrangements and, therefore, does not change entropy.
\end{theorem}

\begin{proof}
From Theorem~\ref{thm:velocity_arrangement}, the number of spatial arrangements $\Omega$ is independent of molecular velocities:
\begin{equation}
\frac{\partial \Omega}{\partial v_i} = 0 \quad \forall i
\end{equation}

Entropy $S = k_B \ln \Omega$ counts arrangements, which depend on positions $\{\mathbf{r}_i\}$, not velocities $\{\mathbf{v}_i\}$.

From Theorem~\ref{thm:elastic_entropy}, elastic collisions can change velocity distribution (and temperature) without changing entropy. Temperature and entropy are independent variables.

From Theorem~\ref{thm:category_error}, the demon commits a category error: it treats velocity (kinetic property) as if it determined entropy (configurational property), when in fact the two categories are independent. The demon manipulates kinetic properties to affect configurational properties, which is impossible because the two categories are orthogonal.

From Theorem~\ref{thm:complete_failure}, the demon's strategy is orthogonal to entropy at every level: the demon measures velocity (orthogonal to entropy), sorts by velocity (doesn't change arrangements), aims to change temperature (not entropy), while entropy depends on arrangements (not velocity or temperature). \qed
\end{proof}

\begin{theorem}[Temporal Triviality of the Demon]
\label{thm:temporal_triviality}
The demon is temporally redundant: any configuration the demon purportedly creates will occur naturally through thermal fluctuations given sufficient time. The demon does not create anything that would not occur spontaneously; it merely (supposedly) accelerates what statistical mechanics already predicts. Since the Second Law constrains what can happen (entropy cannot decrease), not how quickly it happens (the rate of entropy change), acceleration does not constitute a violation.
\end{theorem}

\begin{proof}
In statistical mechanics, the probability of any microscopic configuration $\Gamma$ (specifying positions and momenta of all molecules) is given by the canonical ensemble distribution:
\begin{equation}
P(\Gamma) = \frac{1}{Z} e^{-E(\Gamma)/k_B T}
\label{eq:boltzmann_distribution}
\end{equation}

Crucially, $P(\Gamma) > 0$ for every configuration $\Gamma$, including configurations that appear highly ordered or "sorted." The sorted configuration $\Gamma_{\text{sorted}}$ has probability:
\begin{equation}
P(\Gamma_{\text{sorted}}) = \frac{1}{Z} e^{-E(\Gamma_{\text{sorted}})/k_B T} > 0
\end{equation}

The Poincaré recurrence theorem guarantees that an isolated system will return arbitrarily close to any configuration in finite time. For any initial configuration $\Gamma_0$, any target configuration $\Gamma_{\text{target}}$, and any tolerance $\epsilon > 0$, there exists a recurrence time $T_{\text{rec}} < \infty$ such that:
\begin{equation}
|\Gamma(T_{\text{rec}}) - \Gamma_{\text{target}}| < \epsilon
\end{equation}

Therefore, the demon does not create anything that would not occur naturally. The demon merely (supposedly) accelerates a rare fluctuation. But acceleration is not violation—the Second Law constrains what can happen, not how quickly. \qed
\end{proof}

\begin{theorem}[Phase-Lock Temperature Independence]
\label{thm:phase_lock_temperature_independence}
The same phase-lock network topology (spatial arrangement and categorical structure) can exist at any temperature. A "snapshot" of the system—a frozen configuration with definite molecular positions and phase-lock relationships—is temperature-independent. Temperature is a statistical property of velocity distributions, not a determinant of spatial or categorical structure. Therefore, rearrangement of molecules according to phase-lock topology (categorical completion) is not "sorting by temperature."
\end{theorem}

\begin{proof}
Consider a snapshot of the system at time $t$: a frozen configuration with definite molecular positions $\{\mathbf{r}_i\}_{i=1}^N$ and phase-lock relationships encoded in the network $\phaselockgraph = (V, E)$.

The phase-lock network depends only on positions:
\begin{equation}
\phaselockgraph = \phaselockgraph(\{\mathbf{r}_i\}) \quad \text{(independent of } \{\mathbf{v}_i\})
\end{equation}

This is because phase-lock relationships are determined by spatial proximity and coupling strength, which depend on intermolecular distances $r_{ij} = |\mathbf{r}_i - \mathbf{r}_j|$ and molecular properties, not on velocities.

Temperature depends only on velocities:
\begin{equation}
T = \frac{2}{3 N k_B} \sum_{i=1}^{N} \frac{1}{2} m_i |\mathbf{v}_i|^2 \quad \text{(independent of } \{\mathbf{r}_i\})
\end{equation}

Since positions and velocities are independent variables, the same spatial arrangement (and therefore the same phase-lock network) can occur with different velocity distributions, corresponding to different temperatures. The phase-lock network topology is temperature-independent.

This has a profound implication: rearranging molecules according to categorical pathways (phase-lock adjacency) is not "sorting by temperature." The same categorical rearrangement occurs whether the system is at 100 K or 1000 K. \qed
\end{proof}

\begin{theorem}[The Retrieval Paradox]
\label{thm:retrieval_paradox}
A demon that sorts molecules by velocity is self-defeating: thermal equilibration continuously randomizes velocities on the collision timescale, requiring infinite retrieval operations to maintain the sorted state. The demon cannot "keep up" with thermal relaxation. Velocity-based sorting is futile because the sorting timescale vastly exceeds the equilibration timescale.
\end{theorem}

\begin{proof}
Suppose the demon successfully "sorts" molecule $A$ into the hot chamber based on its velocity $v_A > v_{\text{threshold}}$ at time $t_0$.

After sorting, molecule $A$ undergoes collisions with collision frequency:
\begin{equation}
\nu_{\text{collision}} = n \sigma \langle v \rangle \approx 10^{10} \text{ s}^{-1}
\end{equation}

After a collision at time $t_1 = t_0 + \tau_{\text{collision}}$, where $\tau_{\text{collision}} \sim 10^{-10}$ s, molecule $A$ has a new velocity $v_A'$. With probability $\sim 1/2$, molecule $A$ has become "slow" and is now in the wrong chamber.

The demon must detect this and retrieve molecule $A$. But during retrieval, molecule $A$ undergoes additional collisions. The demon enters an infinite loop.

For $N$ molecules, the demon must process:
\begin{equation}
\text{Operations per second} \sim N \cdot \nu_{\text{collision}} \sim 10^{23} \times 10^{10} = 10^{33} \text{ s}^{-1}
\end{equation}

The sorting timescale is:
\begin{equation}
\tau_{\text{sorting}} \sim N \cdot \tau_{\text{operation}} \sim 10^{13} \text{ s}
\end{equation}

The equilibration timescale is:
\begin{equation}
\tau_{\text{equilibration}} \sim \tau_{\text{collision}} \sim 10^{-10} \text{ s}
\end{equation}

The ratio is:
\begin{equation}
\frac{\tau_{\text{sorting}}}{\tau_{\text{equilibration}}} \sim 10^{23}
\end{equation}

By the time the demon has sorted a significant fraction of molecules, the first molecules sorted have already equilibrated. The demon can never achieve a fully sorted state. \qed
\end{proof}

\begin{corollary}[Velocity Is the Wrong Criterion]
\label{cor:wrong_criterion}
The demon's failure is not due to information costs, measurement disturbance, or quantum uncertainty. It fails because velocity is not a stable molecular property—it changes on the collision timescale $\tau_{\text{collision}} \sim 10^{-10}$ s, which is much faster than any sorting operation. The demon has chosen the wrong criterion for sorting, independent of any thermodynamic considerations.
\end{corollary}

\subsection{The Dissolution}

With Theorems~\ref{thm:velocity_temperature_non_correspondence}--\ref{thm:retrieval_paradox} established, we now show that each step of the demon's purported operation is either unnecessary, misconceived, or automatically entropy-increasing. The dissolution proceeds through four additional arguments that address the demon's specific operations.

\begin{theorem}[Dissolution of Observation]
\label{thm:dissolution_observation}
The demon's "observation" of molecular velocities is unnecessary because phase-lock network topology encodes categorical structure without measurement. The system's categorical structure—which states are accessible from which—is fully determined by network topology $\phaselockgraph$, which depends on spatial configuration, not on kinetic properties. The "information" about molecular arrangement is structural, encoded in the network, not acquired through observation.
\end{theorem}

\begin{proof}
From Theorem~\ref{thm:kinetic_independence}, the phase-lock network $\phaselockgraph = (V, E)$ is determined by spatial configuration $\{\mathbf{r}_i\}$ and molecular properties, not by velocities $\{\mathbf{v}_i\}$. The network topology is given by:
\begin{equation}
E = \{(m_i, m_j) : \kappa_{ij}(\mathbf{r}_i, \mathbf{r}_j) > \kappa_{\text{threshold}}\}
\end{equation}
where $\kappa_{ij}$ depends on positions, not velocities.

From Theorem~\ref{thm:phase_lock_accessibility}, categorical accessibility is determined by network topology:
\begin{equation}
\accessible(C_i) = \{C_j \in \catspace : (C_i, C_j) \in E_{\text{PL}}\}
\end{equation}

Moreover, from Theorem~\ref{thm:velocity_temperature_non_correspondence}, even if the demon observes velocities, this observation does not determine temperature contribution because velocity meaning is context-dependent. The observation is not only unnecessary but also insufficient for the demon's purported goal.

Therefore, the system's categorical structure is fully determined by $\phaselockgraph$, which is determined by spatial configuration. Velocities are not required to determine categorical structure. The "information" about molecular arrangement is structural, encoded in the network topology, not acquired through observation. \qed
\end{proof}

\begin{theorem}[Dissolution of Decision]
\label{thm:dissolution_decision}
The demon's "decision" to open or close the door is unnecessary because categorical completion follows network topology deterministically. The selection of which categorical state to complete next is determined by the categorical ordering and phase-lock adjacency, not by a deliberative decision made by an agent. Categorical dynamics are self-executing.
\end{theorem}

\begin{proof}
From Theorem~\ref{thm:information_free}, categorical selection is determined by minimizing the categorical distance among accessible states:
\begin{equation}
C^* = \argmin_{C \in \accessible(C_{\text{prev}}) \cap [C]_{\text{spatial}}} d_{\catspace}(C, C_{\text{prev}})
\end{equation}

This selection is determined by three factors: the previous categorical state $C_{\text{prev}}$ (given by the system's history), the network topology determining $\accessible(C_{\text{prev}})$ (structural, encoded in $\phaselockgraph$), and the categorical distance metric $d_{\catspace}$ (defined by network topology).

None of these factors involves a deliberative decision by an agent. The selection is deterministic (given $C_{\text{prev}}$ and $\phaselockgraph$, the next state $C^*$ is uniquely determined) or stochastic (if multiple states have equal minimum distance). In either case, no deliberative decision is required. The categorical dynamics are self-executing: the system follows the topological pathway automatically. \qed
\end{proof}

\begin{theorem}[Dissolution of Sorting]
\label{thm:dissolution_sorting}
The demon's "sorting" by temperature is a misinterpretation of categorical completion through phase-lock pathways. When molecules appear "sorted by temperature," they are actually following categorical pathways determined by phase-lock topology, clustering by phase-lock adjacency (categorical property), not by kinetic similarity (kinetic property). Moreover, from Section~\ref{sec:velocity_overlap}, the demon cannot sort by temperature because temperature is not a molecular property and velocity does not determine temperature contribution.
\end{theorem}

\begin{proof}
From Theorem~\ref{thm:demon_cannot_sort}, temperature is not a molecular attribute but an emergent macroscopic property: $T = \mathcal{T}[\{v_1, v_2, \ldots, v_N\}]$. Individual molecules have kinetic energies, not temperatures.

From Corollary~\ref{cor:no_molecular_temperature}, temperature is a functional of the entire velocity distribution, not a function of individual velocities. An individual molecule contributes to temperature, but the contribution's significance depends on the ensemble context.

From Theorem~\ref{thm:kinetic_independence}, kinetic energy does not determine phase-lock network topology: $\partial \phaselockgraph / \partial E_{\text{kinetic}} = 0$.

From Theorem~\ref{thm:apparent_sorting}, molecules in the same phase-lock cluster have correlated kinetic energies because they share molecular properties (mass, polarizability), not because kinetic energy determines clustering:
\begin{equation}
\text{Cov}(E_i, E_j | i, j \in \mathcal{K}_\alpha) > 0
\end{equation}

From Theorem~\ref{thm:category_change}, when molecules transfer between ensembles, their velocity category changes even though velocity is unchanged. A molecule that is "hot" in A becomes "cold" in B.

When molecules appear "sorted by temperature," they are actually sorted by phase-lock cluster membership. The kinetic energy correlation is a consequence of cluster membership, not a cause. The demon does not create the correlation by sorting; it reveals a pre-existing correlation by completing categorical states that make cluster structure visible. \qed
\end{proof}

\begin{theorem}[Dissolution of Second Law Violation]
\label{thm:dissolution_second_law}
The apparent decrease in entropy attributed to the demon's operation is an artifact of ignoring categorical degrees of freedom. When categorical entropy is properly accounted for, total entropy increases. The Second Law is not violated.
\end{theorem}

\begin{proof}
From Theorem~\ref{thm:sorting_density}, the demon operation—categorical completion through phase-lock pathways—increases network density:
\begin{equation}
|E(\gamma(t_{\text{final}}))| > |E(\gamma(t_{\text{initial}}))|
\end{equation}

From Proposition~\ref{prop:entropy_edge_density}, entropy is proportional to edge count: $S_{\text{categorical}} \propto k_B |E|$.

Therefore, categorical entropy increases: $\Delta S_{\text{categorical}} = k_B \Delta |E| > 0$.

From Corollary~\ref{cor:second_law}, total entropy is:
\begin{equation}
S_{\text{total}} = S_{\text{spatial}} + S_{\text{categorical}}
\end{equation}

The spatial entropy may decrease (molecules spatially segregated), but the categorical entropy increase dominates:
\begin{equation}
\Delta S_{\text{total}} = \Delta S_{\text{spatial}} + \Delta S_{\text{categorical}} \sim -k_B N \log 2 + k_B N \langle d \rangle > 0
\end{equation}
since $\langle d \rangle \sim 5$-$10 \gg \log 2 \approx 0.7$.

Moreover, from Theorem~\ref{thm:velocity_entropy_orthogonality}, the demon's velocity-based sorting does not change entropy directly because velocity and entropy are orthogonal. Any entropy change comes from spatial rearrangement (which increases categorical entropy), not from velocity manipulation.

The Second Law is preserved. The paradox arose from incomplete entropy accounting. \qed
\end{proof}

\subsection{The Demon as Categorical Completion}

\begin{theorem}[Identity Theorem]
\label{thm:identity}
Maxwell's Demon is identical to categorical completion through phase-lock network topology. Every operation attributed to the demon corresponds to a categorical process that requires no external agent, no information acquisition, and no violation of thermodynamics:
\begin{equation}
\boxed{\text{``Maxwell's Demon''} \equiv \text{Categorical Completion}(\phaselockgraph)}
\label{eq:demon_identity}
\end{equation}
\end{theorem}

\begin{proof}
We establish a complete correspondence between demon operations and categorical processes:

\begin{center}
\begin{tabular}{l|l}
\textbf{Demon Operation} & \textbf{Categorical Process} \\
\hline
Observe molecule & Complete categorical state $C_i$ \\
Measure velocity & (Unnecessary—topology determines accessibility) \\
Classify fast/slow & Identify phase-lock cluster membership \\
Open door & Make adjacent states $\accessible(C_i)$ accessible \\
Close door & Categorical irreversibility prevents return \\
Sort molecules & Follow phase-lock pathways \\
Create $\Delta T$ & Reveal cluster structure (correlated with $T$)
\end{tabular}
\end{center}

Every demon operation has a categorical counterpart that requires no external agent (categorical completion is self-executing), requires no information acquisition (categorical structure is structural), follows automatically from network topology (categorical pathways are determined by phase-lock adjacency), and increases entropy rather than decreasing it (network densification increases categorical entropy).

Moreover, from Sections~\ref{sec:velocity_overlap} and~\ref{sec:velocity_entropy}, the demon's purported operations are not only unnecessary but conceptually incoherent: velocity does not determine temperature contribution (Theorem~\ref{thm:velocity_temperature_non_correspondence}), and velocity is orthogonal to entropy (Theorem~\ref{thm:velocity_entropy_orthogonality}).

The demon is not needed because categorical completion through phase-lock topology accomplishes the same apparent effect. But this is not a demon "in disguise"—it is the recognition that no demon was ever required. \qed
\end{proof}

\begin{figure*}[htbp]
\centering
\includegraphics[width=0.95\textwidth]{figures/maxwell_demon_resolution_panel.png}
\caption{\textbf{Maxwell's Demon Resolution: Entropy Increases for ANY Molecule Transfer Regardless of Velocity.}
The figure demonstrates that entropy increases in both containers for slow, medium, and fast molecules, with identical entropy changes regardless of velocity ($\Delta S_A > 0$ and $\Delta S_B > 0$ in all cases). Three scenarios are shown vertically: slow molecule ($v \approx 100$ m/s, top row), medium molecule ($v \approx 400$ m/s, middle row), and fast molecule ($v \approx 800$ m/s, bottom row). Each scenario progresses through three stages (left to right): before transfer, during transfer, and after transfer, with quantified entropy changes shown in the rightmost panels.

\textbf{Slow Molecule (v ≈ 100 m/s):}
\textbf{Before:} Container A (green box) contains multiple green molecules with the partition door closed. Container B (purple box) contains multiple purple molecules. The networks are separate.
\textbf{During:} The blue molecule (highlighted) transfers from A to B through the open door.
\textbf{After:} Container A has $N-1$ molecules in the reconfigured network. Container B has $N+1$ molecules with new phase-lock edges (the purple cluster is denser).
\textbf{Entropy changes:} $\Delta S_A = +0.07 \times 10^{-21}$ J/K (categorical completion), $\Delta S_B = +0.28 \times 10^{-21}$ J/K (mixing densification). Both positive (✓ BOTH > 0).

\textbf{Medium Molecule (v ≈ 400 m/s):}
\textbf{Before:} Same initial configuration as the slow case.
\textbf{During:} The orange molecule (highlighted, medium velocity) transfers from A to B.
\textbf{After:} Container A reconfigures with $N-1$ molecules. Container B densifies with $N+1$ molecules.
\textbf{Entropy changes:} $\Delta S_A = +0.07 \times 10^{-21}$ J/K, $\Delta S_B = +0.28 \times 10^{-21}$ J/K. Identical to slow case (✓ BOTH > 0).

\textbf{Fast Molecule (v ≈ 800 m/s):}
\textbf{Before:} Same initial configuration.
\textbf{During:} The red molecule (highlighted, fast velocity) transfers from A to B.
\textbf{After:} Container A reconfigures, and Container B densifies.
\textbf{Entropy changes:} $\Delta S_A = +0.07 \times 10^{-21}$ J/K, $\Delta S_B = +0.28 \times 10^{-21}$ J/K. Again identical (✓ BOTH > 0).

\textbf{Result (bottom text box):} Entropy increases in BOTH containers regardless of molecular velocity. Container A: Categorical completion—network reconfigures as one molecule leaves, increasing edge density and categorical entropy ($\Delta S_A > 0$). Container B: Mixing-type densification—new phase-lock edges form as one molecule enters, increasing total edges and mixing entropy ($\Delta S_B > 0$). The demon CANNOT decrease entropy. The velocity of the transferred molecule is irrelevant to the entropy change; only the topological reconfiguration matters. Maxwell's paradox is dissolved: the apparent ``sorting'' by velocity cannot reduce total entropy because both containers undergo positive entropy changes through categorical mechanisms that are independent of kinetic energy.}
\label{fig:maxwell_demon_resolution}
\end{figure*}

\subsection{Why Maxwell Saw a Demon: Information Complementarity}

\begin{theorem}[Information Complementarity]
\label{thm:information_complementarity}
Information has two conjugate faces (kinetic and categorical) that cannot be simultaneously observed with equal precision. Maxwell saw a "demon" because he was observing the kinetic face of information (velocities, temperatures) while the dynamics of the conjugate categorical face (phase-lock networks, categorical completion) remained hidden.
\end{theorem}

\begin{proof}
Every categorical state has a conjugate representation in two faces:
\begin{align}
\mathbf{S}_{\text{kinetic}} &= (S_{k}, S_{t}, S_{e})_{\text{kinetic}} \quad \text{(observable kinetic face)} \\
\mathbf{S}_{\text{categorical}} &= (S_{k}, S_{t}, S_{e})_{\text{categorical}} \quad \text{(hidden categorical face)}
\end{align}

This conjugacy is not a quantum effect but a classical measurement constraint, analogous to ammeter/voltmeter complementarity in electrical circuits.

Maxwell observed the kinetic face: molecules with velocities, kinetic energies, temperatures, and spatial configurations. The categorical face—phase-lock network topology, cluster structure, categorical pathways, and categorical completion dynamics—was hidden from his view.

When you observe only one face, the dynamics of the conjugate face appear as external intervention. The structured, non-random "sorting" on the kinetic face appears to require an intelligent agent. Hence, the demon.

But the "demon" is not an agent—it is the projection of categorical dynamics onto the kinetic face:
\begin{equation}
\text{``Demon''} = \Pi_{\text{kinetic}}\left(\frac{d\mathbf{S}_{\text{categorical}}}{dt}\right)
\end{equation}

The categorical dynamics (categorical completion following network topology) are projected onto the kinetic face, where they appear as deliberate sorting by an intelligent agent. \qed
\end{proof}

\begin{corollary}[The Demon as Projection]
\label{cor:demon_projection}
Maxwell's Demon is the projection of hidden categorical dynamics onto the observable kinetic face. The demon is not an entity but a projection artifact arising from incomplete observation.
\end{corollary}

\begin{theorem}[Face-Switching Dissolves the Demon]
\label{thm:face_switching}
If Maxwell had been able to observe the categorical face instead of the kinetic face, no demon would have appeared. The "sorting" would be revealed as categorical completion through phase-lock pathways—a physical process requiring no agent.
\end{theorem}

\begin{proof}
On the kinetic face, molecules appear to be sorted by velocity. An agent seems required to select which molecules pass through the door.

On the categorical face, molecules are nodes in a phase-lock network. Categorical completion follows network adjacency. No selection occurs in the sense of choosing among alternatives; the system follows topological pathways determined by network structure.

The same physical process appears differently on different faces:

\begin{center}
\begin{tabular}{l|l}
\textbf{Kinetic Face (Maxwell's View)} & \textbf{Categorical Face (Phase-Lock View)} \\
\hline
Molecules moving with velocities & Nodes in phase-lock network \\
"Fast" and "slow" classification & Phase-lock cluster membership \\
Door opening/closing & Adjacent states becoming accessible \\
Agent making decisions & Topological navigation (automatic) \\
Apparent entropy decrease & Categorical entropy increase \\
Demon required & No agent required
\end{tabular}
\end{center}

The demon is an artifact of the observable face, not a feature of the physical process. \qed
\end{proof}

\subsection{Why the Paradox Persisted}

\begin{proposition}[Source of the Paradox]
\label{prop:paradox_source}
Maxwell's Demon paradox persisted for 150 years due to four conceptual errors that prevented recognition of the categorical resolution: observing only one face of information, treating molecules as independent, privileging kinetic energy, and incomplete entropy accounting. These errors were compounded by the failure to recognize the fundamental incoherence of velocity-based sorting established in Sections~\ref{sec:velocity_overlap} and~\ref{sec:velocity_entropy}.
\end{proposition}

\begin{proof}
\textbf{(1) Single-face observation:}
Maxwell and subsequent analysts observed molecular systems through the kinetic face: velocities, kinetic energies, temperatures, and spatial configurations. The conjugate categorical face—phase-lock networks, cluster structure, categorical pathways—was not accessible to their theoretical framework.

Classical thermodynamics and kinetic theory focus on macroscopic observables (temperature, pressure) and molecular velocities. Neither framework includes concepts of phase-lock networks or categorical structure. When dynamics occur on the hidden face, they must be explained through the observable face. The most parsimonious explanation for structured, non-random "sorting" is an intelligent agent.

\textbf{(2) Independent particle assumption:}
Classical statistical mechanics treats molecules as independent particles whose only interactions are instantaneous collisions. This approximation ignores the persistent phase-lock relationships through Van der Waals forces and dipole interactions that create network structure.

With independent particles, "sorting" would require external information to distinguish molecules. With networked particles, categorical structure already distinguishes molecules through phase-lock relationships.

\textbf{(3) Kinetic energy privilege:}
The thought experiment assumes the demon sorts by velocity—a kinetic property. This privileges kinetic energy as the fundamental variable. But Theorem~\ref{thm:kinetic_independence} establishes that phase-lock networks are kinetically independent: $\partial \phaselockgraph / \partial E_{\text{kinetic}} = 0$.

Moreover, from Section~\ref{sec:velocity_entropy}, velocity and entropy are orthogonal: the demon cannot affect entropy by manipulating velocity because the two quantities are in different categories (kinetic vs. configurational).

\textbf{(4) Incomplete entropy accounting:}
Traditional analyses compute spatial entropy $S_{\text{spatial}}$ while ignoring categorical entropy $S_{\text{categorical}}$. Since categorical completion always increases $S_{\text{categorical}}$, and the increase dominates any decrease in $S_{\text{spatial}}$, total entropy increases.

\textbf{(5) Failure to recognize velocity-based sorting incoherence:}
Even before thermodynamic analysis, the demon's strategy is conceptually incoherent. From Section~\ref{sec:velocity_overlap}, velocity does not determine temperature contribution because temperature is ensemble-relative. From Section~\ref{sec:velocity_entropy}, velocity is orthogonal to entropy. The demon attempts to sort by a property that doesn't exist at the molecular level (temperature) using a measurement that doesn't determine the property (velocity) to affect a quantity that doesn't depend on the measurement (entropy).

These five errors are interconnected and self-reinforcing. They form a conceptual framework that prevents recognition of the categorical resolution. \qed
\end{proof}

\begin{figure*}[htbp]
\centering
\includegraphics[width=0.95\textwidth]{figures/panel_arg7_information_complementarity.png}
\caption{\textbf{Argument 7: Information Complementarity—The Demon is a Projection Artifact.}
\textbf{(A)} Two complementary faces of information. Venn diagram showing the kinetic face (red circle: velocities, energy, temperature) and categorical face (purple circle: network topology, phase-lock structure) with minimal overlap (small purple square in center). The two faces are complementary: observing one face renders the other hidden, analogous to conjugate observables in quantum mechanics. The annotation ``Complementary: cannot observe both simultaneously'' emphasizes measurement incompatibility. Maxwell observed only the kinetic face; the categorical face remained hidden, creating the illusion of a demon.
\textbf{(B)} Ammeter-voltmeter analogy. Schematic of an electrical component with ammeter (A, red) measuring current and voltmeter (V, purple) measuring voltage. The fundamental constraint ``Cannot use both meters simultaneously on same element'' illustrates complementarity: inserting an ammeter (low resistance) changes the circuit, making voltage measurement impossible, and vice versa. Similarly, observing molecular velocities (kinetic face) obscures phase-lock network structure (categorical face). The demon paradox arises from observing only one meter while the other remains hidden.
\textbf{(C)} Demon as projection artifact. Schematic showing categorical dynamics (purple box, hidden) projecting onto the kinetic face (red box, observed). The demon (yellow box with annotation ``DEMON = Shadow of hidden dynamics'') is not an agent but a projection artifact—the shadow cast by categorical completion onto the observable kinetic face. Three downward arrows represent multiple projection paths from hidden categorical dynamics to observed kinetic behavior. What Maxwell interpreted as intelligent sorting is actually the visible manifestation of automatic topological navigation occurring on the hidden face. The demon is an epiphenomenon, not a causal agent.
\textbf{(D)} Complete picture resolves the paradox. Two-column comparison showing Maxwell's incomplete view versus the complete picture. \textit{Left column (Maxwell's View)}: Observing kinetic properties only (``Kinetic only $\to$'') leads to the interpretation of ``Demon sorting'' (red text)—an apparent agent performing intelligent operations. \textit{Right column (Complete View)}: Observing both faces (``Both faces $\to$'') reveals ``Automatic topology'' (green text)—deterministic categorical completion through phase-lock networks. The yellow box at bottom states the resolution: ``NO DEMON EXISTS / Only categorical completion / along network topology.'' The vertical dashed line separates incomplete from complete understanding. The paradox dissolves when both faces are visible: what appeared to require an information-processing demon is revealed as automatic navigation through categorical state space, visible only from the complementary face. This is the deepest resolution: the demon was never real, only a shadow of hidden dynamics.}
\label{fig:information_complementarity}
\end{figure*}

\subsection{Final Statement}

\begin{theorem}[Non-Existence of the Demon]
\label{thm:nonexistence}
Maxwell's Demon does not exist. The thought experiment describes categorical completion through phase-lock network topology—a physical process requiring no intelligent agent, no information acquisition or processing, and no violation of the Second Law. The demon is the null set:
\begin{equation}
\boxed{\text{``Maxwell's Demon''} = \varnothing}
\label{eq:demon_null}
\end{equation}
\end{theorem}

\begin{proof}
From Theorems~\ref{thm:dissolution_observation}, \ref{thm:dissolution_decision}, \ref{thm:dissolution_sorting}, and~\ref{thm:dissolution_second_law}, every aspect of the demon's purported operation is either unnecessary (observation and decision), misconceived (sorting by temperature), or automatically entropy-increasing (Second Law preservation).

From Theorems~\ref{thm:velocity_temperature_non_correspondence} and~\ref{thm:velocity_entropy_orthogonality}, the demon's measurement and sorting strategy is conceptually incoherent before any thermodynamic analysis: velocity does not determine temperature contribution, and velocity is orthogonal to entropy.

From Theorem~\ref{thm:identity}, the physical process attributed to the demon is categorical completion through phase-lock topology. Categorical completion is a physical process, not an agent. It has no intentionality, no information processing, no decision-making.

From Theorem~\ref{thm:information_complementarity}, the demon is a projection artefact: categorical dynamics projected onto the kinetic face appear as deliberate sorting by an intelligent agent. When the categorical face is observed, the demon dissolves.

Therefore, Maxwell's Demon—as an information-processing agent that sorts molecules by temperature and violates the Second Law—does not exist. What exists is phase-lock network topology and categorical completion dynamics. These are not a demon; they are physics. \qed
\end{proof}

\subsection{Summary of the Nine-Fold Dissolution}

Table~\ref{tab:dissolution_summary} summarises the nine-fold dissolution of Maxwell's demon, showing how each claim attributed to the demon is dissolved by categorical analysis.

\begin{table}[htbp]
\centering
\begin{tabular}{p{4cm}|p{5cm}|c}
\textbf{Demon Claim} & \textbf{Dissolution} & \textbf{Theorem} \\
\hline
Measures velocity to determine temperature & Velocity doesn't determine temperature contribution (context-dependent) & \ref{thm:velocity_temperature_non_correspondence} \\
\hline
Manipulates velocity to affect entropy & Velocity and entropy are orthogonal (category error) & \ref{thm:velocity_entropy_orthogonality} \\
\hline
Creates special configuration & Natural fluctuations produce same configuration (Poincaré recurrence) & \ref{thm:temporal_triviality} \\
\hline
Sorts by temperature & Same phase-lock arrangement exists at any temperature & \ref{thm:phase_lock_temperature_independence} \\
\hline
Maintains sorted state & Cannot outpace thermal equilibration; infinite retrieval loop & \ref{thm:retrieval_paradox} \\
\hline
Observes molecules & Topology doesn't depend on velocity; observation unnecessary & \ref{thm:dissolution_observation} \\
\hline
Makes sorting decisions & Categorical pathways determined by network topology & \ref{thm:dissolution_decision} \\
\hline
Decreases entropy & Categorical entropy increases through network densification & \ref{thm:dissolution_second_law} \\
\hline
Exists as agent & Projection of hidden categorical dynamics onto kinetic face & \ref{thm:information_complementarity}
\end{tabular}
\caption{The nine-fold dissolution of Maxwell's Demon. Each row shows a claim attributed to the demon, how categorical analysis dissolves the claim, and the theorem establishing the dissolution.}
\label{tab:dissolution_summary}
\end{table}

\begin{remark}[The Deepest Resolution]
\label{rem:deepest_resolution}
The ninth argument—information complementarity—is the deepest resolution because it explains not only why the demon does not exist, but why Maxwell and others saw a demon in the first place. The demon was not a failure of imagination or a deliberate puzzle; it was the inevitable consequence of observing one face of a two-faced information structure. Any observer confined to the kinetic face will see "sorting" and require an agent to explain it. The agent dissolves the moment the observer gains access to the categorical face. On the categorical face, the "sorting" is revealed as automatic categorical completion following network topology—a physical process requiring no agent, no purpose, no intelligence.

However, the first two arguments—velocity-temperature non-correspondence and velocity-entropy orthogonality—are the most fundamental refutations because they establish that the demon's strategy is conceptually incoherent before any thermodynamic or information-theoretic analysis is required. The demon fails not because of hidden entropy costs or measurement disturbances, but because it attempts to sort by a property that doesn't exist (molecular temperature) using a measurement that doesn't determine the property (velocity) to affect a quantity that doesn't depend on the measurement (entropy).
\end{remark}


%==============================================================================
%==============================================================================
\section{Conclusion}
\label{sec:conclusion}
%==============================================================================

\subsection{Summary of Results}

A complete resolution of Maxwell's Demon paradox has been presented through the theory of categorical phase-lock networks. The resolution rests on eleven independent results that collectively demonstrate the impossibility of the demon's operation.

The first result establishes temporal triviality: any configuration the demon purportedly creates will occur naturally through thermal fluctuations via Poincaré recurrence. The demon merely accelerates what statistical mechanics predicts will happen spontaneously, but acceleration does not constitute violation of the second law.

The second result demonstrates phase-lock temperature independence: a snapshot of the system consisting of frozen molecular positions and phase-lock relationships can exist at any temperature. The same spatial arrangement is compatible with temperatures ranging from 100~K to 1000~K, implying that the demon's purported sorting operates on phase-lock structure rather than kinetic properties, rendering it velocity-blind.

The third result reveals the retrieval paradox: velocity-based sorting is fundamentally self-defeating. Thermal equilibration occurs on the collision timescale of approximately $10^{-10}$~s, continuously randomizing velocities. Any demon attempting to sort by velocity must retrieve molecules that change speed after sorting, requiring approximately $10^{33}$ operations per second and creating an infinite loop of sorting and retrieval operations.

The fourth result establishes phase-lock kinetic independence through Theorem~\ref{thm:kinetic_independence_intro}: the interactions forming phase-lock relationships, including Van der Waals forces, dipole couplings, and vibrational synchronization, depend on spatial configuration and electronic structure rather than molecular velocity. The mathematical statement $\partial \phaselockgraph / \partial E_{\text{kin}} = 0$ demonstrates that network topology is independent of kinetic energy.

The fifth result shows categorical-physical distance inequivalence: molecules can be categorically adjacent through phase-lock relationships while remaining physically distant, and conversely can be physically proximate while categorically separated. The categorical state space $\catspace$ possesses geometry determined by phase-lock topology rather than Euclidean metrics.

The sixth result demonstrates temperature emergence: temperature is a macroscopic observable that emerges from the statistical properties of phase-lock clusters rather than serving as a sorting criterion. While correlation between phase-lock structure and kinetic energy exists, this correlation is not causal.

The seventh result establishes information complementarity: information possesses two conjugate faces, the kinetic face comprising velocities and temperatures, and the categorical face comprising phase-lock networks and categorical completion. These faces cannot be simultaneously observed, analogous to ammeter-voltmeter complementarity in electrical circuits. Maxwell observed only the kinetic face, and the apparent demon represents the projection of hidden categorical dynamics onto the observable kinetic face. The demon appears intelligent because categorical completion follows structured topological pathways that resemble purposeful selection when projected onto the kinetic face.

The eighth result proves symmetric entropy increase: every door operation increases entropy in both containers simultaneously. In the losing container A, the remaining $N-1$ molecules must form a new phase-lock network, and by categorical completion this new state $C_{A'}$ satisfies $C_A \prec C_{A'}$, representing categorical advancement and higher entropy. In the receiving container B, the new molecule introduces additional phase-lock edges through a process identical to mixing, causing network densification with $|E_{B'}| > |E_B|$ and consequently higher entropy. This follows directly from the categorical resolution of Gibbs' paradox: mixing-reseparation invariably increases entropy through phase-lock network densification. Maxwell observed entropy changes but misattributed them to velocity sorting, when in reality any molecule transfer, regardless of velocity, increases total entropy. The demon cannot decrease entropy regardless of its operational strategy.

The ninth result establishes heat-entropy decoupling: heat and entropy are fundamentally decoupled quantities. Heat represents a statistical emergent property where individual collisions can transfer energy in either direction, from hot to cold or from cold to hot. Entropy represents a categorical fundamental property that invariably increases through phase-lock correlation formation. Even when a particular collision at the demon's door transfers heat from cold to hot, a fluctuation permitted by microscopic dynamics, entropy increases because the collision creates new phase-lock correlations. The Second Law constrains entropy rather than heat direction. Maxwell conflated these quantities by framing the paradox in terms of heat flow when the actual constraint applies to entropy. The demon manipulates heat while entropy remains protected.

The tenth result demonstrates velocity-temperature non-correspondence: velocity distributions of containers at different temperatures overlap completely. The same velocity $v$ corresponds to a fast (hot) molecule in a colder ensemble where $v > \langle v \rangle_{\text{cold}}$, but represents a slow (cold) molecule in a hotter ensemble where $v < \langle v \rangle_{\text{hot}}$. Temperature is not a molecular property but an ensemble property. A molecule moving at 500~m/s is hot in a 300~K container but cold in a 310~K container. When the demon transfers this molecule from cold to hot, its categorical meaning inverts: it contributed to hotness in the source but contributes to coldness in the destination. The demon cannot sort by temperature because molecules do not possess temperature, velocity does not determine temperature contribution, and the contribution depends on the destination ensemble rather than the source. Sorting by velocity does not sort by temperature.

The eleventh result establishes velocity-entropy independence as the most fundamental defeat: entropy counts spatial arrangements according to $S = k_B \ln \Omega$ where $\Omega$ represents the number of possible molecular arrangements. Velocity represents the rate of change of position rather than position itself, yielding $\partial \Omega / \partial v_i = 0$ such that arrangement count is velocity-independent. Elastic collisions redistribute velocities without changing spatial arrangement, temperature can change via kinetic energy redistribution while entropy remains constant, and a configurational snapshot is velocity-blind such that the same arrangement exists at any temperature. The demon commits a category error by treating kinetic properties as determinants of configurational properties. Velocity-sorting is categorically orthogonal to entropy, and the demon manipulates a quantity that has zero effect on the quantity protected by the Second Law.

\subsection{The Dissolution}

Maxwell's Demon does not violate the second law because the demon does not exist. The thought experiment posits an agent that measures molecular velocities, makes decisions based on measurements, controls a door to sort molecules, creates temperature differences without work, and maintains the sorted state. Analysis reveals that each step is either unnecessary, misconceived, or impossible.

No measurement is needed because phase-lock network topology encodes categorical structure without any measurement process. The information about which molecules belong together is structural rather than acquired. No decisions are required because categorical completion follows network topology deterministically, with accessible states determined by phase-lock adjacency rather than deliberation. No door operation is necessary because the partition between categorical clusters is topological rather than physical, and opening the door corresponds to selecting a categorical state that makes phase-lock adjacent states accessible. No sorting by temperature occurs because phase-lock structure is temperature-independent, the same categorical arrangement exists at any temperature, and a snapshot of positions is velocity-blind. No maintenance is possible because even if sorting occurred, thermal equilibration randomizes velocities on the collision timescale of $10^{-10}$~s, requiring infinite retrieval operations that defeat the sorting process. No special outcome is achieved because the sorted configuration will occur naturally through fluctuations, rendering the demon temporally redundant as it creates nothing that would not happen spontaneously.

The demon dissolves into categorical completion:
\begin{equation}
\boxed{\text{Maxwell's Demon} \equiv \text{Categorical Completion through Phase-Lock Topology}}
\end{equation}

\subsection{Relationship to Information-Theoretic Resolutions}

The present resolution does not contradict the Landauer-Bennett framework but renders it unnecessary for resolving the core paradox. Information-theoretic resolutions correctly identify entropy costs of measurement and erasure, and these costs are physically real. However, these resolutions address a demon that need not exist. If one constructs a physical demon as an actual device that measures and sorts, then information-theoretic constraints apply. Maxwell's original thought experiment and the thermodynamic puzzle it poses dissolve once phase-lock topology is recognized as performing the sorting without any agent.

\subsection{Implications}

The resolution carries several implications for fundamental physics. For thermodynamics, the second law is preserved not through information costs but through categorical irreversibility, with entropy increasing because categorical completion densifies phase-lock networks regardless of apparent sorting. For statistical mechanics, temperature is properly understood as emergent from categorical structure rather than as a primitive quantity determining molecular behavior. For information theory, the information content of a physical system resides in its categorical structure, specifically phase-lock topology, rather than in externally acquired measurements. For the foundations of physics, the demon paradox arose from treating molecules as independent particles with measurable properties such as velocity, and recognizing molecules as nodes in phase-lock networks dissolves the paradox while suggesting a more relational ontology. For chemical equilibrium, Le Chatelier's principle is revealed as entropy production rate balance, with equilibrium representing not a static state but a dynamic balance where forward and reverse reactions produce entropy at equal rates, unifying gas thermodynamics with chemical kinetics through categorical dynamics.

\subsection{Experimental Predictions}

The resolution generates testable predictions. Phase-lock correlation spectroscopy should detect categorical structure through correlation measurements independent of temperature. Under isothermal conditions, categorical clusters should remain distinguishable while temperature-based sorting proves impossible. After physical separation, molecules from the same categorical cluster should exhibit residual phase correlations detectable through interference measurements. Molecular dynamics should follow phase-lock adjacency rather than kinetic energy similarity, testable through trajectory analysis. Both forward and reverse reactions should increase total entropy in both reactant and product containers, measurable via precision calorimetry during chemical reactions. At chemical equilibrium, entropy production continues in both directions at equal rates rather than ceasing, with this dynamic balance measurable through isotope labeling and reaction monitoring. When equilibrium is perturbed, the system should shift in the direction of higher instantaneous entropy production rate, providing a kinetic rather than purely thermodynamic interpretation of Le Chatelier's principle.

\subsection{Final Statement}

Maxwell's Demon has challenged thermodynamics for over 150 years, generating profound insights into the relationships between information, entropy, and physical law. The present analysis demonstrates that the demon was never present and could never have been present.

The demon fails on eleven independent grounds. It is redundant because fluctuations produce the same configurations naturally. It is misconceived because phase-lock structure is temperature-independent. It is self-defeating because velocity-based sorting cannot outpace thermal equilibration. It is unnecessary because categorical structure requires no measurement. It is automatic because categorical completion follows topology without decisions. It is entropy-increasing because network densification increases total entropy. It is a projection artifact because the demon represents how hidden categorical dynamics appear when projected onto the observable kinetic face. It is symmetrically defeated because every door operation increases entropy in both containers through categorical completion in the losing container and mixing-type densification in the receiving container, regardless of molecular velocity. It is fundamentally misdirected because the demon manipulates heat, a statistical emergent property, while the Second Law protects entropy, a categorical fundamental property, such that even successful heat reversals increase entropy. It is conceptually incoherent because temperature is not a molecular property, velocity distributions overlap completely, and sorting by velocity cannot sort by temperature since the same velocity is hot in one context and cold in another. It is categorically orthogonal because velocity and entropy belong to different categories, kinetic versus configurational, with entropy counting spatial arrangements such that $\partial \Omega / \partial v = 0$ and velocity-sorting having zero effect on arrangement count, constituting a category error.

The sorting that appeared to require an intelligent agent represents the natural dynamics of categorical completion through phase-lock network topology. The paradox dissolves not through identifying where entropy is produced but through recognizing that the sorting operation was always a manifestation of pre-existing categorical structure, and that any attempt to sort by velocity would be defeated by thermal equilibration before achieving success.

The reason Maxwell perceived a demon becomes clear through information complementarity. Information possesses two conjugate faces that cannot be simultaneously observed. Maxwell, confined to the kinetic face of information comprising velocities, temperatures, and molecular speeds, observed structured sorting that appeared to require intelligent intervention. The demon was simply the categorical face, the phase-lock network completing states according to topology, projected onto the observable kinetic face. Just as an ammeter cannot directly observe voltage, Maxwell's theoretical apparatus could not directly observe categorical dynamics. The demon arose from this observational constraint.

Any observer confined to one face of information will, when dynamics occur on the conjugate face, necessarily perceive those dynamics as external intervention. The demon is universal in this sense, appearing whenever an observer accesses only half of a two-faced information structure. The demon dissolves when the observer gains access to the conjugate face.

The demon does not exist. Only the phase-lock network exists, completing its categorical states according to topology, indifferent to the velocities that Maxwell's thought experiment privileged but that physics does not require. The observer, examining one face of information, invents agents to explain phenomena not directly observable.


%==============================================================================
% Bibliography
%==============================================================================

\bibliographystyle{plainnat}
\bibliography{references}

\end{document}

