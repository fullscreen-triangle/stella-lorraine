%==============================================================================
\section{The Dissolution of Maxwell's Demon}
\label{sec:dissolution}
%==============================================================================

We now synthesize the results established in the preceding sections to provide a systematic dissolution of Maxwell's demon paradox. The dissolution proceeds through nine independent arguments, each sufficient to refute the demon's purported ability to violate the Second Law. These nine arguments build upon the fundamental insights developed in Sections~\ref{sec:velocity_overlap} (velocity-temperature non-correspondence) and~\ref{sec:velocity_entropy} (velocity-entropy orthogonality), which establish that the demon's measurement strategy is fundamentally incoherent. The demon does not exist as an information-processing agent that decreases entropy; rather, the thought experiment describes categorical completion through phase-lock network topology—a physical process that requires no agent, no information acquisition, and no violation of thermodynamic principles. The apparent "demon" is a projection artifact arising from observing only one face of a two-faced information structure. When the conjugate face (categorical structure) is made visible, the demon dissolves, revealing that the "sorting" attributed to the demon is actually categorical completion following network topology, which always increases entropy.

\subsection{Restatement of the Paradox}

Maxwell's thought experiment, as originally formulated in 1867 and refined in subsequent discussions, posits a being (the "demon") that performs the following sequence of operations:

\textbf{Step 1: Observation.} The demon observes molecules approaching a door (aperture) between two chambers A and B, initially at thermal equilibrium with uniform temperature $T_0$.

\textbf{Step 2: Measurement.} The demon measures the velocities of approaching molecules to classify them as "fast" (kinetic energy above average) or "slow" (kinetic energy below average).

\textbf{Step 3: Selective opening.} The demon opens the door selectively: it allows fast molecules to pass from chamber B to chamber A and slow molecules to pass from chamber A to chamber B, while blocking molecules moving in the opposite directions.

\textbf{Step 4: Temperature difference.} Through repeated selective openings, the demon creates a temperature difference between the chambers: chamber A becomes hotter (higher average kinetic energy) and chamber B becomes colder (lower average kinetic energy), starting from an initially uniform temperature.

The paradox arises because this process appears to violate the Second Law of Thermodynamics. In the Clausius formulation, the Second Law states that heat cannot spontaneously flow from a colder body to a hotter body. Yet the demon has created a temperature difference (a hot chamber and a cold chamber) from an initially uniform state, apparently transferring heat from cold to hot without expending work. In the entropy formulation, the Second Law states that the entropy of an isolated system cannot decrease: $\Delta S \geq 0$. Yet the demon has created a more ordered state (molecules sorted by velocity) from a less ordered state (uniform distribution), apparently decreasing entropy: $\Delta S < 0$.

The paradox has generated extensive literature attempting to resolve it through information-theoretic arguments (Szilard 1929, Landauer 1961, Bennett 1982), quantum measurement constraints, and thermodynamic costs of computation. These resolutions argue that the demon must pay an entropy cost to acquire, store, or erase information about molecular velocities, and this cost compensates for the apparent entropy decrease in the gas. However, as we now demonstrate, these information-theoretic resolutions, while correct within their framework, are unnecessary: the paradox dissolves completely when categorical structure is properly accounted for, and when the fundamental incoherence of velocity-based sorting is recognized.

\subsection{Five Decisive Insights}

Before analyzing the demon's purported operations in detail, we establish five fundamental results that independently dissolve the paradox. Each result is sufficient to refute the demon's ability to violate the Second Law; together, they form an overdetermined proof that the paradox rests on conceptual errors. These five insights incorporate the results from Sections~\ref{sec:velocity_overlap} and~\ref{sec:velocity_entropy}, which establish that the demon's measurement and sorting strategy is fundamentally incoherent before any thermodynamic analysis is required.

\begin{theorem}[Velocity-Temperature Non-Correspondence]
\label{thm:velocity_temperature_non_correspondence}
The demon cannot sort by temperature because velocity does not determine temperature contribution. From Section~\ref{sec:velocity_overlap}, the same velocity has different "temperature meaning" in different ensembles. A molecule classified as "hot" in one chamber becomes "cold" upon transfer to another chamber, even though its velocity is unchanged. The demon's sorting strategy is conceptually incoherent.
\end{theorem}

\begin{proof}
From Theorem~\ref{thm:context_dependent}, for the same velocity $v$ in two ensembles at temperatures $T_A < T_B$:
\begin{equation}
P_{T_A}(v) > P_{T_B}(v)
\end{equation}

The same velocity represents a higher percentile (is "faster" relative to the ensemble) in the colder container than in the hotter container.

From Theorem~\ref{thm:sorting_paradox}, when the demon transfers a molecule from container A to container B based on its velocity $v^*$ (classified as "fast" in A), the molecule's percentile changes upon transfer. If $v^*$ is in the ambiguous overlap region where $P_{T_A}(v^*) > 0.5$ but $P_{T_B}(v^*) < 0.5$, the molecule is "hot" in A but "cold" in B. The demon's sorting achieves the opposite of its intention.

From Theorem~\ref{thm:temp_sort_impossible}, the demon cannot sort by temperature because temperature is not a molecular property (Corollary~\ref{cor:no_molecular_temperature}), velocity determines only kinetic energy not temperature contribution (context-dependent), temperature contribution is ensemble-relative (changes upon transfer), and sorting changes ensemble composition hence all molecules' contributions (each transfer invalidates previous classifications).

The demon's strategy is conceptually incoherent: it attempts to sort by a property (temperature) that molecules do not possess, using a measurement (velocity) that does not determine the property even statistically. \qed
\end{proof}

\begin{theorem}[Velocity-Entropy Orthogonality]
\label{thm:velocity_entropy_orthogonality}
The demon manipulates a quantity (velocity) that is categorically orthogonal to the quantity (entropy) protected by the Second Law. From Section~\ref{sec:velocity_entropy}, velocity and entropy are independent variables: changing velocities does not change spatial arrangements and, therefore, does not change entropy.
\end{theorem}

\begin{proof}
From Theorem~\ref{thm:velocity_arrangement}, the number of spatial arrangements $\Omega$ is independent of molecular velocities:
\begin{equation}
\frac{\partial \Omega}{\partial v_i} = 0 \quad \forall i
\end{equation}

Entropy $S = k_B \ln \Omega$ counts arrangements, which depend on positions $\{\mathbf{r}_i\}$, not velocities $\{\mathbf{v}_i\}$.

From Theorem~\ref{thm:elastic_entropy}, elastic collisions can change velocity distribution (and temperature) without changing entropy. Temperature and entropy are independent variables.

From Theorem~\ref{thm:category_error}, the demon commits a category error: it treats velocity (kinetic property) as if it determined entropy (configurational property), when in fact the two categories are independent. The demon manipulates kinetic properties to affect configurational properties, which is impossible because the two categories are orthogonal.

From Theorem~\ref{thm:complete_failure}, the demon's strategy is orthogonal to entropy at every level: the demon measures velocity (orthogonal to entropy), sorts by velocity (doesn't change arrangements), aims to change temperature (not entropy), while entropy depends on arrangements (not velocity or temperature). \qed
\end{proof}

\begin{theorem}[Temporal Triviality of the Demon]
\label{thm:temporal_triviality}
The demon is temporally redundant: any configuration the demon purportedly creates will occur naturally through thermal fluctuations given sufficient time. The demon does not create anything that would not occur spontaneously; it merely (supposedly) accelerates what statistical mechanics already predicts. Since the Second Law constrains what can happen (entropy cannot decrease), not how quickly it happens (the rate of entropy change), acceleration does not constitute a violation.
\end{theorem}

\begin{proof}
In statistical mechanics, the probability of any microscopic configuration $\Gamma$ (specifying positions and momenta of all molecules) is given by the canonical ensemble distribution:
\begin{equation}
P(\Gamma) = \frac{1}{Z} e^{-E(\Gamma)/k_B T}
\label{eq:boltzmann_distribution}
\end{equation}

Crucially, $P(\Gamma) > 0$ for every configuration $\Gamma$, including configurations that appear highly ordered or "sorted." The sorted configuration $\Gamma_{\text{sorted}}$ has probability:
\begin{equation}
P(\Gamma_{\text{sorted}}) = \frac{1}{Z} e^{-E(\Gamma_{\text{sorted}})/k_B T} > 0
\end{equation}

The Poincaré recurrence theorem guarantees that an isolated system will return arbitrarily close to any configuration in finite time. For any initial configuration $\Gamma_0$, any target configuration $\Gamma_{\text{target}}$, and any tolerance $\epsilon > 0$, there exists a recurrence time $T_{\text{rec}} < \infty$ such that:
\begin{equation}
|\Gamma(T_{\text{rec}}) - \Gamma_{\text{target}}| < \epsilon
\end{equation}

Therefore, the demon does not create anything that would not occur naturally. The demon merely (supposedly) accelerates a rare fluctuation. But acceleration is not violation—the Second Law constrains what can happen, not how quickly. \qed
\end{proof}

\begin{theorem}[Phase-Lock Temperature Independence]
\label{thm:phase_lock_temperature_independence}
The same phase-lock network topology (spatial arrangement and categorical structure) can exist at any temperature. A "snapshot" of the system—a frozen configuration with definite molecular positions and phase-lock relationships—is temperature-independent. Temperature is a statistical property of velocity distributions, not a determinant of spatial or categorical structure. Therefore, rearrangement of molecules according to phase-lock topology (categorical completion) is not "sorting by temperature."
\end{theorem}

\begin{proof}
Consider a snapshot of the system at time $t$: a frozen configuration with definite molecular positions $\{\mathbf{r}_i\}_{i=1}^N$ and phase-lock relationships encoded in the network $\phaselockgraph = (V, E)$.

The phase-lock network depends only on positions:
\begin{equation}
\phaselockgraph = \phaselockgraph(\{\mathbf{r}_i\}) \quad \text{(independent of } \{\mathbf{v}_i\})
\end{equation}

This is because phase-lock relationships are determined by spatial proximity and coupling strength, which depend on intermolecular distances $r_{ij} = |\mathbf{r}_i - \mathbf{r}_j|$ and molecular properties, not on velocities.

Temperature depends only on velocities:
\begin{equation}
T = \frac{2}{3 N k_B} \sum_{i=1}^{N} \frac{1}{2} m_i |\mathbf{v}_i|^2 \quad \text{(independent of } \{\mathbf{r}_i\})
\end{equation}

Since positions and velocities are independent variables, the same spatial arrangement (and therefore the same phase-lock network) can occur with different velocity distributions, corresponding to different temperatures. The phase-lock network topology is temperature-independent.

This has a profound implication: rearranging molecules according to categorical pathways (phase-lock adjacency) is not "sorting by temperature." The same categorical rearrangement occurs whether the system is at 100 K or 1000 K. \qed
\end{proof}

\begin{theorem}[The Retrieval Paradox]
\label{thm:retrieval_paradox}
A demon that sorts molecules by velocity is self-defeating: thermal equilibration continuously randomizes velocities on the collision timescale, requiring infinite retrieval operations to maintain the sorted state. The demon cannot "keep up" with thermal relaxation. Velocity-based sorting is futile because the sorting timescale vastly exceeds the equilibration timescale.
\end{theorem}

\begin{proof}
Suppose the demon successfully "sorts" molecule $A$ into the hot chamber based on its velocity $v_A > v_{\text{threshold}}$ at time $t_0$.

After sorting, molecule $A$ undergoes collisions with collision frequency:
\begin{equation}
\nu_{\text{collision}} = n \sigma \langle v \rangle \approx 10^{10} \text{ s}^{-1}
\end{equation}

After a collision at time $t_1 = t_0 + \tau_{\text{collision}}$, where $\tau_{\text{collision}} \sim 10^{-10}$ s, molecule $A$ has a new velocity $v_A'$. With probability $\sim 1/2$, molecule $A$ has become "slow" and is now in the wrong chamber.

The demon must detect this and retrieve molecule $A$. But during retrieval, molecule $A$ undergoes additional collisions. The demon enters an infinite loop.

For $N$ molecules, the demon must process:
\begin{equation}
\text{Operations per second} \sim N \cdot \nu_{\text{collision}} \sim 10^{23} \times 10^{10} = 10^{33} \text{ s}^{-1}
\end{equation}

The sorting timescale is:
\begin{equation}
\tau_{\text{sorting}} \sim N \cdot \tau_{\text{operation}} \sim 10^{13} \text{ s}
\end{equation}

The equilibration timescale is:
\begin{equation}
\tau_{\text{equilibration}} \sim \tau_{\text{collision}} \sim 10^{-10} \text{ s}
\end{equation}

The ratio is:
\begin{equation}
\frac{\tau_{\text{sorting}}}{\tau_{\text{equilibration}}} \sim 10^{23}
\end{equation}

By the time the demon has sorted a significant fraction of molecules, the first molecules sorted have already equilibrated. The demon can never achieve a fully sorted state. \qed
\end{proof}

\begin{corollary}[Velocity Is the Wrong Criterion]
\label{cor:wrong_criterion}
The demon's failure is not due to information costs, measurement disturbance, or quantum uncertainty. It fails because velocity is not a stable molecular property—it changes on the collision timescale $\tau_{\text{collision}} \sim 10^{-10}$ s, which is much faster than any sorting operation. The demon has chosen the wrong criterion for sorting, independent of any thermodynamic considerations.
\end{corollary}

\subsection{The Dissolution}

With Theorems~\ref{thm:velocity_temperature_non_correspondence}--\ref{thm:retrieval_paradox} established, we now show that each step of the demon's purported operation is either unnecessary, misconceived, or automatically entropy-increasing. The dissolution proceeds through four additional arguments that address the demon's specific operations.

\begin{theorem}[Dissolution of Observation]
\label{thm:dissolution_observation}
The demon's "observation" of molecular velocities is unnecessary because phase-lock network topology encodes categorical structure without measurement. The system's categorical structure—which states are accessible from which—is fully determined by network topology $\phaselockgraph$, which depends on spatial configuration, not on kinetic properties. The "information" about molecular arrangement is structural, encoded in the network, not acquired through observation.
\end{theorem}

\begin{proof}
From Theorem~\ref{thm:kinetic_independence}, the phase-lock network $\phaselockgraph = (V, E)$ is determined by spatial configuration $\{\mathbf{r}_i\}$ and molecular properties, not by velocities $\{\mathbf{v}_i\}$. The network topology is given by:
\begin{equation}
E = \{(m_i, m_j) : \kappa_{ij}(\mathbf{r}_i, \mathbf{r}_j) > \kappa_{\text{threshold}}\}
\end{equation}
where $\kappa_{ij}$ depends on positions, not velocities.

From Theorem~\ref{thm:phase_lock_accessibility}, categorical accessibility is determined by network topology:
\begin{equation}
\accessible(C_i) = \{C_j \in \catspace : (C_i, C_j) \in E_{\text{PL}}\}
\end{equation}

Moreover, from Theorem~\ref{thm:velocity_temperature_non_correspondence}, even if the demon observes velocities, this observation does not determine temperature contribution because velocity meaning is context-dependent. The observation is not only unnecessary but also insufficient for the demon's purported goal.

Therefore, the system's categorical structure is fully determined by $\phaselockgraph$, which is determined by spatial configuration. Velocities are not required to determine categorical structure. The "information" about molecular arrangement is structural, encoded in the network topology, not acquired through observation. \qed
\end{proof}

\begin{theorem}[Dissolution of Decision]
\label{thm:dissolution_decision}
The demon's "decision" to open or close the door is unnecessary because categorical completion follows network topology deterministically. The selection of which categorical state to complete next is determined by the categorical ordering and phase-lock adjacency, not by a deliberative decision made by an agent. Categorical dynamics are self-executing.
\end{theorem}

\begin{proof}
From Theorem~\ref{thm:information_free}, categorical selection is determined by minimizing the categorical distance among accessible states:
\begin{equation}
C^* = \argmin_{C \in \accessible(C_{\text{prev}}) \cap [C]_{\text{spatial}}} d_{\catspace}(C, C_{\text{prev}})
\end{equation}

This selection is determined by three factors: the previous categorical state $C_{\text{prev}}$ (given by the system's history), the network topology determining $\accessible(C_{\text{prev}})$ (structural, encoded in $\phaselockgraph$), and the categorical distance metric $d_{\catspace}$ (defined by network topology).

None of these factors involves a deliberative decision by an agent. The selection is deterministic (given $C_{\text{prev}}$ and $\phaselockgraph$, the next state $C^*$ is uniquely determined) or stochastic (if multiple states have equal minimum distance). In either case, no deliberative decision is required. The categorical dynamics are self-executing: the system follows the topological pathway automatically. \qed
\end{proof}

\begin{theorem}[Dissolution of Sorting]
\label{thm:dissolution_sorting}
The demon's "sorting" by temperature is a misinterpretation of categorical completion through phase-lock pathways. When molecules appear "sorted by temperature," they are actually following categorical pathways determined by phase-lock topology, clustering by phase-lock adjacency (categorical property), not by kinetic similarity (kinetic property). Moreover, from Section~\ref{sec:velocity_overlap}, the demon cannot sort by temperature because temperature is not a molecular property and velocity does not determine temperature contribution.
\end{theorem}

\begin{proof}
From Theorem~\ref{thm:demon_cannot_sort}, temperature is not a molecular attribute but an emergent macroscopic property: $T = \mathcal{T}[\{v_1, v_2, \ldots, v_N\}]$. Individual molecules have kinetic energies, not temperatures.

From Corollary~\ref{cor:no_molecular_temperature}, temperature is a functional of the entire velocity distribution, not a function of individual velocities. An individual molecule contributes to temperature, but the contribution's significance depends on the ensemble context.

From Theorem~\ref{thm:kinetic_independence}, kinetic energy does not determine phase-lock network topology: $\partial \phaselockgraph / \partial E_{\text{kinetic}} = 0$.

From Theorem~\ref{thm:apparent_sorting}, molecules in the same phase-lock cluster have correlated kinetic energies because they share molecular properties (mass, polarizability), not because kinetic energy determines clustering:
\begin{equation}
\text{Cov}(E_i, E_j | i, j \in \mathcal{K}_\alpha) > 0
\end{equation}

From Theorem~\ref{thm:category_change}, when molecules transfer between ensembles, their velocity category changes even though velocity is unchanged. A molecule that is "hot" in A becomes "cold" in B.

When molecules appear "sorted by temperature," they are actually sorted by phase-lock cluster membership. The kinetic energy correlation is a consequence of cluster membership, not a cause. The demon does not create the correlation by sorting; it reveals a pre-existing correlation by completing categorical states that make cluster structure visible. \qed
\end{proof}

\begin{theorem}[Dissolution of Second Law Violation]
\label{thm:dissolution_second_law}
The apparent decrease in entropy attributed to the demon's operation is an artifact of ignoring categorical degrees of freedom. When categorical entropy is properly accounted for, total entropy increases. The Second Law is not violated.
\end{theorem}

\begin{proof}
From Theorem~\ref{thm:sorting_density}, the demon operation—categorical completion through phase-lock pathways—increases network density:
\begin{equation}
|E(\gamma(t_{\text{final}}))| > |E(\gamma(t_{\text{initial}}))|
\end{equation}

From Proposition~\ref{prop:entropy_edge_density}, entropy is proportional to edge count: $S_{\text{categorical}} \propto k_B |E|$.

Therefore, categorical entropy increases: $\Delta S_{\text{categorical}} = k_B \Delta |E| > 0$.

From Corollary~\ref{cor:second_law}, total entropy is:
\begin{equation}
S_{\text{total}} = S_{\text{spatial}} + S_{\text{categorical}}
\end{equation}

The spatial entropy may decrease (molecules spatially segregated), but the categorical entropy increase dominates:
\begin{equation}
\Delta S_{\text{total}} = \Delta S_{\text{spatial}} + \Delta S_{\text{categorical}} \sim -k_B N \log 2 + k_B N \langle d \rangle > 0
\end{equation}
since $\langle d \rangle \sim 5$-$10 \gg \log 2 \approx 0.7$.

Moreover, from Theorem~\ref{thm:velocity_entropy_orthogonality}, the demon's velocity-based sorting does not change entropy directly because velocity and entropy are orthogonal. Any entropy change comes from spatial rearrangement (which increases categorical entropy), not from velocity manipulation.

The Second Law is preserved. The paradox arose from incomplete entropy accounting. \qed
\end{proof}

\subsection{The Demon as Categorical Completion}

\begin{theorem}[Identity Theorem]
\label{thm:identity}
Maxwell's Demon is identical to categorical completion through phase-lock network topology. Every operation attributed to the demon corresponds to a categorical process that requires no external agent, no information acquisition, and no violation of thermodynamics:
\begin{equation}
\boxed{\text{``Maxwell's Demon''} \equiv \text{Categorical Completion}(\phaselockgraph)}
\label{eq:demon_identity}
\end{equation}
\end{theorem}

\begin{proof}
We establish a complete correspondence between demon operations and categorical processes:

\begin{center}
\begin{tabular}{l|l}
\textbf{Demon Operation} & \textbf{Categorical Process} \\
\hline
Observe molecule & Complete categorical state $C_i$ \\
Measure velocity & (Unnecessary—topology determines accessibility) \\
Classify fast/slow & Identify phase-lock cluster membership \\
Open door & Make adjacent states $\accessible(C_i)$ accessible \\
Close door & Categorical irreversibility prevents return \\
Sort molecules & Follow phase-lock pathways \\
Create $\Delta T$ & Reveal cluster structure (correlated with $T$)
\end{tabular}
\end{center}

Every demon operation has a categorical counterpart that requires no external agent (categorical completion is self-executing), requires no information acquisition (categorical structure is structural), follows automatically from network topology (categorical pathways are determined by phase-lock adjacency), and increases entropy rather than decreasing it (network densification increases categorical entropy).

Moreover, from Sections~\ref{sec:velocity_overlap} and~\ref{sec:velocity_entropy}, the demon's purported operations are not only unnecessary but conceptually incoherent: velocity does not determine temperature contribution (Theorem~\ref{thm:velocity_temperature_non_correspondence}), and velocity is orthogonal to entropy (Theorem~\ref{thm:velocity_entropy_orthogonality}).

The demon is not needed because categorical completion through phase-lock topology accomplishes the same apparent effect. But this is not a demon "in disguise"—it is the recognition that no demon was ever required. \qed
\end{proof}

\begin{figure*}[htbp]
\centering
\includegraphics[width=0.95\textwidth]{figures/maxwell_demon_resolution_panel.png}
\caption{\textbf{Maxwell's Demon Resolution: Entropy Increases for ANY Molecule Transfer Regardless of Velocity.}
The figure demonstrates that entropy increases in both containers for slow, medium, and fast molecules, with identical entropy changes regardless of velocity ($\Delta S_A > 0$ and $\Delta S_B > 0$ in all cases). Three scenarios are shown vertically: slow molecule ($v \approx 100$ m/s, top row), medium molecule ($v \approx 400$ m/s, middle row), and fast molecule ($v \approx 800$ m/s, bottom row). Each scenario progresses through three stages (left to right): before transfer, during transfer, and after transfer, with quantified entropy changes shown in the rightmost panels.

\textbf{Slow Molecule (v ≈ 100 m/s):}
\textbf{Before:} Container A (green box) contains multiple green molecules with the partition door closed. Container B (purple box) contains multiple purple molecules. The networks are separate.
\textbf{During:} The blue molecule (highlighted) transfers from A to B through the open door.
\textbf{After:} Container A has $N-1$ molecules in the reconfigured network. Container B has $N+1$ molecules with new phase-lock edges (the purple cluster is denser).
\textbf{Entropy changes:} $\Delta S_A = +0.07 \times 10^{-21}$ J/K (categorical completion), $\Delta S_B = +0.28 \times 10^{-21}$ J/K (mixing densification). Both positive (✓ BOTH > 0).

\textbf{Medium Molecule (v ≈ 400 m/s):}
\textbf{Before:} Same initial configuration as the slow case.
\textbf{During:} The orange molecule (highlighted, medium velocity) transfers from A to B.
\textbf{After:} Container A reconfigures with $N-1$ molecules. Container B densifies with $N+1$ molecules.
\textbf{Entropy changes:} $\Delta S_A = +0.07 \times 10^{-21}$ J/K, $\Delta S_B = +0.28 \times 10^{-21}$ J/K. Identical to slow case (✓ BOTH > 0).

\textbf{Fast Molecule (v ≈ 800 m/s):}
\textbf{Before:} Same initial configuration.
\textbf{During:} The red molecule (highlighted, fast velocity) transfers from A to B.
\textbf{After:} Container A reconfigures, and Container B densifies.
\textbf{Entropy changes:} $\Delta S_A = +0.07 \times 10^{-21}$ J/K, $\Delta S_B = +0.28 \times 10^{-21}$ J/K. Again identical (✓ BOTH > 0).

\textbf{Result (bottom text box):} Entropy increases in BOTH containers regardless of molecular velocity. Container A: Categorical completion—network reconfigures as one molecule leaves, increasing edge density and categorical entropy ($\Delta S_A > 0$). Container B: Mixing-type densification—new phase-lock edges form as one molecule enters, increasing total edges and mixing entropy ($\Delta S_B > 0$). The demon CANNOT decrease entropy. The velocity of the transferred molecule is irrelevant to the entropy change; only the topological reconfiguration matters. Maxwell's paradox is dissolved: the apparent ``sorting'' by velocity cannot reduce total entropy because both containers undergo positive entropy changes through categorical mechanisms that are independent of kinetic energy.}
\label{fig:maxwell_demon_resolution}
\end{figure*}

\subsection{Why Maxwell Saw a Demon: Information Complementarity}

\begin{theorem}[Information Complementarity]
\label{thm:information_complementarity}
Information has two conjugate faces (kinetic and categorical) that cannot be simultaneously observed with equal precision. Maxwell saw a "demon" because he was observing the kinetic face of information (velocities, temperatures) while the dynamics of the conjugate categorical face (phase-lock networks, categorical completion) remained hidden.
\end{theorem}

\begin{proof}
Every categorical state has a conjugate representation in two faces:
\begin{align}
\mathbf{S}_{\text{kinetic}} &= (S_{k}, S_{t}, S_{e})_{\text{kinetic}} \quad \text{(observable kinetic face)} \\
\mathbf{S}_{\text{categorical}} &= (S_{k}, S_{t}, S_{e})_{\text{categorical}} \quad \text{(hidden categorical face)}
\end{align}

This conjugacy is not a quantum effect but a classical measurement constraint, analogous to ammeter/voltmeter complementarity in electrical circuits.

Maxwell observed the kinetic face: molecules with velocities, kinetic energies, temperatures, and spatial configurations. The categorical face—phase-lock network topology, cluster structure, categorical pathways, and categorical completion dynamics—was hidden from his view.

When you observe only one face, the dynamics of the conjugate face appear as external intervention. The structured, non-random "sorting" on the kinetic face appears to require an intelligent agent. Hence, the demon.

But the "demon" is not an agent—it is the projection of categorical dynamics onto the kinetic face:
\begin{equation}
\text{``Demon''} = \Pi_{\text{kinetic}}\left(\frac{d\mathbf{S}_{\text{categorical}}}{dt}\right)
\end{equation}

The categorical dynamics (categorical completion following network topology) are projected onto the kinetic face, where they appear as deliberate sorting by an intelligent agent. \qed
\end{proof}

\begin{corollary}[The Demon as Projection]
\label{cor:demon_projection}
Maxwell's Demon is the projection of hidden categorical dynamics onto the observable kinetic face. The demon is not an entity but a projection artifact arising from incomplete observation.
\end{corollary}

\begin{theorem}[Face-Switching Dissolves the Demon]
\label{thm:face_switching}
If Maxwell had been able to observe the categorical face instead of the kinetic face, no demon would have appeared. The "sorting" would be revealed as categorical completion through phase-lock pathways—a physical process requiring no agent.
\end{theorem}

\begin{proof}
On the kinetic face, molecules appear to be sorted by velocity. An agent seems required to select which molecules pass through the door.

On the categorical face, molecules are nodes in a phase-lock network. Categorical completion follows network adjacency. No selection occurs in the sense of choosing among alternatives; the system follows topological pathways determined by network structure.

The same physical process appears differently on different faces:

\begin{center}
\begin{tabular}{l|l}
\textbf{Kinetic Face (Maxwell's View)} & \textbf{Categorical Face (Phase-Lock View)} \\
\hline
Molecules moving with velocities & Nodes in phase-lock network \\
"Fast" and "slow" classification & Phase-lock cluster membership \\
Door opening/closing & Adjacent states becoming accessible \\
Agent making decisions & Topological navigation (automatic) \\
Apparent entropy decrease & Categorical entropy increase \\
Demon required & No agent required
\end{tabular}
\end{center}

The demon is an artifact of the observable face, not a feature of the physical process. \qed
\end{proof}

\subsection{Why the Paradox Persisted}

\begin{proposition}[Source of the Paradox]
\label{prop:paradox_source}
Maxwell's Demon paradox persisted for 150 years due to four conceptual errors that prevented recognition of the categorical resolution: observing only one face of information, treating molecules as independent, privileging kinetic energy, and incomplete entropy accounting. These errors were compounded by the failure to recognize the fundamental incoherence of velocity-based sorting established in Sections~\ref{sec:velocity_overlap} and~\ref{sec:velocity_entropy}.
\end{proposition}

\begin{proof}
\textbf{(1) Single-face observation:}
Maxwell and subsequent analysts observed molecular systems through the kinetic face: velocities, kinetic energies, temperatures, and spatial configurations. The conjugate categorical face—phase-lock networks, cluster structure, categorical pathways—was not accessible to their theoretical framework.

Classical thermodynamics and kinetic theory focus on macroscopic observables (temperature, pressure) and molecular velocities. Neither framework includes concepts of phase-lock networks or categorical structure. When dynamics occur on the hidden face, they must be explained through the observable face. The most parsimonious explanation for structured, non-random "sorting" is an intelligent agent.

\textbf{(2) Independent particle assumption:}
Classical statistical mechanics treats molecules as independent particles whose only interactions are instantaneous collisions. This approximation ignores the persistent phase-lock relationships through Van der Waals forces and dipole interactions that create network structure.

With independent particles, "sorting" would require external information to distinguish molecules. With networked particles, categorical structure already distinguishes molecules through phase-lock relationships.

\textbf{(3) Kinetic energy privilege:}
The thought experiment assumes the demon sorts by velocity—a kinetic property. This privileges kinetic energy as the fundamental variable. But Theorem~\ref{thm:kinetic_independence} establishes that phase-lock networks are kinetically independent: $\partial \phaselockgraph / \partial E_{\text{kinetic}} = 0$.

Moreover, from Section~\ref{sec:velocity_entropy}, velocity and entropy are orthogonal: the demon cannot affect entropy by manipulating velocity because the two quantities are in different categories (kinetic vs. configurational).

\textbf{(4) Incomplete entropy accounting:}
Traditional analyses compute spatial entropy $S_{\text{spatial}}$ while ignoring categorical entropy $S_{\text{categorical}}$. Since categorical completion always increases $S_{\text{categorical}}$, and the increase dominates any decrease in $S_{\text{spatial}}$, total entropy increases.

\textbf{(5) Failure to recognize velocity-based sorting incoherence:}
Even before thermodynamic analysis, the demon's strategy is conceptually incoherent. From Section~\ref{sec:velocity_overlap}, velocity does not determine temperature contribution because temperature is ensemble-relative. From Section~\ref{sec:velocity_entropy}, velocity is orthogonal to entropy. The demon attempts to sort by a property that doesn't exist at the molecular level (temperature) using a measurement that doesn't determine the property (velocity) to affect a quantity that doesn't depend on the measurement (entropy).

These five errors are interconnected and self-reinforcing. They form a conceptual framework that prevents recognition of the categorical resolution. \qed
\end{proof}

\begin{figure*}[htbp]
\centering
\includegraphics[width=0.95\textwidth]{figures/panel_arg7_information_complementarity.png}
\caption{\textbf{Argument 7: Information Complementarity—The Demon is a Projection Artifact.}
\textbf{(A)} Two complementary faces of information. Venn diagram showing the kinetic face (red circle: velocities, energy, temperature) and categorical face (purple circle: network topology, phase-lock structure) with minimal overlap (small purple square in center). The two faces are complementary: observing one face renders the other hidden, analogous to conjugate observables in quantum mechanics. The annotation ``Complementary: cannot observe both simultaneously'' emphasizes measurement incompatibility. Maxwell observed only the kinetic face; the categorical face remained hidden, creating the illusion of a demon.
\textbf{(B)} Ammeter-voltmeter analogy. Schematic of an electrical component with ammeter (A, red) measuring current and voltmeter (V, purple) measuring voltage. The fundamental constraint ``Cannot use both meters simultaneously on same element'' illustrates complementarity: inserting an ammeter (low resistance) changes the circuit, making voltage measurement impossible, and vice versa. Similarly, observing molecular velocities (kinetic face) obscures phase-lock network structure (categorical face). The demon paradox arises from observing only one meter while the other remains hidden.
\textbf{(C)} Demon as projection artifact. Schematic showing categorical dynamics (purple box, hidden) projecting onto the kinetic face (red box, observed). The demon (yellow box with annotation ``DEMON = Shadow of hidden dynamics'') is not an agent but a projection artifact—the shadow cast by categorical completion onto the observable kinetic face. Three downward arrows represent multiple projection paths from hidden categorical dynamics to observed kinetic behavior. What Maxwell interpreted as intelligent sorting is actually the visible manifestation of automatic topological navigation occurring on the hidden face. The demon is an epiphenomenon, not a causal agent.
\textbf{(D)} Complete picture resolves the paradox. Two-column comparison showing Maxwell's incomplete view versus the complete picture. \textit{Left column (Maxwell's View)}: Observing kinetic properties only (``Kinetic only $\to$'') leads to the interpretation of ``Demon sorting'' (red text)—an apparent agent performing intelligent operations. \textit{Right column (Complete View)}: Observing both faces (``Both faces $\to$'') reveals ``Automatic topology'' (green text)—deterministic categorical completion through phase-lock networks. The yellow box at bottom states the resolution: ``NO DEMON EXISTS / Only categorical completion / along network topology.'' The vertical dashed line separates incomplete from complete understanding. The paradox dissolves when both faces are visible: what appeared to require an information-processing demon is revealed as automatic navigation through categorical state space, visible only from the complementary face. This is the deepest resolution: the demon was never real, only a shadow of hidden dynamics.}
\label{fig:information_complementarity}
\end{figure*}

\subsection{Final Statement}

\begin{theorem}[Non-Existence of the Demon]
\label{thm:nonexistence}
Maxwell's Demon does not exist. The thought experiment describes categorical completion through phase-lock network topology—a physical process requiring no intelligent agent, no information acquisition or processing, and no violation of the Second Law. The demon is the null set:
\begin{equation}
\boxed{\text{``Maxwell's Demon''} = \varnothing}
\label{eq:demon_null}
\end{equation}
\end{theorem}

\begin{proof}
From Theorems~\ref{thm:dissolution_observation}, \ref{thm:dissolution_decision}, \ref{thm:dissolution_sorting}, and~\ref{thm:dissolution_second_law}, every aspect of the demon's purported operation is either unnecessary (observation and decision), misconceived (sorting by temperature), or automatically entropy-increasing (Second Law preservation).

From Theorems~\ref{thm:velocity_temperature_non_correspondence} and~\ref{thm:velocity_entropy_orthogonality}, the demon's measurement and sorting strategy is conceptually incoherent before any thermodynamic analysis: velocity does not determine temperature contribution, and velocity is orthogonal to entropy.

From Theorem~\ref{thm:identity}, the physical process attributed to the demon is categorical completion through phase-lock topology. Categorical completion is a physical process, not an agent. It has no intentionality, no information processing, no decision-making.

From Theorem~\ref{thm:information_complementarity}, the demon is a projection artefact: categorical dynamics projected onto the kinetic face appear as deliberate sorting by an intelligent agent. When the categorical face is observed, the demon dissolves.

Therefore, Maxwell's Demon—as an information-processing agent that sorts molecules by temperature and violates the Second Law—does not exist. What exists is phase-lock network topology and categorical completion dynamics. These are not a demon; they are physics. \qed
\end{proof}

\subsection{Summary of the Nine-Fold Dissolution}

Table~\ref{tab:dissolution_summary} summarises the nine-fold dissolution of Maxwell's demon, showing how each claim attributed to the demon is dissolved by categorical analysis.

\begin{table}[htbp]
\centering
\begin{tabular}{p{4cm}|p{5cm}|c}
\textbf{Demon Claim} & \textbf{Dissolution} & \textbf{Theorem} \\
\hline
Measures velocity to determine temperature & Velocity doesn't determine temperature contribution (context-dependent) & \ref{thm:velocity_temperature_non_correspondence} \\
\hline
Manipulates velocity to affect entropy & Velocity and entropy are orthogonal (category error) & \ref{thm:velocity_entropy_orthogonality} \\
\hline
Creates special configuration & Natural fluctuations produce same configuration (Poincaré recurrence) & \ref{thm:temporal_triviality} \\
\hline
Sorts by temperature & Same phase-lock arrangement exists at any temperature & \ref{thm:phase_lock_temperature_independence} \\
\hline
Maintains sorted state & Cannot outpace thermal equilibration; infinite retrieval loop & \ref{thm:retrieval_paradox} \\
\hline
Observes molecules & Topology doesn't depend on velocity; observation unnecessary & \ref{thm:dissolution_observation} \\
\hline
Makes sorting decisions & Categorical pathways determined by network topology & \ref{thm:dissolution_decision} \\
\hline
Decreases entropy & Categorical entropy increases through network densification & \ref{thm:dissolution_second_law} \\
\hline
Exists as agent & Projection of hidden categorical dynamics onto kinetic face & \ref{thm:information_complementarity}
\end{tabular}
\caption{The nine-fold dissolution of Maxwell's Demon. Each row shows a claim attributed to the demon, how categorical analysis dissolves the claim, and the theorem establishing the dissolution.}
\label{tab:dissolution_summary}
\end{table}

\begin{remark}[The Deepest Resolution]
\label{rem:deepest_resolution}
The ninth argument—information complementarity—is the deepest resolution because it explains not only why the demon does not exist, but why Maxwell and others saw a demon in the first place. The demon was not a failure of imagination or a deliberate puzzle; it was the inevitable consequence of observing one face of a two-faced information structure. Any observer confined to the kinetic face will see "sorting" and require an agent to explain it. The agent dissolves the moment the observer gains access to the categorical face. On the categorical face, the "sorting" is revealed as automatic categorical completion following network topology—a physical process requiring no agent, no purpose, no intelligence.

However, the first two arguments—velocity-temperature non-correspondence and velocity-entropy orthogonality—are the most fundamental refutations because they establish that the demon's strategy is conceptually incoherent before any thermodynamic or information-theoretic analysis is required. The demon fails not because of hidden entropy costs or measurement disturbances, but because it attempts to sort by a property that doesn't exist (molecular temperature) using a measurement that doesn't determine the property (velocity) to affect a quantity that doesn't depend on the measurement (entropy).
\end{remark}
