%==============================================================================
\section{Categorical Selection and Accessibility Pathways}
\label{sec:selection}
%==============================================================================

The preceding sections established that molecular systems evolve through categorical state space according to phase-lock network topology, independently of kinetic energy. We now address the central mechanism of Maxwell's demon: the selection process by which specific molecules are chosen to pass through the door. In categorical terms, selection is not an external decision imposed by an intelligent agent, but rather an intrinsic consequence of network topology and physical dynamics. This section proves that categorical selection requires no external information and that the apparent "sorting" behavior attributed to the demon emerges naturally from correlations between phase-lock structure and kinetic properties.

\subsection{The Selection Problem}

In Maxwell's original thought experiment, the demon "selects" fast molecules to pass through a door connecting two chambers, allowing only high-velocity molecules to traverse in one direction and low-velocity molecules in the other. This selection process appears to require measurement of molecular velocities followed by intelligent decision-making about which molecules to allow passage. The demon must distinguish between fast and slow molecules, implying the acquisition of information about individual molecular states.

We now analyze what selection means in categorical terms, demonstrating that it is fundamentally a process of categorical state completion rather than information acquisition and decision-making by an external agent.

\begin{definition}[Categorical Selection]
\label{def:categorical_selection}
A \textbf{categorical selection} is the completion of a specific categorical state $C^* \in [C]_{\text{spatial}}$ from an equivalence class of spatially indistinguishable states. The equivalence class $[C]_{\text{spatial}}$ contains all categorical states that are compatible with the same spatial configuration $\mathbf{q} = (q_1, \ldots, q_N)$ of molecular positions but differ in their phase-lock network topology, S-entropy coordinates, or other categorical properties. Selection specifies which particular categorical state within this equivalence class is physically realized.
\end{definition}

The key insight is that spatial measurements (determining molecular positions) do not uniquely specify the categorical state. Multiple categorical states can correspond to the same spatial configuration, differing in their oscillatory phase relationships, network connectivity, and categorical entropy coordinates. Categorical selection is the process by which the system "chooses" one particular categorical state from this equivalence class.

\begin{proposition}[Selection as Equivalence Class Reduction]
\label{prop:selection_reduction}
Categorical selection reduces the equivalence class $[C]_{\text{spatial}}$ to a singleton set containing only the selected state. Formally, selection is a mapping:
\begin{equation}
[C]_{\text{spatial}} \xrightarrow{\text{selection}} \{C^*\}
\end{equation}
where $C^* \in [C]_{\text{spatial}}$ is the uniquely categorical state that is physically realised. This is an information-gain process in the sense that $\log_2 |[C]_{\text{spatial}}|$ bits of categorical information are specified by the selection. However, this information is not acquired from external measurement but rather determined by network topology and physical dynamics.
\end{proposition}

\begin{proof}
Before selection, any state in the equivalence class $[C]_{\text{spatial}}$ is compatible with the observed spatial configuration. The uncertainty about which categorical state is realised is quantified by the logarithm of the equivalence class size. After selection, exactly one state $C^*$ is completed, reducing the uncertainty to zero. The information gained through selection is:
\begin{equation}
I_{\text{selection}} = \log_2 |[C]_{\text{spatial}}| - \log_2 1 = \log_2 |[C]_{\text{spatial}}|
\end{equation}

For typical gas systems with $N \sim 10^{23}$ molecules, the number of possible phase-lock network configurations is enormous. Even restricting to configurations compatible with a particular spatial arrangement, the equivalence class size can be $|[C]_{\text{spatial}}| \sim 10^6$ or larger, corresponding to $I_{\text{selection}} \sim 20$ bits of categorical information per selection event.

Crucially, this information is not acquired through measurement of molecular properties. Rather, it is determined by which categorical state is accessible from the previous state through phase-lock network transitions. The information is structural, encoded in the network topology, rather than observational, acquired through measurement. \qed
\end{proof}

\begin{remark}[Information Without Measurement]
\label{rem:information_without_measurement}
The information gain in Proposition~\ref{prop:selection_reduction} does not violate information-theoretic bounds on Maxwell's demon because it is not information about molecular velocities that could be used to violate the Second Law. Instead, it is information about categorical state identity within an equivalence class of states that are thermodynamically equivalent. All states in $[C]_{\text{spatial}}$ have the same spatial configuration and, by Theorem~\ref{thm:kinetic_independence}, the same kinetic energy distribution. The categorical information specifies network topology, not kinetic properties, and therefore does not enable thermodynamic work extraction.
\end{remark}

\subsection{Accessibility Through Phase-Lock Networks}

The central mechanism governing categorical selection is accessibility: which categorical states can be reached from the current state through physical processes. We now prove that accessibility is determined entirely by phase-lock network topology, providing the structural foundation for information-free selection.

\begin{theorem}[Phase-Lock Accessibility]
\label{thm:phase_lock_accessibility}
When categorical state $C_i$ is completed at time $t_i$, the set of categorical states accessible for subsequent completion at time $t_{i+1} > t_i$ is determined by phase-lock adjacency in the network $\phaselockgraph = (V, E)$. Specifically:
\begin{equation}
\accessible(C_i) = \{C_j \in \catspace : \exists (m_k, m_l) \in E(\phaselockgraph) \text{ connecting } C_i \text{ to } C_j\}
\label{eq:accessible_states}
\end{equation}
where "connecting $C_i$ to $C_j$" means there exists a path in the phase-lock network from molecules in state $C_i$ to molecules in state $C_j$. States not connected through phase-lock network edges are inaccessible regardless of their spatial proximity or kinetic energy compatibility.
\end{theorem}

\begin{figure}[htbp]
    \centering
    \includegraphics[width=\textwidth]{figures/panel_harmonic.png}
    \caption{\textbf{Harmonic Coincidence Interactions in Phase-Lock Networks.}
    \textbf{(A)} Frequency spectrum of molecular oscillators showing characteristic
    distribution around $\log_{10}(f + 1) \approx 13.0$--13.6.
    \textbf{(B)} Harmonic network topology: nodes represent molecules, edges connect
    molecules with harmonic frequency relationships (integer ratios). Network exhibits
    dense connectivity characteristic of phase-lock coupling.
    \textbf{(C)} Interaction strength distribution showing bimodal character: weak
    interactions (0.1--0.3) dominate numerically, but strong interactions ($\sim$0.5)
    form structural backbone.
    \textbf{(D)} Harmonic order distribution $(n + m)$ for frequency ratios $f_i/f_j = n/m$:
    low-order harmonics (2--5) dominate, establishing primary phase-lock structure, while
    higher orders (10--20) provide fine-grained coupling.
    \textbf{(E)} Phase-amplitude distribution in polar coordinates showing uniform phase
    distribution with amplitude concentration near unit circle, characteristic of stable
    phase-lock states.
    \textbf{(F)} Frequency ratio matrix revealing hierarchical harmonic structure:
    red regions (ratio $\sim$0.5) indicate strong harmonic coupling, blue regions
    (ratio $>$ 2) indicate weak coupling or higher-order harmonics.}
    \label{fig:harmonic}
\end{figure}

\begin{proof}
Categorical transitions require physical mechanisms that can modify phase relationships, network topology, or S-entropy coordinates. The physical mechanisms available in molecular systems are:

\begin{enumerate}
    \item \textbf{Molecular collisions} Transfer kinetic energy and momentum between molecules. During collisions, oscillatory phases interact, potentially establishing or breaking phase-lock relationships. Collision-induced phase coupling is the primary mechanism for phase-lock network evolution in gases.

    \item \textbf{Phase-lock coupling} is the direct synchronisation of oscillatory states through electromagnetic interaction. Molecules with coupled oscillators can exchange phase information without physical collisions, mediated by dipole-dipole interactions, van der Waals forces, or other long-range coupling mechanisms.

    \item \textbf{Electromagnetic interaction} involves the modification of electronic configurations through photon exchange or near-field electromagnetic coupling. This can alter molecular polarizability, dipole moment, and vibrational frequencies, thereby changing phase-lock coupling strengths.
\end{enumerate}

All these mechanisms operate through intermolecular interactions, either direct (collision, near-field coupling) or mediated (electromagnetic field exchange). The strength of interaction between molecules $m_k$ and $m_l$ determines whether they can mediate categorical transitions.

From Definition~\ref{def:phase_lock_network}, molecules are connected in the phase-lock network $\phaselockgraph$ if and only if their coupling strength exceeds the threshold:
\begin{equation}
\kappa_{kl} > \kappa_{\text{threshold}} = \frac{\Delta\omega_{\text{max}}}{2}
\end{equation}
where $\kappa_{kl}$ is the coupling strength and $\Delta\omega_{\text{max}}$ is the maximum frequency detuning for phase-lock formation.

Molecules not connected in $\phaselockgraph$ have a coupling strength of $\kappa_{kl} < \kappa_{\text{threshold}}$, which is insufficient to establish phase coherence or mediate categorical transitions. The interaction strength falls below the threshold required to modify phase relationships on timescales relevant to categorical evolution.

Therefore, categorical transitions can only occur between states connected through phase-lock network edges. States not connected in $\phaselockgraph$ are categorically inaccessible regardless of their spatial proximity or energy compatibility. The accessible states from $C_i$ are precisely those reachable through paths in the phase-lock network, as stated in equation~\eqref{eq:accessible_states}. \qed
\end{proof}

\begin{corollary}[Pathway Opening]
\label{cor:pathway_opening}
Completing categorical state $C_i$ "opens" pathways to all states in the connected component of categorical state space containing $C_i$. Formally, the pathways opened by completing $C_i$ are:
\begin{equation}
\text{Pathways}(C_i) = \{C_j \in \catspace : d_{\catspace}(C_i, C_j) < \infty\}
\end{equation}
where $d_{\catspace}(C_i, C_j)$ is the categorical distance defined as the minimum number of phase-lock network edges connecting states $C_i$ and $C_j$. States with $d_{\catspace}(C_i, C_j) = \infty$ are in disconnected components of the phase-lock network and remain inaccessible regardless of subsequent evolution.
\end{corollary}

\begin{proof}
From Theorem~\ref{thm:phase_lock_accessibility}, accessible states are those connected through phase-lock network edges. A state $C_j$ is reachable from $C_i$ if there exists a sequence of accessible states $C_i \to C_{i_1} \to C_{i_2} \to \cdots \to C_j$ where each transition is phase-lock accessible. This is equivalent to requiring that $C_j$ be in the same connected component as $C_i$, i.e., $d_{\catspace}(C_i, C_j) < \infty$. States in disconnected components have $d_{\catspace}(C_i, C_j) = \infty$ and are never reachable through any sequence of phase-lock transitions. \qed
\end{proof}

\begin{remark}[Topological Constraint on Evolution]
\label{rem:topological_constraint}
Corollary~\ref{cor:pathway_opening} establishes that categorical evolution is topologically constrained. The system can only explore the connected component of categorical state space containing the initial state. Disconnected components are permanently inaccessible, regardless of energy availability or time evolution. This topological constraint is invisible in purely kinetic descriptions but becomes manifest in categorical coordinates. It explains why certain molecular configurations, though energetically favorable, are never observed: they lie in disconnected components of the phase-lock network.
\end{remark}

\subsection{The Cascade Effect}

A single categorical selection does not occur in isolation. Rather, it initiates a cascade of subsequent selections as newly accessible states become available. This cascade effect is central to understanding how categorical completion propagates through the system, producing the appearance of coordinated molecular behavior without requiring external coordination.

\begin{theorem}[Categorical Cascade]
\label{thm:categorical_cascade}
Selection of a single categorical state $C_1$ at time $t_1$ initiates a cascade of accessible completions that propagates through the phase-lock network. The cascade structure is:
\begin{align}
C_1 &\to \accessible(C_1) = \{C_2^{(1)}, C_2^{(2)}, \ldots, C_2^{(k_1)}\} \label{eq:cascade_step1}\\
C_2^{(k)} &\to \accessible(C_2^{(k)}) = \{C_3^{(k,1)}, C_3^{(k,2)}, \ldots, C_3^{(k,k_2)}\} \label{eq:cascade_step2}\\
&\vdots \nonumber
\end{align}
where each completed state makes its phase-lock adjacent states accessible for subsequent completion. The cascade continues until either the entire connected component of $\catspace$ is exhausted or thermodynamic constraints (energy conservation, entropy maximisation) halt the propagation.
\end{theorem}

\begin{proof}
From Theorem~\ref{thm:phase_lock_accessibility}, completing state $C_1$ makes all states in $\accessible(C_1)$ available for subsequent completion. These are the states at categorical distance $d_{\catspace}(C_1, C) = 1$ from $C_1$.

When any state $C_2^{(k)} \in \accessible(C_1)$ is completed at time $t_2$, it makes its adjacent states $\accessible(C_2^{(k)})$ available. These are states at distance $d_{\catspace}(C_2^{(k)}, C) = 1$ from $C_2^{(k)}$, which may be at distance $d_{\catspace}(C_1, C) = 2$ from the original state $C_1$.

This propagation continues recursively. At step $n$, states at categorical distance $n$ from $C_1$ become accessible. The cascade propagates through the phase-lock network according to its topology, with the structure of $\phaselockgraph$ determining which states become accessible at each step.

The cascade terminates when one of two conditions is met:

\begin{enumerate}
    \item \textbf{Topological exhaustion:} All states in the connected component containing $C_1$ have been completed. No further states are accessible because $\accessible(C_{\text{current}}) \subseteq \gamma(t)$, where $\gamma(t)$ is the set of already-completed states.

    \item \textbf{Thermodynamic constraints:} Energy conservation or entropy maximisation prevents further transitions. For example, if all accessible states require energy input exceeding the available thermal energy, the cascade halts even if topologically accessible states remain.
\end{enumerate}

In typical molecular systems, thermodynamic constraints halt the cascade before topological exhaustion, producing partial completion of the connected component. The structure of the partial completion is determined by the interplay between network topology (which states are accessible) and thermodynamics (which accessible states are energetically favorable). \qed
\end{proof}

\begin{definition}[Cascade Wavefront]
\label{def:cascade_wavefront}
The \textbf{cascade wavefront} at step $n$ is the set of categorical states at a categorical distance $n$ from the initial selection $C_1$:
\begin{equation}
W_n = \{C \in \catspace : d_{\catspace}(C_1, C) = n\}
\end{equation}
The wavefront represents the "frontier" of categorical completion, separating completed states (distance $< n$) from not-yet-accessible states (distance $> n$). The wavefront propagates through categorical state space as the cascade advances, with its shape determined by phase-lock network topology.
\end{definition}

\begin{proposition}[Wavefront Propagation]
\label{prop:wavefront_propagation}
The size of the cascade wavefront evolves according to:
\begin{equation}
|W_{n+1}| = \sum_{C \in W_n} |\accessible(C) \setminus \gamma(t_n)|
\end{equation}
where $\gamma(t_n)$ is the set of already-completed states at time $t_n$ when wavefront $W_n$ is reached, and $\accessible(C) \setminus \gamma(t_n)$ is the set of states accessible from $C$ that have not yet been completed. The wavefront size grows when states have many accessible neighbours (high degree in $\phaselockgraph$) and shrinks when states have few accessible neighbours or when most neighbours are already completed.
\end{proposition}

\begin{proof}
By definition, $W_{n+1}$ consists of states at a distance $n+1$ from $C_1$. These are precisely the states accessible from $W_n$ that have not been reached at earlier steps. For each state $C \in W_n$, the newly accessible states are $\accessible(C) \setminus \gamma(t_n)$, excluding states already completed. Summing over all states in $W_n$ gives the size of $W_{n+1}$. \qed
\end{proof}

\begin{corollary}[Exponential Cascade Growth]
\label{cor:exponential_cascade}
In phase-lock networks with approximately constant degree $\langle k \rangle$ (average number of connexions per node), the wavefront size grows approximately exponentially:
\begin{equation}
|W_n| \sim \langle k \rangle^n
\end{equation}
until saturation effects (finite system size, thermodynamic constraints) halt the growth. This exponential cascade explains the rapid propagation of categorical completion observed in molecular systems.
\end{corollary}

\begin{figure}[htbp]
\centering
\includegraphics[width=\textwidth]{figures/panel_maxwell_demon.png}
\caption{Categorical selection mechanism demonstrating information-free sorting through phase-lock network topology. (A) Sorted compartments in S-entropy space showing separation of hot (high $S_e$) and cold (low $S_e$) molecules along the evolution entropy axis, with negligible variation in knowledge entropy $S_k \approx 1.0$, indicating that sorting occurs in categorical coordinates without kinetic energy measurement. (B) Sorting efficiency by criterion showing that categorical sorting (based on $S_e$ coordinate) achieves high efficiency ($\sim 0.8$) while all other operations (predict, navigate, sort by other coordinates, classify, observe) have zero efficiency, confirming that selection is determined by categorical position, not by external decision-making. (C) Zero decision cost demonstrating that all categorical operations (predict, navigate, sort, classify, observe) require zero energy expenditure in categorical coordinates, consistent with Theorem~\ref{thm:information_free} showing that selection requires no external information input. (D) Temperature gradient before and after sorting, showing clear separation of temperature distributions in the hot and cold compartments, demonstrating the apparent temperature sorting predicted by Theorem~\ref{thm:apparent_sorting}. (E) Zero backaction distribution showing that categorical measurements produce zero backaction on physical states (distribution centered at zero with negligible spread), confirming that categorical coordinates commute with physical observables and that selection does not disturb kinetic energy. (F) Categorical sorting flow illustrating the complete process: mixed gas enters, the demon operates at zero categorical cost, and hot/cold molecules emerge sorted by categorical pathways, not by velocity measurement. The demon's "sorting" is categorical completion through phase-lock network accessibility (Theorem~\ref{thm:phase_lock_accessibility}), not intelligent decision-making.}
\label{fig:maxwell_demon_selection}
\end{figure}


\subsection{Selection Without Information}

We now prove the central result of this section: categorical selection requires no external information input. The selection process that Maxwell attributed to an intelligent demon is actually determined by phase-lock network topology and physical dynamics, without any need for measurement or decision-making by an external agent.

\begin{theorem}[Information-Free Selection]
\label{thm:information_free}
Categorical selection from equivalence class $[C]_{\text{spatial}}$ is determined by phase-lock network topology and physical dynamics without external information input. Specifically, the selected categorical state is:
\begin{equation}
C^* = \argmin_{C \in [C]_{\text{spatial}}} d_{\catspace}(C, C_{\text{prev}})
\end{equation}
where $C_{\text{prev}}$ is the previously completed categorical state and $d_{\catspace}(C, C_{\text{prev}})$ is the categorical distance in the phase-lock network. The system selects the categorical state that is closest to the previous state in the network topology, minimising the categorical distance traversed. This selection is deterministic given the network structure and requires no measurement of molecular properties or external decision-making.
\end{theorem}

\begin{proof}
Consider a molecular system at categorical state $C_{\text{prev}}$ at time $t$ undergoing a spatial transition to configuration $\mathbf{q}_{\text{new}}$ at time $t + \Delta t$. The spatial transition could be due to molecular diffusion, convection, or any other physical process that changes molecular positions.

The spatial configuration $\mathbf{q}_{\text{new}}$ is compatible with multiple categorical states, forming the equivalence class $[C]_{\text{spatial}}(\mathbf{q}_{\text{new}})$. The question is: which categorical state $C^* \in [C]_{\text{spatial}}(\mathbf{q}_{\text{new}})$ is selected?

From Theorem~\ref{thm:phase_lock_accessibility}, only states in $\accessible(C_{\text{prev}})$ are reachable from $C_{\text{prev}}$ through phase-lock network transitions. Therefore, the selected state must satisfy:
\begin{equation}
C^* \in \accessible(C_{\text{prev}}) \cap [C]_{\text{spatial}}(\mathbf{q}_{\text{new}})
\end{equation}
the intersection of phase-lock accessible states and spatially compatible states.

\textbf{Case 1: Unique accessible state.}
If $|\accessible(C_{\text{prev}}) \cap [C]_{\text{spatial}}(\mathbf{q}_{\text{new}})| = 1$, there is exactly one categorical state that is both phase-lock accessible from $C_{\text{prev}}$ and compatible with spatial configuration $\mathbf{q}_{\text{new}}$. Selection is deterministic: $C^* $ is the unique state in the intersection. No choice is required, and no information is needed.

\textbf{Case 2: Multiple accessible states.}
If $|\accessible(C_{\text{prev}}) \cap [C]_{\text{spatial}}(\mathbf{q}_{\text{new}})| > 1$, multiple categorical states are both accessible and spatially compatible. In this case, physical dynamics select among them according to optimization principles.

The relevant physical principles are:
\begin{enumerate}
    \item \textbf{Principle of least action:} The system follows the path through categorical state space that minimizes the action integral. In categorical coordinates, this corresponds to minimizing categorical distance $d_{\catspace}(C_{\text{prev}}, C^*)$.

    \item \textbf{Maximum entropy production:} Among accessible states, the system selects the state that maximizes the rate of entropy production $d S/dt$. From Proposition~\ref{prop:nonnegative_completion}, categorical completion increases entropy, so this favors states with shorter paths to equilibrium.

    \item \textbf{Minimum oscillatory termination time:} The system favors categorical states with higher oscillatory termination probability $\alpha$, which corresponds to states closer to phase-lock saturation.
\end{enumerate}

These principles are equivalent for systems near equilibrium and all lead to the same selection criterion:
\begin{equation}
C^* = \argmax_{C \in \accessible(C_{\text{prev}}) \cap [C]_{\text{spatial}}(\mathbf{q}_{\text{new}})} \frac{d\alpha}{dt}
\end{equation}
where $\alpha$ is the oscillatory termination probability from Definition~\ref{def:oscillatory_termination}. States with higher $d\alpha/dt$ are closer to categorical completion and are therefore favored by physical dynamics.

For phase-lock networks with approximately uniform edge weights, $d\alpha/dt$ is approximately inversely proportional to categorical distance $d_{\catspace}(C_{\text{prev}}, C)$. Therefore:
\begin{equation}
C^* \approx \argmin_{C \in \accessible(C_{\text{prev}}) \cap [C]_{\text{spatial}}(\mathbf{q}_{\text{new}})} d_{\catspace}(C, C_{\text{prev}})
\end{equation}
The system selects the categorical state that is closest to the previous state in network topology.

In both cases (unique accessible state and multiple accessible states), selection is determined by network topology and physical dynamics. No external measurement is required to determine which state to select. No external information input is needed to make the selection. The selection is intrinsic to the system's categorical structure.

The "information" specifying which categorical state to occupy is structural information encoded in the phase-lock network $\phaselockgraph$, not observational information acquired through measurement. This structural information exists prior to selection and is not created or acquired during the selection process. Therefore, categorical selection is information-free in the sense relevant to Maxwell's demon: it requires no information acquisition about molecular velocities or other kinetic properties. \qed
\end{proof}

\begin{corollary}[No Demon Required]
\label{cor:no_demon}
The selection process attributed to Maxwell's demon is categorical completion through phase-lock topology. No intelligent agent is required because:
\begin{enumerate}
    \item Selection is determined by network structure (Theorem~\ref{thm:information_free}), not by external decision-making
    \item Accessibility follows from phase-lock adjacency (Theorem~\ref{thm:phase_lock_accessibility}), not from the measurement of molecular properties
    \item Cascade propagation is automatic (Theorem~\ref{thm:categorical_cascade}), not requiring coordination by an external agent
    \item The appearance of intelligent sorting emerges from correlations between network topology and kinetic properties (Theorem~\ref{thm:apparent_sorting}, proven below), not from actual measurement and sorting of velocities
\end{enumerate}
The demon is dissolved not by defeating it, but by recognising that the selection process it supposedly performs is actually intrinsic categorical evolution.
\end{corollary}

\subsection{Apparent Sorting Through Categorical Pathways}

The final piece of the dissolution is explaining why molecules following categorical pathways appear to be sorted by temperature or velocity, despite the selection process being independent of kinetic energy. The explanation lies in correlations between phase-lock structure and kinetic properties that arise from shared dependence on molecular characteristics.

\begin{theorem}[Apparent Temperature Sorting]
\label{thm:apparent_sorting}
Molecules following categorical pathways appear sorted by temperature because phase-lock clusters correlate with kinetic properties, despite phase-lock formation being kinetically independent. Specifically, for molecules $i$ and $j$ in the same phase-lock cluster (connected component of $\phaselockgraph$):
\begin{equation}
\text{Cov}(E_{\text{kin},i}, E_{\text{kin},j}) > 0
\label{eq:kinetic_correlation}
\end{equation}
where $E_{\text{kin},i} = \frac{1}{2} m_i v_i^2$ is the kinetic energy of molecule $i$. This positive covariance creates the appearance of temperature sorting when molecules are selected according to categorical pathways, even though the selection mechanism does not measure or respond to kinetic energy.
\end{theorem}

\begin{proof}
Phase-lock clusters form based on molecular properties that determine coupling strength $\kappa_{ij}$ in equation~\eqref{eq:phase_lock_threshold}. The relevant molecular properties include:

\begin{enumerate}
    \item \textbf{Polarizability $\alpha_i$:} Determines the strength of induced dipole interactions. Molecules with similar polarizabilities exhibit stronger coupling and are more likely to form phase-locks.

    \item \textbf{Permanent dipole moment $\mu_i$:} Determines the strength of dipole-dipole interactions. Molecules with similar dipole moments couple more strongly.

    \item \textbf{Vibrational frequencies $\omega_i$:} Determine the frequency detuning $|\omega_i - \omega_j|$ that must be overcome for phase-lock formation. Molecules with similar vibrational frequencies form phase-locks more easily.

    \item \textbf{Molecular size and geometry} determine collision cross-sections and near-field interaction ranges, affecting coupling strength.
\end{enumerate}

These molecular properties are not independent of molecular mass $m_i$. Empirically, for many classes of molecules:
\begin{align}
\alpha_i &\propto m_i^{2/3} \quad \text{(polarizability scales with molecular volume)} \\
\mu_i &\sim \text{const} \text{ or } \propto m_i^{1/2} \quad \text{(dipole moment weakly correlated with mass)} \\
\omega_i &\propto m_i^{-1/2} \quad \text{(vibrational frequency inversely proportional to reduced mass)}
\end{align}

Therefore, molecules with similar masses tend to have similar polarizabilities, dipole moments, and vibrational frequencies, leading to stronger coupling and a higher probability of phase-lock formation. Phase-lock clusters consequently tend to contain molecules with similar masses.

At thermal equilibrium, molecular velocities follow the Maxwell-Boltzmann distribution:
\begin{equation}
f(v) = 4\pi \left(\frac{m}{2\pi k_B T}\right)^{3/2} v^2 \exp\left(-\frac{mv^2}{2k_B T}\right)
\end{equation}
The most probable speed is $v_p = \sqrt{2k_B T/m}$, and the mean kinetic energy is $\langle E_{\text{kin}} \rangle = \frac{3}{2} k_B T$ independent of mass. However, the velocity distribution width depends on mass: lighter molecules have broader velocity distributions at fixed temperatures.

For molecules with similar masses $m_i \approx m_j$, the velocity distributions are similar, leading to a positive correlation in kinetic energies:
\begin{equation}
\text{Cov}(E_{\text{kin},i}, E_{\text{kin},j}) = \langle E_{\text{kin},i} E_{\text{kin},j} \rangle - \langle E_{\text{kin},i} \rangle \langle E_{\text{kin},j} \rangle > 0
\end{equation}

The causal structure is:
\begin{equation}
\text{Molecular properties } (m, \alpha, \mu, \omega) \to \begin{cases} \text{Phase-lock clustering (via } \kappa_{ij}\text{)} \\ \text{Kinetic energy distribution (via } f(v)\text{)} \end{cases}
\end{equation}

Both phase-lock structure and kinetic properties are downstream consequences of molecular properties. Neither determines the other. The correlation in equation~\eqref{eq:kinetic_correlation} is non-causal: phase-lock clustering does not cause kinetic energy correlation, nor does kinetic energy determine phase-lock structure. Rather, both are correlated because they depend on the same underlying molecular properties.

When molecules are selected according to categorical pathways (phase-lock accessibility), they are selected based on network topology, which correlates with mass and kinetic energy distribution. The result is that selected molecules appear sorted by kinetic energy, even though kinetic energy played no role in the selection mechanism.

This is analogous to observing that people who attend opera performances tend to wear formal attire. The opera house does not select attendees based on clothing—it selects based on ticket purchases. But ticket purchasers correlate with socioeconomic status, which correlates with clothing choices. The apparent "sorting by attire" is a correlation, not a causal selection mechanism. Similarly, categorical selection produces apparent "sorting by temperature" through correlation, not through measurement and sorting of kinetic energy. \qed
\end{proof}


\begin{corollary}[Sorting Is Correlation, Not Causation]
\label{cor:correlation_not_causation}
When molecules "sorted" by categorical pathways appear to separate by temperature, this reflects:
\begin{enumerate}
    \item Pre-existing phase-lock cluster structure determined by molecular properties
    \item Correlation between cluster membership and kinetic properties arising from shared dependence on molecular mass and other characteristics
    \item NOT a measurement of velocity followed by a sorting decision based on kinetic energy
\end{enumerate}
The appearance of intelligent temperature sorting is a projection of categorical structure onto kinetic observables, not evidence of an intelligent sorting agent.
\end{corollary}

\begin{proof}
From Theorem~\ref{thm:apparent_sorting}, the correlation between phase-lock clustering and kinetic energy is non-causal. From Theorem~\ref{thm:information_free}, categorical selection is determined by network topology, not by kinetic energy measurement. From Theorem~\ref{thm:kinetic_independence}, phase-lock network evolution is independent of kinetic energy dynamics. Therefore, any apparent temperature sorting must arise from the correlation between pre-existing network structure and kinetic properties, not from the causal influence of kinetic energy on selection or vice versa. \qed
\end{proof}

\begin{remark}[Experimental Verification]
\label{rem:experimental_verification}
The correlation predicted by Theorem~\ref{thm:apparent_sorting} is experimentally testable. One could measure phase-lock network structure (through spectroscopic techniques sensitive to intermolecular coupling) and kinetic energy distributions (through Doppler broadening or time-of-flight measurements) independently, then compute the covariance in equation~\eqref{eq:kinetic_correlation}. The theory predicts positive covariance for molecules in the same phase-lock cluster, with magnitude depending on the strength of the correlation between molecular properties (mass, polarisability, etc.) and kinetic energy distribution. Observing this correlation would confirm the mechanism proposed here. Observing that categorical selection (identified through network topology) produces apparent temperature sorting without measuring kinetic energy would provide strong evidence for the information-free selection mechanism of Theorem~\ref{thm:information_free}.
\end{remark}

\begin{remark}[Resolution of the Paradox]
\label{rem:paradox_resolution}
Maxwell's demon paradox arises from the apparent ability to sort molecules by velocity without paying the thermodynamic cost of measurement. The resolution presented here is that no velocity measurement occurs. Categorical selection operates on network topology, which correlates with kinetic properties but does not require measuring them. The demon appears to measure and sort velocities because categorical structure (invisible in a purely kinetic description) correlates with kinetic properties (visible in a kinetic description). When categorical structure is recognised, the demon dissolves: there is no measurement, no sorting decision, no intelligent agent—only categorical completion through phase-lock network topology, which happens to correlate with kinetic energy due to shared dependence on molecular properties. The Second Law is preserved because categorical completion increases entropy (Proposition~\ref{prop:nonnegative_completion}), and no information about kinetic energy is acquired or exploited.
\end{remark}
