%==============================================================================
\section{Temperature as Emergent from Phase-Lock Statistics}
\label{sec:temperature}
%==============================================================================

The preceding sections established that molecular systems evolve through categorical state space according to phase-lock network topology, independently of kinetic energy. We now address a fundamental conceptual question: what is the relationship between temperature and categorical structure? In classical thermodynamics, temperature appears as a fundamental quantity that determines molecular behavior. We prove that this causal relationship is inverted: temperature is an emergent statistical property of phase-lock cluster structure, not a fundamental determinant of molecular dynamics. This inversion resolves a conceptual difficulty in understanding Maxwell's demon: the demon cannot sort molecules "by temperature" because temperature is not a molecular property but rather a macroscopic statistical functional of categorical structure.

\subsection{The Standard View of Temperature}

In classical thermodynamics, temperature is typically presented as a fundamental quantity with two equivalent definitions. The thermodynamic definition expresses temperature as the derivative of entropy with respect to energy:
\begin{equation}
\frac{1}{T} = \left(\frac{\partial S}{\partial E}\right)_{V,N}
\label{eq:temperature_standard}
\end{equation}
where $S$ is entropy, $E$ is internal energy, $V$ is volume, and $N$ is particle number. This definition establishes temperature as an intensive thermodynamic variable conjugate to energy.

Alternatively, temperature can be defined operationally through thermal equilibrium: two systems are at the same temperature if no net heat flows between them when brought into thermal contact. This operational definition underlies thermometry and provides the basis for temperature measurement.

For ideal gases, temperature relates to mean kinetic energy through the equipartition theorem:
\begin{equation}
\langle E_{\text{kin}} \rangle = \frac{3}{2} N k_B T
\label{eq:equipartition}
\end{equation}
where $\langle E_{\text{kin}} \rangle$ is the mean kinetic energy of $N$ molecules, $k_B$ is Boltzmann's constant, and the factor $3/2$ arises from three translational degrees of freedom. This relationship is often interpreted causally: higher temperatures cause molecules to move faster.

The Maxwell-Boltzmann velocity distribution reinforces this causal interpretation:
\begin{equation}
f(\mathbf{v}) = \left(\frac{m}{2\pi k_B T}\right)^{3/2} \exp\left(-\frac{m|\mathbf{v}|^2}{2k_B T}\right)
\label{eq:maxwell_boltzmann}
\end{equation}
where $m$ is molecular mass and $\mathbf{v}$ is the velocity. Temperature appears as a parameter that determines the velocity distribution, suggesting that temperature is a fundamental cause of molecular motion.

This standard framing suggests a causal hierarchy: temperature is fundamental, and molecular behaviour (velocities, kinetic energies, collision rates) is determined by temperature. We now demonstrate that this hierarchy is inverted when categorical structure is recognised.

\begin{figure}[htbp]
\centering
\includegraphics[width=\textwidth]{figures/panel_arg2_temperature_independence.png}
\caption{Phase-lock network topology is independent of temperature, demonstrating that temperature does not determine categorical structure. (A) Network edges remain constant at $|E| \approx 106$ across temperatures from $T = 0.5$ to $T = 10$ (black line, left axis), while kinetic energy increases linearly with temperature (red line, right axis) from $\sim 100$ to $\sim 1400$ arbitrary units. The independence $\partial G/\partial T = 0$ proves that cluster structure $\{\mathcal{K}_\alpha\}$ (determined by network connectivity) is independent of temperature, supporting Corollary~\ref{cor:temperature_not_causal}. (B) Scatter plot of network edges versus kinetic energy shows zero correlation ($r = 0.0242$, essentially zero within statistical noise), with network edges fluctuating around constant mean $\langle |E| \rangle \approx 130$ independent of kinetic energy ranging from $0$ to $1600$. This confirms Theorem~\ref{thm:kinetic_independence}: phase-lock network topology evolves independently of kinetic energy and therefore independently of temperature. (C) Van der Waals coupling strength $U \propto r^{-6}$ (solid line) decreases with intermolecular distance $r$, crossing the coupling threshold (dashed line) at $r \approx 2.5$ (shaded region indicates coupled molecules). Crucially, the coupling strength depends only on distance and molecular properties (polarizability, dipole moment), not on velocity or kinetic energy, explaining why phase-lock networks are velocity-independent. (D) Network topology remains constant across different temperatures (horizontal bands at $T = 2, 5, 8$) and across configuration space (horizontal axis), showing that network properties (color intensity) are invariant along temperature direction. The horizontal banding confirms that $\partial G/\partial T = 0$: changing temperature does not change which molecules are phase-locked, supporting the causal structure of Theorem~\ref{thm:causal_structure} in which temperature is downstream of network topology, not upstream.}
\label{fig:temperature_independence}
\end{figure}


\subsection{The Categorical View: Temperature as Emergent}

We now prove that temperature emerges from phase-lock cluster statistics rather than determining them. Temperature is not a fundamental property that causes molecular behavior but rather a derived statistical quantity that summarizes the collective kinetic state of phase-lock clusters.

\begin{definition}[Cluster Kinetic Distribution]
\label{def:cluster_kinetic}
For a phase-lock cluster $\mathcal{K}_\alpha \subset V$ (a connected component of the phase-lock network $\phaselockgraph = (V,E)$), define the cluster kinetic energy as:
\begin{equation}
E_{\alpha} = \sum_{i \in \mathcal{K}_\alpha} \frac{1}{2} m_i |\mathbf{v}_i|^2
\end{equation}
where the sum runs over all molecules $i$ in cluster $\alpha$, $m_i$ is the mass of molecule $i$, and $\mathbf{v}_i$ is its velocity. The cluster temperature is defined as:
\begin{equation}
T_\alpha = \frac{2 E_\alpha}{3 |\mathcal{K}_\alpha| k_B}
\end{equation}
where $|\mathcal{K}_\alpha|$ is the number of molecules in cluster $\alpha$. This definition extends the equipartition relation~\eqref{eq:equipartition} to individual clusters, treating each cluster as a subsystem with its own effective temperature.
\end{definition}

\begin{remark}[Cluster Temperature Interpretation]
\label{rem:cluster_temperature}
The cluster temperature $T_\alpha$ is the temperature that would be measured if cluster $\alpha$ were isolated and allowed to equilibrate internally. At any instant, different clusters have different temperatures, even when the macroscopic system is at thermal equilibrium. This is not a violation of equilibrium but rather a manifestation of thermal fluctuations: equilibrium is a statistical property of the ensemble, not a statement that all subsystems have identical instantaneous properties.
\end{remark}

\begin{theorem}[Temperature Emergence]
\label{thm:temperature_emergence}
The macroscopic temperature $T$ of a gas is a statistical functional of phase-lock cluster structure. Specifically, temperature is the cluster-size-weighted average of cluster temperatures:
\begin{equation}
T = \mathcal{F}[\{(\mathcal{K}_\alpha, T_\alpha, |\mathcal{K}_\alpha|)\}_{\alpha=1}^{N_c}]
\label{eq:temperature_functional}
\end{equation}
where $N_c$ is the number of phase-lock clusters, and the functional $\mathcal{F}$ is explicitly given by:
\begin{equation}
T = \frac{\sum_{\alpha=1}^{N_c} |\mathcal{K}_\alpha| T_\alpha}{\sum_{\alpha=1}^{N_c} |\mathcal{K}_\alpha|} = \frac{1}{N}\sum_{\alpha=1}^{N_c} |\mathcal{K}_\alpha| T_\alpha
\label{eq:temperature_average}
\end{equation}
where $N = \sum_{\alpha} |\mathcal{K}_\alpha|$ is the total number of molecules. Temperature is thus an emergent property: it is computed from cluster structure and cluster kinetic energies, not imposed as a fundamental parameter.
\end{theorem}

\begin{proof}
The total kinetic energy of the gas is the sum of the kinetic energies of all molecules:
\begin{equation}
E_{\text{total}} = \sum_{i=1}^N \frac{1}{2} m_i |\mathbf{v}_i|^2
\end{equation}

Since every molecule belongs to exactly one phase-lock cluster (clusters partition the molecule set), we can reorganise this sum by cluster:
\begin{equation}
E_{\text{total}} = \sum_{\alpha=1}^{N_c} \sum_{i \in \mathcal{K}_\alpha} \frac{1}{2} m_i |\mathbf{v}_i|^2 = \sum_{\alpha=1}^{N_c} E_\alpha
\end{equation}

From Definition~\ref{def:cluster_kinetic}, each cluster energy is related to cluster temperature by:
\begin{equation}
E_\alpha = \frac{3}{2} |\mathcal{K}_\alpha| k_B T_\alpha
\end{equation}

Therefore:
\begin{equation}
E_{\text{total}} = \sum_{\alpha=1}^{N_c} \frac{3}{2} |\mathcal{K}_\alpha| k_B T_\alpha
\end{equation}

The macroscopic temperature $T$ is defined through the equipartition theorem~\eqref{eq:equipartition}:
\begin{equation}
E_{\text{total}} = \frac{3}{2} N k_B T
\end{equation}

Equating these two expressions for total energy:
\begin{equation}
\frac{3}{2} N k_B T = \sum_{\alpha=1}^{N_c} \frac{3}{2} |\mathcal{K}_\alpha| k_B T_\alpha
\end{equation}

Canceling common factors $\frac{3}{2} k_B$:
\begin{equation}
N T = \sum_{\alpha=1}^{N_c} |\mathcal{K}_\alpha| T_\alpha
\end{equation}

Solving for $T$:
\begin{equation}
T = \frac{1}{N}\sum_{\alpha=1}^{N_c} |\mathcal{K}_\alpha| T_\alpha = \frac{\sum_{\alpha=1}^{N_c} |\mathcal{K}_\alpha| T_\alpha}{\sum_{\alpha=1}^{N_c} |\mathcal{K}_\alpha|}
\end{equation}

This expresses macroscopic temperature as a weighted average of cluster temperatures, with weights proportional to cluster sizes. The functional form~\eqref{eq:temperature_functional} is proven, with explicit formula~\eqref{eq:temperature_average}.

Crucially, this derivation shows that temperature is computed from cluster structure $\{\mathcal{K}_\alpha\}$ and cluster kinetic energies $\{E_\alpha\}$. Temperature does not appear as an input or parameter but rather as an output—a statistical summary of the system's categorical and kinetic state. \qed
\end{proof}

\begin{corollary}[Temperature Does Not Determine Clusters]
\label{cor:temperature_not_causal}
The cluster structure $\{\mathcal{K}_\alpha\}$ is determined by phase-lock network topology (Theorem~\ref{thm:kinetic_independence}), which is independent of kinetic energy and therefore independent of temperature. Formally:
\begin{equation}
\frac{\partial \mathcal{K}_\alpha}{\partial T} = 0
\end{equation}
for all clusters $\alpha$. Changing the system temperature (by adding or removing kinetic energy) does not change which molecules belong to which phase-lock clusters at a fixed spatial configuration. The cluster structure is a categorical property determined by molecular properties (polarizability, dipole moment, vibrational frequencies), not by kinetic properties (velocities, kinetic energies, temperature).
\end{corollary}

\begin{proof}
From Theorem~\ref{thm:kinetic_independence}, phase-lock network topology evolves independently of kinetic energy dynamics. The phase-lock coupling strength $\kappa_{ij}$ between molecules $i$ and $j$ depends on molecular properties (polarizability $\alpha_i$, dipole moment $\mu_i$, vibrational frequency $\omega_i$) but not on velocities $\mathbf{v}_i$ or kinetic energies $E_i$. Therefore, the phase-lock network $\phaselockgraph$ is independent of temperature.

Clusters $\{\mathcal{K}_\alpha\}$ are defined as connected components of $\phaselockgraph$, a purely graph-theoretic construction. Since $\phaselockgraph$ is independent of temperature, clusters are also independent of temperature. Changing temperature changes cluster kinetic energies $E_\alpha$ and cluster temperatures $T_\alpha$, but does not change cluster membership or connectivity. \qed
\end{proof}

\begin{remark}[Contrast with Standard Thermodynamics]
\label{rem:contrast_standard}
Corollary~\ref{cor:temperature_not_causal} inverts the standard causal interpretation. In classical thermodynamics, temperature is treated as a control parameter that determines system behavior: raising temperature increases molecular velocities, collision rates, and reaction rates. In the categorical framework, temperature is a response variable: it is computed from categorical structure and kinetic state, not imposed as a cause. The categorical structure (phase-lock network and clusters) is determined by molecular properties and spatial configuration, independent of temperature. This inversion is essential for understanding Maxwell's demon: the demon cannot control molecular behavior by controlling temperature because temperature does not determine categorical accessibility.
\end{remark}

\subsection{Cluster Temperature Distribution}

Having established that macroscopic temperature is a statistical average over cluster temperatures, we now characterize the distribution of cluster temperatures. Even at thermal equilibrium, individual clusters exhibit temperature fluctuations around the mean.

\begin{proposition}[Cluster Temperature Variance]
\label{prop:cluster_variance}
At thermal equilibrium with macroscopic temperature $T$, the variance of cluster temperatures satisfies:
\begin{equation}
\text{Var}(T_\alpha) = \frac{2 T^2}{3 \langle |\mathcal{K}_\alpha| \rangle}
\label{eq:cluster_variance}
\end{equation}
where $\langle |\mathcal{K}_\alpha| \rangle$ is the mean cluster size. Smaller clusters exhibit larger temperature fluctuations, with variance inversely proportional to cluster size. For single-molecule "clusters" (isolated molecules not phase-locked to others), the temperature variance diverges, reflecting the fact that single-molecule temperature is undefined.
\end{proposition}

\begin{proof}
Consider a phase-lock cluster $\alpha$ containing $n = |\mathcal{K}_\alpha|$ molecules at thermal equilibrium with temperature $T$. The cluster kinetic energy $E_\alpha$ is the sum of $n$ independent molecular kinetic energies, each distributed according to the Maxwell-Boltzmann distribution.

For a single molecule with three translational degrees of freedom, the kinetic energy follows a Gamma distribution:
\begin{equation}
E_i \sim \text{Gamma}\left(\frac{3}{2}, k_B T\right)
\end{equation}
with shape parameter $k = 3/2$ and scale parameter $\theta = k_B T$.

The cluster kinetic energy, being the sum of $n$ independent Gamma-distributed variables with the same parameters, follows:
\begin{equation}
E_\alpha = \sum_{i \in \mathcal{K}_\alpha} E_i \sim \text{Gamma}\left(\frac{3n}{2}, k_B T\right)
\end{equation}

The mean and variance of $E_\alpha$ are:
\begin{align}
\langle E_\alpha \rangle &= \frac{3n}{2} k_B T \\
\text{Var}(E_\alpha) &= \frac{3n}{2} (k_B T)^2
\end{align}

From Definition~\ref{def:cluster_kinetic}, cluster temperature is:
\begin{equation}
T_\alpha = \frac{2 E_\alpha}{3 n k_B}
\end{equation}

The variance of $T_\alpha$ is obtained by propagating the variance of $E_\alpha$:
\begin{equation}
\text{Var}(T_\alpha) = \left(\frac{2}{3 n k_B}\right)^2 \text{Var}(E_\alpha) = \frac{4}{9 n^2 k_B^2} \cdot \frac{3n}{2} (k_B T)^2
\end{equation}

Simplifying:
\begin{equation}
\text{Var}(T_\alpha) = \frac{4 \cdot 3n \cdot k_B^2 T^2}{9 n^2 k_B^2 \cdot 2} = \frac{12 n k_B^2 T^2}{18 n^2 k_B^2} = \frac{2 T^2}{3 n}
\end{equation}

Taking the expectation over the distribution of cluster sizes (averaging over all clusters or over time for a single cluster):
\begin{equation}
\langle \text{Var}(T_\alpha) \rangle = \frac{2 T^2}{3 \langle n \rangle} = \frac{2 T^2}{3 \langle |\mathcal{K}_\alpha| \rangle}
\end{equation}

This proves equation~\eqref{eq:cluster_variance}. The inverse dependence on cluster size reflects the law of large numbers: larger clusters have more molecules over which to average, reducing temperature fluctuations. \qed
\end{proof}

\begin{corollary}[Hot and Cold Clusters at Equilibrium]
\label{cor:hot_cold_clusters}
Even at thermal equilibrium (uniform macroscopic temperature $T$), individual phase-lock clusters have different instantaneous temperatures. At any given moment, some clusters are "hot" ($T_\alpha > T$) and others are "cold" ($T_\alpha < T$), with the distribution of cluster temperatures having variance given by Proposition~\ref{prop:cluster_variance}. The existence of hot and cold clusters at equilibrium is not a violation of thermodynamics but rather a manifestation of thermal fluctuations at the cluster scale.
\end{corollary}

\begin{proof}
From Proposition~\ref{prop:cluster_variance}, $\text{Var}(T_\alpha) > 0$ for finite cluster sizes. A non-zero variance implies that cluster temperatures are distributed around the mean $T$, with some clusters having $T_\alpha > T$ (hot) and others having $T_\alpha < T$ (cold).

The distribution of cluster temperatures is approximately Gaussian for large clusters (by the central limit theorem applied to the sum of molecular kinetic energies):
\begin{equation}
T_\alpha \sim \mathcal{N}\left(T, \frac{2T^2}{3|\mathcal{K}_\alpha|}\right)
\end{equation}

For small clusters, the distribution is more accurately described by a scaled chi-squared distribution (since $T_\alpha \propto E_\alpha$ and $E_\alpha$ follows a Gamma distribution).

In either case, the probability of finding a cluster with $T_\alpha > T$ is approximately $1/2$ (by symmetry of the distribution around the mean), and similarly for $T_\alpha < T$. Therefore, at any instant, approximately half the clusters are hotter than average and half are colder than average.

This is consistent with equilibrium thermodynamics: equilibrium is defined by the absence of net heat flow between subsystems, not by the absence of temperature fluctuations within subsystems. The fluctuations in cluster temperatures are thermal fluctuations, analogous to density fluctuations, pressure fluctuations, or energy fluctuations in small subsystems. \qed
\end{proof}

\begin{remark}[Connection to Maxwell's Demon]
\label{rem:demon_fluctuations}
Corollary~\ref{cor:hot_cold_clusters} is crucial for understanding Maxwell's demon. The demon supposedly creates a temperature difference by allowing only hot molecules to pass in one direction and cold molecules in the other. But Corollary~\ref{cor:hot_cold_clusters} shows that hot and cold clusters exist naturally at equilibrium. The demon does not create temperature differences—it reveals pre-existing fluctuations in cluster temperatures. The question then becomes: can the demon exploit these fluctuations to extract work? We address this in Section~\ref{sec:entropy_mechanism}.
\end{remark}

\subsection{The Inversion of Causality}

We now formalize the causal structure relating molecular properties, phase-lock networks, cluster structure, kinetic energy, and temperature. This causal analysis demonstrates that temperature is downstream of categorical structure, not upstream.

\begin{theorem}[Causal Structure of Temperature]
\label{thm:causal_structure}
The causal relationships between molecular properties, phase-lock network topology, cluster structure, velocity distribution, kinetic energy, and temperature are represented by the following directed acyclic graph:
\begin{equation}
\begin{tikzcd}[row sep=large, column sep=large]
& \text{Molecular Properties } (m, \alpha, \mu, \omega) \arrow[dl] \arrow[dr] & \\
\text{Phase-Lock Network } \phaselockgraph \arrow[d] & & \text{Velocity Distribution } f(\mathbf{v}) \arrow[d] \\
\text{Cluster Structure } \{\mathcal{K}_\alpha\} \arrow[dr] & & \text{Individual Kinetic Energies } \{E_i\} \arrow[dl] \\
& \text{Temperature } T &
\end{tikzcd}
\label{eq:causal_dag}
\end{equation}
where arrows represent causal influence. Temperature is a downstream consequence of both categorical structure (via clusters $\{\mathcal{K}_\alpha\}$) and kinetic state (via energies $\{E_i\}$). Temperature does not causally influence phase-lock network topology, cluster structure, or the functional form of the velocity distribution. This inverts the standard thermodynamic framing in which temperature appears as a fundamental cause of molecular behavior.
\end{theorem}

\begin{proof}
We establish each causal arrow in the diagram~\eqref{eq:causal_dag}.

\textbf{Molecular properties $\to$ Phase-lock network:}
From Section~\ref{sec:phase_lock}, the phase-lock network $\phaselockgraph = (V, E)$ is determined by intermolecular coupling strengths $\kappa_{ij}$, which depend on molecular polarizability $\alpha_i$, permanent dipole moment $\mu_i$, vibrational frequency $\omega_i$, and molecular geometry. These are intrinsic molecular properties independent of velocity or kinetic energy. The phase-lock threshold condition~\eqref{eq:phase_lock_threshold} is:
\begin{equation}
\kappa_{ij} > \frac{|\omega_i - \omega_j|}{2}
\end{equation}
where $\kappa_{ij} = \kappa(\alpha_i, \alpha_j, \mu_i, \mu_j, r_{ij})$ depends on molecular properties and separation $r_{ij}$ but not on velocities. Therefore, molecular properties causally determine $\phaselockgraph$.

\textbf{Phase-lock network $\to$ Cluster structure:}
Clusters $\{\mathcal{K}_\alpha\}$ are defined as connected components of $\phaselockgraph$, a purely graph-theoretic construction. Given $\phaselockgraph$, the cluster structure is uniquely determined by graph connectivity. Therefore, $\phaselockgraph$ causally determines $\{\mathcal{K}_\alpha\}$.

\textbf{Molecular properties $\to$ Velocity distribution:}
The Maxwell-Boltzmann velocity distribution~\eqref{eq:maxwell_boltzmann} depends on molecular mass $m$:
\begin{equation}
f(\mathbf{v}) = \left(\frac{m}{2\pi k_B T}\right)^{3/2} \exp\left(-\frac{m|\mathbf{v}|^2}{2k_B T}\right)
\end{equation}
Heavier molecules have narrower velocity distributions at fixed temperature. Molecular mass is an intrinsic property, so molecular properties causally influence the velocity distribution functional form.

\textbf{Velocity distribution $\to$ Individual kinetic energies:}
Individual molecular velocities $\mathbf{v}_i$ are sampled from the velocity distribution $f(\mathbf{v})$. Kinetic energies are $E_i = \frac{1}{2} m_i |\mathbf{v}_i|^2$. Therefore, the velocity distribution causally determines the statistical properties of kinetic energies.

\textbf{Cluster structure and kinetic energies $\to$ Temperature:}
From Theorem~\ref{thm:temperature_emergence}, macroscopic temperature is computed as:
\begin{equation}
T = \frac{1}{N} \sum_{\alpha=1}^{N_c} |\mathcal{K}_\alpha| T_\alpha = \frac{1}{N} \sum_{\alpha=1}^{N_c} |\mathcal{K}_\alpha| \cdot \frac{2 E_\alpha}{3 |\mathcal{K}_\alpha| k_B} = \frac{2}{3 N k_B} \sum_{\alpha=1}^{N_c} E_\alpha
\end{equation}
where $E_\alpha = \sum_{i \in \mathcal{K}_\alpha} E_i$. Temperature is a function of cluster structure (which molecules belong to which clusters) and individual kinetic energies. Both inputs are required to compute $T$. Therefore, $\{\mathcal{K}_\alpha\}$ and $\{E_i\}$ causally determine $T$.

\textbf{No reverse arrows:}
Crucially, there are no arrows pointing from temperature back to the phase-lock network, cluster structure, or velocity distribution functional form. From Corollary~\ref{cor:temperature_not_causal}, $\partial \mathcal{K}_\alpha / \partial T = 0$: cluster structure is independent of temperature. From Theorem~\ref{thm:kinetic_independence}, phase-lock network topology evolves independently of kinetic energy and, therefore, independently of temperature. The velocity distribution functional form depends on mass, not on temperature (temperature appears as a parameter in the distribution but does not determine the functional form $f(\mathbf{v}) \propto v^2 \exp(-mv^2/(2k_BT))$, which is fixed by statistical mechanics).

Therefore, the causal structure is as depicted in diagram~\eqref{eq:causal_dag}, with temperature as a downstream consequence, not an upstream cause. \qed
\end{proof}

\begin{figure*}[htbp]
\centering
\includegraphics[width=0.95\textwidth]{figures/arg2_temperature_independence.png}
\caption{\textbf{Phase-Lock Temperature Independence—Network Topology $\partial G/\partial E_{\text{kin}} = 0$: Independent of Kinetic Energy.}
\textbf{(A)} Same network topology across all temperatures. Three-dimensional visualization showing molecular positions in $(X, Y, Z)$ space for four different temperatures: $T = 0.5$ (dark blue/teal points), $T = 1.0$ (cyan points), $T = 2.0$ (orange points), and $T = 5.0$ (red points). Despite the four-fold temperature variation, all four distributions occupy the same spatial region and exhibit identical clustering structure. The points form distinct horizontal layers corresponding to each temperature, but within each layer, the spatial arrangement is statistically identical. This demonstrates that network topology (determined by spatial proximity) is temperature-independent. The phase-lock network structure $G$ depends only on positions, not on kinetic energy: $\partial G/\partial E_{\text{kin}} = 0$.
\textbf{(B)} Network properties versus kinetic properties: $\partial G/\partial T = 0$. Dual-axis plot showing network edges (black circles, left y-axis, "constant") and kinetic energy (red squares, right y-axis, proportional to $T$) versus temperature $T \in [0, 10]$. Network edges remain constant at approximately 107 edges across the entire temperature range (flat black line with annotation "Edges (constant)"). Kinetic energy increases linearly from $\approx 100$ at $T = 1$ to $\approx 1400$ at $T = 10$ (red line with annotation "KE $\propto T$"). The horizontal network line and diagonal kinetic line demonstrate complete decoupling: $\partial(\text{Network Edges})/\partial T = 0$ while $\partial(\text{KE})/\partial T > 0$. Network topology is invariant under temperature changes, proving that categorical structure is independent of kinetic properties.
\textbf{(C)} Maxwell-Boltzmann distributions at different temperatures: velocity distribution widens with $T$, network unchanged. Heat map showing probability density (color scale from 0.0 blue to 0.5+ red) of velocity distributions versus temperature. Horizontal axis: velocity $v \in [-4, 4]$. Vertical axis: temperature $T \in [0.5, 10.0]$. At each temperature, the distribution forms a Gaussian centered at $v = 0$ (red/orange peak). As temperature increases from $T = 0.5$ (top) to $T = 10.0$ (bottom), the distribution widens (blue wings extend further from center), confirming $\sigma_v \propto \sqrt{T}$ (Maxwell-Boltzmann). However, the text annotation states "network unchanged"—while velocity distributions change dramatically with temperature, the underlying network topology remains constant. This is the key insight: kinetic properties (velocity spread) and categorical properties (network structure) evolve independently.
\textbf{(D)} Property correlation matrix: network properties versus kinetic properties ($r \approx 0$). Correlation matrix heat map with color scale from $-1.00$ (blue, negative correlation) to $+1.00$ (red, positive correlation). Matrix divided into four blocks by dashed yellow lines. Top-left block (3×3, dark red): network properties (Network Edges, Mean Degree, Clustering) show strong positive correlations ($r = 0.78$ to $1.00$) with each other—network properties are internally correlated. Bottom-right block (2×2, dark red): kinetic properties (Kinetic Energy, Temperature) show perfect correlation ($r = 0.99$ to $1.00$) with each other—kinetic properties are internally correlated. Off-diagonal blocks (top-right and bottom-left, white/pale): network-kinetic cross-correlations are near-zero ($r = -0.03$ to $0.02$)—network and kinetic properties are uncorrelated. This block-diagonal structure proves that network topology and kinetic energy are orthogonal properties: $\partial G/\partial E_{\text{kin}} = 0$ is confirmed by near-zero cross-correlations.}
\label{fig:arg2_temperature_independence_v1}
\end{figure*}


\begin{corollary}[Temperature as Summary Statistic]
\label{cor:temperature_summary}
Temperature is a summary statistic of the system's categorical and kinetic state, analogous to the mean of a dataset. Just as the mean does not determine individual data points (rather, data points determine the mean), temperature does not determine individual molecular behaviours (rather, molecular behaviours collectively determine temperature). Treating temperature as a fundamental cause of molecular motion is a category error, confusing a statistical summary with a physical cause.
\end{corollary}

\begin{remark}[Philosophical Implications]
\label{rem:philosophical}
Theorem~\ref{thm:causal_structure} has profound implications for the interpretation of thermodynamics. Temperature is often treated as a primitive concept in thermodynamics, with entropy and other quantities defined in terms of temperature (as in equation~\eqref{eq:temperature_standard}). The categorical framework reveals that this is an inversion of the true causal order: temperature is emergent from microscopic structure, not fundamental. This parallels the emergence of other thermodynamic quantities (pressure from momentum transfer, entropy from phase space volume) but is particularly important for temperature because of its central role in thermodynamic reasoning. Recognising temperature as emergent clarifies why Maxwell's demon cannot "sort by temperature"—there is no fundamental temperature property to sort by, only emergent statistical correlations between categorical structure and kinetic energy.
\end{remark}

\subsection{Implications for Maxwell's Demon}

The emergence of temperature from phase-lock cluster statistics has direct implications for understanding Maxwell's demon. We now prove that the demon cannot sort molecules "by temperature" in any meaningful sense.

\begin{theorem}[Demon Cannot Sort by Temperature]
\label{thm:demon_cannot_sort}
A hypothetical Maxwell's demon cannot sort molecules "by temperature" because:
\begin{enumerate}
    \item Temperature is a macroscopic emergent property of ensembles, not a molecular attribute that can be measured or used as a sorting criterion for individual molecules.
    \item Individual molecules have kinetic energies $E_i = \frac{1}{2} m_i |\mathbf{v}_i|^2$, not temperatures. The concept of single-molecule temperature is undefined.
    \item Kinetic energy does not determine categorical accessibility (Theorem~\ref{thm:kinetic_independence}). A molecule's kinetic energy does not predict or control which categorical states it can access.
    \item The demon's apparent "sorting by temperature" is actually navigation through categorical space along phase-lock pathways (Corollary~\ref{cor:demon_actual}), which correlates with kinetic energy but is not caused by kinetic energy.
\end{enumerate}
Therefore, "sorting by temperature" is categorically meaningless. What appears as temperature sorting is actually categorical completion revealing pre-existing cluster structure that happens to correlate with kinetic properties.
\end{theorem}

\begin{proof}
We prove each statement in turn.

\textbf{(1) Temperature is macroscopic:}
From Definition~\ref{def:cluster_kinetic}, even cluster temperature $T_\alpha$ requires multiple molecules:
\begin{equation}
T_\alpha = \frac{2 E_\alpha}{3 |\mathcal{K}_\alpha| k_B} = \frac{2 \sum_{i \in \mathcal{K}_\alpha} E_i}{3 |\mathcal{K}_\alpha| k_B}
\end{equation}

For a single molecule ($|\mathcal{K}_\alpha| = 1$), this formula gives:
\begin{equation}
T_{\text{single}} = \frac{2 E_i}{3 k_B}
\end{equation}

However, this is not a meaningful temperature. Temperature is defined thermodynamically through equation~\eqref{eq:temperature_standard} as a derivative of entropy with respect to energy. For a single molecule, entropy is zero (a single microstate), so $\partial S / \partial E$ is undefined. Temperature is a statistical property that emerges from ensembles of molecules, not a property of individual molecules.

The demon supposedly measures the "temperature" of individual molecules to decide which to allow passage. But individual molecules do not have temperatures—they have kinetic energies. The demon must be measuring kinetic energy, not temperature.

\textbf{(2) Kinetic energy vs. temperature:}
A molecule has instantaneous kinetic energy $E_i = \frac{1}{2} m_i |\mathbf{v}_i|^2$, a well-defined mechanical quantity at each moment. Temperature $T$, by contrast, is a statistical property of ensembles, defined through time or ensemble averaging.

In Maxwell's original formulation, the demon distinguishes "fast" molecules (high $E_i$) from "slow" molecules (low $E_i$) and allows only fast molecules to pass in one direction. This is sorting by kinetic energy, not by temperature. The conflation of "fast" with "hot" and "slow" with "cold" is imprecise: "hot" and "cold" are temperature concepts applicable to ensembles, while "fast" and "slow" are kinetic energy concepts applicable to individuals.

The demon's operation, if interpreted literally, is: measure $E_i$, compare to threshold $E_{\text{threshold}}$, allow passage if $E_i > E_{\text{threshold}}$. This is kinetic energy sorting, not temperature sorting.

\textbf{(3) Kinetic energy does not determine accessibility}:
From Theorem~\ref{thm:kinetic_independence}, phase-lock network topology evolves independently of kinetic energy. The accessible categorical states from state $C_i$ are determined by phase-lock adjacency (Theorem~\ref{thm:phase_lock_accessibility}):
\begin{equation}
\accessible(C_i) = \{C_j \in \catspace : \exists \text{ phase-lock path from } C_i \text{ to } C_j\}
\end{equation}

This accessibility is independent of kinetic energy. Two molecules with the same categorical state $C_i$ but different kinetic energies $E_1 \neq E_2$ have the same accessible states $\accessible(C_i)$. Conversely, two molecules with the same kinetic energy $E_1 = E_2$ but different categorical states $C_i \neq C_j$ have different accessible states $\accessible(C_i) \neq \accessible(C_j)$.

Therefore, measuring kinetic energy does not reveal which categorical states are accessible. A "fast" molecule (high $E_i$) in cluster $\mathcal{K}_\alpha$ has the same categorical accessibility as a "slow" molecule (low $E_i$) in the same cluster. The demon cannot use kinetic energy to predict or control categorical transitions.

\textbf{(4) Apparent sorting is categorical navigation:}
From Theorem~\ref{thm:apparent_sorting}, molecules following categorical pathways appear sorted by kinetic energy because phase-lock clusters correlate with molecular properties (mass, polarizability) that are associated with the kinetic energy distribution. The correlation structure is:
\begin{equation}
\text{Molecular properties } (m, \alpha, \mu) \to \begin{cases} \text{Phase-lock clustering} \\ \text{Kinetic energy distribution} \end{cases}
\end{equation}

Both phase-lock structure and kinetic properties are downstream of molecular properties. The correlation is non-causal: neither determines the other.

When the demon "sorts by temperature," it is actually navigating categorical space along phase-lock pathways (Corollary~\ref{cor:demon_actual}). The categorical pathways happen to correlate with kinetic energy, creating the appearance of kinetic sorting. But the sorting mechanism is categorical (following phase-lock accessibility), not kinetic (measuring and comparing energies).

Combining these four observations: the demon cannot sort by temperature because (1) temperature is not a molecular property, (2) individual molecules have kinetic energies not temperatures, (3) kinetic energy does not determine categorical accessibility, and (4) apparent temperature sorting is actually categorical navigation that correlates with but is not caused by kinetic energy. Therefore, "sorting by temperature" is a categorical error—a confusion of emergent statistical properties with fundamental molecular attributes. \qed
\end{proof}

\begin{corollary}[What the Demon Actually Does]
\label{cor:demon_actual}
If we reinterpret the demon's operation in categorical terms, the actions traditionally attributed to the demon correspond to categorical processes:
\begin{enumerate}
    \item \textbf{"Observing" a molecule:} Completing a categorical state $C_i$, which makes phase-lock adjacent states $\accessible(C_i)$ available for subsequent completion (Theorem~\ref{thm:phase_lock_accessibility}).

    \item \textbf{"Opening the door":} Following phase-lock pathways from cluster $\mathcal{K}_\alpha$ to adjacent cluster $\mathcal{K}_\beta$ through network edges connecting the clusters. The "door" is not a physical barrier but a categorical boundary between clusters.

    \item \textbf{"Sorting" molecules:} Revealing pre-existing phase-lock cluster structure. The clusters already exist (determined by molecular properties and spatial configuration). The demon does not create the clusters or sort molecules into clusters—it navigates the existing cluster structure.

    \item \textbf{"Creating temperature difference":} Revealing pre-existing cluster temperature fluctuations (Corollary~\ref{cor:hot_cold_clusters}). Hot and cold clusters exist naturally at equilibrium. The demon does not create temperature differences—it reveals and amplifies pre-existing fluctuations by separating clusters spatially.
\end{enumerate}
The demon is not an intelligent agent making decisions based on measurements. It is a personification of categorical completion: the deterministic process by which categorical states are completed according to phase-lock network topology, independent of kinetic energy, revealing categorical structure that correlates with but is not caused by temperature.
\end{corollary}

\begin{proof}
Each reinterpretation follows from the theorems established in previous sections:

\textbf{(1) Observation as categorical completion:}
From Theorem~\ref{thm:information_free}, categorical selection is the completion of a specific state $C^* \in [C]_{\text{spatial}}$ from an equivalence class. This completion makes adjacent states accessible (Theorem~\ref{thm:phase_lock_accessibility}). The demon's "observation" is this completion process.

\textbf{(2) Door opening as pathway following:}
From Theorem~\ref{thm:categorical_cascade}, completing state $C_i$ initiates a cascade of accessible completions through phase-lock pathways. The "door" is the set of edges $E(\phaselockgraph)$ connecting clusters. "Opening the door" is traversing these edges.

\textbf{(3) Sorting as structure revelation:}
From Corollary~\ref{cor:temperature_not_causal}, cluster structure is independent of temperature and kinetic energy. Clusters exist prior to any "sorting" operation. The demon reveals this structure by navigating it.

\textbf{(4) Temperature difference as fluctuation amplification:}
From Corollary~\ref{cor:hot_cold_clusters}, hot and cold clusters exist at equilibrium. The demon separates these clusters spatially, converting temporal fluctuations into spatial gradients. This is amplification of pre-existing fluctuations, not creation of new temperature differences. \qed
\end{proof}

\begin{remark}[Resolution of the Paradox]
\label{rem:paradox_resolution_temperature}
The traditional Maxwell's demon paradox arises from the apparent ability to create temperature differences (and extract work) without paying thermodynamic costs. The resolution is that the demon does not create temperature differences—it reveals pre-existing cluster temperature fluctuations that exist naturally at equilibrium. The Second Law is preserved because revealing fluctuations does not decrease entropy; in fact, as we prove in Section~\ref{sec:entropy_mechanism}, the process of revealing and separating clusters increases entropy in both compartments symmetrically. The demon cannot extract work because the revealed temperature differences are fluctuations, not systematic gradients that can drive heat engines. The categorical framework dissolves the paradox by showing that "sorting by temperature" is a mischaracterization of categorical navigation through phase-lock cluster structure.
\end{remark}
