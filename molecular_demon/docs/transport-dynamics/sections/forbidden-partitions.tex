%==============================================================================
\section{Dissipationless States: Forbidden Partition Regimes}
\label{sec:forbidden}
%==============================================================================

The partition extinction theorem (Section~\ref{sec:extinction}) establishes that transport coefficients vanish when partition operations become undefined. We now examine three physical realisations of this principle: superconductivity, superfluidity, and Bose-Einstein condensation. Despite their different physical manifestations, all three share a common origin: the categorical unification of carriers that makes partition operations impossible.

\subsection{Superconductivity}

\subsubsection{BCS Theory and Partition Extinction}

BCS theory describes superconductivity through Cooper pairing of electrons mediated by phonon exchange \citep{bardeen1957}. Two electrons with opposite spins $(\uparrow, \downarrow)$ and opposite momenta $(\mathbf{k}, -\mathbf{k})$ form a bound state (Cooper pair) with total spin zero and total momentum zero. The pair wavefunction is:
\begin{equation}
\Psi_{\text{pair}}(\mathbf{r}_1, \mathbf{r}_2) = \phi(\mathbf{r}_1 - \mathbf{r}_2) \cdot \chi_{\text{singlet}},
\label{eq:cooper_wavefunction}
\end{equation}
where $\phi(\mathbf{r}_1 - \mathbf{r}_2)$ is the pair envelope (size $\sim \xi$, the coherence length, typically $\sim 100$ nm) and $\chi_{\text{singlet}} = (|\uparrow\downarrow\rangle - |\downarrow\uparrow\rangle)/\sqrt{2}$ is the spin singlet state.

In the partition framework, Cooper pairing is categorical unification: two electrons become a single categorical entity. The pair has no internal structure accessible to measurement. Asking ``which electron is which?'' is undefined because the wavefunction is antisymmetric under exchange.

\begin{theorem}[Superconducting Resistivity]
\label{thm:superconducting}
Below the critical temperature $T_c$, the electrical resistivity of a superconductor is:
\begin{equation}
\rho(T < T_c) = 0
\label{eq:rho_super}
\end{equation}
exactly, not approximately.
\end{theorem}

\begin{proof}
Cooper pairs form when $k_B T < \Delta$, where $\Delta$ is the superconducting gap (the energy required to break a pair). Below $T_c$, all conduction electrons form Cooper pairs. The pairs are categorically unified—partition operations between them are undefined.

From the resistivity formula (Eq.~\ref{eq:resistivity_partition}):
\begin{equation}
\rho = \frac{m}{ne^2} \sum_{i,j} \tau_{p,ij} g_{ij},
\end{equation}
where the sum is over carrier-scatterer pairs.

\textbf{Above $T_c$:} Electrons are distinguishable. Scattering events (electron-phonon, electron-impurity) are partition operations that randomise momentum. The partition lag $\tau_p > 0$, giving $\rho > 0$.

\textbf{Below $T_c$:} Cooper pairs are categorically unified. They form a single macroscopic wavefunction (the BCS ground state). Partition operations between pairs are undefined because pairs are indistinguishable. The partition lag $\tau_p = 0$, giving $\rho = 0$ exactly.

The transition is discontinuous: at $T = T_c$, the partition lag drops from $\tau_p(T_c^+) > 0$ to $\tau_p(T_c^-) = 0$. \qed
\end{proof}

\begin{figure*}[htbp]
\centering
\includegraphics[width=\textwidth]{figures/panel_partition.png}
\caption{\textbf{Partition Lag Across Transport Types: Time Required for Categorical Determination.} 
(\textbf{Electric: Partition Lag $\tau_p$}) The partition lag for electrical transport decreases with temperature for all scattering mechanisms. Phonon scattering (orange curve) shows strong decrease from $\tau_p \sim 10^2$ fs at 50 K to $\sim 10^1$ fs at 500 K—higher temperature means faster categorical determination. Impurity scattering (magenta curve) shows similar trend but with longer lag times ($\tau_p \sim 10^5$ fs at low $T$)—defects create persistent barriers that require more time to resolve. 
(\textbf{Diffusive: Partition Lag $\tau_p$}) The partition lag for diffusive transport spans an enormous range: 15 orders of magnitude from $10^1$ fs to $10^{16}$ fs. Vacancy jump (bright green curve) shows the longest lag times ($\tau_p \sim 10^{16}$ fs at 400 K)—vacancies are rare, so waiting for a vacancy to arrive takes enormous time. Interstitial diffusion (green curve) is faster ($\tau_p \sim 10^{13}$ fs)—interstitials are more mobile. Grain boundary diffusion (dark green curve) is much faster ($\tau_p \sim 10^9$ fs)—boundaries provide fast pathways. All mechanisms show exponential decrease with temperature: $\tau_p \propto \exp(\Phi/kT)$. This demonstrates that partition lag is the microscopic origin of diffusion barriers. 
(\textbf{Thermal: Partition Lag $\tau_p$}) The partition lag for thermal transport varies with phonon frequency and scattering mechanism. Normal scattering (green curve) shows constant lag ($\tau_p \sim 10^3$ ps) independent of frequency—normal processes conserve momentum and require no categorical determination. Umklapp scattering (orange curve) shows decreasing lag with frequency—high-frequency phonons scatter more frequently. Boundary scattering (magenta curve) shows weak frequency dependence. Impurity scattering (cyan curve) shows intermediate behavior. The dramatic difference between normal ($\tau_p \sim 10^3$ ps) and umklapp ($\tau_p \sim 10^2$ ps) explains why umklapp processes dominate thermal resistance at high temperature—they have shorter partition lag and thus higher scattering rate. 
(\textbf{Viscous: Partition Lag $\tau_p$}) The partition lag for viscous flow decreases with temperature for all fluids. Water (cyan curve) has the shortest lag ($\tau_p \sim 10^9$ ps at 200 K, decreasing to $10^8$ ps at 600 K)—water molecules rearrange quickly. Glycerol (magenta curve) has much longer lag ($\tau_p \sim 10^{17}$ ps at 200 K)—glycerol is highly viscous and rearranges slowly. n-Hexane (green curve) shows intermediate behavior. The enormous variation (9 orders of magnitude) demonstrates that partition lag captures the microscopic origin of viscosity: $\mu \propto \tau_p \cdot g$. Longer partition lag means slower categorical determination, which manifests as higher viscosity.}
\label{fig:partition_lag}
\end{figure*}

\subsubsection{The BCS Gap Relation}

The BCS theory predicts \citep{bardeen1957}:
\begin{equation}
\Delta(T=0) = 1.76 \, k_B T_c
\label{eq:bcs_gap}
\end{equation}
for weak-coupling superconductors. This relation follows from the partition framework as the condition for categorical unification.

\begin{proof}
The superconducting gap $\Delta$ is the energy required to break a Cooper pair, creating two quasiparticle excitations. At $T = T_c$, thermal energy $k_B T_c$ is sufficient to break pairs, destroying the condensate.

The BCS gap equation is:
\begin{equation}
\Delta(T) = \Delta(0) \tanh\left(\frac{\Delta(T)}{2k_B T}\right).
\label{eq:gap_equation}
\end{equation}

At $T = T_c$, $\Delta(T_c) = 0$ (gap closes). Expanding near $T_c$ gives:
\begin{equation}
T_c = \frac{\Delta(0)}{1.76 \, k_B}.
\end{equation}

The factor 1.76 arises from the density of states at the Fermi surface and the logarithmic dependence of the gap equation. In partition terms, it represents the ratio of phase-locking energy (zero-temperature gap $\Delta(0)$) to thermal energy at the transition ($k_B T_c$). \qed
\end{proof}

For typical superconductors: $T_c \sim 1$--$10$ K, giving $\Delta(0) \sim 0.2$--$2$ meV. The coherence length is $\xi \sim \hbar v_F/\Delta \sim 100$--$1000$ nm, where $v_F$ is the Fermi velocity.

\subsubsection{Meissner Effect}

Superconductors expel magnetic fields (Meissner effect) \citep{meissner1933}. A magnetic field applied to a superconductor is expelled from the interior, with the field decaying exponentially over the London penetration depth $\lambda_L \sim 50$--$500$ nm.

\begin{proposition}[Meissner Effect as Partition Resistance]
\label{prop:meissner}
Magnetic fields create Landau levels that partition electron momentum space. Cooper pairs resist partition because it would break the categorical unity of the condensate. The field is expelled to preserve categorical unification.
\end{proposition}

\begin{proof}
A magnetic field $\mathbf{B}$ quantises electron motion into Landau levels with energies $E_n = (n + 1/2)\hbar\omega_c$, where $\omega_c = eB/m$ is the cyclotron frequency. These levels partition momentum space: electrons are assigned to specific Landau levels.

Cooper pairs are categorically unified. Assigning pairs to different Landau levels would distinguish them, breaking the categorical unity. The condensate resists this partition by generating screening currents that expel the field.

The critical field $H_c$ is the field strength at which the magnetic partition energy equals the pairing energy:
\begin{equation}
\frac{\mu_0 H_c^2}{2} \sim n\Delta,
\label{eq:critical_field}
\end{equation}
where $n$ is the carrier density. Above $H_c$, magnetic partition overwhelms pairing, and superconductivity is destroyed. \qed
\end{proof}

For typical superconductors: $H_c \sim 0.01$--$0.1$ T (Type I), $H_{c2} \sim 1$--$100$ T (Type II upper critical field).

\subsubsection{Persistent Currents}

In a superconducting ring, current flows indefinitely without decay. Experiments have observed persistent currents lasting years without measurable decay \citep{file1964}. This is direct evidence of zero partition lag: with no scattering to dissipate momentum, current persists.

The current is quantised in units of the flux quantum:
\begin{equation}
\Phi_0 = \frac{h}{2e} = 2.07 \times 10^{-15} \text{ Wb},
\label{eq:flux_quantum}
\end{equation}
reflecting the single categorical state of the condensate. The factor of $2e$ (charge of a Cooper pair) rather than $e$ (charge of an electron) confirms that the carriers are pairs, not individual electrons.

The quantisation arises from the single-valuedness of the macroscopic wavefunction. The phase $\theta(\mathbf{r})$ of the condensate must satisfy:
\begin{equation}
\oint \nabla\theta \cdot d\mathbf{l} = 2\pi n, \quad n \in \mathbb{Z},
\label{eq:phase_quantization}
\end{equation}
around any closed loop. This topological constraint quantises the circulation and hence the magnetic flux.

\begin{figure*}[htbp]
\centering
\includegraphics[width=\textwidth]{figures/panel_current_cross_sectional_validation.png}
\caption{\textbf{Cross-Sectional Validation of S-Transformation in Current Flow.} 
(\textbf{A}) S-coordinate evolution along 10 cm wires of copper, aluminum, and tungsten under constant current. Each point represents a cross-sectional measurement (equivalent to electric field measurement). The S-coordinate remains constant along uniform wires, confirming that current is an S-transformation phenomenon. 
(\textbf{B}) Transformation validation: Predicted $S_k$ from partition lag $\tau_{p,k}$ versus calculated $S_k$ from resistivity formula. Perfect agreement ($R^2 = 1.0000$) for all three materials validates the S-transformation theory. 
(\textbf{C}) Electric field profile along wires. The E-field is constant in uniform conductors ($E = V/L$), confirming that gradients indicate non-uniformity. 
(\textbf{D}) Resistance accumulation according to Ohm's law: $R = \rho L/A$. Cumulative resistance increases linearly with position for uniform materials. Copper has the lowest resistivity ($\rho = 1.68~\mu\Omega\cdot$cm), tungsten the highest ($\rho = 5.60~\mu\Omega\cdot$cm). 
(\textbf{E}) Scattering memory accumulation: $\text{Memory} = \int \tau_s \cdot g_{\text{lat}} \cdot |dS|$, analogous to viscosity in fluids. The memory saturates when scattering events fill the available phase space. Copper saturates fastest (highest conductivity), tungsten slowest (highest resistivity). 
(\textbf{F}) Newton's cradle model of current: Electrons push adjacent electrons in a chain reaction. Cross-sections measure S-coordinates at each position $x_i$. The S-transformation propagates as $S(x_{i+1}) = \tau_{p,k}[S(x_i)]$, where $\tau_{p,k}$ is the partition lag operator.}
\label{fig:cross_sectional_validation}
\end{figure*}

\subsection{Superfluidity}

\subsubsection{Helium-4 Below $T_\lambda$}

Liquid helium-4 undergoes a phase transition at the $\lambda$-point $T_\lambda = 2.17$ K, below which it becomes superfluid \citep{kapitza1938,allen1938}. The viscosity drops to exactly zero, and the liquid exhibits remarkable properties: it flows without friction, climbs container walls, and supports quantised vortices.

\begin{theorem}[Superfluid Viscosity]
\label{thm:superfluid}
Below $T_\lambda$, the viscosity of helium-4 is:
\begin{equation}
\mu(T < T_\lambda) = 0
\label{eq:mu_super}
\end{equation}
exactly, not approximately.
\end{theorem}

\begin{proof}
Helium-4 atoms are bosons (spin-0). Below $T_\lambda$, a macroscopic fraction condenses into the ground state, forming a Bose-Einstein condensate. Condensed atoms are categorically unified—they form a single categorical entity described by a macroscopic wavefunction $\Psi(\mathbf{r}, t)$.

From the viscosity formula (Eq.~\ref{eq:viscosity_partition}):
\begin{equation}
\mu = \sum_{i,j} \tau_{c,ij} g_{ij},
\end{equation}
where the sum is over molecular collision pairs.

\textbf{Above $T_\lambda$:} Atoms are distinguishable. Collisions between atoms are partition operations that randomize velocities. The partition lag $\tau_c > 0$, giving $\mu > 0$.

\textbf{Below $T_\lambda$:} Condensed atoms are categorically unified. They occupy the same quantum state, so they are indistinguishable. Partition operations between condensed atoms are undefined. The partition lag $\tau_c = 0$, giving $\mu = 0$ exactly.

The transition is continuous (second-order phase transition): the superfluid fraction $\rho_s/\rho$ grows continuously from zero as $T$ decreases below $T_\lambda$. \qed
\end{proof}

\subsubsection{Two-Fluid Model}

Landau's two-fluid model describes superfluid helium as a mixture of two interpenetrating fluids \citep{landau1941}:

\begin{itemize}
\item \textbf{Superfluid component:} Density $\rho_s$. Atoms in the condensate (categorically unified). Zero viscosity, zero entropy. Flows without friction.

\item \textbf{Normal component:} Density $\rho_n$. Atoms in excited states (categorically distinguishable). Finite viscosity, carries entropy. Behaves like a normal fluid.
\end{itemize}

The total density is $\rho = \rho_s + \rho_n$. The superfluid fraction increases as $T \to 0$:
\begin{equation}
\frac{\rho_s}{\rho} = 1 - \left(\frac{T}{T_\lambda}\right)^{\alpha},
\label{eq:superfluid_fraction}
\end{equation}
where $\alpha \approx 5.6$ near $T_\lambda$ \citep{donnelly1998}. At $T = 0$, $\rho_s/\rho = 1$ (all atoms in condensate). At $T = T_\lambda$, $\rho_s/\rho = 0$ (no condensate).

The two-fluid model explains several phenomena:

\textbf{Second sound:} Temperature waves propagate through the superfluid. The normal and superfluid components oscillate out of phase, creating a temperature oscillation without density oscillation.

\textbf{Thermomechanical effects:} Heating one end of a capillary filled with superfluid creates a pressure difference (fountain effect). The normal fluid is created by heating, while the superfluid flows to maintain equilibrium.

\subsubsection{Wall Climbing and Fountain Effect}

Superfluid helium climbs container walls and can empty an open container \citep{rollin1936}. A thin film (thickness $\sim 30$ nm) forms on walls and flows upward, driven by van der Waals attraction to the wall.

This occurs because:
\begin{enumerate}
\item Zero viscosity means no partition barrier to flow. The superfluid experiences no friction as it flows along the wall.

\item The superfluid redistributes to minimize total energy. The film lowers the system energy by maximizing contact with the wall.

\item If the container is open, the film can flow up and over the rim, emptying the container at a rate $\sim 10^{-7}$ m$^3$/s.
\end{enumerate}

The fountain effect demonstrates thermomechanical coupling. When superfluid in a capillary is heated, normal fluid is created (breaking categorical unity). The resulting osmotic pressure drives superfluid flow, which can produce a fountain several centimeters high.

\subsubsection{Quantized Vortices}

Rotation in a superfluid is quantised. Vortices carry circulation in units of:
\begin{equation}
\kappa = \frac{h}{m_4} = 9.97 \times 10^{-8} \text{ m}^2/\text{s},
\label{eq:circulation_quantum}
\end{equation}
where $m_4 = 6.65 \times 10^{-27}$ kg is the mass of a helium-4 atom \citep{vinen1961}.

This quantisation reflects the single categorical state of the condensate. The phase $\theta(\mathbf{r})$ of the macroscopic wavefunction must satisfy:
\begin{equation}
\oint \nabla\theta \cdot d\mathbf{l} = 2\pi n, \quad n \in \mathbb{Z},
\label{eq:phase_quantization_sf}
\end{equation}
around any closed loop. The velocity field is $\mathbf{v} = (\hbar/m_4)\nabla\theta$, giving circulation:
\begin{equation}
\Gamma = \oint \mathbf{v} \cdot d\mathbf{l} = \frac{\hbar}{m_4} \oint \nabla\theta \cdot d\mathbf{l} = n\kappa.
\end{equation}

Classical vorticity is continuous; superfluid vorticity is discrete because the condensate cannot support arbitrary phase gradients. Each vortex has a core (radius $\sim 0.1$ nm) where the condensate density vanishes, surrounded by circulating superfluid.

\subsection{Bose-Einstein Condensation}

\subsubsection{Dilute Atomic Gases}

In dilute atomic gases cooled to nanokelvin temperatures, atoms condense into the ground state, forming a Bose-Einstein condensate (BEC). The first experimental realisation was in 1995 using rubidium-87 \citep{anderson1995} and sodium-23 \citep{davis1995}, achieving temperatures $T \sim 100$--$500$ nK and densities $n \sim 10^{13}$--$10^{15}$ cm$^{-3}$.

The condensate is a macroscopic occupation of a single quantum state. Typical condensates contain $10^4$--$10^7$ atoms, all in the same state $|\psi_0\rangle$. The macroscopic wavefunction is:
\begin{equation}
\Psi(\mathbf{r}, t) = \sqrt{n_0(\mathbf{r})} \, e^{i\theta(\mathbf{r}, t)},
\label{eq:bec_wavefunction}
\end{equation}
where $n_0(\mathbf{r})$ is the condensate density and $\theta(\mathbf{r}, t)$ is the phase.

\begin{theorem}[BEC Transition]
\label{thm:BEC}
Below the critical temperature $T_{\text{BEC}}$, a macroscopic fraction of atoms occupies the ground state:
\begin{equation}
\frac{N_0}{N} = 1 - \left(\frac{T}{T_{\text{BEC}}}\right)^{3/2},
\label{eq:condensate_fraction}
\end{equation}
where $N_0$ is the number of condensed atoms and $N$ is the total number of atoms.
\end{theorem}

\begin{proof}
From Bose-Einstein statistics, the number of atoms in excited states (with chemical potential $\mu = 0$ at the transition) is:
\begin{equation}
N - N_0 = \int_0^\infty \frac{g(\varepsilon)}{e^{\varepsilon/k_B T} - 1} d\varepsilon,
\end{equation}
where $g(\varepsilon) \propto \varepsilon^{1/2}$ is the density of states for a three-dimensional harmonic trap. Evaluating the integral:
\begin{equation}
N - N_0 = C \, T^{3/2},
\end{equation}
where $C$ is a constant determined by the trap geometry. At $T = T_{\text{BEC}}$, $N_0 = 0$ (condensate just forms), giving $N = C \, T_{\text{BEC}}^{3/2}$. For $T < T_{\text{BEC}}$:
\begin{equation}
\frac{N_0}{N} = 1 - \frac{N - N_0}{N} = 1 - \left(\frac{T}{T_{\text{BEC}}}\right)^{3/2}.
\end{equation}
\qed
\end{proof}

At $T = 0$, all atoms are in the condensate ($N_0/N = 1$). At $T = T_{\text{BEC}}$, the condensate vanishes ($N_0/N = 0$). The transition is continuous (second-order).

\subsubsection{Partition Interpretation}

In a classical gas, atoms are distinguishable. Collisions between atoms are partition operations that randomise trajectories. The mean free path is $\lambda \sim 1/(n\sigma)$, where $\sigma$ is the collision cross-section.

In a BEC, condensed atoms are indistinguishable. They occupy the same categorical state $|\psi_0\rangle$. Partition operations between condensed atoms are undefined because there is no way to distinguish which atom is which.

The normal-to-BEC transition is partition extinction:
\begin{itemize}
\item \textbf{Above $T_{\text{BEC}}$:} Atoms scatter. Partition operations occur with a lag $\tau_p > 0$. Diffusivity is finite: $D = \lambda^2/(2\tau_p)$.

\item \textbf{Below $T_{\text{BEC}}$:} Condensed atoms form a single categorical entity. Partition operations are undefined: $\tau_p = 0$. Diffusivity diverges: $D \to \infty$ (ballistic transport).
\end{itemize}

The BEC does not diffuse in the usual sense—it propagates coherently as a single quantum object.

\subsubsection{Coherence and Interference}

BEC exhibits macroscopic coherence—all condensed atoms share the same quantum phase $\theta(\mathbf{r}, t)$. Interference between two BECs produces fringes \citep{andrews1997}, demonstrating the single categorical state of each condensate.

In the experiment, two condensates are created in separate traps, then released and allowed to overlap. The density distribution shows interference fringes:
\begin{equation}
n(\mathbf{r}) = n_1(\mathbf{r}) + n_2(\mathbf{r}) + 2\sqrt{n_1(\mathbf{r}) n_2(\mathbf{r})} \cos[\theta_1(\mathbf{r}) - \theta_2(\mathbf{r})],
\label{eq:interference}
\end{equation}
where $n_1, n_2$ are the densities and $\theta_1, \theta_2$ are the phases of the two condensates.

This coherence is the signature of categorical unification. Distinguishable atoms would not interfere (the phases would average to zero). Unified atoms produce coherent interference patterns because they share a common phase.

\subsubsection{Quantised Vortices in BECs}

Like superfluid helium, BECs support quantised vortices. Stirring a BEC creates vortices with circulation quantised in units of $\kappa = h/m$, where $m$ is the atomic mass \citep{madison2000}. For rubidium-87, $\kappa = 4.6 \times 10^{-9}$ m$^2$/s.

The vortex core (radius $\sim \xi$, the healing length, typically $\sim 0.5$ $\mu$m) is a region where the condensate density vanishes. The phase winds by $2\pi$ around the core, creating quantised circulation.

\subsection{Unified Perspective}

Despite their different physical manifestations, superconductivity, superfluidity, and Bose-Einstein condensation share a common origin: the extinction of partition operations when carriers become categorically unified.

\begin{table}[h]
\centering
\caption{Comparison of dissipationless states}
\label{tab:dissipationless}
\begin{tabular}{lccc}
\toprule
\textbf{Property} & \textbf{Superconductor} & \textbf{Superfluid He-4} & \textbf{BEC} \\
\midrule
Carriers & Electrons & He-4 atoms & Various atoms \\
Statistics & Fermions & Bosons & Bosons \\
Pairing & Cooper pairs & None & None \\
Critical temp & $T_c \sim 1$--$10$ K & $T_\lambda = 2.17$ K & $T_{\text{BEC}} \sim$ nK \\
Zero coefficient & Resistivity $\rho$ & Viscosity $\mu$ & Scattering rate \\
Quantization & Flux $\Phi_0 = h/2e$ & Circulation $\kappa = h/m_4$ & Circulation $\kappa = h/m$ \\
Coherence length & $\xi \sim 100$ nm & $\xi \sim 0.1$ nm & $\xi \sim 0.5$ $\mu$m \\
\midrule
\textbf{Mechanism} & \multicolumn{3}{c}{Partition extinction through categorical unification} \\
\bottomrule
\end{tabular}
\end{table}

In each case:
\begin{enumerate}
\item \textbf{Carriers that were distinguishable become indistinguishable:} Electrons form Cooper pairs (superconductor), helium atoms condense (superfluid), and trapped atoms condense (BEC).

\item \textbf{Partition operations that were defined become undefined:} Scattering events that distinguished individual carriers can no longer occur because carriers are categorically unified.

\item \textbf{Transport coefficients that were finite become exactly zero:} Resistivity (superconductor), viscosity (superfluid), and scattering rate (BEC) all vanish exactly, not approximately.

\item \textbf{Macroscopic quantum coherence emerges:} The system is described by a single macroscopic wavefunction with a well-defined phase. Interference and quantisation phenomena appear.

\item \textbf{Topological constraints arise:} The single-valuedness of the macroscopic wavefunction quantizes circulation (flux quanta, vortex quanta).
\end{enumerate}

The dissipationless states are not anomalies requiring special explanation. They are the natural consequence of partition dynamics when partition becomes impossible. Transport without partition is transport without dissipation.

\begin{figure*}[htbp]
\centering
\includegraphics[width=\textwidth]{figures/panel_transformation_operator.png}
\caption{\textbf{The S-Transformation Operator: Decomposition and Experimental Validation.} 
(\textbf{A}) Operator decomposition: An initial S-profile (black dashed) undergoes three sequential transformations: advection $\mathcal{T}_{\text{adv}}$ (blue), diffusion $\mathcal{T}_{\text{diff}}$ (green), and partition $\mathcal{T}_{\text{part}}$ (red, final). Each operator modifies the S-coordinate distribution in a characteristic way. 
(\textbf{B}) Partition operator equilibration: S-coordinate evolution toward stationary state $S_{\text{stat}}$ (red dashed line). Four trajectories with initial values $S_0 = 1.0, 3.0, 7.0, 9.0$ all converge to $S_{\text{stat}} = 5.0$. The approach is exponential with time constant $\tau_p$. 
(\textbf{C}) Diffusion operator S-spreading: S-density profiles at times $t = 0, 0.5, 1.0, 2.0, 4.0$. An initially sharp peak (purple, $t = 0$) spreads according to $\sigma = \sqrt{2 D_S t}$. The peak height decreases and width increases while conserving total S-coordinate. 
(\textbf{D}) Advection operator S-translation: S-profiles at times $t = 0, 1, 2, 3, 4$ (cyan to magenta). The profile translates rigidly with velocity $v = 2.0$ (purple arrow). The profile shape is preserved during translation. 
(\textbf{E}) Operator composition: Relative error in $\mathcal{T}_{0 \to x} = \mathcal{T}_{dx}^{(x/dx)}$ versus number of steps. The error decreases exponentially, reaching $< 0.1\%$ after 100 steps. This validates the composition property of S-transformations. 
(\textbf{F}) Partition coefficient: $K = K_0 \exp(-d_S/\sigma_S)$ versus S-distance $d_S$ for four values of $\sigma_S = 0.5, 1.0, 2.0, 3.0$ (red to yellow). Larger $\sigma_S$ gives slower decay, indicating weaker selectivity. The partition coefficient quantifies how readily a system transitions between categorical states separated by S-distance $d_S$.}
\label{fig:transformation_operator}
\end{figure*}

\subsection{Experimental Verification}

The partition extinction framework makes several testable predictions:

\begin{enumerate}
\item \textbf{Discontinuous transitions:} Transport coefficients should drop discontinuously (first-order) or with a sharp kink (second-order) at $T_c$. \textit{Observed:} Superconductors show sharp resistivity drops \citep{onnes1911}, and superfluids show sharp viscosity drops \citep{allen1938}.

\item \textbf{Exactly zero, not small:} Transport coefficients should be exactly zero below $T_c$, not merely very small. \textit{Observed:} Persistent currents in superconductors last years without decay \citep{file1964}, superfluid viscosity is unmeasurably small ($< 10^{-11}$ Pa$\cdot$s).

\item \textbf{Quantization:} Collective excitations should be quantised due to topological constraints. \textit{Observed:} Flux quantisation in superconductors \citep{doll1961,deaver1961}, circulation quantisation in superfluids \citep{vinen1961}, and vortex quantisation in BECs \citep{madison2000}.

\item \textbf{Coherence:} Macroscopic quantum coherence should be observable through interference. \textit{Observed:} Josephson effect in superconductors \citep{josephson1962}, BEC interference \citep{andrews1997}.

\item \textbf{Gap energy:} An energy gap should separate the ground state from excited states. \textit{Observed:} Superconducting gap measured by tunneling \citep{giaever1960}, roton minimum in superfluid helium \citep{henshaw1953}.
\end{enumerate}

All predictions are confirmed, supporting the partition extinction framework.
