\section{Ternary Representation of Transport Dynamics}
\label{sec:ternary}

Traditional transport theory employs continuous variables (velocity, position, energy) or binary logic (occupied/unoccupied, spin up/down). We demonstrate that transport phenomena are naturally encoded in ternary (base-3) representation due to the three-fold partition-oscillation-category equivalence established in Section~\ref{sec:universal}. This ternary structure is not merely a convenient encoding but reflects the fundamental three-dimensional nature of the categorical state space and the tripartite structure of transport processes.

\subsection{Transport as Three-Dimensional Navigation}

The partition-oscillation-category equivalence underlying transport phenomena maps naturally to ternary representation. Each transport event involves three equivalent aspects that are not independent perspectives but identical descriptions in different coordinates:

\begin{enumerate}
    \item \textbf{Oscillatory aspect}: The carrier oscillation frequency $\omega$ that drives transport, with characteristic time $\tau_{\text{osc}} = 2\pi/\omega$
    \item \textbf{Categorical aspect}: The state transition in S-entropy space $\mathbf{S} = (S_k, S_t, S_e)$ that constitutes transport
    \item \textbf{Partition aspect}: The selection operation with partition lag $\taulag$ that determines the destination state
\end{enumerate}

These three aspects are related by the fundamental equivalence:
\begin{equation}
\text{Oscillatory dynamics} \equiv \text{Categorical transitions} \equiv \text{Partition operations}
\end{equation}

The identity (not merely equivalence) of these three descriptions suggests base-3 as the natural representation for transport dynamics, where each ternary digit (trit) encodes one aspect of the transport process.

\subsection{The Trit-Transport Correspondence}

\begin{definition}[Transport Trit Encoding]
\label{def:transport_trit}
A transport event at hierarchical level $j$ encodes as a trit $t_j \in \{0, 1, 2\}$ according to the dominant aspect:
\begin{center}
\begin{tabular}{clp{6cm}}
\toprule
\textbf{Trit} & \textbf{Transport Aspect} & \textbf{Physical Meaning} \\
\midrule
0 & Oscillatory phase & Carrier wave propagation; coherent transport dominated by $\omega$ \\
1 & Categorical state & Scattering/transition event; state change in S-space \\
2 & Partition operation & Channel selection; branching into distinct trajectories \\
\bottomrule
\end{tabular}
\end{center}
\end{definition}

A complete transport trajectory through a material of depth $k$ is thus a $k$-trit ternary string $T = t_1 t_2 \cdots t_k$, with each trit recording which aspect dominated at each hierarchical level. The string encodes both the final state and the path taken to reach it—a property unique to ternary representation that distinguishes it from binary encoding.

\begin{example}[Electron Transport in Copper]
Consider an electron traversing a copper crystal with $k=6$ scattering events:
\begin{equation}
T = 012102
\end{equation}
This encodes:
\begin{itemize}
    \item $t_1 = 0$: Initial propagation (oscillatory)
    \item $t_2 = 1$: First scattering (categorical transition)
    \item $t_3 = 2$: Channel selection (partition)
    \item $t_4 = 1$: Second scattering
    \item $t_5 = 0$: Propagation segment
    \item $t_6 = 2$: Final channel selection
\end{itemize}
The complete trajectory is uniquely specified by this 6-trit string, which addresses one of $3^6 = 729$ possible transport pathways.
\end{example}

\subsection{The $3^k$ Transport Hierarchy}

Transport processes exhibit a hierarchical structure that matches the $3^k$ ternary hierarchy, a direct consequence of the three-dimensional S-entropy space.

\begin{proposition}[$3^k$ Transport Structure]
\label{prop:3k_hierarchy}
At hierarchical depth $k$, the transport structure satisfies:
\begin{enumerate}
    \item There exist exactly $3^k$ distinct transport channels
    \item Each channel is uniquely addressed by a $k$-trit string $T = t_1 t_2 \cdots t_k$
    \item The trit string encodes both the channel identity and the navigation path through S-space
    \item The channel volume scales as $V_k = 3^{-k}$ in normalized S-coordinates
\end{enumerate}
\end{proposition}

\begin{proof}
Each level of refinement divides the current cell into three subcells along one of the three S-entropy axes (cyclic: $S_k$, $S_t$, $S_e$). After $k$ refinements:
\begin{itemize}
    \item Number of cells: $3^k$ (each refinement triples the count)
    \item Cell volume: $(1/3)^{k/3}$ per dimension $\times$ 3 dimensions $= 3^{-k}$ total
    \item Unique addressing: Each $k$-trit string specifies a unique refinement sequence
\end{itemize}
The bijection between $k$-trit strings and cells follows from the construction.
\end{proof}

\begin{figure}[htbp]
\centering
\includegraphics[width=\textwidth]{figures/panel_vap_results.png}
\caption{\textbf{Virtual Aperture Potentiometer (VAP) results showing aperture potentials and selectivities.} 
\textbf{(Top left)} Aperture potentials by material showing distribution of $\Phi_a/k_B T$ for different aperture types. Copper (orange bars) has moderate aperture potentials ($\Phi/k_B T \sim 1.2$) from phonon, impurity, and boundary scattering. Silicon (green bars) has higher potentials ($\Phi/k_B T \sim 1.6$) due to larger band gap and stronger scattering. YBCO below $T_c$ (cyan bars) has very low potentials, approaching zero as Cooper pairs bypass apertures.
\textbf{(Top right)} Selectivity spectrum showing aperture selectivity $s_a = \Omega_{\text{pass}}/\Omega_{\text{total}}$ for copper (orange/green stacked bars) and YBCO below $T_c$ (cyan bar). Copper has moderate selectivity ($s \sim 0.1$--$1$) with contributions from phonon (orange) and impurity (green) apertures. YBCO has very high selectivity ($s \sim 3$, effectively unity) as Cooper pairs pass through all apertures without scattering.
\textbf{(Bottom left)} Categorical potential vs. selectivity showing universal relationship $\Phi/k_B T = -\ln s$ (white line). Data points for copper (orange), silicon (green), and YBCO below $T_c$ (cyan) all fall on this line, confirming the categorical interpretation of aperture potentials. High selectivity ($s \to 1$) gives low potential ($\Phi \to 0$). Low selectivity ($s \ll 1$) gives high potential ($\Phi \gg k_B T$).
\textbf{(Bottom right)} Total aperture potential (transport coefficient) showing sum $\sum_a \Phi_a/k_B T$ for different materials. YBCO below $T_c$ (green) has zero total potential, corresponding to zero resistivity (superconductor). Silicon (green) has moderate total potential $\sum \Phi_a/k_B T \sim 3.17$. Copper (orange) has low total potential $\sum \Phi_a/k_B T \sim 4.02$, corresponding to low resistivity. The total aperture potential is proportional to the transport coefficient: $\rho \propto \sum_a \Phi_a$.}
\label{fig:vap_results}
\end{figure}

\subsubsection{Application to Phonon Transport}

For thermal transport via phonons, the $3^k$ hierarchy manifests in the phonon mode structure:

\begin{itemize}
    \item \textbf{$k = 1$ (3 channels)}: Three acoustic branches
    \begin{itemize}
        \item $t = 0$: Longitudinal acoustic (LA)
        \item $t = 1$: Transverse acoustic 1 (TA$_1$)
        \item $t = 2$: Transverse acoustic 2 (TA$_2$)
    \end{itemize}
    
    \item \textbf{$k = 2$ (9 channels)}: Mode-mode interactions
    \begin{itemize}
        \item $t_1 t_2 = 00$: LA-LA scattering
        \item $t_1 t_2 = 01$: LA-TA$_1$ scattering
        \item $t_1 t_2 = 12$: TA$_1$-TA$_2$ scattering
        \item $\ldots$ (9 combinations total)
    \end{itemize}
    
    \item \textbf{$k = 3$ (27 channels)}: Three-phonon processes
    \begin{itemize}
        \item Normal processes (N-processes): momentum-conserving
        \item Umklapp processes (U-processes): momentum-changing
        \item Each is encoded by a specific 3-trit string
    \end{itemize}
\end{itemize}

The phonon chromatograph (Section~\ref{sec:instruments}) exploits this hierarchy to perform mode-resolved thermal conductivity measurements, separating contributions from different branches by their distinct partition lags.

\subsection{Partition Lag in Ternary Decomposition}

The partition lag $\taulag$ between carriers admits a natural ternary decomposition corresponding to the three equivalent aspects of transport.

\begin{theorem}[Ternary Partition Lag Decomposition]
\label{thm:ternary_tau}
The total partition lag decomposes uniquely as:
\begin{equation}
\taulag = \taulag^{(0)} + \taulag^{(1)} + \taulag^{(2)}
\end{equation}
where:
\begin{align}
\taulag^{(0)} &= \text{oscillatory dephasing time} = \frac{2\pi}{|\omega_i - \omega_j|} \\
\taulag^{(1)} &= \text{categorical transition time} = \frac{\|\mathbf{S}_i - \mathbf{S}_j\|}{|\dot{\mathbf{S}}|} \\
\taulag^{(2)} &= \text{partition selection time} = \tau_{\text{branch}}
\end{align}
Each component measures the delay from one of the three equivalent perspectives. The components are not independent but are related through the partition-oscillation-category equivalence.
\end{theorem}

\begin{proof}
From the equivalence established in Section~\ref{sec:universal}:
\begin{enumerate}
    \item \textbf{Oscillatory}: Carriers $i$ and $j$ with frequencies $\omega_i$ and $\omega_j$ dephase over time $\taulag^{(0)} = 2\pi/|\omega_i - \omega_j|$
    
    \item \textbf{Categorical}: The same carriers occupy states $\mathbf{S}_i$ and $\mathbf{S}_j$ in S-space. The time to traverse this distance at velocity $|\dot{\mathbf{S}}|$ is $\taulag^{(1)} = \|\mathbf{S}_i - \mathbf{S}_j\|/|\dot{\mathbf{S}}|$
    
    \item \textbf{Partition}: The branching operation that distinguishes carriers requires time $\taulag^{(2)} = \tau_{\text{branch}}$ to complete
\end{enumerate}
The equivalence guarantees these three times sum to the total partition lag. The decomposition is unique because each component corresponds to a distinct coordinate axis in the three-dimensional description space.
\end{proof}

The universal transport formula (Eq.~\ref{eq:universal_transport}) now admits ternary form:
\begin{equation}
\label{eq:ternary_transport}
\Tcoeff = \mathcal{N}^{-1} \sum_{i,j} \left(\taulag^{(0)}_{ij} + \taulag^{(1)}_{ij} + \taulag^{(2)}_{ij}\right) g_{ij}
\end{equation}

This form reveals that transport coefficients measure the sum of delays across all three aspects of carrier interaction. Dissipationless transport requires all three components to vanish simultaneously---a condition satisfied only when carriers become categorically indistinguishable.

\subsection{Partition Extinction as Trit Collapse}

The discontinuous nature of partition extinction finds natural explanation in ternary representation.

\begin{theorem}[Extinction as Ternary Collapse]
\label{thm:trit_collapse}
Partition extinction occurs when the three trit values become indistinguishable. For carriers $i$ and $j$:
\begin{equation}
T \to T_c^- \Rightarrow \text{trit}_i(0) \equiv \text{trit}_i(1) \equiv \text{trit}_i(2) \equiv \text{trit}_j(0) \equiv \text{trit}_j(1) \equiv \text{trit}_j(2)
\end{equation}
Below $T_c$, the ternary structure collapses to a single value: carriers cannot be distinguished along any of the three axes (oscillatory, categorical, partition).
\end{theorem}

\begin{proof}
Consider the partition operation between carriers $i$ and $j$. This operation is defined only if the carriers are distinguishable along at least one of the three axes:
\begin{itemize}
    \item \textbf{Oscillatory}: $\omega_i \neq \omega_j$ (different frequencies)
    \item \textbf{Categorical}: $\mathbf{S}_i \neq \mathbf{S}_j$ (different S-states)
    \item \textbf{Partition}: Different branch assignments possible
\end{itemize}

Phase-locking at $T_c$ enforces:
\begin{align}
\omega_i &= \omega_j \quad \text{(oscillatory indistinguishability)} \\
\mathbf{S}_i &= \mathbf{S}_j \quad \text{(categorical indistinguishability)} \\
\text{branch}_i &= \text{branch}_j \quad \text{(partition indistinguishability)}
\end{align}

When all three conditions hold simultaneously, the partition operation becomes undefined: there is no basis for distinguishing the carriers. The partition lag does not approach zero continuously but undergoes discontinuous collapse:
\begin{equation}
\taulag(T) = \begin{cases}
\tau_0 \exp(-\Delta/k_B T) & T > T_c \quad \text{(finite)} \\
\text{undefined} \to 0 & T = T_c \quad \text{(discontinuous)} \\
0 & T < T_c \quad \text{(extinct)}
\end{cases}
\end{equation}
\end{proof}

This explains why partition extinction is discontinuous: the ternary structure either exists (three distinct values, partition possible) or it doesn't (all values identical, partition undefined). There is no partial collapse because a partition operation is discrete—it either occurs or it doesn't.

\begin{corollary}[Dissipationless Transport]
When the ternary structure collapses, the transport coefficient vanishes identically:
\begin{equation}
\text{Trit collapse} \Rightarrow \taulag = 0 \Rightarrow \Tcoeff = 0
\end{equation}
\end{corollary}

\subsection{S-Entropy Coordinates for Transport}

Transport processes map naturally to the three-dimensional S-entropy coordinate space $\mathbf{S} = (S_k, S_t, S_e) \in [0,1]^3$.

\begin{definition}[Transport S-Coordinates]
\label{def:transport_S}
For a transport event involving carrier $i$:
\begin{align}
S_k^{(i)} &= \text{knowledge entropy} = -\sum_{\alpha} p_\alpha \ln p_\alpha \quad \text{(accessible channels)} \\
S_t^{(i)} &= \text{temporal entropy} = \ln(\Delta t / t_{\text{min}}) \quad \text{(timing uncertainty)} \\
S_e^{(i)} &= \text{evolution entropy} = \ln(\mathcal{N}_{\text{traj}}) \quad \text{(trajectory diversity)}
\end{align}
where $p_\alpha$ is the probability of channel $\alpha$, $\Delta t$ is the timing uncertainty, and $\mathcal{N}_{\text{traj}}$ is the number of accessible trajectories.
\end{definition}

The transport coefficient relates to S-entropy gradients through:
\begin{equation}
\label{eq:transport_S_gradient}
\Tcoeff \propto \left|\frac{\partial \mathbf{S}}{\partial x}\right|^{-1}
\end{equation}

\begin{interpretation}
A high S-entropy gradient (steep change in categorical structure) corresponds to a low transport coefficient (high resistance). Physically:
\begin{itemize}
    \item Large $\partial S_k/\partial x$: Rapid change in available channels $\Rightarrow$ strong scattering
    \item Large $\partial S_t/\partial x$: Rapid change in timing structure $\Rightarrow$ temporal disorder
    \item Large $\partial S_e/\partial x$: Rapid change in trajectory diversity $\Rightarrow$ path interference
\end{itemize}
Conversely, uniform S-entropy (small gradient) enables dissipationless transport.
\end{interpretation}

\subsection{Ternary Operations for Transport Analysis}

Transport analysis employs three fundamental ternary primitives that replace Boolean operations (AND, OR, NOT) in binary logic.

\begin{definition}[Ternary Transport Primitives]
\label{def:ternary_ops}
The three fundamental operations on transport trajectories $T = t_1 t_2 \cdots t_k$ are:

\begin{enumerate}
    \item \textbf{Mode Projection} $\pi_i: \{0,1,2\}^k \to \{0,1,2\}^{k/3}$
    \begin{equation}
    \pi_i(t_1 t_2 \cdots t_k) = t_i t_{i+3} t_{i+6} \cdots \quad (i \in \{0,1,2\})
    \end{equation}
    Extracts the component along axis $i$ (oscillatory, categorical, or partition).
    
    \item \textbf{Trajectory Completion} $\kappa: \{0,1,2\}^j \to \{0,1,2\}^k$ ($j < k$)
    \begin{equation}
    \kappa(t_1 \cdots t_j) = t_1 \cdots t_j \cdot \hat{t}_{j+1} \cdots \hat{t}_k
    \end{equation}
    Predicts the continuation of a partial trajectory based on categorical state dynamics.
    
    \item \textbf{Channel Composition} $\circ: \{0,1,2\}^{k_1} \times \{0,1,2\}^{k_2} \to \{0,1,2\}^{k_1+k_2}$
    \begin{equation}
    T_1 \circ T_2 = t_1^{(1)} \cdots t_{k_1}^{(1)} \cdot t_1^{(2)} \cdots t_{k_2}^{(2)}
    \end{equation}
    Combines sequential transport events into a single trajectory.
\end{enumerate}
\end{definition}

\begin{example}[Mode Projection for Phonon Analysis]
Consider a thermal transport trajectory:
\begin{equation}
T = 012102
\end{equation}
Mode projection extracts:
\begin{align}
\pi_0(T) &= 00 \quad \text{(oscillatory: LA modes)} \\
\pi_1(T) &= 11 \quad \text{(categorical: TA$_1$ modes)} \\
\pi_2(T) &= 22 \quad \text{(partition: TA$_2$ modes)}
\end{align}
This separates the contributions from different phonon branches, enabling mode-resolved thermal conductivity analysis.
\end{example}

\begin{figure*}[htbp]
\centering
\includegraphics[width=\textwidth]{figures/panel2_entropy_derivation.png}
\caption{\textbf{Three Derivations of the Entropy Formula $S = k_B M \ln n$.} 
(\textbf{A}) Oscillatory derivation: For $M = 3$ oscillator modes with $n = 4$ quantum states each, the total number of microstates is $W_{\text{osc}} = 4^3 = 64$. 
(\textbf{B}) Categorical derivation: For $M = 2$ categorical dimensions with $n = 4$ distinguishable states each, the total number of configurations is $|C| = 4 \times 4 = 16$. 
(\textbf{C}) Partition derivation: A tree with $M = 2$ levels and branching factor $n = 3$ has $3^2 = 9$ terminal paths (leaves). One path is highlighted in red. 
(\textbf{D}) Boltzmann's fundamental relation $S = k_B \ln W$ combined with $W = n^M$ yields $S = k_B M \ln n$. 
(\textbf{E}) All three perspectives—oscillators, categorical states, and partition paths—yield the same formula $W = n^M$ and thus $S = k_B M \ln n$. 
(\textbf{F}) Entropy scaling as a function of degrees of freedom $M$ and states per degree of freedom $n$. The contour plot shows $S/k_B$ in the $(M, n)$ plane. The pendulum example (red point) has $M = 1$ mode and $n = 4$ states, giving $S = k_B \ln 4$. The entropy increases linearly with $M$ (horizontal direction) and logarithmically with $n$ (vertical direction).}
\label{fig:entropy_derivation}
\end{figure*}

\subsection{Application to Dissipationless States}

The ternary framework provides a unified understanding of dissipationless transport phenomena.

\subsubsection{Superconductivity}

In the superconducting state below $T_c$:

\begin{itemize}
    \item \textbf{Cooper pair formation} collapses the electronic ternary structure:
    \begin{equation}
    \text{All electrons} \to \text{Single categorical entity (Cooper pair condensate)}
    \end{equation}
    
    \item \textbf{Ternary uniformity}: All carriers share identical trits:
    \begin{equation}
    \text{trit}_i(j) = \text{trit}_{\text{condensate}}(j) \quad \forall i, \, j \in \{0,1,2\}
    \end{equation}
    
    \item \textbf{Hierarchy collapse}: The $3^k$ transport hierarchy reduces to $1^k = 1$:
    \begin{equation}
    \text{Number of distinct channels} = 3^k \to 1
    \end{equation}
    
    \item \textbf{Zero resistivity}: No ternary diversity $\Rightarrow$ no scattering:
    \begin{equation}
    \rho = \frac{1}{ne^2} \sum_{i,j} \taulag_{ij} g_{ij} = 0 \quad \text{(all } \taulag_{ij} = 0\text{)}
    \end{equation}
\end{itemize}

The BCS gap $\Delta_{\text{BCS}} = 1.76 k_B T_c$ represents the energy required to break the ternary uniformity by exciting a quasiparticle out of the condensate.

\subsubsection{Superfluidity}

In the superfluid state of $^4$He below $T_\lambda = 2.17$ K:

\begin{itemize}
    \item \textbf{Bose-Einstein statistics} enforces atomic indistinguishability, collapsing the particle ternary structure:
    \begin{equation}
    \text{All atoms} \to \text{Single quantum state (macroscopic wavefunction)}
    \end{equation}
    
    \item \textbf{Ternary uniformity}: All atoms share an identical categorical state:
    \begin{equation}
    \mathbf{S}_i = \mathbf{S}_{\text{condensate}} \quad \forall i
    \end{equation}
    
    \item \textbf{Momentum hierarchy collapse}: The $3^k$ momentum channel hierarchy reduces to $1^k = 1$
    
    \item \textbf{Zero viscosity}: No ternary diversity $\Rightarrow$; no momentum dissipation:
    \begin{equation}
    \mu = \sum_{i,j} \taulag_{ij} g_{ij} = 0 \quad \text{(all } \taulag_{ij} = 0\text{)}
    \end{equation}
\end{itemize}

The roton minimum in the excitation spectrum at $\Delta_{\text{roton}} \approx 8.6$ K represents the energy cost to create ternary diversity (distinguishable excitations) in the superfluid.

\begin{figure*}[htbp]
\centering
\includegraphics[width=\textwidth]{figures/panel_ternary_computation_1.png}
\caption{\textbf{Ternary Representation for Gas Dynamics: S-Entropy Compression.} 
\textbf{Top Left:} Full phase space representation of 200 molecules in 18-dimensional space (x, y, z positions and velocities for each molecule). Points colored by categorical state show complex high-dimensional structure. 
\textbf{Top Center:} S-entropy compression reduces 18 dimensions to 3 dimensions via transforms $S_k$ (knowledge), $S_t$ (time), $S_e$ (evolution). Each molecule (one point) compressed from 18 dims $\to$ 3 dims over 18 mins $\to$ 3 dims of computation. Color gradient indicates temporal evolution through state space. 
\textbf{Top Right:} Ternary addresses in $3^k$ hierarchy showing trit position (depth, 0--50) versus sample index (0--10). Heatmap displays ternary encoding: 0 = oscillatory perspective (blue), 1 = categorical perspective (yellow/cream), 2 = partition perspective (red). Each column represents one molecule's complete ternary address. 
\textbf{Bottom Left:} Sliding window spectrometer tracking mean S-coordinates across 30 window positions. Three traces show $S_k$ (knowledge, yellow), $S_t$ (time, cyan), and $S_e$ (evolution, red) with characteristic oscillations. Window slides through ensemble capturing local S-entropy statistics. 
\textbf{Bottom Center:} $3^k$ ternary address tree visualizing hierarchical navigation in S-space. Red spheres mark oscillatory states (trit 0), blue spheres mark categorical states (trit 1), green spheres mark partition states (trit 2). Tree depth shown on z-axis (partition), with oscillatory and categorical axes forming base plane. Coverage shown as trajectory through tree. 
\textbf{Bottom Right:} Ternary gas computation framework. Phase space (18D: positions, velocities) maps to S-entropy (3D: $S_k$, $S_t$, $S_e$) yielding ternary addresses [0,1,2,0,2,1,...]. Trit encoding: 0 = oscillatory perspective (phase), 1 = categorical perspective (state), 2 = partition perspective (transition). Key insight: oscillator = processor, computing = solving gas dynamics, memory address = trajectory in S-space.}
\label{fig:ternary_gas}
\end{figure*}

\subsubsection{Bose-Einstein Condensation}

In dilute atomic gas BECs below $T_{\text{BEC}}$:

\begin{itemize}
    \item \textbf{Quantum indistinguishability} collapses the state ternary structure:
    \begin{equation}
    \text{Macroscopic occupation: } N_0 / N \to 1 \quad \text{(all atoms in ground state)}
    \end{equation}
    
    \item \textbf{Ternary uniformity}: All condensed atoms share identical trits:
    \begin{equation}
    \text{trit}_i(j) = \text{trit}_0(j) \quad \forall i \in \text{condensate}, \, j \in \{0,1,2\}
    \end{equation}
    
    \item \textbf{Energy hierarchy collapse}: The $3^k$ energy level hierarchy reduces to $1^k = 1$ (single ground state)
    
    \item \textbf{Enhanced diffusivity}: Coherent transport without scattering:
    \begin{equation}
    D_{\text{BEC}} \gg D_{\text{thermal}} \quad \text{(ballistic transport)}
    \end{equation}
\end{itemize}

The critical temperature $T_{\text{BEC}} = (2\pi\hbar^2/mk_B)(n/\zeta(3/2))^{2/3}$ marks the point where thermal energy equals the ternary collapse energy.

\subsubsection{Unified Understanding}

All three dissipationless states share the same underlying mechanism:

\begin{theorem}[Unified Dissipationless Transport]
\label{thm:unified_dissipationless}
Superconductivity, superfluidity, and Bose-Einstein condensation are manifestations of the same phenomenon: the collapse of the ternary structure through the categorical unification of carriers.
\begin{equation}
\text{Dissipationless transport} \Leftrightarrow \text{Ternary collapse} \Leftrightarrow \text{Partition extinction}
\end{equation}
\end{theorem}

The differences lie only in which carriers undergo unification (electrons, helium atoms, dilute atoms) and which transport coefficient vanishes (resistivity, viscosity, or inverse diffusivity).

\subsection{Ternary Hardware for Transport Measurement}

The categorical instruments (Section~\ref{sec:instruments}) exploit a ternary structure for measurement without physical contact.

\subsubsection{Three-Phase Oscillator Architecture}

\begin{itemize}
    \item \textbf{Three-phase oscillator banks}: Measure all three S-coordinates simultaneously
    \begin{align}
    \text{Phase 1 (} \phi = 0 \text{)} &\to S_k \quad \text{(knowledge entropy)} \\
    \text{Phase 2 (} \phi = 2\pi/3 \text{)} &\to S_t \quad \text{(temporal entropy)} \\
    \text{Phase 3 (} \phi = 4\pi/3 \text{)} &\to S_e \quad \text{(evolution entropy)}
    \end{align}
    
    \item \textbf{Ternary comparators}: Detect partition operations through trit transitions
    \begin{equation}
    \Delta \text{trit} = \text{trit}_{\text{after}} - \text{trit}_{\text{before}} \pmod{3}
    \end{equation}
    
    \item \textbf{$3^k$ hierarchy navigators}: Address specific transport channels using $k$-trit strings
    \begin{equation}
    \text{Channel address} = t_1 t_2 \cdots t_k \in \{0,1,2\}^k
    \end{equation}
\end{itemize}

\subsubsection{Industrial Three-Phase Systems}

Three-phase AC power systems, ubiquitous in industrial settings, provide natural hardware substrates for ternary transport measurement:

\begin{itemize}
    \item \textbf{Voltage phases}: $V_1$, $V_2$, $V_3$ are separated by $2\pi/3$ $\to$ natural ternary clock
    \item \textbf{Current phases}: $I_1$, $I_2$, $I_3$ encode the transport state
    \item \textbf{Power analysis}: Three-phase power $P = V_1 I_1 + V_2 I_2 + V_3 I_3$ measures the total transport
\end{itemize}

This explains the efficiency of three-phase systems: they naturally match the ternary structure of transport phenomena.

\subsection{Computational Complexity: Ternary vs Binary}

\begin{proposition}[Ternary Efficiency for Transport]
\label{prop:ternary_efficiency}
For transport problems in three-dimensional S-space:
\begin{itemize}
    \item \textbf{Binary encoding}: Requires $3 \times \log_2 N$ bits (three separate coordinates)
    \item \textbf{Ternary encoding}: Requires $\log_3 N$ trits (intrinsic 3D encoding)
    \item \textbf{Efficiency gain}: Factor of $3 \log_3 2 \approx 1.89$
\end{itemize}
\end{proposition}

\begin{proof}
To address $N$ states in 3D space:
\begin{itemize}
    \item Binary: Each dimension requires $\log_2 N^{1/3}$ bits, totalling $3 \log_2 N^{1/3} = \log_2 N$ bits per dimension for $\times$ 3 dimensions, resulting in $= 3 \log_2 N$ bits
    \item Ternary: Single coordinate system requires $\log_3 N$ trits
    \item Ratio: $(3 \log_2 N) / (\log_3 N) = 3 \log_3 2 = 3 / \log_2 3 \approx 1.89$
\end{itemize}
\end{proof}

For the phonon chromatograph measuring $3^6 = 729$ channels:
\begin{itemize}
    \item Binary: $3 \times \log_2 729 = 3 \times 9.51 = 28.5$ bits
    \item Ternary: $\log_3 729 = 6$ trits
    \item Efficiency: $28.5 / 6 = 4.75\times$ fewer symbols
\end{itemize}

\subsection{Summary: Ternary Transport}

Ternary representation provides the natural encoding for transport dynamics through the following correspondences:

\begin{enumerate}
    \item \textbf{Triple aspect equivalence}: 
    \begin{equation}
    \{\text{Oscillatory, Categorical, Partition}\} \leftrightarrow \{\text{Trit } 0, 1, 2\}
    \end{equation}
    
    \item \textbf{$3^k$ hierarchical structure}: Transport channels form a ternary tree with $3^k$ leaves at depth $k$
    
    \item \textbf{Partition lag decomposition}: 
    \begin{equation}
    \taulag = \taulag^{(0)} + \taulag^{(1)} + \taulag^{(2)}
    \end{equation}
    
    \item \textbf{Extinction as ternary collapse}: Dissipationless states exhibit a collapsed ternary structure:
    \begin{equation}
    \text{trit}(0) \equiv \text{trit}(1) \equiv \text{trit}(2) \Rightarrow \Tcoeff = 0
    \end{equation}
    
    \item \textbf{S-coordinate mapping}: Transport processes navigate three-dimensional entropy space $\mathbf{S} = (S_k, S_t, S_e)$
    
    \item \textbf{Ternary primitives}: Projection $\pi_i$, completion $\kappa$, and composition $\circ$ replace Boolean logic
    
    \item \textbf{Hardware efficiency}: Three-phase systems naturally implement ternary measurement
\end{enumerate}

The transport coefficient $\Tcoeff$ measures the ternary complexity of carrier dynamics. When this complexity collapses to unity at $T_c$ through phase-locking, all three components of the partition lag vanish simultaneously:
\begin{equation}
\taulag^{(0)} = \taulag^{(1)} = \taulag^{(2)} = 0 \quad \Rightarrow \quad \Tcoeff = 0
\end{equation}

This collapse is discontinuous because the ternary structure is discrete: it either exists (three distinct values, partition possible) or it doesn't (all values identical, partition undefined). There is no intermediate state, explaining the sharp transitions observed in superconductivity, superfluidity, and Bose-Einstein condensation.

The ternary framework thus unifies the description of normal and dissipationless transport, revealing both as manifestations of the same underlying categorical dynamics in three-dimensional S-entropy space.
