\section{Categorical Computing for Transport Analysis}
\label{sec:categorical_computing}

The partition framework established in previous sections reveals a profound identity: transport phenomena are not merely analogous to computational processes but are mathematically identical to computation in bounded categorical phase space. This section develops the computational interpretation of transport, connecting to Poincaré Computing \cite{sachikonye2025poincare} and categorical memory architectures \cite{sachikonye2025memory}, and demonstrates how categorical instruments function as Maxwell demon controllers that characterise transport without physical perturbation.

\subsection{Transport as Computation}

Traditional transport theory treats carrier motion as a physical process governed by Newton's laws (classical) or Schrödinger's equation (quantum). The categorical framework reveals a deeper structure: transport is computation.

\begin{theorem}[Transport-Computation Identity]
\label{thm:transport_computation}
Transport of carriers through a material is mathematically identical to computation in the bounded categorical phase space $\mathcal{S} = [0,1]^3$. The correspondence is:

\begin{center}
\begin{tabular}{lll}
\toprule
\textbf{Transport Domain} & \textbf{Computational Domain} & \textbf{Mathematical Object} \\
\midrule
Material & Phase space $\mathcal{S} = [0,1]^3$ & Bounded manifold \\
Carriers (e$^-$, phonons, atoms) & Categorical states $\mathbf{S}_i$ & Points in $\mathcal{S}$ \\
Scattering events & Constraint satisfaction & Categorical transitions \\
Mean free path $\lambda$ & Trajectory segment length & Arc length in $\mathcal{S}$ \\
Transport coefficient $\Tcoeff$ & Computational complexity $\Pi$ & Dimensionless measure \\
Partition lag $\taulag$ & Categorical completion time & Time scale \\
Partition extinction & Solution recognition & Boundary condition \\
Dissipationless transport & Trivial computation & $\Pi = 1$ \\
\bottomrule
\end{tabular}
\end{center}
\end{theorem}

\begin{proof}[Proof sketch]
Both transport and computation are trajectory dynamics in bounded spaces:
\begin{enumerate}
    \item \textbf{State space}: Material carriers occupy states $\mathbf{S}_i \in \mathcal{S}$; computational problems encode as initial states $\mathbf{S}_0 \in \mathcal{S}$
    
    \item \textbf{Dynamics}: Carrier evolution follows Hamiltonian dynamics in $\mathcal{S}$; computational trajectories follow the same Hamiltonian
    
    \item \textbf{Constraints}: Conservation laws (energy, momentum, particle number) are constraints $\mathcal{C}$ on trajectories; computational problems impose identical constraint structure
    
    \item \textbf{Completion}: Transport from $\mathbf{S}_0$ to $\mathbf{S}_f$ completes when $\|\mathbf{S}(T) - \mathbf{S}_f\| < \epsilon$; computation completes under the same condition
\end{enumerate}
The mathematical structures are identical, not merely isomorphic. Transport \emph{is} computation.
\end{proof}

This identity is not metaphorical but mathematical: both processes are trajectory completion in bounded categorical phase space with identical governing equations. The distinction between ``transport'' and ``computation'' is linguistic, not mathematical.

\begin{figure*}[htbp]
\centering
\includegraphics[width=\textwidth]{figures/precision_by_difference_panel.png}
\caption{\textbf{Precision-by-Difference Network: Temporal Coordination Framework for S-Entropy Navigation.} 
\textbf{(A)} Distribution of precision-by-difference values $\Delta P = T_{\text{ref}} - t_{\text{local}}$ across 100 samples. Gaussian distribution centered at $\mu = -0.02$ $\mu$s with standard deviation $\sigma = 0.98$ $\mu$s. Red dashed line marks reference zero. 
\textbf{(B)} Hierarchy branch selection based on $\Delta P$ sign: Branch 0 ($\Delta P > 0$, 31.3\%), Branch 1 ($\Delta P = 0$, 38.1\%), Branch 2 ($\Delta P < 0$, 30.6\%). Nearly uniform distribution indicates balanced temporal sampling across the $3^k$ hierarchy. 
\textbf{(C)} Temporal coherence windows showing width (blue, left axis, 2.5--5.5 ms range) and quality (purple, right axis, 0.9984--0.9994 range) across 50 window indices. Both metrics exhibit correlated fluctuations, indicating dynamic coherence maintenance. 
\textbf{(D)} Navigation through $3^k$ hierarchy in S-entropy space ($S_k$ vs. $S_e$). Four points colored by depth (yellow $\to$ purple) show 22.0\% coverage of accessible state space. Hierarchical structure enables efficient exploration without exhaustive enumeration. 
\textbf{(E)} Collective state coordination across 35 rounds. Window width (blue bars) fluctuates between 6--14 ms. Synchronization rate remains at 0\% (green dashed line labeled ``No''), indicating asynchronous operation. Red dashed line (``Yes'') would indicate synchronization threshold. 
\textbf{(F)} Categorical completion prediction accuracy. Histogram shows prediction error distribution centered at mean = 0.0078 with narrow spread. Red dashed line marks mean value. Low error validates framework's ability to predict categorical completions. 
\textbf{(G)} Network latency comparison: traditional approach (red, mean 69.7 ms) versus Sango Rine Shumba framework (green, mean 2.8 ms) across 120 requests. Sango achieves 96.1\% latency reduction through precision-by-difference temporal coordination. 
\textbf{(H)} Framework performance radar chart showing five metrics: hierarchy coverage (0.22), synchronization rate (0.0), window quality (0.999), prediction accuracy (high), and latency improvement (0.961). Framework excels in latency and precision while maintaining low synchronization overhead.}
\label{fig:precision_network}
\end{figure*}

\subsection{Poincaré Computing for Transport}

Poincaré Computing \cite{sachikonye2025poincare} provides the natural computational framework for transport analysis. The key insight is that transport satisfies the conditions of the Poincaré recurrence theorem \cite{poincare1890}.

\begin{definition}[Transport as Poincaré Recurrence]
\label{def:transport_poincare}
A transport process from initial state $\mathbf{S}_0$ to final state $\mathbf{S}_f$ is a trajectory $\gamma: [0,T] \to \mathcal{S}$ satisfying:

\begin{enumerate}
    \item \textbf{Boundary conditions}: 
    \begin{equation}
    \gamma(0) = \mathbf{S}_0, \quad \|\gamma(T) - \mathbf{S}_f\| < \epsilon
    \end{equation}
    where $\epsilon$ is the categorical resolution (related to thermal energy $k_B T$)
    
    \item \textbf{Conservation laws}: Energy $E$, momentum $\mathbf{p}$, particle number $N$ as constraints:
    \begin{equation}
    \mathcal{C}(\gamma) = \{E[\gamma] = E_0, \, \mathbf{p}[\gamma] = \mathbf{p}_0, \, N[\gamma] = N_0\}
    \end{equation}
    
    \item \textbf{Measure preservation}: The trajectory preserves the categorical measure:
    \begin{equation}
    \mu(\gamma([0,T])) = \mu(\mathcal{S}_0)
    \end{equation}
    where $\mu$ is the Liouville measure on $\mathcal{S}$
    
    \item \textbf{Recurrence}: For almost all initial conditions, the trajectory returns arbitrarily close to $\mathbf{S}_0$:
    \begin{equation}
    \forall \delta > 0, \, \exists T_{\text{rec}} : \|\gamma(T_{\text{rec}}) - \mathbf{S}_0\| < \delta
    \end{equation}
\end{enumerate}
\end{definition}

The transport coefficient measures the computational complexity—the number of categorical completions required for the trajectory to satisfy all constraints:

\begin{equation}
\label{eq:transport_complexity}
\Tcoeff \propto \Pi(\text{transport problem})
\end{equation}

where $\Pi$ is the Poincaré complexity, defined as the number of categorical state transitions from $\mathbf{S}_0$ to solution recognition at the $\epsilon$-boundary.

\begin{example}[Electrical Resistivity as Computational Complexity]
For electrical transport in a metal:
\begin{itemize}
    \item \textbf{Problem}: Move an electron from position $x_0$ to $x_f$ while conserving energy and momentum
    \item \textbf{Complexity}: $\Pi \sim L/\lambda$ where $L = |x_f - x_0|$ and $\lambda$ is mean free path
    \item \textbf{Resistivity}: $\rho \propto \Pi$ (more scattering events $\Rightarrow$ higher resistance)
    \item \textbf{Superconducting limit}: $\lambda \to \infty \Rightarrow \Pi \to 1 \Rightarrow \rho = 0$
\end{itemize}
\end{example}

\subsection{Non-Halting Transport Dynamics}

A crucial feature of Poincaré Computing is non-halting dynamics: the system never reaches a static final state but continuously explores phase space. Transport systems exhibit identical behavior.

\begin{proposition}[Continuous Transport Dynamics]
\label{prop:nonhalting_transport}
Transport processes never truly ``halt'' in the computational sense:

\begin{enumerate}
    \item \textbf{Thermal fluctuations}: Continuous perturbations with energy scale $k_B T$ prevent static equilibrium:
    \begin{equation}
    \langle \delta \mathbf{S}^2 \rangle \sim k_B T
    \end{equation}
    
    \item \textbf{Trajectory initiation}: New transport trajectories are continuously initiated at rate $\Gamma \sim \tau_{\text{thermal}}^{-1}$
    
    \item \textbf{Dynamic equilibrium}: Equilibrium is not static but satisfies detailed balance:
    \begin{equation}
    \Gamma_{i \to j} = \Gamma_{j \to i} \quad \text{(forward rate = reverse rate)}
    \end{equation}
    
    \item \textbf{Emergent memory}: The current state $\mathbf{S}(t)$ encodes the complete trajectory history $\gamma([0,t])$ through exploration
\end{enumerate}
\end{proposition}

\begin{proof}
At finite temperature $T > 0$:
\begin{enumerate}
    \item Thermal energy $k_B T$ provides a continuous driving force for state transitions
    \item The partition function $Z = \sum_i e^{-E_i/k_B T}$ has multiple accessible states ($Z > 1$)
    \item Ergodicity ensures the exploration of all accessible states over time
    \item Memory emerges because the probability distribution $P(\mathbf{S}, t)$ evolves according to the master equation, which depends on the full history
\end{enumerate}
Only at $T = 0$ does the system halt (ground state occupation).
\end{proof}

This explains why transport is inherently time-dependent even at ``steady state'': the steady state is a dynamic balance of continuous microscopic motion, not static rest.

\subsubsection{Emergent Memory in Materials}

Memory emerges from exploration history: the current state of the material encodes the trajectory history of all carriers. This is the physical manifestation of emergent memory in Poincaré Computing \cite{sachikonye2025poincare}.

\begin{theorem}[Material Memory Emergence]
\label{thm:material_memory}
A material's current state $\mathbf{S}_{\text{material}}(t)$ contains information about its complete thermal history $\{\mathbf{S}(\tau) : 0 \leq \tau \leq t\}$ through:
\begin{equation}
I[\mathbf{S}_{\text{material}}(t) : \text{history}] = S_{\text{total}} - S_{\text{current}} \geq 0
\end{equation}
where $I$ is mutual information and $S$ is entropy.
\end{theorem}

The memory is distributed across multiple timescales:
\begin{itemize}
    \item \textbf{Femtosecond}: Electronic excitations (working memory)
    \item \textbf{Picosecond}: Phonon modes (cache memory)
    \item \textbf{Nanosecond} to millisecond: Defect dynamics (RAM)
    \item \textbf{Seconds-years}: Structural defects (long-term storage)
\end{itemize}

\subsection{The $\epsilon$-Boundary in Transport}

In Poincaré Computing, solutions are recognised at the $\epsilon$-boundary rather than computed to exact completion. Transport exhibits the same structure.

\begin{theorem}[Transport Solution Recognition]
\label{thm:transport_epsilon}
A transport process is ``complete'' when the trajectory reaches within $\epsilon$ of the target state:
\begin{equation}
\|\mathbf{S}(T) - \mathbf{S}_{\text{target}}\| < \epsilon
\end{equation}
where the categorical resolution $\epsilon$ is temperature-dependent:
\begin{equation}
\epsilon(T) \sim \sqrt{k_B T / E_{\text{char}}}
\end{equation}
with $E_{\text{char}}$ the characteristic energy scale of the transport process.
\end{theorem}

\begin{proof}
Thermal fluctuations create uncertainty in state determination:
\begin{enumerate}
    \item Energy uncertainty: $\Delta E \sim k_B T$ (thermal energy scale)
    \item Position uncertainty: $\Delta x \sim \sqrt{k_B T / k_{\text{spring}}}$ (thermal length scale)
    \item Momentum uncertainty: $\Delta p \sim \sqrt{m k_B T}$ (thermal momentum scale)
\end{enumerate}
These combine to give categorical resolution $\epsilon \sim \sqrt{k_B T / E_{\text{char}}}$.
\end{proof}

\begin{corollary}[Temperature Dependence of Transport]
The temperature dependence of transport coefficients arises from $\epsilon(T)$:
\begin{itemize}
    \item \textbf{High $T$}: Large $\epsilon$ (coarse resolution) $\Rightarrow$ fewer categorical completions needed $\Rightarrow$ efficient transport
    \item \textbf{Low $T$}: Small $\epsilon$ (fine resolution) $\Rightarrow$ more categorical completions needed $\Rightarrow$ reduced transport
    \item \textbf{$T \to T_c$}: $\epsilon$ becomes irrelevant as partition extinction occurs
\end{itemize}
\end{corollary}

This explains the general trend of increasing resistivity at low temperatures (before superconductivity): finer categorical resolution requires more computational steps.

\subsection{Identity Unification in Transport}

The identity unification theorem from categorical computing \cite{sachikonye2025memory} applies directly to transport: a carrier's state simultaneously encodes multiple aspects that are traditionally treated as independent.

\begin{proposition}[Carrier Identity Unification]
\label{prop:carrier_identity}
A carrier's categorical state $\mathbf{S}_i = (S_k, S_t, S_e)$ simultaneously encodes:

\begin{enumerate}
    \item \textbf{Position} (memory address): Location in real space $\mathbf{r}_i$
    \begin{equation}
    \mathbf{r}_i = \pi_{\text{position}}(\mathbf{S}_i)
    \end{equation}
    
    \item \textbf{Momentum} (processor state): Velocity/energy state $\mathbf{p}_i$
    \begin{equation}
    \mathbf{p}_i = \pi_{\text{momentum}}(\mathbf{S}_i)
    \end{equation}
    
    \item \textbf{Type} (semantic content): Species/mode identity $\alpha_i$
    \begin{equation}
    \alpha_i = \pi_{\text{type}}(\mathbf{S}_i)
    \end{equation}
\end{enumerate}

These are not independent properties but projections of the same categorical state:
\begin{equation}
\mathbf{S}_i = \text{span}\{\mathbf{r}_i, \mathbf{p}_i, \alpha_i\}
\end{equation}
\end{proposition}

\begin{proof}
The S-entropy coordinates encode:
\begin{itemize}
    \item $S_k$ (knowledge entropy): Uncertainty in which state the carrier occupies $\Rightarrow$ position information
    \item $S_t$ (temporal entropy): Uncertainty in timing $\Rightarrow$ momentum/energy information
    \item $S_e$ (evolution entropy): Uncertainty in trajectory $\Rightarrow$ type/mode information
\end{itemize}
Position, momentum, and type are extracted via projection operators $\pi_i$ acting on $\mathbf{S}$.
\end{proof}

\begin{corollary}[Coupled Transport]
This explains why transport couples position, momentum, and species: they are not independent degrees of freedom but three views of one categorical entity. Changing position necessarily changes momentum and may change type (e.g., electron-hole recombination).
\end{corollary}

This unification resolves the apparent paradox in thermoelectric effects: how can charge transport (position) couple to heat transport (energy) and chemical potential (type)? Answer: they are projections of the same categorical state, so coupling is automatic.

\subsection{Categorical Memory in Materials}

Materials function as categorical memory systems in the sense of \cite{sachikonye2025memory}, with information stored across multiple hierarchical levels.

\begin{theorem}[Material as Categorical Memory]
\label{thm:material_memory_system}
A material stores information through hierarchical categorical structures:

\begin{enumerate}
    \item \textbf{Defect structure} (long-term memory):
    \begin{itemize}
        \item Vacancies, interstitials, dislocations
        \item Frozen trajectory endpoints from non-equilibrium processing
        \item Retention time: $\tau_{\text{defect}} \sim \tau_0 e^{E_{\text{migration}}/k_B T}$ (years at room temperature)
    \end{itemize}
    
    \item \textbf{Phonon modes} (short-term memory):
    \begin{itemize}
        \item Lattice vibrations encoding recent thermal history
        \item Coherent oscillations with lifetime $\tau_{\text{phonon}} \sim 10^{-12}$ s
        \item Analogous to cache memory in computers
    \end{itemize}
    
    \item \textbf{Electronic structure} (working memory):
    \begin{itemize}
        \item Band occupation, charge distribution
        \item Responds to applied fields on timescale $\tau_{\text{electronic}} \sim 10^{-15}$ s
        \item Analogous to CPU registers
    \end{itemize}
\end{enumerate}

The thermal history IS the memory content:
\begin{equation}
\text{Memory}[\text{material}] = \int_0^t \mathbf{S}(\tau) \, w(t - \tau) \, d\tau
\end{equation}
where $w(t)$ is a weighting function that decays over the relevant timescale.
\end{theorem}

\subsubsection{Thermal Processing as Memory Operations}

Common thermal treatments are memory operations in the categorical sense:

\begin{enumerate}
    \item \textbf{Annealing} (memory consolidation):
    \begin{itemize}
        \item Slow cooling allows system to explore phase space and find low-energy configurations
        \item Analogous to memory consolidation during sleep
        \item Result: Defects migrate to grain boundaries, stress relief
    \end{itemize}
    
    \item \textbf{Quenching} (memory write):
    \begin{itemize}
        \item Rapid cooling freezes non-equilibrium state
        \item Analogous to writing data to non-volatile memory
        \item Result: Metastable phases, retained high-temperature structure
    \end{itemize}
    
    \item \textbf{Thermal cycling} (memory training):
    \begin{itemize}
        \item Repeated heating/cooling creates preferred pathways
        \item Analogous to training neural networks
        \item Result: Shape memory alloys, training effects in phase transitions
    \end{itemize}
    
    \item \textbf{Aging} (memory degradation):
    \begin{itemize}
        \item Slow structural relaxation at constant temperature
        \item Analogous to bit rot in digital memory
        \item Result: Creep, stress relaxation, property drift
    \end{itemize}
\end{enumerate}

\subsection{Maxwell Demon Controllers}

The categorical instruments (Section~\ref{sec:instruments}) operate as Maxwell demon controllers---devices that observe and predict carrier behavior without disturbing the transport process. We use the term ``Maxwell demon'' by convention, though the categorical framework resolves the apparent thermodynamic paradox \cite{sachikonye2025memory}.

\begin{definition}[Transport Maxwell Demon]
\label{def:maxwell_demon}
A transport Maxwell demon is a categorical measurement device that:

\begin{enumerate}
    \item \textbf{Observes} carrier categorical states $\mathbf{S}_i$ without measuring position or momentum (zero backaction)
    
    \item \textbf{Predicts} trajectory endpoints from partial trajectory data:
    \begin{equation}
    \hat{\mathbf{S}}_f = \kappa(\mathbf{S}_0, \gamma([0,t])) \quad \text{(trajectory completion)}
    \end{equation}
    
    \item \textbf{Identifies} partition extinction conditions:
    \begin{equation}
    T_c = \arg\min_T \{\taulag(T) = 0\}
    \end{equation}
    
    \item \textbf{Measures} partition lag spectra:
    \begin{equation}
    P(\taulag) = \text{FFT}[\delta t(\omega)]
    \end{equation}
\end{enumerate}

without disturbing the physical transport, because categorical observables $\hat{O}_{\text{cat}}$ commute with physical observables $\hat{O}_{\text{phys}}$:
\begin{equation}
[\hat{O}_{\text{cat}}, \hat{O}_{\text{phys}}] = 0
\end{equation}
\end{definition}

\begin{theorem}[Zero Backaction Measurement]
\label{thm:zero_backaction}
Categorical measurements of transport properties have zero backaction on the transport process:
\begin{equation}
\frac{d\langle \hat{O}_{\text{phys}} \rangle}{dt}\bigg|_{\text{measurement}} = \frac{d\langle \hat{O}_{\text{phys}} \rangle}{dt}\bigg|_{\text{no measurement}}
\end{equation}
\end{theorem}

\begin{proof}
Categorical observables are functions of S-entropy coordinates $\mathbf{S} = (S_k, S_t, S_e)$, which are orthogonal to position-momentum phase space coordinates $(\mathbf{r}, \mathbf{p})$:
\begin{equation}
\frac{\partial \mathbf{S}}{\partial \mathbf{r}} = 0, \quad \frac{\partial \mathbf{S}}{\partial \mathbf{p}} = 0
\end{equation}
Therefore, measuring $\mathbf{S}$ does not disturb $(\mathbf{r}, \mathbf{p})$, and physical transport (which depends on $(\mathbf{r}, \mathbf{p})$) is unaffected.
\end{proof}

This enables non-invasive transport characterization: the categorical instruments measure transport properties (resistivity, viscosity, thermal conductivity) without applying voltage, shear, or temperature gradients.

\subsubsection{Resolution of the Maxwell Demon Paradox}

The apparent violation of the second law by Maxwell demons is resolved in the categorical framework:

\begin{proposition}[No Entropy Reduction]
A Maxwell demon controller does not reduce total entropy:
\begin{equation}
\Delta S_{\text{total}} = \Delta S_{\text{system}} + \Delta S_{\text{demon}} \geq 0
\end{equation}
because the information gained by the demon ($\Delta I$) is stored with entropy cost:
\begin{equation}
\Delta S_{\text{demon}} \geq k_B \ln 2 \cdot \Delta I
\end{equation}
(Landauer's principle \cite{landauer1961}).
\end{proposition}

In categorical measurement, the ``demon'' is the hardware oscillator network, which stores information in its timing state. The entropy cost is paid by the oscillators, not extracted from the measured system.

\subsection{Complexity Analysis of Transport}

Transport complexity follows the Poincaré complexity scaling derived in \cite{sachikonye2025poincare}.

\begin{proposition}[Transport Complexity Scaling]
\label{prop:transport_complexity}
The Poincaré complexity of a transport process scales as:
\begin{equation}
\Pi(\text{transport}) \sim \frac{L}{\lambda} \times \frac{1}{\epsilon}
\end{equation}
where:
\begin{itemize}
    \item $L$: Transport length (distance from $\mathbf{S}_0$ to $\mathbf{S}_f$)
    \item $\lambda$: Mean free path (average distance between scattering events)
    \item $\epsilon$: Categorical resolution (related to $k_B T$)
\end{itemize}
\end{proposition}

\begin{proof}
The number of categorical completions is:
\begin{enumerate}
    \item \textbf{Scattering events}: $N_{\text{scatter}} \sim L/\lambda$ (number of collisions)
    \item \textbf{Resolution factor}: $1/\epsilon$ (finer resolution requires more steps)
    \item \textbf{Total complexity}: $\Pi = N_{\text{scatter}} \times (1/\epsilon) \sim (L/\lambda) \times (1/\epsilon)$
\end{enumerate}
\end{proof}

\begin{corollary}[Transport Regimes]
The complexity formula predicts three transport regimes:

\begin{enumerate}
    \item \textbf{Ballistic transport} ($L \ll \lambda$):
    \begin{equation}
    \Pi \sim L/\lambda \ll 1 \Rightarrow \text{low resistance}
    \end{equation}
    
    \item \textbf{Diffusive transport} ($L \gg \lambda$):
    \begin{equation}
    \Pi \sim L/\lambda \gg 1 \Rightarrow \text{high resistance}
    \end{equation}
    
    \item \textbf{Dissipationless transport} ($\lambda \to \infty$):
    \begin{equation}
    \Pi \to 1 \Rightarrow \Tcoeff = 0
    \end{equation}
\end{enumerate}
\end{corollary}

The transition to dissipationless transport at partition extinction corresponds to $\lambda \to \infty$: when carriers cannot be distinguished (partitioning is impossible), scattering cannot occur, and the mean free path diverges.

\subsection{Dissipationless States as Trivial Computation}

Partition extinction implies trivial computational complexity, providing a computational interpretation of dissipationless transport.

\begin{theorem}[Dissipationless = Trivial Complexity]
\label{thm:trivial_complexity}
Partition extinction implies trivial transport complexity:
\begin{equation}
\taulag \to 0 \Rightarrow \lambda \to \infty \Rightarrow \Pi(\text{transport}) \to 1
\end{equation}
Transport becomes a single categorical completion with no intermediate scattering events.
\end{theorem}

\begin{proof}
When partition lag vanishes at $T < T_c$:
\begin{enumerate}
    \item \textbf{No scattering}: Partition operations between carriers are undefined (carriers are indistinguishable)
    \item \textbf{Infinite mean free path}: $\lambda \to \infty$ (no collisions)
    \item \textbf{Complexity collapse}: $\Pi \sim L/\lambda \to 0$ as $\lambda \to \infty$
    \item \textbf{Single completion}: Only the final state transition occurs (initial $\to$ final with no intermediate steps)
\end{enumerate}
Therefore, $\Pi \to 1$ (one categorical completion).
\end{proof}

\begin{interpretation}
This explains why superconductors carry current without resistance: the computation (trajectory from initial to final state) is trivially simple---a single step with no intermediate processing. The transport coefficient measures computational complexity:
\begin{equation}
\rho \propto \Pi \Rightarrow \rho = 0 \text{ when } \Pi = 1
\end{equation}
\end{interpretation}

\begin{corollary}[Energy Gap as Computational Barrier]
The energy gap in superconductors ($\Delta_{\text{BCS}}$) and superfluids ($\Delta_{\text{roton}}$) represents the energy cost to increase computational complexity from $\Pi = 1$ to $\Pi > 1$:
\begin{equation}
\Delta = k_B T_c = \text{energy to create scattering event}
\end{equation}
Below $T_c$, thermal energy $k_B T < \Delta$ is insufficient to create scattering, so $\Pi$ remains at 1.
\end{corollary}

\subsection{Virtual Gas Ensemble in Materials}

The carrier ensemble in a material is mathematically identical to the virtual gas ensemble developed in the companion Poincaré Computing paper \cite{sachikonye2025poincare}.

\begin{proposition}[Carrier Gas Identity]
\label{prop:carrier_gas}
The correspondence between virtual gas and material carriers is:

\begin{center}
\begin{tabular}{lll}
\toprule
\textbf{Virtual Gas} & \textbf{Material Carriers} & \textbf{Mathematical Object} \\
\midrule
Container & Material boundaries & Bounded domain $\Omega$ \\
Molecules & Electrons/phonons/atoms & Particles with mass $m_i$ \\
Temperature $T$ & Kinetic energy distribution & $\langle E_{\text{kin}} \rangle = \frac{3}{2} k_B T$ \\
Pressure $P$ & Carrier density $n$ & $P = n k_B T$ (ideal gas) \\
Collisions & Scattering events & Interaction events \\
Mean free path $\lambda$ & Transport length scale & $\lambda = 1/(n \sigma)$ \\
Velocity distribution & Fermi-Dirac/Bose-Einstein & $f(E) = [e^{(E-\mu)/k_B T} \pm 1]^{-1}$ \\
\bottomrule
\end{tabular}
\end{center}
\end{proposition}

The ideal gas laws derived in \cite{sachikonye2025poincare} apply directly to material carriers:

\begin{enumerate}
    \item \textbf{Electron gas} (Fermi-Dirac statistics):
    \begin{itemize}
        \item Drude model: $\rho = m/(ne^2 \tau)$
        \item Fermi energy: $E_F = \hbar^2 (3\pi^2 n)^{2/3} / (2m)$
        \item Heat capacity: $C_V = (\pi^2/3) n k_B^2 T / E_F$
    \end{itemize}
    
    \item \textbf{Phonon gas} (Bose-Einstein statistics):
    \begin{itemize}
        \item Debye model: $C_V = 9 N k_B (T/\Theta_D)^3 \int_0^{\Theta_D/T} \frac{x^4 e^x}{(e^x - 1)^2} dx$
        \item Thermal conductivity: $\kappa = (1/3) C_V v_s \lambda_{\text{phonon}}$
        \item Umklapp scattering: $\lambda_{\text{phonon}}^{-1} \propto T$ at high $T$
    \end{itemize}
    
    \item \textbf{Magnon gas} (spin wave excitations):
    \begin{itemize}
        \item Bloch $T^{3/2}$ law: $M(T) = M(0) [1 - (T/T_C)^{3/2}]$
        \item Spin Seebeck effect: a thermal gradient drives a spin current
        \item Magnon thermal conductivity: $\kappa_{\text{magnon}}$
    \end{itemize}
\end{enumerate}

\begin{figure*}[htbp]
\centering
\includegraphics[width=\textwidth]{figures/panel_ternary_computation_2.png}
\caption{\textbf{Ternary Computation as Gas Dynamics: Oscillator = Processor Identity.} 
\textbf{Top Left:} Ternary computation trajectories in S-space showing individual molecular paths (each line = 1 molecule). Yellow trajectory highlighted shows evolution from origin through S-space coordinates ($S_k$, $S_t$, $S_e$). Multiple molecules (green lines) explore different regions of accessible state space. 
\textbf{Top Center:} Ensemble equilibration showing computation converging to thermalization. Three S-coordinate traces ($S_k$ categorical, $S_t$ oscillatory, $S_e$ partition) evolve over 140 computation steps. Initial transient (0--20 steps) followed by equilibration to steady-state fluctuations around mean values ($S_k \approx 0.25$, $S_t \approx 0.00$, $S_e \approx 0.28$). Shaded regions indicate variance envelopes. 
\textbf{Top Right:} Ternary operations in S-space showing three fundamental operations as vectors: Op 0 (oscillate, blue), Op 1 (categorize, green), Op 2 (partition, red). Operations span the S-space, enabling complete navigation through categorical states. 
\textbf{Bottom Left:} Thermodynamics from ternary computation. Temperature $T$ (red) and pressure $P$ (purple) evolution over 140 computation steps. Initial spike ($T \approx 280$ K, $P \approx 0.75$ bar) during initialization, followed by equilibration to steady state ($T \approx 260$ K, $P \approx 0.50$ bar). Demonstrates emergence of thermodynamic variables from purely computational operations. 
\textbf{Bottom Center:} Trit state evolution heatmap for single molecule (12 trits) across 100+ computation steps. Color encoding: 0 = oscillatory (blue), 1 = categorical (yellow/cream), 2 = partition (red). Vertical bands show state persistence; transitions indicate computational operations. Pattern demonstrates ergodic exploration of trit configurations. 
\textbf{Bottom Right:} Computation-thermodynamics identity table establishing correspondence. Ternary operations (trit increment) map to thermodynamic processes (phase oscillation, category transition, partition rearrangement). Computational state (12-trit register, S-entropy coordinates, random walk) is identical to gas state (molecular microstate, phase space coordinates, thermal fluctuations). Computation complete = equilibrium (Poincaré recurrence = Maxwell distribution). Core identity: oscillator = processor, memory address = trajectory in S-space, solving gas dynamics = running ternary program.}
\label{fig:ternary_computation}
\end{figure*}

\subsection{Ternary Encoding of Transport Computation}

Connecting to Section~\ref{sec:ternary}, transport computation naturally encodes in ternary:

\begin{proposition}[Ternary Transport Computation]
Each transport event is a ternary operation:
\begin{itemize}
    \item \textbf{Trit 0}: Oscillatory propagation (coherent transport)
    \item \textbf{Trit 1}: Categorical transition (scattering event)
    \item \textbf{Trit 2}: Partition operation (channel selection)
\end{itemize}
A complete transport trajectory is a $k$-trit string encoding $3^k$ possible pathways.
\end{proposition}

The computational complexity in ternary form:
\begin{equation}
\Pi = \sum_{j=1}^k w_j \cdot \text{trit}_j
\end{equation}
where $w_j$ are weights depending on the transport regime.

\subsection{Summary: Transport as Computation}

Categorical computing provides a unified framework for understanding transport phenomena:

\begin{enumerate}
    \item \textbf{Transport IS computation}: Not analogy but mathematical identity---trajectory completion in bounded categorical phase space $\mathcal{S} = [0,1]^3$
    
    \item \textbf{Poincaré recurrence}: Transport satisfies recurrence conditions; solutions are recognised at $\epsilon$-boundary
    
    \item \textbf{Non-halting dynamics}: Continuous thermal exploration; equilibrium is dynamic balance, not static rest
    
    \item \textbf{Temperature-dependent resolution}: $\epsilon(T) \sim \sqrt{k_B T / E_{\text{char}}}$ sets categorical resolution; it explains low-$T$ transport reduction
    
    \item \textbf{Identity unification}: Position, momentum, and type are projections of a single categorical state $\mathbf{S}$; it explains coupled transport
    
    \item \textbf{Material memory}: Thermal history stored across hierarchical timescales (defects, phonons, electrons); thermal processing = memory operations
    
    \item \textbf{Maxwell demon instruments}: Zero-backaction measurement via categorical observables orthogonal to $(\mathbf{r}, \mathbf{p})$; resolves thermodynamic paradox
    
    \item \textbf{Complexity scaling}: $\Pi \sim (L/\lambda) \times (1/\epsilon)$; predicts ballistic, diffusive, and dissipationless regimes
    
    \item \textbf{Trivial complexity}: Partition extinction $\Rightarrow$ $\Pi = 1$ $\Rightarrow$ $\Tcoeff = 0$; dissipationless transport is single-step computation
    
    \item \textbf{Virtual gas identity}: Material carriers = virtual gas molecules; ideal gas laws apply with quantum statistics
    
    \item \textbf{Ternary encoding}: Transport naturally encodes as ternary strings; $3^k$ pathways at depth $k$
\end{enumerate}

The transport coefficient $\Tcoeff$ is the computational complexity of moving carriers through a material. Dissipationless transport is computation with complexity $\Pi = 1$---a single categorical step with no intermediate processing. The energy gap $\Delta$ is the computational barrier: the energy cost to increase complexity from $\Pi = 1$ to $\Pi > 1$.

This computational interpretation unifies normal and dissipationless transport, revealing both as manifestations of the same underlying categorical dynamics. Transport is not a physical process that can be computed; transport \emph{is} computation.
