%==============================================================================
\section{Phase Transitions as Partition Extinction}
\label{sec:phase_transitions}
%==============================================================================

\subsection{Two Partition Structures in Condensed Matter}

Condensed matter systems possess two independent categorical partition structures that determine their phase and transport properties:

\begin{definition}[Particle Identity Partition]
\label{def:particle_identity}
The \emph{particle identity partition} assigns a categorical identity to each particle: Atom$_1$, Atom$_2$, Atom$_3$, etc. This partition distinguishes individual particles from one another, allowing them to be tracked and counted separately.
\end{definition}

\begin{definition}[Site Assignment Partition]
\label{def:site_assignment}
The \emph{site assignment partition} maps each particle to a spatial location: Atom$_i \to$ Site$_j$. In crystalline solids, this partition assigns atoms to specific lattice sites with well-defined equilibrium positions.
\end{definition}

These partitions are logically independent. A system may have:
\begin{itemize}
\item \textbf{Both partitions intact:} Crystalline solid (atoms assigned to sites, atoms distinguishable)
\item \textbf{Site partition extinct, identity partition intact:} Liquid (atoms not assigned to sites, but atoms still distinguishable)
\item \textbf{Identity partition extinct, site partition intact:} Not observed (would require indistinguishable atoms locked to lattice sites)
\item \textbf{Identity partition extinct, site partition extinct:} Superfluid (atoms neither assigned to sites nor distinguishable)
\item \textbf{Both partitions extinct:} Singular state (not physically realizable at $T > 0$)
\end{itemize}

The phase diagram of matter can be understood as the space of these two partition structures. Phase transitions correspond to the extinction or formation of one or both partitions.

\subsection{Phonon Propagation as Amplitude Transfer}

In a crystalline solid, atoms oscillate about equilibrium lattice positions. Heat propagates through collective lattice vibrations—phonons—rather than through atomic transport \citep{kittel2005}. This is the thermal analogue of Newton's cradle mechanism for electrical conduction.

\begin{proposition}[Phonon Newton's Cradle]
\label{prop:phonon_cradle}
Phonon heat conduction is the thermal analogue of the Newton's cradle mechanism in electrical conduction. Energy transfers through sequential atomic displacement without net atomic transport.
\end{proposition}

\begin{proof}
Consider a one-dimensional chain of atoms with equilibrium positions $\{x_n^{(0)}\}$ and displacements $\{u_n(t)\}$ from equilibrium. The equation of motion for the $n$-th atom is:
\begin{equation}
m\ddot{u}_n = K(u_{n+1} - 2u_n + u_{n-1}),
\label{eq:chain_eom}
\end{equation}
where $m$ is atomic mass and $K$ is the spring constant (force constant) between nearest neighbors.

The dispersion relation for this chain is:
\begin{equation}
\omega(k) = 2\sqrt{\frac{K}{m}} \left|\sin\left(\frac{ka}{2}\right)\right|,
\label{eq:dispersion}
\end{equation}
where $a$ is the lattice spacing and $k$ is the wavevector. For long-wavelength modes ($ka \ll 1$), this reduces to:
\begin{equation}
\omega(k) \approx v_s |k|, \quad v_s = a\sqrt{\frac{K}{m}},
\label{eq:sound_velocity}
\end{equation}
where $v_s$ is the sound velocity (phonon group velocity).

Energy propagates at a velocity of $v_s \sim 5000$ m/s in typical solids. Individual atoms oscillate with thermal velocities $v_{\text{atom}} \sim \sqrt{k_B T/m} \sim 500$ m/s at room temperature. The energy propagation is ten times faster than atomic motion—confirming the collective (Newton's cradle) nature of phonon transport.

When atom $n$ is displaced, it pushes atom $n+1$, which pushes atom $n+2$, and so on. The displacement propagates as a wave at velocity $v_s$, but individual atoms do not travel with the wave. They oscillate locally about their equilibrium positions. This is precisely the Newton's cradle mechanism: energy transfer through sequential local interactions, not through carrier transport. \qed
\end{proof}

\subsection{The Amplitude-Spacing Ratio}

Each atom oscillates with an amplitude $u$ about its equilibrium position. The ratio of oscillation amplitude to lattice spacing determines the categorical sharpness of site assignment:

\begin{equation}
\eta = \frac{\langle u^2 \rangle^{1/2}}{a},
\label{eq:lindemann_ratio}
\end{equation}
where $\langle u^2 \rangle^{1/2}$ is the root-mean-square (RMS) displacement, and $a$ is the nearest-neighbour lattice spacing.

The ratio $\eta$ has a clear categorical interpretation:

\begin{itemize}
\item \textbf{For $\eta \ll 1$:} Atoms remain close to their assigned sites. The site assignment partition is sharp. An atom at Site$_j$ is unambiguously closer to Site$_j$ than to any neighbouring site.

\item \textbf{For $\eta \sim 1$:} Atoms approach neighbouring sites. The site assignment becomes ambiguous. An atom nominally at Site$_j$ may instantaneously be closer to Site$_{j+1}$.

\item \textbf{For $\eta \gg 1$:} Atoms wander far from assigned sites. The site assignment partition is extinct. There is no meaningful sense in which atoms are ``at'' specific sites.
\end{itemize}

\subsection{Site Assignment Ambiguity}

When the oscillation amplitude $u$ approaches the lattice spacing $a$, a categorical ambiguity arises: to which site does the atom belong?

\begin{proposition}[Partition Ambiguity Condition]
\label{prop:ambiguity}
An atom with instantaneous displacement $u$ from its assigned Site$_j$ is closer to the neighbouring Site$_{j+1}$ when:
\begin{equation}
|u| > \frac{a}{2}.
\label{eq:ambiguity_condition}
\end{equation}
At this point, the site assignment partition is undefined for that atom.
\end{proposition}

\begin{proof}
Consider an atom assigned to Site$_j$ at position $x_j^{(0)}$. The neighbouring site Site$_{j+1}$ is at position $x_{j+1}^{(0)} = x_j^{(0)} + a$. If the atom is displaced by $u$ from Site$_j$, its instantaneous position is $x_j^{(0)} + u$.

The distance to Site$_j$ is $|u|$. The distance to Site$_{j+1}$ is $|a - u|$. The atom is closer to Site$_{j+1}$ when:
\begin{equation}
|a - u| < |u|.
\end{equation}

For $u > 0$ (displacement toward Site$_{j+1}$), this gives $a - u < u$, or $u > a/2$. For $u < 0$ (displacement toward Site$_{j-1}$), the condition is $|u| > a/2$. \qed
\end{proof}

The mean-square displacement in a harmonic solid is given by quantum statistical mechanics \citep{ashcroft1976}:
\begin{equation}
\langle u^2 \rangle = \frac{3\hbar}{2m\omega_D} \coth\left(\frac{\hbar\omega_D}{2k_B T}\right),
\label{eq:msd_quantum}
\end{equation}
where $\omega_D$ is the Debye frequency (characteristic frequency of lattice vibrations).

At high temperatures ($k_B T \gg \hbar\omega_D$), the hyperbolic cotangent simplifies:
\begin{equation}
\coth\left(\frac{\hbar\omega_D}{2k_B T}\right) \approx \frac{2k_B T}{\hbar\omega_D},
\end{equation}
giving the classical result:
\begin{equation}
\langle u^2 \rangle \approx \frac{3k_B T}{m\omega_D^2} = \frac{3k_B T}{K},
\label{eq:msd_classical}
\end{equation}
where $K = m\omega_D^2$ is the effective spring constant.

The amplitude grows linearly with temperature until it exceeds the threshold for site assignment ambiguity.

\subsection{The Lindemann Melting Criterion}

Lindemann proposed in 1910 that melting occurs when the RMS displacement reaches a critical fraction of the lattice spacing \citep{lindemann1910}:

\begin{equation}
\eta_c = \frac{\langle u^2 \rangle_m^{1/2}}{a} \approx 0.1 \text{--} 0.2,
\label{eq:lindemann_criterion}
\end{equation}
where the subscript $m$ denotes evaluation at the melting temperature $T_m$.

This empirical criterion successfully predicts melting temperatures for a wide range of materials \citep{gilvarry1956}. The critical value $\eta_c \approx 0.15$ is typical for most elements and simple compounds.

\begin{theorem}[Melting as Site Partition Extinction]
\label{thm:melting_partition}
The Lindemann criterion is the condition for the extinction of the site assignment partition. When $\eta > \eta_c$, atoms cannot be categorically assigned to specific lattice sites, and the solid melts.
\end{theorem}

\begin{proof}
The site assignment partition requires each atom to be closer to its assigned site than to any other site. For a simple cubic lattice with nearest-neighbour spacing $a$, this requires $|u| < a/2$ for all atoms.

For a Gaussian distribution of displacements with an RMS value $\sigma = \langle u^2 \rangle^{1/2}$, the probability that an atom has $|u| > a/2$ is:
\begin{equation}
P(|u| > a/2) = 2\int_{a/2}^\infty \frac{1}{\sqrt{2\pi\sigma^2}} \exp\left(-\frac{u^2}{2\sigma^2}\right) du = \text{erfc}\left(\frac{a}{2\sqrt{2}\sigma}\right),
\end{equation}
where $\text{erfc}(x) = 1 - \text{erf}(x)$ is the complementary error function.

When $\eta = \sigma/a = 0.15$:
\begin{equation}
P(|u| > a/2) = \text{erfc}\left(\frac{1}{2\sqrt{2} \times 0.15}\right) = \text{erfc}(2.36) \approx 0.0009.
\end{equation}

Approximately $0.09\%$ of atoms are in partition-ambiguous positions at any instant. This seems small, but the transition is cooperative: once some atoms lose site assignment, the potential wells for neighboring atoms become shallower (the restoring force weakens), increasing their amplitudes.

The transition is self-reinforcing: site partition extinction for a critical fraction of atoms triggers extinction for all atoms. The solid cannot maintain long-range order when a significant fraction of atoms cannot be assigned to sites. The lattice collapses into a liquid. \qed
\end{proof}

\begin{figure}[htbp]
\centering
\includegraphics[width=\textwidth]{figures/panel_lam_results.png}
\caption{\textbf{Lindemann Amplitude Monitor (LAM) results showing melting as partition extinction.} 
\textbf{(Top left)} Lindemann parameter $\eta$ vs. temperature for copper showing approach to melting transition. At low temperature, $\eta \sim 0.02$ as atoms vibrate in well-defined lattice sites. As temperature increases, $\eta$ grows approximately linearly, reaching critical value $\eta_c \approx 0.15$ (cyan dashed line) at melting temperature $T_m = 1358$ K (red dashed line). Above $T_m$, site partition becomes extinct as atoms can no longer be assigned to specific lattice sites. The Lindemann criterion $\eta_c \approx 0.15$ is universal across materials, arising from geometric constraint: when RMS displacement reaches $\sim$15\% of nearest-neighbor distance, site assignment becomes undefined.
\textbf{(Top right)} Lindemann parameter comparison across materials showing universal melting criterion. Copper (cyan), aluminum (magenta), gold (yellow), iron (orange), and lead (red) all follow similar curves when temperature is normalized by melting point $T/T_m$. All materials reach $\eta_c \approx 0.15$ (gray dashed line) at $T/T_m = 1$, confirming universality of partition extinction criterion. Low-melting materials (lead) have steeper curves due to weaker bonding. High-melting materials (iron) have shallower curves due to stronger bonding.
\textbf{(Bottom left)} RMS displacement vs. temperature for different materials showing atomic vibration amplitudes. Copper (orange) reaches RMS displacement $\sim 0.32$ Å at melting. Aluminum (white) reaches $\sim 0.22$ Å. Gold (yellow) reaches $\sim 0.18$ Å. Iron (magenta) reaches $\sim 0.14$ Å. Lead (blue) reaches $\sim 0.22$ Å. Despite different absolute displacements, all materials satisfy $\eta = u_{\text{RMS}}/a \approx 0.15$ at melting, where $a$ is the nearest-neighbor distance.
\textbf{(Bottom right)} Site assignment partition showing solid-liquid transition as partition extinction. In solid phase ($T < T_m$, green region), site partition is well-defined: atoms occupy specific lattice sites with high probability (green curve). Site assignment probability (cyan) remains near unity. Categorical potential $\Phi$ (magenta) increases gradually as thermal disorder grows. At melting point $T_m$ (orange dashed line), site partition becomes extinct: atoms cannot be assigned to specific sites. In liquid phase ($T > T_m$, red region), site partition is undefined. The discontinuous transition from defined to extinct partition explains first-order nature of melting: latent heat arises from categorical restructuring, not just energy increase.}
\label{fig:lam_results}
\end{figure}

\subsection{The Melting Temperature}

Setting $\eta = \eta_c$ in Eq.~\eqref{eq:msd_classical} and solving for temperature:
\begin{equation}
\frac{\sqrt{3k_B T_m/K}}{a} = \eta_c.
\label{eq:melting_condition}
\end{equation}

Squaring and rearranging:
\begin{equation}
T_m = \frac{\eta_c^2 K a^2}{3k_B} = \frac{\eta_c^2 m \omega_D^2 a^2}{3k_B}.
\label{eq:Tm_formula}
\end{equation}

Using the Debye approximation $\omega_D = v_s/a$ (where $v_s$ is the sound velocity):
\begin{equation}
T_m = \frac{\eta_c^2 m v_s^2}{3k_B}.
\label{eq:Tm_sound}
\end{equation}

This relates melting temperature to sound velocity and atomic mass—both measurable quantities. For typical values $\eta_c \approx 0.15$, $m \sim 10^{-25}$ kg, $v_s \sim 5000$ m/s:
\begin{equation}
T_m \sim \frac{(0.15)^2 \times 10^{-25} \times (5000)^2}{3 \times 1.38 \times 10^{-23}} \sim 1400 \text{ K},
\end{equation}
What is the correct order of magnitude for many metals.

\subsection{The Forgetting of Equilibrium}

The physical picture is that of an atom ``forgetting'' its equilibrium position as the temperature increases:

\subsubsection{Low Temperature ($T \ll T_m$)}

\begin{itemize}
\item Atom oscillates with small amplitude $\langle u^2 \rangle^{1/2} \ll a$ about equilibrium
\item Restoring force $F = -Ku$ always points toward the home site
\item Site assignment is unambiguous: atom is always closer to Site$_j$ than to any neighbor
\item Phonon transport dominates heat conduction $\kappa = (1/3)C_v v_s \lambda_{\text{phonon}}$
\end{itemize}

\subsubsection{High Temperature ($T \to T_m$)}

\begin{itemize}
\item Amplitude approaches lattice spacing: $\langle u^2 \rangle^{1/2} \sim 0.15a$
\item An atom spends a significant amount of time closer to neighbouring sites than to its home site
\item Restoring force direction becomes ambiguous: which site is ``home''?
\item Site assignment partition weakens: categorical identity of sites blurs
\end{itemize}

\subsubsection{At Melting ($T = T_m$)}

\begin{itemize}
\item Amplitude exceeds critical fraction: $\langle u^2 \rangle^{1/2} > \eta_c a$
\item Atom can no longer ``remember'' which site is home
\item Site assignment partition is extinct: atoms are not at sites; they are between sites
\item Solid becomes liquid: long-range order collapses
\end{itemize}

This is not a gradual crossover but a sharp transition. Below $T_m$, the lattice has long-range order (atoms assigned to sites). Above $T_m$, the liquid has only short-range order (atoms near neighbors, but no site assignment).

\subsection{Transport Mechanism Change at Melting}

The extinction of site assignment partition fundamentally changes the heat transport mechanism:

\subsubsection{In Solid (Site Partition Intact)}

Heat propagates via collective lattice modes (phonons):
\begin{equation}
\kappa_{\text{solid}} = \frac{1}{3} C_v v_s \lambda_{\text{phonon}},
\label{eq:kappa_solid}
\end{equation}
where $C_v$ is volumetric heat capacity, $v_s$ is sound velocity, and $\lambda_{\text{phonon}}$ is phonon mean free path (limited by phonon-phonon scattering, Umklapp processes).

The mechanism is collective: energy propagates through correlated atomic motion (phonon waves), not through individual atomic transport. Newton's cradle operates efficiently because atoms remain near their equilibrium positions.

\begin{figure*}[htbp]
\centering
\includegraphics[width=\textwidth]{figures/panel_epc_results.png}
\caption{\textbf{Entropy Production Camera (EPC) Visualization of Current Flow.} 
The entropy production rate $\sigma' = dS/dt$ per unit volume is visualized as a heat map. Cyan grid lines mark measurement positions. 
(\textbf{Top left}) Uniform temperature gradient: A 10 mm $\times$ 8 mm conductor with linear temperature gradient from left (cold) to right (hot). Entropy production is uniform ($\sigma' \approx 0.007$ entropy units), giving total $\Sigma \sigma' = 12.47$. 
(\textbf{Top right}) Central hot spot: A localized high-temperature region at the center creates radial entropy production. Cyan arrows show entropy flux direction (outward from hot spot). Maximum $\sigma' \approx 0.014$ at center. Total $\Sigma \sigma' = 7.94$. 
(\textbf{Bottom left}) Defect (high resistance region): A circular defect at $(x, y) \approx (5, 5)$ mm creates localized entropy production. The defect has higher resistivity, causing increased Joule heating. Maximum $\sigma' \approx 0.5$ at defect center. Total $\Sigma \sigma' = 28.57$. 
(\textbf{Bottom right}) Superconducting region ($x < 3$ mm): A vertical superconducting strip (orange, $x < 3$ mm) has zero entropy production ($\sigma' = 0$). The normal region ($x > 3$ mm, blue) has finite entropy production. The sharp boundary demonstrates the phase transition at $T_c$. Total $\Sigma \sigma' = 15.93$ (only from normal region). 
The EPC technique visualizes where energy is dissipated in conductors, enabling identification of defects, hot spots, and superconducting regions.}
\label{fig:epc_results}
\end{figure*}

\subsubsection{In Liquid (Site Partition Extinct)}

Heat propagates via molecular collisions:
\begin{equation}
\kappa_{\text{liquid}} = \frac{1}{3} n k_B v_{\text{mol}} \lambda_{\text{coll}},
\label{eq:kappa_liquid}
\end{equation}
where $n$ is number density, $v_{\text{mol}} = \sqrt{8k_B T/\pi m}$ is mean molecular velocity, and $\lambda_{\text{coll}}$ is mean free path between collisions.

The mechanism is individual: energy propagates through uncorrelated molecular motion (kinetic theory), not through collective modes. Molecules carry energy between collisions, similar to viscous momentum transport.

\subsubsection{Thermal Conductivity Ratio}

The ratio is approximately:
\begin{equation}
\frac{\kappa_{\text{solid}}}{\kappa_{\text{liquid}}} \sim \frac{v_s \lambda_{\text{phonon}}}{v_{\text{mol}} \lambda_{\text{coll}}} \sim \frac{5000 \times 10^{-6}}{500 \times 10^{-9}} \sim 10 \text{--} 100,
\label{eq:kappa_ratio}
\end{equation}
where typical values are: $v_s \sim 5000$ m/s, $\lambda_{\text{phonon}} \sim 1$ $\mu$m (limited by Umklapp), $v_{\text{mol}} \sim 500$ m/s, $\lambda_{\text{coll}} \sim 1$ nm (molecular diameter).

Solids typically have thermal conductivity 10--100 times higher than their liquids, reflecting the efficiency of collective phonon transport over individual molecular transport. This is a direct consequence of site partition extinction: the loss of long-range order destroys the collective modes that carry heat efficiently.

\subsection{Hierarchy of Partition Extinctions}

The various phase transitions form a hierarchy of partition extinctions, each corresponding to the loss of a specific categorical structure:

\begin{table}[h]
\centering
\caption{Phase transitions as partition extinctions}
\label{tab:phase_hierarchy}
\begin{tabular}{lccc}
\toprule
\textbf{Transition} & \textbf{Partition Extinct} & \textbf{$T_c$} & \textbf{Transport Change} \\
\midrule
Melting & Site assignment & $T_m$ & Phonon $\to$ collision \\
Vaporization & Spatial localization & $T_b$ & Collision $\to$ free flight \\
BEC/Superfluidity & Particle identity & $T_{\text{BEC}}$ & Dissipative $\to$ lossless \\
Superconductivity & Electron distinguishability & $T_c$ & Resistive $\to$ lossless \\
\bottomrule
\end{tabular}
\end{table}

Each transition corresponds to the extinction of a specific categorical partition:

\subsubsection{Melting ($T_m$)}

\textbf{Partition extinct:} Site assignment (atoms lose their assignment to specific lattice sites)

\textbf{What remains:} Particle identity (atoms retain individual identities) and spatial localisation (atoms remain in a condensed phase)

\textbf{Transport change:} Phonon transport $\to$ collision transport. Thermal conductivity drops by factor 10--100. Viscosity appears (liquids have finite viscosity, crystals do not flow).

\subsubsection{Vaporization ($T_b$)}

\textbf{Partition extinct:} Spatial localisation (atoms lose confinement to the condensed phase)

\textbf{What remains:} Particle identity (atoms retain individual identities)

\textbf{Transport change:} Collision transport $\to$ free flight transport. Mean free path increases from $\sim$nm (liquid) to $\sim$$\mu$m (gas). Thermal conductivity drops by a factor of $\sim$100. Viscosity decreases (gas viscosity $\ll$ liquid viscosity).

\subsubsection{BEC/Superfluidity ($T_{\text{BEC}}$ or $T_\lambda$)}

\textbf{Partition extinct:} Particle identity (atoms lose their individual identities and form a macroscopic wavefunction)

\textbf{What remains:} Spatial localisation (the condensate remains in the condensed phase)

\textbf{Transport change:} Dissipative transport $\to$ lossless transport. Viscosity drops to exactly zero. Thermal conductivity diverges (no scattering). Quantized circulation appears.

\subsubsection{Superconductivity ($T_c$)}

\textbf{Partition extinct:} Electron distinguishability (electron pairs, Cooper pairs lose individual identities)

\textbf{What remains:} Lattice structure (metal remains crystalline)

\textbf{Transport change:} Resistive transport $\to$ lossless transport. Resistivity drops to exactly zero. Meissner effect (perfect diamagnetism). Flux quantisation.

\subsection{Why Liquids Can Become Superfluid}

Helium-4 remains liquid down to absolute zero (at normal pressure) because its zero-point motion prevents site assignment partition from forming—it never crystallises \citep{wilks1967}. This allows it to undergo a different transition: extinction of particle identity partition, producing superfluidity.

\begin{proposition}[Helium Anomaly]
\label{prop:helium}
Helium-4 atoms have such large zero-point energy that $\eta > \eta_c$ even at $T = 0$:
\begin{equation}
\eta_0 = \frac{\langle u^2 \rangle_0^{1/2}}{a} = \frac{1}{a}\sqrt{\frac{3\hbar}{2m\omega_D}} > \eta_c.
\label{eq:helium_eta}
\end{equation}
Site assignment is never established; helium remains liquid.
\end{proposition}

\begin{proof}
From Eq.~\eqref{eq:msd_quantum}, the zero-point ($T = 0$) mean-square displacement is:
\begin{equation}
\langle u^2 \rangle_0 = \frac{3\hbar}{2m\omega_D}.
\end{equation}

For helium-4: $m = 6.65 \times 10^{-27}$ kg, $\omega_D \approx 2\pi \times 3 \times 10^{12}$ Hz (Debye frequency), $a \approx 3.6 \times 10^{-10}$ m (nearest-neighbour spacing in the hypothetical solid). This gives:
\begin{equation}
\langle u^2 \rangle_0^{1/2} \approx 0.7 \times 10^{-10} \text{ m}, \quad \eta_0 \approx \frac{0.7 \times 10^{-10}}{3.6 \times 10^{-10}} \approx 0.19.
\end{equation}

Since $\eta_0 \approx 0.19 > \eta_c \approx 0.15$, the zero-point motion alone exceeds the Lindemann criterion. Helium cannot form a solid at $T = 0$ (at normal pressure) because quantum fluctuations are too large. \qed
\end{proof}

Because helium never forms a solid (no site partition), it can undergo a different transition: the extinction of particle identity partition, producing superfluidity at $T_\lambda = 2.17$ K.

The sequence is:
\begin{itemize}
\item \textbf{Normal solids:} Site partition intact at low $T$, melts at $T_m$ when site partition extincts
\item \textbf{Helium:} The site partition never forms (quantum zero-point motion is too large); the identity partition extinguishes at $T_\lambda$ (superfluidity)
\end{itemize}

This explains why helium is the only element that remains liquid at $T = 0$: it is too quantum-mechanical to solidify.

\begin{figure}[htbp]
\centering
\includegraphics[width=\textwidth]{figures/panel_thermal_properties.png}
\caption{\textbf{Thermal transport material properties showing temperature dependence of thermal parameters.} 
\textbf{(Top left)} Thermal conductivity vs. temperature for different materials. Diamond (cyan) has highest conductivity $\kappa \sim 2000$ W/(m$\cdot$K) at room temperature due to light atoms, strong covalent bonds, and high Debye temperature. Copper (orange) has $\kappa \sim 400$ W/(m$\cdot$K) from electron transport. Aluminum (green) has $\kappa \sim 200$ W/(m$\cdot$K). Silicon (yellow) has $\kappa \sim 150$ W/(m$\cdot$K) from phonon transport. Glass (magenta) has low $\kappa \sim 1$ W/(m$\cdot$K) due to disordered structure. All materials show decreasing conductivity with increasing temperature (except glass) as phonon-phonon scattering increases.
\textbf{(Top right)} Thermal diffusivity $\alpha = \kappa/(\rho c_p)$ showing rate of temperature equilibration. Copper (yellow/orange) has highest diffusivity $\alpha \sim 140$ mm$^2$/s, equilibrating quickly. Aluminum (orange/red) has $\alpha \sim 100$ mm$^2$/s. Iron (purple) has $\alpha \sim 20$ mm$^2$/s. Silicon (magenta) has $\alpha \sim 60$ mm$^2$/s. SiO$_2$ (black) and H$_2$O (black) have low diffusivity $\alpha \sim 0.1$ mm$^2$/s, equilibrating slowly. Diffusivity determines transient thermal response: high $\alpha$ means fast equilibration, low $\alpha$ means slow equilibration.
\textbf{(Bottom left)} Thermal effusivity $e = \sqrt{\kappa\rho c_p}$ showing thermal inertia for contact heating. Copper (yellow) has highest effusivity $e \sim 35$ kW$\cdot$s$^{1/2}$/(m$^2\cdot$K), feeling cold to touch. Aluminum (salmon) has $e \sim 25$ kW$\cdot$s$^{1/2}$/(m$^2\cdot$K). Iron (magenta) has $e \sim 15$ kW$\cdot$s$^{1/2}$/(m$^2\cdot$K). Silicon (purple) has $e \sim 10$ kW$\cdot$s$^{1/2}$/(m$^2\cdot$K). SiO$_2$ (blue) and H$_2$O (gray) have low effusivity $e \sim 1$ kW$\cdot$s$^{1/2}$/(m$^2\cdot$K), feeling warm to touch. Effusivity determines initial heat flux during contact: high $e$ extracts heat quickly, low $e$ extracts heat slowly.
\textbf{(Bottom right)} Thermal inertia $I = \rho c_p$ showing volumetric heat capacity. Copper (lime) has $I \sim 4$ MJ/(m$^3\cdot$K). Aluminum (cyan) has $I \sim 3$ MJ/(m$^3\cdot$K). Iron (yellow) has $I \sim 2.5$ MJ/(m$^3\cdot$K). Silicon (cyan) has $I \sim 2$ MJ/(m$^3\cdot$K). H$_2$O (yellow/purple) has $I \sim 4$ MJ/(m$^3\cdot$K) despite low conductivity, providing excellent thermal storage. Thermal inertia determines temperature change for given heat input: high $I$ means small temperature change, low $I$ means large temperature change.}
\label{fig:thermal_properties}
\end{figure}

\subsection{Entropy of Melting}

The entropy change at melting is given by the Clausius relation:
\begin{equation}
\Delta S_m = \frac{\Delta H_m}{T_m},
\label{eq:entropy_melting}
\end{equation}
where $\Delta H_m$ is the enthalpy (latent heat) of melting.

For most elements, $\Delta S_m \approx R$ (Richard's rule), where $R = 8.314$ J/(mol$\cdot$K) is the gas constant \citep{richard1897}. This corresponds to $\Delta S_m \approx k_B$ per atom.

\begin{theorem}[Entropy of Site Partition Extinction]
\label{thm:entropy_site}
The entropy of melting represents the categorical information lost when site assignment becomes undefined:
\begin{equation}
\Delta S_m = k_B \ln \Omega_{\text{liquid}} - k_B \ln \Omega_{\text{solid}},
\label{eq:entropy_partition}
\end{equation}
where $\Omega$ is the number of accessible microstates.
\end{theorem}

\begin{proof}
In the solid, each atom is assigned to a specific lattice site. The number of accessible configurations is limited by the constraint that atoms remain near their assigned sites. The configurational entropy is approximately zero (atoms are localized).

In the liquid, atoms are not assigned to specific sites. Each atom can be anywhere within the liquid volume (subject to excluded volume constraints from other atoms). The number of accessible configurations is vastly larger.

The excess configurational entropy is approximately:
\begin{equation}
\Delta S_m \approx k_B \ln\left(\frac{V_{\text{free}}}{V_{\text{atom}}}\right) \approx k_B,
\end{equation}
where $V_{\text{free}}$ is the free volume available to each atom in the liquid and $V_{\text{atom}}$ is the atomic volume. The ratio is of order unity, giving $\Delta S_m \sim k_B$ per atom, or $R$ per mole. \qed
\end{proof}

This explains Richard's rule: the entropy of melting is approximately one gas constant per mole because melting destroys one categorical partition (site assignment), liberating approximately $k_B$ of configurational entropy per atom.

\subsection{Connexion to the Universal Transport Formula}

The transport mechanism change at melting connects directly to the universal transport formula. In the solid:
\begin{equation}
\kappa_{\text{solid}}^{-1} = \frac{1}{\mathcal{N}_{\text{phonon}}} \sum_{\text{phonon modes}} \tau_{\text{phonon}} g_{\text{phonon}},
\end{equation}
where the sum is over collective phonon modes. The partition operations are phonon-phonon scattering events (Umklapp processes).

In the liquid:
\begin{equation}
\kappa_{\text{liquid}}^{-1} = \frac{1}{\mathcal{N}_{\text{molecule}}} \sum_{\text{molecules}} \tau_{\text{collision}} g_{\text{collision}},
\end{equation}
where the sum is over individual molecular collisions. The partition operations are molecule-molecule scattering events.

The extinction of site partition changes the categorical structure of the system: from collective modes (phonons) to individual particles (molecules). This changes the partition operations from phonon scattering to molecular scattering, fundamentally altering the transport mechanism.
