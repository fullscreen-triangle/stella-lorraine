%==============================================================================
\section{The Partition Extinction Theorem}
\label{sec:extinction}
%==============================================================================

\subsection{Categorical Unification of Carriers}

The transport coefficients derived in Sections~\ref{sec:electrical}--\ref{sec:thermal} all depend on partition operations between carriers. Electrical resistivity arises from electron-phonon scattering partitions, viscosity from molecular collision partitions, diffusivity from atomic scattering partitions, and thermal resistance from phonon-phonon scattering partitions. In every case, the transport coefficient is proportional to the partition lag:
\begin{equation}
\Xi = \frac{1}{\mathcal{N}} \sum_{i,j} \tau_{p,ij} g_{ij}.
\label{eq:transport_recap}
\end{equation}

If partition operations cease, transport coefficients vanish. We now establish the conditions under which partition operations become impossible.

\begin{definition}[Phase-Locking]
\label{def:phase_locking}
Two carriers $i$ and $j$ are \emph{phase-locked} if their oscillatory modes maintain a fixed phase relationship. Phase-locked carriers form a single categorical entity: they cannot be distinguished by any partition operation.
\end{definition}

Phase-locking is not merely correlation or synchronization---it is categorical unification. The carriers lose their individual identities and become a single quantum-mechanical entity described by a macroscopic wavefunction. Any attempt to distinguish them (to perform a partition operation) fails because there is only one entity present, not multiple entities.

Phase-locking requires energy. Let $\Delta_{\text{lock}}$ be the \emph{phase-locking energy}---the binding energy that maintains the fixed phase relationship. Thermal fluctuations with energy $k_B T$ can disrupt phase-locking if $k_B T > \Delta_{\text{lock}}$.

\begin{theorem}[Phase-Locking Condition]
\label{thm:phase_lock}
Carriers become phase-locked when thermal energy falls below the phase-locking energy:
\begin{equation}
k_B T < \Delta_{\text{lock}}.
\label{eq:phase_lock_condition}
\end{equation}
This defines the critical temperature:
\begin{equation}
T_c = \frac{\Delta_{\text{lock}}}{k_B}.
\label{eq:critical_temperature}
\end{equation}
\end{theorem}

\begin{proof}
Phase-locking is maintained by an attractive interaction (phonon-mediated for Cooper pairs, van der Waals for helium-4 atoms, etc.) with characteristic energy $\Delta_{\text{lock}}$. Thermal fluctuations provide energy $\sim k_B T$ that can break the phase-lock.

When $k_B T \gg \Delta_{\text{lock}}$, thermal fluctuations dominate, and carriers are distinguishable (normal state). When $k_B T \ll \Delta_{\text{lock}}$, the phase-locking interaction dominates, and carriers unify (ordered state). The transition occurs at $k_B T \sim \Delta_{\text{lock}}$, defining $T_c = \Delta_{\text{lock}}/k_B$. \qed
\end{proof}

The phase-locking energy $\Delta_{\text{lock}}$ has different physical origins in different systems:
\begin{itemize}
\item \textbf{Superconductors:} $\Delta_{\text{lock}} = \Delta_{\text{BCS}}$ (BCS gap energy from phonon-mediated attraction)
\item \textbf{Superfluid helium-4:} $\Delta_{\text{lock}} \sim k_B T_\lambda$ (van der Waals interaction energy)
\item \textbf{Bose-Einstein condensates:} $\Delta_{\text{lock}} \sim k_B T_{\text{BEC}}$ (quantum degeneracy energy)
\end{itemize}

\subsection{Partition Extinction}

\begin{theorem}[Partition Extinction]
\label{thm:partition_extinction}
When carriers become phase-locked, partition operations between them are undefined. The partition lag does not approach zero continuously but transitions discontinuously:
\begin{equation}
\tau_p(T) = \begin{cases}
\tau_{p,\text{normal}}(T) & T > T_c \\
0 & T < T_c
\end{cases}.
\label{eq:tau_discontinuous}
\end{equation}
\end{theorem}

\begin{proof}
Partition is a categorical operation that distinguishes entities. For partition to occur, entities must be distinguishable—they must have separate identities that can be tracked and separated.

Phase-locked carriers are categorically identical. They do not merely behave similarly; they \textit{are} the same entity. A Cooper pair is not two electrons that happen to move together—it is a single quantum state with no internal structure accessible to measurement. A Bose-Einstein condensate is not a collection of atoms that happen to occupy the same state—it is a single macroscopic wavefunction.

Attempting to partition a single entity is undefined. There is nothing to partition. The question ``which carrier is which?'' has no answer because there is only one carrier (the unified state), not multiple carriers.

The partition lag is not ``very small'' but exactly zero because no partition operation occurs. The transition is discontinuous because categorical identity is discrete: carriers are either distinguishable (partition possible, $\tau_p > 0$) or indistinguishable (partition impossible, $\tau_p = 0$). There is no intermediate state of ``partial distinguishability.''

Mathematically, the partition lag is:
\begin{equation}
\tau_p = \frac{1}{\Gamma_{\text{partition}}},
\end{equation}
where $\Gamma_{\text{partition}}$ is the partition rate (number of partition operations per unit time). When carriers are phase-locked, $\Gamma_{\text{partition}} = 0$ (no partition operations occur), giving $\tau_p = 1/0 = \infty$ (partition never completes) or equivalently $\tau_p = 0$ (no partition lag because no partition occurs). The latter interpretation is correct: the partition lag is zero because the partition operation is undefined, not because it completes instantaneously. \qed
\end{proof}

\begin{figure}[htbp]
\centering
\includegraphics[width=\textwidth]{figures/panel_partition_lag.png}
\caption{\textbf{Partition lag $\tau_p$ across all four transport types showing universal temperature dependence.} 
\textbf{(Top left)} Electrical partition lag showing scattering mechanism contributions. Phonon scattering (orange) dominates at high temperature with $\tau_p \sim 10^2$ fs at 500 K, decreasing from $\sim 10^3$ fs at low $T$ as phonon population increases ($\propto T$). Impurity scattering (magenta) is temperature-independent at $\tau_p \sim 10^4$ fs, providing residual scattering even at $T \to 0$. Electron-electron scattering (green) shows weak temperature dependence with $\tau_p \sim 10^4$ fs. All mechanisms contribute to total resistivity through $\rho = \mathcal{N}^{-1}\sum_{ij}\tau_{p,ij}g_{ij}$.
\textbf{(Top right)} Diffusive partition lag showing atomic jump mechanisms. Vacancy diffusion (bright green) has longest partition lag $\tau_p \sim 10^{15}$ fs ($\sim 1$ s) at 400 K, decreasing exponentially with temperature as thermal activation enables atomic jumps: $\tau_p \propto \exp(E_a/k_B T)$. Interstitial diffusion (medium green) has shorter lag $\tau_p \sim 10^{13}$ fs ($\sim 10$ ms) due to lower activation barrier. Grain boundary diffusion (dark green) has intermediate lag $\tau_p \sim 10^7$ fs ($\sim 10$ ns) as atoms diffuse along defects with reduced barriers. The enormous range of partition lags (10$^2$--10$^{15}$ fs) reflects the wide range of diffusion timescales from fast interstitial motion to slow vacancy migration.
\textbf{(Bottom left)} Thermal partition lag showing phonon scattering vs. frequency. Normal scattering (cyan) has constant partition lag $\tau_p \sim 10^3$ ps across all frequencies, as normal processes conserve crystal momentum and don't limit thermal transport. Umklapp scattering (orange) shows strong frequency dependence: $\tau_p \sim 10^1$ ps at low frequency ($\omega \sim 1$ THz), decreasing to $\sim 10^0$ ps at high frequency ($\omega \sim 14$ THz) as umklapp phase space increases. Boundary scattering (green) is frequency-independent at $\tau_p \sim 10^3$ ps. Impurity scattering (magenta) shows weak frequency dependence with $\tau_p \sim 10^2$ ps. The frequency-dependent partition lag determines thermal conductivity spectrum $\kappa(\omega)$.
\textbf{(Bottom right)} Viscous partition lag showing molecular collision times. Water (cyan) has shortest partition lag $\tau_p \sim 10^0$ ps at 600 K, increasing to $\sim 10^2$ ps at 200 K as molecular collision rate decreases with temperature. Glycerol (magenta) has much longer lag $\tau_p \sim 10^{17}$ ps ($\sim 10^5$ s) at 200 K due to strong hydrogen bonding, decreasing exponentially to $\sim 10^9$ ps ($\sim 1$ s) at 600 K as bonds break. n-Hexane (green) has intermediate lag $\tau_p \sim 10^2$ ps. 
\textbf{Universal structure:} All four transport types show partition lag decreasing with temperature (or frequency), following Arrhenius-like behavior $\tau_p \propto \exp(E_a/k_B T)$ where activation energy $E_a$ represents the energy barrier for partition operations. The universal formula $\text{Transport coefficient} \propto \sum_{ij}\tau_{p,ij}g_{ij}$ applies across all modes, with only the carrier type and coupling structure differing. This demonstrates the deep unity of transport phenomena: all arise from the same categorical partition dynamics, differing only in timescales and interaction strengths.}
\label{fig:partition_lag_comparison}
\end{figure}

\subsection{Transport Coefficient Vanishing}

\begin{corollary}[Transport Coefficient Vanishing]
\label{cor:transport_vanishing}
Below $T_c$, the transport coefficient vanishes exactly:
\begin{equation}
\Xi(T < T_c) = \frac{1}{\mathcal{N}} \sum_{i,j} \tau_{p,ij}(T) g_{ij} = 0.
\label{eq:transport_zero}
\end{equation}
\end{corollary}

\begin{proof}
When all carriers are phase-locked, $\tau_{p,ij} = 0$ for all pairs $(i,j)$. The sum vanishes identically:
\begin{equation}
\sum_{i,j} \tau_{p,ij} g_{ij} = \sum_{i,j} 0 \cdot g_{ij} = 0.
\end{equation}

This is not an asymptotic limit ($\Xi \to 0$ as $T \to T_c$) but an exact result ($\Xi = 0$ for $T < T_c$). The transport coefficient does not become small—it becomes zero. \qed
\end{proof}

This explains the defining property of dissipationless states: exactly zero resistance, viscosity, or diffusive scattering. The transport coefficient is not merely very small (which would allow slow dissipation); it is exactly zero (which forbids dissipation entirely).

\subsection{Physical Interpretation}

The partition extinction theorem has a physical interpretation in terms of Newton's cradle mechanism discussed in Section~\ref{sec:electrical}.

\subsubsection{Normal State ($T > T_c$)}

At high temperatures, carriers (electrons, molecules, atoms) undergo thermal motion in three dimensions. Sequential momentum transfer—the Newton's cradle mechanism—is disrupted by thermal jiggling. Each collision event is a partition operation that randomises carrier trajectories, producing resistance, viscosity, or diffusive scattering.

Consider electrons in a metal. Each electron has thermal energy $\sim k_B T$ and moves with a velocity $\sim v_F$ (Fermi velocity). When an electric field is applied, electrons acquire a drift velocity $v_d \ll v_F$ superimposed on their thermal motion. Collisions with phonons randomise the drift momentum, creating resistance. Each collision is a partition operation: the electron's pre-collision state (momentum $\mathbf{k}_i$) is distinguished from its post-collision state (momentum $\mathbf{k}_f$).

\subsubsection{Approaching $T_c$}

As temperature decreases, thermal motion decreases. Carriers align more precisely. Newton's cradle operates more cleanly—collisions become less frequent, and the partition lag increases (or equivalently, the scattering rate decreases). The transport coefficient decreases continuously as $T \to T_c^+$.

However, this continuous decrease does not extend through $T_c$. At $T_c$, a qualitative change occurs.

\subsubsection{Ordered State ($T < T_c$)}

Below $T_c$, carriers become phase-locked. They no longer behave as individual entities but as a single collective mode. The Newton's cradle analogy breaks down: there are no individual balls to collide, only a single unified object.

Transport occurs without scattering because there are no individual carriers to scatter. The macroscopic wavefunction propagates coherently through the material. Obstacles that would scatter individual carriers (phonons, impurities, defects) cannot scatter the collective state because they cannot distinguish individual carriers within it.

This is not a matter of degree (fewer scattering events) but of kind (no scattering events). The partition operations that create resistance in the normal state are undefined in the ordered state.

\subsection{The Role of Bosonic Statistics}

The mechanism of phase-locking depends on the quantum statistics of the carriers.

\subsubsection{Bosons}

For bosonic carriers (helium-4 atoms, photons, phonons), phase-locking produces \emph{Bose-Einstein condensation}. All carriers occupy the same quantum state $|\psi_0\rangle$, which is the same as saying they form a single categorical entity.

The wavefunction of the condensate is:
\begin{equation}
\Psi(\mathbf{r}_1, \mathbf{r}_2, \ldots, \mathbf{r}_N) = \prod_{i=1}^N \psi_0(\mathbf{r}_i),
\label{eq:bec_wavefunction}
\end{equation}
where $N$ is the number of atoms. This is a product state, not an entangled state, but it is categorically unified: all atoms are in the same state, so they are indistinguishable.

Attempting to partition this state---to ask ``which atom is which?''---is undefined. The atoms have no individual identities. They are all $|\psi_0\rangle$.

\subsubsection{Fermions}

For fermionic carriers (electrons), phase-locking requires pairing. The Pauli exclusion principle prevents multiple fermions from occupying the same state. However, pairs of fermions (with opposite spin) form composite bosons that can condense.

This is the \emph{Cooper pairing} mechanism \citep{cooper1956}. Two electrons with opposite spins $(\uparrow, \downarrow)$ and opposite momenta $(\mathbf{k}, -\mathbf{k})$ form a bound state (Cooper pair) with total spin zero and total momentum zero. The pair is a boson and can condense.

\begin{proposition}[Cooper Pairing as Categorical Unification]
\label{prop:cooper}
Cooper pairs are categorically unified electron pairs. The pairing breaks the distinguishability of individual electrons, extinguishing partition operations.
\end{proposition}

\begin{proof}
The Cooper pair wavefunction is:
\begin{equation}
\Psi_{\text{pair}}(\mathbf{r}_1, \mathbf{r}_2) = \phi(\mathbf{r}_1 - \mathbf{r}_2) \cdot e^{i\mathbf{K} \cdot (\mathbf{r}_1 + \mathbf{r}_2)/2},
\label{eq:cooper_wavefunction_ext}
\end{equation}
where $\phi(\mathbf{r}_1 - \mathbf{r}_2)$ is the pair envelope (size $\sim \xi$, the coherence length) and $\mathbf{K}$ is the centre-of-mass momentum.

The pair has no internal structure accessible to measurement. Asking ``which electron is electron 1 and which is electron 2?'' is undefined because the wavefunction is symmetric under exchange: $\Psi_{\text{pair}}(\mathbf{r}_1, \mathbf{r}_2) = \Psi_{\text{pair}}(\mathbf{r}_2, \mathbf{r}_1)$.

Partition operations require distinguishability. Since the electrons in a Cooper pair are indistinguishable, partition is undefined. \qed
\end{proof}

The BCS gap $\Delta$ is the energy required to break a Cooper pair---equivalently, to restore distinguishability and enable partition. The relation $\Delta = 1.76 k_B T_c$ follows from the partition extinction condition \citep{bardeen1957}: at $T = T_c$, thermal energy $k_B T_c$ equals the average gap energy $\Delta/1.76$, breaking pairs and restoring partition.

\subsection{Thermal de Broglie Wavelength}

For atomic gases, phase-locking occurs when quantum wavefunctions overlap. The thermal de Broglie wavelength is:
\begin{equation}
\lambda_{\text{dB}} = \sqrt{\frac{2\pi\hbar^2}{m k_B T}}.
\label{eq:de_broglie}
\end{equation}

This is the characteristic size of the wavefunction of an atom with thermal energy $k_B T$. At high temperatures, $\lambda_{\text{dB}}$ is small compared to the interatomic spacing $d \sim n^{-1/3}$ (where $n$ is the number density). Wavefunctions do not overlap, and atoms are distinguishable.

As the temperature decreases, $\lambda_{\text{dB}}$ increases. When $\lambda_{\text{dB}}$ exceeds the interatomic spacing:
\begin{equation}
\lambda_{\text{dB}} > n^{-1/3},
\label{eq:overlap_condition}
\end{equation}
wavefunctions overlap, and atoms become indistinguishable. This is the onset of Bose-Einstein condensation.

Setting $\lambda_{\text{dB}} = n^{-1/3}$ gives the BEC critical temperature:
\begin{equation}
T_{\text{BEC}} = \frac{2\pi\hbar^2}{m k_B} \left( \frac{n}{\zeta(3/2)} \right)^{2/3},
\label{eq:T_BEC}
\end{equation}
where $\zeta(3/2) \approx 2.612$ is the Riemann zeta function \citep{bose1924,einstein1924,pethick2008}.

For typical experimental parameters ($n \sim 10^{13}$--$10^{15}$ cm$^{-3}$, $m \sim 10^{-25}$ kg), this gives $T_{\text{BEC}} \sim 100$ nK--$1$ $\mu$K, consistent with observations in ultracold atomic gases.

\subsection{Connection to Absolute Zero}

The partition extinction theorem connects to the thermodynamic limit $T \to 0$. As established in prior work \citep{sachikonye2025kelvin}, absolute zero is the categorical boundary where time ceases to exist—where no categorical completions occur.

Transport requires categorical operations (partition, scattering, measurement). As $T \to 0$:
\begin{enumerate}
\item Thermal fluctuations vanish ($k_B T \to 0$)
\item Carriers phase-lock (wavefunction overlap becomes total)
\item Partition becomes impossible (no distinguishability)
\item Transport becomes dissipationless (no scattering events)
\end{enumerate}

The dissipationless states (superconductivity, superfluidity, BEC) are partial approaches to the $T = 0$ limit. Carriers achieve categorical unification while the system remains at $T > 0$. Transport occurs (current flows, mass flows, heat flows), but without partition—without the entropy-producing scattering events that constitute dissipation.

This explains why dissipationless states are fundamentally different from merely low-resistance states. A very pure metal at low temperature can have very low resistivity ($\rho \sim 10^{-10}$ $\Omega\cdot$m), but it is not a superconductor. Partition operations still occur; they are merely rare. In a superconductor, partition operations are undefined. The difference is categorical, not quantitative.

\begin{figure}[htbp]
\centering
\includegraphics[width=\textwidth]{figures/panel_temperature_superconductivity.png}
\caption{\textbf{Temperature dependence of resistivity and the superconducting transition as aperture bypass.} 
\textbf{(A)} Matthiessen's rule showing additive resistivities: $\rho(T) = \rho_0 + \rho_{\text{ph}}(T)$. Residual resistivity $\rho_0$ (green dashed) arises from temperature-independent impurity scattering. Phonon resistivity $\rho_{\text{ph}}(T)$ (red dotted) increases linearly with temperature at high $T$ as phonon population grows. Total resistivity (blue solid) is the sum, demonstrating that independent scattering channels add: $\tau_{\text{total}}^{-1} = \sum_i \tau_i^{-1}$. 
\textbf{(B)} Superconducting transition in niobium ($T_c = 9.2$ K, red dashed line) showing discontinuous drop in resistivity. Above $T_c$, resistivity follows normal-state behavior (blue line). Below $T_c$, resistivity drops to exactly zero (green region) within millikelvins. This discontinuity arises from partition extinction: Cooper pairs become categorically indistinguishable, making partition operations undefined and eliminating all scattering. 
\textbf{(C)} Cooper pair aperture bypass mechanism. In normal state ($T > T_c$, blue box), individual electrons (blue dots) are distinguishable and scatter from apertures (red X). In superconducting state ($T < T_c$, green box), electrons form Cooper pairs (paired blue dots) that are categorically unified. Partition operations between unified carriers are undefined, so pairs bypass all apertures without scattering. 
\textbf{(D)} Skin effect as frequency-dependent aperture selectivity. Skin depth $\delta = \sqrt{2/(\mu_0 \sigma \omega)}$ decreases with frequency. At 60 Hz (power frequency), $\delta \sim 10^4$ $\mu$m (bulk conduction). At 1 MHz (RF), $\delta \sim 10^2$ $\mu$m (surface conduction). At 1 GHz (microwave), $\delta \sim 1$ $\mu$m (extreme surface conduction). High-frequency fields partition electrons only near the surface, effectively reducing the aperture cross-section.}
\label{fig:temperature_superconductivity}
\end{figure}

\subsection{Experimental Signatures}

The discontinuous nature of partition extinction predicts several experimental signatures, all of which are observed:

\begin{enumerate}
\item \textbf{Sharp transitions at $T_c$:} The transport coefficient drops discontinuously from $\Xi(T_c^+) > 0$ to $\Xi(T_c^-) = 0$. This is observed as a sharp resistivity drop in superconductors, a sharp viscosity drop in superfluids, and a sharp change in heat capacity at the BEC transition.

\item \textbf{Exactly zero transport coefficient below $T_c$:} Not merely small, but zero. Superconducting currents persist for years without decay \citep{file1964}. Superfluid flow shows no measurable viscosity \citep{allen1938}. These are not asymptotic limits but exact zeros.

\item \textbf{Macroscopic quantum coherence:} The system is described by a single macroscopic wavefunction (order parameter). This coherence is observable through interference experiments (Josephson effect \citep{josephson1962}, superfluid interferometry \citep{packard1970}).

\item \textbf{Quantized collective excitations:} Flux quanta in superconductors ($\Phi_0 = h/2e$), quantised vortices in superfluids ($\kappa = h/m$), and quantised circulation in BECs. These arise because the macroscopic wave function must be single-valued, imposing topological constraints.

\item \textbf{Energy gap:} A finite energy $\Delta$ is required to create excitations (break pairs, create quasiparticles). This gap is observable in tunnelling experiments \citep{giaever1960}, specific heat measurements \citep{corak1954}, and spectroscopy.

\item \textbf{Meissner effect (superconductors):} The expulsion of magnetic fields due to perfect diamagnetism. This is a consequence of the macroscopic phase coherence of the Cooper pair wave function \citep{meissner1933}.

\item \textbf{Fountain effect (superfluids):} Superfluid helium flows through capillaries without viscosity and can flow upward against gravity (thermomechanical effect) \citep{allen1938}.

\item \textbf{Persistent currents (BECs):} Stirred BECs exhibit persistent rotation with quantised angular momentum, analogous to superconducting persistent currents \citep{madison2000}.
\end{enumerate}

All these signatures are manifestations of partition extinction: the categorical unification of carriers that makes partition operations undefined.

\subsection{Two-Fluid Model}

In real systems, partition extinction is not always complete. Thermal excitations above the ground state (quasiparticles, rotons, etc.) remain distinguishable and can undergo partition operations. This leads to the \emph{two-fluid model} \citep{tisza1938,landau1941}.

The system is described as a mixture of two components:
\begin{itemize}
\item \textbf{Superfluid/superconducting component:} Fraction $\rho_s/\rho$ (or $n_s/n$) in the ground state. Phase-locked, $\tau_p = 0$, zero viscosity/resistivity.
\item \textbf{Normal component:} Fraction $\rho_n/\rho$ (or $n_n/n$) in excited states. Distinguishable, $\tau_p > 0$, finite viscosity/resistivity.
\end{itemize}

The total transport coefficient is:
\begin{equation}
\Xi(T) = \frac{\rho_n(T)}{\rho} \Xi_{\text{normal}}(T),
\label{eq:two_fluid_transport}
\end{equation}
where $\rho_n(T)/\rho$ is the normal fraction. As $T \to 0$, $\rho_n \to 0$, so $\Xi \to 0$.

The temperature dependence of the normal fraction depends on the excitation spectrum:
\begin{itemize}
\item \textbf{Superconductors:} $\rho_n/\rho \propto \exp(-\Delta/k_B T)$ (exponential suppression due to gap)
\item \textbf{Superfluid helium-4:} $\rho_n/\rho \propto T^4$ (roton excitations at low $T$)
\item \textbf{BECs:} $\rho_n/\rho \propto (T/T_c)^{3/2}$ (thermal depletion of condensate)
\end{itemize}

The two-fluid model shows that partition extinction can be partial: some carriers are phase-locked (partition extinct), while others remain distinguishable (partition active). The macroscopic transport coefficient reflects the weighted average.
