\section{Continuous Emergence}
\label{sec:continuous}

\subsection{The Limit Construction}

We now establish that infinite ternary strings correspond exactly to points in the continuous space $[0,1]^3$.

\begin{theorem}[Continuous Emergence]\label{thm:continuous-emergence}
The mapping $\phi : \mathbb{T}^\infty \to [0,1]^3$ defined by:
\begin{equation}
\phi(t_1, t_2, t_3, \ldots) = \lim_{k \to \infty} \phi_k(t_1, \ldots, t_k)
\end{equation}
is well-defined and surjective.
\end{theorem}

\begin{proof}
\textbf{Well-definedness:} For fixed infinite string $\mathbf{t} = (t_1, t_2, \ldots)$, consider the sequence of cells $C_k = \phi_k(t_1, \ldots, t_k)$.

By the nesting theorem, $C_{k+1} \subseteq C_k$ for all $k$. The cells are closed sets with:
\begin{equation}
\text{diam}(C_k) = \sqrt{3} \cdot 3^{-\lfloor k/3 \rfloor} \to 0 \text{ as } k \to \infty
\end{equation}

By the nested closed set theorem (Cantor's intersection theorem), since $[0,1]^3$ is complete:
\begin{equation}
\bigcap_{k=0}^\infty C_k = \{\Scoord\}
\end{equation}
is a singleton. Define $\phi(\mathbf{t}) = \Scoord$.

\textbf{Surjectivity:} For any $\Scoord \in [0,1]^3$, Theorem~\ref{thm:address-from-coords} constructs an infinite ternary string $\mathbf{t}$ with $\phi(\mathbf{t}) = \Scoord$. \qed
\end{proof}

\begin{theorem}[Continuity of Limit Map]\label{thm:continuity}
The map $\phi : \mathbb{T}^\infty \to [0,1]^3$ is continuous, where $\mathbb{T}^\infty$ carries the product topology.
\end{theorem}

\begin{proof}
The product topology on $\mathbb{T}^\infty$ has basis sets:
\begin{equation}
U_{t_1, \ldots, t_k} = \{(s_1, s_2, \ldots) \in \mathbb{T}^\infty : s_i = t_i \text{ for } i = 1, \ldots, k\}
\end{equation}

For any open ball $B_\epsilon(\Scoord) \subset [0,1]^3$, we need $\phi^{-1}(B_\epsilon(\Scoord))$ to be open in $\mathbb{T}^\infty$.

Choose $k$ such that $\sqrt{3} \cdot 3^{-\lfloor k/3 \rfloor} < \epsilon$. Let $\mathbf{t} \in \phi^{-1}(B_\epsilon(\Scoord))$ with $\phi(\mathbf{t}) = \Scoord' \in B_\epsilon(\Scoord)$.

The basis set $U_{t_1, \ldots, t_k}$ contains $\mathbf{t}$. For any $\mathbf{t}' \in U_{t_1, \ldots, t_k}$:
\begin{equation}
d(\phi(\mathbf{t}'), \phi(\mathbf{t})) \leq \sqrt{3} \cdot 3^{-\lfloor k/3 \rfloor} < \epsilon
\end{equation}

by Theorem~\ref{thm:distance}. Therefore, $\phi(\mathbf{t}') \in B_{2\epsilon}(\Scoord)$. Since $\epsilon$ was arbitrary, continuity follows. \qed
\end{proof}

\subsection{The Ternary-Continuous Bridge}

\begin{definition}[Finite Precision Approximation]
For $\Scoord \in [0,1]^3$ and precision $\delta > 0$, the \textbf{$\delta$-approximation} is the shortest ternary string $\mathbf{t}$ such that:
\begin{equation}
d(\phi(\mathbf{t}), \Scoord) < \delta
\end{equation}
\end{definition}

\begin{theorem}[Approximation Depth]\label{thm:approx-depth}
A $\delta$-approximation requires at most:
\begin{equation}
k = 3 \left\lceil \log_3 \frac{\sqrt{3}}{\delta} \right\rceil
\end{equation}
trits.
\end{theorem}

\begin{proof}
A $k$-trit string addresses a cell of diameter at most $\sqrt{3} \cdot 3^{-\lfloor k/3 \rfloor}$.

For $d(\phi(\mathbf{t}), \Scoord) < \delta$, we need:
\begin{equation}
\sqrt{3} \cdot 3^{-\lfloor k/3 \rfloor} < \delta
\end{equation}

Solving:
\begin{equation}
\lfloor k/3 \rfloor > \log_3 \frac{\sqrt{3}}{\delta} \implies k \geq 3 \left\lceil \log_3 \frac{\sqrt{3}}{\delta} \right\rceil
\end{equation}
\qed
\end{proof}

\begin{example}
For $\delta = 0.01$ (1\% precision):
\begin{equation}
k = 3 \left\lceil \log_3 \frac{\sqrt{3}}{0.01} \right\rceil = 3 \left\lceil \log_3 173.2 \right\rceil = 3 \cdot 5 = 15 \text{ trits}
\end{equation}

A 15-trit string (2.5 trytes) specifies S-coordinates to 1\% precision.
\end{example}

\begin{figure}[htbp]
  \centering
  \includegraphics[width=\textwidth]{figures/figure_2_hierarchical_partition.png}
  \caption{\textbf{The 3$^k$ Hierarchical Partition of S-Space Converges to Continuous Points as k→∞, Establishing Ternary Expansion as Bridge Between Discrete Computation and Continuous Dynamics.}
  \textbf{(Top Left)} First refinement level k=1: 3$^1$=3 cells. Three purple spheres positioned along Sk axis at coordinates (0, 1/3, 2/3), representing three cells created by initial ternary split. Trit addresses: "0" (left sphere, Sk∈[0,1/3]), "1" (middle sphere, Sk∈[1/3,2/3]), "2" (right sphere, Sk∈[2/3,1]). Each cell occupies volume 1/3 of unit cube. First trit selects which third of Sk axis to refine.
  \textbf{(Top Right)} Second refinement level k=2: 3$^2$=9 cells. Nine spheres (purple to cyan gradient) positioned in 3×3 grid on Sk-St plane. Each k=1 cell subdivided into three cells along St axis. Trit addresses: "00" (purple, Sk∈[0,1/3], St∈[0,1/3]), "01" (Sk∈[0,1/3], St∈[1/3,2/3]), "02" (Sk∈[0,1/3], St∈[2/3,1]), continuing through "22" (cyan, Sk∈[2/3,1], St∈[2/3,1]). Each cell occupies volume 1/9. Second trit selects which third of St axis to refine within chosen Sk slice.
  \textbf{(Bottom Left)} Third refinement level k=3: 3$^3$=27 cells. Twenty-seven spheres (purple to yellow gradient) filling 3D volume in 3×3×3 grid. Each k=2 cell subdivided into three cells along Se axis. Color coding indicates coordinate values: purple (low Sk, St, Se), cyan (intermediate), green (mid-range), yellow (high). Annotated addresses visible: "00" (purple sphere, origin corner), "11" (green sphere, center), "12" (green-yellow sphere), "21" (yellow-green sphere), "22" (yellow sphere, opposite corner). Each cell occupies volume 1/27. Third trit selects which third of Se axis to refine within chosen Sk-St slice. Complete 3D space coverage achieved: all 27 cells tile [0,1]$^3$ without gaps or overlaps.
  \textbf{(Bottom Right)} Convergence to continuous point as k→∞. Log-linear plot showing distance from cell center to target point S=(0.7, 0.5, 0.3) versus trit depth k. Blue circles: measured distances at k=1,2,3,4,5,6,7,8,9,10. Red dashed line: theoretical bound $\sqrt{3} \cdot 3^{-k/3}$ (diagonal of rectangular cell). Distance decreases exponentially: k=1 (distance ≈0.6), k=3 (≈0.2), k=6 (≈0.07), k=9 (≈0.02), k=10 (≈0.015). Convergence rate O(3$^{-k/3}$) confirms each additional trit reduces uncertainty by factor of 3$^{1/3}$≈1.44 per dimension. After k=9 trits (three 3-trit coordinates), distance <0.02, sufficient precision for most applications. Infinite ternary string converges to unique point: $\lim_{k \to \infty}$ cell$_k$ = S. Theorem 5.2 establishes bijection between infinite trit sequences and points in [0,1]$^3$, resolving discrete-continuous duality.}
  \label{fig:hierarchical_partition}
\end{figure}

\subsection{Cantor Set Connection}

\begin{definition}[Ternary Cantor Set]
The \textbf{Cantor set} $\mathcal{K}$ is the set of points in $[0,1]$ with ternary expansions using only digits $\{0, 2\}$:
\begin{equation}
\mathcal{K} = \left\{ \sum_{i=1}^\infty t_i \cdot 3^{-i} : t_i \in \{0, 2\} \right\}
\end{equation}
\end{definition}

\begin{remark}
The Cantor set is the prototypical example of a set that is ``large'' (uncountable, same cardinality as $\mathbb{R}$) yet ``small'' (measure zero, totally disconnected). It arises naturally from ternary representation when the middle digit is excluded.

In our framework, the full ternary set $\{0, 1, 2\}$ is used, so we obtain the full interval $[0,1]$ rather than the Cantor set. The Cantor set can be viewed as the ``skeleton'' of the ternary structure.
\end{remark}

\subsection{Space-Filling Properties}

\begin{theorem}[Ternary Space-Filling]\label{thm:space-filling}
The map $\phi : \mathbb{T}^\infty \to [0,1]^3$ covers the entire cube: for every $\Scoord \in [0,1]^3$, there exists $\mathbf{t} \in \mathbb{T}^\infty$ with $\phi(\mathbf{t}) = \Scoord$.
\end{theorem}

\begin{proof}
This is the surjectivity part of Theorem~\ref{thm:continuous-emergence}. \qed
\end{proof}

\begin{remark}
This theorem establishes that the ternary representation is \textit{complete}: every point in S-space has a ternary address. There are no ``gaps'' or ``holes'' in the coverage.

The connection to space-filling curves \citep{peano1890courbe, sagan1994space} is deep. While Peano's original curve uses a continuous surjection $[0,1] \to [0,1]^2$, our construction uses the product structure of $\mathbb{T}^\infty$ to achieve a similar covering of $[0,1]^3$.
\end{remark}

\subsection{Measure-Theoretic Properties}

\begin{theorem}[Measure Correspondence]\label{thm:measure}
The Lebesgue measure on $[0,1]^3$ corresponds to the product measure on $\mathbb{T}^\infty$ (with uniform distribution on each $\{0, 1, 2\}$).
\end{theorem}

\begin{proof}
A basic open set in $\mathbb{T}^\infty$ is:
\begin{equation}
U_{t_1, \ldots, t_k} = \{(s_1, s_2, \ldots) : s_i = t_i \text{ for } i \leq k\}
\end{equation}
with product measure $3^{-k}$.

This set maps to cell $\phi_k(t_1, \ldots, t_k)$, which has Lebesgue measure:
\begin{equation}
\lambda(\phi_k(t_1, \ldots, t_k)) = (3^{-\lfloor k/3 \rfloor})^3 = 3^{-3\lfloor k/3 \rfloor}
\end{equation}

For $k = 3m$, this equals $3^{-3m} = 3^{-k}$, matching the product measure.

For general $k$, the cell has one or two dimensions at finer resolution, but the limiting measure correspondence holds. \qed
\end{proof}

\subsection{Fractal Structure}

\begin{theorem}[Self-Similarity]\label{thm:self-similarity}
The ternary addressing scheme exhibits $3^k$ self-similarity: the structure at any scale is identical to the structure at any other scale.
\end{theorem}

\begin{proof}
Consider a cell $C = \phi_k(t_1, \ldots, t_k)$. The subcells of $C$ are $\{\phi_{k+m}(t_1, \ldots, t_k, s_1, \ldots, s_m) : (s_1, \ldots, s_m) \in \mathbb{T}^m\}$.

The structure of these subcells within $C$ is identical to the structure of level-$m$ cells within $[0,1]^3$, up to scaling by factor $3^{-\lfloor k/3 \rfloor}$.

This self-similarity at all scales is characteristic of fractal structures \citep{falconer2003fractal}. \qed
\end{proof}

\begin{corollary}[Scale Ambiguity]
Given only the local structure of a ternary-addressed region, it is impossible to determine the absolute scale (the value of $k$).
\end{corollary}

\begin{remark}
This scale ambiguity is the mathematical basis for the ``scale ambiguity theorem'' in categorical computing: local and global problems have identical structure, differing only in the number of prefix trits.
\end{remark}

\subsection{Uniqueness and Boundary Representations}

\begin{theorem}[Unique Ternary Expansion]\label{thm:unique-expansion}
Every $\Scoord \in [0,1]^3$ has a unique infinite ternary expansion:
\begin{equation}
\Scoord = (S_k, S_t, S_e) = \left(\sum_{i=1}^\infty \frac{t_{3i}}{3^i}, \sum_{i=1}^\infty \frac{t_{3i+1}}{3^i}, \sum_{i=1}^\infty \frac{t_{3i+2}}{3^i}\right)
\end{equation}
except for boundary points with two representations:
\begin{equation}
S = 0.t_1t_2\ldots t_k222\ldots = 0.t_1t_2\ldots(t_k+1)000\ldots
\end{equation}
\end{theorem}

\begin{proof}
\textbf{Existence:} For any $\Scoord \in [0,1]^3$, Theorem~\ref{thm:address-from-coords} constructs a ternary expansion by iterative refinement. The sequence of cells $C_k$ nests with $\bigcap_{k=0}^\infty C_k = \{\Scoord\}$.

\textbf{Uniqueness (generic case):} Suppose $\Scoord$ has two distinct ternary expansions $\mathbf{t} = (t_1, t_2, \ldots)$ and $\mathbf{t}' = (t'_1, t'_2, \ldots)$ with $\mathbf{t} \neq \mathbf{t}'$.

Let $j$ be the first position where $t_j \neq t'_j$. Then $\phi_j(\mathbf{t})$ and $\phi_j(\mathbf{t}')$ are distinct cells at level $j$, which are disjoint. But $\Scoord \in \phi_j(\mathbf{t})$ and $\Scoord \in \phi_j(\mathbf{t}')$, contradiction.

\textbf{Boundary exception:} Consider a point on a cell boundary, e.g., $S_k = 1/3$. This can be represented as:
\begin{align}
S_k &= 0.1000\ldots = \frac{1}{3} \\
S_k &= 0.0222\ldots = \frac{0}{3} + \frac{2}{9} + \frac{2}{27} + \cdots = \frac{2/9}{1-1/3} = \frac{1}{3}
\end{align}

These are the only non-unique representations, occurring precisely when a coordinate equals $m/3^n$ for integers $m, n$.
\qed
\end{proof}

\begin{corollary}[Almost-Everywhere Uniqueness]
The set of points in $[0,1]^3$ with non-unique ternary expansions has Lebesgue measure zero.
\end{corollary}

\begin{proof}
Boundary points form a countable union of lower-dimensional manifolds (planes, lines, points), each with measure zero in $\mathbb{R}^3$. \qed
\end{proof}
