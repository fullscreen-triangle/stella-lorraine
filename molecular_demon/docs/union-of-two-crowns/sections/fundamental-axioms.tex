\section{Foundational Axioms for Observation}

\subsection{The Axiomatic Approach}

We adopt the axiomatic method: state minimal assumptions explicitly, then derive all consequences through rigorous deduction. This approach provides two advantages. First, it makes logical dependencies transparent—every result traces back to specific axioms. Second, it enables falsification—if any axiom is shown to be false, we know exactly which results depend on it.

We require two axioms. The first establishes that physical systems occupy bounded regions. The second establishes that observation distinguishes among finite alternatives. From these two axioms alone, we derive the complete thermodynamic and statistical mechanical structure developed in subsequent sections.

No additional assumptions about probability, ergodicity, ensembles, or quantum mechanics are required. The structure emerges from geometry and logic.

\subsection{Axiom I: Bounded Phase Space}

\begin{axiom}[Bounded Phase Space]
\label{axiom:bounded}
Every physical system $\Sigma$ observable for finite time $t_{\text{obs}}$ occupies a bounded region of phase space.

Formally: There exist finite constants $L$, $E_{\max}$, and $T$ such that:
\begin{enumerate}[label=(\roman*)]
    \item \textbf{Spatial boundedness:} All position coordinates satisfy $|q_i| \leq L$ for the characteristic length $L < \infty$
    \item \textbf{Energetic boundedness:} Total energy satisfies $E \leq E_{\max} < \infty$
    \item \textbf{Temporal boundedness:} Any distinguishable dynamical process completes within finite time $T < \infty$
\end{enumerate}
\end{axiom}

\begin{remark}[Empirical Basis]
This axiom reflects empirical fact, not a modelling assumption. Every observable system satisfies these conditions:
\begin{itemize}
    \item Gas molecules: bounded by container walls ($L = $ container dimension)
    \item Atoms: bounded by Coulomb potential ($L \sim a_0$, Bohr radius)
    \item Nuclei: bounded by strong force ($L \sim 10^{-15}$ m)
    \item Planetary systems: bounded by gravitational potential ($L \sim $ orbital radius)
    \item Observable universe: bounded by cosmological horizon ($L \sim $ Hubble radius)
\end{itemize}

Unbounded systems—infinite planes, unlimited energy reservoirs, and eternal processes—are mathematical idealisations that never occur in nature. We exclude them.
\end{remark}

\begin{remark}[Logical Relations]
The three boundedness conditions are related through fundamental constraints:

\textbf{Spatial → Momentum:} Position boundedness ($\Delta q \leq L$) and the uncertainty principle $\Delta q \cdot \Delta p \geq \hbar/2$ imply momentum boundedness: $\Delta p \geq \hbar/(2L)$.

\textbf{Momentum → Energy:} Momentum boundedness ($|p| \leq p_{\max}$) implies energy boundedness: $E_{\text{kin}} = p^2/(2m) \leq p_{\max}^2/(2m) = E_{\max}$.

\textbf{Energy → Temporal:} Energy boundedness and the energy-time uncertainty $\Delta E \cdot \Delta t \geq \hbar/2$ imply temporal boundedness: $\Delta t \geq \hbar/(2\Delta E)$.

We state all three explicitly for clarity, though only one is logically independent.
\end{remark}

\begin{remark}[Necessity for Observability]
Boundedness is a precondition for observability. Consider an unbounded system with $E \to \infty$ or $L \to \infty$:
\begin{itemize}
    \item Infinite spatial extent requires infinite time to traverse (even at $c$)
    \item Infinite energy implies infinite degrees of freedom, requiring infinite information to specify
    \item Infinite temporal extent means processes never complete, yielding no definite measurement outcome
\end{itemize}

Systems violating Axiom~\ref{axiom:bounded} are not merely difficult to observe—they are logically unobservable within finite time.
\end{remark}

\begin{definition}[Phase Space Volume]
\label{def:phase_volume}
For a system with $d$ degrees of freedom, the phase space volume is:
\begin{equation}
\Omega = \int_{\mathcal{M}} \prod_{i=1}^{d} dq_i \, dp_i
\end{equation}
where $\mathcal{M} \subset \mathbb{R}^{2d}$ is the accessible region.

By Axiom~\ref{axiom:bounded}, $\Omega < \infty$.
\end{definition}

\begin{definition}[Characteristic Scales]
\label{def:scales}
From boundedness constants, we define characteristic scales:
\begin{align}
\text{Length:} \quad \lambda &= L \\
\text{Momentum:} \quad \pi &= \sqrt{2mE_{\max}} \\
\text{Time:} \quad \tau &= T
\end{align}

For an ideal gas at temperature $T$ in volume $V$:
\begin{align}
\lambda &= V^{1/3} \quad \text{(container size)} \\
\pi &= \sqrt{2mk_BT} \quad \text{(thermal momentum)} \\
\tau &= \lambda/v_{\text{th}} \quad \text{(collision time, } v_{\text{th}} = \sqrt{k_BT/m}\text{)}
\end{align}
\end{definition}

\subsection{Axiom II: Finite Observational Resolution}

\begin{axiom}[Finite Observational Resolution]
\label{axiom:resolution}
Any observation of a physical system distinguishes among a finite number of alternatives.

Formally: For any observable $Q$ and measurement procedure $\mathcal{M}$, there exists a finite set of distinguishable outcomes $\{q_1, q_2, \ldots, q_n\}$ where $n < \infty$.

Equivalently: Any observation partitions phase space $\mathcal{M} \subset \mathbb{R}^{2d}$ into finite cells:
\begin{equation}
\mathcal{M} = \bigcup_{k=1}^{n} C_k
\end{equation}
where:
\begin{enumerate}[label=(\roman*)]
    \item Each $C_k$ is a measurable subset of $\mathcal{M}$
    \item Cells are mutually exclusive: $C_i \cap C_j = \emptyset$ for $i \neq j$
    \item Cells are exhaustive: $\bigcup_{k=1}^{n} C_k = \mathcal{M}$
    \item The number of cells is finite: $n < \infty$
\end{enumerate}
\end{axiom}

\begin{remark}[Physical Basis]
This axiom reflects a fundamental constraint: to distinguish state A from state B requires measurable difference. This difference requires:
\begin{itemize}
    \item Spatial separation: positions differ by at least $\Delta q > 0$
    \item Momentum separation: momenta differ by at least $\Delta p > 0$
    \item Temporal separation: states occur at times differing by at least $\Delta t > 0$
\end{itemize}

With finite resolution ($\Delta q > 0$, $\Delta p > 0$) and bounded phase space (Axiom~\ref{axiom:bounded}), the number of distinguishable states is:
\begin{equation}
n = \frac{\Omega}{\Delta q \cdot \Delta p} < \infty
\end{equation}

This is finite because both numerator and denominator are finite.
\end{remark}

\begin{remark}[Quantum Interpretation]
In quantum mechanics, minimum resolution is set by Planck's constant:
\begin{equation}
\Delta q \cdot \Delta p \geq \hbar
\end{equation}

The minimum phase space cell size is $\hbar^d$ for $d$ degrees of freedom. The maximum number of distinguishable quantum states is:
\begin{equation}
n_{\max} = \frac{\Omega}{\hbar^d}
\end{equation}

This is the quantum density of states. Axiom~\ref{axiom:resolution} does not assume quantum mechanics—it requires only finite resolution. Quantum mechanics provides one specific value.
\end{remark}

\begin{remark}[Classical Limit]
In the classical limit, resolution can be made arbitrarily fine ($\Delta q \to 0$, $\Delta p \to 0$), but remains finite for any actual measurement. The number of distinguishable states grows as resolution improves:
\begin{equation}
n(\Delta q, \Delta p) = \frac{\Omega}{\Delta q \cdot \Delta p} \to \infty \quad \text{as } \Delta q \cdot \Delta p \to 0
\end{equation}

For any fixed resolution, $n$ is finite. Axiom~\ref{axiom:resolution} applies to actual measurements, not idealized limits.
\end{remark}

\begin{definition}[Partition Depth]
\label{def:partition_depth}
The partition depth $n$ is the number of distinguishable cells in a phase space partition:
\begin{equation}
n = |\{C_1, C_2, \ldots, C_n\}|
\end{equation}

For fixed phase space volume $\Omega$ and resolution $\Delta q \cdot \Delta p$:
\begin{equation}
n = \frac{\Omega}{\Delta q \cdot \Delta p}
\end{equation}

Finer resolution (smaller $\Delta q \cdot \Delta p$) yields larger partition depth (larger $n$).
\end{definition}

\begin{definition}[Categorical State]
\label{def:categorical_state}
A categorical state is a phase space cell $C_k$ in a partition. The system is in categorical state $k$ if its phase point lies in $C_k$:
\begin{equation}
\text{System in state } k \iff (q, p) \in C_k
\end{equation}

Two systems are in the same categorical state if their phase points lie in the same cell, even if exact positions differ. This is the coarse-graining inherent in finite-resolution observation.
\end{definition}

\subsection{Logical Independence of Axioms}

The two axioms are logically independent—neither can be derived from the other.

\begin{proposition}[Independence of Axiom~\ref{axiom:bounded}]
\label{prop:independence_I}
Axiom~\ref{axiom:bounded} (bounded phase space) does not imply Axiom~\ref{axiom:resolution} (finite resolution).
\end{proposition}

\begin{proof}
Consider a bounded system (satisfying Axiom~\ref{axiom:bounded}) observed with infinite resolution ($\Delta q \to 0$, $\Delta p \to 0$). Phase space volume $\Omega$ is finite, but the number of distinguishable states is:
\begin{equation}
n = \frac{\Omega}{\Delta q \cdot \Delta p} \to \infty
\end{equation}

This violates Axiom~\ref{axiom:resolution}. Therefore, Axiom~\ref{axiom:bounded} does not imply Axiom~\ref{axiom:resolution}.
\end{proof}

\begin{proposition}[Independence of Axiom~\ref{axiom:resolution}]
\label{prop:independence_II}
Axiom~\ref{axiom:resolution} (finite resolution) does not imply Axiom~\ref{axiom:bounded} (bounded phase space).
\end{proposition}

\begin{proof}
Consider an unbounded system (violating Axiom~\ref{axiom:bounded}) observed with finite resolution. Example: a free particle on an infinite line with position resolution $\Delta q = 1$ m and momentum resolution $\Delta p = 1$ kg·m/s.

Phase space is unbounded ($q \in \mathbb{R}$, $p \in \mathbb{R}$), so $\Omega = \infty$. However, observing only for finite time $t_{\text{obs}}$ distinguishes only finite states:
\begin{equation}
n = \frac{v_{\max} \cdot t_{\text{obs}}}{\Delta q} < \infty
\end{equation}
where $v_{\max}$ is maximum observable velocity.

This satisfies Axiom~\ref{axiom:resolution} but violates Axiom~\ref{axiom:bounded}. Therefore, Axiom~\ref{axiom:resolution} does not imply Axiom~\ref{axiom:bounded}.
\end{proof}

Both axioms are required.

\subsection{Sufficiency of Axioms}

Axioms~\ref{axiom:bounded} and~\ref{axiom:resolution} are sufficient to derive all results in this paper. No additional assumptions are required.

Specifically, we do not assume:
\begin{itemize}
    \item Probability distributions (Maxwell-Boltzmann, canonical ensemble)
    \item Ergodicity (time averages equal ensemble averages)
    \item Equal a priori probabilities (microcanonical postulate)
    \item Quantum mechanics (wave functions, operators, commutation relations)
    \item Specific dynamics (Hamiltonian, Lagrangian, equations of motion)
    \item Thermodynamic postulates (zeroth, first, second, third laws)
\end{itemize}

All emerge as consequences of the two axioms.

\begin{remark}[Methodological Principle]
Throughout this paper: \textit{every result must be derivable from Axioms~\ref{axiom:bounded} and~\ref{axiom:resolution} through explicit logical steps}.

If a result requires additional assumptions, we state them explicitly as lemmas or propositions. If no derivation from the axioms is possible, we state the result as a conjecture requiring empirical validation.

This discipline ensures logical dependencies remain transparent.
\end{remark}

\subsection{Notation and Conventions}

\begin{notation}[Phase Space]
\label{not:phase_space}
\begin{itemize}
    \item $\mathcal{M}$: Phase space (set of all possible states)
    \item $(q, p) \in \mathcal{M}$: Phase point (position $q$, momentum $p$)
    \item $d$: Number of degrees of freedom
    \item $\Omega$: Phase space volume
    \item $L$: Characteristic length scale
    \item $E_{\max}$: Maximum energy
    \item $T$: Characteristic time scale
\end{itemize}
\end{notation}

\begin{notation}[Partitions]
\label{not:partitions}
\begin{itemize}
    \item $\{C_k\}_{k=1}^{n}$: Partition of phase space into $n$ cells
    \item $C_k \subset \mathcal{M}$: The $k$-th cell (categorical state)
    \item $n$: Partition depth (number of cells)
    \item $\Delta q$: Position resolution
    \item $\Delta p$: Momentum resolution
    \item $\tau$: Temporal resolution
\end{itemize}
\end{notation}

\begin{notation}[Thermodynamic Quantities]
\label{not:thermodynamic}
\begin{itemize}
    \item $S$: Entropy
    \item $T$: Temperature
    \item $P$: Pressure
    \item $V$: Volume
    \item $N$: Number of particles
    \item $E$ or $U$: Internal energy
    \item $M$: Number of active categorical states
    \item $k_B$: Boltzmann constant
\end{itemize}
\end{notation}

\begin{convention}[Natural Units]
\label{conv:units}
We use natural units where convenient:
\begin{itemize}
    \item $k_B = 1$ (temperature has units of energy)
    \item $\hbar = 1$ (action is dimensionless)
    \item $c = 1$ (time and length have same units)
\end{itemize}

Physical units are restored in final results for experimental comparison.
\end{convention}

\begin{convention}[Summation]
\label{conv:summation}
Unless otherwise stated:
\begin{itemize}
    \item Sums over states: $\sum_k \equiv \sum_{k=1}^{n}$
    \item Sums over particles: $\sum_i \equiv \sum_{i=1}^{N}$
    \item Integrals over phase space: $\int \equiv \int_{\mathcal{M}}$
\end{itemize}
\end{convention}

\subsection{Scope and Limitations}

\textbf{Within scope:}
\begin{itemize}
    \item Systems satisfying Axioms~\ref{axiom:bounded} and~\ref{axiom:resolution}
    \item Equilibrium thermodynamics (temperature, pressure, entropy, free energy)
    \item Statistical mechanics (partition functions, distributions, fluctuations)
    \item Discrete state structure (quantum numbers, energy levels, selection rules)
    \item Classical mechanics (as limiting case of fine resolution)
\end{itemize}

\textbf{Outside scope:}
\begin{itemize}
    \item Unbounded systems (infinite volume, energy, or time)
    \item Continuous observations (infinite resolution limits)
    \item Non-equilibrium dynamics (time-dependent processes, irreversibility)
    \item Field theories (infinite degrees of freedom)
    \item General relativistic systems (where spacetime geometry is dynamical)
\end{itemize}

Some limitations may be relaxed in future work. We restrict attention to systems where both axioms hold exactly.

\subsection{Relation to Existing Frameworks}

\subsubsection{Statistical Mechanics}

Standard statistical mechanics begins with:
\begin{enumerate}
    \item Hamiltonian $H(q, p)$ specifying dynamics
    \item Probability distribution $\rho(q, p)$ over phase space
    \item Ergodic hypothesis (time averages = ensemble averages)
    \item Equal a priori probability postulate (microcanonical ensemble)
\end{enumerate}

Thermodynamic quantities follow: entropy $S = -k_B \sum_k p_k \ln p_k$, temperature $1/T = \partial S/\partial E$.

Our approach inverts this logic:
\begin{enumerate}
    \item Begin with geometric constraints (Axioms~\ref{axiom:bounded} and~\ref{axiom:resolution})
    \item Derive discrete states (partition cells)
    \item Derive entropy from state counting ($S = k_B \ln n$)
    \item Derive probability distributions as maximum entropy consequences
\end{enumerate}

No Hamiltonian, probability postulates, or ergodicity assumptions required. These emerge as derived concepts.

\subsubsection{Quantum Mechanics}

Standard quantum mechanics begins with:
\begin{enumerate}
    \item Hilbert space $\mathcal{H}$ of state vectors $|\psi\rangle$
    \item Operators $\hat{O}$ representing observables
    \item Schrödinger equation $i\hbar \partial_t |\psi\rangle = \hat{H}|\psi\rangle$
    \item Born rule $P(a) = |\langle a|\psi\rangle|^2$
\end{enumerate}

Discrete energy levels, quantum numbers, and selection rules follow.

Our approach derives these structures without assuming Hilbert spaces:
\begin{enumerate}
    \item Discrete states arise from finite partition depth (Axiom~\ref{axiom:resolution})
    \item Quantum numbers arise from nested partition geometry (Section 4)
    \item Energy levels arise from partition coordinate structure (Section 5)
    \item Selection rules arise from transition continuity constraints (Section 6)
\end{enumerate}

Hilbert space formalism can be recovered as a convenient representation, but it is not fundamental.

\subsubsection{Thermodynamics}

Classical thermodynamics begins with empirical laws:
\begin{enumerate}
    \item Zeroth law (transitivity of thermal equilibrium)
    \item First law (energy conservation: $dU = \delta Q - \delta W$)
    \item Second law (entropy increase: $dS \geq \delta Q/T$)
    \item Third law (entropy vanishes at $T=0$: $S(T=0) = 0$)
\end{enumerate}

Our approach derives these from axioms:
\begin{enumerate}
    \item Zeroth law: from transitivity of partition depth matching (Section 5)
    \item First law: from energy as a partition coordinate (Section 5)
    \item Second law: from partition depth being non-decreasing under coarse-graining (Section 7)
    \item Third law: from minimum partition depth $n = 1$ at zero energy (Section 5)
\end{enumerate}

Thermodynamics is not separate theory—it is the macroscopic limit of partition geometry.

\subsection{Summary}

We have established two axioms:

\textbf{Axiom~\ref{axiom:bounded} (Bounded Phase Space):} Every observable system occupies finite phase space region with volume $\Omega < \infty$.

\textbf{Axiom~\ref{axiom:resolution} (Finite Resolution):} Every observation distinguishes among finite number $n < \infty$ of categorical states.

These axioms are:
\begin{itemize}
    \item \textbf{Empirically grounded:} All observed systems satisfy them
    \item \textbf{Logically independent:} Neither implies the other
    \item \textbf{Sufficient:} All results derive from them
    \item \textbf{Minimal:} No additional assumptions required
\end{itemize}

Subsequent sections develop mathematical consequences. We derive:
\begin{itemize}
    \item Triple equivalence (oscillatory, categorical, partition descriptions)
    \item Entropy and its equivalence across descriptions
    \item Thermodynamic variables (temperature, pressure, free energy)
    \item Discrete state structure (quantum numbers, energy levels)
    \item Statistical distributions (Maxwell-Boltzmann, Fermi-Dirac, Bose-Einstein)
\end{itemize}

All from two axioms. No additional postulates. This is the power of the axiomatic method: minimal assumptions, maximal consequences.
