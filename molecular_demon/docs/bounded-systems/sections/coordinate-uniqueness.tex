\section{Coordinate Uniqueness: The Exclusion Principle}
\label{sec:coordinate_uniqueness}

We prove that no two categorical states can share identical partition coordinates. This \emph{exclusion principle} is a fundamental consequence of categorical distinguishability.

\subsection{Categorical Distinguishability}

\begin{axiom}[Identity of Indiscernibles]
\label{ax:identity_indiscernibles}
Two categorical states are identical if and only if they have identical partition coordinates:
\begin{equation}
    S_1 = S_2 \iff (n_1, l_1, m_1, s_1) = (n_2, l_2, m_2, s_2)
\end{equation}
\end{axiom}

\begin{theorem}[Coordinate-State Bijection]
\label{thm:coordinate_bijection}
There is a one-to-one correspondence between valid partition coordinates and categorical states.
\end{theorem}

\begin{proof}
\textbf{Surjectivity}: Every categorical state has a partition coordinate (Theorem~\ref{thm:completeness}).

\textbf{Injectivity}: Suppose two states $S_1, S_2$ have the same coordinate $(n, l, m, s)$. Then they occupy the same partition depth, have the same boundary complexity, the same orientation, and the same chirality. By Axiom~\ref{ax:identity_indiscernibles}, $S_1 = S_2$.
\end{proof}

\subsection{The Exclusion Principle}

\begin{theorem}[Exclusion Principle]
\label{thm:exclusion_principle}
No two distinct categorical states can have identical partition coordinates:
\begin{equation}
    S_1 \neq S_2 \implies (n_1, l_1, m_1, s_1) \neq (n_2, l_2, m_2, s_2)
\end{equation}
Equivalently: each coordinate can be occupied by at most one state.
\end{theorem}

\begin{proof}
This is the contrapositive of the injectivity statement in Theorem~\ref{thm:coordinate_bijection}. If two states are distinct, they must have different coordinates (otherwise they would be identical by the Identity of Indiscernibles).
\end{proof}

\subsection{Consequences of Exclusion}

\begin{corollary}[Maximum Occupancy]
\label{cor:maximum_occupancy}
The maximum number of states at partition depth $n$ is exactly $2n^2$---no more states can be accommodated because all coordinates would be occupied.
\end{corollary}

\begin{corollary}[Filling Necessity]
\label{cor:filling_necessity}
When adding states to a system, each new state must occupy a previously unoccupied coordinate. This forces the filling order derived in Section~\ref{sec:energy_ordering}.
\end{corollary}

\begin{corollary}[Degeneracy Pressure]
\label{cor:degeneracy_pressure}
In a bounded region with many states, the exclusion principle creates an effective ``pressure'' that resists compression---states cannot be squeezed into already-occupied coordinates.
\end{corollary}

\subsection{Mathematical Formulation}

\begin{definition}[Occupation Number]
\label{def:occupation_number}
The \emph{occupation number} $N_{(n,l,m,s)}$ for coordinate $(n, l, m, s)$ is:
\begin{equation}
    N_{(n,l,m,s)} \in \{0, 1\}
\end{equation}
where 0 indicates unoccupied and 1 indicates occupied.
\end{definition}

\begin{theorem}[Occupation Constraint]
\label{thm:occupation_constraint}
For categorical states satisfying the exclusion principle:
\begin{equation}
    \sum_{(n,l,m,s)} N_{(n,l,m,s)}^2 = \sum_{(n,l,m,s)} N_{(n,l,m,s)}
\end{equation}
This is equivalent to requiring $N \in \{0, 1\}$ for all coordinates.
\end{theorem}

\begin{proof}
If $N \in \{0, 1\}$, then $N^2 = N$ for all occupation numbers, so the sums are equal.

Conversely, if the sums are equal, then $\sum N(N-1) = 0$. Since $N(N-1) \geq 0$ for $N \geq 0$, each term must vanish, requiring $N \in \{0, 1\}$.
\end{proof}

\subsection{Antisymmetry}

\begin{theorem}[State Antisymmetry]
\label{thm:antisymmetry}
A system of multiple categorical states is described by an antisymmetric function that changes sign when any two coordinates are exchanged:
\begin{equation}
    \Psi(\ldots, (n_i, l_i, m_i, s_i), \ldots, (n_j, l_j, m_j, s_j), \ldots) = -\Psi(\ldots, (n_j, l_j, m_j, s_j), \ldots, (n_i, l_i, m_i, s_i), \ldots)
\end{equation}
\end{theorem}

\begin{proof}
Antisymmetry ensures that $\Psi = 0$ when any two coordinates are identical (since $\Psi = -\Psi$ implies $\Psi = 0$). This enforces the exclusion principle at the level of the state function.
\end{proof}

\subsection{Connection to Chirality}

\begin{theorem}[Half-Integer Chirality and Exclusion]
\label{thm:chirality_exclusion}
The half-integer chirality values $s = \pm\frac{1}{2}$ are directly connected to the exclusion principle. States with half-integer chirality obey exclusion; states with integer chirality do not.
\end{theorem}

\begin{proof}
Under exchange of two states, the total state function acquires a phase $e^{i\pi(2s_1)(2s_2)}$. For half-integer $s$, this is $e^{i\pi} = -1$ (antisymmetric). For integer $s$, this is $e^{i 2\pi} = +1$ (symmetric).

Antisymmetry (half-integer chirality) enforces exclusion. Symmetry (integer chirality) allows multiple occupation.
\end{proof}

\begin{remark}[Structural Similarity]
The exclusion principle derived here has the same mathematical form as the Pauli exclusion principle of quantum mechanics:
\begin{itemize}
    \item No two fermions (half-integer spin) can have identical quantum numbers
    \item The occupation number is restricted to $\{0, 1\}$
    \item Multi-particle states are described by antisymmetric wave functions
\end{itemize}
The connection between half-integer chirality and exclusion mirrors the connection between half-integer spin and fermionic statistics. This suggests that the Pauli principle may be a consequence of categorical partitioning in bounded phase space, with spin being the physical manifestation of boundary chirality.
\end{remark}

