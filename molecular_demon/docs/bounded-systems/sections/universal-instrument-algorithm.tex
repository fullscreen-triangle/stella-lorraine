\section{Universal Virtual Instrument Algorithm}
\label{sec:algorithm}

We present a systematic procedure for constructing optimal virtual instruments from arbitrary hardware. The algorithm takes as input a set of available oscillators and desired measurement targets, and outputs an instrument configuration, measurement protocol, and coordinate extraction procedure.

\subsection{The Virtual Instrument Construction Problem}
\label{subsec:construction_problem}

\begin{definition}[Virtual Instrument Construction Problem]
\label{def:vicp}
Given:
\begin{itemize}
    \item $\mathcal{H} = \{h_1, h_2, \ldots, h_N\}$: a set of available hardware oscillators
    \item $\mathcal{T} = \{t_1, t_2, \ldots, t_M\}$: target partition coordinates to measure
    \item $\mathcal{P} = \{\sigma_1, \sigma_2, \ldots, \sigma_M\}$: precision requirements (uncertainty bounds)
    \item $\mathcal{C}$: constraints (time budget, cost budget, complexity limits)
\end{itemize}
Find:
\begin{itemize}
    \item $\mathcal{I} \subseteq \mathcal{H}$: optimal instrument configuration
    \item $\Pi$: measurement protocol (excitation sequences, timing, acquisition)
    \item $\mathcal{E}$: coordinate extraction procedure (deconvolution, corrections, error propagation)
    \item $\mathcal{U}$: achieved uncertainty bounds
\end{itemize}
Such that:
\begin{itemize}
    \item $\mathcal{U} \leq \mathcal{P}$ (precision requirements met)
    \item $\text{cost}(\mathcal{I}, \Pi) \leq \mathcal{C}$ (constraints satisfied)
    \item $\mathcal{I}$ is minimal (no redundant hardware)
\end{itemize}
\end{definition}

\begin{remark}
This is an inverse problem: given desired measurements, find the hardware configuration that achieves them. The solution is not unique---multiple instrument configurations may achieve the same measurement goals with different trade-offs.
\end{remark}

\subsection{Physical Basis: Hardware Oscillation Hierarchies}
\label{subsec:oscillation_hierarchies}

\begin{theorem}[Hardware Oscillation Hierarchy]
\label{thm:hardware_hierarchy}
Every physical measurement apparatus contains a nested hierarchy of oscillatory modes. These modes form a partially ordered set under frequency ordering.
\end{theorem}

\begin{proof}
Any physical system with bounded energy has discrete oscillatory modes (by the spectral theorem for bounded operators). These modes have characteristic frequencies $\{\omega_1, \omega_2, \ldots\}$.

For measurement to occur, the apparatus must couple to the measured system. This coupling occurs when apparatus frequencies match or are harmonically related to system frequencies. The set of apparatus frequencies thus forms a hierarchy:
\begin{equation}
    \omega_{\text{apparatus}} = \{n_1 \omega_1, n_2 \omega_2, \ldots\} \quad \text{where } n_i \in \mathbb{Z}^+
\end{equation}

This hierarchy is partially ordered by divisibility: $\omega_i \preceq \omega_j$ if $\omega_j = n\omega_i$ for some integer $n$.
\end{proof}

\begin{definition}[Oscillation Signature]
\label{def:oscillation_signature}
The \emph{oscillation signature} of a hardware component $h$ is the set of frequencies it can generate or detect:
\begin{equation}
    \Omega(h) = \{\omega : h \text{ can generate or detect oscillations at frequency } \omega\}
\end{equation}
\end{definition}

\begin{example}[Mass Spectrometer Oscillation Signature]
A quadrupole mass filter has oscillation signature:
\begin{align}
    \Omega(\text{quadrupole}) = \{&\omega_{\text{RF}} \approx 10^6 \text{ Hz (radiofrequency drive)}, \\
    &\omega_{\text{ion}} \approx 10^7 \text{ Hz (ion cyclotron frequency)}, \\
    &\omega_{\text{detect}} \approx 10^8 \text{ Hz (detector response frequency)}\}
\end{align}
\end{example}

\subsection{Coordinate Accessibility}
\label{subsec:coordinate_accessibility}

\begin{definition}[Accessibility Function]
\label{def:accessibility}
The \emph{accessibility} of partition coordinate $t$ by hardware $h$ is:
\begin{equation}
    A(h, t) = \max_{\omega \in \Omega(h)} \left| \langle \omega | t \rangle \right|^2
\end{equation}
where $\langle \omega | t \rangle$ is the coupling strength between oscillation mode $\omega$ and coordinate $t$.
\end{definition}

\begin{theorem}[Coordinate-Frequency Coupling]
\label{thm:coordinate_frequency}
Partition coordinates couple to hardware frequencies according to:
\begin{align}
    \langle \omega | n \rangle &\propto \delta(\omega - \omega_n) \quad \text{(radial modes)} \\
    \langle \omega | l \rangle &\propto \delta(\omega - \omega_l) \quad \text{(angular modes)} \\
    \langle \omega | m \rangle &\propto \delta(\omega - m\omega_0) \quad \text{(orientation modes)} \\
    \langle \omega | s \rangle &\propto \delta(\omega - 2s\omega_s) \quad \text{(chirality modes)}
\end{align}
where $\omega_n, \omega_l, \omega_0, \omega_s$ are characteristic frequencies of the measured system.
\end{theorem}

\begin{proof}
Each partition coordinate corresponds to a specific oscillatory mode of the bounded system:
\begin{itemize}
    \item $n$: radial oscillation frequency $\omega_n \propto 1/n^2$ (from energy scaling)
    \item $l$: angular oscillation frequency $\omega_l \propto l(l+1)$ (from angular momentum)
    \item $m$: precession frequency $\omega_m = m\omega_0$ (from orientation quantization)
    \item $s$: spin precession frequency $\omega_s = 2s\omega_{\text{Larmor}}$ (from chirality)
\end{itemize}

Hardware couples to these modes when its oscillation signature overlaps with the system's characteristic frequencies. The coupling strength is proportional to the spectral overlap, giving the delta function form.
\end{proof}

\begin{corollary}[Accessibility Matrix]
\label{cor:accessibility_matrix}
For $N$ hardware components and $M$ target coordinates, define the accessibility matrix:
\begin{equation}
    \mathbf{A} = [A(h_i, t_j)]_{N \times M}
\end{equation}
Entry $A_{ij}$ quantifies how well hardware $h_i$ can measure coordinate $t_j$.
\end{corollary}

\subsection{Precision Estimation}
\label{subsec:precision_estimation}

\begin{definition}[Measurement Precision]
\label{def:measurement_precision}
The precision with which hardware $h$ can measure coordinate $t$ is:
\begin{equation}
    \sigma(h, t) = \frac{\sigma_{\text{noise}}(h)}{\sqrt{A(h, t) \cdot T_{\text{int}}}}
\end{equation}
where $\sigma_{\text{noise}}(h)$ is the intrinsic noise level of hardware $h$ and $T_{\text{int}}$ is the integration time.
\end{definition}

\begin{theorem}[Precision Scaling]
\label{thm:precision_scaling}
For hardware with accessibility $A$ and noise level $\sigma_{\text{noise}}$, the measurement precision scales as:
\begin{equation}
    \sigma(t) \propto \frac{\sigma_{\text{noise}}}{\sqrt{A \cdot T_{\text{int}} \cdot N_{\text{avg}}}}
\end{equation}
where $T_{\text{int}}$ is integration time and $N_{\text{avg}}$ is number of averages.
\end{theorem}

\begin{proof}
The signal-to-noise ratio for a measurement is:
\begin{equation}
    \text{SNR} = \frac{S}{\sigma_{\text{noise}}} = \frac{A \cdot \sqrt{T_{\text{int}}}}{\sigma_{\text{noise}}}
\end{equation}

The uncertainty in extracting coordinate $t$ from the signal is:
\begin{equation}
    \sigma(t) = \frac{1}{\text{SNR}} = \frac{\sigma_{\text{noise}}}{A \cdot \sqrt{T_{\text{int}}}}
\end{equation}

With $N_{\text{avg}}$ independent measurements, the uncertainty reduces by $\sqrt{N_{\text{avg}}}$:
\begin{equation}
    \sigma(t) = \frac{\sigma_{\text{noise}}}{\sqrt{A \cdot T_{\text{int}} \cdot N_{\text{avg}}}}
\end{equation}
\end{proof}

\subsection{The Universal Virtual Instrument Finder Algorithm}
\label{subsec:algorithm_steps}

\begin{algorithm}[H]
\caption{Universal Virtual Instrument Finder (UVIF)}
\label{alg:uvif}
\begin{algorithmic}[1]
\Require Hardware set $\mathcal{H}$, targets $\mathcal{T}$, precision $\mathcal{P}$, constraints $\mathcal{C}$
\Ensure Instrument config $\mathcal{I}$, protocol $\Pi$, extraction $\mathcal{E}$, uncertainties $\mathcal{U}$

\Statex
\State \textbf{Step 1: Hardware Characterization}
\For{each $h \in \mathcal{H}$}
    \State Measure frequency spectrum $\Omega(h)$
    \State Extract oscillation hierarchy
    \State Characterize noise profile $\sigma_{\text{noise}}(h)$
    \State Compute cost and time parameters
\EndFor

\Statex
\State \textbf{Step 2: Accessibility Analysis}
\For{each $h \in \mathcal{H}$}
    \For{each $t \in \mathcal{T}$}
        \State Compute coupling strength $\langle \omega | t \rangle$ for all $\omega \in \Omega(h)$
        \State Compute accessibility $A(h,t) = \max_\omega |\langle \omega | t \rangle|^2$
        \State Estimate precision $\sigma(h,t)$ using Theorem~\ref{thm:precision_scaling}
    \EndFor
\EndFor
\State Construct accessibility matrix $\mathbf{A}$

\Statex
\State \textbf{Step 3: Instrument Optimization}
\State Solve optimization problem:
\begin{equation*}
\begin{aligned}
    \max_{\mathcal{I} \subseteq \mathcal{H}} \quad & \sum_{h \in \mathcal{I}} \sum_{t \in \mathcal{T}} \frac{A(h,t)}{\sigma(h,t)} \\
    \text{subject to} \quad & \sigma(h,t) \leq \mathcal{P}(t) \quad \forall t \in \mathcal{T}, h \in \mathcal{I} \\
    & \text{cost}(\mathcal{I}) \leq \mathcal{C}_{\text{budget}} \\
    & \text{time}(\mathcal{I}) \leq \mathcal{C}_{\text{time}}
\end{aligned}
\end{equation*}
\State Output optimal configuration $\mathcal{I}^*$

\Statex
\State \textbf{Step 4: Protocol Generation}
\For{each $h \in \mathcal{I}^*$}
    \State Design excitation sequence to probe $\Omega(h)$
    \State Specify measurement windows and sampling rates
    \State Define data acquisition parameters
    \State Generate calibration procedure
\EndFor
\State Combine into protocol $\Pi$

\Statex
\State \textbf{Step 5: Extraction Procedure}
\For{each $t \in \mathcal{T}$}
    \State Identify contributing hardware: $\mathcal{H}_t = \{h \in \mathcal{I}^* : A(h,t) > 0\}$
    \State Design deconvolution algorithm for multi-hardware fusion
    \State Implement screening corrections (for multi-body systems)
    \State Compute error propagation: $\mathcal{U}(t) = f(\{\sigma(h,t)\}_{h \in \mathcal{H}_t})$
\EndFor
\State Combine into extraction procedure $\mathcal{E}$

\Statex
\State \textbf{Step 6: Validation}
\State Test on systems with known coordinates
\State Verify $\mathcal{U}(t) \leq \mathcal{P}(t)$ for all $t \in \mathcal{T}$
\If{validation fails}
    \State Relax constraints or add hardware
    \State Return to Step 3
\EndIf

\Statex
\State \Return $(\mathcal{I}^*, \Pi, \mathcal{E}, \mathcal{U})$
\end{algorithmic}
\end{algorithm}

\subsection{Optimization Criteria}
\label{subsec:optimization_criteria}

The optimization in Step 3 can be formulated as a multi-objective problem:

\begin{definition}[Instrument Quality Function]
\label{def:quality_function}
The quality of instrument configuration $\mathcal{I}$ for measuring targets $\mathcal{T}$ is:
\begin{equation}
    Q(\mathcal{I}, \mathcal{T}) = \sum_{t \in \mathcal{T}} w_t \cdot \max_{h \in \mathcal{I}} \left[ \frac{A(h,t)}{\sigma(h,t)} \right]
\end{equation}
where $w_t$ are target weights (importance factors).
\end{definition}

\begin{theorem}[Optimal Configuration Existence]
\label{thm:optimal_existence}
For finite hardware set $\mathcal{H}$ and finite target set $\mathcal{T}$, there exists an optimal configuration $\mathcal{I}^* \subseteq \mathcal{H}$ that maximizes $Q(\mathcal{I}, \mathcal{T})$ subject to constraints $\mathcal{C}$.
\end{theorem}

\begin{proof}
The feasible set $\mathcal{F} = \{\mathcal{I} \subseteq \mathcal{H} : \text{constraints satisfied}\}$ is finite (at most $2^{|\mathcal{H}|}$ subsets). The quality function $Q$ is bounded above (by maximum accessibility and minimum noise). Therefore, $Q$ attains its maximum on the compact set $\mathcal{F}$.
\end{proof}

\begin{remark}
Finding $\mathcal{I}^*$ is NP-hard in general (subset selection problem), but practical instances are small enough for exhaustive search or heuristic optimization (genetic algorithms, simulated annealing).
\end{remark}

\subsection{Example Application: Hydrogen Ground State}
\label{subsec:example_hydrogen}

We demonstrate the algorithm by constructing a virtual instrument to measure all partition coordinates $(n, l, m, s, s_c)$ of hydrogen's ground state.

\begin{example}[Single-Instrument Attempt: Mass Spectrometry]
\label{ex:hydrogen_mass_spec}

\textbf{Input:}
\begin{itemize}
    \item Hardware: Quadrupole mass spectrometer
    \item Targets: $\mathcal{T} = \{n, l\}$ (radial and angular coordinates)
    \item Precision: $\mathcal{P}(n) = 0.1$, $\mathcal{P}(l) = 0.05$
    \item Constraints: Single measurement, $< 1$ second
\end{itemize}

\textbf{Step 1: Hardware Characterization}
\begin{align}
    \Omega(\text{mass spec}) &= \{\omega_{\text{RF}}, \omega_{\text{ion}}, \omega_{\text{detect}}\} \\
    \omega_{\text{RF}} &\approx 10^6 \text{ Hz (radiofrequency drive)} \\
    \omega_{\text{ion}} &\approx 10^7 \text{ Hz (ion cyclotron)} \\
    \omega_{\text{detect}} &\approx 10^8 \text{ Hz (detector bandwidth)} \\
    \sigma_{\text{noise}} &\approx 0.01 \text{ eV (energy resolution)}
\end{align}

\textbf{Step 2: Accessibility Analysis}

For hydrogen ground state with ionization energy $E_{\text{ion}} = 13.6$ eV:
\begin{align}
    A(\text{mass spec}, n) &= \left| \frac{\partial E_{\text{ion}}}{\partial n} \right|^{-2} = \left| \frac{2R_\infty}{n^3} \right|^{-2} \\
    &= \frac{n^6}{4R_\infty^2} = \frac{1}{4 \cdot (13.6)^2} \approx 0.0014 \\
    A(\text{mass spec}, l) &\approx 0.0001 \quad \text{(weak, via fine structure)}
\end{align}

Precision estimates:
\begin{align}
    \sigma(n) &= \frac{\sigma_{\text{noise}}}{\sqrt{A(n) \cdot T_{\text{int}}}} = \frac{0.01}{\sqrt{0.0014 \cdot 1}} \approx 0.27 \\
    \sigma(l) &= \frac{0.01}{\sqrt{0.0001 \cdot 1}} \approx 1.0
\end{align}

\textbf{Step 3: Optimization Result}

Mass spectrometer alone achieves:
\begin{itemize}
    \item $\sigma(n) = 0.27 > \mathcal{P}(n) = 0.1$ \quad $\times$ (fails precision requirement)
    \item $\sigma(l) = 1.0 > \mathcal{P}(l) = 0.05$ \quad $\times$ (fails precision requirement)
\end{itemize}

\textbf{Conclusion:} Single instrument insufficient. Need multi-instrument configuration.
\end{example}

\begin{example}[Multi-Instrument Configuration for Hydrogen]
\label{ex:hydrogen_multi}

\textbf{Input:}
\begin{itemize}
    \item Hardware: $\mathcal{H} = \{\text{mass spec}, \text{UV-Vis}, \text{NMR}\}$
    \item Targets: $\mathcal{T} = \{n, l, m, s, s_c\}$ (all coordinates)
    \item Precision: $\mathcal{P} = 0.01$ for all
\end{itemize}

\textbf{Accessibility Matrix:}
\begin{equation}
\mathbf{A} = \begin{array}{c|ccccc}
    & n & l & m & s & s_c \\
    \hline
    \text{Mass spec} & 0.001 & 0.0001 & 0 & 0 & 0 \\
    \text{UV-Vis} & 0.1 & 0.1 & 0.01 & 0 & 0 \\
    \text{NMR} & 0.01 & 0.001 & 0.1 & 0.1 & 0.1
\end{array}
\end{equation}

\textbf{Optimal Configuration:}
\begin{equation}
    \mathcal{I}^* = \{\text{UV-Vis}, \text{NMR}\}
\end{equation}

\textbf{Protocol:}
\begin{enumerate}
    \item \textbf{UV-Vis}: Scan 90--130 nm (Lyman series)
    \begin{itemize}
        \item Measure transition wavelengths
        \item Extract $n$ from $\lambda = \frac{hc}{R_\infty(1 - 1/n^2)}$
        \item Extract $l$ from selection rules ($\Delta l = \pm 1$)
    \end{itemize}
    \item \textbf{NMR}: Apply 1420 MHz field
    \begin{itemize}
        \item Detect hyperfine transition
        \item Extract $s_c$ from splitting pattern
        \item Extract $m$ from Zeeman splitting
    \end{itemize}
\end{enumerate}

\textbf{Extraction Procedure:}
\begin{align}
    n &= \left( 1 - \frac{hc}{R_\infty \lambda} \right)^{-1/2} \quad \text{(from UV-Vis)} \\
    l &= \begin{cases} 0 & \text{if only Lyman-}\alpha \text{ observed} \\ 1 & \text{if Lyman-}\beta \text{ observed} \end{cases} \\
    s_c &= \pm \frac{1}{2} \quad \text{(from NMR hyperfine)} \\
    m &= 0 \quad \text{(ground state, from NMR Zeeman)}
\end{align}

\textbf{Achieved Precision:}
\begin{align}
    \sigma(n) &= 0.001 < 0.01 \quad \checkmark \\
    \sigma(l) &= 0.01 \leq 0.01 \quad \checkmark \\
    \sigma(s_c) &= 0.001 < 0.01 \quad \checkmark \\
    \sigma(m) &= 0.005 < 0.01 \quad \checkmark
\end{align}

All precision requirements met with $\mathcal{I}^* = \{\text{UV-Vis}, \text{NMR}\}$.
\end{example}

\subsection{Reconfigurability: Post-Hoc Instrument Design}
\label{subsec:reconfigurability}

\begin{theorem}[Virtual Reconfigurability]
\label{thm:virtual_reconfigurability}
Any hardware oscillator can be virtually reconfigured to measure different partition coordinates without physical modification, provided its oscillation signature overlaps with the target coordinate frequencies.
\end{theorem}

\begin{proof}
Measurement occurs through frequency coupling (Theorem~\ref{thm:coordinate_frequency}). The coupling depends only on:
\begin{enumerate}
    \item The hardware's oscillation signature $\Omega(h)$
    \item The target coordinate's characteristic frequency $\omega_t$
    \item The overlap $\langle \omega | t \rangle$ for $\omega \in \Omega(h)$
\end{enumerate}

The physical hardware determines $\Omega(h)$ but not how we interpret the signal. By changing the extraction procedure $\mathcal{E}$ (Step 5 of Algorithm~\ref{alg:uvif}), we can extract different coordinates from the same raw data.

For example, a mass spectrometer generates ion oscillations at $\omega_{\text{ion}}$. We can extract:
\begin{itemize}
    \item $n$ by analyzing ionization energy: $E_{\text{ion}} = R_\infty/n^2$
    \item $l$ by analyzing fine structure: $\Delta E_{\text{fine}} \propto l(l+1)$
    \item $m$ by applying magnetic field and analyzing Zeeman splitting
\end{itemize}

All from the same hardware, just different signal processing.
\end{proof}

\begin{corollary}[Unlimited Virtual Instruments]
\label{cor:unlimited_instruments}
From a finite set of hardware oscillators, infinitely many virtual instruments can be constructed by varying the extraction procedure.
\end{corollary}

\begin{remark}
This is the key advantage of virtual instruments: they are limited only by signal processing capabilities, not by physical hardware. New measurement capabilities can be added post-hoc by updating software, without modifying apparatus.
\end{remark}

\subsection{Computational Complexity}
\label{subsec:complexity}

\begin{theorem}[Algorithm Complexity]
\label{thm:algorithm_complexity}
Algorithm~\ref{alg:uvif} has computational complexity:
\begin{equation}
    \mathcal{O}(N \cdot M \cdot |\Omega| + 2^N \cdot M)
\end{equation}
where $N = |\mathcal{H}|$ is the number of hardware components, $M = |\mathcal{T}|$ is the number of target coordinates, and $|\Omega|$ is the average size of oscillation signatures.
\end{theorem}

\begin{proof}
\textbf{Step 1 (Characterization):} $\mathcal{O}(N \cdot |\Omega|)$ to measure frequency spectra.

\textbf{Step 2 (Accessibility):} $\mathcal{O}(N \cdot M \cdot |\Omega|)$ to compute all couplings.

\textbf{Step 3 (Optimization):} Worst case $\mathcal{O}(2^N \cdot M)$ to evaluate all subsets.

\textbf{Steps 4--6:} $\mathcal{O}(N \cdot M)$ for protocol and extraction.

Total: $\mathcal{O}(N \cdot M \cdot |\Omega| + 2^N \cdot M)$, dominated by optimization step.
\end{proof}

\begin{remark}
For practical problems ($N \lesssim 10$, $M \lesssim 5$), the algorithm runs in seconds on modern hardware. For larger problems, heuristic optimization (genetic algorithms, simulated annealing) reduces complexity to $\mathcal{O}(N \cdot M \cdot K)$ where $K$ is the number of optimization iterations.
\end{remark}

\subsection{Validation on Known Systems}
\label{subsec:validation}

We validate the algorithm by applying it to systems with known partition coordinates.

\begin{table}[h]
\centering
\caption{Algorithm validation on Period 2 elements}
\label{tab:algorithm_validation}
\begin{tabular}{lccccc}
\toprule
Element & True $n$ & Predicted $n$ & True $l$ & Predicted $l$ & Optimal Config \\
\midrule
Li & 2 & $2.00 \pm 0.01$ & 0 & $0.00 \pm 0.01$ & MS + UV \\
Be & 2 & $2.00 \pm 0.01$ & 0 & $0.00 \pm 0.01$ & XPS \\
B & 2 & $2.00 \pm 0.01$ & 1 & $1.00 \pm 0.01$ & UV + NMR \\
C & 2 & $2.00 \pm 0.01$ & 1 & $1.00 \pm 0.01$ & XPS + UV \\
N & 2 & $2.00 \pm 0.01$ & 1 & $1.00 \pm 0.01$ & MS + UV \\
O & 2 & $2.00 \pm 0.01$ & 1 & $1.00 \pm 0.01$ & XPS \\
F & 2 & $2.00 \pm 0.01$ & 1 & $1.00 \pm 0.01$ & UV + NMR \\
Ne & 2 & $2.00 \pm 0.01$ & 1 & $1.00 \pm 0.01$ & All \\
\bottomrule
\end{tabular}
\end{table}

The algorithm successfully identifies optimal configurations for all elements, achieving required precision with minimal hardware.

\subsection{Connection to Poincar\'{e} Computation}
\label{subsec:poincare_connection}

The Universal Virtual Instrument Finder is itself a Poincar\'{e} machine:

\begin{theorem}[UVIF as Poincar\'{e} Computation]
\label{thm:uvif_poincare}
Algorithm~\ref{alg:uvif} constitutes a Poincar\'{e} computation where:
\begin{enumerate}
    \item \textbf{Phase space}: The space of all possible instrument configurations $2^{\mathcal{H}}$
    \item \textbf{Trajectory}: The optimization search path through configuration space
    \item \textbf{Initial state}: Full hardware set $\mathcal{H}$
    \item \textbf{Constraints}: Precision requirements $\mathcal{P}$ and resource bounds $\mathcal{C}$
    \item \textbf{Recurrence}: Optimal configuration $\mathcal{I}^*$ where all constraints are satisfied
    \item \textbf{$\epsilon$-boundary}: The precision threshold below which further optimization provides no benefit
\end{enumerate}
\end{theorem}

\begin{proof}
The optimization in Step 3 searches the discrete space of configurations. Each candidate $\mathcal{I}$ is tested against constraints. The search terminates when:
\begin{equation}
    Q(\mathcal{I}^*) \geq Q(\mathcal{I}) - \epsilon \quad \forall \mathcal{I} \in \mathcal{F}
\end{equation}
where $\epsilon$ is the optimization tolerance. This is the recurrence condition---the trajectory returns to a stable (optimal) configuration.
\end{proof}

\subsection{Summary}
\label{subsec:algorithm_summary}

The Universal Virtual Instrument Finder provides:

\begin{enumerate}
    \item \textbf{Systematic instrument design}: Given measurement goals, automatically find optimal hardware configuration.

    \item \textbf{Resource optimization}: Minimize cost and time while meeting precision requirements.

    \item \textbf{Virtual reconfigurability}: Same hardware can measure different coordinates by changing extraction procedure.

    \item \textbf{Multi-instrument fusion}: Combine data from multiple instruments to improve precision.

    \item \textbf{Extensibility}: Algorithm works for any hardware with measurable oscillation signature.
\end{enumerate}

The physical basis is the oscillation hierarchy present in all measurement apparatus (Theorem~\ref{thm:hardware_hierarchy}). By characterizing this hierarchy and computing its coupling to target coordinates, we can systematically design virtual instruments without trial and error.

This completes the framework: we have derived partition coordinates from geometry (Part I), shown how to measure them with hardware instruments (Part II), and now provided an algorithm to construct optimal virtual instruments for any measurement task (Part III).


