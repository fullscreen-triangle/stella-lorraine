\section{Partition Coordinates in Bounded Phase Space}
\label{sec:partition_coordinates}

We develop a coordinate system for addressing categorical states in bounded phase space. The coordinates arise naturally from the structure of nested partitioning operations.

\subsection{Bounded Phase Space}

\begin{definition}[Bounded Phase Space]
\label{def:bounded_phase_space}
A \emph{bounded phase space} $\Omega$ is a compact region of categorical state space with finite volume:
\begin{equation}
    \text{Vol}(\Omega) = \int_\Omega d\mu < \infty
\end{equation}
where $d\mu$ is the natural measure on categorical states.
\end{definition}

\begin{axiom}[Partitioning]
\label{ax:partitioning}
Any bounded region $\Omega$ can be partitioned into disjoint subregions:
\begin{equation}
    \Omega = \bigcup_{i=1}^{k} \Omega_i \quad \text{with} \quad \Omega_i \cap \Omega_j = \emptyset \text{ for } i \neq j
\end{equation}
\end{axiom}

\begin{axiom}[Nesting]
\label{ax:nesting}
Partitioning operations can be nested: if $\Omega_i$ is a partition of $\Omega$, then $\Omega_i$ can itself be partitioned:
\begin{equation}
    \Omega_i = \bigcup_{j=1}^{m} \Omega_{i,j}
\end{equation}
\end{axiom}

\subsection{The Partition Depth Parameter}

\begin{definition}[Partition Depth]
\label{def:partition_depth}
The \emph{partition depth} $n$ of a categorical state is the number of nested partition boundaries enclosing that state:
\begin{equation}
    n = |\{B : B \text{ is a boundary enclosing the state}\}|
\end{equation}
where $n \geq 1$ (every state is enclosed by at least the outer boundary of $\Omega$).
\end{definition}

\begin{theorem}[Discrete Depth]
\label{thm:discrete_depth}
Partition depth takes only positive integer values: $n \in \{1, 2, 3, \ldots\}$.
\end{theorem}

\begin{proof}
Each boundary is either present or absent. The count of enclosing boundaries is therefore a non-negative integer. Since every state in $\Omega$ is enclosed by at least the outer boundary, $n \geq 1$.
\end{proof}

\subsection{The Angular Complexity Parameter}

\begin{definition}[Boundary Complexity]
\label{def:boundary_complexity}
For a partition boundary at depth $n$, the \emph{angular complexity} $l$ measures the number of independent angular variations in the boundary surface:
\begin{equation}
    l = \dim(\text{angular degrees of freedom of boundary})
\end{equation}
\end{definition}

\begin{theorem}[Complexity Constraint]
\label{thm:complexity_constraint}
For a state at partition depth $n$, the angular complexity satisfies:
\begin{equation}
    0 \leq l \leq n - 1
\end{equation}
\end{theorem}

\begin{proof}
At depth $n = 1$ (the outermost boundary), the boundary is a simple closed surface with no internal angular structure, hence $l = 0$.

At depth $n = 2$, the boundary can have at most one independent angular variation (a single nodal plane), hence $l \in \{0, 1\}$.

By induction: at depth $n$, there can be at most $n - 1$ independent angular variations, since each additional nesting level permits at most one additional angular degree of freedom. Thus $l \in \{0, 1, \ldots, n-1\}$.
\end{proof}

\subsection{The Orientation Parameter}

\begin{definition}[Spatial Orientation]
\label{def:spatial_orientation}
For a boundary with angular complexity $l$, the \emph{orientation parameter} $m$ specifies which of the $2l + 1$ possible spatial orientations the boundary occupies:
\begin{equation}
    m \in \{-l, -l+1, \ldots, 0, \ldots, l-1, l\}
\end{equation}
\end{definition}

\begin{theorem}[Orientation Degeneracy]
\label{thm:orientation_degeneracy}
For angular complexity $l$, there are exactly $2l + 1$ distinct orientations.
\end{theorem}

\begin{proof}
Consider a boundary with $l$ independent angular variations. In three-dimensional space, each angular variation can be oriented along any axis. The number of distinct orientations for a structure with $l$ angular nodes is the number of ways to orient $l$ nodal planes in space, which is $2l + 1$ (corresponding to the $2l + 1$ spherical harmonics of order $l$).
\end{proof}

\subsection{The Chirality Parameter}

\begin{definition}[Boundary Chirality]
\label{def:chirality}
Each partition boundary has a \emph{chirality} $s \in \{-\frac{1}{2}, +\frac{1}{2}\}$ corresponding to its handedness---whether the boundary curves ``left'' or ``right'' relative to the traversal direction.
\end{definition}

\begin{theorem}[Binary Chirality]
\label{thm:binary_chirality}
Chirality is strictly binary: $s = \pm\frac{1}{2}$ with no intermediate values.
\end{theorem}

\begin{proof}
Chirality is a topological property of oriented surfaces. A surface either has one handedness or the other; there is no continuous interpolation between them. The values $\pm\frac{1}{2}$ are conventional, chosen for algebraic convenience.
\end{proof}

\subsection{The Complete Partition Coordinate}

\begin{definition}[Partition Coordinate]
\label{def:partition_coordinate}
A \emph{partition coordinate} is a 4-tuple $(n, l, m, s)$ satisfying:
\begin{align}
    n &\in \{1, 2, 3, \ldots\} \\
    l &\in \{0, 1, \ldots, n-1\} \\
    m &\in \{-l, -l+1, \ldots, l\} \\
    s &\in \{-\tfrac{1}{2}, +\tfrac{1}{2}\}
\end{align}
Each valid coordinate addresses a unique categorical state in bounded phase space.
\end{definition}

\begin{theorem}[Completeness]
\label{thm:completeness}
Every categorical state in bounded phase space has a unique partition coordinate $(n, l, m, s)$.
\end{theorem}

\begin{proof}
By construction: $n$ specifies the partition depth, $l$ specifies the boundary complexity at that depth, $m$ specifies the orientation, and $s$ specifies the chirality. These four parameters exhaust the degrees of freedom for specifying a categorical state in bounded space.
\end{proof}

\begin{remark}[Structural Similarity]
The partition coordinate system $(n, l, m, s)$ has the same algebraic structure as the quantum numbers $(n, l, m_l, m_s)$ used in atomic physics to label electron states. This suggests a possible connection between categorical partitioning and atomic structure, which we explore in later sections.
\end{remark}

