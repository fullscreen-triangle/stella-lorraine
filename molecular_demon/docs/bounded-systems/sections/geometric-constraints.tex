\section{Geometric Constraints on Partition Space}
\label{sec:geometric_constraints}

We derive the fundamental capacity theorem: the maximum number of distinct categorical states at partition depth $n$ is exactly $2n^2$. This result follows purely from the geometry of nested partitioning.

\subsection{Counting States at Fixed Depth}

\begin{lemma}[States per Complexity Level]
\label{lem:states_per_l}
For a fixed angular complexity $l$, the number of distinct states is:
\begin{equation}
    N(l) = 2(2l + 1)
\end{equation}
accounting for all orientations and both chiralities.
\end{lemma}

\begin{proof}
At complexity $l$:
\begin{itemize}
    \item There are $(2l + 1)$ orientation values: $m \in \{-l, \ldots, +l\}$
    \item Each orientation has 2 chirality values: $s \in \{-\frac{1}{2}, +\frac{1}{2}\}$
\end{itemize}
Total: $N(l) = (2l + 1) \times 2 = 2(2l + 1)$.
\end{proof}

\begin{theorem}[Shell Capacity]
\label{thm:shell_capacity}
The total number of distinct states at partition depth $n$ is:
\begin{equation}
    C(n) = 2n^2
\end{equation}
\end{theorem}

\begin{proof}
At depth $n$, the allowed complexity values are $l \in \{0, 1, \ldots, n-1\}$.

The total number of states is:
\begin{align}
    C(n) &= \sum_{l=0}^{n-1} N(l) \\
         &= \sum_{l=0}^{n-1} 2(2l + 1) \\
         &= 2 \sum_{l=0}^{n-1} (2l + 1)
\end{align}

The sum $\sum_{l=0}^{n-1} (2l + 1)$ is the sum of the first $n$ odd numbers:
\begin{equation}
    \sum_{l=0}^{n-1} (2l + 1) = 1 + 3 + 5 + \cdots + (2n-1) = n^2
\end{equation}

Therefore:
\begin{equation}
    C(n) = 2n^2
\end{equation}
\end{proof}

\subsection{Explicit Capacity Values}

\begin{corollary}[Capacity Table]
\label{cor:capacity_table}
The capacity at each partition depth is:
\begin{center}
\begin{tabular}{ccc}
\toprule
Depth $n$ & Allowed $l$ values & Capacity $C(n) = 2n^2$ \\
\midrule
1 & $\{0\}$ & 2 \\
2 & $\{0, 1\}$ & 8 \\
3 & $\{0, 1, 2\}$ & 18 \\
4 & $\{0, 1, 2, 3\}$ & 32 \\
5 & $\{0, 1, 2, 3, 4\}$ & 50 \\
6 & $\{0, 1, 2, 3, 4, 5\}$ & 72 \\
7 & $\{0, 1, 2, 3, 4, 5, 6\}$ & 98 \\
\bottomrule
\end{tabular}
\end{center}
\end{corollary}

\subsection{Subshell Structure}

\begin{definition}[Subshell]
\label{def:subshell}
A \emph{subshell} $(n, l)$ is the set of all states with fixed depth $n$ and complexity $l$:
\begin{equation}
    S_{n,l} = \{(n, l, m, s) : m \in \{-l, \ldots, l\}, s \in \{\pm\tfrac{1}{2}\}\}
\end{equation}
\end{definition}

\begin{theorem}[Subshell Capacity]
\label{thm:subshell_capacity}
Each subshell $(n, l)$ contains exactly $2(2l + 1)$ states:
\begin{center}
\begin{tabular}{ccc}
\toprule
Complexity $l$ & Designation & Capacity \\
\midrule
0 & $s$ & 2 \\
1 & $p$ & 6 \\
2 & $d$ & 10 \\
3 & $f$ & 14 \\
4 & $g$ & 18 \\
\bottomrule
\end{tabular}
\end{center}
where we use letters $s, p, d, f, g$ as conventional labels for complexity values $0, 1, 2, 3, 4$.
\end{theorem}

\subsection{Cumulative Capacity}

\begin{theorem}[Total States up to Depth $N$]
\label{thm:cumulative_capacity}
The total number of distinct states with partition depth $\leq N$ is:
\begin{equation}
    T(N) = \sum_{n=1}^{N} 2n^2 = \frac{N(N+1)(2N+1)}{3}
\end{equation}
\end{theorem}

\begin{proof}
\begin{align}
    T(N) &= \sum_{n=1}^{N} 2n^2 = 2 \sum_{n=1}^{N} n^2 \\
         &= 2 \cdot \frac{N(N+1)(2N+1)}{6} = \frac{N(N+1)(2N+1)}{3}
\end{align}
\end{proof}

\begin{corollary}[Cumulative Values]
\label{cor:cumulative_values}
\begin{center}
\begin{tabular}{cc}
\toprule
Maximum depth $N$ & Total states $T(N)$ \\
\midrule
1 & 2 \\
2 & 10 \\
3 & 28 \\
4 & 60 \\
5 & 110 \\
6 & 182 \\
7 & 280 \\
\bottomrule
\end{tabular}
\end{center}
\end{corollary}

\subsection{Geometric Interpretation}

\begin{theorem}[Boundary Surface Interpretation]
\label{thm:boundary_interpretation}
The $2n^2$ capacity at depth $n$ can be understood geometrically:
\begin{itemize}
    \item Factor of $n^2$: surface area of a sphere at radius $\propto n$ scales as $n^2$
    \item Factor of 2: binary chirality doubles the state count
\end{itemize}
\end{theorem}

\begin{proof}
Consider nested spherical partition boundaries. The $n$-th boundary has surface area $\propto n^2$. Each point on the surface can have two chiralities (handedness). Thus the total ``area'' available for categorical states scales as $2n^2$.

This geometric argument confirms the algebraic result from counting $(l, m, s)$ combinations.
\end{proof}

\begin{remark}[Structural Similarity]
The capacity formula $C(n) = 2n^2$ is identical to the electron capacity of atomic shells. In atomic physics, each shell with principal quantum number $n$ can hold at most $2n^2$ electrons. The subshell capacities (2, 6, 10, 14 for $s$, $p$, $d$, $f$) also match exactly. This correspondence suggests that atomic shell structure may be a physical manifestation of partition coordinate geometry.
\end{remark}

