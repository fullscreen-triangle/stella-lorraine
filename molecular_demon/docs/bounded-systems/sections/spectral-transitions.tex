\section{Spectral Transitions Between Coordinates}
\label{sec:spectral_transitions}

We derive the selection rules governing transitions between partition coordinates and show that these transitions produce discrete spectral signatures.

\subsection{Transition Energy}

\begin{definition}[Coordinate Transition]
\label{def:coordinate_transition}
A \emph{coordinate transition} is a change from partition coordinate $(n_i, l_i, m_i, s_i)$ to $(n_f, l_f, m_f, s_f)$, accompanied by energy exchange:
\begin{equation}
    \Delta E = E(n_f, l_f) - E(n_i, l_i)
\end{equation}
\end{definition}

\begin{theorem}[Transition Energy Formula]
\label{thm:transition_energy}
For transitions between states with partition depths $n_i$ and $n_f$, the energy exchanged is:
\begin{equation}
    \Delta E = E_0 \left( \frac{1}{n_f^2} - \frac{1}{n_i^2} \right)
\end{equation}
where $E_0$ is the characteristic energy scale of the system.
\end{theorem}

\begin{proof}
From Theorem~\ref{thm:depth_dependence}, the energy at depth $n$ is $E_n = -E_0/n^2$. The transition energy is:
\begin{align}
    \Delta E &= E_{n_f} - E_{n_i} \\
             &= -\frac{E_0}{n_f^2} - \left( -\frac{E_0}{n_i^2} \right) \\
             &= E_0 \left( \frac{1}{n_i^2} - \frac{1}{n_f^2} \right)
\end{align}
For emission (energy released), $n_f < n_i$ gives $\Delta E < 0$. For absorption, $n_f > n_i$ gives $\Delta E > 0$.
\end{proof}

\subsection{Selection Rules}

\begin{theorem}[Complexity Selection Rule]
\label{thm:complexity_selection}
Transitions are only permitted when the complexity changes by exactly one:
\begin{equation}
    \Delta l = l_f - l_i = \pm 1
\end{equation}
\end{theorem}

\begin{proof}
Consider the partition boundary as a continuous surface. A transition corresponds to the boundary deforming from one configuration to another. 

The number of nodal planes (which determines $l$) can only change by one through continuous deformation: either a new nodal plane appears ($\Delta l = +1$) or an existing nodal plane disappears ($\Delta l = -1$). Larger changes would require discontinuous deformation, which violates boundary continuity.
\end{proof}

\begin{theorem}[Orientation Selection Rule]
\label{thm:orientation_selection}
Transitions are permitted only when:
\begin{equation}
    \Delta m = m_f - m_i \in \{-1, 0, +1\}
\end{equation}
\end{theorem}

\begin{proof}
The orientation parameter $m$ corresponds to the angular momentum projection of the boundary. During a transition, angular momentum can change by at most one unit (absorbed or emitted with the energy quantum). Thus $\Delta m \in \{-1, 0, +1\}$.
\end{proof}

\begin{theorem}[Chirality Conservation]
\label{thm:chirality_conservation}
Transitions conserve chirality:
\begin{equation}
    \Delta s = s_f - s_i = 0
\end{equation}
\end{theorem}

\begin{proof}
Chirality is a topological invariant of the boundary surface. It cannot change through continuous deformation. A transition that changes chirality would require the boundary to ``flip inside out,'' which is topologically forbidden.
\end{proof}

\subsection{Spectral Series}

\begin{definition}[Spectral Series]
\label{def:spectral_series}
A \emph{spectral series} is the set of all transitions terminating at a common final depth $n_f$:
\begin{equation}
    \mathcal{S}_{n_f} = \left\{ \Delta E = E_0 \left( \frac{1}{n_f^2} - \frac{1}{n_i^2} \right) : n_i > n_f \right\}
\end{equation}
\end{definition}

\begin{theorem}[Series Structure]
\label{thm:series_structure}
For each final depth $n_f$, the spectral series has:
\begin{itemize}
    \item A \emph{series limit} as $n_i \to \infty$: $\Delta E_{\text{limit}} = E_0 / n_f^2$
    \item A \emph{first line} at $n_i = n_f + 1$: $\Delta E_1 = E_0 \left( \frac{1}{n_f^2} - \frac{1}{(n_f+1)^2} \right)$
    \item Lines that converge toward the series limit as $n_i$ increases
\end{itemize}
\end{theorem}

\begin{corollary}[Named Series]
\label{cor:named_series}
The first several series are:
\begin{center}
\begin{tabular}{cccc}
\toprule
Series & Final depth $n_f$ & First line ($n_i \to n_f$) & Series limit \\
\midrule
$\alpha$ & 1 & $2 \to 1$ & $E_0$ \\
$\beta$ & 2 & $3 \to 2$ & $E_0/4$ \\
$\gamma$ & 3 & $4 \to 3$ & $E_0/9$ \\
$\delta$ & 4 & $5 \to 4$ & $E_0/16$ \\
\bottomrule
\end{tabular}
\end{center}
\end{corollary}

\subsection{Spectral Wavelengths}

\begin{definition}[Transition Wavelength]
\label{def:transition_wavelength}
If the energy quantum $\Delta E$ is carried by a wave, the wavelength is:
\begin{equation}
    \lambda = \frac{hc}{\Delta E} = \frac{hc}{E_0} \cdot \frac{n_i^2 n_f^2}{n_i^2 - n_f^2}
\end{equation}
where $h$ is a fundamental action constant and $c$ is the propagation speed.
\end{definition}

\begin{theorem}[Discrete Spectrum]
\label{thm:discrete_spectrum}
The transition wavelengths form a discrete set, not a continuous distribution. This discreteness arises from the integer values of $n_i$ and $n_f$.
\end{theorem}

\subsection{Transition Intensity}

\begin{theorem}[Selection Rule Intensity]
\label{thm:transition_intensity}
Transitions satisfying the selection rules ($\Delta l = \pm 1$, $\Delta m \in \{0, \pm 1\}$, $\Delta s = 0$) have finite intensity. Transitions violating these rules have zero intensity (forbidden).
\end{theorem}

\begin{proof}
The intensity of a transition depends on the overlap integral between initial and final boundary configurations. For transitions violating the selection rules, symmetry arguments show that this integral vanishes identically.
\end{proof}

\begin{remark}[Structural Similarity]
The transition energy formula $\Delta E = E_0(1/n_f^2 - 1/n_i^2)$ is identical to the Rydberg formula for atomic spectral lines, with $E_0$ corresponding to the Rydberg energy (13.6 eV for hydrogen). The selection rules ($\Delta l = \pm 1$, $\Delta m \in \{0, \pm 1\}$, $\Delta s = 0$) match the electric dipole selection rules of atomic spectroscopy exactly. The series structure (with series named after Lyman, Balmer, Paschen, etc. in atomic physics) emerges directly from partition coordinate geometry.
\end{remark}

