\section{Multi-Body Partition Coordinates: Hyperfine Structure}
\label{sec:hyperfine}

We extend partition coordinate theory to systems where both the boundary and the center have internal structure. This leads to \emph{hyperfine splitting}---energy differences arising from coupling between boundary chirality and center chirality.

\subsection{Center Partition Coordinates}

\begin{definition}[Composite Partition System]
\label{def:composite_system}
A \emph{composite partition system} has:
\begin{itemize}
    \item Boundary partition coordinates $(n, l, m, s)$ describing the categorical boundary
    \item Center partition coordinates $(n_c, l_c, m_c, s_c)$ describing internal structure of the center
\end{itemize}
The complete state requires specifying both sets of coordinates.
\end{definition}

\begin{theorem}[Center Has Chirality]
\label{thm:center_chirality}
The central concentration $N_Z$ (Theorem~\ref{thm:center_exists} from previous work) is not structureless. At minimum, it has chirality $s_c \in \{-\frac{1}{2}, +\frac{1}{2}\}$.
\end{theorem}

\begin{proof}
The center is created by the convergence of negation fields. This convergence can occur with either handedness---the negations can ``spiral in'' clockwise or counterclockwise.

Once established, the center's chirality is fixed. It cannot continuously interpolate between $+\frac{1}{2}$ and $-\frac{1}{2}$ (Theorem~\ref{thm:binary_chirality}).

Therefore, every center has an intrinsic chirality $s_c = \pm\frac{1}{2}$.
\end{proof}

\subsection{Chirality-Chirality Coupling}

\begin{definition}[Chirality Coupling]
\label{def:chirality_coupling}
When a boundary with chirality $s$ encloses a center with chirality $s_c$, there is a coupling energy that depends on their relative orientation:
\begin{equation}
    E_{\text{coupling}} = A \cdot \mathbf{s} \cdot \mathbf{s}_c
\end{equation}
where $A$ is the coupling constant and $\mathbf{s} \cdot \mathbf{s}_c$ is the chirality product.
\end{definition}

\begin{theorem}[Two Coupling States]
\label{thm:coupling_states}
For a boundary with $s = \pm\frac{1}{2}$ enclosing a center with $s_c = \pm\frac{1}{2}$, there are exactly two distinct coupling configurations:
\begin{enumerate}
    \item \textbf{Parallel}: $s$ and $s_c$ have the same sign. Total chirality $F = |s + s_c| = 1$.
    \item \textbf{Antiparallel}: $s$ and $s_c$ have opposite signs. Total chirality $F = |s + s_c| = 0$.
\end{enumerate}
\end{theorem}

\begin{proof}
The chirality product $\mathbf{s} \cdot \mathbf{s}_c$ takes values:
\begin{align}
    (+\tfrac{1}{2})(+\tfrac{1}{2}) &= +\tfrac{1}{4} \quad \text{(parallel)} \\
    (+\tfrac{1}{2})(-\tfrac{1}{2}) &= -\tfrac{1}{4} \quad \text{(antiparallel)} \\
    (-\tfrac{1}{2})(+\tfrac{1}{2}) &= -\tfrac{1}{4} \quad \text{(antiparallel)} \\
    (-\tfrac{1}{2})(-\tfrac{1}{2}) &= +\tfrac{1}{4} \quad \text{(parallel)}
\end{align}
There are only two distinct values: $+\frac{1}{4}$ (parallel) and $-\frac{1}{4}$ (antiparallel).
\end{proof}

\subsection{Hyperfine Energy Splitting}

\begin{theorem}[Hyperfine Energy Difference]
\label{thm:hyperfine_energy}
The energy difference between parallel and antiparallel configurations is:
\begin{equation}
    \Delta E_{\text{hf}} = E_{\text{parallel}} - E_{\text{antiparallel}} = A \left( \frac{1}{4} - \left(-\frac{1}{4}\right) \right) = \frac{A}{2}
\end{equation}
\end{theorem}

\begin{definition}[Hyperfine Coupling Constant]
\label{def:hyperfine_constant}
The coupling constant $A$ depends on the overlap between boundary and center:
\begin{equation}
    A = \frac{8\pi}{3} g_s g_c \mu_s \mu_c |\psi(0)|^2
\end{equation}
where:
\begin{itemize}
    \item $g_s, g_c$ are the chirality $g$-factors (gyromagnetic ratios) of boundary and center
    \item $\mu_s, \mu_c$ are the chirality magnetic moments
    \item $|\psi(0)|^2$ is the boundary probability density at the center location
\end{itemize}
\end{definition}

\begin{theorem}[Only $l = 0$ Boundaries Contribute]
\label{thm:s_orbital_hyperfine}
Only boundaries with angular complexity $l = 0$ have nonzero density at the center:
\begin{equation}
    |\psi_{n,l}(0)|^2 = \begin{cases}
        \frac{1}{\pi a_0^3 n^3} & \text{if } l = 0 \\
        0 & \text{if } l > 0
    \end{cases}
\end{equation}
where $a_0$ is the characteristic length scale.
\end{theorem}

\begin{proof}
Boundaries with $l > 0$ have angular nodes---surfaces where the probability density vanishes. At $r = 0$, all $l > 0$ boundaries pass through a node (the angular structure requires at least one nodal plane through the origin).

Only $l = 0$ boundaries are spherically symmetric with no nodes, allowing nonzero density at $r = 0$.
\end{proof}

\subsection{The 21 cm Transition}

\begin{theorem}[Ground State Hyperfine Splitting]
\label{thm:21cm_derivation}
For a single-partition configuration ($Z = 1$) in its ground state $(n=1, l=0, m=0)$, the hyperfine energy splitting is:
\begin{equation}
    \Delta E_{\text{hf}} = 5.87 \times 10^{-6} \text{ eV}
\end{equation}
\end{theorem}

\begin{proof}
For the ground state of the simplest partition configuration:
\begin{itemize}
    \item $n = 1, l = 0$ gives $|\psi(0)|^2 = 1/(\pi a_0^3)$
    \item The boundary chirality moment is $\mu_s = g_s \mu_B$ where $\mu_B$ is the Bohr magneton
    \item The center chirality moment is $\mu_c = g_c \mu_N$ where $\mu_N$ is the nuclear magneton
    \item $\mu_N / \mu_B \approx 1/1836$ (the mass ratio)
\end{itemize}

Substituting:
\begin{equation}
    A = \frac{8\pi}{3} g_s g_c \frac{\mu_B^2}{1836} \cdot \frac{1}{\pi a_0^3} = \frac{8 g_s g_c \mu_B^2}{3 \cdot 1836 \cdot a_0^3}
\end{equation}

With $g_s \approx 2$ and $g_c \approx 5.59$:
\begin{equation}
    \Delta E_{\text{hf}} = \frac{A}{2} \approx 5.87 \times 10^{-6} \text{ eV}
\end{equation}
\end{proof}

\begin{corollary}[Transition Frequency and Wavelength]
\label{cor:21cm_frequency}
The hyperfine transition has:
\begin{align}
    \text{Frequency:} \quad \nu &= \frac{\Delta E_{\text{hf}}}{h} = 1420.405 \text{ MHz} \\
    \text{Wavelength:} \quad \lambda &= \frac{c}{\nu} = 21.106 \text{ cm}
\end{align}
\end{corollary}

\subsection{Virtual NMR Measurement}

\begin{definition}[Virtual Nuclear Magnetic Resonance]
\label{def:virtual_nmr}
A \emph{virtual NMR spectrometer} measures center chirality by:
\begin{enumerate}
    \item Applying an oscillating magnetic field at frequency $\nu$
    \item Detecting resonance when $\nu$ matches $\Delta E_{\text{hf}} / h$
    \item Recording the transition between $F = 1$ and $F = 0$ states
\end{enumerate}
\end{definition}

\begin{theorem}[NMR Measures Center Chirality]
\label{thm:nmr_chirality}
The NMR spectrum encodes the center chirality $s_c$ and its coupling to boundary chirality $s$:
\begin{itemize}
    \item \textbf{Chemical shift}: Position relative to reference frequency reveals local partition environment
    \item \textbf{Spin-spin coupling}: Splitting pattern reveals coupling between multiple centers
    \item \textbf{Relaxation times}: Decay rates reveal center dynamics
\end{itemize}
\end{theorem}

\subsection{Generalisation to Multi-Partition Systems}

\begin{theorem}[Multi-Center Coupling]
\label{thm:multi_center}
For a $Z$-partition configuration with $Z$ centers (each with chirality $s_{c,i}$) and $Z$ boundaries (each with chirality $s_j$), the hyperfine energy is:
\begin{equation}
    E_{\text{hf}} = \sum_{i,j} A_{ij} \, \mathbf{s}_{c,i} \cdot \mathbf{s}_j + \sum_{i < j} J_{ij} \, \mathbf{s}_{c,i} \cdot \mathbf{s}_{c,j}
\end{equation}
where:
\begin{itemize}
    \item $A_{ij}$ is the coupling between center $i$ and boundary $j$
    \item $J_{ij}$ is the direct coupling between centers $i$ and $j$
\end{itemize}
\end{theorem}

\begin{remark}[Structural Similarity]
The hyperfine splitting derived here is identical to the atomic hyperfine structure:
\begin{itemize}
    \item Boundary chirality $s$ corresponds to electron spin
    \item Center chirality $s_c$ corresponds to nuclear spin
    \item The 21 cm line (1420 MHz) is the hydrogen hyperfine transition used in radio astronomy
    \item NMR spectroscopy measures transitions between center chirality states
\end{itemize}
This demonstrates that partition coordinate theory can derive not only the gross structure of elements (shells, subshells, filling order) but also the fine and hyperfine structure arising from chirality couplings. The framework naturally extends to nuclear magnetic resonance phenomena.
\end{remark}

