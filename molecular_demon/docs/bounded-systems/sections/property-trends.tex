\section{Property Trends Across Partition Space}
\label{sec:property_trends}

We derive systematic trends in measurable properties as functions of partition coordinates. These trends emerge from the geometry of bounded phase space.

\subsection{Binding Energy}

\begin{definition}[Binding Energy]
\label{def:binding_energy}
The \emph{binding energy} $E_B(n, l)$ is the energy required to remove a categorical state from depth $(n, l)$ to infinity:
\begin{equation}
    E_B(n, l) = E_\infty - E(n, l) = 0 - \left( -\frac{E_0}{(n + \alpha l)^2} \right) = \frac{E_0}{(n + \alpha l)^2}
\end{equation}
\end{definition}

\begin{theorem}[Binding Energy Trends]
\label{thm:binding_trends}
The binding energy exhibits systematic trends:
\begin{enumerate}
    \item \textbf{Across a period} (constant $n$, increasing coordinate count): $E_B$ generally increases
    \item \textbf{Down a group} (increasing $n$, similar configuration): $E_B$ decreases
\end{enumerate}
\end{theorem}

\begin{proof}
\textbf{Across a period}: As states fill within a shell, the effective depth $n_{\text{eff}} = n + \alpha l$ increases slowly. However, the ``effective charge'' (attractive force toward the partition centre) increases faster due to incomplete shielding. The net effect is increasing $E_B$.

\textbf{Down a group}: Moving to higher $n$ with similar $l$ configuration increases the characteristic size, reducing binding. Thus $E_B \propto 1/n^2$ decreases.
\end{proof}

\subsection{Effective Size}

\begin{definition}[Characteristic Radius]
\label{def:characteristic_radius}
The \emph{characteristic radius} $r(n, l)$ of a partition boundary is:
\begin{equation}
    r(n, l) = r_0 \cdot \frac{(n + \alpha l)^2}{Z_{\text{eff}}}
\end{equation}
where $r_0$ is a fundamental length scale and $Z_{\text{eff}}$ is the effective central attraction.
\end{definition}

\begin{theorem}[Size Trends]
\label{thm:size_trends}
The characteristic radius exhibits systematic trends:
\begin{enumerate}
    \item \textbf{Across a period}: $r$ decreases as $Z_{\text{eff}}$ increases faster than $n_{\text{eff}}$
    \item \textbf{Down a group}: $r$ increases as $n$ increases
\end{enumerate}
\end{theorem}

\begin{proof}
\textbf{Across a period}: As states fill, incomplete shielding causes $Z_{\text{eff}}$ to increase. Since $r \propto 1/Z_{\text{eff}}$, the radius decreases.

\textbf{Down a group}: Moving to higher $n$ increases the numerator $(n + \alpha l)^2$, while $Z_{\text{eff}}$ increases more slowly. The net effect is increasing $r$.
\end{proof}

\subsection{Boundary Affinity}

\begin{definition}[Boundary Affinity]
\label{def:boundary_affinity}
The \emph{boundary affinity} $\chi$ measures the tendency of a partition boundary to attract additional categorical structure:
\begin{equation}
    \chi = \frac{E_B + E_A}{2}
\end{equation}
where $E_B$ is the binding energy and $E_A$ is the energy released when adding structure.
\end{definition}

\begin{theorem}[Affinity Trends]
\label{thm:affinity_trends}
Boundary affinity exhibits systematic trends:
\begin{enumerate}
    \item \textbf{Across a period}: $\chi$ generally increases
    \item \textbf{Down a group}: $\chi$ decreases
    \item \textbf{Maximum affinity}: Occurs near but not at complete shells
\end{enumerate}
\end{theorem}

\subsection{Completeness Stability}

\begin{definition}[Shell Completeness]
\label{def:shell_completeness}
A \emph{complete shell} at depth $n$ has all $2n^2$ states occupied. A \emph{complete subshell} at $(n, l)$ has all $2(2l+1)$ states occupied.
\end{definition}

\begin{theorem}[Completeness Stability]
\label{thm:completeness_stability}
Complete shells and subshells are exceptionally stable:
\begin{enumerate}
    \item Complete shells: Maximum binding energy, minimum size
    \item One state beyond complete shell: Minimum binding energy, maximum size
    \item One state before complete shell: High affinity for additional states
\end{enumerate}
\end{theorem}

\begin{proof}
Complete configurations have:
\begin{itemize}
    \item All orientations filled, cancelling angular asymmetries
    \item All chiralities paired, cancelling chirality effects
    \item Maximum symmetry, minimising energy
\end{itemize}

Breaking this symmetry by adding or removing states costs energy, hence completeness confers stability.
\end{proof}

\subsection{Property Periodicity}

\begin{theorem}[Periodic Recurrence]
\label{thm:periodic_recurrence}
Properties recur periodically as coordinates advance through the filling order:
\begin{enumerate}
    \item States with similar $(l, m, s)$ but different $n$ have similar properties
    \item The period length matches the capacity of the shells being filled
    \item Period lengths are: 2, 8, 8, 18, 18, 32, 32, \ldots
\end{enumerate}
\end{theorem}

\begin{proof}
The filling order places states with similar complexity $l$ at similar positions within each period. Since properties depend primarily on $l$ (boundary shape) and the number of states in the outermost shell, states at corresponding positions in successive periods have similar properties.
\end{proof}

\subsection{Group Structure}

\begin{definition}[Group]
\label{def:group}
A \emph{group} is the set of all states with the same position within their respective periods---i.e., the same number of states in the outermost incomplete shell.
\end{definition}

\begin{theorem}[Group Property Similarity]
\label{thm:group_similarity}
States in the same group have similar:
\begin{enumerate}
    \item Number of loosely-bound outer states
    \item Boundary affinity
    \item Interaction patterns with other configurations
\end{enumerate}
\end{theorem}

\begin{remark}[Structural Similarity]
The property trends derived here match the periodic trends observed in chemistry:
\begin{itemize}
    \item Binding energy corresponds to ionisation energy
    \item Characteristic radius corresponds to atomic radius
    \item Boundary affinity corresponds to electronegativity
    \item Completeness stability explains noble gas inertness
    \item Periodic recurrence explains the periodic table structure
    \item Groups correspond to chemical families (alkali metals, halogens, etc.)
\end{itemize}
The partition coordinate framework provides a geometric foundation for understanding why chemical properties are periodic.
\end{remark}

