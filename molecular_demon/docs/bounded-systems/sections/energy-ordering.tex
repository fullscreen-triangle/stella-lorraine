\section{Energy Ordering in Partition Space}
\label{sec:energy_ordering}

We derive the energy ordering of partition coordinates, showing that energy minimisation produces a specific filling sequence determined by the $(n + \alpha l)$ rule.

\subsection{Energy of Partition Coordinates}

\begin{definition}[Partition Energy]
\label{def:partition_energy}
The \emph{energy} $E(n, l)$ of a partition coordinate is the work required to establish the partition boundary configuration $(n, l)$:
\begin{equation}
    E(n, l) = E_0 \cdot f(n, l)
\end{equation}
where $E_0$ is a characteristic energy scale and $f(n, l)$ is a dimensionless function determined by boundary geometry.
\end{equation}
\end{definition}

\begin{theorem}[Depth Dependence]
\label{thm:depth_dependence}
The energy scales inversely with the square of partition depth:
\begin{equation}
    E(n, l) \propto -\frac{1}{n^2}
\end{equation}
where the negative sign indicates that deeper partitions are more stable (lower energy).
\end{theorem}

\begin{proof}
Consider a partition boundary at depth $n$. The boundary has characteristic size $r_n \propto n^2$ (from the geometric constraint that surface area scales as $n^2$).

The energy to maintain this boundary has two contributions:
\begin{enumerate}
    \item \textbf{Kinetic energy}: The traversal of the boundary requires momentum $p \propto 1/r_n \propto 1/n^2$ (uncertainty principle for categorical states).
    \item \textbf{Potential energy}: The binding to the partition centre scales as $1/r_n \propto 1/n^2$.
\end{enumerate}

Both contributions scale as $1/n^2$, giving:
\begin{equation}
    E_n = -\frac{E_0}{n^2}
\end{equation}
where $E_0 > 0$ is a constant and the negative sign reflects binding.
\end{proof}

\subsection{Complexity Correction}

\begin{theorem}[Complexity Raises Energy]
\label{thm:complexity_correction}
Higher angular complexity $l$ increases the energy (makes the state less stable):
\begin{equation}
    E(n, l) = -\frac{E_0}{(n + \alpha l)^2}
\end{equation}
where $\alpha > 0$ is a shielding parameter (typically $\alpha \approx 0.3$ to $0.4$).
\end{theorem}

\begin{proof}
Angular complexity introduces nodal structures in the partition boundary. States with higher $l$ have boundaries that penetrate less deeply toward the partition centre (they are ``pushed out'' by the angular nodes).

This effective increase in the characteristic radius can be modelled by replacing $n$ with an effective depth:
\begin{equation}
    n_{\text{eff}} = n + \alpha l
\end{equation}
where $\alpha$ captures the degree of penetration reduction per unit of complexity.

The energy becomes:
\begin{equation}
    E(n, l) = -\frac{E_0}{n_{\text{eff}}^2} = -\frac{E_0}{(n + \alpha l)^2}
\end{equation}
\end{proof}

\subsection{The Filling Order}

\begin{definition}[Filling Order]
\label{def:filling_order}
The \emph{filling order} is the sequence in which partition coordinates are occupied as states are added to a system, determined by increasing energy.
\end{definition}

\begin{theorem}[The $(n + l)$ Rule]
\label{thm:n_plus_l_rule}
For $\alpha \approx 1$, the filling order is determined by:
\begin{enumerate}
    \item Lower $(n + l)$ fills before higher $(n + l)$
    \item For equal $(n + l)$, lower $n$ fills first
\end{enumerate}
\end{theorem}

\begin{proof}
When $\alpha \approx 1$, we have $n_{\text{eff}} \approx n + l$. States with lower $n_{\text{eff}}$ have lower (more negative) energy and fill first.

For states with equal $n + l$, the one with smaller $n$ has been at that depth longer and has established more stable boundaries, hence fills first.
\end{proof}

\begin{corollary}[Explicit Filling Sequence]
\label{cor:filling_sequence}
The filling order for the first several subshells is:
\begin{center}
\begin{tabular}{cccc}
\toprule
Order & Subshell $(n, l)$ & $n + l$ & Capacity \\
\midrule
1 & $(1, 0)$ = 1$s$ & 1 & 2 \\
2 & $(2, 0)$ = 2$s$ & 2 & 2 \\
3 & $(2, 1)$ = 2$p$ & 3 & 6 \\
4 & $(3, 0)$ = 3$s$ & 3 & 2 \\
5 & $(3, 1)$ = 3$p$ & 4 & 6 \\
6 & $(4, 0)$ = 4$s$ & 4 & 2 \\
7 & $(3, 2)$ = 3$d$ & 5 & 10 \\
8 & $(4, 1)$ = 4$p$ & 5 & 6 \\
9 & $(5, 0)$ = 5$s$ & 5 & 2 \\
10 & $(4, 2)$ = 4$d$ & 6 & 10 \\
11 & $(5, 1)$ = 5$p$ & 6 & 6 \\
12 & $(6, 0)$ = 6$s$ & 6 & 2 \\
13 & $(4, 3)$ = 4$f$ & 7 & 14 \\
14 & $(5, 2)$ = 5$d$ & 7 & 10 \\
\bottomrule
\end{tabular}
\end{center}
\end{corollary}

\subsection{Period Structure}

\begin{definition}[Period]
\label{def:period}
A \emph{period} is a sequence of consecutive states in the filling order that begins with an $s$ subshell ($l = 0$).
\end{definition}

\begin{theorem}[Period Lengths]
\label{thm:period_lengths}
The filling order produces periods with lengths:
\begin{center}
\begin{tabular}{cc}
\toprule
Period & Length (number of states) \\
\midrule
1 & 2 \\
2 & 8 \\
3 & 8 \\
4 & 18 \\
5 & 18 \\
6 & 32 \\
7 & 32 \\
\bottomrule
\end{tabular}
\end{center}
\end{theorem}

\begin{proof}
Each period contains:
\begin{itemize}
    \item Period 1: 1$s$ only $\rightarrow$ 2 states
    \item Period 2: 2$s$ + 2$p$ $\rightarrow$ 2 + 6 = 8 states
    \item Period 3: 3$s$ + 3$p$ $\rightarrow$ 2 + 6 = 8 states
    \item Period 4: 4$s$ + 3$d$ + 4$p$ $\rightarrow$ 2 + 10 + 6 = 18 states
    \item Period 5: 5$s$ + 4$d$ + 5$p$ $\rightarrow$ 2 + 10 + 6 = 18 states
    \item Period 6: 6$s$ + 4$f$ + 5$d$ + 6$p$ $\rightarrow$ 2 + 14 + 10 + 6 = 32 states
    \item Period 7: 7$s$ + 5$f$ + 6$d$ + 7$p$ $\rightarrow$ 2 + 14 + 10 + 6 = 32 states
\end{itemize}
\end{proof}

\subsection{Block Structure}

\begin{definition}[Block]
\label{def:block}
A \emph{block} is the set of all states with a particular complexity $l$ value:
\begin{itemize}
    \item $s$-block: $l = 0$ (2 states per period)
    \item $p$-block: $l = 1$ (6 states per period)
    \item $d$-block: $l = 2$ (10 states per period)
    \item $f$-block: $l = 3$ (14 states per period)
\end{itemize}
\end{definition}

\begin{remark}[Structural Similarity]
The filling order derived here is identical to the aufbau principle of atomic physics. The period lengths (2, 8, 8, 18, 18, 32, 32) match the periods of the periodic table. The block structure ($s$, $p$, $d$, $f$) matches the block structure of chemical elements. This suggests that the periodic table may be a manifestation of partition coordinate geometry under energy minimisation.
\end{remark}

