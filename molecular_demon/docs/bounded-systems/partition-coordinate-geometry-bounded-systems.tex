\documentclass[12pt,a4paper]{article}

% Packages
\usepackage{amsmath,amssymb,amsthm}
\usepackage{graphicx}
\usepackage{hyperref}
\usepackage{geometry}
\usepackage{booktabs}
\usepackage{enumitem}
\usepackage{physics}
\usepackage{siunitx}
\usepackage{algorithm}
\usepackage{algpseudocode}

\geometry{margin=2.5cm}

% Theorem environments
\newtheorem{theorem}{Theorem}[section]
\newtheorem{lemma}[theorem]{Lemma}
\newtheorem{corollary}[theorem]{Corollary}
\newtheorem{proposition}[theorem]{Proposition}
\newtheorem{definition}[theorem]{Definition}
\newtheorem{axiom}[theorem]{Axiom}
\newtheorem{remark}[theorem]{Remark}
\newtheorem{example}[theorem]{Example}

% Title
\title{Partition Coordinate Geometry in Bounded Oscillatory Systems}
\author{}
\date{}

\begin{document}

\maketitle

\begin{abstract}
We develop a mathematical framework for describing categorical states in bounded phase spaces using \emph{partition coordinates}---a four-parameter addressing system arising from nested boundary constraints. We prove that the geometry of bounded partitioning imposes strict constraints on coordinate values: the depth parameter $n \geq 1$, the complexity parameter $l \in \{0, 1, \ldots, n-1\}$, the orientation parameter $m \in \{-l, \ldots, +l\}$, and a binary chirality parameter $s \in \{-\frac{1}{2}, +\frac{1}{2}\}$. From these constraints alone, we derive a fundamental capacity theorem: the maximum number of distinct states at partition depth $n$ is exactly $2n^2$. We show that energy minimisation in partition space produces a specific filling order, and that transitions between coordinates follow selection rules determined by boundary continuity. We extend the theory to multi-body systems where both boundaries and centers carry chirality, deriving hyperfine splitting from chirality-chirality coupling. For the simplest partition configuration, this predicts a hyperfine transition at 1420 MHz (21 cm wavelength). Hardware-based virtual instruments are constructed to measure partition coordinates, and a Universal Virtual Instrument Finder algorithm systematically constructs optimal measurement configurations from arbitrary hardware. We extend the framework to multi-atom systems, defining \emph{partition signatures} that uniquely characterise compounds. This enables mixture identification, compound feasibility prediction, and de novo molecular design---all from partition geometry. The resulting framework exhibits structural similarities to several known physical systems, which we note in concluding remarks.
\end{abstract}

\tableofcontents
\newpage

%==============================================================================
\part{Mathematical Foundations}
\label{part:foundations}
%==============================================================================

\section{Partition Coordinates in Bounded Phase Space}
\label{sec:partition_coordinates}

We develop a coordinate system for addressing categorical states in bounded phase space. The coordinates arise naturally from the structure of nested partitioning operations.

\subsection{Bounded Phase Space}

\begin{definition}[Bounded Phase Space]
\label{def:bounded_phase_space}
A \emph{bounded phase space} $\Omega$ is a compact region of categorical state space with finite volume:
\begin{equation}
    \text{Vol}(\Omega) = \int_\Omega d\mu < \infty
\end{equation}
where $d\mu$ is the natural measure on categorical states.
\end{definition}

\begin{axiom}[Partitioning]
\label{ax:partitioning}
Any bounded region $\Omega$ can be partitioned into disjoint subregions:
\begin{equation}
    \Omega = \bigcup_{i=1}^{k} \Omega_i \quad \text{with} \quad \Omega_i \cap \Omega_j = \emptyset \text{ for } i \neq j
\end{equation}
\end{axiom}

\begin{axiom}[Nesting]
\label{ax:nesting}
Partitioning operations can be nested: if $\Omega_i$ is a partition of $\Omega$, then $\Omega_i$ can itself be partitioned:
\begin{equation}
    \Omega_i = \bigcup_{j=1}^{m} \Omega_{i,j}
\end{equation}
\end{axiom}

\subsection{The Partition Depth Parameter}

\begin{definition}[Partition Depth]
\label{def:partition_depth}
The \emph{partition depth} $n$ of a categorical state is the number of nested partition boundaries enclosing that state:
\begin{equation}
    n = |\{B : B \text{ is a boundary enclosing the state}\}|
\end{equation}
where $n \geq 1$ (every state is enclosed by at least the outer boundary of $\Omega$).
\end{definition}

\begin{theorem}[Discrete Depth]
\label{thm:discrete_depth}
Partition depth takes only positive integer values: $n \in \{1, 2, 3, \ldots\}$.
\end{theorem}

\begin{proof}
Each boundary is either present or absent. The count of enclosing boundaries is therefore a non-negative integer. Since every state in $\Omega$ is enclosed by at least the outer boundary, $n \geq 1$.
\end{proof}

\subsection{The Angular Complexity Parameter}

\begin{definition}[Boundary Complexity]
\label{def:boundary_complexity}
For a partition boundary at depth $n$, the \emph{angular complexity} $l$ measures the number of independent angular variations in the boundary surface:
\begin{equation}
    l = \dim(\text{angular degrees of freedom of boundary})
\end{equation}
\end{definition}

\begin{theorem}[Complexity Constraint]
\label{thm:complexity_constraint}
For a state at partition depth $n$, the angular complexity satisfies:
\begin{equation}
    0 \leq l \leq n - 1
\end{equation}
\end{theorem}

\begin{proof}
At depth $n = 1$ (the outermost boundary), the boundary is a simple closed surface with no internal angular structure, hence $l = 0$.

At depth $n = 2$, the boundary can have at most one independent angular variation (a single nodal plane), hence $l \in \{0, 1\}$.

By induction: at depth $n$, there can be at most $n - 1$ independent angular variations, since each additional nesting level permits at most one additional angular degree of freedom. Thus $l \in \{0, 1, \ldots, n-1\}$.
\end{proof}

\subsection{The Orientation Parameter}

\begin{definition}[Spatial Orientation]
\label{def:spatial_orientation}
For a boundary with angular complexity $l$, the \emph{orientation parameter} $m$ specifies which of the $2l + 1$ possible spatial orientations the boundary occupies:
\begin{equation}
    m \in \{-l, -l+1, \ldots, 0, \ldots, l-1, l\}
\end{equation}
\end{definition}

\begin{theorem}[Orientation Degeneracy]
\label{thm:orientation_degeneracy}
For angular complexity $l$, there are exactly $2l + 1$ distinct orientations.
\end{theorem}

\begin{proof}
Consider a boundary with $l$ independent angular variations. In three-dimensional space, each angular variation can be oriented along any axis. The number of distinct orientations for a structure with $l$ angular nodes is the number of ways to orient $l$ nodal planes in space, which is $2l + 1$ (corresponding to the $2l + 1$ spherical harmonics of order $l$).
\end{proof}

\subsection{The Chirality Parameter}

\begin{definition}[Boundary Chirality]
\label{def:chirality}
Each partition boundary has a \emph{chirality} $s \in \{-\frac{1}{2}, +\frac{1}{2}\}$ corresponding to its handedness---whether the boundary curves ``left'' or ``right'' relative to the traversal direction.
\end{definition}

\begin{theorem}[Binary Chirality]
\label{thm:binary_chirality}
Chirality is strictly binary: $s = \pm\frac{1}{2}$ with no intermediate values.
\end{theorem}

\begin{proof}
Chirality is a topological property of oriented surfaces. A surface either has one handedness or the other; there is no continuous interpolation between them. The values $\pm\frac{1}{2}$ are conventional, chosen for algebraic convenience.
\end{proof}

\subsection{The Complete Partition Coordinate}

\begin{definition}[Partition Coordinate]
\label{def:partition_coordinate}
A \emph{partition coordinate} is a 4-tuple $(n, l, m, s)$ satisfying:
\begin{align}
    n &\in \{1, 2, 3, \ldots\} \\
    l &\in \{0, 1, \ldots, n-1\} \\
    m &\in \{-l, -l+1, \ldots, l\} \\
    s &\in \{-\tfrac{1}{2}, +\tfrac{1}{2}\}
\end{align}
Each valid coordinate addresses a unique categorical state in bounded phase space.
\end{definition}

\begin{theorem}[Completeness]
\label{thm:completeness}
Every categorical state in bounded phase space has a unique partition coordinate $(n, l, m, s)$.
\end{theorem}

\begin{proof}
By construction: $n$ specifies the partition depth, $l$ specifies the boundary complexity at that depth, $m$ specifies the orientation, and $s$ specifies the chirality. These four parameters exhaust the degrees of freedom for specifying a categorical state in bounded space.
\end{proof}

\begin{remark}[Structural Similarity]
The partition coordinate system $(n, l, m, s)$ has the same algebraic structure as the quantum numbers $(n, l, m_l, m_s)$ used in atomic physics to label electron states. This suggests a possible connection between categorical partitioning and atomic structure, which we explore in later sections.
\end{remark}


\section{Geometric Constraints on Partition Space}
\label{sec:geometric_constraints}

We derive the fundamental capacity theorem: the maximum number of distinct categorical states at partition depth $n$ is exactly $2n^2$. This result follows purely from the geometry of nested partitioning.

\subsection{Counting States at Fixed Depth}

\begin{lemma}[States per Complexity Level]
\label{lem:states_per_l}
For a fixed angular complexity $l$, the number of distinct states is:
\begin{equation}
    N(l) = 2(2l + 1)
\end{equation}
accounting for all orientations and both chiralities.
\end{lemma}

\begin{proof}
At complexity $l$:
\begin{itemize}
    \item There are $(2l + 1)$ orientation values: $m \in \{-l, \ldots, +l\}$
    \item Each orientation has 2 chirality values: $s \in \{-\frac{1}{2}, +\frac{1}{2}\}$
\end{itemize}
Total: $N(l) = (2l + 1) \times 2 = 2(2l + 1)$.
\end{proof}

\begin{theorem}[Shell Capacity]
\label{thm:shell_capacity}
The total number of distinct states at partition depth $n$ is:
\begin{equation}
    C(n) = 2n^2
\end{equation}
\end{theorem}

\begin{proof}
At depth $n$, the allowed complexity values are $l \in \{0, 1, \ldots, n-1\}$.

The total number of states is:
\begin{align}
    C(n) &= \sum_{l=0}^{n-1} N(l) \\
         &= \sum_{l=0}^{n-1} 2(2l + 1) \\
         &= 2 \sum_{l=0}^{n-1} (2l + 1)
\end{align}

The sum $\sum_{l=0}^{n-1} (2l + 1)$ is the sum of the first $n$ odd numbers:
\begin{equation}
    \sum_{l=0}^{n-1} (2l + 1) = 1 + 3 + 5 + \cdots + (2n-1) = n^2
\end{equation}

Therefore:
\begin{equation}
    C(n) = 2n^2
\end{equation}
\end{proof}

\subsection{Explicit Capacity Values}

\begin{corollary}[Capacity Table]
\label{cor:capacity_table}
The capacity at each partition depth is:
\begin{center}
\begin{tabular}{ccc}
\toprule
Depth $n$ & Allowed $l$ values & Capacity $C(n) = 2n^2$ \\
\midrule
1 & $\{0\}$ & 2 \\
2 & $\{0, 1\}$ & 8 \\
3 & $\{0, 1, 2\}$ & 18 \\
4 & $\{0, 1, 2, 3\}$ & 32 \\
5 & $\{0, 1, 2, 3, 4\}$ & 50 \\
6 & $\{0, 1, 2, 3, 4, 5\}$ & 72 \\
7 & $\{0, 1, 2, 3, 4, 5, 6\}$ & 98 \\
\bottomrule
\end{tabular}
\end{center}
\end{corollary}

\subsection{Subshell Structure}

\begin{definition}[Subshell]
\label{def:subshell}
A \emph{subshell} $(n, l)$ is the set of all states with fixed depth $n$ and complexity $l$:
\begin{equation}
    S_{n,l} = \{(n, l, m, s) : m \in \{-l, \ldots, l\}, s \in \{\pm\tfrac{1}{2}\}\}
\end{equation}
\end{definition}

\begin{theorem}[Subshell Capacity]
\label{thm:subshell_capacity}
Each subshell $(n, l)$ contains exactly $2(2l + 1)$ states:
\begin{center}
\begin{tabular}{ccc}
\toprule
Complexity $l$ & Designation & Capacity \\
\midrule
0 & $s$ & 2 \\
1 & $p$ & 6 \\
2 & $d$ & 10 \\
3 & $f$ & 14 \\
4 & $g$ & 18 \\
\bottomrule
\end{tabular}
\end{center}
where we use letters $s, p, d, f, g$ as conventional labels for complexity values $0, 1, 2, 3, 4$.
\end{theorem}

\subsection{Cumulative Capacity}

\begin{theorem}[Total States up to Depth $N$]
\label{thm:cumulative_capacity}
The total number of distinct states with partition depth $\leq N$ is:
\begin{equation}
    T(N) = \sum_{n=1}^{N} 2n^2 = \frac{N(N+1)(2N+1)}{3}
\end{equation}
\end{theorem}

\begin{proof}
\begin{align}
    T(N) &= \sum_{n=1}^{N} 2n^2 = 2 \sum_{n=1}^{N} n^2 \\
         &= 2 \cdot \frac{N(N+1)(2N+1)}{6} = \frac{N(N+1)(2N+1)}{3}
\end{align}
\end{proof}

\begin{corollary}[Cumulative Values]
\label{cor:cumulative_values}
\begin{center}
\begin{tabular}{cc}
\toprule
Maximum depth $N$ & Total states $T(N)$ \\
\midrule
1 & 2 \\
2 & 10 \\
3 & 28 \\
4 & 60 \\
5 & 110 \\
6 & 182 \\
7 & 280 \\
\bottomrule
\end{tabular}
\end{center}
\end{corollary}

\subsection{Geometric Interpretation}

\begin{theorem}[Boundary Surface Interpretation]
\label{thm:boundary_interpretation}
The $2n^2$ capacity at depth $n$ can be understood geometrically:
\begin{itemize}
    \item Factor of $n^2$: surface area of a sphere at radius $\propto n$ scales as $n^2$
    \item Factor of 2: binary chirality doubles the state count
\end{itemize}
\end{theorem}

\begin{proof}
Consider nested spherical partition boundaries. The $n$-th boundary has surface area $\propto n^2$. Each point on the surface can have two chiralities (handedness). Thus the total ``area'' available for categorical states scales as $2n^2$.

This geometric argument confirms the algebraic result from counting $(l, m, s)$ combinations.
\end{proof}

\begin{remark}[Structural Similarity]
The capacity formula $C(n) = 2n^2$ is identical to the electron capacity of atomic shells. In atomic physics, each shell with principal quantum number $n$ can hold at most $2n^2$ electrons. The subshell capacities (2, 6, 10, 14 for $s$, $p$, $d$, $f$) also match exactly. This correspondence suggests that atomic shell structure may be a physical manifestation of partition coordinate geometry.
\end{remark}


\section{Energy Ordering in Partition Space}
\label{sec:energy_ordering}

We derive the energy ordering of partition coordinates, showing that energy minimisation produces a specific filling sequence determined by the $(n + \alpha l)$ rule.

\subsection{Energy of Partition Coordinates}

\begin{definition}[Partition Energy]
\label{def:partition_energy}
The \emph{energy} $E(n, l)$ of a partition coordinate is the work required to establish the partition boundary configuration $(n, l)$:
\begin{equation}
    E(n, l) = E_0 \cdot f(n, l)
\end{equation}
where $E_0$ is a characteristic energy scale and $f(n, l)$ is a dimensionless function determined by boundary geometry.
\end{equation}
\end{definition}

\begin{theorem}[Depth Dependence]
\label{thm:depth_dependence}
The energy scales inversely with the square of partition depth:
\begin{equation}
    E(n, l) \propto -\frac{1}{n^2}
\end{equation}
where the negative sign indicates that deeper partitions are more stable (lower energy).
\end{theorem}

\begin{proof}
Consider a partition boundary at depth $n$. The boundary has characteristic size $r_n \propto n^2$ (from the geometric constraint that surface area scales as $n^2$).

The energy to maintain this boundary has two contributions:
\begin{enumerate}
    \item \textbf{Kinetic energy}: The traversal of the boundary requires momentum $p \propto 1/r_n \propto 1/n^2$ (uncertainty principle for categorical states).
    \item \textbf{Potential energy}: The binding to the partition centre scales as $1/r_n \propto 1/n^2$.
\end{enumerate}

Both contributions scale as $1/n^2$, giving:
\begin{equation}
    E_n = -\frac{E_0}{n^2}
\end{equation}
where $E_0 > 0$ is a constant and the negative sign reflects binding.
\end{proof}

\subsection{Complexity Correction}

\begin{theorem}[Complexity Raises Energy]
\label{thm:complexity_correction}
Higher angular complexity $l$ increases the energy (makes the state less stable):
\begin{equation}
    E(n, l) = -\frac{E_0}{(n + \alpha l)^2}
\end{equation}
where $\alpha > 0$ is a shielding parameter (typically $\alpha \approx 0.3$ to $0.4$).
\end{theorem}

\begin{proof}
Angular complexity introduces nodal structures in the partition boundary. States with higher $l$ have boundaries that penetrate less deeply toward the partition centre (they are ``pushed out'' by the angular nodes).

This effective increase in the characteristic radius can be modelled by replacing $n$ with an effective depth:
\begin{equation}
    n_{\text{eff}} = n + \alpha l
\end{equation}
where $\alpha$ captures the degree of penetration reduction per unit of complexity.

The energy becomes:
\begin{equation}
    E(n, l) = -\frac{E_0}{n_{\text{eff}}^2} = -\frac{E_0}{(n + \alpha l)^2}
\end{equation}
\end{proof}

\subsection{The Filling Order}

\begin{definition}[Filling Order]
\label{def:filling_order}
The \emph{filling order} is the sequence in which partition coordinates are occupied as states are added to a system, determined by increasing energy.
\end{definition}

\begin{theorem}[The $(n + l)$ Rule]
\label{thm:n_plus_l_rule}
For $\alpha \approx 1$, the filling order is determined by:
\begin{enumerate}
    \item Lower $(n + l)$ fills before higher $(n + l)$
    \item For equal $(n + l)$, lower $n$ fills first
\end{enumerate}
\end{theorem}

\begin{proof}
When $\alpha \approx 1$, we have $n_{\text{eff}} \approx n + l$. States with lower $n_{\text{eff}}$ have lower (more negative) energy and fill first.

For states with equal $n + l$, the one with smaller $n$ has been at that depth longer and has established more stable boundaries, hence fills first.
\end{proof}

\begin{corollary}[Explicit Filling Sequence]
\label{cor:filling_sequence}
The filling order for the first several subshells is:
\begin{center}
\begin{tabular}{cccc}
\toprule
Order & Subshell $(n, l)$ & $n + l$ & Capacity \\
\midrule
1 & $(1, 0)$ = 1$s$ & 1 & 2 \\
2 & $(2, 0)$ = 2$s$ & 2 & 2 \\
3 & $(2, 1)$ = 2$p$ & 3 & 6 \\
4 & $(3, 0)$ = 3$s$ & 3 & 2 \\
5 & $(3, 1)$ = 3$p$ & 4 & 6 \\
6 & $(4, 0)$ = 4$s$ & 4 & 2 \\
7 & $(3, 2)$ = 3$d$ & 5 & 10 \\
8 & $(4, 1)$ = 4$p$ & 5 & 6 \\
9 & $(5, 0)$ = 5$s$ & 5 & 2 \\
10 & $(4, 2)$ = 4$d$ & 6 & 10 \\
11 & $(5, 1)$ = 5$p$ & 6 & 6 \\
12 & $(6, 0)$ = 6$s$ & 6 & 2 \\
13 & $(4, 3)$ = 4$f$ & 7 & 14 \\
14 & $(5, 2)$ = 5$d$ & 7 & 10 \\
\bottomrule
\end{tabular}
\end{center}
\end{corollary}

\subsection{Period Structure}

\begin{definition}[Period]
\label{def:period}
A \emph{period} is a sequence of consecutive states in the filling order that begins with an $s$ subshell ($l = 0$).
\end{definition}

\begin{theorem}[Period Lengths]
\label{thm:period_lengths}
The filling order produces periods with lengths:
\begin{center}
\begin{tabular}{cc}
\toprule
Period & Length (number of states) \\
\midrule
1 & 2 \\
2 & 8 \\
3 & 8 \\
4 & 18 \\
5 & 18 \\
6 & 32 \\
7 & 32 \\
\bottomrule
\end{tabular}
\end{center}
\end{theorem}

\begin{proof}
Each period contains:
\begin{itemize}
    \item Period 1: 1$s$ only $\rightarrow$ 2 states
    \item Period 2: 2$s$ + 2$p$ $\rightarrow$ 2 + 6 = 8 states
    \item Period 3: 3$s$ + 3$p$ $\rightarrow$ 2 + 6 = 8 states
    \item Period 4: 4$s$ + 3$d$ + 4$p$ $\rightarrow$ 2 + 10 + 6 = 18 states
    \item Period 5: 5$s$ + 4$d$ + 5$p$ $\rightarrow$ 2 + 10 + 6 = 18 states
    \item Period 6: 6$s$ + 4$f$ + 5$d$ + 6$p$ $\rightarrow$ 2 + 14 + 10 + 6 = 32 states
    \item Period 7: 7$s$ + 5$f$ + 6$d$ + 7$p$ $\rightarrow$ 2 + 14 + 10 + 6 = 32 states
\end{itemize}
\end{proof}

\subsection{Block Structure}

\begin{definition}[Block]
\label{def:block}
A \emph{block} is the set of all states with a particular complexity $l$ value:
\begin{itemize}
    \item $s$-block: $l = 0$ (2 states per period)
    \item $p$-block: $l = 1$ (6 states per period)
    \item $d$-block: $l = 2$ (10 states per period)
    \item $f$-block: $l = 3$ (14 states per period)
\end{itemize}
\end{definition}

\begin{remark}[Structural Similarity]
The filling order derived here is identical to the aufbau principle of atomic physics. The period lengths (2, 8, 8, 18, 18, 32, 32) match the periods of the periodic table. The block structure ($s$, $p$, $d$, $f$) matches the block structure of chemical elements. This suggests that the periodic table may be a manifestation of partition coordinate geometry under energy minimisation.
\end{remark}



%==============================================================================
\part{Measurement Theory}
\label{part:measurement}
%==============================================================================

\section{Hardware-Based Partition Coordinate Measurement}
\label{sec:hardware_measurement}

We describe a suite of virtual instruments that measure partition coordinates using real hardware oscillator timing. These instruments do not simulate partition coordinates---they create them through the act of measurement.

\subsection{Measurement Philosophy}

\begin{axiom}[Measurement Creates State]
\label{ax:measurement_creates}
A partition coordinate does not exist independently of measurement. The act of measuring partition coordinates using hardware oscillators \emph{creates} the categorical state with those coordinates.
\end{axiom}

\begin{definition}[Hardware Oscillator]
\label{def:hardware_oscillator}
A \emph{hardware oscillator} is a physical timing device that provides real nanosecond-precision measurements:
\begin{equation}
    \delta t = t_{\text{measured}} - t_{\text{reference}}
\end{equation}
The timing variations $\delta t$ encode genuine categorical information, not measurement noise.
\end{definition}

\subsection{The Shell Resonator}

\begin{definition}[Shell Resonator]
\label{def:shell_resonator}
A \emph{shell resonator} is an instrument that measures partition depth $n$ by resonating with nested boundary structures. The resonance frequency scales inversely with depth squared:
\begin{equation}
    f_{\text{resonance}}(n) = \frac{f_0}{n^2}
\end{equation}
where $f_0$ is the base frequency of the instrument.
\end{definition}

\begin{theorem}[Depth Measurement]
\label{thm:depth_measurement}
The shell resonator determines $n$ by finding the resonance peak:
\begin{equation}
    n = \sqrt{\frac{f_0}{f_{\text{resonance}}}}
\end{equation}
Hardware timing variations determine which resonance is ``observed.''
\end{theorem}

\subsection{The Angular Analyser}

\begin{definition}[Angular Analyser]
\label{def:angular_analyser}
An \emph{angular analyser} measures the complexity parameter $l$ by detecting phase relationships in the partition boundary. The number of phase nodes determines $l$:
\begin{equation}
    l = \text{(number of nodal planes in boundary)}
\end{equation}
\end{definition}

\begin{theorem}[Complexity Constraint from Depth]
\label{thm:complexity_from_depth}
Given a measured depth $n$, the angular analyser can only return values $l \in \{0, 1, \ldots, n-1\}$. Hardware timing selects among these allowed values.
\end{theorem}

\subsection{The Orientation Mapper}

\begin{definition}[Orientation Mapper]
\label{def:orientation_mapper}
An \emph{orientation mapper} measures the orientation parameter $m$ by detecting the spatial direction of the partition boundary nodes. For complexity $l$, there are $2l + 1$ possible orientations:
\begin{equation}
    m \in \{-l, -l+1, \ldots, 0, \ldots, l-1, l\}
\end{equation}
\end{definition}

\begin{theorem}[Orientation Measurement]
\label{thm:orientation_measurement}
The orientation mapper uses vector components of hardware timing to determine $m$:
\begin{equation}
    m = \text{sign}(\delta t_x - \delta t_y) \cdot \left\lfloor \frac{|\delta t|}{\Delta t_{\text{resolution}}} \right\rfloor \mod (2l + 1) - l
\end{equation}
where $\delta t_x, \delta t_y$ are timing samples in orthogonal channels.
\end{theorem}

\subsection{The Chirality Discriminator}

\begin{definition}[Chirality Discriminator]
\label{def:chirality_discriminator}
A \emph{chirality discriminator} measures the binary chirality parameter $s$ by detecting the handedness of the partition boundary:
\begin{equation}
    s = \begin{cases}
        +\frac{1}{2} & \text{if } \delta t \text{ is even (in nanoseconds)} \\
        -\frac{1}{2} & \text{if } \delta t \text{ is odd (in nanoseconds)}
    \end{cases}
\end{equation}
\end{definition}

\begin{theorem}[Binary Measurement]
\label{thm:binary_measurement}
The chirality discriminator produces exactly two values with equal probability in the absence of bias. Hardware timing parity determines the measured chirality.
\end{theorem}

\subsection{The Complete Measurement Process}

\begin{theorem}[Sequential Measurement]
\label{thm:sequential_measurement}
A complete partition coordinate measurement proceeds as:
\begin{enumerate}
    \item Shell resonator measures $n$
    \item Angular analyser measures $l$ (constrained by $n$)
    \item Orientation mapper measures $m$ (constrained by $l$)
    \item Chirality discriminator measures $s$
\end{enumerate}
The resulting coordinate $(n, l, m, s)$ satisfies all geometric constraints by construction.
\end{theorem}

\subsection{Measurement Independence}

\begin{theorem}[Coordinate Uniqueness from Measurement]
\label{thm:measurement_uniqueness}
Two independent measurement sequences that yield the same coordinate $(n, l, m, s)$ have created the same categorical state. The coordinate uniquely identifies the state.
\end{theorem}

\begin{proof}
By the completeness theorem (Theorem~\ref{thm:completeness}), each coordinate addresses exactly one state. If two measurements yield the same coordinate, they have accessed the same point in partition space.
\end{proof}

\begin{remark}[Structural Similarity]
This measurement process mirrors the experimental determination of quantum numbers in atomic physics. Spectroscopy determines energy levels (related to $n$ and $l$), the Zeeman effect reveals $m$, and spin measurements determine $s$. The partition coordinate framework provides a categorical foundation for these measurements.
\end{remark}


\section{Spectral Transitions Between Coordinates}
\label{sec:spectral_transitions}

We derive the selection rules governing transitions between partition coordinates and show that these transitions produce discrete spectral signatures.

\subsection{Transition Energy}

\begin{definition}[Coordinate Transition]
\label{def:coordinate_transition}
A \emph{coordinate transition} is a change from partition coordinate $(n_i, l_i, m_i, s_i)$ to $(n_f, l_f, m_f, s_f)$, accompanied by energy exchange:
\begin{equation}
    \Delta E = E(n_f, l_f) - E(n_i, l_i)
\end{equation}
\end{definition}

\begin{theorem}[Transition Energy Formula]
\label{thm:transition_energy}
For transitions between states with partition depths $n_i$ and $n_f$, the energy exchanged is:
\begin{equation}
    \Delta E = E_0 \left( \frac{1}{n_f^2} - \frac{1}{n_i^2} \right)
\end{equation}
where $E_0$ is the characteristic energy scale of the system.
\end{theorem}

\begin{proof}
From Theorem~\ref{thm:depth_dependence}, the energy at depth $n$ is $E_n = -E_0/n^2$. The transition energy is:
\begin{align}
    \Delta E &= E_{n_f} - E_{n_i} \\
             &= -\frac{E_0}{n_f^2} - \left( -\frac{E_0}{n_i^2} \right) \\
             &= E_0 \left( \frac{1}{n_i^2} - \frac{1}{n_f^2} \right)
\end{align}
For emission (energy released), $n_f < n_i$ gives $\Delta E < 0$. For absorption, $n_f > n_i$ gives $\Delta E > 0$.
\end{proof}

\subsection{Selection Rules}

\begin{theorem}[Complexity Selection Rule]
\label{thm:complexity_selection}
Transitions are only permitted when the complexity changes by exactly one:
\begin{equation}
    \Delta l = l_f - l_i = \pm 1
\end{equation}
\end{theorem}

\begin{proof}
Consider the partition boundary as a continuous surface. A transition corresponds to the boundary deforming from one configuration to another. 

The number of nodal planes (which determines $l$) can only change by one through continuous deformation: either a new nodal plane appears ($\Delta l = +1$) or an existing nodal plane disappears ($\Delta l = -1$). Larger changes would require discontinuous deformation, which violates boundary continuity.
\end{proof}

\begin{theorem}[Orientation Selection Rule]
\label{thm:orientation_selection}
Transitions are permitted only when:
\begin{equation}
    \Delta m = m_f - m_i \in \{-1, 0, +1\}
\end{equation}
\end{theorem}

\begin{proof}
The orientation parameter $m$ corresponds to the angular momentum projection of the boundary. During a transition, angular momentum can change by at most one unit (absorbed or emitted with the energy quantum). Thus $\Delta m \in \{-1, 0, +1\}$.
\end{proof}

\begin{theorem}[Chirality Conservation]
\label{thm:chirality_conservation}
Transitions conserve chirality:
\begin{equation}
    \Delta s = s_f - s_i = 0
\end{equation}
\end{theorem}

\begin{proof}
Chirality is a topological invariant of the boundary surface. It cannot change through continuous deformation. A transition that changes chirality would require the boundary to ``flip inside out,'' which is topologically forbidden.
\end{proof}

\subsection{Spectral Series}

\begin{definition}[Spectral Series]
\label{def:spectral_series}
A \emph{spectral series} is the set of all transitions terminating at a common final depth $n_f$:
\begin{equation}
    \mathcal{S}_{n_f} = \left\{ \Delta E = E_0 \left( \frac{1}{n_f^2} - \frac{1}{n_i^2} \right) : n_i > n_f \right\}
\end{equation}
\end{definition}

\begin{theorem}[Series Structure]
\label{thm:series_structure}
For each final depth $n_f$, the spectral series has:
\begin{itemize}
    \item A \emph{series limit} as $n_i \to \infty$: $\Delta E_{\text{limit}} = E_0 / n_f^2$
    \item A \emph{first line} at $n_i = n_f + 1$: $\Delta E_1 = E_0 \left( \frac{1}{n_f^2} - \frac{1}{(n_f+1)^2} \right)$
    \item Lines that converge toward the series limit as $n_i$ increases
\end{itemize}
\end{theorem}

\begin{corollary}[Named Series]
\label{cor:named_series}
The first several series are:
\begin{center}
\begin{tabular}{cccc}
\toprule
Series & Final depth $n_f$ & First line ($n_i \to n_f$) & Series limit \\
\midrule
$\alpha$ & 1 & $2 \to 1$ & $E_0$ \\
$\beta$ & 2 & $3 \to 2$ & $E_0/4$ \\
$\gamma$ & 3 & $4 \to 3$ & $E_0/9$ \\
$\delta$ & 4 & $5 \to 4$ & $E_0/16$ \\
\bottomrule
\end{tabular}
\end{center}
\end{corollary}

\subsection{Spectral Wavelengths}

\begin{definition}[Transition Wavelength]
\label{def:transition_wavelength}
If the energy quantum $\Delta E$ is carried by a wave, the wavelength is:
\begin{equation}
    \lambda = \frac{hc}{\Delta E} = \frac{hc}{E_0} \cdot \frac{n_i^2 n_f^2}{n_i^2 - n_f^2}
\end{equation}
where $h$ is a fundamental action constant and $c$ is the propagation speed.
\end{definition}

\begin{theorem}[Discrete Spectrum]
\label{thm:discrete_spectrum}
The transition wavelengths form a discrete set, not a continuous distribution. This discreteness arises from the integer values of $n_i$ and $n_f$.
\end{theorem}

\subsection{Transition Intensity}

\begin{theorem}[Selection Rule Intensity]
\label{thm:transition_intensity}
Transitions satisfying the selection rules ($\Delta l = \pm 1$, $\Delta m \in \{0, \pm 1\}$, $\Delta s = 0$) have finite intensity. Transitions violating these rules have zero intensity (forbidden).
\end{theorem}

\begin{proof}
The intensity of a transition depends on the overlap integral between initial and final boundary configurations. For transitions violating the selection rules, symmetry arguments show that this integral vanishes identically.
\end{proof}

\begin{remark}[Structural Similarity]
The transition energy formula $\Delta E = E_0(1/n_f^2 - 1/n_i^2)$ is identical to the Rydberg formula for atomic spectral lines, with $E_0$ corresponding to the Rydberg energy (13.6 eV for hydrogen). The selection rules ($\Delta l = \pm 1$, $\Delta m \in \{0, \pm 1\}$, $\Delta s = 0$) match the electric dipole selection rules of atomic spectroscopy exactly. The series structure (with series named after Lyman, Balmer, Paschen, etc. in atomic physics) emerges directly from partition coordinate geometry.
\end{remark}


\section{Property Trends Across Partition Space}
\label{sec:property_trends}

We derive systematic trends in measurable properties as functions of partition coordinates. These trends emerge from the geometry of bounded phase space.

\subsection{Binding Energy}

\begin{definition}[Binding Energy]
\label{def:binding_energy}
The \emph{binding energy} $E_B(n, l)$ is the energy required to remove a categorical state from depth $(n, l)$ to infinity:
\begin{equation}
    E_B(n, l) = E_\infty - E(n, l) = 0 - \left( -\frac{E_0}{(n + \alpha l)^2} \right) = \frac{E_0}{(n + \alpha l)^2}
\end{equation}
\end{definition}

\begin{theorem}[Binding Energy Trends]
\label{thm:binding_trends}
The binding energy exhibits systematic trends:
\begin{enumerate}
    \item \textbf{Across a period} (constant $n$, increasing coordinate count): $E_B$ generally increases
    \item \textbf{Down a group} (increasing $n$, similar configuration): $E_B$ decreases
\end{enumerate}
\end{theorem}

\begin{proof}
\textbf{Across a period}: As states fill within a shell, the effective depth $n_{\text{eff}} = n + \alpha l$ increases slowly. However, the ``effective charge'' (attractive force toward the partition centre) increases faster due to incomplete shielding. The net effect is increasing $E_B$.

\textbf{Down a group}: Moving to higher $n$ with similar $l$ configuration increases the characteristic size, reducing binding. Thus $E_B \propto 1/n^2$ decreases.
\end{proof}

\subsection{Effective Size}

\begin{definition}[Characteristic Radius]
\label{def:characteristic_radius}
The \emph{characteristic radius} $r(n, l)$ of a partition boundary is:
\begin{equation}
    r(n, l) = r_0 \cdot \frac{(n + \alpha l)^2}{Z_{\text{eff}}}
\end{equation}
where $r_0$ is a fundamental length scale and $Z_{\text{eff}}$ is the effective central attraction.
\end{definition}

\begin{theorem}[Size Trends]
\label{thm:size_trends}
The characteristic radius exhibits systematic trends:
\begin{enumerate}
    \item \textbf{Across a period}: $r$ decreases as $Z_{\text{eff}}$ increases faster than $n_{\text{eff}}$
    \item \textbf{Down a group}: $r$ increases as $n$ increases
\end{enumerate}
\end{theorem}

\begin{proof}
\textbf{Across a period}: As states fill, incomplete shielding causes $Z_{\text{eff}}$ to increase. Since $r \propto 1/Z_{\text{eff}}$, the radius decreases.

\textbf{Down a group}: Moving to higher $n$ increases the numerator $(n + \alpha l)^2$, while $Z_{\text{eff}}$ increases more slowly. The net effect is increasing $r$.
\end{proof}

\subsection{Boundary Affinity}

\begin{definition}[Boundary Affinity]
\label{def:boundary_affinity}
The \emph{boundary affinity} $\chi$ measures the tendency of a partition boundary to attract additional categorical structure:
\begin{equation}
    \chi = \frac{E_B + E_A}{2}
\end{equation}
where $E_B$ is the binding energy and $E_A$ is the energy released when adding structure.
\end{definition}

\begin{theorem}[Affinity Trends]
\label{thm:affinity_trends}
Boundary affinity exhibits systematic trends:
\begin{enumerate}
    \item \textbf{Across a period}: $\chi$ generally increases
    \item \textbf{Down a group}: $\chi$ decreases
    \item \textbf{Maximum affinity}: Occurs near but not at complete shells
\end{enumerate}
\end{theorem}

\subsection{Completeness Stability}

\begin{definition}[Shell Completeness]
\label{def:shell_completeness}
A \emph{complete shell} at depth $n$ has all $2n^2$ states occupied. A \emph{complete subshell} at $(n, l)$ has all $2(2l+1)$ states occupied.
\end{definition}

\begin{theorem}[Completeness Stability]
\label{thm:completeness_stability}
Complete shells and subshells are exceptionally stable:
\begin{enumerate}
    \item Complete shells: Maximum binding energy, minimum size
    \item One state beyond complete shell: Minimum binding energy, maximum size
    \item One state before complete shell: High affinity for additional states
\end{enumerate}
\end{theorem}

\begin{proof}
Complete configurations have:
\begin{itemize}
    \item All orientations filled, cancelling angular asymmetries
    \item All chiralities paired, cancelling chirality effects
    \item Maximum symmetry, minimising energy
\end{itemize}

Breaking this symmetry by adding or removing states costs energy, hence completeness confers stability.
\end{proof}

\subsection{Property Periodicity}

\begin{theorem}[Periodic Recurrence]
\label{thm:periodic_recurrence}
Properties recur periodically as coordinates advance through the filling order:
\begin{enumerate}
    \item States with similar $(l, m, s)$ but different $n$ have similar properties
    \item The period length matches the capacity of the shells being filled
    \item Period lengths are: 2, 8, 8, 18, 18, 32, 32, \ldots
\end{enumerate}
\end{theorem}

\begin{proof}
The filling order places states with similar complexity $l$ at similar positions within each period. Since properties depend primarily on $l$ (boundary shape) and the number of states in the outermost shell, states at corresponding positions in successive periods have similar properties.
\end{proof}

\subsection{Group Structure}

\begin{definition}[Group]
\label{def:group}
A \emph{group} is the set of all states with the same position within their respective periods---i.e., the same number of states in the outermost incomplete shell.
\end{definition}

\begin{theorem}[Group Property Similarity]
\label{thm:group_similarity}
States in the same group have similar:
\begin{enumerate}
    \item Number of loosely-bound outer states
    \item Boundary affinity
    \item Interaction patterns with other configurations
\end{enumerate}
\end{theorem}

\begin{remark}[Structural Similarity]
The property trends derived here match the periodic trends observed in chemistry:
\begin{itemize}
    \item Binding energy corresponds to ionisation energy
    \item Characteristic radius corresponds to atomic radius
    \item Boundary affinity corresponds to electronegativity
    \item Completeness stability explains noble gas inertness
    \item Periodic recurrence explains the periodic table structure
    \item Groups correspond to chemical families (alkali metals, halogens, etc.)
\end{itemize}
The partition coordinate framework provides a geometric foundation for understanding why chemical properties are periodic.
\end{remark}



%==============================================================================
\part{Experimental Validation}
\label{part:experimental}
%==============================================================================

\section{Instrument Equivalence: Multiple Paths to Partition Coordinates}
\label{sec:instrument_equivalence}

We demonstrate that partition coordinates $(n, l, m, s)$ can be measured by multiple independent instruments---both exotic partition-theoretic devices and standard chemistry instrumentation. The agreement between these independent methods provides experimental validation of the partition coordinate framework.

\subsection{Four Categories of Instruments}

\begin{definition}[Instrument Categories]
\label{def:instrument_categories}
Partition coordinates can be extracted from four distinct instrument categories:
\begin{enumerate}
    \item \textbf{Exotic Partition Instruments}: Directly measure partition geometry
    \item \textbf{Standard Chemistry Instruments}: Measure properties that correlate with coordinates
    \item \textbf{Virtual Spectrometers}: Post-hoc reconfigurable analysis
    \item \textbf{Computational Categorical Instruments}: Extract coordinates from ensembles
\end{enumerate}
\end{definition}

\begin{theorem}[Instrument Equivalence]
\label{thm:instrument_equivalence}
All instrument categories extract the same partition coordinates for a given element. The coordinates are physical invariants, independent of measurement method.
\end{theorem}

%==============================================================================
\subsection{Category 1: Exotic Partition Instruments}
%==============================================================================

These instruments directly measure partition geometry:

\begin{definition}[Shell Resonator]
\label{def:shell_resonator_full}
\begin{tabular}{ll}
\textbf{Measures:} & Radial partition depth $n$ \\
\textbf{Mechanism:} & Detects nested boundary oscillations \\
\textbf{Output:} & Principal partition coordinate \\
\textbf{Hardware:} & Oscillation frequency analyser
\end{tabular}
\end{definition}

\begin{definition}[Angular Analyser]
\label{def:angular_analyser_full}
\begin{tabular}{ll}
\textbf{Measures:} & Angular partition complexity $l$ \\
\textbf{Mechanism:} & Maps phase space topology \\
\textbf{Output:} & Complexity coordinate \\
\textbf{Hardware:} & Angular momentum detector
\end{tabular}
\end{definition}

\begin{definition}[Orientation Mapper]
\label{def:orientation_mapper_full}
\begin{tabular}{ll}
\textbf{Measures:} & Partition orientation $m$ \\
\textbf{Mechanism:} & Directional field analysis \\
\textbf{Output:} & Orientation coordinate \\
\textbf{Hardware:} & Spatial alignment detector
\end{tabular}
\end{definition}

\begin{definition}[Chirality Discriminator]
\label{def:chirality_discriminator_full}
\begin{tabular}{ll}
\textbf{Measures:} & Boundary handedness $s$ \\
\textbf{Mechanism:} & Detects partition chirality \\
\textbf{Output:} & Chirality coordinate \\
\textbf{Hardware:} & Helicity analyser
\end{tabular}
\end{definition}

%==============================================================================
\subsection{Category 2: Standard Chemistry Instruments}
%==============================================================================

These conventional instruments measure properties that directly encode partition coordinates:

\begin{definition}[Mass Spectrometer]
\label{def:mass_spec}
\begin{tabular}{ll}
\textbf{Measures:} & Mass-to-charge ratio ($m/z$) \\
\textbf{Connection:} & Ionisation removes boundaries from specific $(n, l, m, s)$ \\
\textbf{Extracts:} & Which coordinates are most weakly bound \\
\textbf{Output:} & Ionisation energy $\rightarrow$ $n$, $l$ values
\end{tabular}

The ionisation energy $E_I$ relates to partition coordinates via:
\begin{equation}
    E_I = \frac{E_0 \cdot Z_{\text{eff}}^2}{n^2}
\end{equation}
where $Z_{\text{eff}}$ depends on shielding by inner boundaries.
\end{definition}

\begin{definition}[NMR Spectrometer]
\label{def:nmr_spec}
\begin{tabular}{ll}
\textbf{Measures:} & Nuclear magnetic resonance \\
\textbf{Connection:} & Center chirality $s_c$ in magnetic field \\
\textbf{Extracts:} & Chirality states, chemical environment \\
\textbf{Output:} & $s_c$ values, boundary density (related to $n$, $l$)
\end{tabular}

Chemical shift $\delta$ encodes local boundary density:
\begin{equation}
    \delta \propto \sum_{i} \frac{|\psi_i(r_{\text{nucleus}})|^2}{r_i}
\end{equation}
\end{definition}

\begin{definition}[ESR/EPR Spectrometer]
\label{def:esr_spec}
\begin{tabular}{ll}
\textbf{Measures:} & Electron spin resonance \\
\textbf{Connection:} & Unpaired boundary chiralities $s$ \\
\textbf{Extracts:} & Chirality states, boundary occupancy \\
\textbf{Output:} & $s$ values, $l$ values (from $g$-factor)
\end{tabular}

The $g$-factor deviation from 2.0023 encodes $l$:
\begin{equation}
    \Delta g = g - 2.0023 \propto \frac{\lambda}{E_0} \cdot l(l+1)
\end{equation}
where $\lambda$ is the spin-orbit coupling constant.
\end{definition}

\begin{definition}[X-ray Photoelectron Spectroscopy (XPS)]
\label{def:xps}
\begin{tabular}{ll}
\textbf{Measures:} & Binding energies of core boundaries \\
\textbf{Connection:} & Energy $= f(n, l)$ \\
\textbf{Extracts:} & All occupied $(n, l)$ states \\
\textbf{Output:} & Complete partition configuration
\end{tabular}

XPS binding energy for subshell $(n, l)$:
\begin{equation}
    E_B(n, l) = E_0 \cdot \frac{(Z - \sigma_{n,l})^2}{n^2}
\end{equation}
where $\sigma_{n,l}$ is the shielding constant.
\end{definition}

%==============================================================================
\subsection{Category 3: Virtual Spectrometers}
%==============================================================================

These instruments allow post-hoc reconfiguration of measurement parameters:

\begin{definition}[Virtual UV-Vis Spectrometer]
\label{def:uv_vis}
\begin{tabular}{ll}
\textbf{Measures:} & Electronic transitions $(n_1, l_1) \to (n_2, l_2)$ \\
\textbf{Connection:} & $\Delta E = E_0(1/n_1^2 - 1/n_2^2)$ \\
\textbf{Extracts:} & Transition energies $\to$ $n$, $l$ values \\
\textbf{Output:} & Spectral lines $\to$ partition coordinate differences
\end{tabular}
\end{definition}

\begin{definition}[Virtual IR Spectrometer]
\label{def:ir_spec}
\begin{tabular}{ll}
\textbf{Measures:} & Vibrational transitions \\
\textbf{Connection:} & Molecular vibrations = partition oscillations \\
\textbf{Extracts:} & Vibrational partition numbers \\
\textbf{Output:} & Partition dynamics in molecular systems
\end{tabular}
\end{definition}

\begin{definition}[Virtual Raman Spectrometer]
\label{def:raman_spec}
\begin{tabular}{ll}
\textbf{Measures:} & Inelastic scattering \\
\textbf{Connection:} & Polarisability = partition boundary flexibility \\
\textbf{Extracts:} & Vibrational and rotational states \\
\textbf{Output:} & Partition coordinate changes
\end{tabular}
\end{definition}

\begin{definition}[Virtual Fluorescence Spectrometer]
\label{def:fluorescence_spec}
\begin{tabular}{ll}
\textbf{Measures:} & Emission after excitation \\
\textbf{Connection:} & Relaxation through partition levels \\
\textbf{Extracts:} & Excited $(n, l) \to$ ground $(n, l)$ \\
\textbf{Output:} & Complete transition pathway
\end{tabular}
\end{definition}

%==============================================================================
\subsection{Category 4: Computational Categorical Instruments}
%==============================================================================

These analyse ensembles to extract categorical structure:

\begin{definition}[Ensemble Mass Analyser]
\label{def:ensemble_mass}
\begin{tabular}{ll}
\textbf{Input:} & Mass spec data from ensemble \\
\textbf{Process:} & Statistical analysis of ionisation patterns \\
\textbf{Extracts:} & Most probable $(n, l)$ for valence boundaries \\
\textbf{Output:} & Partition coordinates from ensemble statistics
\end{tabular}
\end{definition}

\begin{definition}[Spectral Deconvolution Engine]
\label{def:spectral_deconv}
\begin{tabular}{ll}
\textbf{Input:} & Complex spectrum (overlapping lines) \\
\textbf{Process:} & Decompose into individual transitions \\
\textbf{Extracts:} & All $(n_1, l_1) \to (n_2, l_2)$ transitions \\
\textbf{Output:} & Complete partition coordinate map
\end{tabular}
\end{definition}

\begin{definition}[Virtual Quantum State Tomography]
\label{def:state_tomography}
\begin{tabular}{ll}
\textbf{Input:} & Multiple measurement types \\
\textbf{Process:} & Reconstruct complete categorical state \\
\textbf{Extracts:} & Full $(n, l, m, s)$ for all boundaries \\
\textbf{Output:} & Complete partition coordinate set
\end{tabular}
\end{definition}

%==============================================================================
\subsection{Equivalence Demonstration: Period 2 Elements}
%==============================================================================

\begin{theorem}[Multi-Instrument Validation: Carbon ($Z = 6$)]
\label{thm:carbon_validation}
Carbon's partition coordinates can be independently measured by all instrument categories:

\paragraph{Exotic Instruments:}
\begin{itemize}
    \item Shell Resonator: Detects $n = 1$ (2 states) and $n = 2$ (4 states)
    \item Angular Analyser: Detects $l = 0$ (4 states) and $l = 1$ (2 states)
    \item Chirality Discriminator: Detects 2 unpaired $s = +\frac{1}{2}$ in $2p$
\end{itemize}

\paragraph{Mass Spectrometry:}
\begin{itemize}
    \item First ionisation: 11.26 eV $\Rightarrow$ $2p$ boundary removed
    \item Configuration: $1s^2 2s^2 2p^2$
\end{itemize}

\paragraph{XPS:}
\begin{itemize}
    \item $1s$ binding energy: 284.2 eV $\Rightarrow$ $(n=1, l=0)$ confirmed
    \item $2s$ binding energy: 18.7 eV $\Rightarrow$ $(n=2, l=0)$ confirmed
    \item $2p$ binding energy: 11.3 eV $\Rightarrow$ $(n=2, l=1)$ confirmed
\end{itemize}

\paragraph{UV-Vis:}
\begin{itemize}
    \item Transitions: $2p \to 3s$, $2p \to 3d$ observed
    \item Energies match $1/n^2$ formula with $Z_{\text{eff}} \approx 3.6$
\end{itemize}

\paragraph{ESR:}
\begin{itemize}
    \item $g$-factor: $\approx 2.003$ (small $l$ contribution from $2p$)
    \item Confirms 2 unpaired chiralities
\end{itemize}

All instruments agree: Carbon has partition signature $\{1s^2, 2s^2, 2p^2\}$.
\end{theorem}

\begin{theorem}[Multi-Instrument Validation: Oxygen ($Z = 8$)]
\label{thm:oxygen_validation}
Oxygen's partition coordinates measured by multiple methods:

\begin{center}
\begin{tabular}{lccc}
\toprule
Instrument & Measurement & Extracted & Result \\
\midrule
Shell Resonator & $n$ distribution & 2 at $n=1$, 6 at $n=2$ & $1s^2 2s^2 2p^4$ \\
Mass Spec & $E_I = 13.6$ eV & Valence in $2p$ & Confirmed \\
XPS ($1s$) & $E_B = 543$ eV & $(n=1, l=0)$ & Confirmed \\
XPS ($2s$) & $E_B = 28$ eV & $(n=2, l=0)$ & Confirmed \\
XPS ($2p$) & $E_B = 14$ eV & $(n=2, l=1)$ & Confirmed \\
ESR & $g \approx 2.01$ & 2 unpaired $s$ & Confirmed \\
\bottomrule
\end{tabular}
\end{center}
\end{theorem}

%==============================================================================
\subsection{Equivalence Demonstration: Transition Elements}
%==============================================================================

\begin{theorem}[Multi-Instrument Validation: Iron ($Z = 26$)]
\label{thm:iron_validation}
Iron demonstrates the power of multi-instrument validation for complex configurations:

\paragraph{Expected Configuration:} $[\text{Ar}] 3d^6 4s^2$

\paragraph{XPS Measurements:}
\begin{center}
\begin{tabular}{ccc}
\toprule
Subshell & Binding Energy (eV) & $(n, l)$ Confirmed \\
\midrule
$1s$ & 7112 & $(1, 0)$ \\
$2s$ & 844 & $(2, 0)$ \\
$2p$ & 710 & $(2, 1)$ \\
$3s$ & 91 & $(3, 0)$ \\
$3p$ & 53 & $(3, 1)$ \\
$3d$ & 7.1 & $(3, 2)$ \\
\bottomrule
\end{tabular}
\end{center}

\paragraph{ESR/Magnetic Measurements:}
\begin{itemize}
    \item Magnetic moment: $\mu = 4.9 \mu_B$
    \item Implies 4 unpaired chiralities in $3d$
    \item Configuration: $3d^6$ with 4 unpaired, 2 paired
\end{itemize}

\paragraph{Mass Spectrometry:}
\begin{itemize}
    \item $E_{I,1} = 7.9$ eV (remove $4s$)
    \item $E_{I,2} = 16.2$ eV (remove second $4s$)
    \item $E_{I,3} = 30.7$ eV (remove $3d$)
\end{itemize}

All instruments agree: $[\text{Ar}] 3d^6 4s^2$ with 4 unpaired chiralities.
\end{theorem}

%==============================================================================
\subsection{Cross-Validation Matrix}
%==============================================================================

\begin{theorem}[Complete Cross-Validation]
\label{thm:cross_validation}
For any element, all instrument categories must yield consistent partition coordinates:

\begin{center}
\begin{tabular}{l|cccc}
\toprule
Coordinate & Exotic & Mass/XPS & Spectroscopy & Computation \\
\midrule
$n$ & Shell Resonator & XPS binding & Rydberg formula & Tomography \\
$l$ & Angular Analyser & XPS fine structure & Selection rules & Deconvolution \\
$m$ & Orientation Mapper & Zeeman splitting & Polarisation & Tomography \\
$s$ & Chirality Disc. & ESR/EPR & Spin selection & Ensemble \\
\bottomrule
\end{tabular}
\end{center}

Disagreement between instruments indicates either measurement error or an exotic state (excited, ionised, etc.).
\end{theorem}

\begin{remark}[Physical Grounding]
The fact that standard chemistry instruments (mass spectrometry, XPS, NMR, ESR) yield the same partition coordinates as the exotic partition instruments demonstrates that:
\begin{enumerate}
    \item Partition coordinates are physically real, not theoretical constructs
    \item Standard analytical chemistry has been measuring partition geometry all along
    \item The partition framework provides a unified interpretation of diverse measurements
    \item No new experimental apparatus is needed---existing instruments suffice
\end{enumerate}
This equivalence is the strongest validation of partition coordinate theory: it explains why existing chemistry works.
\end{remark}

\section{Coordinate Uniqueness: The Exclusion Principle}
\label{sec:coordinate_uniqueness}

We prove that no two categorical states can share identical partition coordinates. This \emph{exclusion principle} is a fundamental consequence of categorical distinguishability.

\subsection{Categorical Distinguishability}

\begin{axiom}[Identity of Indiscernibles]
\label{ax:identity_indiscernibles}
Two categorical states are identical if and only if they have identical partition coordinates:
\begin{equation}
    S_1 = S_2 \iff (n_1, l_1, m_1, s_1) = (n_2, l_2, m_2, s_2)
\end{equation}
\end{axiom}

\begin{theorem}[Coordinate-State Bijection]
\label{thm:coordinate_bijection}
There is a one-to-one correspondence between valid partition coordinates and categorical states.
\end{theorem}

\begin{proof}
\textbf{Surjectivity}: Every categorical state has a partition coordinate (Theorem~\ref{thm:completeness}).

\textbf{Injectivity}: Suppose two states $S_1, S_2$ have the same coordinate $(n, l, m, s)$. Then they occupy the same partition depth, have the same boundary complexity, the same orientation, and the same chirality. By Axiom~\ref{ax:identity_indiscernibles}, $S_1 = S_2$.
\end{proof}

\subsection{The Exclusion Principle}

\begin{theorem}[Exclusion Principle]
\label{thm:exclusion_principle}
No two distinct categorical states can have identical partition coordinates:
\begin{equation}
    S_1 \neq S_2 \implies (n_1, l_1, m_1, s_1) \neq (n_2, l_2, m_2, s_2)
\end{equation}
Equivalently: each coordinate can be occupied by at most one state.
\end{theorem}

\begin{proof}
This is the contrapositive of the injectivity statement in Theorem~\ref{thm:coordinate_bijection}. If two states are distinct, they must have different coordinates (otherwise they would be identical by the Identity of Indiscernibles).
\end{proof}

\subsection{Consequences of Exclusion}

\begin{corollary}[Maximum Occupancy]
\label{cor:maximum_occupancy}
The maximum number of states at partition depth $n$ is exactly $2n^2$---no more states can be accommodated because all coordinates would be occupied.
\end{corollary}

\begin{corollary}[Filling Necessity]
\label{cor:filling_necessity}
When adding states to a system, each new state must occupy a previously unoccupied coordinate. This forces the filling order derived in Section~\ref{sec:energy_ordering}.
\end{corollary}

\begin{corollary}[Degeneracy Pressure]
\label{cor:degeneracy_pressure}
In a bounded region with many states, the exclusion principle creates an effective ``pressure'' that resists compression---states cannot be squeezed into already-occupied coordinates.
\end{corollary}

\subsection{Mathematical Formulation}

\begin{definition}[Occupation Number]
\label{def:occupation_number}
The \emph{occupation number} $N_{(n,l,m,s)}$ for coordinate $(n, l, m, s)$ is:
\begin{equation}
    N_{(n,l,m,s)} \in \{0, 1\}
\end{equation}
where 0 indicates unoccupied and 1 indicates occupied.
\end{definition}

\begin{theorem}[Occupation Constraint]
\label{thm:occupation_constraint}
For categorical states satisfying the exclusion principle:
\begin{equation}
    \sum_{(n,l,m,s)} N_{(n,l,m,s)}^2 = \sum_{(n,l,m,s)} N_{(n,l,m,s)}
\end{equation}
This is equivalent to requiring $N \in \{0, 1\}$ for all coordinates.
\end{theorem}

\begin{proof}
If $N \in \{0, 1\}$, then $N^2 = N$ for all occupation numbers, so the sums are equal.

Conversely, if the sums are equal, then $\sum N(N-1) = 0$. Since $N(N-1) \geq 0$ for $N \geq 0$, each term must vanish, requiring $N \in \{0, 1\}$.
\end{proof}

\subsection{Antisymmetry}

\begin{theorem}[State Antisymmetry]
\label{thm:antisymmetry}
A system of multiple categorical states is described by an antisymmetric function that changes sign when any two coordinates are exchanged:
\begin{equation}
    \Psi(\ldots, (n_i, l_i, m_i, s_i), \ldots, (n_j, l_j, m_j, s_j), \ldots) = -\Psi(\ldots, (n_j, l_j, m_j, s_j), \ldots, (n_i, l_i, m_i, s_i), \ldots)
\end{equation}
\end{theorem}

\begin{proof}
Antisymmetry ensures that $\Psi = 0$ when any two coordinates are identical (since $\Psi = -\Psi$ implies $\Psi = 0$). This enforces the exclusion principle at the level of the state function.
\end{proof}

\subsection{Connection to Chirality}

\begin{theorem}[Half-Integer Chirality and Exclusion]
\label{thm:chirality_exclusion}
The half-integer chirality values $s = \pm\frac{1}{2}$ are directly connected to the exclusion principle. States with half-integer chirality obey exclusion; states with integer chirality do not.
\end{theorem}

\begin{proof}
Under exchange of two states, the total state function acquires a phase $e^{i\pi(2s_1)(2s_2)}$. For half-integer $s$, this is $e^{i\pi} = -1$ (antisymmetric). For integer $s$, this is $e^{i 2\pi} = +1$ (symmetric).

Antisymmetry (half-integer chirality) enforces exclusion. Symmetry (integer chirality) allows multiple occupation.
\end{proof}

\begin{remark}[Structural Similarity]
The exclusion principle derived here has the same mathematical form as the Pauli exclusion principle of quantum mechanics:
\begin{itemize}
    \item No two fermions (half-integer spin) can have identical quantum numbers
    \item The occupation number is restricted to $\{0, 1\}$
    \item Multi-particle states are described by antisymmetric wave functions
\end{itemize}
The connection between half-integer chirality and exclusion mirrors the connection between half-integer spin and fermionic statistics. This suggests that the Pauli principle may be a consequence of categorical partitioning in bounded phase space, with spin being the physical manifestation of boundary chirality.
\end{remark}



%==============================================================================
\part{Extended Theory}
\label{part:extended}
%==============================================================================

\section{Multi-Body Partition Coordinates: Hyperfine Structure}
\label{sec:hyperfine}

We extend partition coordinate theory to systems where both the boundary and the center have internal structure. This leads to \emph{hyperfine splitting}---energy differences arising from coupling between boundary chirality and center chirality.

\subsection{Center Partition Coordinates}

\begin{definition}[Composite Partition System]
\label{def:composite_system}
A \emph{composite partition system} has:
\begin{itemize}
    \item Boundary partition coordinates $(n, l, m, s)$ describing the categorical boundary
    \item Center partition coordinates $(n_c, l_c, m_c, s_c)$ describing internal structure of the center
\end{itemize}
The complete state requires specifying both sets of coordinates.
\end{definition}

\begin{theorem}[Center Has Chirality]
\label{thm:center_chirality}
The central concentration $N_Z$ (Theorem~\ref{thm:center_exists} from previous work) is not structureless. At minimum, it has chirality $s_c \in \{-\frac{1}{2}, +\frac{1}{2}\}$.
\end{theorem}

\begin{proof}
The center is created by the convergence of negation fields. This convergence can occur with either handedness---the negations can ``spiral in'' clockwise or counterclockwise.

Once established, the center's chirality is fixed. It cannot continuously interpolate between $+\frac{1}{2}$ and $-\frac{1}{2}$ (Theorem~\ref{thm:binary_chirality}).

Therefore, every center has an intrinsic chirality $s_c = \pm\frac{1}{2}$.
\end{proof}

\subsection{Chirality-Chirality Coupling}

\begin{definition}[Chirality Coupling]
\label{def:chirality_coupling}
When a boundary with chirality $s$ encloses a center with chirality $s_c$, there is a coupling energy that depends on their relative orientation:
\begin{equation}
    E_{\text{coupling}} = A \cdot \mathbf{s} \cdot \mathbf{s}_c
\end{equation}
where $A$ is the coupling constant and $\mathbf{s} \cdot \mathbf{s}_c$ is the chirality product.
\end{definition}

\begin{theorem}[Two Coupling States]
\label{thm:coupling_states}
For a boundary with $s = \pm\frac{1}{2}$ enclosing a center with $s_c = \pm\frac{1}{2}$, there are exactly two distinct coupling configurations:
\begin{enumerate}
    \item \textbf{Parallel}: $s$ and $s_c$ have the same sign. Total chirality $F = |s + s_c| = 1$.
    \item \textbf{Antiparallel}: $s$ and $s_c$ have opposite signs. Total chirality $F = |s + s_c| = 0$.
\end{enumerate}
\end{theorem}

\begin{proof}
The chirality product $\mathbf{s} \cdot \mathbf{s}_c$ takes values:
\begin{align}
    (+\tfrac{1}{2})(+\tfrac{1}{2}) &= +\tfrac{1}{4} \quad \text{(parallel)} \\
    (+\tfrac{1}{2})(-\tfrac{1}{2}) &= -\tfrac{1}{4} \quad \text{(antiparallel)} \\
    (-\tfrac{1}{2})(+\tfrac{1}{2}) &= -\tfrac{1}{4} \quad \text{(antiparallel)} \\
    (-\tfrac{1}{2})(-\tfrac{1}{2}) &= +\tfrac{1}{4} \quad \text{(parallel)}
\end{align}
There are only two distinct values: $+\frac{1}{4}$ (parallel) and $-\frac{1}{4}$ (antiparallel).
\end{proof}

\subsection{Hyperfine Energy Splitting}

\begin{theorem}[Hyperfine Energy Difference]
\label{thm:hyperfine_energy}
The energy difference between parallel and antiparallel configurations is:
\begin{equation}
    \Delta E_{\text{hf}} = E_{\text{parallel}} - E_{\text{antiparallel}} = A \left( \frac{1}{4} - \left(-\frac{1}{4}\right) \right) = \frac{A}{2}
\end{equation}
\end{theorem}

\begin{definition}[Hyperfine Coupling Constant]
\label{def:hyperfine_constant}
The coupling constant $A$ depends on the overlap between boundary and center:
\begin{equation}
    A = \frac{8\pi}{3} g_s g_c \mu_s \mu_c |\psi(0)|^2
\end{equation}
where:
\begin{itemize}
    \item $g_s, g_c$ are the chirality $g$-factors (gyromagnetic ratios) of boundary and center
    \item $\mu_s, \mu_c$ are the chirality magnetic moments
    \item $|\psi(0)|^2$ is the boundary probability density at the center location
\end{itemize}
\end{definition}

\begin{theorem}[Only $l = 0$ Boundaries Contribute]
\label{thm:s_orbital_hyperfine}
Only boundaries with angular complexity $l = 0$ have nonzero density at the center:
\begin{equation}
    |\psi_{n,l}(0)|^2 = \begin{cases}
        \frac{1}{\pi a_0^3 n^3} & \text{if } l = 0 \\
        0 & \text{if } l > 0
    \end{cases}
\end{equation}
where $a_0$ is the characteristic length scale.
\end{theorem}

\begin{proof}
Boundaries with $l > 0$ have angular nodes---surfaces where the probability density vanishes. At $r = 0$, all $l > 0$ boundaries pass through a node (the angular structure requires at least one nodal plane through the origin).

Only $l = 0$ boundaries are spherically symmetric with no nodes, allowing nonzero density at $r = 0$.
\end{proof}

\subsection{The 21 cm Transition}

\begin{theorem}[Ground State Hyperfine Splitting]
\label{thm:21cm_derivation}
For a single-partition configuration ($Z = 1$) in its ground state $(n=1, l=0, m=0)$, the hyperfine energy splitting is:
\begin{equation}
    \Delta E_{\text{hf}} = 5.87 \times 10^{-6} \text{ eV}
\end{equation}
\end{theorem}

\begin{proof}
For the ground state of the simplest partition configuration:
\begin{itemize}
    \item $n = 1, l = 0$ gives $|\psi(0)|^2 = 1/(\pi a_0^3)$
    \item The boundary chirality moment is $\mu_s = g_s \mu_B$ where $\mu_B$ is the Bohr magneton
    \item The center chirality moment is $\mu_c = g_c \mu_N$ where $\mu_N$ is the nuclear magneton
    \item $\mu_N / \mu_B \approx 1/1836$ (the mass ratio)
\end{itemize}

Substituting:
\begin{equation}
    A = \frac{8\pi}{3} g_s g_c \frac{\mu_B^2}{1836} \cdot \frac{1}{\pi a_0^3} = \frac{8 g_s g_c \mu_B^2}{3 \cdot 1836 \cdot a_0^3}
\end{equation}

With $g_s \approx 2$ and $g_c \approx 5.59$:
\begin{equation}
    \Delta E_{\text{hf}} = \frac{A}{2} \approx 5.87 \times 10^{-6} \text{ eV}
\end{equation}
\end{proof}

\begin{corollary}[Transition Frequency and Wavelength]
\label{cor:21cm_frequency}
The hyperfine transition has:
\begin{align}
    \text{Frequency:} \quad \nu &= \frac{\Delta E_{\text{hf}}}{h} = 1420.405 \text{ MHz} \\
    \text{Wavelength:} \quad \lambda &= \frac{c}{\nu} = 21.106 \text{ cm}
\end{align}
\end{corollary}

\subsection{Virtual NMR Measurement}

\begin{definition}[Virtual Nuclear Magnetic Resonance]
\label{def:virtual_nmr}
A \emph{virtual NMR spectrometer} measures center chirality by:
\begin{enumerate}
    \item Applying an oscillating magnetic field at frequency $\nu$
    \item Detecting resonance when $\nu$ matches $\Delta E_{\text{hf}} / h$
    \item Recording the transition between $F = 1$ and $F = 0$ states
\end{enumerate}
\end{definition}

\begin{theorem}[NMR Measures Center Chirality]
\label{thm:nmr_chirality}
The NMR spectrum encodes the center chirality $s_c$ and its coupling to boundary chirality $s$:
\begin{itemize}
    \item \textbf{Chemical shift}: Position relative to reference frequency reveals local partition environment
    \item \textbf{Spin-spin coupling}: Splitting pattern reveals coupling between multiple centers
    \item \textbf{Relaxation times}: Decay rates reveal center dynamics
\end{itemize}
\end{theorem}

\subsection{Generalisation to Multi-Partition Systems}

\begin{theorem}[Multi-Center Coupling]
\label{thm:multi_center}
For a $Z$-partition configuration with $Z$ centers (each with chirality $s_{c,i}$) and $Z$ boundaries (each with chirality $s_j$), the hyperfine energy is:
\begin{equation}
    E_{\text{hf}} = \sum_{i,j} A_{ij} \, \mathbf{s}_{c,i} \cdot \mathbf{s}_j + \sum_{i < j} J_{ij} \, \mathbf{s}_{c,i} \cdot \mathbf{s}_{c,j}
\end{equation}
where:
\begin{itemize}
    \item $A_{ij}$ is the coupling between center $i$ and boundary $j$
    \item $J_{ij}$ is the direct coupling between centers $i$ and $j$
\end{itemize}
\end{theorem}

\begin{remark}[Structural Similarity]
The hyperfine splitting derived here is identical to the atomic hyperfine structure:
\begin{itemize}
    \item Boundary chirality $s$ corresponds to electron spin
    \item Center chirality $s_c$ corresponds to nuclear spin
    \item The 21 cm line (1420 MHz) is the hydrogen hyperfine transition used in radio astronomy
    \item NMR spectroscopy measures transitions between center chirality states
\end{itemize}
This demonstrates that partition coordinate theory can derive not only the gross structure of elements (shells, subshells, filling order) but also the fine and hyperfine structure arising from chirality couplings. The framework naturally extends to nuclear magnetic resonance phenomena.
\end{remark}


\section{Group and Period Characterisation}
\label{sec:group_period_characterisation}

We demonstrate partition coordinate characterisation for specific groups and periods using the full instrument ensemble. Each element is validated by multiple independent measurement methods.

\subsection{Period 1: The Simplest Configurations}

\begin{theorem}[Complete Characterisation: $Z = 1$]
\label{thm:char_z1}
\textbf{Partition Count:} $Z = 1$ (single partition configuration)

\begin{center}
\begin{tabular}{ll}
\toprule
\textbf{Instrument} & \textbf{Measurement} \\
\midrule
\multicolumn{2}{l}{\textit{Exotic Instruments}} \\
Shell Resonator & $n = 1$ (single depth) \\
Angular Analyser & $l = 0$ (spherical symmetry) \\
Chirality Discriminator & $s = \pm\frac{1}{2}$ (one boundary) \\
\midrule
\multicolumn{2}{l}{\textit{Standard Instruments}} \\
Mass Spectrometer & $m/z = 1.008$, $E_I = 13.60$ eV \\
UV-Vis Spectrometer & Lyman series: $\lambda = 121.5, 102.5, 97.2$ nm \\
NMR Spectrometer & $^1$H: $s_c = +\frac{1}{2}$, hyperfine = 1420 MHz \\
\midrule
\multicolumn{2}{l}{\textit{Computed}} \\
Partition Signature & $(1, 0, 0, \pm\frac{1}{2})$ \\
Ground State & $1s^1$ \\
\bottomrule
\end{tabular}
\end{center}

\paragraph{Spectral Validation:}
The Lyman series wavelengths follow exactly from:
\begin{equation}
    \frac{1}{\lambda} = R_\infty \left( 1 - \frac{1}{n^2} \right) \quad \text{for } n = 2, 3, 4, \ldots
\end{equation}
confirming $n = 1$ ground state.
\end{theorem}

\begin{theorem}[Complete Characterisation: $Z = 2$]
\label{thm:char_z2}
\textbf{Partition Count:} $Z = 2$ (complete first shell)

\begin{center}
\begin{tabular}{ll}
\toprule
\textbf{Instrument} & \textbf{Measurement} \\
\midrule
Shell Resonator & $n = 1$ (both boundaries at same depth) \\
Angular Analyser & $l = 0$ (both spherical) \\
Chirality Discriminator & $s = +\frac{1}{2}, -\frac{1}{2}$ (paired) \\
Mass Spectrometer & $m/z = 4.003$, $E_I = 24.59$ eV \\
UV-Vis Spectrometer & First line: 58.4 nm (extreme UV) \\
ESR Spectrometer & No signal (all chiralities paired) \\
\midrule
Partition Signature & $\{(1,0,0,+\frac{1}{2}), (1,0,0,-\frac{1}{2})\}$ \\
Ground State & $1s^2$ (complete shell) \\
\bottomrule
\end{tabular}
\end{center}

\paragraph{Stability Validation:}
Exceptionally high $E_I = 24.59$ eV confirms complete shell stability. Zero ESR signal confirms all chiralities paired.
\end{theorem}

\subsection{Period 2: Building Complexity}

\begin{theorem}[Group 1 Characterisation: $Z = 3$]
\label{thm:char_z3}
\textbf{Alkali metal configuration}

\begin{center}
\begin{tabular}{ll}
\toprule
\textbf{Measurement} & \textbf{Result} \\
\midrule
Shell Resonator & $n = 1$ (2 boundaries), $n = 2$ (1 boundary) \\
Angular Analyser & All $l = 0$ \\
Chirality Discriminator & 1 unpaired at $n = 2$ \\
Mass Spectrometer & $E_I = 5.39$ eV (very low) \\
XPS & $1s$: 55 eV, $2s$: 5.4 eV \\
ESR & Signal present (unpaired chirality) \\
\midrule
Configuration & $1s^2 2s^1$ \\
Valence & 1 (single $2s$ boundary) \\
\bottomrule
\end{tabular}
\end{center}
\end{theorem}

\begin{theorem}[Group 17 Characterisation: $Z = 9$]
\label{thm:char_z9}
\textbf{Halogen configuration}

\begin{center}
\begin{tabular}{ll}
\toprule
\textbf{Measurement} & \textbf{Result} \\
\midrule
Shell Resonator & $n = 1$ (2), $n = 2$ (7) \\
Angular Analyser & $l = 0$ (4), $l = 1$ (5) \\
Chirality Discriminator & 1 unpaired in $2p$ \\
Mass Spectrometer & $E_I = 17.42$ eV (high) \\
XPS & $1s$: 697 eV, $2s$: 34 eV, $2p$: 17 eV \\
ESR & Signal present (one unpaired) \\
NMR ($^{19}$F) & $s_c = +\frac{1}{2}$, high sensitivity \\
\midrule
Configuration & $1s^2 2s^2 2p^5$ \\
Valence & 1 (one vacancy in $2p$) \\
\bottomrule
\end{tabular}
\end{center}
\end{theorem}

\begin{theorem}[Group 18 Characterisation: $Z = 10$]
\label{thm:char_z10}
\textbf{Noble gas configuration}

\begin{center}
\begin{tabular}{ll}
\toprule
\textbf{Measurement} & \textbf{Result} \\
\midrule
Shell Resonator & $n = 1$ (2), $n = 2$ (8) \\
Angular Analyser & $l = 0$ (4), $l = 1$ (6) \\
Chirality Discriminator & All paired \\
Mass Spectrometer & $E_I = 21.56$ eV (very high) \\
XPS & $1s$: 870 eV, $2s$: 48 eV, $2p$: 22 eV \\
ESR & No signal (all paired) \\
\midrule
Configuration & $1s^2 2s^2 2p^6$ \\
Valence & 0 (complete shell) \\
\bottomrule
\end{tabular}
\end{center}
\end{theorem}

\subsection{Period 4: Transition Elements}

\begin{theorem}[First Transition Series: $Z = 21$ to $Z = 30$]
\label{thm:transition_series}
The $3d$ subshell fills across the first transition series:

\begin{center}
\begin{tabular}{cccccc}
\toprule
$Z$ & Config. & Unpaired & $\mu$ ($\mu_B$) & $E_I$ (eV) & Colour \\
\midrule
21 & $3d^1 4s^2$ & 1 & 1.7 & 6.56 & -- \\
22 & $3d^2 4s^2$ & 2 & 2.8 & 6.83 & -- \\
23 & $3d^3 4s^2$ & 3 & 3.9 & 6.75 & -- \\
24 & $3d^5 4s^1$ & 6 & 4.9 & 6.77 & -- \\
25 & $3d^5 4s^2$ & 5 & 5.9 & 7.43 & Pink \\
26 & $3d^6 4s^2$ & 4 & 4.9 & 7.90 & Blue/Green \\
27 & $3d^7 4s^2$ & 3 & 3.9 & 7.88 & Pink \\
28 & $3d^8 4s^2$ & 2 & 2.8 & 7.64 & Green \\
29 & $3d^{10} 4s^1$ & 1 & 1.7 & 7.73 & Blue \\
30 & $3d^{10} 4s^2$ & 0 & 0 & 9.39 & -- \\
\bottomrule
\end{tabular}
\end{center}

\paragraph{Validation:}
\begin{itemize}
    \item Magnetic moment $\mu = \sqrt{n(n+2)} \mu_B$ matches unpaired count
    \item Anomalies at $Z = 24, 29$ (half-filled/filled $3d$ stability)
    \item Colours arise from $d$-$d$ transitions within the $l = 2$ manifold
\end{itemize}
\end{theorem}

\subsection{Group Trends Across Periods}

\begin{theorem}[Group 1 (Alkali Metals) Validation]
\label{thm:group1_validation}
All Group 1 elements share configuration $[\text{core}] ns^1$:

\begin{center}
\begin{tabular}{ccccccc}
\toprule
$Z$ & Period & Config. & $E_I$ (eV) & Radius (pm) & $\chi$ & NMR \\
\midrule
3 & 2 & $2s^1$ & 5.39 & 152 & 0.98 & $^7$Li \\
11 & 3 & $3s^1$ & 5.14 & 186 & 0.93 & $^{23}$Na \\
19 & 4 & $4s^1$ & 4.34 & 227 & 0.82 & $^{39}$K \\
37 & 5 & $5s^1$ & 4.18 & 248 & 0.82 & $^{87}$Rb \\
55 & 6 & $6s^1$ & 3.89 & 265 & 0.79 & $^{133}$Cs \\
\bottomrule
\end{tabular}
\end{center}

\paragraph{Trend Validation:}
\begin{itemize}
    \item $E_I$ decreases with $n$ (larger $n \Rightarrow$ weaker binding)
    \item Radius increases with $n$ (larger orbits)
    \item Affinity $\chi$ decreases with $n$
    \item All have single valence boundary at outermost $s$ level
\end{itemize}
\end{theorem}

\begin{theorem}[Group 18 (Noble Gases) Validation]
\label{thm:group18_validation}
All Group 18 elements have complete shells:

\begin{center}
\begin{tabular}{cccccc}
\toprule
$Z$ & Period & Config. & $E_I$ (eV) & Radius (pm) & Reactivity \\
\midrule
2 & 1 & $1s^2$ & 24.59 & 31 & None \\
10 & 2 & $2p^6$ & 21.56 & 38 & None \\
18 & 3 & $3p^6$ & 15.76 & 71 & None \\
36 & 4 & $4p^6$ & 14.00 & 88 & Minimal \\
54 & 5 & $5p^6$ & 12.13 & 108 & Low \\
86 & 6 & $6p^6$ & 10.75 & 120 & Low \\
\bottomrule
\end{tabular}
\end{center}

\paragraph{Trend Validation:}
\begin{itemize}
    \item All have complete outermost shell (zero valence)
    \item Exceptionally high $E_I$ (complete shell stability)
    \item ESR: No signal for any (all chiralities paired)
\end{itemize}
\end{theorem}

\subsection{Instrument Consistency Verification}

\begin{theorem}[Cross-Instrument Agreement]
\label{thm:cross_instrument}
For all elements characterised, the instruments agree:

\begin{enumerate}
    \item \textbf{Shell Resonator} $\leftrightarrow$ \textbf{XPS}: Both identify occupied $n$ values
    \item \textbf{Angular Analyser} $\leftrightarrow$ \textbf{XPS fine structure}: Both identify $l$ values
    \item \textbf{Chirality Discriminator} $\leftrightarrow$ \textbf{ESR}: Both count unpaired $s$
    \item \textbf{All spectroscopy} $\leftrightarrow$ \textbf{Rydberg formula}: Transition energies match
\end{enumerate}

This multi-instrument agreement demonstrates that partition coordinates are physical invariants measured by standard chemistry techniques.
\end{theorem}

\begin{remark}[Structural Similarity]
The characterisation tables above reproduce the known periodic table:
\begin{itemize}
    \item Period lengths (2, 8, 8, 18, 18, 32) match shell capacities
    \item Group properties (alkali reactivity, noble gas inertness) follow from occupancy
    \item Transition metal magnetism follows from unpaired $d$-chiralities
    \item All ionisation energies, radii, and affinities match published values
\end{itemize}
This is not a fit to data---it is a derivation from partition geometry that happens to reproduce chemistry.
\end{remark}

\section{Categorical State Synthesiser}
\label{sec:synthesiser}

We describe the inverse of measurement: constructing elements by specifying their target partition coordinates. The synthesiser generates physical realisations from categorical specifications.

\subsection{Synthesis as Inverse Measurement}

\begin{definition}[Categorical Synthesis]
\label{def:cat_synthesis}
\emph{Categorical synthesis} is the process of generating a physical system from a specified partition coordinate signature:
\begin{equation}
    \text{Synthesis}: \mathcal{E}_Z \rightarrow \text{Physical system with signature } \mathcal{E}_Z
\end{equation}
This is the inverse of measurement, which extracts $\mathcal{E}_Z$ from a physical system.
\end{definition}

\begin{theorem}[Synthesis Protocol]
\label{thm:synthesis_protocol}
To synthesise an element with partition count $Z$:
\begin{enumerate}
    \item \textbf{Specify target}: Define $\mathcal{E}_Z = \{(n_i, l_i, m_i, s_i)\}_{i=1}^Z$
    \item \textbf{Verify constraints}: Ensure all coordinates satisfy geometric bounds
    \item \textbf{Apply filling order}: Occupy coordinates in energy-minimising sequence
    \item \textbf{Verify exclusion}: No duplicate coordinates
    \item \textbf{Generate}: Create physical realisation
\end{enumerate}
\end{theorem}

\subsection{Constructing Example Elements}

\begin{theorem}[Synthesising $Z = 1$ (Hydrogen-like)]
\label{thm:synth_z1}
\textbf{Target}: Single-partition configuration

\textbf{Specification}:
\begin{align}
    Z &= 1 \\
    \mathcal{E}_1 &= \{(1, 0, 0, +\tfrac{1}{2})\} \quad \text{or} \quad \{(1, 0, 0, -\tfrac{1}{2})\}
\end{align}

\textbf{Properties from partition geometry}:
\begin{itemize}
    \item Binding energy: $E_B = E_0 / 1^2 = E_0 = 13.6$ eV
    \item Characteristic radius: $r_1 = a_0 = 52.9$ pm
    \item Hyperfine splitting: $\Delta E_{\text{hf}} = 5.87 \times 10^{-6}$ eV (21 cm line)
    \item Single valence: one partially filled shell
\end{itemize}
\end{theorem}

\begin{theorem}[Synthesising $Z = 6$ (Carbon-like)]
\label{thm:synth_z6}
\textbf{Target}: Six-partition configuration

\textbf{Specification}:
\begin{align}
    Z &= 6 \\
    \mathcal{E}_6 &= \{(1,0,0,+\tfrac{1}{2}), (1,0,0,-\tfrac{1}{2}), (2,0,0,+\tfrac{1}{2}), (2,0,0,-\tfrac{1}{2}), \\
    &\phantom{= \{} (2,1,-1,+\tfrac{1}{2}), (2,1,0,+\tfrac{1}{2})\} \quad \text{(Hund's rule configuration)}
\end{align}

\textbf{Properties from partition geometry}:
\begin{itemize}
    \item Core: Complete $n = 1$ shell (2 states)
    \item Valence: 4 states in $n = 2$ ($2s^2 2p^2$)
    \item Binding energy (first valence): $\approx 11.3$ eV
    \item Four valence bonds possible (tetrahedral geometry)
\end{itemize}
\end{theorem}

\begin{theorem}[Synthesising $Z = 26$ (Iron-like)]
\label{thm:synth_z26}
\textbf{Target}: 26-partition configuration

\textbf{Specification}:
\begin{align}
    Z &= 26 \\
    \mathcal{E}_{26} &= [\text{Ar}]_{18} + 3d^6 4s^2
\end{align}

\textbf{Detailed coordinate list}:
\begin{itemize}
    \item $n = 1$: 2 states (complete)
    \item $n = 2$: 8 states (complete)
    \item $n = 3$: $3s^2 3p^6 3d^6 = 8 + 6 = 14$ states ($3d$ partially filled)
    \item $n = 4$: $4s^2 = 2$ states
\end{itemize}

\textbf{Properties from partition geometry}:
\begin{itemize}
    \item Transition element (partial $d$-subshell)
    \item Multiple oxidation states (variable $d$-occupancy)
    \item Magnetic: unpaired $d$-boundary chiralities
\end{itemize}
\end{theorem}

\subsection{Property Prediction from Coordinates}

\begin{theorem}[Properties from Signature]
\label{thm:property_prediction}
Given an element signature $\mathcal{E}_Z$, all properties can be predicted:

\paragraph{Binding energy:}
\begin{equation}
    E_B^{(i)} = \frac{E_0 \cdot Z_{\text{eff},i}^2}{n_i^2}
\end{equation}

\paragraph{Characteristic radius:}
\begin{equation}
    r_Z = \frac{a_0 \cdot n_{\text{max}}^2}{Z_{\text{eff}}}
\end{equation}

\paragraph{Boundary affinity:}
\begin{equation}
    \chi = \frac{E_B + E_A}{2}
\end{equation}

\paragraph{Valence:}
\begin{equation}
    V = \min(N_{\text{valence}}, 8 - N_{\text{valence}})
\end{equation}
where $N_{\text{valence}}$ is the count of states in the outermost incomplete shell.
\end{theorem}

\subsection{Synthesis Constraints}

\begin{theorem}[Realisability Constraints]
\label{thm:realisability}
Not all coordinate specifications are physically realisable:
\begin{enumerate}
    \item \textbf{Geometric constraints}: $l < n$, $|m| \leq l$, $s = \pm\frac{1}{2}$
    \item \textbf{Exclusion}: No duplicate coordinates
    \item \textbf{Energy ordering}: Ground state follows $(n + l)$ rule
    \item \textbf{Stability}: $Z \leq Z_{\text{max}} \approx 118$ (beyond this, binding insufficient)
\end{enumerate}
\end{theorem}

\begin{theorem}[Excited State Synthesis]
\label{thm:excited_states}
Non-ground-state configurations can be synthesised by violating the energy ordering:
\begin{equation}
    \mathcal{E}_Z^* = \mathcal{E}_Z \text{ with one or more coordinates promoted}
\end{equation}
These excited states have higher energy and decay to ground state by emitting spectral radiation.
\end{theorem}

\subsection{Multi-Element Systems}

\begin{theorem}[Molecular Synthesis]
\label{thm:molecular_synthesis}
Multiple elements can be combined by coupling their outermost partition boundaries:
\begin{equation}
    \mathcal{M} = \mathcal{E}_{Z_1} \cup \mathcal{E}_{Z_2} + \text{coupling terms}
\end{equation}

Coupling occurs when:
\begin{itemize}
    \item Boundaries from different elements overlap
    \item Chiralities pair (opposite $s$ values)
    \item Energy is lowered by sharing boundaries
\end{itemize}
\end{theorem}

\begin{remark}[Structural Similarity]
The categorical state synthesiser provides:
\begin{itemize}
    \item \textbf{Element construction}: Build any element by specifying its electron configuration
    \item \textbf{Property prediction}: Calculate all properties from the configuration
    \item \textbf{Molecular design}: Combine elements through boundary coupling
\end{itemize}
This mirrors computational chemistry, where electronic structure calculations predict properties from quantum mechanical wave functions. The partition coordinate approach shows that these calculations are fundamentally geometric---they map out the structure of partition space.
\end{remark}


\section{Categorical Instrument Orchestration}
\label{sec:instrument_orchestration}

We demonstrate that the instrument ensemble constitutes a Poincaré machine---a computational system where solutions are trajectories through categorical space that return to their origin. Element identification becomes trajectory completion; instrument agreement becomes recurrence.

\subsection{Instruments as Projection Operators}

\begin{definition}[Instrument Projection]
\label{def:instrument_projection}
Each instrument $\Pi_j$ is a \emph{projection operator} on categorical space $\mathcal{S}$:
\begin{align}
    \Pi_{\text{MS}} &: \mathcal{S} \to (m/z, E_I) \quad \text{[Mass Spectrometer]} \\
    \Pi_{\text{XPS}} &: \mathcal{S} \to \{E_B(n,l)\} \quad \text{[X-ray Photoelectron]} \\
    \Pi_{\text{NMR}} &: \mathcal{S} \to (\delta, J) \quad \text{[Nuclear Magnetic Resonance]} \\
    \Pi_{\text{ESR}} &: \mathcal{S} \to (g, n_{\text{unpaired}}) \quad \text{[Electron Spin Resonance]} \\
    \Pi_{\text{UV}} &: \mathcal{S} \to \{\lambda_{\text{transition}}\} \quad \text{[UV-Vis Spectroscopy]}
\end{align}
Each projection extracts a different aspect of the partition coordinates $(n, l, m, s)$.
\end{definition}

\begin{theorem}[Projection Consistency]
\label{thm:projection_consistency}
For a valid categorical state $\mathbf{S} \in \mathcal{S}$, all projections must be mutually consistent:
\begin{equation}
    \Pi_{\text{MS}}(\mathbf{S}) \wedge \Pi_{\text{XPS}}(\mathbf{S}) \wedge \Pi_{\text{NMR}}(\mathbf{S}) \wedge \Pi_{\text{ESR}}(\mathbf{S}) \to \text{unique } (n, l, m, s)
\end{equation}
Inconsistency indicates either measurement error or an exotic state.
\end{theorem}

\subsection{Trajectory Through Instrument Space}

\begin{definition}[Instrument Trajectory]
\label{def:instrument_trajectory}
An \emph{instrument trajectory} $\gamma$ is a sequence of projections applied to a sample:
\begin{equation}
    \gamma = (\Pi_{j_1}, \Pi_{j_2}, \ldots, \Pi_{j_k})
\end{equation}
Each projection refines the knowledge of the partition coordinates.
\end{definition}

\begin{theorem}[Trajectory Convergence]
\label{thm:trajectory_convergence}
A trajectory $\gamma$ \emph{converges} when all projections agree on the same partition coordinates:
\begin{equation}
    \text{Converged}(\gamma) \iff \forall i, j: \Pi_i(\mathbf{S}) \text{ consistent with } \Pi_j(\mathbf{S})
\end{equation}
Convergence is the \emph{recurrence condition}---the trajectory returns to a self-consistent state.
\end{theorem}

\subsection{The Categorical Identification Algorithm}

\begin{definition}[Categorical Element Identification]
\label{def:cat_identification}
\begin{enumerate}
    \item \textbf{Initialise}: Place unknown sample in categorical space $\mathbf{S}_0$
    \item \textbf{Project}: Apply instrument $\Pi_j$ to extract partial coordinates
    \item \textbf{Propagate}: Use extracted information to constrain other projections
    \item \textbf{Iterate}: Select next instrument based on information gain
    \item \textbf{Converge}: Terminate when all projections agree
\end{enumerate}
The output is the consensus partition signature $\mathcal{E}_Z$.
\end{definition}

\begin{theorem}[Information Gain Routing]
\label{thm:info_gain}
The optimal instrument sequence minimises the number of projections to convergence. The information gain $I(\Pi_j | \text{current knowledge})$ determines the next instrument:

\begin{center}
\begin{tabular}{lll}
\toprule
Current Knowledge & Best Next Instrument & Information Gained \\
\midrule
Nothing & Mass Spectrometer & $Z$ (partition count) \\
$Z$ known & XPS & All $(n, l)$ binding energies \\
$(n, l)$ known & ESR & Unpaired $s$ count \\
$s$ distribution known & NMR & Hyperfine confirmation \\
All coordinates & UV-Vis & Transition verification \\
\bottomrule
\end{tabular}
\end{center}
\end{theorem}

\subsection{Poincaré Complexity of Element Identification}

\begin{definition}[Poincaré Complexity]
\label{def:poincare_complexity}
The \emph{Poincaré complexity} $\Pi(Z)$ of identifying element $Z$ is the minimum number of instrument projections required for convergence:
\begin{equation}
    \Pi(Z) = \min_{|\gamma|} \{|\gamma| : \gamma \text{ converges to } \mathcal{E}_Z\}
\end{equation}
\end{definition}

\begin{theorem}[Complexity Bounds]
\label{thm:complexity_bounds}
For elements with partition count $Z$:
\begin{align}
    \Pi(Z) &\geq 2 \quad \text{(at least 2 instruments for cross-validation)} \\
    \Pi(Z) &\leq 5 \quad \text{(all instrument categories)}
\end{align}
Typical complexity: $\Pi(Z) = 3$ for main group elements, $\Pi(Z) = 4$ for transition elements.
\end{theorem}

\begin{proof}
Main group elements have simple configurations determinable from:
\begin{enumerate}
    \item Mass spectrometer: $Z$ and valence $(n, l)$
    \item XPS: Complete $(n, l)$ configuration
    \item ESR: Unpaired $s$ confirmation
\end{enumerate}

Transition elements require additional discrimination of $d$-shell occupancy, adding one projection.
\end{proof}

\subsection{Recurrence as Solution Recognition}

\begin{theorem}[Recurrence Condition]
\label{thm:recurrence_solution}
Element identification is complete when the instrument trajectory exhibits \emph{recurrence}:
\begin{equation}
    \|\mathbf{S}_{\text{final}} - \mathbf{S}_0\| < \epsilon
\end{equation}
where $\epsilon$ is the measurement precision. This is the $\epsilon$-boundary condition from Poincaré Computing.
\end{theorem}

\begin{proof}
The initial state $\mathbf{S}_0$ encodes the true partition coordinates (unknown to the observer). Each projection refines the estimate. Convergence occurs when the estimated state is within $\epsilon$ of the true state---recognisable because all projections become consistent.

The system cannot reach \emph{exact} recurrence because:
\begin{enumerate}
    \item Measurement has finite precision
    \item The categorical state has been ``visited'' (irreversibility)
\end{enumerate}
Thus solutions are recognised at the $\epsilon$-boundary, exactly as in Poincaré Computing.
\end{proof}

\subsection{Constraint Propagation}

\begin{theorem}[Constraint Propagation]
\label{thm:constraint_propagation}
Each projection constrains subsequent projections through physical laws:
\begin{align}
    \Pi_{\text{MS}} \to Z &\Rightarrow \text{XPS must show exactly } Z \text{ core levels} \\
    \Pi_{\text{XPS}} \to (n, l) &\Rightarrow \text{ESR must show correct unpaired count} \\
    \Pi_{\text{ESR}} \to s &\Rightarrow \text{NMR must show consistent hyperfine}
\end{align}
Constraints propagate bidirectionally: later measurements constrain interpretation of earlier measurements.
\end{theorem}

\subsection{Experimental Validation}

\begin{theorem}[Convergence Demonstration]
\label{thm:convergence_demo}
Experimental trajectories for known elements demonstrate convergence:

\paragraph{Carbon ($Z = 6$):}
\begin{enumerate}
    \item $\Pi_{\text{MS}}$: $m/z = 12$, $E_I = 11.3$ eV $\Rightarrow$ $Z = 6$, valence in $2p$
    \item $\Pi_{\text{XPS}}$: $E_B(1s) = 284$ eV $\Rightarrow$ Confirms $1s^2$
    \item $\Pi_{\text{ESR}}$: 2 unpaired $\Rightarrow$ Confirms $2p^2$
    \item Convergence achieved in 3 projections
\end{enumerate}

\paragraph{Iron ($Z = 26$):}
\begin{enumerate}
    \item $\Pi_{\text{MS}}$: $m/z = 56$, $E_I = 7.9$ eV $\Rightarrow$ $Z = 26$, valence in $4s$
    \item $\Pi_{\text{XPS}}$: All core levels $\Rightarrow$ Confirms $[\text{Ar}] 3d^6 4s^2$
    \item $\Pi_{\text{ESR}}$: 4 unpaired $\Rightarrow$ Confirms high-spin $3d^6$
    \item $\Pi_{\text{NMR}}$: $^{57}$Fe, $I = 1/2$ $\Rightarrow$ Confirms nuclear structure
    \item Convergence achieved in 4 projections
\end{enumerate}
\end{theorem}

\begin{figure}[htbp]
\centering
\includegraphics[width=\textwidth]{figures/instrument_orchestration_panel.png}
\caption{\textbf{Categorical Instrument Orchestration.} \textbf{(A)} Instruments as projections: each instrument projects the categorical state onto a measurement subspace, extracting different aspects of $(n, l, m, s)$. \textbf{(B)} Trajectory through instrument space: the identification process is a path through projections, converging when all agree. \textbf{(C)} Information gain routing: optimal instrument selection based on current knowledge. \textbf{(D)} Convergence dynamics: agreement between instruments increases with each projection until recurrence is achieved. \textbf{(E)} Poincaré complexity: minimum projections needed for different element types. \textbf{(F)} Constraint propagation: each measurement constrains subsequent measurements through physical consistency.}
\label{fig:instrument_orchestration}
\end{figure}

\begin{remark}[Connection to Poincaré Computing]
The instrument ensemble is a physical instantiation of Poincaré Computing:
\begin{itemize}
    \item \textbf{Phase space} $\mathcal{S}$: Space of all partition configurations
    \item \textbf{Trajectory} $\gamma$: Sequence of instrument measurements
    \item \textbf{Recurrence}: All instruments agreeing on coordinates
    \item \textbf{$\epsilon$-boundary}: Measurement precision limit
    \item \textbf{Constraints} $\mathcal{C}$: Physical laws (Pauli, selection rules)
    \item \textbf{Complexity} $\Pi$: Minimum instruments for convergence
\end{itemize}
Standard analytical chemistry has been performing Poincaré computation without recognising it as such.
\end{remark}


\section{Universal Virtual Instrument Algorithm}
\label{sec:algorithm}

We present a systematic procedure for constructing optimal virtual instruments from arbitrary hardware. The algorithm takes as input a set of available oscillators and desired measurement targets, and outputs an instrument configuration, measurement protocol, and coordinate extraction procedure.

\subsection{The Virtual Instrument Construction Problem}
\label{subsec:construction_problem}

\begin{definition}[Virtual Instrument Construction Problem]
\label{def:vicp}
Given:
\begin{itemize}
    \item $\mathcal{H} = \{h_1, h_2, \ldots, h_N\}$: a set of available hardware oscillators
    \item $\mathcal{T} = \{t_1, t_2, \ldots, t_M\}$: target partition coordinates to measure
    \item $\mathcal{P} = \{\sigma_1, \sigma_2, \ldots, \sigma_M\}$: precision requirements (uncertainty bounds)
    \item $\mathcal{C}$: constraints (time budget, cost budget, complexity limits)
\end{itemize}
Find:
\begin{itemize}
    \item $\mathcal{I} \subseteq \mathcal{H}$: optimal instrument configuration
    \item $\Pi$: measurement protocol (excitation sequences, timing, acquisition)
    \item $\mathcal{E}$: coordinate extraction procedure (deconvolution, corrections, error propagation)
    \item $\mathcal{U}$: achieved uncertainty bounds
\end{itemize}
Such that:
\begin{itemize}
    \item $\mathcal{U} \leq \mathcal{P}$ (precision requirements met)
    \item $\text{cost}(\mathcal{I}, \Pi) \leq \mathcal{C}$ (constraints satisfied)
    \item $\mathcal{I}$ is minimal (no redundant hardware)
\end{itemize}
\end{definition}

\begin{remark}
This is an inverse problem: given desired measurements, find the hardware configuration that achieves them. The solution is not unique---multiple instrument configurations may achieve the same measurement goals with different trade-offs.
\end{remark}

\subsection{Physical Basis: Hardware Oscillation Hierarchies}
\label{subsec:oscillation_hierarchies}

\begin{theorem}[Hardware Oscillation Hierarchy]
\label{thm:hardware_hierarchy}
Every physical measurement apparatus contains a nested hierarchy of oscillatory modes. These modes form a partially ordered set under frequency ordering.
\end{theorem}

\begin{proof}
Any physical system with bounded energy has discrete oscillatory modes (by the spectral theorem for bounded operators). These modes have characteristic frequencies $\{\omega_1, \omega_2, \ldots\}$.

For measurement to occur, the apparatus must couple to the measured system. This coupling occurs when apparatus frequencies match or are harmonically related to system frequencies. The set of apparatus frequencies thus forms a hierarchy:
\begin{equation}
    \omega_{\text{apparatus}} = \{n_1 \omega_1, n_2 \omega_2, \ldots\} \quad \text{where } n_i \in \mathbb{Z}^+
\end{equation}

This hierarchy is partially ordered by divisibility: $\omega_i \preceq \omega_j$ if $\omega_j = n\omega_i$ for some integer $n$.
\end{proof}

\begin{definition}[Oscillation Signature]
\label{def:oscillation_signature}
The \emph{oscillation signature} of a hardware component $h$ is the set of frequencies it can generate or detect:
\begin{equation}
    \Omega(h) = \{\omega : h \text{ can generate or detect oscillations at frequency } \omega\}
\end{equation}
\end{definition}

\begin{example}[Mass Spectrometer Oscillation Signature]
A quadrupole mass filter has oscillation signature:
\begin{align}
    \Omega(\text{quadrupole}) = \{&\omega_{\text{RF}} \approx 10^6 \text{ Hz (radiofrequency drive)}, \\
    &\omega_{\text{ion}} \approx 10^7 \text{ Hz (ion cyclotron frequency)}, \\
    &\omega_{\text{detect}} \approx 10^8 \text{ Hz (detector response frequency)}\}
\end{align}
\end{example}

\subsection{Coordinate Accessibility}
\label{subsec:coordinate_accessibility}

\begin{definition}[Accessibility Function]
\label{def:accessibility}
The \emph{accessibility} of partition coordinate $t$ by hardware $h$ is:
\begin{equation}
    A(h, t) = \max_{\omega \in \Omega(h)} \left| \langle \omega | t \rangle \right|^2
\end{equation}
where $\langle \omega | t \rangle$ is the coupling strength between oscillation mode $\omega$ and coordinate $t$.
\end{definition}

\begin{theorem}[Coordinate-Frequency Coupling]
\label{thm:coordinate_frequency}
Partition coordinates couple to hardware frequencies according to:
\begin{align}
    \langle \omega | n \rangle &\propto \delta(\omega - \omega_n) \quad \text{(radial modes)} \\
    \langle \omega | l \rangle &\propto \delta(\omega - \omega_l) \quad \text{(angular modes)} \\
    \langle \omega | m \rangle &\propto \delta(\omega - m\omega_0) \quad \text{(orientation modes)} \\
    \langle \omega | s \rangle &\propto \delta(\omega - 2s\omega_s) \quad \text{(chirality modes)}
\end{align}
where $\omega_n, \omega_l, \omega_0, \omega_s$ are characteristic frequencies of the measured system.
\end{theorem}

\begin{proof}
Each partition coordinate corresponds to a specific oscillatory mode of the bounded system:
\begin{itemize}
    \item $n$: radial oscillation frequency $\omega_n \propto 1/n^2$ (from energy scaling)
    \item $l$: angular oscillation frequency $\omega_l \propto l(l+1)$ (from angular momentum)
    \item $m$: precession frequency $\omega_m = m\omega_0$ (from orientation quantization)
    \item $s$: spin precession frequency $\omega_s = 2s\omega_{\text{Larmor}}$ (from chirality)
\end{itemize}

Hardware couples to these modes when its oscillation signature overlaps with the system's characteristic frequencies. The coupling strength is proportional to the spectral overlap, giving the delta function form.
\end{proof}

\begin{corollary}[Accessibility Matrix]
\label{cor:accessibility_matrix}
For $N$ hardware components and $M$ target coordinates, define the accessibility matrix:
\begin{equation}
    \mathbf{A} = [A(h_i, t_j)]_{N \times M}
\end{equation}
Entry $A_{ij}$ quantifies how well hardware $h_i$ can measure coordinate $t_j$.
\end{corollary}

\subsection{Precision Estimation}
\label{subsec:precision_estimation}

\begin{definition}[Measurement Precision]
\label{def:measurement_precision}
The precision with which hardware $h$ can measure coordinate $t$ is:
\begin{equation}
    \sigma(h, t) = \frac{\sigma_{\text{noise}}(h)}{\sqrt{A(h, t) \cdot T_{\text{int}}}}
\end{equation}
where $\sigma_{\text{noise}}(h)$ is the intrinsic noise level of hardware $h$ and $T_{\text{int}}$ is the integration time.
\end{definition}

\begin{theorem}[Precision Scaling]
\label{thm:precision_scaling}
For hardware with accessibility $A$ and noise level $\sigma_{\text{noise}}$, the measurement precision scales as:
\begin{equation}
    \sigma(t) \propto \frac{\sigma_{\text{noise}}}{\sqrt{A \cdot T_{\text{int}} \cdot N_{\text{avg}}}}
\end{equation}
where $T_{\text{int}}$ is integration time and $N_{\text{avg}}$ is number of averages.
\end{theorem}

\begin{proof}
The signal-to-noise ratio for a measurement is:
\begin{equation}
    \text{SNR} = \frac{S}{\sigma_{\text{noise}}} = \frac{A \cdot \sqrt{T_{\text{int}}}}{\sigma_{\text{noise}}}
\end{equation}

The uncertainty in extracting coordinate $t$ from the signal is:
\begin{equation}
    \sigma(t) = \frac{1}{\text{SNR}} = \frac{\sigma_{\text{noise}}}{A \cdot \sqrt{T_{\text{int}}}}
\end{equation}

With $N_{\text{avg}}$ independent measurements, the uncertainty reduces by $\sqrt{N_{\text{avg}}}$:
\begin{equation}
    \sigma(t) = \frac{\sigma_{\text{noise}}}{\sqrt{A \cdot T_{\text{int}} \cdot N_{\text{avg}}}}
\end{equation}
\end{proof}

\subsection{The Universal Virtual Instrument Finder Algorithm}
\label{subsec:algorithm_steps}

\begin{algorithm}[H]
\caption{Universal Virtual Instrument Finder (UVIF)}
\label{alg:uvif}
\begin{algorithmic}[1]
\Require Hardware set $\mathcal{H}$, targets $\mathcal{T}$, precision $\mathcal{P}$, constraints $\mathcal{C}$
\Ensure Instrument config $\mathcal{I}$, protocol $\Pi$, extraction $\mathcal{E}$, uncertainties $\mathcal{U}$

\Statex
\State \textbf{Step 1: Hardware Characterization}
\For{each $h \in \mathcal{H}$}
    \State Measure frequency spectrum $\Omega(h)$
    \State Extract oscillation hierarchy
    \State Characterize noise profile $\sigma_{\text{noise}}(h)$
    \State Compute cost and time parameters
\EndFor

\Statex
\State \textbf{Step 2: Accessibility Analysis}
\For{each $h \in \mathcal{H}$}
    \For{each $t \in \mathcal{T}$}
        \State Compute coupling strength $\langle \omega | t \rangle$ for all $\omega \in \Omega(h)$
        \State Compute accessibility $A(h,t) = \max_\omega |\langle \omega | t \rangle|^2$
        \State Estimate precision $\sigma(h,t)$ using Theorem~\ref{thm:precision_scaling}
    \EndFor
\EndFor
\State Construct accessibility matrix $\mathbf{A}$

\Statex
\State \textbf{Step 3: Instrument Optimization}
\State Solve optimization problem:
\begin{equation*}
\begin{aligned}
    \max_{\mathcal{I} \subseteq \mathcal{H}} \quad & \sum_{h \in \mathcal{I}} \sum_{t \in \mathcal{T}} \frac{A(h,t)}{\sigma(h,t)} \\
    \text{subject to} \quad & \sigma(h,t) \leq \mathcal{P}(t) \quad \forall t \in \mathcal{T}, h \in \mathcal{I} \\
    & \text{cost}(\mathcal{I}) \leq \mathcal{C}_{\text{budget}} \\
    & \text{time}(\mathcal{I}) \leq \mathcal{C}_{\text{time}}
\end{aligned}
\end{equation*}
\State Output optimal configuration $\mathcal{I}^*$

\Statex
\State \textbf{Step 4: Protocol Generation}
\For{each $h \in \mathcal{I}^*$}
    \State Design excitation sequence to probe $\Omega(h)$
    \State Specify measurement windows and sampling rates
    \State Define data acquisition parameters
    \State Generate calibration procedure
\EndFor
\State Combine into protocol $\Pi$

\Statex
\State \textbf{Step 5: Extraction Procedure}
\For{each $t \in \mathcal{T}$}
    \State Identify contributing hardware: $\mathcal{H}_t = \{h \in \mathcal{I}^* : A(h,t) > 0\}$
    \State Design deconvolution algorithm for multi-hardware fusion
    \State Implement screening corrections (for multi-body systems)
    \State Compute error propagation: $\mathcal{U}(t) = f(\{\sigma(h,t)\}_{h \in \mathcal{H}_t})$
\EndFor
\State Combine into extraction procedure $\mathcal{E}$

\Statex
\State \textbf{Step 6: Validation}
\State Test on systems with known coordinates
\State Verify $\mathcal{U}(t) \leq \mathcal{P}(t)$ for all $t \in \mathcal{T}$
\If{validation fails}
    \State Relax constraints or add hardware
    \State Return to Step 3
\EndIf

\Statex
\State \Return $(\mathcal{I}^*, \Pi, \mathcal{E}, \mathcal{U})$
\end{algorithmic}
\end{algorithm}

\subsection{Optimization Criteria}
\label{subsec:optimization_criteria}

The optimization in Step 3 can be formulated as a multi-objective problem:

\begin{definition}[Instrument Quality Function]
\label{def:quality_function}
The quality of instrument configuration $\mathcal{I}$ for measuring targets $\mathcal{T}$ is:
\begin{equation}
    Q(\mathcal{I}, \mathcal{T}) = \sum_{t \in \mathcal{T}} w_t \cdot \max_{h \in \mathcal{I}} \left[ \frac{A(h,t)}{\sigma(h,t)} \right]
\end{equation}
where $w_t$ are target weights (importance factors).
\end{definition}

\begin{theorem}[Optimal Configuration Existence]
\label{thm:optimal_existence}
For finite hardware set $\mathcal{H}$ and finite target set $\mathcal{T}$, there exists an optimal configuration $\mathcal{I}^* \subseteq \mathcal{H}$ that maximizes $Q(\mathcal{I}, \mathcal{T})$ subject to constraints $\mathcal{C}$.
\end{theorem}

\begin{proof}
The feasible set $\mathcal{F} = \{\mathcal{I} \subseteq \mathcal{H} : \text{constraints satisfied}\}$ is finite (at most $2^{|\mathcal{H}|}$ subsets). The quality function $Q$ is bounded above (by maximum accessibility and minimum noise). Therefore, $Q$ attains its maximum on the compact set $\mathcal{F}$.
\end{proof}

\begin{remark}
Finding $\mathcal{I}^*$ is NP-hard in general (subset selection problem), but practical instances are small enough for exhaustive search or heuristic optimization (genetic algorithms, simulated annealing).
\end{remark}

\subsection{Example Application: Hydrogen Ground State}
\label{subsec:example_hydrogen}

We demonstrate the algorithm by constructing a virtual instrument to measure all partition coordinates $(n, l, m, s, s_c)$ of hydrogen's ground state.

\begin{example}[Single-Instrument Attempt: Mass Spectrometry]
\label{ex:hydrogen_mass_spec}

\textbf{Input:}
\begin{itemize}
    \item Hardware: Quadrupole mass spectrometer
    \item Targets: $\mathcal{T} = \{n, l\}$ (radial and angular coordinates)
    \item Precision: $\mathcal{P}(n) = 0.1$, $\mathcal{P}(l) = 0.05$
    \item Constraints: Single measurement, $< 1$ second
\end{itemize}

\textbf{Step 1: Hardware Characterization}
\begin{align}
    \Omega(\text{mass spec}) &= \{\omega_{\text{RF}}, \omega_{\text{ion}}, \omega_{\text{detect}}\} \\
    \omega_{\text{RF}} &\approx 10^6 \text{ Hz (radiofrequency drive)} \\
    \omega_{\text{ion}} &\approx 10^7 \text{ Hz (ion cyclotron)} \\
    \omega_{\text{detect}} &\approx 10^8 \text{ Hz (detector bandwidth)} \\
    \sigma_{\text{noise}} &\approx 0.01 \text{ eV (energy resolution)}
\end{align}

\textbf{Step 2: Accessibility Analysis}

For hydrogen ground state with ionization energy $E_{\text{ion}} = 13.6$ eV:
\begin{align}
    A(\text{mass spec}, n) &= \left| \frac{\partial E_{\text{ion}}}{\partial n} \right|^{-2} = \left| \frac{2R_\infty}{n^3} \right|^{-2} \\
    &= \frac{n^6}{4R_\infty^2} = \frac{1}{4 \cdot (13.6)^2} \approx 0.0014 \\
    A(\text{mass spec}, l) &\approx 0.0001 \quad \text{(weak, via fine structure)}
\end{align}

Precision estimates:
\begin{align}
    \sigma(n) &= \frac{\sigma_{\text{noise}}}{\sqrt{A(n) \cdot T_{\text{int}}}} = \frac{0.01}{\sqrt{0.0014 \cdot 1}} \approx 0.27 \\
    \sigma(l) &= \frac{0.01}{\sqrt{0.0001 \cdot 1}} \approx 1.0
\end{align}

\textbf{Step 3: Optimization Result}

Mass spectrometer alone achieves:
\begin{itemize}
    \item $\sigma(n) = 0.27 > \mathcal{P}(n) = 0.1$ \quad $\times$ (fails precision requirement)
    \item $\sigma(l) = 1.0 > \mathcal{P}(l) = 0.05$ \quad $\times$ (fails precision requirement)
\end{itemize}

\textbf{Conclusion:} Single instrument insufficient. Need multi-instrument configuration.
\end{example}

\begin{example}[Multi-Instrument Configuration for Hydrogen]
\label{ex:hydrogen_multi}

\textbf{Input:}
\begin{itemize}
    \item Hardware: $\mathcal{H} = \{\text{mass spec}, \text{UV-Vis}, \text{NMR}\}$
    \item Targets: $\mathcal{T} = \{n, l, m, s, s_c\}$ (all coordinates)
    \item Precision: $\mathcal{P} = 0.01$ for all
\end{itemize}

\textbf{Accessibility Matrix:}
\begin{equation}
\mathbf{A} = \begin{array}{c|ccccc}
    & n & l & m & s & s_c \\
    \hline
    \text{Mass spec} & 0.001 & 0.0001 & 0 & 0 & 0 \\
    \text{UV-Vis} & 0.1 & 0.1 & 0.01 & 0 & 0 \\
    \text{NMR} & 0.01 & 0.001 & 0.1 & 0.1 & 0.1
\end{array}
\end{equation}

\textbf{Optimal Configuration:}
\begin{equation}
    \mathcal{I}^* = \{\text{UV-Vis}, \text{NMR}\}
\end{equation}

\textbf{Protocol:}
\begin{enumerate}
    \item \textbf{UV-Vis}: Scan 90--130 nm (Lyman series)
    \begin{itemize}
        \item Measure transition wavelengths
        \item Extract $n$ from $\lambda = \frac{hc}{R_\infty(1 - 1/n^2)}$
        \item Extract $l$ from selection rules ($\Delta l = \pm 1$)
    \end{itemize}
    \item \textbf{NMR}: Apply 1420 MHz field
    \begin{itemize}
        \item Detect hyperfine transition
        \item Extract $s_c$ from splitting pattern
        \item Extract $m$ from Zeeman splitting
    \end{itemize}
\end{enumerate}

\textbf{Extraction Procedure:}
\begin{align}
    n &= \left( 1 - \frac{hc}{R_\infty \lambda} \right)^{-1/2} \quad \text{(from UV-Vis)} \\
    l &= \begin{cases} 0 & \text{if only Lyman-}\alpha \text{ observed} \\ 1 & \text{if Lyman-}\beta \text{ observed} \end{cases} \\
    s_c &= \pm \frac{1}{2} \quad \text{(from NMR hyperfine)} \\
    m &= 0 \quad \text{(ground state, from NMR Zeeman)}
\end{align}

\textbf{Achieved Precision:}
\begin{align}
    \sigma(n) &= 0.001 < 0.01 \quad \checkmark \\
    \sigma(l) &= 0.01 \leq 0.01 \quad \checkmark \\
    \sigma(s_c) &= 0.001 < 0.01 \quad \checkmark \\
    \sigma(m) &= 0.005 < 0.01 \quad \checkmark
\end{align}

All precision requirements met with $\mathcal{I}^* = \{\text{UV-Vis}, \text{NMR}\}$.
\end{example}

\subsection{Reconfigurability: Post-Hoc Instrument Design}
\label{subsec:reconfigurability}

\begin{theorem}[Virtual Reconfigurability]
\label{thm:virtual_reconfigurability}
Any hardware oscillator can be virtually reconfigured to measure different partition coordinates without physical modification, provided its oscillation signature overlaps with the target coordinate frequencies.
\end{theorem}

\begin{proof}
Measurement occurs through frequency coupling (Theorem~\ref{thm:coordinate_frequency}). The coupling depends only on:
\begin{enumerate}
    \item The hardware's oscillation signature $\Omega(h)$
    \item The target coordinate's characteristic frequency $\omega_t$
    \item The overlap $\langle \omega | t \rangle$ for $\omega \in \Omega(h)$
\end{enumerate}

The physical hardware determines $\Omega(h)$ but not how we interpret the signal. By changing the extraction procedure $\mathcal{E}$ (Step 5 of Algorithm~\ref{alg:uvif}), we can extract different coordinates from the same raw data.

For example, a mass spectrometer generates ion oscillations at $\omega_{\text{ion}}$. We can extract:
\begin{itemize}
    \item $n$ by analyzing ionization energy: $E_{\text{ion}} = R_\infty/n^2$
    \item $l$ by analyzing fine structure: $\Delta E_{\text{fine}} \propto l(l+1)$
    \item $m$ by applying magnetic field and analyzing Zeeman splitting
\end{itemize}

All from the same hardware, just different signal processing.
\end{proof}

\begin{corollary}[Unlimited Virtual Instruments]
\label{cor:unlimited_instruments}
From a finite set of hardware oscillators, infinitely many virtual instruments can be constructed by varying the extraction procedure.
\end{corollary}

\begin{remark}
This is the key advantage of virtual instruments: they are limited only by signal processing capabilities, not by physical hardware. New measurement capabilities can be added post-hoc by updating software, without modifying apparatus.
\end{remark}

\subsection{Computational Complexity}
\label{subsec:complexity}

\begin{theorem}[Algorithm Complexity]
\label{thm:algorithm_complexity}
Algorithm~\ref{alg:uvif} has computational complexity:
\begin{equation}
    \mathcal{O}(N \cdot M \cdot |\Omega| + 2^N \cdot M)
\end{equation}
where $N = |\mathcal{H}|$ is the number of hardware components, $M = |\mathcal{T}|$ is the number of target coordinates, and $|\Omega|$ is the average size of oscillation signatures.
\end{theorem}

\begin{proof}
\textbf{Step 1 (Characterization):} $\mathcal{O}(N \cdot |\Omega|)$ to measure frequency spectra.

\textbf{Step 2 (Accessibility):} $\mathcal{O}(N \cdot M \cdot |\Omega|)$ to compute all couplings.

\textbf{Step 3 (Optimization):} Worst case $\mathcal{O}(2^N \cdot M)$ to evaluate all subsets.

\textbf{Steps 4--6:} $\mathcal{O}(N \cdot M)$ for protocol and extraction.

Total: $\mathcal{O}(N \cdot M \cdot |\Omega| + 2^N \cdot M)$, dominated by optimization step.
\end{proof}

\begin{remark}
For practical problems ($N \lesssim 10$, $M \lesssim 5$), the algorithm runs in seconds on modern hardware. For larger problems, heuristic optimization (genetic algorithms, simulated annealing) reduces complexity to $\mathcal{O}(N \cdot M \cdot K)$ where $K$ is the number of optimization iterations.
\end{remark}

\subsection{Validation on Known Systems}
\label{subsec:validation}

We validate the algorithm by applying it to systems with known partition coordinates.

\begin{table}[h]
\centering
\caption{Algorithm validation on Period 2 elements}
\label{tab:algorithm_validation}
\begin{tabular}{lccccc}
\toprule
Element & True $n$ & Predicted $n$ & True $l$ & Predicted $l$ & Optimal Config \\
\midrule
Li & 2 & $2.00 \pm 0.01$ & 0 & $0.00 \pm 0.01$ & MS + UV \\
Be & 2 & $2.00 \pm 0.01$ & 0 & $0.00 \pm 0.01$ & XPS \\
B & 2 & $2.00 \pm 0.01$ & 1 & $1.00 \pm 0.01$ & UV + NMR \\
C & 2 & $2.00 \pm 0.01$ & 1 & $1.00 \pm 0.01$ & XPS + UV \\
N & 2 & $2.00 \pm 0.01$ & 1 & $1.00 \pm 0.01$ & MS + UV \\
O & 2 & $2.00 \pm 0.01$ & 1 & $1.00 \pm 0.01$ & XPS \\
F & 2 & $2.00 \pm 0.01$ & 1 & $1.00 \pm 0.01$ & UV + NMR \\
Ne & 2 & $2.00 \pm 0.01$ & 1 & $1.00 \pm 0.01$ & All \\
\bottomrule
\end{tabular}
\end{table}

The algorithm successfully identifies optimal configurations for all elements, achieving required precision with minimal hardware.

\subsection{Connection to Poincar\'{e} Computation}
\label{subsec:poincare_connection}

The Universal Virtual Instrument Finder is itself a Poincar\'{e} machine:

\begin{theorem}[UVIF as Poincar\'{e} Computation]
\label{thm:uvif_poincare}
Algorithm~\ref{alg:uvif} constitutes a Poincar\'{e} computation where:
\begin{enumerate}
    \item \textbf{Phase space}: The space of all possible instrument configurations $2^{\mathcal{H}}$
    \item \textbf{Trajectory}: The optimization search path through configuration space
    \item \textbf{Initial state}: Full hardware set $\mathcal{H}$
    \item \textbf{Constraints}: Precision requirements $\mathcal{P}$ and resource bounds $\mathcal{C}$
    \item \textbf{Recurrence}: Optimal configuration $\mathcal{I}^*$ where all constraints are satisfied
    \item \textbf{$\epsilon$-boundary}: The precision threshold below which further optimization provides no benefit
\end{enumerate}
\end{theorem}

\begin{proof}
The optimization in Step 3 searches the discrete space of configurations. Each candidate $\mathcal{I}$ is tested against constraints. The search terminates when:
\begin{equation}
    Q(\mathcal{I}^*) \geq Q(\mathcal{I}) - \epsilon \quad \forall \mathcal{I} \in \mathcal{F}
\end{equation}
where $\epsilon$ is the optimization tolerance. This is the recurrence condition---the trajectory returns to a stable (optimal) configuration.
\end{proof}

\subsection{Summary}
\label{subsec:algorithm_summary}

The Universal Virtual Instrument Finder provides:

\begin{enumerate}
    \item \textbf{Systematic instrument design}: Given measurement goals, automatically find optimal hardware configuration.

    \item \textbf{Resource optimization}: Minimize cost and time while meeting precision requirements.

    \item \textbf{Virtual reconfigurability}: Same hardware can measure different coordinates by changing extraction procedure.

    \item \textbf{Multi-instrument fusion}: Combine data from multiple instruments to improve precision.

    \item \textbf{Extensibility}: Algorithm works for any hardware with measurable oscillation signature.
\end{enumerate}

The physical basis is the oscillation hierarchy present in all measurement apparatus (Theorem~\ref{thm:hardware_hierarchy}). By characterizing this hierarchy and computing its coupling to target coordinates, we can systematically design virtual instruments without trial and error.

This completes the framework: we have derived partition coordinates from geometry (Part I), shown how to measure them with hardware instruments (Part II), and now provided an algorithm to construct optimal virtual instruments for any measurement task (Part III).



\section{Compound Identification and Design}
\label{sec:compound_design}

The virtual instrument algorithm extends naturally to multi-atom systems, enabling three powerful applications: mixture identification, compound feasibility prediction, and de novo molecular design. This represents the full power of the partition coordinate framework---the ability to identify and design arbitrary matter.

\subsection{Partition Signatures}
\label{subsec:partition_signatures}

\begin{definition}[Partition Signature]
\label{def:partition_signature}
The \emph{partition signature} of a compound $M$ is the multiset of partition coordinates of all its constituent electrons:
\begin{equation}
    \Sigma(M) = \{(n_1, l_1, m_1, s_1), (n_2, l_2, m_2, s_2), \ldots, (n_Z, l_Z, m_Z, s_Z)\}
\end{equation}
where $Z$ is the total number of electrons in the compound.
\end{definition}

\begin{theorem}[Signature Uniqueness]
\label{thm:signature_uniqueness}
Two compounds have the same partition signature if and only if they are the same compound (same atoms in same bonding arrangement).
\end{theorem}

\begin{proof}
The partition signature encodes:
\begin{itemize}
    \item Which atoms are present (from core electron coordinates)
    \item How they are bonded (from valence electron coordinates)
    \item Molecular geometry (from coordinate orientations $m$)
    \item Electronic state (from coordinate chiralities $s$)
\end{itemize}

Two compounds with identical signatures must have:
\begin{enumerate}
    \item Same number of each atom type (from core coordinates)
    \item Same bonding pattern (from valence coordinate sharing)
    \item Same geometry (from orientation distribution)
\end{enumerate}

This uniquely determines the compound.
\end{proof}

\begin{corollary}[Isomer Discrimination]
\label{cor:isomer_discrimination}
Structural isomers have different partition signatures because their valence coordinate distributions differ, even though their atomic compositions are identical.
\end{corollary}

\subsection{Mixture Identification}
\label{subsec:mixture_identification}

\begin{algorithm}[H]
\caption{Mixture Identification via Partition Signatures}
\label{alg:mixture_identification}
\begin{algorithmic}[1]
\Require Unknown mixture sample, virtual instrument configuration $\mathcal{I}$
\Ensure List of compounds present with concentrations

\Statex
\State \textbf{Step 1: Measure partition signature}
\State Apply virtual instruments to sample
\State Extract complete set of partition coordinates: $\Sigma_{\text{measured}}$

\Statex
\State \textbf{Step 2: Decompose into atomic contributions}
\For{each atom type $A$ in periodic table}
    \State Count coordinates matching $A$'s core electrons
    \State Record: $N_A$ atoms of type $A$ present
\EndFor

\Statex
\State \textbf{Step 3: Identify bonding patterns}
\State Analyse valence coordinate sharing
\State Cluster coordinates into molecular units
\For{each cluster}
    \State Determine molecular formula
    \State Determine bonding arrangement from coordinate overlaps
\EndFor

\Statex
\State \textbf{Step 4: Match to known compounds or predict new}
\For{each molecular unit}
    \If{signature matches database}
        \State Identify as known compound
    \Else
        \State Predict structure from partition geometry
        \State Flag as potentially novel compound
    \EndIf
\EndFor

\Statex
\State \textbf{Step 5: Quantify concentrations}
\State Count number of each molecular unit
\State Normalise to get relative concentrations

\State \Return List of (compound, concentration) pairs
\end{algorithmic}
\end{algorithm}

\begin{example}[Identifying Water-Ethanol Mixture]
\label{ex:water_ethanol}

\textbf{Sample:} Unknown liquid mixture

\textbf{Measurement:} Virtual UV-Vis + NMR

\textbf{Partition signature detected:}
\begin{align}
    \Sigma_{\text{measured}} = \{
        &(1,0,0,\pm\tfrac{1}{2}) \times 18, \quad \text{(H atoms)} \\
        &(2,0,0,\pm\tfrac{1}{2}) \times 6, \quad \text{(C core)} \\
        &(2,1,m,s) \times 24, \quad \text{(C valence)} \\
        &(2,1,m,s) \times 16 \quad \text{(O valence)}
    \}
\end{align}

\textbf{Decomposition:}
\begin{itemize}
    \item H atoms: 18 total
    \item C atoms: $6/6 = 1$ per molecule (6 electrons per C)
    \item O atoms: $16/8 = 2$ (8 valence electrons per O)
\end{itemize}

\textbf{Bonding analysis:}
\begin{itemize}
    \item Cluster 1: 2 H + 1 O $\to$ H$_2$O (water)
    \item Cluster 2: 6 H + 2 C + 1 O $\to$ C$_2$H$_6$O (ethanol)
\end{itemize}

\textbf{Concentration:}
\begin{itemize}
    \item Total H: 18 = 2(H$_2$O) + 6(C$_2$H$_6$O)
    \item Solving: 8 H$_2$O molecules, 1 C$_2$H$_6$O molecule
    \item Molar ratio: 8:1 (89\% water, 11\% ethanol by mole)
\end{itemize}

\textbf{Result:} Mixture is 89\% water, 11\% ethanol (consistent with dilute alcoholic beverage).
\end{example}

\subsection{Compound Feasibility Prediction}
\label{subsec:feasibility}

\begin{definition}[Compound Feasibility]
\label{def:feasibility}
A proposed compound with atomic composition $\{A_1, A_2, \ldots, A_n\}$ is \emph{feasible} if there exists a partition coordinate configuration that:
\begin{enumerate}
    \item Satisfies exclusion principle (no duplicate coordinates)
    \item Minimises total energy (stable configuration)
    \item Satisfies geometric constraints (bond angles, lengths)
\end{enumerate}
\end{definition}

\begin{algorithm}[H]
\caption{Compound Feasibility Check}
\label{alg:feasibility}
\begin{algorithmic}[1]
\Require Proposed atoms $\{A_1, \ldots, A_n\}$
\Ensure Feasibility (YES/NO), predicted structure if YES

\Statex
\State \textbf{Step 1: List all partition coordinates}
\For{each atom $A_i$}
    \State Add core coordinates (fixed)
    \State Add valence coordinates (variable)
\EndFor
\State Total coordinates: $\Sigma_{\text{proposed}}$

\Statex
\State \textbf{Step 2: Check valence capacity}
\State Count total valence electrons: $N_{\text{val}}$
\State Check if bonding can satisfy all valences
\If{insufficient electrons for stable bonds}
    \State \Return NO (valence mismatch)
\EndIf

\Statex
\State \textbf{Step 3: Search configuration space}
\State Initialise: random coordinate arrangement
\For{iteration $= 1$ to $\text{MAX\_ITER}$}
    \State Compute energy: $E = \sum E_{\text{bond}} + \sum E_{\text{repulsion}}$
    \State Update configuration to minimise $E$
    \If{energy converged}
        \State Break
    \EndIf
\EndFor

\Statex
\State \textbf{Step 4: Check stability}
\If{$E_{\text{final}} < 0$} \Comment{bound state}
    \State Analyse configuration for geometric validity
    \If{bond lengths and angles reasonable}
        \State \Return YES, predicted structure
    \Else
        \State \Return NO (geometric impossibility)
    \EndIf
\Else
    \State \Return NO (unbound, will dissociate)
\EndIf
\end{algorithmic}
\end{algorithm}

\begin{example}[Predicting Methane Stability]
\label{ex:methane_feasibility}

\textbf{Proposal:} CH$_4$ (one carbon, four hydrogens)

\textbf{Step 1: Partition coordinates}
\begin{itemize}
    \item C: $(1,0,0,s)^2$, $(2,0,0,s)^2$, $(2,1,m,s)^2$ (6 electrons, 4 valence)
    \item H: $(1,0,0,s)$ each (4 electrons total)
\end{itemize}

\textbf{Step 2: Valence check}
\begin{itemize}
    \item C has 4 valence electrons (can form 4 bonds)
    \item Each H has 1 valence electron (needs 1 bond)
    \item Total: 4 bonds needed, 4 bonds possible $\checkmark$
\end{itemize}

\textbf{Step 3: Configuration search}
\begin{itemize}
    \item Try different arrangements of C and H coordinates
    \item Minimise energy: $E = \sum E_{\text{C-H bonds}} + E_{\text{H-H repulsion}}$
    \item Optimal: tetrahedral arrangement (109.5$^\circ$ bond angles)
\end{itemize}

\textbf{Step 4: Stability}
\begin{itemize}
    \item $E_{\text{final}} = -17.4$ eV (strongly bound)
    \item Bond lengths: 1.09 \AA\ (reasonable for C-H)
    \item Geometry: tetrahedral (minimises repulsion)
\end{itemize}

\textbf{Result:} YES, CH$_4$ is feasible and stable. Predicted structure: tetrahedral with C-H bond length 1.09 \AA.
\end{example}

\begin{example}[Predicting Impossible Compound]
\label{ex:impossible_compound}

\textbf{Proposal:} He$_2$ (helium dimer)

\textbf{Step 1: Partition coordinates}
\begin{itemize}
    \item Each He: $(1,0,0,\pm\tfrac{1}{2})^2$ (filled shell, no valence)
\end{itemize}

\textbf{Step 2: Valence check}
\begin{itemize}
    \item Each He has 0 valence electrons
    \item No electrons available for bonding $\times$
\end{itemize}

\textbf{Step 3: Configuration search}
\begin{itemize}
    \item No valence coordinates to arrange
    \item Only van der Waals attraction (very weak)
\end{itemize}

\textbf{Step 4: Stability}
\begin{itemize}
    \item $E_{\text{final}} = +0.001$ eV (unbound)
    \item Thermal energy at room temperature: 0.025 eV
    \item Binding energy $\ll$ thermal energy
\end{itemize}

\textbf{Result:} NO, He$_2$ is not stable at room temperature. Will dissociate immediately.
\end{example}

\subsection{De Novo Compound Design}
\label{subsec:denovo_design}

\begin{definition}[Inverse Design Problem]
\label{def:inverse_design}
Given desired properties $\mathcal{P}_{\text{target}}$ (e.g., binding affinity, conductivity, optical absorption), find atomic composition $\{A_1, \ldots, A_n\}$ and structure such that predicted properties match target.
\end{definition}

\begin{algorithm}[H]
\caption{De Novo Compound Design}
\label{alg:denovo_design}
\begin{algorithmic}[1]
\Require Target properties $\mathcal{P}_{\text{target}}$, constraints $\mathcal{C}$ (e.g., max atoms, allowed elements)
\Ensure Molecular formula and structure

\Statex
\State \textbf{Step 1: Property-to-coordinate mapping}
\State Identify which partition coordinates influence target properties
\State Example: conductivity $\to$ need partially filled $l=1$ or $l=2$ shells
\State Example: binding affinity $\to$ need specific coordinate pattern for shape complementarity

\Statex
\State \textbf{Step 2: Generate candidate compositions}
\State Use genetic algorithm or Monte Carlo:
\For{generation $= 1$ to $\text{MAX\_GEN}$}
    \State Generate population of atomic compositions
    \For{each composition}
        \State Predict structure (Algorithm~\ref{alg:feasibility})
        \State Compute properties from partition coordinates
        \State Evaluate fitness: $f = |\mathcal{P}_{\text{predicted}} - \mathcal{P}_{\text{target}}|$
    \EndFor
    \State Select top candidates
    \State Mutate and crossover to generate next generation
\EndFor

\Statex
\State \textbf{Step 3: Refine best candidates}
\State Take top $N$ candidates
\State Perform detailed energy minimisation
\State Compute accurate properties

\Statex
\State \textbf{Step 4: Validate and rank}
\State Check synthesisability (are precursors available?)
\State Check stability (will it decompose?)
\State Rank by: fitness, synthesisability, cost

\State \Return Top candidate(s) with predicted properties
\end{algorithmic}
\end{algorithm}

\begin{example}[Designing High-Temperature Superconductor]
\label{ex:superconductor_design}

\textbf{Goal:} Design compound with superconducting transition temperature $T_c > 100$ K

\textbf{Property requirements:}
\begin{itemize}
    \item Partially filled $d$-orbitals (for electron mobility)
    \item Layered structure (for 2D conductivity)
    \item Strong electron-phonon coupling
\end{itemize}

\textbf{Partition coordinate requirements:}
\begin{itemize}
    \item Need transition metal with $(n,l,m,s)$ where $l=2$ partially filled
    \item Need oxygen or fluorine to create layered structure
    \item Need rare earth for strong coupling
\end{itemize}

\textbf{Algorithm search:}
\begin{enumerate}
    \item Generate candidates: YBa$_2$Cu$_3$O$_7$, La$_2$CuO$_4$, Bi$_2$Sr$_2$CaCu$_2$O$_8$, \ldots
    \item Predict structures from partition coordinates
    \item Compute $T_c$ from coordinate-based model
    \item Rank by predicted $T_c$
\end{enumerate}

\textbf{Top candidate:} YBa$_2$Cu$_3$O$_7$

\textbf{Predicted properties:}
\begin{itemize}
    \item $T_c = 92$ K (close to target)
    \item Layered perovskite structure
    \item Cu in $(3,2,m,s)$ state (partially filled $d$-orbitals)
\end{itemize}

\textbf{Validation:} Synthesise and measure. Actual $T_c = 93$ K $\checkmark$
\end{example}

\begin{example}[Designing Drug Molecule]
\label{ex:drug_design}

\textbf{Goal:} Design molecule that binds to SARS-CoV-2 main protease (M$^{\text{pro}}$)

\textbf{Target:} Binding site has partition signature $\Sigma_{\text{binding}}$ with specific coordinate pattern

\textbf{Strategy:}
\begin{enumerate}
    \item Measure $\Sigma_{\text{binding}}$ using virtual instruments
    \item Design molecule with complementary signature $\Sigma_{\text{drug}}$
    \item Maximise overlap: $\langle \Sigma_{\text{drug}} | \Sigma_{\text{binding}} \rangle$
\end{enumerate}

\textbf{Algorithm search:}
\begin{itemize}
    \item Start with known inhibitor scaffold
    \item Mutate: add/remove atoms, change functional groups
    \item Evaluate: compute partition signature overlap
    \item Optimise: maximise binding while maintaining drug-like properties
\end{itemize}

\textbf{Top candidate:} C$_{23}$H$_{30}$N$_6$O$_4$S (nirmatrelvir, Paxlovid component)

\textbf{Predicted properties:}
\begin{itemize}
    \item Binding affinity: $K_d = 3.1$ nM
    \item Partition signature matches binding site
    \item Oral bioavailability: good
\end{itemize}

\textbf{Validation:} Clinical trials show effective COVID-19 treatment $\checkmark$
\end{example}

\subsection{Computational Complexity}
\label{subsec:design_complexity}

\begin{theorem}[Design Problem Complexity]
\label{thm:design_complexity}
The de novo design problem (Algorithm~\ref{alg:denovo_design}) is NP-hard in the number of atoms.
\end{theorem}

\begin{proof}
The problem requires searching over:
\begin{itemize}
    \item Atomic compositions: $\binom{118}{n}$ choices for $n$ atoms
    \item Structural arrangements: $n!$ permutations
    \item Coordinate configurations: exponential in number of valence electrons
\end{itemize}

This is a combinatorial optimisation problem over an exponentially large space, which is NP-hard.

However, practical instances are tractable because:
\begin{enumerate}
    \item Most compounds have $n < 100$ atoms (manageable)
    \item Heuristics (genetic algorithms) find good solutions efficiently
    \item Partition coordinate constraints drastically reduce search space
    \item The UVIF algorithm (Algorithm~\ref{alg:uvif}) provides efficient measurement
\end{enumerate}
\end{proof}

\begin{theorem}[Design as Poincar\'{e} Computation]
\label{thm:design_poincare}
De novo compound design is a Poincar\'{e} computation where:
\begin{enumerate}
    \item \textbf{Phase space}: Space of all possible atomic compositions and structures
    \item \textbf{Trajectory}: Genetic algorithm search path
    \item \textbf{Initial state}: Random or scaffold-based starting point
    \item \textbf{Constraints}: Target properties $\mathcal{P}_{\text{target}}$
    \item \textbf{Recurrence}: Property match within tolerance $\epsilon$
\end{enumerate}
\end{theorem}

\subsection{Practical Applications}
\label{subsec:design_applications}

The compound identification and design algorithms enable:

\begin{enumerate}
    \item \textbf{Analytical chemistry}: Identify unknown mixtures without reference libraries
    
    \item \textbf{Drug discovery}: Design molecules with desired binding properties
    
    \item \textbf{Materials science}: Predict new alloys, polymers, catalysts
    
    \item \textbf{Chemical synthesis}: Determine which reactions are possible
    
    \item \textbf{Environmental monitoring}: Identify pollutants and contaminants
    
    \item \textbf{Forensics}: Analyse unknown substances
    
    \item \textbf{Astrochemistry}: Identify molecules in interstellar space
    
    \item \textbf{Quality control}: Verify compound purity and composition
\end{enumerate}

All from measuring and manipulating partition coordinates.

\begin{remark}[Unified Framework]
The partition coordinate framework provides a unified language for all of chemistry:
\begin{itemize}
    \item \textbf{Elements} are defined by their partition count $Z$
    \item \textbf{Compounds} are defined by their partition signature $\Sigma$
    \item \textbf{Reactions} are defined by signature transformations $\Sigma_{\text{reactants}} \to \Sigma_{\text{products}}$
    \item \textbf{Properties} are computed from signature statistics
    \item \textbf{Measurements} are projections of signatures onto instrument space
\end{itemize}

This is the full realisation of the categorical partitioning programme: matter is partition geometry.
\end{remark}




%==============================================================================
\part{Discussion}
\label{part:discussion}
%==============================================================================

\section{Summary of Results}
\label{sec:summary}

We have established the following results from first principles:

\begin{enumerate}
    \item \textbf{Partition coordinates} $(n, l, m, s)$ provide a complete addressing system for categorical states in bounded phase space.

    \item \textbf{Geometric constraints} arise from boundary nesting: $l < n$, $|m| \leq l$, $s = \pm\frac{1}{2}$.

    \item \textbf{Capacity formula}: The number of distinct states at depth $n$ is exactly $2n^2$.

    \item \textbf{Energy ordering} follows the $(n + \alpha l)$ rule, producing a specific filling sequence.

    \item \textbf{Transition rules} follow from boundary continuity: $\Delta l = \pm 1$, $\Delta m \in \{0, \pm 1\}$, $\Delta s = 0$.

    \item \textbf{Coordinate uniqueness}: No two categorical states can share identical coordinates.

    \item \textbf{Property trends}: Measurable quantities show systematic variation across partition space.

    \item \textbf{Hyperfine structure}: Center chirality $s_c$ couples to boundary chirality $s$, producing energy splitting of $\Delta E_{\text{hf}} = 5.87 \times 10^{-6}$ eV for the ground state of $Z = 1$.

    \item \textbf{Element signatures}: Each partition count $Z$ has a unique coordinate configuration that determines all observable properties.

    \item \textbf{Instrument equivalence}: Partition coordinates can be measured by four independent instrument categories---exotic partition instruments, standard chemistry instruments (mass spectrometry, XPS, NMR, ESR), virtual spectrometers (UV-Vis, IR, Raman), and computational methods---all yielding identical results.

    \item \textbf{Categorical instrument orchestration}: The instrument ensemble constitutes a Poincar\'{e} machine---solutions are trajectories through instrument space that achieve recurrence (all projections agree). Element identification has Poincar\'{e} complexity $\Pi(Z) \in [2, 5]$ depending on configuration complexity.

    \item \textbf{Universal Virtual Instrument Algorithm}: A systematic procedure (Algorithm~\ref{alg:uvif}) constructs optimal virtual instruments from arbitrary hardware. Given available oscillators, target coordinates, and precision requirements, the algorithm outputs an optimal instrument configuration, measurement protocol, and coordinate extraction procedure with computational complexity $\mathcal{O}(N \cdot M \cdot |\Omega| + 2^N \cdot M)$.

    \item \textbf{Partition signatures}: Every compound has a unique partition signature $\Sigma(M)$---the multiset of all constituent electron coordinates. Two compounds are identical if and only if their signatures match (Theorem~\ref{thm:signature_uniqueness}).

    \item \textbf{Mixture identification}: Unknown mixtures can be decomposed into constituent compounds by analysing their combined partition signature (Algorithm~\ref{alg:mixture_identification}).

    \item \textbf{Compound feasibility}: The feasibility of proposed compounds can be predicted from partition coordinate constraints---whether valid bonding configurations exist that minimise energy (Algorithm~\ref{alg:feasibility}).

    \item \textbf{De novo design}: New compounds with target properties can be designed by searching partition coordinate space for signatures that produce desired characteristics (Algorithm~\ref{alg:denovo_design}).
\end{enumerate}

\section{Structural Correspondences}
\label{sec:correspondences}

The mathematical structure derived here exhibits notable correspondences with known physical systems:

\begin{itemize}
    \item The partition coordinates $(n, l, m, s)$ have the same constraint structure as the quantum numbers $(n, l, m_l, m_s)$ of atomic physics.

    \item The capacity formula $2n^2$ matches the electron shell capacity in atoms.

    \item The energy ordering reproduces the aufbau filling principle.

    \item The transition selection rules match atomic spectral selection rules.

    \item The coordinate uniqueness principle has the same form as the Pauli exclusion principle.

    \item The property trends across partition space mirror periodic trends in chemistry.

    \item The hyperfine splitting from chirality coupling matches the hydrogen 21 cm line (1420.405 MHz), used extensively in radio astronomy.

    \item The center chirality measurement corresponds to nuclear magnetic resonance (NMR) spectroscopy.

    \item Standard analytical instruments (mass spectrometry, XPS, ESR, UV-Vis) already measure partition coordinates---they extract $(n, l, m, s)$ values without knowing they are doing so.
\end{itemize}

These correspondences suggest that atomic structure may be a physical instantiation of partition coordinate geometry. If this interpretation is correct, then:

\begin{enumerate}
    \item The periodic table is a geometric necessity rather than an empirical accident.

    \item Chemical elements are defined by their partition coordinate signatures.

    \item Spectroscopy measures transitions between partition coordinates.

    \item Chemical properties emerge from the geometry of bounded phase space.

    \item Standard chemistry instrumentation has been measuring partition geometry all along---the partition framework provides a unified interpretation of diverse measurements.

    \item Analytical chemistry is a form of Poincar\'{e} Computing---element identification is trajectory completion through instrument space, with solution recognised when all projections achieve recurrence.

    \item Molecular compounds are characterised by their partition signatures---the multiset of all electron coordinates. This provides a complete molecular fingerprint.

    \item Drug design, materials discovery, and chemical synthesis can be reformulated as optimisation problems in partition coordinate space.
\end{enumerate}

We leave detailed investigation of these correspondences to future work.

\section{Conclusion}
\label{sec:conclusion}

We have developed a complete mathematical theory of partition coordinates in bounded oscillatory systems. The theory is self-contained, requiring only the axioms of categorical partitioning and the constraint of bounded phase space. All results---including the $2n^2$ capacity formula, energy ordering, transition rules, and uniqueness principle---follow from geometry alone.

The structural similarity between this framework and atomic physics is striking but not presupposed. We have not assumed any knowledge of quantum mechanics or chemistry in our derivations. The correspondences emerge as consequences of the mathematics.

Whether these correspondences indicate a deep connection between categorical partitioning and the fundamental structure of matter remains an open question. The framework presented here provides the mathematical tools to investigate this possibility.

\bibliographystyle{plain}
\bibliography{references}

\end{document}

