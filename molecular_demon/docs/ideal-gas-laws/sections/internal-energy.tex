\section{Internal Energy: Active Category Counting}
\label{sec:internal_energy}

\subsection{Classical Internal Energy and Equipartition}

Classical statistical mechanics assigns energy $k_B T/2$ to each quadratic degree of freedom through the equipartition theorem:
\begin{equation}
U_{\text{classical}} = \frac{f}{2} N k_B T
\end{equation}

where $f$ is the number of degrees of freedom per particle, and $N$ is the number of particles.

For a monatomic ideal gas with three translational degrees of freedom:
\begin{equation}
U = \frac{3}{2} N k_B T
\end{equation}

\textbf{Unresolved questions in classical theory:}
\begin{itemize}
\item \textbf{Why $k_B T/2$ per mode?} What physical principle determines this specific energy allocation?
\item \textbf{Why does equipartition fail at low temperatures?} Rotational and vibrational modes ``freeze out'' below characteristic temperatures.
\item \textbf{Why do some degrees of freedom remain inactive?} Not all modes participate equally in energy storage.
\item \textbf{What is the mechanism of mode activation?} How does temperature determine which modes are energetically accessible?
\end{itemize}

The triple equivalence framework answers these questions: energy is stored in \textit{active categories}, and categories activate discretely when their characteristic energy scale becomes comparable to $k_B T$.

\subsection{Categorical Internal Energy}

Internal energy counts the number of active categorical modes, each storing energy $k_B T$:

\begin{definition}
The \textit{categorical internal energy} is:
\begin{equation}
\boxed{U_{\text{cat}} = k_B T \cdot M_{\text{active}}}
\label{eq:categorical_energy}
\end{equation}
where $M_{\text{active}}$ is the number of categories with non-zero thermal occupation.
\end{definition}

\textbf{Physical interpretation:} Energy is not distributed continuously; it is stored in discrete categorical ``slots.'' Each active slot holds approximately $k_B T$ of thermal energy. The total energy is the product of the energy scale ($k_B T$) and the number of active slots ($M_{\text{active}}$).

\subsubsection{Why $k_B T$ per Category?}

The energy per category emerges from the fundamental thermodynamic relation:
\begin{equation}
dU = T \, dS - P \, dV + \mu \, dN
\end{equation}

At constant volume and particle number:
\begin{equation}
dU = T \, dS
\end{equation}

Using the categorical entropy $S = k_B M \ln n$:
\begin{equation}
dU = T \cdot k_B \ln n \, dM
\end{equation}

For the natural choice $\ln n = 1$ (one unit of information per category):
\begin{equation}
dU = k_B T \, dM
\end{equation}

Integrating:
\begin{equation}
U = k_B T \cdot M + U_0
\end{equation}

where $U_0$ is the zero-point energy. Each additional category contributes $k_B T$ to the thermal energy.

\textbf{Physical reason:} A category represents one distinguishable state. The thermal energy required to maintain distinguishability against entropic mixing is $k_B T$ per category—this is the characteristic energy scale of thermal fluctuations.

\subsubsection{Active vs. Total Categories}

Not all categories are thermally active. A category is active if its characteristic energy scale is comparable to or less than $k_B T$:

\begin{definition}
A category $i$ is \textit{active} if:
\begin{equation}
\hbar\omega_i \lesssim k_B T
\end{equation}
where $\omega_i$ is the characteristic frequency of that category.
\end{definition}

The number of active categories is:
\begin{equation}
M_{\text{active}} = \sum_i \Theta(k_B T - \hbar\omega_i)
\end{equation}

where $\Theta$ is the Heaviside step function (or a smoothed version accounting for thermal broadening).

\textbf{Physical interpretation:} Categories with $\hbar\omega_i > k_B T$ are ``frozen out''---they require more energy than thermal fluctuations can provide; thus, they remain in their ground state and do not contribute to thermal energy.

This explains the temperature-dependent activation of degrees of freedom:
\begin{itemize}
\item \textbf{Translational modes}: Always active (very low $\hbar\omega$)
\item \textbf{Rotational modes}: Activate at $T \sim \hbar^2/(2Ik_B)$ where $I$ is the moment of inertia
\item \textbf{Vibrational modes}: Activate at $T \sim \hbar\omega_{\text{vib}}/k_B$ where $\omega_{\text{vib}}$ is the vibrational frequency
\item \textbf{Electronic modes}: Activate only at very high $T$ (eV energy scales)
\end{itemize}

\subsubsection{Recovery of Classical Equipartition}

For a classical system where all modes are active ($k_B T \gg \hbar\omega_i$ for all $i$), each quadratic degree of freedom corresponds to one category. For $f$ degrees of freedom per particle and $N$ particles:
\begin{equation}
M_{\text{active}} = \frac{f \cdot N}{2}
\end{equation}

The factor of 2 arises because each physical degree of freedom (e.g., position $x$) has both kinetic ($p_x^2/2m$) and potential ($kx^2/2$) contributions; however, in categorical counting, each quadratic term contributes independently.

Thus:
\begin{equation}
U = k_B T \cdot \frac{f N}{2} = \frac{f N k_B T}{2}
\end{equation}

This is the classical equipartition result. The categorical framework provides the underlying mechanism: energy is stored in categories, and equipartition is the high-temperature limit in which all categories are active.

\begin{figure}[htbp]
\centering
\includegraphics[width=\textwidth]{figures/fig_internal_energy.png}
\caption{\textbf{Internal Energy: Triple Equivalence Perspectives on Thermodynamic Energy.} 
\textbf{(A) Categorical energy versus temperature:} Reduced internal energy $U/(Nk_BT)$ versus temperature (10$^{-1}$ to 10$^4$ K). Black dashed line: classical equipartition ($U = 3Nk_BT/2$, giving $U/(Nk_BT) = 1.5$). Green solid line: categorical prediction $U = M_{\text{active}}k_BT/2$. Step-like increases at $T \sim 100$ K (orange annotation: ``Rotation activates'') and $T \sim 1000$ K (red annotation: ``Vibration activates''). At low temperature ($T < 10$ K), $U/(Nk_BT) \approx 1.5$ (translational modes only). At high temperature ($T > 1000$ K), $U/(Nk_BT) \approx 3.5$ (translation + rotation + vibration).
\textbf{(B) Oscillatory energy (quantum):} Absolute internal energy $U$ (joules) versus temperature (0-10000 K), logarithmic vertical scale (10$^1$ to 10$^5$ J). Blue solid line: oscillatory prediction $U = \sum_i \hbar\omega_i(n_i + 1/2)$. Purple dashed line: zero-point energy $U_0 = N\hbar\omega/2 \approx 300$ J (constant). Black dotted line: classical limit $U = Nk_BT$ (linear). At low temperature, zero-point energy dominates. At high temperature, classical limit is approached.
\textbf{(C) Partition energy (aperture contributions):} Reduced energy $\sum_a \Phi_a N_a/(Nk_BT)$ versus temperature (10$^1$ to 10$^4$ K). Stacked area chart: green (translational contribution, constant $\approx 1.5$), orange (rotational contribution, activates at $\sim$100 K, adds $\approx 1.0$), red (vibrational contribution, activates at $\sim$1000 K, adds $\approx 1.0$). Total energy at high temperature: $\approx 3.5 \times Nk_BT$, matching panel A.
\textbf{(D) Heat capacity (mode activation):} Heat capacity $C_V/(Nk_B)$ versus temperature (10$^0$ to 10$^4$ K). Gray dashed line: classical value (3/2). Green solid line: categorical prediction. Purple dotted line: Einstein model. Gray annotation at $T \sim 1$ K: ``Quantum freeze-out'' where $C_V/(Nk_B) \approx 1.5$. Green annotation at $T \sim 100$ K: ``Classical plateau'' where $C_V/(Nk_B) \approx 2.5$ (translation + rotation). Green annotation at $T \sim 1000$ K: ``Vibrational activation'' where $C_V/(Nk_B) \to 3.5$. Discrete steps in heat capacity correspond to sequential activation of categorical modes, in contrast to continuous classical prediction.}
\label{fig:internal_energy}
\end{figure}

\subsection{Oscillatory Internal Energy}

In the oscillatory perspective, energy is the sum over all oscillator modes:

\begin{definition}
The \textit{oscillatory internal energy} is:
\begin{equation}
\boxed{U_{\text{osc}} = \sum_{i=1}^{N_{\text{modes}}} \hbar\omega_i \left(n_i + \frac{1}{2}\right)}
\label{eq:oscillatory_energy}
\end{equation}
where $n_i$ is the occupation number of mode $i$ and $\omega_i$ is its frequency.
\end{definition}

\textbf{Physical interpretation:} Each mode is a quantum harmonic oscillator. The energy includes both the thermal excitation energy $\hbar\omega_i n_i$ and the zero-point energy $\hbar\omega_i/2$. The zero-point energy persists even at $T = 0$, while the excitation energy vanishes.

\subsubsection{Thermal Occupation}

For a system in thermal equilibrium at temperature $T$, the average occupation number follows the Bose-Einstein distribution:
\begin{equation}
\langle n_i \rangle = \frac{1}{e^{\hbar\omega_i/k_B T} - 1}
\end{equation}

\textbf{High-temperature limit} ($k_B T \gg \hbar\omega_i$): Expanding the exponential:
\begin{equation}
e^{\hbar\omega_i/k_B T} \approx 1 + \frac{\hbar\omega_i}{k_B T} + \cdots
\end{equation}

Thus:
\begin{equation}
\langle n_i \rangle \approx \frac{k_B T}{\hbar\omega_i}
\end{equation}

The thermal energy per mode becomes:
\begin{equation}
\langle E_i \rangle = \hbar\omega_i \left(\frac{k_B T}{\hbar\omega_i} + \frac{1}{2}\right) = k_B T + \frac{\hbar\omega_i}{2}
\end{equation}

Ignoring the subdominant zero-point term:
\begin{equation}
\langle E_i \rangle \approx k_B T
\end{equation}

Summing over $M$ active modes:
\begin{equation}
U \approx M k_B T
\end{equation}

For a monatomic gas with 3 translational degrees of freedom per particle, counting only kinetic energy contributions:
\begin{equation}
U = \frac{3}{2} N k_B T
\end{equation}

\subsubsection{Low-Temperature Behavior}

\textbf{Low-temperature limit} ($k_B T \ll \hbar\omega_i$):
\begin{equation}
\langle n_i \rangle \approx e^{-\hbar\omega_i/k_B T} \to 0
\end{equation}

The mode is exponentially suppressed (frozen out). Only the zero-point energy remains:
\begin{equation}
\langle E_i \rangle \to \frac{\hbar\omega_i}{2}
\end{equation}

This correctly captures quantum freeze-out: modes with $\hbar\omega_i > k_B T$ do not contribute to thermal energy, only to ground-state energy.

\subsubsection{Crossover Temperature}

The crossover between active and frozen regimes occurs at:
\begin{equation}
T_{\text{crossover}} \sim \frac{\hbar\omega_i}{k_B}
\end{equation}

For $T \ll T_{\text{crossover}}$: mode frozen, $\langle n_i \rangle \approx 0$

For $T \gg T_{\text{crossover}}$: mode active, $\langle n_i \rangle \approx k_B T/\hbar\omega_i$

\subsection{Partition Internal Energy}

In the partition perspective, energy is stored in the categorical potentials of occupied apertures:

\begin{definition}
The \textit{partition internal energy} is:
\begin{equation}
\boxed{U_{\text{part}} = \sum_{a=1}^{M} \Phi_a \cdot N_a}
\label{eq:partition_energy}
\end{equation}
where $\Phi_a = k_B T \ln n_a$ is the potential of aperture $a$ and $N_a$ is its occupancy.
\end{definition}

\textbf{Physical interpretation:} Each aperture stores energy proportional to its categorical depth $\ln n_a$. Deep apertures (high $n_a$, many accessible states) store more energy than shallow apertures (low $n_a$, few accessible states).

\subsubsection{Connection to Categorical and Oscillatory Formulations}

For uniform apertures with $n_a = n$ for all $a$ and total occupancy $\sum_a N_a = N$:
\begin{equation}
U_{\text{part}} = k_B T \ln n \cdot N
\end{equation}

For the natural choice $\ln n = 1$:
\begin{equation}
U_{\text{part}} = N k_B T
\end{equation}

Comparing with oscillatory energy at high $T$ (ignoring zero-point):
\begin{equation}
U_{\text{osc}} \approx \sum_{i=1}^{N} k_B T = N k_B T
\end{equation}

And categorical energy:
\begin{equation}
U_{\text{cat}} = k_B T \cdot M_{\text{active}} = k_B T \cdot N = N k_B T
\end{equation}

All three formulations agree in the classical high-temperature limit.

\subsection{Equivalence of Three Definitions}

\begin{theorem}[Internal Energy Equivalence]
\label{thm:energy_equivalence}
For thermal systems in the classical limit, the three energy definitions are equivalent:
\begin{equation}
U_{\text{cat}} = U_{\text{osc}} = U_{\text{part}} = M_{\text{active}} k_B T
\end{equation}
\end{theorem}

\begin{proof}
At high temperature ($k_B T \gg \hbar\omega_i$ for all active modes), each active mode contributes $k_B T$:

\textit{Categorical:} $M_{\text{active}}$ modes $\times$ $k_B T$ per mode = $M_{\text{active}} k_B T$

\textit{Oscillatory:} 
\begin{equation}
\sum_i \hbar\omega_i n_i \approx \sum_i \hbar\omega_i \cdot \frac{k_B T}{\hbar\omega_i} = \sum_i k_B T = M_{\text{active}} k_B T
\end{equation}

\textit{Partition:} 
\begin{equation}
\sum_a \Phi_a N_a = k_B T \ln n \cdot M_{\text{active}} \approx M_{\text{active}} k_B T
\end{equation}
(for $\ln n = 1$)

All three reduce to $M_{\text{active}} k_B T$ in the classical limit.
\end{proof}

\begin{figure*}[htbp]
\centering
\includegraphics[width=\textwidth]{figures/panel_thermodynamics.png}
\caption{\textbf{Real Thermodynamics from Hardware Timing.} 
\textbf{(A)} Temperature evolution over 3 seconds, defined as timing jitter variance. Initial spike to $T \approx 0.09$ (arbitrary units) during system initialization, followed by equilibration to steady-state $T \approx 0.08$. Temperature is a categorical observable extracted from hardware oscillation statistics, not a simulation parameter. 
\textbf{(B)} Pressure-count relationship showing exponential decay from $P \approx 13000$ (rate units) at low molecule count to $P \approx 0$ at high count ($N \sim 1000$). Color gradient (purple $\to$ yellow) indicates temporal evolution. Pressure is measured as the rate of partition operations per unit time. 
\textbf{(C)} Maxwell-Boltzmann distribution fit to measured S-entropy coordinate $S_e$. Blue histogram shows measured probability density; red dashed curve shows theoretical MB distribution. Excellent agreement validates that hardware timing statistics obey thermodynamic distributions without simulation. 
\textbf{(D)} Entropy growth during system evolution, increasing from $S \approx 0$ to $S \approx 2.3$ as molecule count grows from 0 to 1000. Entropy is computed directly from configuration space sampling via $S = k_B \ln \Omega$. 
\textbf{(E)} Pressure-internal energy (P-U) diagram showing thermodynamic trajectory from start (green circle, high pressure $P \approx 13000$, low energy $U \approx 0$) to end (red square, low pressure $P \approx 0$, moderate energy $U \approx 120$). Trajectory follows expected adiabatic expansion path. 
\textbf{(F)} Heat capacity $C_v = dU/dT$ versus temperature. Scatter points show measured values with large fluctuations due to finite sampling. Red dashed line indicates mean $\langle C_v \rangle \approx 0$ (units: dU/dT). Fluctuations arise from discrete partition events in the categorical measurement process.}
\label{fig:thermodynamics}
\end{figure*}

\subsection{Heat Capacity}

The heat capacity at constant volume is:
\begin{equation}
C_V = \left(\frac{\partial U}{\partial T}\right)_V
\end{equation}

\subsubsection{Categorical Heat Capacity}

From $U = k_B T \cdot M_{\text{active}}$:
\begin{equation}
C_V = k_B M_{\text{active}} + k_B T \frac{\partial M_{\text{active}}}{\partial T}
\end{equation}

\textbf{High-temperature regime:} All modes are active, $\partial M_{\text{active}}/\partial T = 0$:
\begin{equation}
C_V = k_B M_{\text{active}} = \frac{f N k_B}{2}
\end{equation}

This is the classical Dulong-Petit law for solids ($C_V = 3Nk_B$) or the ideal gas result.

\textbf{Low-temperature regime:} As modes freeze out, $\partial M_{\text{active}}/\partial T < 0$ and $C_V$ decrease. The heat capacity exhibits steps as each mode's activation threshold $\hbar\omega_i \sim k_B T$ is crossed.

\subsubsection{Oscillatory Heat Capacity (Einstein Model)}

For a collection of oscillators with a single frequency $\omega$:
\begin{equation}
U = N \hbar\omega \left(\frac{1}{e^{\hbar\omega/k_B T} - 1} + \frac{1}{2}\right)
\end{equation}

Taking the derivative:
\begin{equation}
C_V = N k_B \left(\frac{\hbar\omega}{k_B T}\right)^2 \frac{e^{\hbar\omega/k_B T}}{(e^{\hbar\omega/k_B T} - 1)^2}
\end{equation}

This is the Einstein heat capacity formula:
\begin{itemize}
\item \textbf{High $T$}: $C_V \to N k_B$ (classical limit, full activation)
\item \textbf{Low $T$}: $C_V \propto e^{-\hbar\omega/k_B T} \to 0$ (exponential freeze-out)
\end{itemize}

\subsubsection{Discrete Steps in Heat Capacity}

The categorical perspective predicts discrete steps in $C_V(T)$ as modes activate:
\begin{equation}
C_V(T) = k_B \sum_i \Theta(k_B T - \hbar\omega_i) + k_B T \sum_i \delta(k_B T - \hbar\omega_i)
\end{equation}

where $\delta$ is the Dirac delta function (or a smoothed version representing the activation region).

This stepwise behaviour is observable in:
\begin{itemize}
\item \textbf{Molecular gases}: Rotational modes activate around $T \sim 10$ K, vibrational modes around $T \sim 1000$ K
\item \textbf{Quantum solids}: Phonon modes activate progressively according to the Debye spectrum
\item \textbf{Magnetic systems}: Spin degrees of freedom activate at characteristic temperatures
\end{itemize}

\subsection{Summary}

Internal energy admits three equivalent definitions:
\begin{align}
U_{\text{cat}} &= k_B T \cdot M_{\text{active}} \quad \text{(active category count)} \\
U_{\text{osc}} &= \sum_i \hbar\omega_i (n_i + 1/2) \quad \text{(oscillator sum)} \\
U_{\text{part}} &= \sum_a \Phi_a N_a \quad \text{(aperture potential)}
\end{align}

All three:
\begin{enumerate}
\item \textbf{Explain equipartition}: Each active mode stores $k_B T$ of thermal energy
\item \textbf{Explain quantum freeze-out}: Modes with $\hbar\omega > k_B T$ are categorically inactive
\item \textbf{Classical correspondence}: Reduce to $U = fNk_B T/2$ when all modes are active
\item \textbf{Predict discrete heat capacity}: $C_V(T)$ exhibits steps as modes activate
\item \textbf{Distinguish thermal from zero-point energy}: Only thermal energy scales with $M_{\text{active}}$
\end{enumerate}

The categorical framework resolves the equipartition mystery: the $k_B T/2$ per quadratic degree of freedom is the energy required to maintain one categorical distinction against thermal fluctuations. Equipartition is not a fundamental postulate---it is the high-temperature limit of discrete categorical activation.
