\section{Enthalpy: The Equivalence Proof}
\label{sec:enthalpy}

Having derived entropy from three perspectives—categorical, oscillatory, and partition—we now demonstrate that the triple equivalence extends to enthalpy. We derive the same enthalpy formula from each perspective, proving that the three frameworks yield identical thermodynamic predictions.

\subsection{Classical Enthalpy}

In classical thermodynamics, enthalpy is defined as:
\begin{equation}
H = U + PV
\end{equation}

where $U$ is internal energy, $P$ is pressure, and $V$ is volume. The $PV$ term represents the work done against external pressure to establish the system's volume.

Enthalpy is the natural thermodynamic potential for processes at constant pressure. Its differential form:
\begin{equation}
dH = TdS + VdP + \mu dN
\end{equation}

It makes it particularly useful for chemical reactions and phase transitions.

We will derive enthalpy from each of the three perspectives and prove they all reduce to this classical form.

\subsection{Categorical Enthalpy}

\subsubsection{Categorical Potential}

In the categorical framework, each category (or aperture) has an associated potential that measures the ``cost'' of maintaining that categorical distinction.

\begin{definition}
The \textit{categorical potential} of aperture $a$ is:
\begin{equation}
\Phi_a = -k_B T \ln s_a = k_B T \ln n_a
\end{equation}
where $s_a = 1/n_a$ is the selectivity and $n_a$ is the number of accessible states.
\end{definition}

\textbf{Physical interpretation:}
\begin{itemize}
\item $\Phi_a$ measures the free energy cost of maintaining aperture $a$ in its current state
\item High categorical depth ($n_a \gg 1$) implies high potential: many states must be kept accessible
\item Low categorical depth ($n_a \to 1$) implies low potential: few states need maintenance
\item $\Phi_a$ is the work required to keep aperture $a$ ``open'' against the system's tendency to collapse to fewer states
\end{itemize}

The categorical potential is thermodynamically conjugate to the occupancy: $\Phi_a$ is the chemical potential for particles in category $a$.

\subsubsection{Categorical Enthalpy Definition}

The categorical enthalpy is internal energy plus the sum of aperture potentials weighted by occupancy:
\begin{equation}
\boxed{H_{\text{cat}} = U + \sum_{a=1}^{M} N_a \Phi_a = U + k_B T \sum_{a=1}^{M} N_a \ln n_a}
\label{eq:categorical_enthalpy}
\end{equation}

where $N_a$ is the number of particles (or excitations) occupying aperture $a$.

\textbf{Physical interpretation:}
\begin{itemize}
\item $U$ is the kinetic energy of particles
\item $\sum_a N_a \Phi_a$ is the potential energy associated with maintaining the categorical structure
\item Enthalpy accounts for both the energy of motion and the energy of configuration
\end{itemize}

\subsubsection{Reduction to Classical Enthalpy}

For an ideal gas with volume $V$ and $N$ particles, we establish the connection to $PV$.

\textbf{Spatial categories:} The number of spatial categories scales as:
\begin{equation}
M \propto \frac{V}{V_0}
\end{equation}
where $V_0$ is the elementary volume (typically on the order of molecular size).

\textbf{Categorical depth:} The number of accessible states per spatial category is:
\begin{equation}
n \approx \frac{V}{N V_0}
\end{equation}
This is the volume per particle in units of $V_0$.

\textbf{Total occupancy:} The conservation of particles gives:
\begin{equation}
\sum_{a=1}^{M} N_a = N
\end{equation}

For uniform distribution ($N_a = N/M$):
\begin{equation}
\sum_a N_a \Phi_a = N \cdot k_B T \ln\left(\frac{V}{N V_0}\right)
\end{equation}

Taking the volume derivative at a constant temperature:
\begin{equation}
\left(\frac{\partial}{\partial V}\right)_{T,N} \sum_a N_a \Phi_a = N k_B T \cdot \frac{1}{V} = \frac{Nk_B T}{V}
\end{equation}

By the ideal gas law $PV = Nk_B T$, we have:
\begin{equation}
\left(\frac{\partial}{\partial V}\right)_{T,N} \sum_a N_a \Phi_a = P
\end{equation}

This shows that the aperture potential term is thermodynamically conjugate to pressure. Integrating:
\begin{equation}
\sum_a N_a \Phi_a = PV + \text{const}
\end{equation}

Choosing the constant such that $\Phi_a \to 0$ as $V \to V_0$ (minimal volume):
\begin{equation}
\sum_a N_a \Phi_a = PV
\end{equation}

Therefore:
\begin{equation}
H_{\text{cat}} = U + \sum_a N_a \Phi_a = U + PV = H_{\text{classical}}
\end{equation}

\begin{figure*}[htbp]
\centering
\includegraphics[width=\textwidth]{figures/panel3_categorical_enthalpy.png}
\caption{\textbf{Categorical Enthalpy and the Emergence of Pressure.} 
(\textbf{A}) Aperture selectivity: Small molecules pass through apertures while large molecules are blocked. Selectivity $s_a = \Omega_{\text{pass}}/\Omega_{\text{total}}$ ranges from 0 (impermeable) to 1 (fully permeable). 
(\textbf{B}) Categorical potential $\Phi_a = -k_B T \ln s_a$ as a function of selectivity. At $s_a = 0.5$, the potential barrier is $\Phi_a = 0.69 k_B T$. As $s_a \to 1$, the barrier vanishes ($\Phi_a \to 0$). As $s_a \to 0$, the barrier diverges ($\Phi_a \to \infty$). 
(\textbf{C}) Categorical enthalpy definition: $\mathcal{H} = U + \sum_a n_a \Phi_a$, where $U$ is internal energy, $n_a$ is the number of type-$a$ apertures, and $\Phi_a$ is the categorical potential of aperture $a$. The aperture energy $\sum_a n_a \Phi_a$ represents the work required to maintain selective boundaries. 
(\textbf{D}) Classical limit: As selectivity $s_a \to 1$ and aperture density $\rho_a \to \infty$, individual apertures become non-selective and the aperture energy becomes continuous. 
(\textbf{E}) Pressure emerges from aperture statistics: The total aperture contribution is $\rho_a \cdot A \cdot \Phi_a$. Taking the limit $s_a \to 1$ and $\rho_a \to \infty$ while keeping the product finite defines pressure $P = \lim_{s_a \to 1} \rho_a \cdot (-k_B T \ln s_a)$. 
(\textbf{F}) Enthalpy transition from categorical to classical: The fundamental categorical form $\mathcal{H} = U + \int \sigma(x) \phi(x) \, dA$ reduces to the classical form $H = U + PV$ in the coarse-grained limit where aperture density $\sigma(x) \to 1$ and potential $\phi(x) \to P$. Classical thermodynamics emerges as the coarse-grained limit of categorical aperture dynamics.}
\label{fig:categorical_enthalpy}
\end{figure*}

\subsection{Oscillatory Enthalpy}

\subsubsection{Mode Energy}

In the oscillatory framework, the system decomposes into modes with frequencies $\{\omega_i\}$ and occupation numbers $\{n_i\}$. Each mode carries energy:
\begin{equation}
E_i = \hbar\omega_i \left(n_i + \frac{1}{2}\right)
\end{equation}

The internal energy is:
\begin{equation}
U = \sum_{i=1}^{N} E_i = \sum_{i=1}^{N} \hbar\omega_i \left(n_i + \frac{1}{2}\right)
\end{equation}

\subsubsection{Mode Potential}

Define the mode potential as the work required to maintain oscillation amplitude against dissipative forces and thermal fluctuations.

\begin{definition}
The \textit{mode potential} is:
\begin{equation}
\Psi_i = \hbar\omega_i
\end{equation}
This is the energy quantum of mode $i$---the cost of adding one excitation.
\end{definition}

\textbf{Physical interpretation:}
\begin{itemize}
\item $\Psi_i$ is the minimum energy required to activate mode $i$
\item Higher frequency modes have higher potentials
\item $\Psi_i$ sets the energy scale for thermal activation: modes with $\hbar\omega_i \gg k_B T$ are thermally inaccessible
\end{itemize}

\subsubsection{Oscillatory Enthalpy Definition}

The oscillatory enthalpy is internal energy plus the sum of mode potentials weighted by occupation:
\begin{equation}
\boxed{H_{\text{osc}} = U + \sum_{i=1}^{N} \Psi_i n_i = U + \sum_{i=1}^{N} \hbar\omega_i n_i}
\label{eq:oscillatory_enthalpy}
\end{equation}

Note that this differs from internal energy: $U$ includes zero-point energy ($\frac{1}{2}\hbar\omega_i$ per mode), while the enthalpy correction term includes only the excitation energy ($\hbar\omega_i n_i$).

\textbf{Alternative form using amplitude:} Since $n_i \propto A_i^2/A_0^2$ (occupation is proportional to amplitude squared):
\begin{equation}
H_{\text{osc}} = U + \sum_i \hbar\omega_i \left\langle \frac{A_i^2}{A_0^2} \right\rangle
\end{equation}

\subsubsection{Equivalence with Categorical Enthalpy}

For a system in thermal equilibrium at temperature $T$, the classical equipartition theorem states:
\begin{equation}
\langle E_i \rangle = k_B T \quad \Rightarrow \quad n_i = \frac{k_B T}{\hbar\omega_i}
\end{equation}

The mode potential term becomes:
\begin{equation}
\sum_{i=1}^{N} \hbar\omega_i n_i = \sum_{i=1}^{N} \hbar\omega_i \cdot \frac{k_B T}{\hbar\omega_i} = N k_B T
\end{equation}

From categorical enthalpy with uniform apertures ($n_a = n$ for all $a$) and one particle per aperture ($N_a = 1$):
\begin{equation}
\sum_{a=1}^{M} N_a \Phi_a = M \cdot k_B T \ln n
\end{equation}

For the natural choice $\ln n = 1$ (one nat of information per category) and $M = N$ (one mode per categorical dimension):
\begin{equation}
\sum_{a=1}^{M} N_a \Phi_a = N k_B T = \sum_{i=1}^{N} \hbar\omega_i n_i
\end{equation}

\textbf{Therefore:} $H_{\text{osc}} = H_{\text{cat}}$.

Both reduce to $H = U + Nk_BT$, which, for an ideal gas, equals $U + PV$.

\subsection{Partition Enthalpy}

\subsubsection{Transition Work}

In the partition framework, each partition transition requires work against the selectivity barrier. This work is the free energy cost of passing through the aperture.

\begin{definition}
The \textit{transition work} for partition $a$ is:
\begin{equation}
W_a = -k_B T \ln s_a = k_B T \ln n_a
\end{equation}
\end{definition}

This is identical to the categorical potential $\Phi_a$, reflecting the equivalence between partitions and categories.

\textbf{Physical interpretation:}
\begin{itemize}
\item $W_a$ is the activation energy for transition through partition $a$
\item Low selectivity ($s_a \to 0$) means a high barrier ($W_a \to \infty$)
\item High selectivity ($s_a \to 1$) means a low barrier ($W_a \to 0$)
\end{itemize}

\subsubsection{Partition Rate and Occupancy}

The rate at which transitions occur through partition $a$ is:
\begin{equation}
\dot{N}_a = \frac{N_a}{\tau_{p,a}}
\end{equation}

where $N_a$ is the number of particles in partition $a$ and $\tau_{p,a}$ is the partition lag (transition time).

The steady-state occupancy is determined by the balance between influx and efflux:
\begin{equation}
N_a = \dot{N}_a \cdot \tau_{p,a}
\end{equation}

\subsubsection{Partition Enthalpy Definition}

The partition enthalpy is internal energy plus the total transition work, weighted by the relative partition lags:
\begin{equation}
\boxed{H_{\text{part}} = U + \sum_{a=1}^{M} W_a \cdot \frac{N_a \tau_{p,a}}{\langle\tau_p\rangle} = U + k_B T \sum_{a=1}^{M} N_a \ln n_a}
\label{eq:partition_enthalpy}
\end{equation}

where $\langle\tau_p\rangle = (1/M)\sum_a \tau_{p,a}$ is the average partition lag.

For uniform partition lags ($\tau_{p,a} = \langle\tau_p\rangle$ for all $a$):
\begin{equation}
H_{\text{part}} = U + k_B T \sum_{a=1}^{M} N_a \ln n_a
\end{equation}

This is identical to the categorical enthalpy (Equation~\ref{eq:categorical_enthalpy}).

\subsubsection{Equivalence with Categorical and Oscillatory Enthalpy}

Comparing the three formulations:

\begin{align}
H_{\text{cat}} &= U + k_B T \sum_{a=1}^{M} N_a \ln n_a \\
H_{\text{osc}} &= U + \sum_{i=1}^{N} \hbar\omega_i n_i \\
H_{\text{part}} &= U + k_B T \sum_{a=1}^{M} N_a \ln n_a
\end{align}

\textbf{Categorical-Partition equivalence:} Immediate from identical functional forms.

\textbf{Oscillatory equivalence:} For thermal equilibrium with $\hbar\omega_i n_i = k_B T$ and uniform occupancy:
\begin{equation}
\sum_i \hbar\omega_i n_i = N k_B T = k_B T \sum_a N_a \ln n
\end{equation}
(for $\ln n = 1$ and $\sum_a N_a = N$).

All three reduce to:
\begin{equation}
H = U + N k_B T
\end{equation}

For an ideal gas, using $PV = Nk_BT$:
\begin{equation}
H = U + PV = H_{\text{classical}}
\end{equation}

\subsection{The Triple Equivalence Theorem}

We have now established the central result:

\begin{theorem}[Enthalpy Equivalence]
\label{thm:enthalpy_equivalence}
The categorical, oscillatory, and partition formulations of enthalpy are mathematically equivalent:
\begin{equation}
H_{\text{cat}} = H_{\text{osc}} = H_{\text{part}}
\end{equation}
and all reduce to the classical enthalpy $H = U + PV$ in the appropriate limits.
\end{theorem}

\begin{proof}
We have shown:
\begin{enumerate}
\item Categorical enthalpy: $H_{\text{cat}} = U + k_B T \sum_a N_a \ln n_a$ reduces to $U + PV$ for ideal gases (Section 5.2.3)
\item Oscillatory enthalpy: $H_{\text{osc}} = U + \sum_i \hbar\omega_i n_i$ equals $U + Nk_BT = U + PV$ at thermal equilibrium (Section 5.3.4)
\item Partition enthalpy: $H_{\text{part}} = U + k_B T \sum_a N_a \ln n_a$ is identical to $H_{\text{cat}}$ (Section 5.4.4)
\end{enumerate}
Therefore $H_{\text{cat}} = H_{\text{osc}} = H_{\text{part}} = U + PV$.
\end{proof}

\begin{figure}[htbp]
\centering
\includegraphics[width=\textwidth]{figures/panel_categorical_enthalpy.png}
\caption{\textbf{Categorical enthalpy $H = k_B T \sum_a \ln(1/s_a)$ quantifies partition energy cost across transport types.} 
\textbf{(Top left)} Electrical categorical enthalpy showing energy required to partition electrons through scattering apertures. Copper (orange) has $H \sim 0$ at low $T$, increasing linearly to $H \sim 1.8$ eV at 500 K as phonon population grows. Aluminum (yellow) shows similar behavior. Tungsten (gray) has higher enthalpy due to stronger electron-phonon coupling.
\textbf{(Top right)} Diffusive categorical enthalpy showing energy required for atomic diffusion. Bulk diffusion (green) has highest enthalpy $H \sim 3.5$ eV, corresponding to breaking bonds and moving through lattice. Grain boundary diffusion (cyan) has lower enthalpy $H \sim 1.5$ eV as atoms move along defects. Surface diffusion (magenta) has lowest enthalpy $H \sim 0.8$ eV as atoms hop along surface.
\textbf{(Bottom left)} Thermal categorical enthalpy showing energy required for phonon scattering. Diamond (cyan) has highest enthalpy $H \sim 1$ eV due to strong covalent bonds. Silicon (green) has moderate enthalpy $H \sim 0.6$ eV. Copper (orange) has lower enthalpy $H \sim 0.3$ eV as electron transport dominates. Lead (gray) has lowest enthalpy $H \sim 0.1$ eV due to heavy atoms and weak bonds.
\textbf{(Bottom right)} Viscous categorical enthalpy showing energy required for momentum transfer between molecules. Water (cyan) has low enthalpy $H \sim 0.2$ eV, decreasing with temperature. Glycerol (magenta) has high enthalpy $H \sim 0.7$ eV at low $T$, decreasing to $\sim 0.4$ eV at 600 K as hydrogen bonds break. Silicone oil (yellow) has intermediate enthalpy $H \sim 0.3$ eV. The categorical enthalpy $H = k_B T \sum_a \ln(1/s_a)$ provides a unified measure of transport resistance across all four transport modes, quantifying the total energy cost of partition operations.}
\label{fig:categorical_enthalpy}
\end{figure}

\subsection{Extension to Other Thermodynamic Quantities}

The equivalence proven for entropy (Sections~\ref{sec:categorical}--\ref{sec:partition}) and enthalpy (this section) extends to all thermodynamic quantities. Since temperature, pressure, chemical potential, free energy, and all other state functions can be derived from entropy and enthalpy through thermodynamic relations, the triple equivalence propagates throughout the entire framework.

\textbf{Temperature:} From the fundamental identity (Equation~\ref{eq:fundamental}):
\begin{align}
T_{\text{cat}} &= \frac{\hbar}{k_B} \frac{dM}{dt} \quad \text{(categorical actualization rate)} \\
T_{\text{osc}} &= \frac{\hbar}{k_B} \langle\omega\rangle \quad \text{(average oscillation frequency)} \\
T_{\text{part}} &= \frac{\hbar}{k_B} \frac{1}{\langle\tau_p\rangle} \quad \text{(inverse partition lag)}
\end{align}

All three are equal: $T_{\text{cat}} = T_{\text{osc}} = T_{\text{part}}$.

\textbf{Pressure:} From the volume derivatives:
\begin{align}
P_{\text{cat}} &= k_B T \left(\frac{\partial M}{\partial V}\right)_{T,N} \quad \text{(categorical density)} \\
P_{\text{osc}} &= \frac{1}{3V}\sum_i m_i \omega_i^2 A_i^2 \quad \text{(momentum flux)} \\
P_{\text{part}} &= \frac{k_B T}{V} \sum_a \frac{1}{\tau_{p,a}} \quad \text{(transition rate density)}
\end{align}

For equilibrium systems, all reduce to $P = Nk_B T/V$.

\textbf{Internal Energy:} From the energy-entropy relations:
\begin{align}
U_{\text{cat}} &= k_B T \cdot M_{\text{active}} \quad \text{(active categorical dimensions)} \\
U_{\text{osc}} &= \sum_i \hbar\omega_i \left(n_i + \frac{1}{2}\right) \quad \text{(mode energies)} \\
U_{\text{part}} &= \sum_a N_a \Phi_a \quad \text{(partition occupancy)}
\end{align}

For classical ideal gases, all exhibit $U = \frac{3}{2}Nk_B T$ (three translational degrees of freedom).

\subsection{Summary}

The categorical, oscillatory, and partition formulations of enthalpy are equivalent:
\begin{equation}
H_{\text{cat}} = H_{\text{osc}} = H_{\text{part}} = U + PV
\end{equation}

This equivalence, combined with the entropy equivalence proven in Sections~\ref{sec:categorical}--\ref{sec:partition}, demonstrates that the triple framework yields a complete and self-consistent thermodynamics. All classical results are recovered, while the discrete categorical structure provides:
\begin{itemize}
\item A foundation that resolves conceptual issues (infinite velocity tails, negative absolute temperatures)
\item Natural connexion to quantum mechanics (through $\hbar\omega$ quantisation)
\item Unified treatment of equilibrium and non-equilibrium processes (through partition dynamics)
\item Direct link to information theory (through categorical counting)
\end{itemize}

\textbf{Key insight:} Enthalpy, like entropy, admits three complementary interpretations:
\begin{enumerate}
\item \textbf{Categorical}: Enthalpy is internal energy plus the work to maintain aperture structure
\begin{equation}
H = U + \sum_a N_a \Phi_a
\end{equation}

\item \textbf{Oscillatory}: Enthalpy is internal energy plus the excitation energy of modes
\begin{equation}
H = U + \sum_i \hbar\omega_i n_i
\end{equation}

\item \textbf{Partition}: Enthalpy is internal energy plus the work against selectivity barriers
\begin{equation}
H = U + \sum_a W_a N_a
\end{equation}
\end{enumerate}

All three descriptions are mathematically identical and physically equivalent. The choice of perspective depends on which aspect of the system is most natural for the problem at hand.
