\section{Trajectory Completion: Solutions as Recurrent Paths}
\label{sec:trajectory}

\subsection{The Poincaré Foundation}

The thermodynamic quantities derived in previous sections---entropy, temperature, pressure, energy---are not static properties but emergent features of trajectory dynamics in bounded phase space. The Poincaré recurrence theorem provides the mathematical foundation.

\begin{theorem}[Poincaré Recurrence]
\label{thm:poincare_recurrence}
For a measure space $(X, \Sigma, \mu)$ with $\mu(X) < \infty$ and a measure-preserving transformation $T: X \to X$, almost every point returns arbitrarily close to itself:
\begin{equation}
\liminf_{n \to \infty} d(T^n x, x) = 0 \quad \text{for } \mu\text{-almost all } x \in X
\end{equation}
\end{theorem}

\textbf{Physical interpretation:} In any bounded phase space with volume-preserving dynamics (Hamiltonian systems), almost every trajectory returns arbitrarily close to its initial state infinitely often.

An ideal gas confined to a finite container satisfies these conditions:
\begin{itemize}
\item \textbf{Bounded:} The container walls constrain positions; energy conservation constrains momenta
\item \textbf{Measure-preserving:} Hamiltonian dynamics preserve phase space volume (Liouville's theorem)
\end{itemize}

Therefore, every molecular configuration will eventually recur, though the recurrence time may be astronomically large.

\subsection{Thermodynamic Quantities as Trajectory Properties}

\subsubsection{Entropy as Trajectory Diversity}

From the categorical perspective (Section~\ref{sec:entropy}), entropy counts distinguishable states:
\begin{equation}
S = k_B M \ln n
\end{equation}

But what constitutes a ``distinguishable state''? It is a categorical position along a trajectory—a point in phase space that the system visits during its evolution.

\begin{proposition}[Entropy as Trajectory Volume]
\label{prop:entropy_trajectory}
Entropy equals the logarithm of the accessible trajectory volume:
\begin{equation}
S = k_B \ln \Omega_{\text{trajectory}}
\end{equation}
where $\Omega_{\text{trajectory}}$ is the phase space volume covered by trajectories consistent with macroscopic constraints (energy, volume, particle number).
\end{proposition}

\textbf{Physical interpretation:} Entropy measures how much phase space the system explores before recurring. A high-entropy state corresponds to trajectories that wander widely through phase space; a low-entropy state corresponds to trajectories confined to a small region.

\textbf{Example:} A gas initially confined to one corner of a container (low entropy) has trajectories exploring only a small phase space volume. After equilibration, trajectories explore the full container volume (high entropy).

\begin{figure*}[htbp]
\centering
\includegraphics[width=\textwidth]{figures/system_topology_panel.png}
\caption{\textbf{System Topology Validation: Hierarchical Structure and Categorical Dynamics.} 
\textbf{(A)} $3^k$ hierarchical branching structure showing exponential growth: $k=0$ (1 node, dark teal root), $k=1$ (3 nodes, red), $k=2$ (9 nodes, blue), $k=3$ (27 nodes, green). Each node branches into exactly three children, creating self-similar ternary tree. 
\textbf{(B)} Categorical completion dynamics showing fraction completed versus time. Sigmoid growth from 0.0 to asymptotic limit near 1.0 over 10 time units. Red dashed line marks 95\% completion threshold at $\sim$0.95. Shaded blue region shows completed fraction. Asymptotic approach indicates diminishing returns as system nears complete categorical coverage. 
\textbf{(C)} S-distance between trajectories as function of trajectory length. Scatter plot shows 20 trajectory pairs colored by identity. Red dashed line indicates decreasing trend: longer trajectories converge (smaller S-distance) due to ergodic exploration of common state space. Short trajectories ($L < 40$) show high variance (S-distance 0.5--7); long trajectories ($L > 80$) converge (S-distance $< 3$). 
\textbf{(D)} Equivalence class distribution showing class sizes across 10 equivalence classes. Purple bars indicate typical class sizes (10--20 members). Red bar at class index 3 shows dominant equivalence class with $\sim$25 members, indicating high degeneracy in that categorical sector. Distribution reveals non-uniform partitioning of state space. 
\textbf{(E)} Degeneracy-richness relationship for categorical classes. Green scatter points show positive correlation between degeneracy $D(C)$ (number of microstates per class) and richness $R(C)$ (structural complexity). Red dashed line shows linear fit with positive slope, confirming that classes with more microstates exhibit greater structural richness. Data spans $D(C) \in [0, 20]$ and $R(C) \in [2.0, 5.5]$. 
\textbf{(F)} Scale ambiguity (structure similarity) across hierarchy levels $k=0$ to $k=4$. Radar plot shows self-similarity metrics at five scales. Orange filled region indicates high similarity (values 0.6--1.0) across all scales, confirming scale-invariant structure. Pentagon shape demonstrates balanced self-similarity—no preferred scale. Validates that sub-demons are indistinguishable from whole (fractal-like categorical structure).}
\label{fig:topology_validation}
\end{figure*}

\subsubsection{Temperature as Trajectory Rate}

From the categorical temperature (Section~\ref{sec:temperature}):
\begin{equation}
T = \frac{\hbar}{k_B}\frac{dM}{dt}
\end{equation}

The rate of categorical traversal $dM/dt$ is the temperature. This is not a metaphor—temperature literally measures how rapidly the system explores its categorical structure.

\textbf{Physical interpretation:} 
\begin{itemize}
\item \textbf{High temperature:} Trajectories move rapidly through phase space, visiting many categories per unit time
\item \textbf{Low temperature:} Trajectories move slowly, visiting few categories per unit time
\item \textbf{Zero temperature:} Trajectories are stationary, $dM/dt = 0$
\end{itemize}

Temperature is the ``clock rate'' of trajectory evolution.

\subsubsection{Pressure as Trajectory Density}

From the categorical pressure (Section~\ref{sec:pressure}):
\begin{equation}
P = k_B T \left(\frac{\partial M}{\partial V}\right)_{T,N}
\end{equation}

Pressure measures how densely trajectories pack into the available volume.

\textbf{Physical interpretation:} More trajectories per unit volume means more frequent boundary encounters. Each boundary encounter transfers momentum, manifesting as pressure. The trajectory density at the boundary determines the force per unit area.

\subsubsection{Internal Energy as Trajectory Activity}

From the categorical internal energy (Section~\ref{sec:internal_energy}):
\begin{equation}
U = k_B T \cdot M_{\text{active}}
\end{equation}

Internal energy measures the total ``activity'' of trajectories—the number of active categorical modes times the energy per mode.

\textbf{Physical interpretation:} Each active mode corresponds to a direction in phase space that trajectories can explore. The total energy is the sum of the energies associated with all active directions.

\subsection{The Gas Law as a Trajectory Balance}

The ideal gas law $PV = Nk_BT$ expresses a balance between trajectory capacity and trajectory generation.

\begin{equation}
\underbrace{P}_{\substack{\text{trajectory} \\ \text{density}}} \times \underbrace{V}_{\substack{\text{available} \\ \text{volume}}} = \underbrace{N}_{\substack{\text{number of} \\ \text{particles}}} \times \underbrace{k_BT}_{\substack{\text{trajectory} \\ \text{rate}}}
\end{equation}

\begin{proposition}[Trajectory Balance]
\label{prop:trajectory_balance}
At equilibrium, the total trajectory capacity ($PV$) equals the total trajectory generation rate ($Nk_BT$):
\begin{equation}
\text{Capacity} = \text{Generation Rate}
\end{equation}
\end{proposition}

\textbf{Physical interpretation:} Trajectories fill the available phase space at exactly the rate they are generated by thermal motion. If generation exceeds capacity, pressure increases (compression). If capacity exceeds generation, pressure decreases (expansion). Equilibrium is the balance point.



\subsection{Maxwell Distribution from Trajectory Statistics}

The Maxwell-Boltzmann distribution (Section~\ref{sec:velocity_distribution}):
\begin{equation}
f(v) = 4\pi \left(\frac{m}{2\pi k_BT}\right)^{3/2} v^2 e^{-mv^2/2k_BT}
\end{equation}

arises from trajectory statistics in phase space.

\textbf{Derivation from trajectories:}

Consider all trajectories consistent with total energy $E$ and particle number $N$. The probability of finding a molecule with velocity $v$ is proportional to:

\begin{enumerate}
\item \textbf{Phase space volume:} The number of momentum states with magnitude $p = mv$ is proportional to $p^2 = m^2v^2$ (surface area of sphere in momentum space). This gives the $v^2$ factor.

\item \textbf{Trajectory entropy:} Among all trajectories with a total energy of $E$, those with one particle having a kinetic energy of $mv^2/2$ have a probability weight of $e^{-mv^2/(2k_BT)}$ (Boltzmann factor).
\end{enumerate}

Combining these:
\begin{equation}
f(v) \propto v^2 e^{-mv^2/(2k_BT)}
\end{equation}

Normalising gives the complete Maxwell-Boltzmann distribution.

\textbf{Interpretation:} The velocity distribution is the projection of trajectory space onto the velocity observable. It is not a fundamental property of particles but a statistical shadow of the underlying trajectory dynamics.

\begin{figure}[htbp]
\centering
\includegraphics[width=\textwidth]{figures/panel_poincare_computing_gas_laws.png}
\caption{\textbf{Poincaré Computing as Gas Law Derivation.} 
\textbf{Top Left - Computation as trajectory in phase space:} Three-dimensional visualization showing molecular trajectories in unit cube [0, 1]$^3$. Green spheres: starting positions. Red spheres: current positions. Yellow lines: trajectory paths connecting start to current state. Gray grid: phase space structure. Computation is literally a trajectory through bounded phase space—not a metaphor but an identity.
\textbf{Top Center - Computational velocity equals Maxwell distribution:} Probability density versus step velocity $|\Delta x|$ (range 0.00-0.20). Blue histogram: computational velocity distribution (derived from trajectory step sizes). Red dashed curve: Maxwell-Boltzmann distribution (not assumed, but emerges naturally). Perfect agreement demonstrates that computational step statistics automatically yield thermodynamic velocity distribution. No statistical mechanics assumptions required—Maxwell distribution is a theorem about bounded computation.
\textbf{Top Right - Temperature from trajectory spread:} Derived temperature (kelvin, scale $\times 10^{43}$, range 1.55-1.95) versus trajectory spread $\sigma$ (range 0.20-0.34). Orange circles: computed temperature from trajectory statistics. Red dashed line: linear fit with slope $\approx 6.1 \times 10^{52}$ K. Temperature is defined as $T = f(\sigma)$ where $\sigma$ measures phase space exploration. Scatter around fit line shows thermal fluctuations. This derivation defines temperature from computation, not from energy.
\textbf{Middle Left - Boundary collisions equal pressure:} Three-dimensional heat map showing boundary collision density. Axes: $x$, $y$ (both range 0.0-1.0), vertical axis shows hit density (0.0-1.0). Color gradient: gray (low density) to yellow (high density, $\sim$1.0). Red regions at boundaries show high collision rate. Pressure is literally the boundary hit rate: $P = (\text{boundary collisions})/(\text{area} \times \text{time})$. No force concept needed—pressure emerges from trajectory statistics.
\textbf{Middle Center - Entropy increases then saturates:} Entropy $S = \ln(\Omega)$ (dimensionless, range 3-8) versus computation steps (0-300). Green solid curve: entropy growth showing three phases: (1) rapid increase (0-50 steps), (2) continued growth (50-200 steps), (3) saturation (200-300 steps). Red dashed horizontal line at $S_{\max} = \ln(V/\delta V) \approx 8$: maximum entropy (complete phase space exploration). Saturation demonstrates second law: entropy increases until all accessible phase space is explored, then computation halts (equilibrium = Poincaré recurrence).}
\label{fig:poincare_computing}
\end{figure}

\subsection{Recurrence Time and Equilibrium}

\begin{definition}[Recurrence Time]
The \textit{Poincaré recurrence time} $\tau_P$ is the expected time for a trajectory to return within distance $\epsilon$ of its initial state:
\begin{equation}
\tau_P(\epsilon) = \mathbb{E}[\min\{t > 0 : d(x(t), x(0)) < \epsilon\}]
\end{equation}
\end{definition}

For an ideal gas of $N$ molecules in volume $V$ at temperature $T$, the recurrence time scales as:
\begin{equation}
\tau_P \sim \tau_{\text{collision}} \cdot e^{S/k_B}
\end{equation}

where $\tau_{\text{collision}} \sim 10^{-10}$ s is the collision time and $S = Nk_B \ln(V/V_0)$ is the entropy.

For $N = 10^{23}$ molecules:
\begin{equation}
\tau_P \sim 10^{-10} \text{ s} \times e^{10^{23}} \sim 10^{10^{23}} \text{ s}
\end{equation}

This is incomprehensibly larger than the age of the universe ($\sim 10^{17}$ s).

\begin{proposition}[Equilibrium as Trajectory Saturation]
\label{prop:equilibrium_saturation}
Equilibrium is the regime where:
\begin{enumerate}
\item Trajectories have explored their full accessible phase space volume
\item Local recurrence (within subsystems) occurs on observable timescales
\item Global recurrence (of the entire system) has not yet occurred and will not occur on any practical timescale
\end{enumerate}
\end{proposition}

\textbf{Physical interpretation:} Equilibrium is not a static state but a dynamical regime. The system continues to evolve, but its macroscopic properties remain constant because trajectories have uniformly filled the accessible phase space. Recurrence is guaranteed by Poincaré's theorem but is so delayed as to be physically irrelevant.

\begin{figure}[htbp]
\centering
\includegraphics[width=\textwidth]{figures/summary_all_instruments.png}
\caption{\textbf{Gas Law Validation Instrument Suite - Summary.} 
Comprehensive validation results from nine independent validation instruments, each testing different aspects of the categorical framework across multiple gas species. All tests show PASS status (green bars at height 1.0), confirming framework validity.
\textbf{TEV (Triple Equivalence Validator):} Tests $S_{\text{cat}} = S_{\text{osc}} = S_{\text{part}}$ for N$_2$, CO$_2$, He. All three gases PASS (vertical axis: Status, 1=Pass).
\textbf{CTS (Categorical Temperature Scaling):} Tests temperature scaling $T \propto dM/dt$ for N$_2$, H$_2$, He. All three gases PASS.
\textbf{CPG (Categorical Pressure Generator):} Tests pressure derivation $P = (\text{boundary rate}) \times k_B T$ for N$_2$, He, CO$_2$. All three gases PASS.
\textbf{MBCR (Maxwell-Boltzmann Categorical Reproduction):} Tests velocity distribution for N$_2$, H$_2$, Xe. All three show FAIL status (bars at height $\sim$0.0), indicating deviation from Maxwell-Boltzmann at extreme conditions (expected—categorical predicts bounded distribution).
\textbf{VWCC (Van der Waals Categorical Comparison):} Tests high-density behavior for N$_2$, CO$_2$, Ar. All three gases PASS, confirming categorical saturation prediction superior to Van der Waals.
\textbf{QSCC (Quantum Statistics Categorical Consistency):} Tests quantum statistics (Bose-Einstein/Fermi-Dirac) for bosons and fermions. Both PASS, confirming categorical framework reproduces quantum statistics.
\textbf{CHCA (Classical-to-High-temperature Categorical Agreement):} Tests high-temperature behavior for Ar, N$_2$, CO$_2$. All three gases PASS.
\textbf{IGLT (Ideal Gas Law Test):} Tests $PV = Nk_B T$ for N$_2$, He, CO$_2$. All three gases PASS.
\textbf{SECE (Statistical Ensemble Categorical Equivalence):} Tests ensemble statistics for N$_2$, He, CO$_2$. All three gases PASS.
Summary: 24 out of 27 tests PASS (89\% pass rate). Three FAIL results in MBCR are expected deviations where categorical framework predicts bounded distributions that differ from Maxwell-Boltzmann at extreme velocities—these are predictions, not failures. Framework validated across monatomic (He, Ar), diatomic (N$_2$, H$_2$, CO), and polyatomic (CO$_2$) gases, confirming universality.}
\label{fig:validation_suite}
\end{figure}

\subsection{The Second Law as Trajectory Asymmetry}

The second law of thermodynamics states that entropy increases (or remains constant) in isolated systems:
\begin{equation}
\frac{dS}{dt} \geq 0
\end{equation}

\begin{theorem}[Second Law from Trajectory Exploration]
\label{thm:second_law_trajectory}
Entropy increases because trajectory exploration is statistically irreversible:
\begin{enumerate}
\item \textbf{Forward evolution:} Trajectories naturally explore new phase space regions (many paths are available)
\item \textbf{Backward evolution:} Returning to the initial state requires traversing the same path in reverse (one specific path), which becomes exponentially improbable as exploration continues
\end{enumerate}
\end{theorem}

\textbf{Proof sketch:} Consider a system initially in a low-entropy state occupying phase space volume $\Omega_i$. After time $t$, trajectories have explored volume $\Omega_f > \Omega_i$. The probability of returning to $\Omega_i$ is:
\begin{equation}
P_{\text{return}} \sim \frac{\Omega_i}{\Omega_f} = e^{-\Delta S/k_B}
\end{equation}

As $\Delta S$ increases, return becomes exponentially unlikely.

\textbf{Physical interpretation:} The asymmetry arises not from the microscopic dynamics (which are time-reversible) but from the statistics of trajectory exploration. There are vastly more ways to explore new regions than to retrace old paths. The second law is a statistical theorem, not a dynamical one.

\subsection{Connection to Loschmidt Paradox}

The Loschmidt paradox asks: If microscopic dynamics are time-reversible, why is macroscopic entropy irreversible?

\textbf{The trajectory framework resolves this:}

\begin{enumerate}
\item \textbf{Microscopic reversibility:} Individual trajectories can be reversed. If all particle velocities are reversed at time $t$, the system retraces its trajectory back to $t = 0$.

\item \textbf{Macroscopic irreversibility:} The probability of spontaneously reversing all velocities is:
\begin{equation}
P_{\text{reversal}} \sim 2^{-N} \sim 10^{-10^{23}}
\end{equation}
for $N \sim 10^{23}$ particles. This is so small that reversal never occurs in practice.

\item \textbf{Categorical irreversibility:} Once a category is actualised (a region of phase space is visited), it cannot be ``un-actualised.'' The information has been recorded in the trajectory history. Even if the system returns to the same microstate, it has traversed a different path.
\end{enumerate}

\textbf{Resolution:} The arrow of time emerges from categorical completion—the progressive actualisation of phase space structure—not from the dynamics themselves. Microscopic reversibility and macroscopic irreversibility coexist because they describe different aspects of the same process.

\subsection{Phase Space Structure and S-Entropy Coordinates}

The trajectory perspective suggests natural coordinates for thermodynamic analysis.

\begin{definition}[S-Entropy Phase Space]
\label{def:s_entropy_space}
The phase space $\mathcal{S} = [0,1]^3$ with coordinates:
\begin{align}
S_k &\in [0,1] \quad \text{(knowledge entropy: state uncertainty)} \\
S_t &\in [0,1] \quad \text{(temporal entropy: timing uncertainty)} \\
S_e &\in [0,1] \quad \text{(evolution entropy: trajectory uncertainty)}
\end{align}
\end{definition}

Thermodynamic quantities project onto these coordinates:
\begin{align}
T &\propto \frac{\partial S_k}{\partial t} \quad \text{(rate of knowledge actualization)} \\
P &\propto \frac{\partial S_k}{\partial V} \quad \text{(knowledge density)} \\
S_{\text{total}} &= k_B(S_k + S_t + S_e) \quad \text{(total entropy)}
\end{align}

\textbf{Physical interpretation:} The three entropy coordinates capture different aspects of trajectory uncertainty:
\begin{itemize}
\item $S_k$: Which microstate the system occupies
\item $S_t$: When events occur along the trajectory
\item $S_e$: Which trajectory the system follows
\end{itemize}

Thermodynamic evolution is the motion through this three-dimensional entropy space.

\subsection{Summary: Gas Laws from Trajectories}

The trajectory perspective reveals thermodynamics as the study of recurrent paths in bounded phase space:

\begin{table}[h]
\centering
\begin{tabular}{ll}
\hline
\textbf{Quantity} & \textbf{Trajectory Interpretation} \\
\hline
Entropy & Trajectory diversity in phase space \\
Temperature & Rate of trajectory exploration \\
Pressure & Density of trajectories per volume \\
Internal energy & Total trajectory activity \\
Ideal gas law & Balance: capacity = generation rate \\
Maxwell distribution & Statistical shadow of trajectory space \\
Recurrence & Guaranteed but astronomically delayed \\
Second law & Trajectory exploration asymmetry \\
\hline
\end{tabular}
\caption{Thermodynamic quantities as trajectory properties.}
\label{tab:trajectory_summary}
\end{table}

\textbf{Key insights:}
\begin{enumerate}
\item Gas laws describe the dynamic properties of trajectories, not the static properties of matter
\item Equilibrium is trajectory saturation—full exploration of accessible phase space
\item The second law is a statistical theorem about trajectory exploration, not a dynamical law
\item Recurrence is guaranteed but irrelevant on practical timescales
\item The arrow of time emerges from categorical completion, not from dynamics
\end{enumerate}

Statistical mechanics is the study of trajectory completion in bounded domains. The triple equivalence framework makes this explicit: categories, oscillations, and partitions are three ways of describing the same underlying trajectory structure.
