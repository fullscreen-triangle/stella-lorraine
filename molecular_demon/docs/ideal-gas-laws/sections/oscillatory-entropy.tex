\section{Oscillatory Entropy}
\label{sec:oscillatory}

\subsection{Oscillators as Fundamental Degrees of Freedom}

From the triple equivalence (Theorem~\ref{thm:triple_equivalence}), bounded dynamics manifests as oscillation. A macroscopic system decomposes into a collection of oscillators---vibrational modes, rotational modes, translational modes---each characterised by frequency $\omega_i$ and amplitude $A_i$.

The oscillatory perspective derives entropy by summing over these modes, weighted by their phase space volumes.

\subsection{Phase Space Volume of an Oscillator}

Consider a harmonic oscillator with mass $m$, frequency $\omega$, and amplitude $A$. Its trajectory traces an ellipse in phase space $(x, p)$:
\begin{equation}
\frac{x^2}{A^2} + \frac{p^2}{(m\omega A)^2} = 1
\end{equation}

The area enclosed by this ellipse is:
\begin{equation}
\Gamma = \pi \cdot A \cdot (m\omega A) = \pi m\omega A^2
\end{equation}

For an oscillator with energy $E = \frac{1}{2}m\omega^2 A^2$, we have $A^2 = 2E/(m\omega^2)$, which gives:
\begin{equation}
\Gamma = \pi m\omega \cdot \frac{2E}{m\omega^2} = \frac{2\pi E}{\omega}
\end{equation}

In quantum mechanics, phase space is quantised in units of $h = 2\pi\hbar$. The number of quantum states enclosed is:
\begin{equation}
n = \frac{\Gamma}{h} = \frac{2\pi E}{\omega \cdot 2\pi\hbar} = \frac{E}{\hbar\omega}
\end{equation}

For a quantum harmonic oscillator with energy $E = (n + 1/2)\hbar\omega$, this gives $n$ as the occupation number, confirming the correspondence between classical phase space volume and quantum state counting.

\begin{figure}[htbp]
\centering
\includegraphics[width=\textwidth]{figures/fig_pendulum_triple_equivalence.png}
\caption{\textbf{Pendulum Demonstrates Triple Equivalence: Oscillation = Category = Partition.} 
\textbf{Top Left - Oscillatory View:} Simple pendulum (black pivot point, black bob, gray reference positions, blue arrow showing current position). Equation: $\theta(t) = \theta_{\max}\cos(\omega t)$. Continuous periodic motion in angle coordinate.
\textbf{Top Center - Continuous Periodic Motion:} Two traces versus time (0-12 in units of $t/T$): blue solid line ($\theta(t)$, angle), cyan dashed line ($\dot{\theta}(t)$, angular velocity). One complete period $T$ spans from $t = 0$ to $t = T$ (black arrow). Sinusoidal oscillation with phase shift between position and velocity.
\textbf{Top Right - Phase Space (Ellipse):} Phase portrait showing $\dot{\theta}$ versus $\theta$ (both axes range $-0.4$ to $0.4$). Blue ellipse: phase space trajectory. Two red dots: current state showing position on ellipse. Closed trajectory indicates periodic motion with no dissipation.
\textbf{Middle Center - Discrete State Structure:} Bar chart showing time in category versus category index (C$_1$ to C$_8$). Green bars with heights ranging 0.15 to 0.8. Peak at categories C$_3$ and C$_4$ (height $\approx 0.8$) corresponds to slow motion near turning points. Minimum at C$_1$ and C$_8$ (height $\approx 0.15$) corresponds to fast motion through equilibrium. Black arrows labeled ``Traversal'' indicate sequential category progression.
\textbf{Bottom Left - Categorical View:} Eight green spheres (C$_1$ through C$_8$) arranged in arc, connected by gray lines to black pivot point above. Text: ``$M = 8$ categories. Each $C_i$ is a distinguishable state.'' Pendulum motion discretized into eight categorical regions.
\textbf{Bottom Right - Partition View:} Eight pink/red rectangles (P$_1$ through P$_8$) arranged horizontally along time axis (0 to $T$). Color gradient from light pink (short duration) to dark red (long duration). Black arrow labeled $t$ points right. Equation: $T = \sum_{i=1}^M \tau_i$. Text: ``Each partition = one category transition.'' Average partition duration: $\langle\tau_p\rangle = T/M$.
\textbf{Bottom - Triple Equivalence Statement:} Yellow box with black border: ``TRIPLE EQUIVALENCE: Oscillation = Category Traversal = Period Partition.'' Below: ``Fundamental Identity: $dM/dt = \omega/(2\pi/M) = 1/\langle\tau_p\rangle$.'' This identity connects categorical rate (left), oscillation frequency (center), and partition rate (right), proving all three perspectives measure the same dynamics.}
\label{fig:pendulum_triple_equivalence}
\end{figure}

\subsection{Amplitude as Measure of Accessible States}

For a classical oscillator, the amplitude $A$ determines the extent of phase space exploration. A larger amplitude means more accessible states. Define a reference amplitude $A_0$ corresponding to the ground state (minimum accessible phase space):
\begin{equation}
\Gamma_0 = \pi m\omega A_0^2
\end{equation}

The ratio of accessible phase space volumes is:
\begin{equation}
\frac{\Gamma}{\Gamma_0} = \frac{A^2}{A_0^2}
\end{equation}

This ratio measures how many times more phase space is accessible at amplitude $A$ compared to the ground state amplitude $A_0$. In quantum terms, $\Gamma/\Gamma_0 = n/n_0$ is the ratio of occupation numbers.

\subsection{Derivation of Oscillatory Entropy}

For a system of $N$ independent oscillators with amplitudes $\{A_i\}$ and frequencies $\{\omega_i\}$, the total accessible phase space volume is:
\begin{equation}
\Gamma_{\text{total}} = \prod_{i=1}^{N} \Gamma_i = \prod_{i=1}^{N} \pi m_i \omega_i A_i^2
\end{equation}

Following Boltzmann's principle, entropy is the logarithm of the accessible phase space volume (in units of $h^N$):
\begin{equation}
S = k_B \ln\left(\frac{\Gamma_{\text{total}}}{\Gamma_0^N}\right) = k_B \sum_{i=1}^{N} \ln\left(\frac{\Gamma_i}{\Gamma_0}\right) = k_B \sum_{i=1}^{N} \ln\left(\frac{A_i^2}{A_0^2}\right)
\end{equation}

This yields the oscillatory entropy formula:
\begin{equation}
\boxed{S_{\text{osc}} = k_B \sum_{i=1}^{N} \ln\left(\frac{A_i}{A_0}\right)^2}
\label{eq:oscillatory_entropy}
\end{equation}

Equivalently, in terms of phase space volumes:
\begin{equation}
S_{\text{osc}} = k_B \sum_{i=1}^{N} \ln\left(\frac{\Gamma_i}{\Gamma_0}\right)
\end{equation}

Or in terms of energies (using $\Gamma \propto E/\omega$):
\begin{equation}
S_{\text{osc}} = k_B \sum_{i=1}^{N} \ln\left(\frac{E_i/\omega_i}{E_0/\omega_0}\right)
\end{equation}

\textbf{Physical interpretation:}
\begin{itemize}
\item Each oscillator $i$ contributes $\ln(\Gamma_i/\Gamma_0)$ to the total entropy
\item Larger amplitudes (more phase space exploration) contribute more entropy
\item The sum over modes reflects the independence of oscillatory degrees of freedom
\item The reference amplitude $A_0$ sets the zero point of entropy (analogous to ground state)
\end{itemize}


\subsection{Connection to Energy and Temperature}

For a harmonic oscillator, amplitude and energy are related by:
\begin{equation}
E = \frac{1}{2}m\omega^2 A^2 \quad \Rightarrow \quad A^2 = \frac{2E}{m\omega^2}
\end{equation}

Thus:
\begin{equation}
\ln\left(\frac{A_i}{A_0}\right)^2 = \ln\left(\frac{E_i}{E_0}\right)
\end{equation}

The oscillatory entropy becomes expressed as follows:
\begin{equation}
S_{\text{osc}} = k_B \sum_{i=1}^{N} \ln\left(\frac{E_i}{E_0}\right)
\end{equation}

For a system in thermal equilibrium at temperature $T$, the classical equipartition theorem gives $\langle E_i \rangle = k_B T$ per quadratic degree of freedom. For $N$ oscillators (each with kinetic and potential energy):
\begin{equation}
\langle E_i \rangle = k_B T
\end{equation}

Substituting into the entropy formula:
\begin{equation}
S_{\text{osc}} = k_B N \ln\left(\frac{k_B T}{E_0}\right)
\end{equation}

This recovers the classical ideal gas entropy structure (the additive constant depends on $E_0$ and other system-specific parameters).

\subsection{Quantum Oscillatory Entropy}

For quantum oscillators, phase space is discretised in units of $h$. An oscillator with occupation number $n_i$ has accessible phase space $\Gamma_i = (n_i + 1)h$ (accounting for zero-point energy). The number of accessible states is:
\begin{equation}
W_i = n_i + 1
\end{equation}

The quantum oscillatory entropy is:
\begin{equation}
S_{\text{osc,quantum}} = k_B \sum_{i=1}^{N} \ln(n_i + 1)
\end{equation}

\textbf{Limiting behavior:}
\begin{itemize}
\item \textbf{High temperature}: $n_i = k_B T/(\hbar\omega_i) \gg 1$, so $\ln(n_i + 1) \approx \ln n_i$, recovering the classical limit
\item \textbf{Low temperature}: $n_i \to 0$, so $S \to 0$, satisfying the third law of thermodynamics
\end{itemize}

This demonstrates that the oscillatory formulation naturally interpolates between quantum and classical regimes.

\subsection{The Bose-Einstein Distribution}

For a system of quantum oscillators in thermal equilibrium, maximising entropy subject to a fixed total energy yields the Bose-Einstein distribution.

\textbf{Setup:} Maximise 
\begin{equation}
S = k_B \sum_{i=1}^{N} \ln(n_i + 1)
\end{equation}
subject to the constraint
\begin{equation}
U = \sum_{i=1}^{N} \hbar\omega_i n_i = \text{constant}
\end{equation}

Using the method of Lagrange multipliers, we extremize:
\begin{equation}
\mathcal{L} = k_B \sum_i \ln(n_i + 1) - \beta \sum_i \hbar\omega_i n_i
\end{equation}

Taking the derivative with respect to $n_i$:
\begin{equation}
\frac{\partial \mathcal{L}}{\partial n_i} = \frac{k_B}{n_i + 1} - \beta \hbar\omega_i = 0
\end{equation}

Solving for $n_i$:
\begin{equation}
n_i + 1 = \frac{k_B}{\beta \hbar\omega_i} \quad \Rightarrow \quad n_i = \frac{k_B}{\beta \hbar\omega_i} - 1
\end{equation}

Identifying $\beta = 1/(k_B T)$:
\begin{equation}
\boxed{\langle n_i \rangle = \frac{1}{e^{\hbar\omega_i/(k_B T)} - 1}}
\label{eq:bose_einstein}
\end{equation}

This is precisely the Bose-Einstein distribution. The oscillatory entropy formulation naturally yields the correct quantum statistical distribution without additional postulates.

\subsection{Oscillatory Temperature}

Temperature emerges from the thermodynamic relation:
\begin{equation}
\frac{1}{T} = \left(\frac{\partial S}{\partial U}\right)_{V,N}
\end{equation}

For the oscillatory entropy $S_{\text{osc}} = k_B \sum_i \ln(A_i/A_0)^2 = k_B \sum_i \ln(E_i/E_0)$:
\begin{equation}
\frac{\partial S}{\partial U} = k_B \sum_i \frac{\partial}{\partial U}\ln E_i = k_B \sum_i \frac{1}{E_i}\frac{\partial E_i}{\partial U}
\end{equation}

For $N$ independent oscillators sharing total energy $U$, if energy is distributed uniformly, $E_i = U/N$:
\begin{equation}
\frac{\partial E_i}{\partial U} = \frac{1}{N}
\end{equation}

Thus:
\begin{equation}
\frac{1}{T} = k_B \sum_i \frac{1}{E_i} \cdot \frac{1}{N} = k_B \sum_i \frac{N}{U} \cdot \frac{1}{N} = \frac{k_B N}{U}
\end{equation}

Solving for $U$:
\begin{equation}
\boxed{U = N k_B T}
\label{eq:oscillatory_internal_energy}
\end{equation}

This is the equipartition result for $N$ oscillators with one degree of freedom each. For oscillators with $f$ quadratic degrees of freedom (e.g., $f=2$ for kinetic + potential energy):
\begin{equation}
U = \frac{f N k_B T}{2}
\end{equation}

The oscillatory perspective thus recovers classical equipartition as a consequence of the entropy-temperature relationship.

\subsection{Equivalence with Categorical Entropy}

The oscillatory entropy relates to the categorical entropy through the correspondence between phase space volumes and categorical dimensions.

\textbf{Key identifications:}
\begin{enumerate}
\item Each oscillator $i$ corresponds to one categorical dimension: $M = N$
\item The phase space ratio $\Gamma_i/\Gamma_0 = A_i^2/A_0^2$ equals the number of accessible states in that dimension: $n_i$
\item Thus $\ln(\Gamma_i/\Gamma_0) = \ln n_i$
\end{enumerate}

The oscillatory entropy becomes expressed as follows:
\begin{equation}
S_{\text{osc}} = k_B \sum_{i=1}^{N} \ln n_i
\end{equation}

For a system with uniform state distribution ($n_i = n$ for all $i$):
\begin{equation}
S_{\text{osc}} = k_B N \ln n = k_B M \ln n = S_{\text{cat}}
\end{equation}

For non-uniform distributions:
\begin{equation}
S_{\text{osc}} = k_B \sum_{i=1}^{M} \ln n_i = k_B M \langle \ln n \rangle
\end{equation}
where $\langle \ln n \rangle = (1/M)\sum_i \ln n_i$ is the geometric mean of the state counts.

\textbf{The oscillatory and categorical entropies are mathematically equivalent.} The oscillatory form explicitly demonstrates the contribution of each mode, while the categorical form emphasises the total dimensionality. They are two representations of the same underlying structure.

\begin{figure*}[htbp]
\centering
\includegraphics[width=\textwidth]{figures/panel2_entropy_derivation.png}
\caption{\textbf{Three Derivations of the Entropy Formula $S = k_B M \ln n$.} 
(\textbf{A}) Oscillatory derivation: For $M = 3$ oscillator modes with $n = 4$ quantum states each, the total number of microstates is $W_{\text{osc}} = 4^3 = 64$. 
(\textbf{B}) Categorical derivation: For $M = 2$ categorical dimensions with $n = 4$ distinguishable states each, the total number of configurations is $|C| = 4 \times 4 = 16$. 
(\textbf{C}) Partition derivation: A tree with $M = 2$ levels and branching factor $n = 3$ has $3^2 = 9$ terminal paths (leaves). One path is highlighted in red. 
(\textbf{D}) Boltzmann's fundamental relation $S = k_B \ln W$ combined with $W = n^M$ yields $S = k_B M \ln n$. 
(\textbf{E}) All three perspectives—oscillators, categorical states, and partition paths—yield the same formula $W = n^M$ and thus $S = k_B M \ln n$. 
(\textbf{F}) Entropy scaling as a function of degrees of freedom $M$ and states per degree of freedom $n$. The contour plot shows $S/k_B$ in the $(M, n)$ plane. The pendulum example (red point) has $M = 1$ mode and $n = 4$ states, giving $S = k_B \ln 4$. The entropy increases linearly with $M$ (horizontal direction) and logarithmically with $n$ (vertical direction).}
\label{fig:entropy_derivation}
\end{figure*}

\subsection{Summary}

The oscillatory perspective yields entropy as:
\begin{equation}
S_{\text{osc}} = k_B \sum_{i=1}^{N} \ln\left(\frac{A_i}{A_0}\right)^2 = k_B \sum_{i=1}^{N} \ln\left(\frac{\Gamma_i}{\Gamma_0}\right)
\end{equation}

Key features:
\begin{enumerate}
\item \textbf{Phase space foundation}: Derives from classical phase space volumes of oscillators
\item \textbf{Quantum correspondence}: Naturally incorporates quantum mechanics through discretization $\Gamma = nh$
\item \textbf{Statistical distributions}: Yields the Bose-Einstein distribution through entropy maximisation.
\item \textbf{Equipartition}: Recovers classical equipartition $U = Nk_BT$ through the definition of temperature.
\item \textbf{Third law compliance}: The quantum version satisfies $S \to 0$ as $T \to 0$
\item \textbf{Equivalence}: Mathematically identical to categorical entropy: $S_{\text{osc}} = S_{\text{cat}}$
\end{enumerate}

The partition perspective (Section~\ref{sec:partition}) will complete the triple equivalence by deriving entropy from the temporal segmentation of the oscillation period.
