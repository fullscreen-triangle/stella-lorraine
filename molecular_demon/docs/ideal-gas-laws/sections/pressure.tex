\section{Pressure: Categorical Density}
\label{sec:pressure}

\subsection{Classical Pressure and Its Limitations}

Classical kinetic theory derives pressure from momentum transfer at the container walls:
\begin{equation}
P_{\text{classical}} = \frac{1}{3}\rho\langle v^2 \rangle = \frac{Nk_B T}{V}
\end{equation}

While this derivation is successful for ideal gases, it faces conceptual challenges:

\textbf{Challenge 1: Boundary localisation paradox.} The standard derivation assumes that pressure arises from wall collisions. Yet we measure pressure in the bulk of fluids—deep ocean pressure, atmospheric pressure at altitude, and pressure inside stars where no walls exist. How can a boundary phenomenon determine a bulk property? Why does pressure exist far from any boundary?

\textbf{Challenge 2: Circular relationship with temperature.} Since temperature is defined through $T \propto \langle v^2 \rangle$ in kinetic theory, the relation $P \propto T$ becomes $P \propto \langle v^2 \rangle$—which is essentially a restatement rather than an independent physical principle. The connexion between pressure and temperature appears tautological.

\textbf{Challenge 3: Unexplained extensivity.} Why does pressure scale as $N/V$? What physical mechanism makes it an intensive variable (independent of system size)? The classical derivation provides the result but not the underlying reason.

The triple equivalence framework resolves these challenges by defining pressure as categorical density—an intrinsic bulk property that exists throughout the system.

\begin{figure}[htbp]
\centering
\includegraphics[width=\textwidth]{figures/panel_vap_results.png}
\caption{\textbf{Virtual Aperture Potentiometer (VAP) results showing aperture potentials and selectivities.} 
\textbf{(Top left)} Aperture potentials by material showing distribution of $\Phi_a/k_B T$ for different aperture types. Copper (orange bars) has moderate aperture potentials ($\Phi/k_B T \sim 1.2$) from phonon, impurity, and boundary scattering. Silicon (green bars) has higher potentials ($\Phi/k_B T \sim 1.6$) due to larger band gap and stronger scattering. YBCO below $T_c$ (cyan bars) has very low potentials, approaching zero as Cooper pairs bypass apertures.
\textbf{(Top right)} Selectivity spectrum showing aperture selectivity $s_a = \Omega_{\text{pass}}/\Omega_{\text{total}}$ for copper (orange/green stacked bars) and YBCO below $T_c$ (cyan bar). Copper has moderate selectivity ($s \sim 0.1$--$1$) with contributions from phonon (orange) and impurity (green) apertures. YBCO has very high selectivity ($s \sim 3$, effectively unity) as Cooper pairs pass through all apertures without scattering.
\textbf{(Bottom left)} Categorical potential vs. selectivity showing universal relationship $\Phi/k_B T = -\ln s$ (white line). Data points for copper (orange), silicon (green), and YBCO below $T_c$ (cyan) all fall on this line, confirming the categorical interpretation of aperture potentials. High selectivity ($s \to 1$) gives low potential ($\Phi \to 0$). Low selectivity ($s \ll 1$) gives high potential ($\Phi \gg k_B T$).
\textbf{(Bottom right)} Total aperture potential (transport coefficient) showing sum $\sum_a \Phi_a/k_B T$ for different materials. YBCO below $T_c$ (green) has zero total potential, corresponding to zero resistivity (superconductor). Silicon (green) has moderate total potential $\sum \Phi_a/k_B T \sim 3.17$. Copper (orange) has low total potential $\sum \Phi_a/k_B T \sim 4.02$, corresponding to low resistivity. The total aperture potential is proportional to the transport coefficient: $\rho \propto \sum_a \Phi_a$.}
\label{fig:vap_results}
\end{figure}

\subsection{Categorical Pressure}

Pressure measures the density of categorical distinctions—how many categories are packed into a given volume.

\begin{definition}
The \textit{categorical pressure} is:
\begin{equation}
\boxed{P_{\text{cat}} = k_B T \left(\frac{\partial M}{\partial V}\right)_{T,N}}
\label{eq:categorical_pressure}
\end{equation}
where $M$ is the number of accessible categorical dimensions and $V$ is the volume.
\end{definition}

\textbf{Physical interpretation:} Compressing a gas (decreasing $V$) forces the same number of categories into a smaller volume, increasing the categorical density $\partial M/\partial V$. The system resists this compression because categories require space to be distinguishable—this resistance is pressure.

\subsubsection{Bulk Property, Not Boundary Effect}

The categorical density $\partial M/\partial V$ exists throughout the volume, not just at the boundaries. Wall collisions are one \textit{manifestation} of categorical density, not its \textit{definition}.

Consider a point in the bulk of a gas. The local categorical density at that point is:
\begin{equation}
\rho_M(\mathbf{r}) = \frac{dM}{dV}\bigg|_{\mathbf{r}}
\end{equation}

This is the number of categorical distinctions per unit volume at position $\mathbf{r}$. The local pressure is:
\begin{equation}
P(\mathbf{r}) = k_B T \cdot \rho_M(\mathbf{r})
\end{equation}

For a uniform gas, $\rho_M$ is constant throughout the volume, resulting in uniform pressure. For non-uniform systems (e.g., atmosphere with altitude), $\rho_M(\mathbf{r})$ varies spatially, producing pressure gradients.

\textbf{Example:} In the ocean at depth $h$, the categorical density increases due to gravitational compression. The pressure $P(h) = k_B T \cdot \rho_M(h)$ reflects the local categorical density, not collisions with any distant surface.

\subsubsection{Derivation from Thermodynamic Relations}

From the fundamental thermodynamic relation:
\begin{equation}
P = -\left(\frac{\partial F}{\partial V}\right)_{T,N} = -\frac{\partial}{\partial V}(U - TS)
\end{equation}

For an ideal gas, where internal energy $U$ is volume-independent:
\begin{equation}
P = T\left(\frac{\partial S}{\partial V}\right)_{T,N}
\end{equation}

Using the categorical entropy $S = k_B M \ln n$:
\begin{equation}
P = T \cdot k_B \left[\ln n \left(\frac{\partial M}{\partial V}\right)_T + M \left(\frac{\partial \ln n}{\partial V}\right)_T\right]
\end{equation}

For the natural choice $\ln n = 1$ (one nat per category) and assuming $n$ is volume-independent:
\begin{equation}
P = k_B T \left(\frac{\partial M}{\partial V}\right)_T
\end{equation}

This establishes pressure as categorical density times temperature.

\subsubsection{Scaling of Categories with Volume}

For an ideal gas, the number of accessible spatial categories scales with volume. Each particle can occupy positions within the volume, with resolution set by the thermal de Broglie wavelength $\lambda_{\text{th}} = h/\sqrt{2\pi m k_B T}$.

The number of distinguishable spatial positions per particle is:
\begin{equation}
m_{\text{spatial}} \propto \frac{V}{\lambda_{\text{th}}^3}
\end{equation}

For $N$ particles:
\begin{equation}
M_{\text{total}} = N \cdot m_{\text{spatial}} \propto N \frac{V}{\lambda_{\text{th}}^3}
\end{equation}

However, the logarithmic nature of entropy means that:
\begin{equation}
S = k_B N \ln\left(\frac{V}{\lambda_{\text{th}}^3}\right) = k_B N \ln V + \text{const}
\end{equation}

This gives:
\begin{equation}
\frac{\partial S}{\partial V} = \frac{k_B N}{V}
\end{equation}

Therefore:
\begin{equation}
P = T \frac{\partial S}{\partial V} = \frac{k_B T N}{V}
\end{equation}

Comparing with $P = k_B T (\partial M/\partial V)$:
\begin{equation}
\frac{\partial M}{\partial V} = \frac{N}{V}
\end{equation}

This is the categorical density for an ideal gas, yielding the ideal gas law:
\begin{equation}
P = \frac{Nk_B T}{V}
\end{equation}

\subsection{Oscillatory Pressure}

In the oscillatory perspective, pressure arises from the spatial extent of oscillations. Particles execute oscillations with amplitudes $\{A_i\}$; these amplitudes exert forces.

\begin{definition}
The \textit{oscillatory pressure} is:
\begin{equation}
\boxed{P_{\text{osc}} = \frac{1}{3V}\sum_{i=1}^{N} m_i \omega_i^2 A_i^2}
\label{eq:oscillatory_pressure}
\end{equation}
\end{definition}

\textbf{Physical interpretation:} Each oscillator exerts an average force $\langle F \rangle = m\omega^2 A^2/L$ where $L$ is a characteristic length. Summing over all oscillators and dividing by the volume gives pressure. The factor of $1/3$ accounts for three spatial dimensions.

\subsubsection{Derivation from Virial Theorem}

For a system of particles in a container, the virial theorem relates pressure to kinetic energy:
\begin{equation}
PV = \frac{2}{3}\langle E_k \rangle_{\text{total}} = \frac{1}{3}\sum_{i=1}^{N} m_i \langle v_i^2 \rangle
\end{equation}

For harmonic oscillators, the time-averaged velocity squared is:
\begin{equation}
\langle v^2 \rangle = \omega^2 A^2
\end{equation}

Substituting:
\begin{equation}
PV = \frac{1}{3}\sum_{i=1}^{N} m_i \omega_i^2 A_i^2
\end{equation}

Dividing by $V$:
\begin{equation}
P = \frac{1}{3V}\sum_{i=1}^{N} m_i \omega_i^2 A_i^2
\end{equation}

\subsubsection{Connection to Thermal Energy}

For oscillators in thermal equilibrium, equipartition gives:
\begin{equation}
\frac{1}{2}m\omega^2 A^2 = \frac{1}{2}k_B T
\end{equation}

per degree of freedom. Thus:
\begin{equation}
m\omega^2 A^2 = k_B T
\end{equation}

Summing over $N$ particles, each with three translational degrees of freedom:
\begin{equation}
\sum_{i=1}^{N} m_i \omega_i^2 A_i^2 = 3N k_B T
\end{equation}

Substituting into the pressure formula:
\begin{equation}
P = \frac{1}{3V} \cdot 3N k_B T = \frac{N k_B T}{V}
\end{equation}

This recovers the ideal gas law from the oscillatory perspective.

\subsection{Partition Pressure}

In the partition perspective, pressure arises from the rate of boundary-crossing transitions.

\begin{definition}
The \textit{partition pressure} is:
\begin{equation}
\boxed{P_{\text{part}} = \frac{k_B T}{V} \sum_{a \in \text{boundary}} \frac{1}{\tau_{p,a}}}
\label{eq:partition_pressure}
\end{equation}
where the sum is over boundary partitions and $\tau_{p,a}$ is the partition lag for boundary crossing $a$.
\end{definition}

\textbf{Physical interpretation:} Faster boundary crossings (shorter partition lags) mean higher pressure. Each crossing transfers momentum; the rate of crossings determines the momentum flux, which is pressure.

\subsubsection{Derivation from Kinetic Theory}

The rate at which particles cross a boundary element $dA$ is given by kinetic theory:
\begin{equation}
\Phi = \frac{n \langle v \rangle}{4}
\end{equation}

where $n = N/V$ is the number density and $\langle v \rangle = \sqrt{8k_B T/\pi m}$ is the mean speed.

Each crossing is a partition event. The total number of boundary crossings per unit time is:
\begin{equation}
\sum_a \frac{1}{\tau_{p,a}} = \Phi \cdot A_{\text{total}} = \frac{N \langle v \rangle}{4V} \cdot A_{\text{total}}
\end{equation}

For a cubic container with side length $L$, $V = L^3$ and $A_{\text{total}} = 6L^2$:
\begin{equation}
\sum_a \frac{1}{\tau_{p,a}} = \frac{N \langle v \rangle}{4L^3} \cdot 6L^2 = \frac{3N \langle v \rangle}{2L}
\end{equation}

The pressure is the momentum flux. Each particle carries momentum $m\langle v \rangle$, giving:
\begin{equation}
P = \frac{k_B T}{V} \cdot \frac{3N \langle v \rangle}{2L} \cdot \frac{2L}{3\langle v \rangle} = \frac{Nk_B T}{V}
\end{equation}

(The geometric factors cancel to give the ideal gas result.)

\subsubsection{Alternative Form}

For an ideal gas, the sum over boundary partition rates equals:
\begin{equation}
\sum_a \frac{1}{\tau_{p,a}} = \frac{N}{\langle\tau_p\rangle_{\text{boundary}}}
\end{equation}

where $\langle\tau_p\rangle_{\text{boundary}}$ is the average time between boundary collisions for a single particle.

Thus:
\begin{equation}
P = \frac{k_B T N}{V \langle\tau_p\rangle_{\text{boundary}}} \cdot \langle\tau_p\rangle_{\text{boundary}} = \frac{Nk_B T}{V}
\end{equation}

\subsection{Equivalence of Three Definitions}

\begin{theorem}[Pressure Equivalence]
\label{thm:pressure_equivalence}
The three pressure definitions are equivalent for ideal gases:
\begin{equation}
P_{\text{cat}} = P_{\text{osc}} = P_{\text{part}} = \frac{Nk_B T}{V}
\end{equation}
\end{theorem}

\begin{proof}
We have shown:

\textit{Categorical:}
\begin{equation}
P_{\text{cat}} = k_B T \left(\frac{\partial M}{\partial V}\right)_T = k_B T \cdot \frac{N}{V}
\end{equation}

\textit{Oscillatory:}
\begin{equation}
P_{\text{osc}} = \frac{1}{3V}\sum_i m_i \omega_i^2 A_i^2 = \frac{3Nk_B T}{3V} = \frac{N k_B T}{V}
\end{equation}

\textit{Partition:}
\begin{equation}
P_{\text{part}} = \frac{k_B T}{V} \sum_a \frac{1}{\tau_{p,a}} = \frac{k_B T \cdot N}{V} = \frac{N k_B T}{V}
\end{equation}

All three yield the ideal gas law.
\end{proof}

\begin{figure}[htbp]
\centering
\includegraphics[width=\textwidth]{figures/fig_pressure_perspectives.png}
\caption{\textbf{Pressure: Triple Equivalence Perspectives.} 
\textbf{(A) Categorical versus classical pressure:} Pressure $P$ (pascals, logarithmic scale 10$^{-9}$ to 10$^{12}$ Pa) versus density $\rho$ (particles/m$^3$, logarithmic scale 10$^{10}$ to 10$^{31}$). Black dashed line: classical ideal gas law $P = \rho k_B T$ (linear on log-log plot). Green solid line: categorical prediction with saturation. Red annotation ``$P_{\text{sat}}$'' at $\rho \sim 10^{29}$ particles/m$^3$ marks onset of pressure saturation where categorical density reaches maximum. Classical prediction continues linearly (unphysical), while categorical prediction saturates at $P_{\text{sat}} \sim 10^9$ Pa.
\textbf{(B) Oscillatory pressure:} Pressure $P$ (pascals, logarithmic scale 10$^{-9}$ to 10$^{12}$ Pa) versus density $\rho$ (particles/m$^3$, logarithmic scale 10$^{10}$ to 10$^{31}$). Blue solid line: oscillatory prediction $P = \frac{1}{3}\rho m \omega^2 A^2$. Gray dashed line: classical reference. Inset diagram (top): blue irregular closed curve represents phase space trajectory with amplitude $A$, black dot at center, red dot on trajectory, arrow labeled ``$A\omega^2$'' showing acceleration. Text annotation: ``Amplitude creates pressure.'' Oscillatory perspective relates pressure to squared amplitude of molecular oscillations.
\textbf{(C) Partition pressure:} Pressure $P$ (pascals, logarithmic scale 10$^{-9}$ to 10$^{12}$ Pa) versus density $\rho$ (particles/m$^3$, logarithmic scale 10$^{10}$ to 10$^{31}$). Red solid line: partition prediction (boundary rate). Gray dashed line: classical reference. Inset graph shows boundary versus bulk ratio: horizontal axis labeled ``Boundary/Bulk,'' vertical axis shows pressure (0-10000 Pa). Two traces: red dashed (ideal), black solid (real). Real trace shows saturation at high density while ideal continues linearly. Partition perspective interprets pressure as rate of boundary encounters.
\textbf{(D) Pressure saturation at high density:} Compressibility factor $Z = P/(\rho k_B T)$ versus density $\rho$ (particles/m$^3$, logarithmic scale 10$^{25}$ to 10$^{32}$). Black dashed line: classical ideal gas ($Z = 1$, horizontal). Green solid line: categorical prediction showing saturation. Purple dotted line: Van der Waals prediction showing unphysical divergence. Green shaded region: saturation regime where $Z$ decreases from 1.0 to near 0 as density increases from 10$^{29}$ to 10$^{31}$ particles/m$^3$. Van der Waals diverges to $Z > 1.5$ (unphysical), while categorical saturates at $Z \to 0$ (all categories occupied, pressure cannot increase further).}
\label{fig:pressure_perspectives}
\end{figure}

\subsection{Pressure as Intensive Variable}

The categorical perspective explains why pressure is intensive. The categorical density:
\begin{equation}
\rho_M = \frac{\partial M}{\partial V} = \frac{N}{V}
\end{equation}

is intensive: doubling both $N$ and $V$ leaves $\rho_M$ unchanged. Since $P = k_B T \cdot \rho_M$, pressure inherits this intensivity.

\textbf{Physical reason:} Categories are local distinctions. The density of local distinctions depends only on the local concentration of particles, not on the total system size. This is why pressure is the same in a small sample and a large sample of the same gas at the same density and temperature.

\subsection{Pressure Saturation at High Density}

At extremely high densities, all available categories become occupied:
\begin{equation}
M \to M_{\max}
\end{equation}

The categorical density approaches a maximum:
\begin{equation}
\frac{\partial M}{\partial V} \to 0 \quad \text{as} \quad M \to M_{\max}
\end{equation}

This predicts \textit{pressure saturation} at extreme densities. The pressure cannot increase indefinitely because there are no more categories to compress— all distinguishable states are already occupied.

\textbf{Physical examples:}
\begin{itemize}
\item \textbf{Neutron stars}: At nuclear densities ($\rho \sim 10^{17}$ kg/m$^3$), the Pauli exclusion principle limits the available quantum states, causing pressure saturation.
\item \textbf{White dwarfs}: Electron degeneracy pressure saturates when all low-energy states are filled.
\item \textbf{Quantum liquids}: Liquid helium exhibits pressure saturation due to quantum effects at low temperatures.
\end{itemize}

The categorical framework predicts this behaviour naturally, without invoking quantum statistics as an additional postulate—it emerges from the finite number of distinguishable categories.

\subsection{Summary}

Pressure admits three equivalent definitions:
\begin{align}
P_{\text{cat}} &= k_B T \left(\frac{\partial M}{\partial V}\right)_T \quad \text{(categorical density)} \\
P_{\text{osc}} &= \frac{1}{3V}\sum_i m_i \omega_i^2 A_i^2 \quad \text{(oscillation amplitude)} \\
P_{\text{part}} &= \frac{k_B T}{V} \sum_a \frac{1}{\tau_{p,a}} \quad \text{(boundary partition rate)}
\end{align}

All three:
\begin{enumerate}
\item \textbf{Bulk properties}: Exist throughout the volume, not just at boundaries
\item \textbf{Explain extensivity}: $P \propto N/V$ follows from categorical density being intensive
\item \textbf{Ideal gas law}: All reduce to $P = Nk_BT/V$ for ideal gases
\item \textbf{Predict saturation}: Finite categorical structure implies pressure saturation at extreme density
\item \textbf{Local interpretation}: Pressure is a local property determined by local categorical density
\end{enumerate}

The categorical perspective resolves the boundary localization paradox: pressure is fundamentally a bulk property (categorical density) that happens to manifest at boundaries through momentum transfer. Wall collisions are the \textit{measurement} of pressure, not its \textit{cause}.
