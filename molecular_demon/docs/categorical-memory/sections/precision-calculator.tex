
\subsection{Definition of Precision-by-Difference}

The precision-by-difference quantity provides the mechanism for navigating the S-entropy hierarchy. It captures the discrepancy between expected and actual timing, transforming what is conventionally regarded as measurement error into navigational information. This reconceptualization is central to the categorical memory architecture: timing ``error'' becomes addressing information.

\begin{definition}[Precision-by-Difference]
\label{def:precision-diff}
Let $\Tref(k)$ be a reference time measurement at step $k$ and let $t_i(k)$ be the local time measurement at node $i$. The precision-by-difference value is:
\begin{equation}
\deltaP_i(k) = \Tref(k) - t_i(k)
\end{equation}
\end{definition}

The sign and magnitude of $\deltaP_i(k)$ encode information, as visualized in Figure~\ref{fig:precision_by_difference}(A):
\begin{itemize}
    \item $\deltaP_i(k) > 0$: Local clock runs slow relative to reference (temporal lag) $\to$ Branch 0.
    \item $|\deltaP_i(k)| \approx 0$: Local clock synchronized with reference $\to$ Branch 1.
    \item $\deltaP_i(k) < 0$: Local clock runs fast relative to reference (temporal advance) $\to$ Branch 2.
\end{itemize}
Figure~\ref{fig:precision_by_difference}(B) shows the resulting branch distribution: 31.3\% / 38.1\% / 30.6\%, demonstrating balanced ternary routing with slight preference for the synchronized branch.

\subsection{Trajectory Construction}

The sequence of precision-by-difference values over time forms a trajectory through S-entropy space.

\begin{definition}[Precision Trajectory]
A precision trajectory is the ordered sequence of precision-by-difference values:
\begin{equation}
\mathcal{T} = \{\deltaP_i(0), \deltaP_i(1), \ldots, \deltaP_i(K)\}
\end{equation}
where $K$ is the trajectory length.
\end{definition}

\begin{proposition}[Trajectory-to-Address Mapping]
The precision trajectory $\mathcal{T}$ maps to a unique address through the trajectory hash:
\begin{equation}
\text{addr}(\mathcal{T}) = \mathcal{H}\left(\bigoplus_{k=0}^{K} \deltaP_i(k)\right)
\end{equation}
where $\mathcal{H}$ is a cryptographic hash function and $\bigoplus$ denotes concatenation.
\end{proposition}

The hash provides a fixed-length identifier for trajectories of arbitrary length. Two trajectories with identical precision-by-difference sequences will have identical addresses; trajectories that differ in any component will (with high probability) have different addresses.

\subsection{Precision Window}

The distribution of precision-by-difference values defines a temporal coherence window.

\begin{definition}[Precision Window]
Given a set of precision-by-difference values $\{\deltaP_j(k)\}_{j \in \mathcal{N}}$ from nodes in neighborhood $\mathcal{N}$, the precision window is the interval:
\begin{equation}
W(k) = \left[\Tref(k) + \min_j \deltaP_j(k), \; \Tref(k) + \max_j \deltaP_j(k)\right]
\end{equation}
\end{definition}

The width of the precision window characterizes the timing spread across nodes:
\begin{equation}
|W(k)| = \max_j \deltaP_j(k) - \min_j \deltaP_j(k)
\end{equation}

Narrow windows indicate tight synchronisation; wide windows indicate temporal dispersion.

\begin{figure}[htbp]
    \centering
    \includegraphics[width=\textwidth]{figures/precision_by_difference_panel.png}
    \caption{
        \textbf{Precision-by-difference network achieves 96.1\% latency reduction (69.7 ms $\to$ 2.8 ms) via S-entropy temporal coordination.} 
        \textbf{(A)} Precision-by-difference distribution (histogram, 60 bins) shows $\Delta P = T_{\text{ref}} - t_{\text{local}}$ over 100 measurements. Distribution approximately Gaussian: $\mu = -0.02$ µs, $\sigma = 0.98$ µs. Red dashed line: reference ($\Delta P = 0$). Bimodal structure visible with peaks at $-1$ µs and $+1$ µs. Validates $\Delta P$ as stable information source for addressing.
        
        \textbf{(B)} Hierarchy branch selection (bar chart) shows routing based on $\Delta P$ sign. Branch 0 ($\Delta P > 0$): 31.3\% (green, 310 samples). Branch 1 ($\Delta P = 0$): 38.1\% (orange, 380 samples, highest). Branch 2 ($\Delta P < 0$): 30.6\% (red, 300 samples). Balanced distribution indicates unbiased routing, validating ternary branching structure.
        
        \textbf{(C)} Temporal coherence windows (dual time series, 50 windows) show width (blue line, left y-axis, 2.5--5.5 ms) and quality (purple line, right y-axis, 0.9984--0.9994). Width oscillates with period $\sim 10$ windows. Quality anticorrelated with width (high width $\to$ low quality). Mean quality: 0.9990 (99.90\% coherence). Validates temporal synchronization mechanism.
        
        \textbf{(D)} $3^k$ hierarchy navigation (2D scatter, $S_k$ vs. $S_e$) shows 4 access points colored by depth (3.0--5.0 scale). Yellow point (depth 5.0, $S_k = 0.4$, $S_e = 0.4$) represents deepest navigation. Purple point (depth 3.25, $S_k = 0.0$, $S_e = 0.15$) represents shallowest. Coverage: 22.0\% of hierarchy accessed. Validates selective navigation where only relevant nodes are visited.
        
        \textbf{(E)} Collective state coordination (bar chart + scatter, 30 rounds) shows window width (blue bars, 0--14 ms) and sync status (green dots = No, red dots = Yes). Sync rate: 0\% (all green dots below red dashed line). Window width varies 4--14 ms across rounds. Zero synchronization indicates independent oscillators, validating asynchronous coordination framework.
        
        \textbf{(F)} Categorical completion prediction (histogram, 6 bins) shows prediction error distribution. Mean error: 0.0078 (red dashed line). Peak at 0.006 (orange bar, 2.0 synchronized). Range: 0.002--0.010. Tight distribution ($\sigma \sim 0.002$) indicates high prediction accuracy. Validates completion mechanism where system predicts trajectory endpoints.
        
        \textbf{(G)} Network latency comparison (time series, 100 requests) shows Traditional (red, top, 69.7 ms mean) vs. Sango Rine Shumba (green, bottom, 2.8 ms mean). Traditional: high variance (60--80 ms), frequent spikes. Sango: low variance (0--20 ms), rare spikes. Improvement: 96.1\% latency reduction. Validates precision-by-difference routing eliminates traditional lookup overhead.
        
        \textbf{(H)} Framework performance (radar plot, 5 axes) shows normalized metrics. Latency Improvement: 1.0 (maximum, validates 96.1\% reduction). Window Quality: 0.8 (validates 99.90\% coherence). Prediction Accuracy: 0.6 (validates mean error 0.0078). Hierarchy Coverage: 0.2 (validates 22.0\% selective navigation). Sync Rate: 0.0 (validates asynchronous coordination). Teal shaded area represents achieved performance.
    }
    \label{fig:precision_by_difference}
\end{figure}

\subsection{Statistical Properties}

\begin{proposition}[Precision Statistics]
For a system with $N$ measurements, the precision-by-difference values satisfy:
\begin{align}
\bar{\deltaP} &= \frac{1}{N}\sum_{k=1}^{N} \deltaP(k) && \text{(Mean precision)} \\
\sigma_{\deltaP}^2 &= \frac{1}{N-1}\sum_{k=1}^{N} (\deltaP(k) - \bar{\deltaP})^2 && \text{(Precision variance)}
\end{align}
\end{proposition}

These statistics characterize the overall timing behavior:
\begin{itemize}
    \item Non-zero mean $\bar{\deltaP} \neq 0$ indicates systematic clock drift.
    \item Large variance $\sigma_{\deltaP}^2$ indicates high timing jitter.
\end{itemize}

\subsection{Conversion to S-Coordinates}

Precision-by-difference values convert to S-entropy coordinates through a defined mapping.

\begin{definition}[Precision-to-S Mapping]
Given a precision signature consisting of $n$ recent precision-by-difference values $\{\deltaP_1, \ldots, \deltaP_n\}$, the S-coordinates are:
\begin{align}
\Sk &= \sigma(\nabla \deltaP) && \text{(Standard deviation of differences)} \\
\St &= \bar{\deltaP} && \text{(Mean value)} \\
\Se &= H(\deltaP) && \text{(Histogram entropy)}
\end{align}
where $\nabla \deltaP = \{\deltaP_{k+1} - \deltaP_k\}$ and $H(\cdot)$ denotes histogram-based entropy estimation.
\end{definition}

This mapping has the following interpretation:
\begin{itemize}
    \item $\Sk$ (kinetic): The rate of change in precision values indicates dynamic behavior. Large $\sigma(\nabla \deltaP)$ means rapidly varying timing; small values indicate stable timing.
    
    \item $\St$ (thermal): The mean precision value indicates the central tendency of timing deviations. This positions the trajectory in the thermal dimension.
    
    \item $\Se$ (entropic): The entropy of the precision distribution indicates the disorder in timing patterns. High entropy means unpredictable timing; low entropy means regular patterns.
\end{itemize}

\subsection{Navigation via Precision}

The precision-by-difference value at each step determines the navigation direction in the $3^k$ hierarchy.

\begin{proposition}[Branch Selection]
At each level of the hierarchy, the branch index $b \in \{0, 1, 2\}$ is determined by:
\begin{equation}
b = \lfloor 3 \cdot |\deltaP(k) \cdot 10^9| \rfloor \mod 3
\end{equation}
where the factor $10^9$ converts from seconds to nanoseconds for numerical stability.
\end{proposition}

This deterministic mapping ensures that identical precision-by-difference sequences produce identical navigation paths, and hence identical addresses. The modulo-3 operation maps the continuous precision value to the discrete three-way branching structure.

\subsection{Trajectory Prediction}

Given a partial trajectory, the endpoint can be predicted through categorical completion.

\begin{theorem}[Trajectory Completion]
Let $\mathcal{T}_{\text{partial}} = \{\deltaP(0), \ldots, \deltaP(K)\}$ be a partial trajectory with $K$ steps. The predicted completion point $\Scoord^*$ is the asymptotic limit:
\begin{equation}
\Scoord^* = \lim_{k \to \infty} \Scoord(k) = \bar{\Scoord} + \frac{\langle \mathbf{v} \rangle}{1 - r}
\end{equation}
where $\bar{\Scoord}$ is the mean S-coordinate, $\langle \mathbf{v} \rangle$ is the mean velocity (rate of coordinate change), and $r$ is the velocity decay rate.
\end{theorem}

\begin{proof}
The trajectory in S-space follows dynamics $\Scoord(k+1) = \Scoord(k) + \mathbf{v}(k)$ where velocities decay geometrically: $\|\mathbf{v}(k+1)\| = r \|\mathbf{v}(k)\|$ for $0 < r < 1$. The infinite sum of velocities from the current position converges to $\sum_{j=0}^{\infty} r^j \langle \mathbf{v} \rangle = \langle \mathbf{v} \rangle / (1-r)$. Adding this to the current position gives the asymptotic limit.
\end{proof}

The predicted completion point represents where the trajectory is ``heading''---the categorical endpoint already encoded in the access pattern history.

