\section{Minimal Hardware Bounds}
\label{sec:hardware_bounds}

Having established the completeness of coupling structures in Section~\ref{sec:completeness}, we now address the question of \emph{physical implementation}: what are the minimal hardware requirements for realizing these coupling structures? The main results are fundamental bounds on: (i) the minimum number of physical oscillators required (Theorem~\ref{thm:minimal_hardware}), (ii) the minimum energy per measurement (Theorem~\ref{thm:min_energy}), and (iii) the minimum measurement time (Theorem~\ref{thm:min_time}). These bounds are not technological limitations but mathematical necessities arising from information theory, thermodynamics, and Fourier analysis. Remarkably, all bounds are achievable (Theorem~\ref{thm:bound_saturation}), establishing that the spectroscopic framework developed in previous sections is not only mathematically complete but also physically optimal.

\begin{figure}[htbp]
\centering
\includegraphics[width=\textwidth]{figures/comprehensive_validation.png}
\caption{Comprehensive validation of spectroscopic measurement framework against synthetic test data. \textbf{Top row:} Peak detection performance (mean F1 = 0.055), spectral correlation distribution (mean = 0.027), RMSE distribution (mean = 0.435), and LED wavelength response validation. \textbf{Middle row:} Four representative spectral comparisons between real (blue) and virtual (red dashed) measurements showing systematic discrepancies. \textbf{Bottom row:} Peak count comparison, correlation vs RMSE scatter plot, and overall performance metrics. The low correlation and high RMSE indicate that the virtual measurement model does not accurately reproduce real spectroscopic data, suggesting fundamental differences between the theoretical framework and physical implementation.}
\label{fig:comprehensive_validation}
\end{figure}

\subsection{Hardware Oscillator Model}

We begin by formalizing the notion of a physical oscillator as the fundamental building block of spectroscopic instrumentation.

\begin{definition}[Hardware Oscillator]
\label{def:hardware_oscillator}
A \emph{hardware oscillator} is a physical system characterized by three parameters:
\begin{enumerate}[label=(\roman*), noitemsep]
    \item \emph{Natural frequency} $\omega_{\text{hw}} \in \Reals^+$: the frequency at which the oscillator resonates,
    \item \emph{Quality factor} $Q = \omega_{\text{hw}} / \Gamma$: the ratio of oscillation frequency to linewidth, quantifying frequency selectivity (Definition~\ref{def:quality_factor}),
    \item \emph{Coupling strength} $g \in \Reals^+$: the rate at which the oscillator exchanges energy with external systems.
\end{enumerate}
The oscillator state space is $\oscillator_{\text{hw}} = \{(A, \phi) : A \in \Reals^+, \, \phi \in [0, 2\pi)\}$, where $A$ is the oscillation amplitude and $\phi$ is the phase. The oscillator energy is $E = \frac{1}{2}m\omega_{\text{hw}}^2 A^2$ (for mechanical oscillators) or $E = \hbar\omega_{\text{hw}}(n + \frac{1}{2})$ (for quantum oscillators with $n$ excitations).
\end{definition}

\begin{remark}
Examples of hardware oscillators include: LC circuits (radio frequencies), cavity resonators (microwave/optical), mechanical resonators (acoustic), atomic transitions (optical), nuclear spins (radio frequencies). The quality factor $Q$ ranges from $\sim 10$ (broadband antennas) to $\sim 10^{15}$ (atomic clocks).
\end{remark}

\begin{definition}[Oscillator Bank]
\label{def:oscillator_bank}
An \emph{oscillator bank} is a collection $\mathcal{B} = \{(\omega_i, Q_i, g_i)\}_{i=1}^N$ of $N$ hardware oscillators with distinct natural frequencies $\omega_1, \omega_2, \ldots, \omega_N$. The bank state is the product $\oscillator_{\mathcal{B}} = \prod_{i=1}^N \oscillator_{\text{hw},i}$.
\end{definition}

\begin{definition}[Frequency Coverage]
\label{def:frequency_coverage}
An oscillator bank $\mathcal{B}$ \emph{covers} frequency $\omega$ if there exists an oscillator $i \in \{1, \ldots, N\}$ such that:
\begin{equation}
|\omega - \omega_i| < \Gamma_i = \omega_i / Q_i,
\end{equation}
i.e., $\omega$ lies within the resonance linewidth of oscillator $i$. The \emph{coverage set} of $\mathcal{B}$ is:
\begin{equation}
\Omega(\mathcal{B}) = \bigcup_{i=1}^N [\omega_i - \Gamma_i, \omega_i + \Gamma_i] \subset \Reals^+.
\end{equation}
\end{definition}

\begin{remark}
The coverage set $\Omega(\mathcal{B})$ is the union of frequency intervals within which the bank can efficiently couple to external systems. Gaps in coverage correspond to frequencies that cannot be accessed without adding new oscillators.
\end{remark}

\subsection{Minimal Oscillator Count}

We now derive the minimum number of hardware oscillators required for complete partition coordinate extraction.

\begin{theorem}[Minimal Oscillator Count]
\label{thm:minimal_hardware}
For a partition $\partition$ with cardinality $|\partition|$, the minimum number of hardware oscillators required for complete coordinate extraction (unique identification of partition elements) is:
\begin{equation}
N_{\min} = \lceil \log_2 |\partition| \rceil,
\end{equation}
where $\lceil \cdot \rceil$ denotes the ceiling function (rounding up to the nearest integer).
\end{theorem}

\begin{proof}
\textbf{Lower bound ($N \geq \log_2 |\partition|$):}

Each hardware oscillator can be in one of two distinguishable states at any given time: \emph{excited} (oscillating with significant amplitude) or \emph{ground} (quiescent). This binary distinction arises from the resonance condition: an oscillator is excited if the external system frequency matches $\omega_i \pm \Gamma_i$, and remains in the ground state otherwise.

With $N$ oscillators, the total number of distinguishable configurations is at most $2^N$ (each oscillator contributes one bit). To uniquely identify $|\partition|$ distinct partition elements requires:
\begin{equation}
2^N \geq |\partition| \quad \Rightarrow \quad N \geq \log_2 |\partition|.
\end{equation}

Since $N$ must be an integer, $N \geq \lceil \log_2 |\partition| \rceil$.

\textbf{Upper bound (achievability):}

We construct an explicit encoding achieving $N = \lceil \log_2 |\partition| \rceil$. Label partition elements as $P_1, P_2, \ldots, P_{|\partition|}$. Represent each index $j \in \{1, \ldots, |\partition|\}$ as a binary string of length $N = \lceil \log_2 |\partition| \rceil$:
\begin{equation}
j = \sum_{i=1}^N b_i(j) \cdot 2^{i-1}, \quad b_i(j) \in \{0, 1\}.
\end{equation}

Assign to each oscillator $i$ a frequency $\omega_i$ such that:
\begin{itemize}[noitemsep]
    \item If $b_i(j) = 1$, then partition element $P_j$ has a spectral component at frequency $\omega_i$ (oscillator $i$ is excited),
    \item If $b_i(j) = 0$, then $P_j$ has no spectral component near $\omega_i$ (oscillator $i$ remains in ground state).
\end{itemize}

By choosing oscillator frequencies $\omega_i$ sufficiently separated (with $|\omega_i - \omega_j| \gg \Gamma_i + \Gamma_j$ for $i \neq j$), the oscillators respond independently. The binary string $(b_1, b_2, \ldots, b_N)$ uniquely identifies the partition element.

This construction achieves $N = \lceil \log_2 |\partition| \rceil$, matching the lower bound.
\end{proof}

\begin{corollary}[Coordinate-Specific Oscillator Requirements]
\label{cor:coord_oscillators}
For extraction of individual partition coordinates, the minimum number of oscillators required for each coordinate is:
\begin{align}
N_n &\geq \lceil \log_2 n_{\max} \rceil && \text{(depth)}, \\
N_\ell &\geq \lceil \log_2 n_{\max} \rceil && \text{(angular complexity)}, \\
N_m &\geq \lceil \log_2 (2\ell_{\max} + 1) \rceil && \text{(orientation)}, \\
N_s &\geq 1 && \text{(chirality)},
\end{align}
where $n_{\max}$ is the maximum depth and $\ell_{\max} = n_{\max} - 1$ is the maximum angular complexity (Theorem~\ref{thm:partition_structure}).
\end{corollary}

\begin{proof}
Apply Theorem~\ref{thm:minimal_hardware} to the range of each coordinate:
\begin{itemize}[noitemsep]
    \item Depth $n \in \{1, 2, \ldots, n_{\max}\}$: $|\{n\}| = n_{\max}$, hence $N_n \geq \lceil \log_2 n_{\max} \rceil$.
    \item Angular complexity $\ell \in \{0, 1, \ldots, n_{\max} - 1\}$: $|\{\ell\}| = n_{\max}$, hence $N_\ell \geq \lceil \log_2 n_{\max} \rceil$.
    \item Orientation $m \in \{-\ell_{\max}, \ldots, \ell_{\max}\}$: $|\{m\}| = 2\ell_{\max} + 1$, hence $N_m \geq \lceil \log_2(2\ell_{\max} + 1) \rceil$.
    \item Chirality $s \in \{-\tfrac{1}{2}, +\tfrac{1}{2}\}$: $|\{s\}| = 2$, hence $N_s \geq \lceil \log_2 2 \rceil = 1$.
\end{itemize}
\end{proof}

\begin{example}
For a partition with $n_{\max} = 10$ (typical atomic system):
\begin{align}
N_n &\geq \lceil \log_2 10 \rceil = 4 \quad \text{oscillators for depth}, \\
N_\ell &\geq \lceil \log_2 10 \rceil = 4 \quad \text{oscillators for complexity}, \\
N_m &\geq \lceil \log_2 19 \rceil = 5 \quad \text{oscillators for orientation}, \\
N_s &\geq 1 \quad \text{oscillator for chirality}.
\end{align}
Total: at least $4 + 4 + 5 + 1 = 14$ oscillators for complete coordinate extraction. This is far fewer than the partition cardinality $|\partition| = 770$ (Corollary~\ref{cor:partition_cardinality}), demonstrating the efficiency of coordinate-based encoding.
\end{example}

\subsection{Frequency Matching and Hardware-Coordinate Compatibility}

Not all oscillators are suitable for extracting all coordinates; frequency matching is essential.

\begin{theorem}[Frequency Matching Necessity]
\label{thm:frequency_matching}
For hardware oscillator $i$ with natural frequency $\omega_i$ and linewidth $\Gamma_i$ to extract coordinate $\xi$, the \emph{frequency matching condition}:
\begin{equation}
\omega_i \in \Omega_\xi,
\end{equation}
is necessary, where $\Omega_\xi$ is the spectral regime for coordinate $\xi$ (Definition~\ref{def:spectral_regime}).
\end{theorem}

\begin{proof}
By Theorem~\ref{thm:resonance_enhancement}, the coupling strength between the oscillator and the system is:
\begin{equation}
\mathcal{C}(\omega_{\text{sys}}, \omega_i) = \frac{\mathcal{C}_0}{1 + 4(\omega_{\text{sys}} - \omega_i)^2 / \Gamma_i^2}.
\end{equation}

For coordinate $\xi$, the system frequencies are concentrated in regime $\Omega_\xi$ (Theorem~\ref{thm:frequency_duality}). If $\omega_i \notin \Omega_\xi$, then the detuning satisfies $|\omega_{\text{sys}} - \omega_i| \geq \Delta_{\min}(\xi)$, where $\Delta_{\min}(\xi)$ is the minimum separation between $\Omega_\xi$ and other regimes (Proposition~\ref{prop:regime_separation}).

By Corollary~\ref{cor:off_resonance}, the coupling is suppressed by:
\begin{equation}
\frac{\mathcal{C}(\omega_{\text{sys}}, \omega_i)}{\mathcal{C}_0} \approx \left(\frac{\Gamma_i}{2\Delta_{\min}}\right)^2.
\end{equation}

For typical systems with $\Delta_{\min}/\Gamma_i \gtrsim 10^3$ (from Proposition~\ref{prop:regime_selectivity}), the suppression is:
\begin{equation}
\frac{\mathcal{C}}{\mathcal{C}_0} \lesssim \left(\frac{1}{2 \times 10^3}\right)^2 \approx 2.5 \times 10^{-7}.
\end{equation}

This seven-orders-of-magnitude suppression renders the oscillator ineffective for extracting coordinate $\xi$. Hence, frequency matching $\omega_i \in \Omega_\xi$ is necessary for efficient coupling.
\end{proof}

\begin{definition}[Hardware-Coordinate Compatibility]
\label{def:compatibility}
An oscillator bank $\mathcal{B}$ is \emph{compatible} with coordinate $\xi$ if its coverage set intersects the spectral regime:
\begin{equation}
\Omega(\mathcal{B}) \cap \Omega_\xi \neq \emptyset.
\end{equation}
The bank is \emph{complete} if it is compatible with all four coordinates:
\begin{equation}
\Omega(\mathcal{B}) \cap \Omega_n \neq \emptyset, \quad \Omega(\mathcal{B}) \cap \Omega_\ell \neq \emptyset, \quad \Omega(\mathcal{B}) \cap \Omega_m \neq \emptyset, \quad \Omega(\mathcal{B}) \cap \Omega_s \neq \emptyset.
\end{equation}
\end{definition}

\begin{proposition}[Minimal Complete Bank]
\label{prop:minimal_bank}
A minimal complete oscillator bank contains at least 4 oscillators, with at least one oscillator in each spectral regime $\Omega_n, \Omega_\ell, \Omega_m, \Omega_s$.
\end{proposition}

\begin{proof}
By Proposition~\ref{prop:regime_separation}, the spectral regimes are pairwise disjoint:
\begin{equation}
\Omega_\xi \cap \Omega_{\xi'} = \emptyset \quad \text{for } \xi \neq \xi'.
\end{equation}

A single oscillator with frequency $\omega_i$ and linewidth $\Gamma_i$ covers an interval $[\omega_i - \Gamma_i, \omega_i + \Gamma_i]$. Since this interval is connected, it can intersect at most one of the disjoint regimes $\{\Omega_n, \Omega_\ell, \Omega_m, \Omega_s\}$.

Hence, each oscillator is compatible with at most one coordinate. To achieve compatibility with all four coordinates requires at least 4 oscillators.
\end{proof}

\begin{corollary}[Fundamental Hardware Requirement]
\label{cor:fundamental_hardware}
Complete spectroscopic characterisation of a bounded measure-preserving system requires at least 4 physical oscillators operating at widely separated frequencies (spanning multiple decades in frequency space).
\end{corollary}

\subsection{Signal Processing and Virtual Instrumentation}

While hardware oscillators are fixed physical components, their outputs can be processed to implement diverse measurements.

\begin{definition}[Virtual Instrument]
\label{def:virtual_instrument}
A \emph{virtual instrument} is a coupling structure implemented through signal processing on fixed hardware. It is specified by the triple:
\begin{equation}
\mathcal{I}_{\text{virtual}} = (\mathcal{B}, \mathcal{F}, \mathcal{D}),
\end{equation}
where:
\begin{itemize}[noitemsep]
    \item $\mathcal{B}$ is an oscillator bank (fixed hardware),
    \item $\mathcal{F}: \oscillator_{\mathcal{B}} \to \Reals^k$ is a signal processing pipeline (filtering, mixing, Fourier transforms, etc.),
    \item $\mathcal{D}: \Reals^k \to \mathfrak{M}$ is a detection/readout scheme mapping processed signals to measurement outcomes.
\end{itemize}
\end{definition}

\begin{theorem}[Reconfigurability]
\label{thm:reconfigurability}
For a complete oscillator bank $\mathcal{B}$ (Definition~\ref{def:compatibility}), any measurement $\mathcal{M} \in \mathfrak{M}$ can be implemented as a virtual instrument through signal processing reconfiguration alone, without modifying the hardware oscillators.
\end{theorem}

\begin{proof}
By Theorem~\ref{thm:derived_construction}, any measurement $\mathcal{M} \in \mathfrak{M}$ is constructible from elementary measurements $\{\mathcal{M}_n, \mathcal{M}_\ell, \mathcal{M}_m, \mathcal{M}_s\}$ via:
\begin{enumerate}[label=(\roman*), noitemsep]
    \item Parallel composition (simultaneous extraction),
    \item Classical post-processing (arithmetic operations, functions).
\end{enumerate}

Since $\mathcal{B}$ is complete, it contains oscillators in all four spectral regimes, enabling extraction of all elementary measurements. Specifically:
\begin{itemize}[noitemsep]
    \item Oscillators in $\Omega_n$ extract $\mathcal{M}_n$ (depth),
    \item Oscillators in $\Omega_\ell$ extract $\mathcal{M}_\ell$ (complexity),
    \item Oscillators in $\Omega_m$ extract $\mathcal{M}_m$ (orientation),
    \item Oscillators in $\Omega_s$ extract $\mathcal{M}_s$ (chirality).
\end{itemize}

The signal processing pipeline $\mathcal{F}$ implements:
\begin{itemize}[noitemsep]
    \item \emph{Filtering}: isolate oscillator outputs corresponding to specific frequencies,
    \item \emph{Mixing}: combine signals from multiple oscillators (parallel composition),
    \item \emph{Fourier analysis}: extract frequency components for coordinate identification,
    \item \emph{Arithmetic operations}: compute functions of extracted coordinates (post-processing).
\end{itemize}

All these operations are performed on the oscillator outputs (voltages, currents, photon counts, etc.) without altering the oscillators themselves. Hence, $\mathcal{M}$ is implementable via reconfiguration of $\mathcal{F}$ and $\mathcal{D}$.
\end{proof}

\begin{corollary}[Software-Defined Spectroscopy]
\label{cor:software_defined}
A single complete oscillator bank can implement an unlimited variety of measurements through software-controlled signal processing, analogous to software-defined radio. The hardware is fixed; the measurement is defined by the processing algorithm.
\end{corollary}

\subsection{Thermodynamic and Temporal Bounds}

Beyond hardware count, fundamental physical limits constrain measurement energy and time.

\begin{theorem}[Minimum Measurement Energy]
\label{thm:min_energy}
The minimum energy required to extract one bit of information about a partition coordinate is:
\begin{equation}
E_{\min} = k_B T \ln 2 \approx 2.87 \times 10^{-21} \, \text{J} \quad \text{at } T = 300 \, \text{K},
\end{equation}
where $k_B = 1.38 \times 10^{-23}$ J/K is Boltzmann's constant and $T$ is the apparatus temperature.
\end{theorem}

\begin{proof}
This is \emph{Landauer's principle} \citep{Landauer1961}: any logically irreversible operation (such as erasing one bit of information) must dissipate at least $k_B T \ln 2$ of energy to the environment.

Measurement involves an irreversible step: the apparatus transitions from an initial state (independent of the system) to a final state correlated with the system state. This correlation requires "erasing" the apparatus's prior state (resetting it to a standard initial condition for the next measurement).

By the second law of thermodynamics, this erasure increases entropy by at least $\Delta S = k_B \ln 2$ per bit. The associated heat dissipation is:
\begin{equation}
Q = T \Delta S = k_B T \ln 2.
\end{equation}

Hence, $E_{\min} = k_B T \ln 2$ is a fundamental lower bound, independent of technology.
\end{proof}

\begin{remark}
At room temperature ($T = 300$ K), $E_{\min} \approx 3 \times 10^{-21}$ J per bit. For comparison, modern digital logic dissipates $\sim 10^{-15}$ J per operation, about $10^6$ times the Landauer limit. Spectroscopic measurements typically dissipate even more energy (photon energies $\hbar\omega \sim 10^{-19}$ J for visible light), but approach the Landauer limit for low-frequency measurements (NMR, ESR).
\end{remark}

\begin{figure}[htbp]
\centering
\includegraphics[width=\textwidth]{figures/panel_unified_spectroscopy.png}
\caption{Unified spectroscopic framework showing correspondence between partition coordinates $(n,\ell,m,s)$ and measurement techniques. \textbf{Top:} Frequency regime separation spanning radio to X-ray frequencies ($10^6$--$10^{18}$ Hz), with each coordinate occupying a distinct spectral regime separated by factors $>10^3$ (Theorem~\ref{thm:frequency_duality}). \textbf{Middle:} Geometric representations of each coordinate: depth $n$ (shell capacity $2n^2$), complexity $\ell$ (angular degeneracy), orientation $m$ (Zeeman levels and Larmor precession), and chirality $s$ (Bloch sphere relaxation). \textbf{Bottom table:} Summary of coordinate-instrument correspondences, showing frequency scaling ($\omega_n \propto n^{-3}$, $\omega_\ell \propto \ell(\ell+1)$, $\omega_m \propto m \cdot B$, $\omega_s \propto s \cdot B$), physical coupling mechanisms, and spectroscopic implementations. The coordinate relationship diagram (right) illustrates the hierarchical structure connecting all four measurements through the partition structure $\mathcal{P}$.}
\label{fig:unified_spectroscopy}
\end{figure}

\begin{theorem}[Minimum Measurement Time]
\label{thm:min_time}
The minimum time required to resolve a frequency $\omega$ with precision $\delta\omega$ is:
\begin{equation}
T_{\min} = \frac{2\pi}{\delta\omega}.
\end{equation}
This is the \emph{Fourier time-frequency uncertainty relation}.
\end{theorem}

\begin{proof}
A signal of finite duration $T$ has a Fourier transform with frequency width $\delta\omega \sim 1/T$. Precisely, for a signal $f(t)$ with support on $[0, T]$, the Fourier transform $\hat{f}(\omega)$ has width:
\begin{equation}
\Delta\omega = \frac{\sqrt{\langle \omega^2 \rangle - \langle \omega \rangle^2}}{\text{(normalization)}} \geq \frac{2\pi}{T},
\end{equation}
by the Heisenberg-type uncertainty relation for Fourier pairs (Proposition~\ref{prop:resolution_bandwidth}).

To resolve two frequencies separated by $\delta\omega$, we require $\Delta\omega < \delta\omega$, hence:
\begin{equation}
T > \frac{2\pi}{\delta\omega}.
\end{equation}

The minimum time saturates this bound: $T_{\min} = 2\pi / \delta\omega$.
\end{proof}

\begin{corollary}[Coordinate Resolution Time]
\label{cor:coord_time}
The minimum time required to distinguish adjacent values of partition coordinate $\xi$ is:
\begin{equation}
T_\xi = \frac{2\pi}{\Delta\omega_\xi},
\end{equation}
where $\Delta\omega_\xi$ is the frequency spacing between adjacent coordinate values (Theorem~\ref{thm:frequency_duality}).
\end{corollary}

\begin{proof}
Adjacent coordinate values $\xi$ and $\xi + 1$ correspond to frequencies $\omega_\xi$ and $\omega_{\xi+1}$ with separation $\Delta\omega_\xi = |\omega_{\xi+1} - \omega_\xi|$. To resolve them requires frequency precision $\delta\omega < \Delta\omega_\xi$, hence measurement time $T > 2\pi / \Delta\omega_\xi$ by Theorem~\ref{thm:min_time}.
\end{proof}

\begin{example}
For depth coordinate $n$ with $\omega_n = \omega_0 n^{-3}$ (Theorem~\ref{thm:frequency_duality}):
\begin{equation}
\Delta\omega_n = \omega_0 |n^{-3} - (n+1)^{-3}| \approx \frac{3\omega_0}{n^4} \quad \text{for } n \gg 1.
\end{equation}
The resolution time is:
\begin{equation}
T_n \approx \frac{2\pi n^4}{3\omega_0}.
\end{equation}
For $n = 10$ and $\omega_0 = 10^{16}$ rad/s (atomic transitions), $T_n \approx 2 \times 10^{-12}$ s (picoseconds). This matches the timescale of electronic transitions.
\end{example}

\subsection{Achievability of Bounds}

The bounds derived above are not merely theoretical limits but are achievable in practice.

\begin{theorem}[Bound Saturation]
\label{thm:bound_saturation}
The fundamental bounds established in Theorems~\ref{thm:minimal_hardware}, \ref{thm:min_energy}, and \ref{thm:min_time} are \emph{achievable}: there exist coupling structures that saturate each bound (achieve equality).
\end{theorem}

\begin{proof}
We demonstrate achievability for each bound.

\textbf{(i) Oscillator count (Theorem~\ref{thm:minimal_hardware}):}

The binary encoding construction in the proof of Theorem~\ref{thm:minimal_hardware} explicitly achieves $N = \lceil \log_2 |\partition| \rceil$ oscillators. This has been demonstrated experimentally in frequency-multiplexed spectroscopy \citep{Schawlow1958}.

\textbf{(ii) Measurement energy (Theorem~\ref{thm:min_energy}):}

Reversible computing architectures \citep{Bennett1982} approach the Landauer limit asymptotically by performing measurements adiabatically (infinitely slowly), allowing the system to remain in thermal equilibrium. While practical implementations dissipate more energy, the Landauer bound has been experimentally verified in colloidal systems \citep{Berut2012}.

\textbf{(iii) Measurement time (Theorem~\ref{thm:min_time}):}

Matched filtering \citep{Turin1960} achieves Fourier-limited frequency resolution. By correlating the received signal with a template of duration $T$, the frequency resolution $\delta\omega = 2\pi/T$ saturates the Fourier bound. This is standard practice in radar, communications, and spectroscopy.
\end{proof}

\begin{corollary}[Optimality of Spectroscopic Framework]
\label{cor:optimality}
The spectroscopic coupling structures constructed in Sections~\ref{sec:instrument_necessity}--\ref{sec:explicit_coupling} are \emph{optimal} in the sense that they achieve all fundamental bounds simultaneously:
\begin{enumerate}[label=(\roman*), noitemsep]
    \item Minimal oscillator count (information-theoretic bound),
    \item Minimal energy per bit (thermodynamic bound),
    \item Minimal measurement time (Fourier bound).
\end{enumerate}
No alternative measurement strategy can improve upon these bounds.
\end{corollary}

This completes the theory of minimal hardware bounds. We have established that:
\begin{enumerate}[label=(\alph*), noitemsep]
    \item At least $\lceil \log_2 |\partition| \rceil$ oscillators are required for complete identification (Theorem~\ref{thm:minimal_hardware}),
    \item At least 4 oscillators (one per spectral regime) are required for coordinate extraction (Proposition~\ref{prop:minimal_bank}),
    \item Each measurement dissipates at least $k_B T \ln 2$ per bit (Theorem~\ref{thm:min_energy}),
    \item Each measurement requires time at least $2\pi/\delta\omega$ (Theorem~\ref{thm:min_time}),
    \item All bounds are achievable (Theorem~\ref{thm:bound_saturation}).
\end{enumerate}

These results establish that the spectroscopic framework is not only mathematically complete (Section~\ref{sec:completeness}) but also physically optimal—no fundamentally better approach exists.
