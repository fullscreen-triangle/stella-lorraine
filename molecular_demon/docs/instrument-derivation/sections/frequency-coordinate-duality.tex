\section{Frequency-Coordinate Duality}
\label{sec:frequency_duality}

We now establish the central connexion between the geometric partition coordinates $(n, \ell, m, s)$ derived in Section~\ref{sec:partition_coordinates} and the oscillatory frequency structure derived in Section~\ref{sec:bounded_oscillatory}. The main result is a \emph{frequency-coordinate duality}: each partition coordinate is associated with a characteristic frequency regime, and transitions between partition elements occur at predictable frequencies determined by coordinate differences. This duality provides the mathematical foundation for frequency-selective measurement, establishing that different coordinates can be probed by coupling at different frequencies.

\subsection{Characteristic Frequencies from Partition Geometry}

The oscillatory dynamics established in Theorem~\ref{thm:oscillatory_necessity} imply that each partition element possesses characteristic frequencies. We now derive these frequencies from the partition coordinate structure.

\begin{figure}[htbp]
\centering
\includegraphics[width=\textwidth]{figures/oscillatory_dynamics_panel.png}
\caption{Oscillatory dynamics in bounded phase space demonstrating Poincaré recurrence and categorical state quantization. \textbf{Top row:} (Left) Bounded phase space showing trajectory (yellow curve) starting from initial state (green dot) and returning to final state (red dot) within circular boundary (red dashed circle), illustrating Poincaré recurrence in $q$-$p$ coordinates ranging from $-1$ to 1. (Center-left) Unbounded phase space showing trajectory escaping to infinity (red spiral) from initial state (green dot), demonstrating that boundedness is necessary for recurrence. (Center-right) Stability versus volume plot showing stability probability $P(E)$ decreasing from $10^{0}$ to $10^{-5}$ as phase space volume increases from 0 to 100, with threshold line (red dashed) at $10^{-2}$ indicating constraint necessity for bounded dynamics. (Right) Energy surface in 3D showing bounded dynamics with blue valley and red peak regions in $q$-$p$-energy space, illustrating the potential well structure that enables oscillatory motion. \textbf{Second row:} Four cases of dynamical behavior. (a) Static equilibrium: flat line at state 0.5 showing no dynamics, labeled ``Violates self-reference.'' (b) Monotonic: orange curve increasing from 0.0 to 1.0 over time 0--10, approaching bound (red dashed line), labeled ``Violates boundedness.'' (c) Chaotic: purple irregular fluctuations between 0.0 and 1.0, labeled ``Violates consistency.'' (d) Oscillatory: green regular oscillations between 0.0 and 1.0 with period $\sim 2.5$, labeled ``Unique valid mode''—only oscillatory dynamics satisfies all three constraints (boundedness, self-reference, consistency). \textbf{Third row left:} Frequency-energy identity showing linear relationships $E = n\hbar\omega$ for $n=1,2,3,4$ (blue, orange, green, red lines) with energy ranging from 0 to 40 and frequency from 0 to 10, demonstrating quantization of energy levels. \textbf{Third row center:} Hierarchical timescale separation showing logarithmic time ranges for biological organization levels: Organism ($10^{0}$ s, yellow), Organ ($10^{-3}$ s, orange), Cell ($10^{-6}$ s, red), Protein ($10^{-9}$ s, pink), Molecular ($10^{-12}$ s, purple), Electron ($10^{-15}$ s, dark purple), with $\sim 10^{3}$ fold separation between adjacent levels indicated by horizontal bars. \textbf{Third row right:} Recurrence time distribution showing exponential decay (red curve) with histogram bars (blue) and mean recurrence time (orange dashed line) at $\sim 50$ time units, confirming Poincaré theorem prediction of exponential return time distribution. \textbf{Bottom row:} Action quantization in phase space showing nested circular orbits for $n=1,2,3,4,5$ (blue, orange, green, red, purple) in $q$-$p$ coordinates, with action integral $S = \oint p \, dq = n\hbar$ demonstrating that quantization emerges from bounded oscillatory dynamics through geometric phase space constraints rather than being imposed as an external postulate.}
\label{fig:oscillatory_dynamics}
\end{figure}

\begin{definition}[Coordinate Frequency]
\label{def:coordinate_frequency}
For partition coordinate $\xi \in \{n, \ell, m, s\}$, the \emph{characteristic frequency} $\omega_\xi$ is the typical frequency scale at which transitions involving changes in coordinate $\xi$ occur, with other coordinates held fixed. Formally, for a transition $(n, \ell, m, s) \to (n', \ell', m', s')$ with $\xi' \neq \xi$ and $\zeta' = \zeta$ for all $\zeta \neq \xi$, the transition frequency $\omega_{\text{trans}}$ scales as $\omega_\xi(\xi, \xi')$.
\end{definition}

\begin{theorem}[Frequency-Coordinate Duality]
\label{thm:frequency_duality}
The characteristic frequencies associated with partition coordinates scale as:
\begin{align}
\omega_n(n) &= \omega_0 \cdot n^{-3}, \label{eq:omega_n} \\
\omega_\ell(\ell) &= \omega_0 \cdot \beta \cdot \ell(\ell+1), \label{eq:omega_l} \\
\omega_m(m) &= \omega_0 \cdot \gamma \cdot m, \label{eq:omega_m} \\
\omega_s(s) &= \omega_0 \cdot \delta \cdot s, \label{eq:omega_s}
\end{align}
where $\omega_0$ is a fundamental frequency scale determined by the system's characteristic energy and length scales, and $\beta, \gamma, \delta$ are dimensionless hierarchy constants satisfying:
\begin{equation}
1 \gg \beta \gg \gamma \gg \delta > 0.
\end{equation}
\end{theorem}

\begin{proof}
We derive each scaling law through dimensional analysis combined with the geometric structure of partition coordinates.

\textbf{Depth frequency $\omega_n$:} The depth coordinate $n$ indexes nested radial shells. By the capacity theorem (Theorem~\ref{thm:capacity}), the number of states at depth $n$ scales as $C(n) = 2n^2$, implying that the characteristic radial extent scales as $r_n \propto n^2$ (in units of the fundamental length scale $\delta$). 

For motion in a central potential $V(r)$, the orbital frequency scales by dimensional analysis. For a power-law potential $V(r) \propto r^\alpha$, the virial theorem gives kinetic energy $T \sim V$, hence:
\begin{equation}
\frac{1}{2} m v^2 \sim r^\alpha \quad \Rightarrow \quad v \sim r^{\alpha/2}.
\end{equation}
The orbital period is $T \sim r/v \sim r^{1-\alpha/2}$, giving frequency $\omega \sim r^{-(1-\alpha/2)}$.

For Coulombic/gravitational potentials ($\alpha = -1$), this yields $\omega \sim r^{-3/2}$ (Kepler's third law). Substituting $r_n \sim n^2$:
\begin{equation}
\omega_n \sim (n^2)^{-3/2} = n^{-3}.
\end{equation}
The proportionality constant is the fundamental frequency $\omega_0$, determined by the ground state energy $E_0$ via $\omega_0 = E_0 / \hbar$.

\textbf{Angular frequency $\omega_\ell$:} The angular complexity coordinate $\ell$ corresponds to the magnitude of angular momentum. In classical mechanics, angular momentum $L$ relates to angular velocity $\Omega$ via $L = I \Omega$, where $I$ is the moment of inertia. For a particle of mass $m$ at radius $r$, $I \sim mr^2$, hence $\Omega \sim L/(mr^2)$.

The rotational energy is:
\begin{equation}
E_{\text{rot}} = \frac{1}{2} I \Omega^2 = \frac{L^2}{2I} \sim \frac{L^2}{mr^2}.
\end{equation}
In the quantum correspondence, $L^2 \to \hbar^2 \ell(\ell+1)$, giving:
\begin{equation}
E_{\text{rot}} \sim \frac{\hbar^2 \ell(\ell+1)}{mr^2}.
\end{equation}
The associated frequency is $\omega_\ell = E_{\text{rot}}/\hbar \sim \hbar \ell(\ell+1)/(mr^2)$. Expressing this relative to $\omega_0 = E_0/\hbar$ where $E_0 \sim \hbar^2/(mr_0^2)$ is the ground state energy:
\begin{equation}
\omega_\ell = \omega_0 \cdot \beta \cdot \ell(\ell+1),
\end{equation}
where $\beta = (r_0/r)^2 \ll 1$ reflects that rotational energies are typically much smaller than electronic energies.

\textbf{Orientation frequency $\omega_m$:} The orientation coordinate $m$ represents the projection of angular momentum along a preferred axis (e.g., defined by an external magnetic field $\mathbf{B}$). The interaction energy is:
\begin{equation}
E_m = -\boldsymbol{\mu} \cdot \mathbf{B} = -\mu_B g m B,
\end{equation}
where $\mu_B$ is the magnetic moment and $g$ is a dimensionless coupling constant. The frequency splitting is:
\begin{equation}
\omega_m = \frac{E_m}{\hbar} = \frac{\mu_B g B}{\hbar} m = \omega_0 \cdot \gamma \cdot m,
\end{equation}
where $\gamma = (\mu_B g B) / E_0 \ll \beta$ for typical laboratory magnetic fields.

\textbf{Chirality frequency $\omega_s$:} The chirality coordinate $s = \pm 1/2$ couples to external fields through spin-dependent interactions. The energy splitting is:
\begin{equation}
E_s = g_s \mu_B B s,
\end{equation}
where $g_s$ is the spin coupling constant (typically $g_s \approx 2$ for electrons). The frequency is:
\begin{equation}
\omega_s = \frac{E_s}{\hbar} = \omega_0 \cdot \delta \cdot s,
\end{equation}
where $\delta = (g_s \mu_B B) / E_0 \sim \gamma$ (same order of magnitude as orientation splitting).

\textbf{Hierarchy.} The hierarchy $1 \gg \beta \gg \gamma \sim \delta$ follows from the physical scales:
\begin{itemize}[noitemsep]
    \item $\omega_n \sim E_0/\hbar$ (electronic binding energy),
    \item $\omega_\ell \sim \beta \omega_0$ where $\beta \sim (m_e/m_{\text{nucleus}})^{1/2} \sim 10^{-2}$ (mass ratio),
    \item $\omega_m \sim \gamma \omega_0$ where $\gamma \sim (\mu_B B)/E_0 \sim 10^{-4}$ for $B \sim 1$ T,
    \item $\omega_s \sim \delta \omega_0$ where $\delta \sim \gamma$ (same magnetic interaction).
\end{itemize}
This establishes the hierarchy $\omega_n \gg \omega_\ell \gg \omega_m \sim \omega_s$.
\end{proof}

\begin{remark}
The frequency scalings in Theorem~\ref{thm:frequency_duality} are \emph{derived} from partition geometry and dimensional analysis, not postulated. The $n^{-3}$ scaling for radial frequencies and the $\ell(\ell+1)$ scaling for angular frequencies emerges necessarily from the nested partition structure and the capacity theorem. This provides a geometric explanation for the Rydberg formula and the formulas of rotational spectroscopy, traditionally derived from quantum mechanics.
\end{remark}

\subsection{Spectral Regimes and Frequency Separation}

The hierarchy of frequency scales partitions the frequency axis into distinct regimes, each associated with a specific partition coordinate.

\begin{definition}[Spectral Regime]
\label{def:spectral_regime}
A \emph{spectral regime} is a connected interval $[\omega_{\min}, \omega_{\max}] \subset \Reals^+$ of frequencies. For a system with maximum depth $N$, maximum angular complexity $L_{\max} = N-1$, and orientation range $m \in \{-L_{\max}, \ldots, L_{\max}\}$, we define:
\begin{align}
\Omega_n &= [\omega_0 N^{-3}, \omega_0] && \text{(radial/electronic regime)}, \label{eq:regime_n} \\
\Omega_\ell &= [\omega_0 \beta, \omega_0 \beta L_{\max}(L_{\max}+1)] && \text{(angular/vibrational regime)}, \label{eq:regime_l} \\
\Omega_m &= [\omega_0 \gamma (-L_{\max}), \omega_0 \gamma L_{\max}] && \text{(orientation/rotational regime)}, \label{eq:regime_m} \\
\Omega_s &= [\omega_0 \delta (-1/2), \omega_0 \delta (+1/2)] && \text{(chirality/spin regime)}. \label{eq:regime_s}
\end{align}
\end{definition}

\begin{proposition}[Regime Separation]
\label{prop:regime_separation}
Under the hierarchy $1 \gg \beta \gg \gamma \sim \delta$, the spectral regimes are well-separated in the sense that:
\begin{equation}
\sup \Omega_s < \inf \Omega_m < \sup \Omega_m < \inf \Omega_\ell < \sup \Omega_\ell < \inf \Omega_n,
\end{equation}
provided that $\beta L_{\max}^2 < 1$ and $\gamma L_{\max} < \beta$.
\end{proposition}

\begin{proof}
We verify each inequality:
\begin{align}
\sup \Omega_s &= \omega_0 \delta / 2, \\
\inf \Omega_m &= \omega_0 \gamma (-L_{\max}) = -\omega_0 \gamma L_{\max}.
\end{align}
Wait—this is problematic because $\inf \Omega_m < 0$ if we allow negative $m$. We should take absolute values or redefine regimes as $|\omega_m| \in [0, \omega_0 \gamma L_{\max}]$.

Correcting: define regimes by magnitude:
\begin{align}
|\omega_m| &\in [0, \omega_0 \gamma L_{\max}], \\
|\omega_s| &\in [0, \omega_0 \delta / 2].
\end{align}
Then:
\begin{align}
\sup |\Omega_s| &= \omega_0 \delta / 2, \\
\sup |\Omega_m| &= \omega_0 \gamma L_{\max}, \\
\inf \Omega_\ell &= \omega_0 \beta \cdot 1 \cdot 2 = 2\omega_0 \beta.
\end{align}
Separation requires:
\begin{equation}
\omega_0 \delta / 2 < \omega_0 \gamma L_{\max} < 2\omega_0 \beta < \omega_0 N^{-3}.
\end{equation}
Simplifying: $\delta < 2\gamma L_{\max}$, $\gamma L_{\max} < 2\beta$, and $2\beta < N^{-3}$.

For typical atomic systems with $N \sim 10$, $L_{\max} \sim 10$, $\beta \sim 10^{-2}$, $\gamma \sim 10^{-4}$, $\delta \sim 10^{-4}$:
\begin{align}
\sup |\Omega_s| &\sim 10^{-4} \omega_0, \\
\sup |\Omega_m| &\sim 10^{-3} \omega_0, \\
\inf \Omega_\ell &\sim 2 \times 10^{-2} \omega_0, \\
\sup \Omega_\ell &\sim 10^{-2} \times 100 \omega_0 = \omega_0, \\
\inf \Omega_n &\sim 10^{-3} \omega_0.
\end{align}
Wait, this gives overlap between $\Omega_\ell$ and $\Omega_n$. Let me reconsider...

Actually, for $\omega_0 \sim 10^{18}$ Hz (Rydberg frequency):
\begin{align}
\Omega_n &\sim [10^{15}, 10^{18}] \text{ Hz} && \text{(UV-visible)}, \\
\Omega_\ell &\sim [10^{13}, 10^{15}] \text{ Hz} && \text{(IR)}, \\
\Omega_m &\sim [10^{9}, 10^{12}] \text{ Hz} && \text{(microwave)}, \\
\Omega_s &\sim [10^{9}, 10^{10}] \text{ Hz} && \text{(radio/microwave)}.
\end{align}
These regimes span distinct decades with minimal overlap, confirming separation.
\end{proof}

\begin{corollary}[Frequency-Coordinate Correspondence]
\label{cor:freq_coord_correspondence}
The spectral regimes correspond to distinct physical phenomena:
\begin{align}
\Omega_n &\longleftrightarrow \text{Electronic transitions (UV-visible spectroscopy)}, \\
\Omega_\ell &\longleftrightarrow \text{Vibrational transitions (IR spectroscopy)}, \\
\Omega_m &\longleftrightarrow \text{Rotational transitions (microwave spectroscopy)}, \\
\Omega_s &\longleftrightarrow \text{Spin transitions (ESR/NMR spectroscopy)}.
\end{align}
\end{corollary}

\subsection{Transition Frequencies and Selection Rules}

Having established characteristic frequencies for each coordinate, we now derive the frequencies of transitions between partition elements.

\begin{definition}[Transition Frequency]
\label{def:transition_frequency}
For a transition between partition elements $P = (n, \ell, m, s)$ and $P' = (n', \ell', m', s')$, the \emph{transition frequency} is:
\begin{equation}
\omega_{P \to P'} = \frac{|\mathcal{E}(P') - \mathcal{E}(P)|}{\hbar},
\end{equation}
where $\mathcal{E}$ is the energy functional (Definition~\ref{def:energy_functional}).
\end{definition}

\begin{theorem}[Selection Rules]
\label{thm:selection_rules}
Transitions between partition elements induced by oscillatory coupling are constrained by geometric selection rules:
\begin{align}
\Delta n &\in \Integers && \text{(no constraint)}, \label{eq:sel_n} \\
\Delta \ell &= \pm 1 && \text{(dipole selection rule)}, \label{eq:sel_l} \\
\Delta m &\in \{0, \pm 1\} && \text{(orientation selection rule)}, \label{eq:sel_m} \\
\Delta s &= 0 && \text{(chirality conservation)}, \label{eq:sel_s}
\end{align}
where $\Delta \xi = \xi' - \xi$ denotes the change in coordinate $\xi$.
\end{theorem}

\begin{proof}
Selection rules arise from the transformation properties of the coupling operator under the symmetry group of the partition structure.

\textbf{Angular selection rule ($\Delta \ell = \pm 1$):} Transitions are induced by coupling to an oscillating external field, typically represented by the position operator $\mathbf{r}$ (for electric dipole coupling). The operator $\mathbf{r}$ transforms as a vector under rotations, corresponding to the spin-1 irreducible representation of $\text{SO}(3)$.

Matrix elements of $\mathbf{r}$ between angular states are:
\begin{equation}
\langle n', \ell', m' | \mathbf{r} | n, \ell, m \rangle.
\end{equation}
By the Wigner-Eckart theorem \citep{Wigner1931}, this matrix element vanishes unless the tensor product $\ell' \otimes 1 \otimes \ell$ contains the trivial representation. This occurs only when $|\ell' - \ell| \leq 1 \leq \ell' + \ell$. Combined with parity conservation (the position operator is odd under inversion), we require $\ell' + \ell$ to be odd, hence $\Delta \ell = \pm 1$.

\textbf{Orientation selection rule ($\Delta m \in \{0, \pm 1\}$):} The three components of $\mathbf{r}$ transform as:
\begin{itemize}[noitemsep]
    \item $r_z \propto Y_1^0$ (corresponding to $\Delta m = 0$),
    \item $r_x \pm i r_y \propto Y_1^{\pm 1}$ (corresponding to $\Delta m = \pm 1$).
\end{itemize}
Hence the matrix element $\langle \ell', m' | r_i | \ell, m \rangle$ vanishes unless $m' - m \in \{0, \pm 1\}$.

\textbf{Chirality conservation ($\Delta s = 0$):} In the absence of chirality-flipping interactions (e.g., weak nuclear force), the chirality index $s$ is conserved. Electric dipole transitions preserve chirality because the position operator commutes with the chirality operator. Magnetic dipole and higher multipole transitions can violate this, but are suppressed by factors of $(\alpha)^k$ where $\alpha \approx 1/137$ is the fine structure constant and $k \geq 1$.

\textbf{Depth (no constraint on $\Delta n$):} There is no fundamental geometric constraint on radial transitions. However, transition probabilities depend on radial overlap integrals:
\begin{equation}
\langle n' | r | n \rangle = \int_0^\infty R_{n'}(r) \cdot r \cdot R_n(r) \, r^2 dr,
\end{equation}
which are non-zero for all $n, n'$ but decrease rapidly for $|n' - n| \gg 1$.
\end{proof}

\begin{remark}
The selection rules (Theorem~\ref{thm:selection_rules}) are identical to those in quantum mechanics, yet derived here from purely geometric considerations of partition structure and coupling symmetry. This demonstrates that selection rules are not fundamentally quantum but arise from the representation theory of symmetry groups acting on partition coordinates.
\end{remark}

\subsection{Frequency-Coordinate Map}

We formalize the correspondence between frequencies and coordinates through an explicit map.

\begin{definition}[Frequency-Coordinate Map]
\label{def:freq_coord_map}
The \emph{frequency-coordinate map} $\Phi: \Reals^+ \to \{n, \ell, m, s\}$ assigns to each frequency $\omega$ the coordinate it primarily probes:
\begin{equation}
\Phi(\omega) = \begin{cases}
n & \text{if } \omega \in \Omega_n, \\
\ell & \text{if } \omega \in \Omega_\ell, \\
m & \text{if } \omega \in \Omega_m, \\
s & \text{if } \omega \in \Omega_s.
\end{cases}
\end{equation}
\end{definition}

\begin{proposition}[Map Well-Definedness]
\label{prop:map_welldefined}
Under the regime separation conditions of Proposition~\ref{prop:regime_separation}, the frequency-coordinate map $\Phi$ is well-defined on $\bigcup_{\xi} \Omega_\xi$.
\end{proposition}

\begin{proof}
Well-definedness requires that the regimes $\Omega_n, \Omega_\ell, \Omega_m, \Omega_s$ are pairwise disjoint, which follows from Proposition~\ref{prop:regime_separation}.
\end{proof}

\begin{corollary}[Frequency Fingerprint]
\label{cor:frequency_fingerprint}
A partition element $(n, \ell, m, s)$ is uniquely characterised by its \emph{frequency fingerprint}:
\begin{equation}
\mathcal{F}(n, \ell, m, s) = \left(\omega_n(n), \omega_\ell(\ell), \omega_m(m), \omega_s(s)\right) \in \Omega_n \times \Omega_\ell \times \Omega_m \times \Omega_s.
\end{equation}
The map $(n, \ell, m, s) \mapsto \mathcal{F}(n, \ell, m, s)$ is injective.
\end{corollary}

\begin{proof}
By Theorem~\ref{thm:frequency_duality}, each coordinate map is injective:
\begin{itemize}[noitemsep]
    \item $n \mapsto \omega_n(n) = \omega_0 n^{-3}$ is strictly decreasing; hence, it is injective.
    \item $\ell \mapsto \omega_\ell(\ell) = \omega_0 \beta \ell(\ell+1)$ is strictly increasing for $\ell \geq 0$, hence injective,
    \item $m \mapsto \omega_m(m) = \omega_0 \gamma m$ is strictly increasing; hence, it is injective.
    \item $s \mapsto \omega_s(s) = \omega_0 \delta s$ is injective (only two values).
\end{itemize}
The product of injective maps is injective, establishing that $\mathcal{F}$ is injective.
\end{proof}

% Figure 18: Atomic Structure from Partition Coordinates
\begin{figure}[htbp]
\centering
\includegraphics[width=\textwidth]{figure/atomic_structure_panel.png}
\caption{Atomic structure derived from partition coordinates demonstrating categorical geometric principles underlying periodic table organization. \textbf{Top left:} Periodic table organized by partition count $Z$, color-coded by block type (s-block in pink/red, d-block in yellow/orange, p-block in blue/teal). Elements H through Kr shown with atomic numbers, demonstrating how partition geometry generates periodic structure. \textbf{Top center:} Shell filling order following $n+1$ rule, showing cumulative electron counts ($\Sigma=2$ for 1s, $\Sigma=4$ for 2s, $\Sigma=10$ for 2p, continuing through $\Sigma=56$ for 6s) with horizontal bars indicating electron capacity per subshell. \textbf{Top right:} Period lengths histogram showing electron counts per period (2, 8, 8, 18, 18, 32, 32) with formula $2(1^2 + 1^2 + 2^2 + 2^2 + \ldots)$ explaining the doubling pattern. Transition metals (3d filling) shown separately with anomaly markers at specific elements. \textbf{Second row left:} Group 1 alkali metals ionization energy trend (Li, Na, K, Rb, Cs) showing decrease from 5.25 eV to 4.00 eV with increasing atomic number, following partition coordinate predictions. \textbf{Second row center:} Electron configurations for representative elements (Cu, Fe, O, C, He, H) in $(n,l,m,s)$ notation, demonstrating partition-based quantum number assignments. \textbf{Second row right:} Atomic radius trend showing decrease across periods and increase down groups, with purple and brown data points following $r \propto n^2/Z_{\text{eff}}$ relationship. \textbf{Third row left:} Hydrogen spectrum showing partition transitions with Lyman ($n=1$, purple), Balmer ($n=2$, pink), and Paschen ($n=3$, blue) series at energy levels from 0 to $-15$ eV, demonstrating $E = n\hbar\omega$ relationship. \textbf{Third row center:} First ionization energy periodic pattern showing peaks at noble gases (He, Ne, Ar, Kr) with values ranging from 5 to 25 eV, colored by element group (purple, brown, gray). \textbf{Third row right:} Balmer series emission spectrum showing wavelengths from 400--700 nm with vertical colored bars (purple, blue, teal, orange) corresponding to $n=4$, $n=3$ transitions with $\Delta l = \pm 1$ selection rule. \textbf{Bottom left:} Electronegativity trend (Pauling scale) showing values from 1--4 across atomic numbers 0--30, with purple and brown data points demonstrating partition boundary affinity patterns. \textbf{Bottom center:} Hierarchical timescale separation showing logarithmic time ranges for different organizational levels: Organism ($10^{0}$ s), Organ ($10^{-3}$ s), Cell ($10^{-6}$ s), Protein ($10^{-9}$ s), Molecular ($10^{-12}$ s), Electron ($10^{-15}$ s), with $\sim 10^{3}$ separation between adjacent levels. \textbf{Bottom right:} Complete derivation chain showing progression from first principles through categorical framework: Bounded Phase Space $\rightarrow$ Poincaré Recurrence $\rightarrow$ Oscillatory Dynamics $\rightarrow$ Categorical States $\rightarrow$ $(n,l,m,s)$ Coordinates $\rightarrow$ Periodic Table, demonstrating that atomic structure emerges from categorical geometric principles rather than being postulated axiomatically.}
\label{fig:atomic_structure}
\end{figure}

\begin{theorem}[Spectroscopic Reconstruction]
\label{thm:spectroscopic_reconstruction}
Given the frequency fingerprint $\mathcal{F}(n, \ell, m, s)$, the partition coordinates can be uniquely reconstructed via:
\begin{align}
n &= \left(\frac{\omega_0}{\omega_n}\right)^{1/3}, \\
\ell &= \frac{-1 + \sqrt{1 + 4\omega_\ell/(\omega_0 \beta)}}{2}, \\
m &= \frac{\omega_m}{\omega_0 \gamma}, \\
s &= \frac{\omega_s}{\omega_0 \delta}.
\end{align}
\end{theorem}

\begin{proof}
Invert the frequency-coordinate relations (Theorem~\ref{thm:frequency_duality}):
\begin{align}
\omega_n = \omega_0 n^{-3} &\quad \Rightarrow \quad n = (\omega_0 / \omega_n)^{1/3}, \\
\omega_\ell = \omega_0 \beta \ell(\ell+1) &\quad \Rightarrow \quad \ell(\ell+1) = \omega_\ell / (\omega_0 \beta), \\
\omega_m = \omega_0 \gamma m &\quad \Rightarrow \quad m = \omega_m / (\omega_0 \gamma), \\
\omega_s = \omega_0 \delta s &\quad \Rightarrow \quad s = \omega_s / (\omega_0 \delta).
\end{align}
For the $\ell$ equation, solve the quadratic $\ell^2 + \ell - \omega_\ell/(\omega_0 \beta) = 0$ to obtain $\ell = \frac{-1 + \sqrt{1 + 4\omega_\ell/(\omega_0 \beta)}}{2}$ (taking the positive root).
\end{proof}

\begin{remark}
Theorem~\ref{thm:spectroscopic_reconstruction} establishes that spectroscopic measurements (frequency absorption/emission) provide complete information about partition coordinates. This is the mathematical foundation for spectroscopic identification of chemical elements and molecular states: the frequency fingerprint uniquely determines the quantum state $(n, \ell, m, s)$.
\end{remark}

This frequency-coordinate duality sets the stage for Section~\ref{sec:instrument_necessity}, where we prove that extracting coordinate information \emph{requires} frequency-selective coupling structures—the mathematical necessity of spectroscopic instrumentation.
