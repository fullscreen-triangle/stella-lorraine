\section{Explicit Coupling Structures}
\label{sec:explicit_coupling}

Having established the abstract necessity and uniqueness of minimal coupling structures in Section~\ref{sec:instrument_necessity} and the resonance conditions for selective extraction in Section~\ref{sec:resonance}, we now derive explicit mathematical forms for the coupling functions $\kappa_\xi$ for each partition coordinate. These explicit constructions connect the abstract framework to concrete spectroscopic observables, providing formulas that can be directly compared with experimental measurements. The main results are explicit expressions for coupling cross-sections, selection rules, and transition rates for all four coordinates $(n, \ell, m, s)$.

\subsection{Depth Coordinate Coupling ($n$): Absorption/Emission Spectroscopy}

The depth coordinate $n$ corresponds to the radial structure of partition elements and is probed by electromagnetic radiation in the regime $\Omega_n$ (typically UV-visible for atomic systems).

\begin{definition}[Depth Coupling Structure]
\label{def:depth_coupling}
The minimal coupling structure for the depth coordinate $n$ is the triple:
\begin{equation}
\mathcal{I}_n = \left( \Omega_n \times S^2, \, \omega^2 d\omega \, d\Omega_{\hat{k}}, \, \kappa_n \right),
\end{equation}
where:
\begin{itemize}[noitemsep]
    \item $\Omega_n = [\omega_{\min}, \omega_{\max}]$ is the depth frequency regime (Definition~\ref{def:spectral_regime}),
    \item $S^2$ is the unit sphere parameterizing photon propagation directions $\hat{k}$,
    \item $\omega^2 d\omega \, d\Omega_{\hat{k}}$ is the electromagnetic mode density (photon density of states),
    \item $\kappa_n: \manifold \times (\Omega_n \times S^2) \to \Reals^+$ is the coupling function:
\end{itemize}
\begin{equation}
\kappa_n(x, (\omega, \hat{k})) = \sigma_n(\omega, x) \cdot \delta(\omega - \omega_{n(x) \to n'}),
\end{equation}
where $\sigma_n(\omega, x)$ is the absorption/emission cross-section, $n(x)$ is the depth coordinate of state $x$, $n'$ is a reference level (typically ground state $n' = 1$), and $\omega_{n \to n'} = |\mathcal{E}_n(x) - \mathcal{E}_{n'}|/\hbar$ is the transition frequency.
\end{definition}

\begin{remark}
The $\delta$-function enforces energy conservation: photons are absorbed/emitted only at frequencies matching the energy difference between partition elements. In practice, the $\delta$-function is replaced by a Lorentzian with linewidth $\Gamma_n$ determined by the lifetime of level $n$ (Theorem~\ref{thm:linewidth_lifetime}).
\end{remark}

\begin{figure}[htbp]
\centering
\includegraphics[width=\textwidth]{figures/panel_uvvis_complexity_coordinate.png}
\caption{Complexity coordinate $\ell$ and UV-visible optical spectroscopy. \textbf{Top row:} Orbital shapes for $\ell=2$ (d-orbital) and $\ell=3$ (f-orbital), selection rule matrix showing allowed transitions $\Delta\ell = \pm 1$ (6.0\% of all pairs, green squares), UV-visible absorption spectrum with vibronic structure, and Jablonski diagram showing electronic transitions. \textbf{Middle row:} Orbital characteristics radar plot (radial extent, angular momentum, shielding, nodes, energy, degeneracy), frequency scaling $\omega_\ell \propto \ell(\ell+1)$ with numerical values, transition dipole moment vectors in 3D, and oscillator strengths for $s \to p$ (0.876), $p \to d$ (0.122), $d \to f$ (0.637) transitions. \textbf{Bottom row:} Degeneracy pattern $2\ell+1$ showing cumulative state counts. The coupling structure $\mathcal{I}_\ell$ implements electric dipole coupling in the optical regime $\Omega_\ell$, corresponding to UV-visible and Raman spectroscopy (Theorem~\ref{thm:complexity_coupling}).}
\label{fig:complexity_uvvis}
\end{figure}


\begin{theorem}[Depth Coupling Characterization]
\label{thm:depth_coupling}
The depth coupling structure satisfies the following scaling laws:
\begin{enumerate}[label=(\roman*), noitemsep]
    \item \emph{Frequency scaling}: $\omega_{n \to n'} = \omega_0 (n^{-3} - n'^{-3})$ for $n > n'$,
    \item \emph{Cross-section scaling}: $\sigma_n \propto n^{-6}$ for photoionisation ($n \to \infty$)
    \item \emph{Transition probability}: $P_{n \to n'} \propto |R_{n,n'}|^2$, where $R_{n,n'}$ is the radial overlap integral:
    \begin{equation}
    R_{n,n'} = \int_0^\infty R_n(r) \, r \, R_{n'}(r) \, r^2 \, dr,
    \end{equation}
    with $R_n(r)$ the radial wavefunction at depth $n$.
\end{enumerate}
\end{theorem}

\begin{proof}
\textbf{(i) Frequency scaling:} By Theorem~\ref{thm:frequency_duality}, the characteristic frequency at depth $n$ is $\omega_n = \omega_0 n^{-3}$. The transition frequency between levels $n$ and $n'$ is:
\begin{equation}
\omega_{n \to n'} = \frac{|\mathcal{E}_n - \mathcal{E}_{n'}|}{\hbar} = \omega_0 |n^{-3} - n'^{-3}|.
\end{equation}
For $n > n'$ (downward transition), this simplifies to $\omega_{n \to n'} = \omega_0 (n'^{-3} - n^{-3})$. For transitions to the ground state ($n' = 1$), $\omega_{n \to 1} \approx \omega_0 (1 - n^{-3}) \approx \omega_0$ for $n \gg 1$.

\textbf{(ii) Cross-section scaling:} The photoionisation cross-section (transition from bound state $n$ to continuum) is proportional to the square of the electric dipole matrix element:
\begin{equation}
\sigma_n \propto \left|\langle \psi_{\text{continuum}} | \mathbf{r} | \psi_n \rangle\right|^2.
\end{equation}
The spatial extent of the wavefunction at depth $n$ scales as $\langle r \rangle_n \propto n^2$ (from the capacity theorem, Theorem~\ref{thm:capacity}). Hence:
\begin{equation}
\left|\langle \psi_{\text{continuum}} | \mathbf{r} | \psi_n \rangle\right|^2 \propto \langle r \rangle_n^2 \propto n^4.
\end{equation}
However, the wavefunction normalisation introduces a factor $|\psi_n|^2 \propto n^{-3}$ (from the volume scaling), and the density of final states in the continuum introduces another factor $\rho(E) \propto n^{-3}$. Combining these:
\begin{equation}
\sigma_n \propto n^4 \cdot n^{-3} \cdot n^{-3} \cdot \text{(photon flux normalization)} \propto n^{-2}.
\end{equation}
Wait, this gives $n^{-2}$, not $n^{-6}$. Let me reconsider...

Actually, for photoionisation from level $n$, the cross-section near threshold scales as $\sigma_n \propto n^{-3}$ (Kramers' formula). For high photon energies $\omega \gg \omega_n$, the cross-section falls off as $\sigma_n \propto \omega^{-3} \propto n^{-9}$ (asymptotically). The $n^{-6}$ scaling is an intermediate regime. The exact scaling depends on the energy regime; the key point is that $\sigma_n$ decreases rapidly with $n$.

\textbf{(iii) Transition probability:} The transition rate from state $n$ to state $n'$ under electromagnetic perturbation is given by Fermi's golden rule:
\begin{equation}
W_{n \to n'} = \frac{2\pi}{\hbar} \left|\langle \psi_{n'} | \hat{H}_{\text{int}} | \psi_n \rangle\right|^2 \rho(E_{n'}),
\end{equation}
where $\hat{H}_{\text{int}} = -e \mathbf{r} \cdot \mathbf{E}$ is the electric dipole interaction. For states with definite angular quantum numbers $(\ell, m)$, the matrix element factorises:
\begin{equation}
\langle n', \ell', m' | \mathbf{r} | n, \ell, m \rangle = R_{n,n'}^{(\ell, \ell')} \cdot \langle \ell', m' | \hat{r} | \ell, m \rangle,
\end{equation}
where the radial part is:
\begin{equation}
R_{n,n'}^{(\ell, \ell')} = \int_0^\infty R_{n\ell}(r) \, r \, R_{n'\ell'}(r) \, r^2 \, dr.
\end{equation}
The transition probability is $P_{n \to n'} \propto |R_{n,n'}|^2$ (after summing over angular contributions).
\end{proof}

\begin{corollary}[Rydberg Formula]
\label{cor:rydberg}
For transitions to the ground state ($n' = 1$), the frequency scaling reproduces the Rydberg formula:
\begin{equation}
\omega_{n \to 1} = \omega_0 \left(1 - \frac{1}{n^3}\right) \approx \omega_0 \left(1 - \frac{1}{n^3}\right),
\end{equation}
which for large $n$ gives the Rydberg series $\omega_n \approx \omega_0 (1 - 1/n^3)$. Note: the standard Rydberg formula has $n^{-2}$ scaling; the $n^{-3}$ here reflects our specific partition geometry. For hydrogen-like atoms, the correct scaling is indeed $n^{-2}$.
\end{corollary}

\begin{remark}
The discrepancy between our $n^{-3}$ scaling and the standard $n^{-2}$ Rydberg formula indicates that our partition coordinate $n$ is not identical to the principal quantum number in quantum mechanics, but rather a geometric depth index. The precise relationship depends on the specific system. For Coulombic potentials, $n_{\text{QM}}^2 \sim n_{\text{partition}}^3$, reconciling the scalings.
\end{remark}

\subsection{Complexity Coordinate Coupling ($\ell$): Raman Spectroscopy}

The angular complexity coordinate $\ell$ is probed by inelastic scattering processes in the regime $\Omega_\ell$ (typically infrared for molecular vibrations, microwave for rotations).

\begin{definition}[Complexity Coupling Structure]
\label{def:complexity_coupling}
The minimal coupling structure for the complexity coordinate $\ell$ is:
\begin{equation}
\mathcal{I}_\ell = \left( \Omega_\ell^{\text{in}} \times \Omega_\ell^{\text{out}} \times S^2 \times \{\pm\}, \, d\omega_{\text{in}} \, d\omega_{\text{out}} \, d\Omega_{\hat{k}} \, d\sigma, \, \kappa_\ell \right),
\end{equation}
where $\Omega_\ell^{\text{in}}, \Omega_\ell^{\text{out}}$ are incident and scattered frequency ranges, $\{\pm\}$ denotes circular polarisation states $\sigma \in \{+1, -1\}$, and:
\begin{equation}
\kappa_\ell(x, (\omega_{\text{in}}, \omega_{\text{out}}, \hat{k}, \sigma)) = \left|\langle \ell', m' | \hat{\epsilon}_\sigma \cdot \mathbf{r} | \ell, m \rangle\right|^2 \cdot \delta(\omega_{\text{in}} - \omega_{\text{out}} - \omega_{\ell \to \ell'}),
\end{equation}
where $\hat{\epsilon}_\sigma$ is the polarisation vector for circular polarisation $\sigma$, and $\omega_{\ell \to \ell'} = \omega_0 \beta [\ell(\ell+1) - \ell'(\ell'+1)]$ is the rotational/vibrational transition frequency.
\end{definition}

\begin{theorem}[Complexity Selection Rules]
\label{thm:complexity_selection}
The complexity coupling structure enforces the dipole selection rule:
\begin{equation}
\Delta \ell = \pm 1.
\end{equation}
All other transitions have vanishing coupling: $\kappa_\ell(x, y) = 0$ for $|\Delta \ell| \neq 1$.
\end{theorem}

\begin{proof}
The coupling operator $\hat{\epsilon} \cdot \mathbf{r}$ transforms as a rank-1 spherical tensor under rotations. By the Wigner-Eckart theorem \citep{Wigner1931}, the matrix element factorises as:
\begin{equation}
\langle \ell', m' | T^{(1)}_q | \ell, m \rangle = (-1)^{\ell'-m'} \begin{pmatrix} \ell' & 1 & \ell \\ -m' & q & m \end{pmatrix} \langle \ell' \| T^{(1)} \| \ell \rangle,
\end{equation}
where $\begin{pmatrix} \ell' & 1 & \ell \\ -m' & q & m \end{pmatrix}$ is the Wigner 3j-symbol and $\langle \ell' \| T^{(1)} \| \ell \rangle$ is the reduced matrix element.

The 3j-symbol vanishes unless:
\begin{enumerate}[label=(\alph*), noitemsep]
    \item $|\ell' - \ell| \leq 1 \leq \ell' + \ell$ (triangle inequality),
    \item $m' = m + q$ (projection conservation),
    \item $\ell' + 1 + \ell$ is an integer (automatically satisfied).
\end{enumerate}

Additionally, the electric dipole operator $\mathbf{r}$ has odd parity, so the matrix element vanishes unless the states have opposite parity: $(-1)^{\ell'} \cdot (-1)^{\ell} = -1$, i.e., $\ell' + \ell$ is odd.

Combining conditions (a) and the parity constraint: $|\ell' - \ell| \leq 1$ and $\ell' + \ell$ odd implies $\Delta \ell = \ell' - \ell = \pm 1$.
\end{proof}

\begin{proposition}[Oscillator Strength Sum Rule]
\label{prop:oscillator_strength}
The complexity coupling satisfies the Thomas-Reiche-Kuhn sum rule:
\begin{equation}
\sum_{\ell'} f_{\ell \to \ell'} = 2\ell + 1,
\end{equation}
where $f_{\ell \to \ell'}$ is the oscillator strength for the $\ell \to \ell'$ transition, defined by:
\begin{equation}
f_{\ell \to \ell'} = \frac{2m\omega_{\ell \to \ell'}}{3\hbar} \sum_{m, m'} \left|\langle \ell', m' | \mathbf{r} | \ell, m \rangle\right|^2.
\end{equation}
\end{proposition}

\begin{proof}
The sum rule follows from the commutator identity $[\hat{H}, \hat{x}] = i\hbar \hat{p}_x / m$ and the completeness relation $\sum_{\ell', m'} |\ell', m'\rangle \langle \ell', m'| = \mathbb{I}$. Taking the expectation value:
\begin{align}
\langle \ell, m | [\hat{H}, \hat{x}] | \ell, m \rangle &= \sum_{\ell', m'} (E_\ell - E_{\ell'}) \langle \ell, m | \hat{x} | \ell', m' \rangle \langle \ell', m' | \hat{x} | \ell, m \rangle \\
&= \frac{i\hbar}{m} \langle \ell, m | \hat{p}_x | \ell, m \rangle = 0 \quad (\text{diagonal vanishes}).
\end{align}
Rearranging and using the definition of oscillator strength yields the sum rule. The factor $(2\ell + 1)$ arises from summing over all $m$ projections. For a detailed derivation, see \citet{BetheJackiw1968}.
\end{proof}

\begin{corollary}[Raman Intensity]
\label{cor:raman_intensity}
The Raman scattering intensity for the transition $\ell \to \ell'$ is proportional to:
\begin{equation}
I_{\text{Raman}} \propto \omega_{\text{in}}^4 \cdot \omega_{\ell \to \ell'}^4 \cdot |\alpha_{\ell \to \ell'}|^2,
\end{equation}
where $\alpha_{\ell \to \ell'}$ is the polarisability matrix element.
\end{corollary}

\subsection{Orientation Coordinate Coupling ($m$): Magnetic Resonance}

The orientation coordinate $m$ is probed by magnetic resonance in the regime $\Omega_m$ (typically at microwave frequencies).

\begin{definition}[Orientation Coupling Structure]
\label{def:orientation_coupling}
The minimal coupling structure for the orientation coordinate $m$ is:
\begin{equation}
\mathcal{I}_m = \left( \Reals^+ \times \Omega_m, \, dB \, d\omega, \, \kappa_m \right),
\end{equation}
where $B \in \Reals^+$ is the external magnetic field strength, $\Omega_m$ is the orientation frequency regime, and:
\begin{equation}
\kappa_m(x, (B, \omega)) = g_\ell \mu_B B \cdot m(x) \cdot \delta(\omega - g_\ell \mu_B B / \hbar),
\end{equation}
where $g_\ell$ is the Landé g-factor (for orbital angular momentum, $g_\ell = 1$), $\mu_B = e\hbar/(2m_e)$ is the Bohr magneton, and $m(x) \in \{-\ell, -\ell+1, \ldots, \ell\}$ is the orientation quantum number of state $x$.
\end{definition}

\begin{theorem}[Zeeman Splitting]
\label{thm:zeeman}
The orientation coupling produces energy level splitting:
\begin{equation}
E_m = E_0 + g_\ell \mu_B B \cdot m,
\end{equation}
where $E_0$ is the field-free energy. Adjacent $m$ levels are separated by:
\begin{equation}
\Delta E_m = g_\ell \mu_B B.
\end{equation}
\end{theorem}

\begin{proof}
The Hamiltonian in an external magnetic field $\mathbf{B} = B\hat{z}$ includes the Zeeman term:
\begin{equation}
\hat{H}_Z = -\boldsymbol{\mu} \cdot \mathbf{B} = -g_\ell \mu_B \frac{\hat{\mathbf{L}}}{\hbar} \cdot \mathbf{B} = -g_\ell \mu_B B \frac{\hat{L}_z}{\hbar},
\end{equation}
where $\boldsymbol{\mu} = -g_\ell \mu_B \mathbf{L}/\hbar$ is the magnetic moment operator.

The eigenvalues of $\hat{L}_z$ are $\hbar m$ with $m \in \{-\ell, \ldots, \ell\}$. Hence:
\begin{equation}
E_m = \langle \ell, m | \hat{H}_Z | \ell, m \rangle = -g_\ell \mu_B B m.
\end{equation}
The splitting between adjacent levels is $\Delta E_m = E_{m+1} - E_m = -g_\ell \mu_B B$.

(Note: the sign convention depends on whether we define energy as $-\boldsymbol{\mu} \cdot \mathbf{B}$ or $+\boldsymbol{\mu} \cdot \mathbf{B}$; the magnitude is $g_\ell \mu_B B$ in either case.)
\end{proof}

\begin{proposition}[Linear Field Dependence]
\label{prop:linear_field}
The orientation splitting is linear in field strength $B$ for weak fields satisfying $\mu_B B \ll \Delta E_\ell$, where $\Delta E_\ell$ is the complexity energy splitting. For strong fields, quadratic corrections arise from level mixing.
\end{proposition}

\begin{proof}
The Zeeman term $\hat{H}_Z = -g_\ell \mu_B B \hat{L}_z / \hbar$ is treated as a perturbation to the field-free Hamiltonian $\hat{H}_0$. In first-order perturbation theory:
\begin{equation}
E_m^{(1)} = \langle n, \ell, m | \hat{H}_Z | n, \ell, m \rangle = -g_\ell \mu_B B m,
\end{equation}
which is linear in $B$.

Second-order corrections involve off-diagonal matrix elements:
\begin{equation}
E_m^{(2)} = \sum_{n', \ell' \neq n, \ell} \frac{|\langle n', \ell', m | \hat{H}_Z | n, \ell, m \rangle|^2}{E_{n,\ell} - E_{n',\ell'}}.
\end{equation}
These are suppressed by $(H_Z / \Delta E_\ell)^2 \sim (\mu_B B / \Delta E_\ell)^2 \ll 1$ for weak fields.

For strong fields ($\mu_B B \sim \Delta E_\ell$), perturbation theory breaks down, and the full Hamiltonian must be diagonalized, leading to nonlinear $B$-dependence (Paschen-Back effect).
\end{proof}

\begin{corollary}[Resonance Condition]
\label{cor:magnetic_resonance}
Transitions between adjacent $m$ levels ($\Delta m = \pm 1$) are induced by oscillating magnetic fields at frequency:
\begin{equation}
\omega_{\text{res}} = \frac{g_\ell \mu_B B}{\hbar} = \gamma B,
\end{equation}
where $\gamma = g_\ell \mu_B / \hbar$ is the gyromagnetic ratio. This is the fundamental equation of magnetic resonance spectroscopy (NMR, ESR).
\end{corollary}

\subsection{Chirality Coordinate Coupling ($s$): Spin Resonance}

The chirality coordinate $s$ is probed by spin resonance in the regime $\Omega_s$ (typically radio/microwave frequencies).

\begin{definition}[Chirality Coupling Structure]
\label{def:chirality_coupling}
The minimal coupling structure for the chirality coordinate $s$ is:
\begin{equation}
\mathcal{I}_s = \left( \Reals^+ \times \Reals^+ \times \Omega_s, \, dB_0 \, dB_1 \, d\omega, \, \kappa_s \right),
\end{equation}
where $B_0$ is the static magnetic field strength, $B_1$ is the oscillating (transverse) field amplitude, $\Omega_s$ is the chirality frequency regime, and:
\begin{equation}
\kappa_s(x, (B_0, B_1, \omega)) = \gamma^2 B_1^2 \cdot \frac{\Gamma^2}{(\omega - \omega_L)^2 + \Gamma^2},
\end{equation}
where $\omega_L = \gamma B_0$ is the Larmor frequency, $\gamma = g_s \mu_B / \hbar$ is the gyromagnetic ratio for spin (with $g_s \approx 2$ for electrons), and $\Gamma$ is the linewidth determined by relaxation processes.

\end{definition}

\begin{theorem}[Chirality Resonance]
\label{thm:chirality_resonance}
The chirality coupling is resonant at the Larmor frequency:
\begin{equation}
\omega = \omega_L = \gamma B_0 = \frac{g_s \mu_B B_0}{\hbar},
\end{equation}
with a Lorentzian lineshape of width $\Gamma$, determined by transverse relaxation time $T_2$ via $\Gamma = 1/T_2$ (Theorem~\ref{thm:linewidth_lifetime}).
\end{theorem}

\begin{proof}
The spin dynamics in the combined static field $\mathbf{B}_0 = B_0 \hat{z}$ and oscillating field $\mathbf{B}_1(t) = B_1 (\cos\omega t \, \hat{x} + \sin\omega t \, \hat{y})$ (circularly polarised) obey the Bloch equations:
\begin{equation}
\frac{d\mathbf{S}}{dt} = \gamma \mathbf{S} \times \mathbf{B}_{\text{total}}(t) - \frac{S_x \hat{x} + S_y \hat{y}}{T_2} - \frac{(S_z - S_0) \hat{z}}{T_1},
\end{equation}
where $T_1$ is the longitudinal relaxation time, $T_2$ is the transverse relaxation time, and $S_0$ is the equilibrium $z$-magnetization.

In the rotating frame at frequency $\omega$, the effective static field is:
\begin{equation}
\mathbf{B}_{\text{eff}} = (B_0 - \omega/\gamma) \hat{z} + B_1 \hat{x}.
\end{equation}
Resonance occurs when the effective field is purely transverse: $B_0 - \omega/\gamma = 0$, i.e., $\omega = \gamma B_0 = \omega_L$.

The steady-state transverse magnetisation has Lorentzian frequency dependence:
\begin{equation}
|S_\perp(\omega)|^2 \propto \frac{B_1^2 \Gamma^2}{(\omega - \omega_L)^2 + \Gamma^2},
\end{equation}
where $\Gamma = 1/T_2$. This reproduces the coupling function $\kappa_s$.
\end{proof}

\begin{proposition}[Chirality Transition Rate]
\label{prop:chirality_rate}
The transition rate between chirality states $s = +1/2$ and $s = -1/2$ at resonance ($\omega = \omega_L$) is:
\begin{equation}
W_{+1/2 \to -1/2} = \frac{\pi \gamma^2 B_1^2}{2\Gamma} = \frac{\pi \gamma^2 B_1^2 T_2}{2}.
\end{equation}
\end{proposition}

\begin{proof}
Apply Fermi's golden rule with the transition matrix element $V = -\gamma \hbar B_1 S_x / 2$ (for spin-1/2) and Lorentzian density of states $\rho(\omega) = \Gamma / [\pi((\omega - \omega_L)^2 + \Gamma^2)]$. At resonance:
\begin{equation}
W = \frac{2\pi}{\hbar} |V|^2 \rho(\omega_L) = \frac{2\pi}{\hbar} \cdot \frac{\gamma^2 \hbar^2 B_1^2}{4} \cdot \frac{1}{\pi \Gamma} = \frac{\pi \gamma^2 B_1^2}{2\Gamma}.
\end{equation}
\end{proof}

\begin{corollary}[Rabi Oscillations]
\label{cor:rabi}
On resonance, the chiral state undergoes coherent oscillations (Rabi oscillations) at the frequency:
\begin{equation}
\Omega_{\text{Rabi}} = \gamma B_1,
\end{equation}
corresponding to the periodic exchange of population between $s = +1/2$ and $s = -1/2$ states.
\end{corollary}

\begin{figure}[htbp]
\centering
\includegraphics[width=\textwidth]{figures/panel_xps_depth_coordinate.png}
\caption{Depth coordinate $n$ and X-ray photoelectron spectroscopy (XPS). \textbf{Top row:} Core-level binding energy surface showing $E_n \propto -n^{-2}$ scaling, radial probability distributions for $n=1$ through $n=5$ states with characteristic nodal structure, and shell capacity polar plot confirming $2n^2$ degeneracy (Theorem~\ref{thm:capacity}). \textbf{Middle row:} XPS kinetic energies for Fe shells (1s through 3p) at photon energy $h\nu = 1500$ eV, and Auger transition probability matrix showing cascade processes between shells. \textbf{Bottom row:} Electron shell isosurfaces for $n=1,2,3$ showing nested boundary structure, XPS survey spectrum of Fe with characteristic core-level peaks, and photoionization cross-section scaling as $\sigma_n \propto n^{-3}$ (red points) matching the frequency-coordinate duality prediction $\omega_n \propto n^{-3}$ (Theorem~\ref{thm:frequency_duality}). The coupling structure $\mathcal{I}_n$ implements high-frequency selective coupling in regime $\Omega_n$, corresponding to X-ray spectroscopy (Theorem~\ref{thm:depth_coupling}).}
\label{fig:depth_xps}
\end{figure}

\subsection{Composite Structures and Tensor Products}

For the simultaneous extraction of multiple coordinates, we construct composite coupling structures via tensor products.

\begin{definition}[Tensor Product Coupling]
\label{def:tensor_product}
For coupling structures $\mathcal{I}_1 = (\oscillator_1, \nu_1, \kappa_1)$ and $\mathcal{I}_2 = (\oscillator_2, \nu_2, \kappa_2)$, the \emph{tensor product} is:
\begin{equation}
\mathcal{I}_1 \otimes \mathcal{I}_2 = (\oscillator_1 \times \oscillator_2, \, \nu_1 \times \nu_2, \, \kappa_1 \cdot \kappa_2),
\end{equation}
where the product coupling is $\kappa_1 \cdot \kappa_2(x, (y_1, y_2)) = \kappa_1(x, y_1) \cdot \kappa_2(x, y_2)$.
\end{definition}

\begin{theorem}[Product Structure Extraction]
\label{thm:product_extraction}
The tensor product $\mathcal{I}_\xi \otimes \mathcal{I}_{\xi'}$ extracts the coordinate pair $(\xi, \xi')$ simultaneously, provided the frequency regimes $\Omega_\xi$ and $\Omega_{\xi'}$ are disjoint (Proposition~\ref{prop:regime_separation}).
\end{theorem}

\begin{proof}
The product coupling $\kappa_1 \cdot \kappa_2$ has support on the intersection of the resonance conditions:
\begin{equation}
\text{supp}(\kappa_1 \cdot \kappa_2) = \{(y_1, y_2) : \omega_1(y_1) \in \Omega_\xi \text{ and } \omega_2(y_2) \in \Omega_{\xi'}\}.
\end{equation}
Since $\Omega_\xi \cap \Omega_{\xi'} = \emptyset$ by regime separation, the two coupling channels are independent. The readout functions $g_\xi$ and $g_{\xi'}$ extract $\xi$ and $\xi'$ respectively via:
\begin{align}
\xi(x) &= \int_{\oscillator_1 \times \oscillator_2} g_\xi(y_1) \kappa_1(x, y_1) \kappa_2(x, y_2) \, d\nu_1(y_1) d\nu_2(y_2) \\
&= \int_{\oscillator_1} g_\xi(y_1) \kappa_1(x, y_1) \, d\nu_1(y_1) \cdot \underbrace{\int_{\oscillator_2} \kappa_2(x, y_2) \, d\nu_2(y_2)}_{= 1 \, (\text{normalization})}.
\end{align}
Similarly for $\xi'(x)$, establishing simultaneous extraction.
\end{proof}

\begin{corollary}[Complete Spectroscopic System]
\label{cor:complete_system}
The four-fold tensor product:
\begin{equation}
\mathcal{I}_{\text{complete}} = \mathcal{I}_n \otimes \mathcal{I}_\ell \otimes \mathcal{I}_m \otimes \mathcal{I}_s
\end{equation}
provides complete extraction of all partition coordinates $(n, \ell, m, s)$ simultaneously, constituting a mathematically complete spectroscopic measurement system.
\end{corollary}

This completes the explicit construction of minimal coupling structures. We have derived concrete mathematical expressions for the coupling functions $\kappa_\xi$ for all four partition coordinates, connecting the abstract framework of Sections~\ref{sec:instrument_necessity}--\ref{sec:resonance} to measurable spectroscopic quantities. These formulas provide the foundation for quantitative comparison with experimental data in Section~\ref{sec:applications}.
