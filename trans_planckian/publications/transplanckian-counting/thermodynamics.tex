\section{Thermodynamics and Statistical Mechanics from Triple Equivalence}
\label{sec:thermodynamics}

The ions in our Penning trap constitute a statistical ensemble. To understand their collective behavior and thermal properties, we derive thermodynamics from the partition-oscillation-category equivalence. This establishes that the trapped ion gas obeys the ideal gas law and related thermodynamic relations.

\subsection{The Triple Equivalence}

\begin{theorem}[Partition-Oscillation-Category Equivalence]
\label{thm:triple_equivalence}
For any bounded dynamical system, the following three descriptions are mathematically equivalent:
\begin{enumerate}
\item \textbf{Oscillatory}: The system exhibits periodic motion with frequency $\omega = 2\pi/T$.
\item \textbf{Categorical}: The system traverses $M$ distinguishable states per period.
\item \textbf{Partition}: The period $T$ is partitioned into $M$ temporal segments.
\end{enumerate}
These are not three separate phenomena but three perspectives on a single underlying structure.
\end{theorem}

\begin{proof}
From Section~\ref{sec:atom_derivation}, boundedness implies oscillation (Poincaré recurrence, Theorem~\ref{thm:poincare_recurrence}). 

Oscillation defines categories: an oscillating system traverses distinct states. The pendulum at its leftmost position is categorically distinct from the pendulum at its rightmost position—the oscillation \emph{is} the traversal between these categories.

Categories partition the period: the period $T$ naturally decomposes into segments, each corresponding to the time spent in a particular category. If the oscillation has $M$ distinguishable categories, the period is partitioned into $M$ segments:
\begin{equation}
T = \sum_{i=1}^{M} \tau_i
\end{equation}
where $\tau_i$ is the duration of partition $i$.

The three descriptions are thus logically equivalent—any one implies the other two.
\end{proof}

\begin{figure}[htbp]
    \centering
    \includegraphics[width=\textwidth]{figures/fig_pendulum_triple_equivalence.png}
    \caption{\textbf{Pendulum Demonstrates Triple Equivalence: Oscillation = Category = Partition.} 
    \textbf{Top Left - Oscillatory View:} Simple pendulum (black pivot point, black bob, gray reference positions, blue arrow showing current position). Equation: $\theta(t) = \theta_{\max}\cos(\omega t)$. Continuous periodic motion in angle coordinate.
    \textbf{Top Center - Continuous Periodic Motion:} Two traces versus time (0-12 in units of $t/T$): blue solid line ($\theta(t)$, angle), cyan dashed line ($\dot{\theta}(t)$, angular velocity). One complete period $T$ spans from $t = 0$ to $t = T$ (black arrow). Sinusoidal oscillation with phase shift between position and velocity.
    \textbf{Top Right - Phase Space (Ellipse):} Phase portrait showing $\dot{\theta}$ versus $\theta$ (both axes range $-0.4$ to $0.4$). Blue ellipse: phase space trajectory. Two red dots: current state showing position on ellipse. Closed trajectory indicates periodic motion with no dissipation.
    \textbf{Middle Center - Discrete State Structure:} Bar chart showing time in category versus category index (C$_1$ to C$_8$). Green bars with heights ranging 0.15 to 0.8. Peak at categories C$_3$ and C$_4$ (height $\approx 0.8$) corresponds to slow motion near turning points. Minimum at C$_1$ and C$_8$ (height $\approx 0.15$) corresponds to fast motion through equilibrium. Black arrows labeled ``Traversal'' indicate sequential category progression.
    \textbf{Bottom Left - Categorical View:} Eight green spheres (C$_1$ through C$_8$) arranged in arc, connected by gray lines to black pivot point above. Text: ``$M = 8$ categories. Each $C_i$ is a distinguishable state.'' Pendulum motion discretized into eight categorical regions.
    \textbf{Bottom Right - Partition View:} Eight pink/red rectangles (P$_1$ through P$_8$) arranged horizontally along time axis (0 to $T$). Color gradient from light pink (short duration) to dark red (long duration). Black arrow labeled $t$ points right. Equation: $T = \sum_{i=1}^M \tau_i$. Text: ``Each partition = one category transition.'' Average partition duration: $\langle\tau_p\rangle = T/M$.
    \textbf{Bottom - Triple Equivalence Statement:} Yellow box with black border: ``TRIPLE EQUIVALENCE: Oscillation = Category Traversal = Period Partition.'' Below: ``Fundamental Identity: $dM/dt = \omega/(2\pi/M) = 1/\langle\tau_p\rangle$.'' This identity connects categorical rate (left), oscillation frequency (center), and partition rate (right), proving all three perspectives measure the same dynamics.}
    \label{fig:pendulum_triple_equivalence}
    \end{figure}

\subsection{Quantitative Relationships}

The triple equivalence establishes precise quantitative relationships:

\textbf{Rate of category traversal:}
\begin{equation}
\frac{dM}{dt} = \frac{M}{T} = \frac{M\omega}{2\pi}
\end{equation}

\textbf{Average partition duration:}
\begin{equation}
\langle\tau_p\rangle = \frac{T}{M} = \frac{2\pi}{M\omega}
\end{equation}

\textbf{Fundamental identity:}
\begin{equation}
\boxed{\frac{dM}{dt} = \frac{\omega}{2\pi/M} = \frac{1}{\langle\tau_p\rangle}}
\label{eq:fundamental_identity}
\end{equation}

Equation~\eqref{eq:fundamental_identity} expresses the triple equivalence quantitatively: the rate of categorical actualization, the (scaled) oscillation frequency, and the inverse partition lag are identical.

\subsection{Entropy from Three Perspectives}

\subsubsection{Categorical Entropy}

\begin{definition}[Categorical Entropy]
\label{def:categorical_entropy}
For a system with $M$ independent categorical dimensions, each with $n$ possible states:
\begin{equation}
\boxed{S_{\text{cat}} = k_B M \ln n}
\end{equation}
where $k_B$ is Boltzmann's constant.
\end{definition}

The total number of categorical states is $n^M$. The entropy $S = k_B \ln \Omega$ with $\Omega = n^M$ gives $S = k_B M \ln n$.

\subsubsection{Oscillatory Entropy}

\begin{definition}[Oscillatory Entropy]
\label{def:oscillatory_entropy}
For a system with $M$ independent oscillation modes, each with amplitude $A_i$ relative to ground state amplitude $A_0$:
\begin{equation}
\boxed{S_{\text{osc}} = k_B \sum_{i=1}^{M} \ln\left(\frac{A_i}{A_0}\right)}
\end{equation}
\end{definition}

The amplitude ratio $A_i/A_0$ measures the number of accessible oscillation states for mode $i$. For equal amplitudes $A_i/A_0 = n$:
\begin{equation}
S_{\text{osc}} = k_B \sum_{i=1}^{M} \ln n = k_B M \ln n = S_{\text{cat}}
\end{equation}

\subsubsection{Partition Entropy}

\begin{definition}[Partition Entropy]
\label{def:partition_entropy}
For a system with $M$ partition operations, each with selectivity $s_a$ (probability of selecting correct partition):
\begin{equation}
\boxed{S_{\text{part}} = k_B \sum_{a=1}^{M} \ln\left(\frac{1}{s_a}\right)}
\end{equation}
\end{definition}

The inverse selectivity $1/s_a$ measures the number of partitions that must be explored. For equal selectivity $s_a = 1/n$:
\begin{equation}
S_{\text{part}} = k_B \sum_{a=1}^{M} \ln n = k_B M \ln n = S_{\text{cat}} = S_{\text{osc}}
\end{equation}

\begin{theorem}[Entropy Equivalence]
\label{thm:entropy_equivalence}
The three entropy formulations are mathematically identical:
\begin{equation}
S_{\text{cat}} = S_{\text{osc}} = S_{\text{part}} = k_B M \ln n
\end{equation}
\end{theorem}

\begin{figure*}[htbp]
    \centering
    \includegraphics[width=\textwidth]{figures/panel1_triple_equivalence.png}
    \caption{\textbf{The Partition-Oscillation-Category Equivalence.} 
    (\textbf{A}) Virtual gas molecules represented as pendulums in a container. Each vibrational mode corresponds to one pendulum oscillator. 
    (\textbf{B}) Oscillatory perspective: A pendulum traces angle $\theta(t) = \theta_0 \cos(\omega t)$ with period $T = 2\pi/\omega$. Quantum states $n = 0, 1, 2, \ldots$ are marked on the amplitude axis. 
    (\textbf{C}) Categorical perspective: The pendulum's period divides into $n = 8$ distinguishable positions. Each position $\theta_i$ corresponds to a categorical state $C_i$. 
    (\textbf{D}) Partition perspective: A tree structure with depth $M$ (levels) and branching factor $n$ (branches per node). The number of terminal states (leaves) is $n^M$. 
    (\textbf{E}) The fundamental equivalence: All three perspectives yield the same entropy $S = k_B M \ln n$, where $M$ is the number of degrees of freedom and $n$ is the number of states per degree of freedom. 
    (\textbf{F}) Parameter correspondence table showing how oscillatory modes, categorical dimensions, and partition levels map to each other. The pendulum demonstrates all three perspectives simultaneously: oscillation $\theta(t) = \theta_0 \cos(\omega t)$, $n$ distinguishable categorical positions $\{C_1, \ldots, C_n\}$, and period $T$ divided into $n$ intervals.}
    \label{fig:triple_equivalence}
    \end{figure*}

\subsection{Temperature as Categorical Actualization Rate}

\begin{definition}[Categorical Temperature]
\label{def:categorical_temperature}
Temperature is the rate of categorical actualization:
\begin{equation}
\boxed{T = \frac{\hbar}{k_B}\frac{dM}{dt}}
\end{equation}
where $\hbar$ is the reduced Planck constant.
\end{definition}

\begin{proposition}[Temperature-Energy Relation]
\label{prop:temperature_energy}
For a system with $M$ active categorical dimensions:
\begin{equation}
T = \frac{U}{k_B M}
\end{equation}
where $U$ is the internal energy.
\end{proposition}

\begin{proof}
Each active category contributes energy $\hbar\omega$ where $\omega = dM/dt$ is the categorical actualization rate. The total internal energy is:
\begin{equation}
U = M \cdot \hbar\omega = M \hbar \frac{dM}{dt}
\end{equation}

From Definition~\ref{def:categorical_temperature}:
\begin{equation}
T = \frac{\hbar}{k_B}\frac{dM}{dt}
\end{equation}

Therefore:
\begin{equation}
k_B T = \hbar\frac{dM}{dt} = \frac{U}{M}
\end{equation}

Rearranging:
\begin{equation}
T = \frac{U}{k_B M}
\end{equation}
\end{proof}

\textbf{Physical interpretation:} Temperature measures how fast the system actualizes categorical states. Higher temperature means faster categorical traversal.


\begin{figure}[htbp]
    \centering
    \includegraphics[width=\textwidth]{figures/fig_temperature_perspectives.png}
    \caption{\textbf{Temperature: Triple Equivalence Perspectives.} 
    \textbf{(A) Categorical actualization rate:} Categorical transition rate $dM/dt$ (transitions/s, logarithmic scale 10$^9$ to 10$^{23}$) versus temperature $T$ (kelvin, logarithmic scale 10$^{-3}$ to 10$^{13}$). Green solid line: categorical prediction (linear on log-log plot). Four colored background regions: purple (quantum regime, $T < 1$ K), light green (classical regime, 1 K $< T < 10^7$ K), light orange (relativistic regime, $T > 10^7$ K). Temperature measures the rate at which categories are actualized: $T = (\hbar/k_B) \cdot dM/dt$.
    \textbf{(B) Oscillatory frequency:} Angular frequency $\omega$ (rad/s, logarithmic scale 10$^8$ to 10$^{48}$) versus temperature $T$ (kelvin, logarithmic scale 10$^{-3}$ to 10$^{13}$). Blue solid line: categorical prediction. Gray dashed line: classical (no bound, linear). Purple dotted horizontal line at $\omega_{\text{Planck}} = 1.85 \times 10^{43}$ rad/s: maximum frequency (Planck frequency). At low temperature, frequency scales linearly with $T$. At high temperature ($T \gtrsim 10^{13}$ K), frequency saturates at Planck frequency (categorical bound). Classical prediction continues linearly (unphysical).
    \textbf{(C) Partition lag:} Average partition duration $\langle\tau_p\rangle$ (seconds, logarithmic scale 10$^{-23}$ to 10$^{-9}$) versus temperature $T$ (kelvin, logarithmic scale 10$^{-3}$ to 10$^{13}$). Red solid line: partition lag decreases with temperature (inverse relationship). Text annotation at top left: ``Long lag (cold)'' indicates cold systems have long partition durations (slow categorical transitions). At $T = 10^{-3}$ K, $\langle\tau_p\rangle \sim 10^{-9}$ s. At $T = 10^{13}$ K, $\langle\tau_p\rangle \sim 10^{-23}$ s (approaching Planck time).
    \textbf{(D) Equivalence test:} Ratio to classical temperature (dimensionless) versus temperature $T$ (kelvin, logarithmic scale 10$^0$ to 10$^{10}$). Three overlapping traces: green circles (categorical), blue squares (oscillatory), red triangles (partition). All three traces overlap at ratio = 1.000 across entire temperature range, confirming triple equivalence. Vertical axis range: 0.900-1.100, showing deviations $<$0.1\% across 10 orders of magnitude in temperature.}
    \label{fig:temperature_perspectives}
    \end{figure}
\subsection{Pressure as Categorical Density}

\begin{definition}[Categorical Pressure]
\label{def:categorical_pressure}
Pressure is the categorical density:
\begin{equation}
\boxed{P = k_B T \left(\frac{\partial M}{\partial V}\right)_S}
\end{equation}
where the derivative is taken at constant entropy $S$.
\end{definition}

\begin{proposition}[Pressure Formula]
\label{prop:pressure_formula}
For a system with $M$ categorical dimensions in volume $V$:
\begin{equation}
P = \frac{k_B T M}{V}
\end{equation}
\end{proposition}

\begin{proof}
For a bounded system, the number of accessible categorical states scales with volume. In three-dimensional space:
\begin{equation}
M \propto V
\end{equation}

Therefore:
\begin{equation}
\frac{\partial M}{\partial V} = \frac{M}{V}
\end{equation}

Substituting into Definition~\ref{def:categorical_pressure}:
\begin{equation}
P = k_B T \frac{M}{V}
\end{equation}
\end{proof}

\textbf{Physical interpretation:} Pressure is not localized at container boundaries. It is an intrinsic property existing throughout the volume—the categorical density. Wall collisions are one manifestation of categorical density, not its definition.

\subsection{The Ideal Gas Law}

\begin{theorem}[Ideal Gas Law]
\label{thm:ideal_gas_law}
For $N$ particles, each with $f$ degrees of freedom, the total number of categorical dimensions is:
\begin{equation}
M = Nf
\end{equation}

From Proposition~\ref{prop:pressure_formula}:
\begin{equation}
P = \frac{k_B T M}{V} = \frac{k_B T Nf}{V}
\end{equation}

For monoatomic ideal gas, $f = 3$ (three translational degrees of freedom). Therefore:
\begin{equation}
\boxed{PV = Nk_BT}
\end{equation}
\end{theorem}

This is the ideal gas law, derived from categorical structure without empirical assumptions.

\begin{corollary}[Universal Gas Constant]
\label{cor:gas_constant}
For $n$ moles of gas ($N = nN_A$ where $N_A$ is Avogadro's number):
\begin{equation}
PV = nRT
\end{equation}
where $R = N_A k_B = 8.314$ J/(mol·K) is the universal gas constant.
\end{corollary}

\begin{figure}[htbp]
    \centering
    \includegraphics[width=\textwidth]{figures/panel_categorical_computing_gas_laws.png}
    \caption{\textbf{Categorical Computing as Gas Law Derivation.} 
    \textbf{Top Left - Categorical operations as molecular trajectories:} Three-dimensional visualization of 27 categories organized as $3^3$ phase cells. Axes: Category $x$, Category $y$, Category $z$ (all range 0.0-2.0). Colored lines (rainbow gradient from blue to red): molecular trajectories connecting different categorical states. Each trajectory represents one computational operation = one molecular transition. The $3^3 = 27$ cell structure provides natural discretization of phase space.
    \textbf{Top Center - Operation types equal energy modes:} Bar chart showing operation count versus operation type. Three bars: Oscillatory/Phase (red, count $\approx 67$), Categorical/Transition (green, count $\approx 68$), Partition/Rearrange (blue, count $\approx 65$). Black error bars show fluctuations. Nearly equal counts demonstrate equipartition across operation types—this IS the equipartition theorem, not an approximation but an exact consequence of balanced categorical structure.
    \textbf{Top Right - Hardware oscillation equals temperature:} Horizontal bar chart showing temperature equivalent (kelvin, logarithmic scale 10$^{-5}$ to 10$^2$) for different hardware components. Five bars (all orange): WiFi 2.4 GHz ($T \approx 1.2 \times 10^{-1}$ K), Quartz 32 kHz ($T \approx 1.6 \times 10^{-5}$ K), LED optical ($T \approx 2.4 \times 10^4$ K), RAM 1.6 GHz ($T \approx 7.7 \times 10^{-2}$ K), CPU 3 GHz ($T \approx 1.4 \times 10^{-1}$ K). Temperature defined by $T = hf/k_B$ where $f$ is oscillation frequency. Hardware oscillations ARE thermal oscillations—not analogous but identical.
    \textbf{Middle Left - T-S relationship from computation:} Derived entropy (dimensionless, range 2.6-3.3) versus derived temperature (range 170-220). Blue circles: computed values from trajectory statistics. Red dashed curve: fit to $S \sim \ln(T)$. Scatter shows thermal fluctuations. This relationship is DERIVED from computation, not assumed. Temperature and entropy emerge simultaneously from bounded trajectory dynamics.
    \textbf{Middle Center - State occupancy equals Boltzmann distribution:} Occupancy (count, range 0-300) versus categorical state/energy level (0-25). Green bars: computed occupancy from categorical operations. Red dashed curve: Maxwell-Boltzmann prediction $\exp(-E/k_B T)$. Perfect agreement demonstrates that categorical occupancy statistics automatically yield Boltzmann distribution. No statistical mechanics postulates required—Boltzmann distribution is a theorem about discrete state occupation.}
    \label{fig:categorical_computing}
    \end{figure}

\subsection{Internal Energy}

\begin{definition}[Internal Energy]
\label{def:internal_energy}
The internal energy is the total energy of active categorical dimensions:
\begin{equation}
\boxed{U = M_{\text{active}} k_B T}
\end{equation}
\end{definition}

\begin{proposition}[Equipartition Theorem]
\label{prop:equipartition}
Each active degree of freedom contributes $k_B T$ to the internal energy.
\end{proposition}

For monoatomic ideal gas with $N$ particles and 3 translational degrees of freedom:
\begin{equation}
U = \frac{3}{2}Nk_BT
\end{equation}

\subsection{Maxwell-Boltzmann Distribution}

\begin{theorem}[Velocity Distribution]
\label{thm:velocity_distribution}
The velocity distribution is the continuum limit of discrete categorical structure:
\begin{equation}
f(v) = 4\pi n \left(\frac{m}{2\pi k_B T}\right)^{3/2} v^2 e^{-mv^2/(2k_BT)}
\end{equation}
for $v \leq c$ (speed of light).
\end{theorem}

\begin{proof}
Velocity corresponds to categorical traversal rate: $v = \Delta x/\tau$. The number of categorical states accessible at velocity $v$ is proportional to $v^2$ (phase space volume in velocity space).

The probability of occupying a state with energy $E = mv^2/2$ follows the Boltzmann distribution:
\begin{equation}
P(E) \propto e^{-E/(k_BT)} = e^{-mv^2/(2k_BT)}
\end{equation}

Combining:
\begin{equation}
f(v) \propto v^2 e^{-mv^2/(2k_BT)}
\end{equation}

Normalization gives the full Maxwell-Boltzmann distribution.

\textbf{Critical constraint:} The distribution is bounded at $v = c$ because categorical traversal cannot exceed the speed of light (the maximum rate of categorical state propagation). This eliminates the unphysical infinite velocity tail of the classical Maxwell distribution.
\end{proof}

\subsection{Application to Penning Trap Ensemble}

In our Penning trap (Section~\ref{sec:experimental_setup}), we have:

\begin{itemize}
\item \textbf{Number of ions}: $N \sim 10^3$ to $10^5$ H$^+$ ions
\item \textbf{Volume}: $V \sim 10^{-9}$ m$^3$ (trap volume)
\item \textbf{Temperature}: $T = 4$ K (cryogenic cooling)
\item \textbf{Degrees of freedom}: $f = 6$ (3 translational + 3 vibrational modes)
\end{itemize}

The ideal gas law predicts:
\begin{equation}
P = \frac{Nk_BT}{V} = \frac{10^4 \times 1.38 \times 10^{-23} \times 4}{10^{-9}} \approx 5.5 \times 10^{-7} \text{ Pa}
\end{equation}

This is an ultra-high vacuum, consistent with our experimental conditions.

The internal energy is:
\begin{equation}
U = 3Nk_BT = 3 \times 10^4 \times 1.38 \times 10^{-23} \times 4 \approx 1.7 \times 10^{-18} \text{ J}
\end{equation}

The mean thermal velocity is:
\begin{equation}
\langle v \rangle = \sqrt{\frac{8k_BT}{\pi m_p}} = \sqrt{\frac{8 \times 1.38 \times 10^{-23} \times 4}{\pi \times 1.67 \times 10^{-27}}} \approx 1450 \text{ m/s}
\end{equation}

These thermodynamic properties are essential for understanding the ion ensemble behavior during electron trajectory measurements.

\subsection{Resolution of Classical Paradoxes}

\subsubsection{Resolution-Dependence of Temperature}

Classical kinetic theory defines temperature through mean square velocity: $T = m\langle v^2\rangle/(3k_B)$. This makes temperature dependent on how velocity is measured—the resolution and averaging procedure.

In the categorical framework, temperature is the rate of categorical actualization:
\begin{equation}
T = \frac{\hbar}{k_B}\frac{dM}{dt}
\end{equation}

Categories are discrete and countable; no resolution ambiguity arises. The categorical rate $dM/dt$ is intrinsic to the system and independent of measurement procedure.

\subsubsection{Localization of Pressure}

Classical kinetic theory derives pressure from molecular collisions with container walls, suggesting pressure is a boundary phenomenon. Yet we measure pressure in the bulk of fluids.

In the categorical framework, pressure is categorical density:
\begin{equation}
P = k_B T \frac{M}{V}
\end{equation}

This is an intrinsic property existing throughout the volume, not localized at boundaries. Wall collisions are one manifestation of categorical density, not its definition.

\subsubsection{Infinite Velocity Tail}

The Maxwell-Boltzmann distribution $f(v) \propto v^2 e^{-mv^2/(2k_BT)}$ extends to $v \to \infty$, violating special relativity ($v < c$).

In the categorical framework, the distribution is over discrete categories $m = 0, 1, \ldots, M_{\max}$, where $M_{\max}$ corresponds to $v_{\max} = c$. The distribution is intrinsically bounded:
\begin{equation}
f(m) = \frac{e^{-\beta E_m}}{\sum_{m=0}^{M_{\max}} e^{-\beta E_m}}
\end{equation}

No particle can occupy category $m > M_{\max}$, automatically enforcing $v \leq c$.

\begin{figure}[htbp]
    \centering
    \includegraphics[width=\textwidth]{figures/panel_mrt_22L.png}
    \caption{Maxwell Relations Tester: Categorical Thermodynamics Validation. 
    \textbf{Top row:} Maxwell relations 1, 2, and 3 showing perfect agreement between reciprocal derivatives:
    - \textbf{Relation 1:} 
    $$\left(\frac{\partial T}{\partial V}\right)_S = -\left(\frac{\partial P}{\partial S}\right)_V$$
    with identical slopes
    - \textbf{Relation 2:} 
    $$\left(\frac{\partial S}{\partial V}\right)_T = \left(\frac{\partial P}{\partial T}\right)_V$$
    with coefficient 7.31$\times$$10^{13}$ Pa/K$^2$
    - \textbf{Relation 3:} 
    $$\left(\frac{\partial S}{\partial P}\right)_T = -\left(\frac{\partial V}{\partial T}\right)_P$$
    showing perfect reciprocal symmetry
    \textbf{Bottom left:} Maxwell relation 4: 
    $$\left(\frac{\partial T}{\partial P}\right)_S = \left(\frac{\partial V}{\partial S}\right)_P$$
    maintaining constant value 0.00108 across temperature range, confirming thermodynamic consistency.
    \textbf{Bottom center:} 3D deviation surface for relation 2 showing deviations < $10^{-7}$ across entire (T,V) parameter space, demonstrating numerical precision of categorical thermodynamics.
    \textbf{Bottom right:} Triple equivalence of entropy showing categorical (green), oscillatory (blue), and partition (purple) methods yielding identical entropy values across 200-1000 K temperature range.}
    \label{fig:maxwell_success}
    \end{figure}

\subsection{Summary}

We have derived thermodynamics from the partition-oscillation-category equivalence:

\begin{itemize}
\item \textbf{Entropy}: $S = k_B M \ln n$ (three equivalent forms)
\item \textbf{Temperature}: $T = U/(k_B M)$ (categorical actualization rate)
\item \textbf{Pressure}: $P = k_B TM/V$ (categorical density)
\item \textbf{Internal Energy}: $U = M_{\text{active}} k_B T$ (active category counting)
\item \textbf{Ideal Gas Law}: $PV = Nk_BT$ (categorical balance)
\item \textbf{Maxwell-Boltzmann}: Continuum limit with $v \leq c$ bound
\end{itemize}

All thermodynamics emerges from:
\begin{equation}
\text{Bounded phase space} \implies \text{Triple equivalence} \implies \text{Thermodynamics}
\end{equation}

This establishes that the ion ensemble in our Penning trap obeys thermodynamic laws derived from the same partition structure that produces atomic states and classical mechanics. The complete framework—atomic structure, classical mechanics, thermodynamics—rests on the single axiom of bounded phase space.
